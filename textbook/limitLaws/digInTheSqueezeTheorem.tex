\documentclass{ximera}

\newcommand{\RR}{\mathbb R}
\renewcommand{\d}{\,d}
\newcommand{\dd}[2][]{\frac{d #1}{d #2}}
\renewcommand{\l}{\ell}
\newcommand{\ddx}{\frac{d}{dx}}
\newcommand{\dfn}{\textbf}
\newcommand{\eval}[1]{\bigg[ #1 \bigg]}


\title[Dig-In]{The squeeze theorem}

\begin{document}
\begin{abstract}
\end{abstract}
\maketitle



\begin{theorem}[Squeeze Theorem]
Suppose that $g(x) \le f(x) \le h(x)$ for all $x$
close to $a$ but not necessarily equal to $a$. If 
\[
\lim_{x\to a} g(x) = L = \lim_{x\to a} h(x),
\] 
then $\lim_{x\to a} f(x) = L$.
\end{theorem}

\begin{question}
\end{question}



%% \marginnote[.2in]{For a nice discussion of this limit, see: Richman,
%%   Fred. \textit{A circular argument}. College Math. J. 24 (1993),
%%   no. 2, 160--162.}
\begin{example}
Compute
\[
\lim_{\theta\to 0} \frac{\sin(\theta)}{\theta}.
\]



To compute this limit, use the Squeeze Theorem. First note that we
only need to examine $\theta\in \left(\frac{-\pi}{2}, \frac{\pi}{2}\right)$
and for the present time, we'll assume that $\theta$ is positive---consider
the diagrams below:


\begin{tabular}{ccc}
\begin{tikzpicture}
	\begin{axis}[
            xmin=-.1,xmax=1.1,ymin=-.1,ymax=1.1,
            axis lines=center,
            ticks=none,
            unit vector ratio*=1 1 1,
            xlabel=$x$, ylabel=$y$,
            every axis y label/.style={at=(current axis.above origin),anchor=south},
            every axis x label/.style={at=(current axis.right of origin),anchor=west},
          ]        
          \addplot [very thick, penColor2, smooth, domain=(-.2:.2+pi/2)] ({cos(deg(x))},{sin(deg(x))});
          \addplot [textColor] plot coordinates {(0,0) (1,.839)}; %% 40 degrees
          \addplot [very thick, penColor] plot coordinates {(.766,0) (.766,.643)}; %% 40 degrees
          \addplot [textColor] plot coordinates {(1,0) (1,.839)}; %% 40 degrees
          \addplot [very thick,penColor,fill=fill1] plot coordinates {(0,0) (.766,.643)}\closedcycle; %% triangle
          \addplot [textColor,smooth, domain=(0:40)] ({.15*cos(x)},{.15*sin(x)});
          \node at (axis cs:.15,.07) [anchor=west] {$\theta$};
          \node at (axis cs:.766,.322) [anchor=east] {$\sin(\theta)$};
          \node at (axis cs:.383,0) [anchor=north] {$\cos(\theta)$};
        \end{axis}
\end{tikzpicture} & 
\begin{tikzpicture}
	\begin{axis}[
            xmin=-.1,xmax=1.1,ymin=-.1,ymax=1.1,
            axis lines=center,
            ticks=none,
            unit vector ratio*=1 1 1,
            xlabel=$x$, ylabel=$y$,
            every axis y label/.style={at=(current axis.above origin),anchor=south},
            every axis x label/.style={at=(current axis.right of origin),anchor=west},
          ]        
          \addplot [draw=none,fill=fill1] plot coordinates {(0,0) (.766,.643)}\closedcycle; %% sector
          \addplot [draw=none, fill=fill1, samples=100, domain=(0:40)] ({cos(x)},{sin(x)})\closedcycle; %% sector 
          \addplot [very thick, penColor2, smooth, domain=(-.2:.2+pi/2)] ({cos(deg(x))},{sin(deg(x))});
          \addplot [textColor] plot coordinates {(0,0) (1,.839)}; %% 40 degrees
          \addplot [textColor] plot coordinates {(.766,0) (.766,.643)}; %% 40 degrees
          \addplot [textColor] plot coordinates {(1,0) (1,.839)}; %% 40 degrees          
          \addplot [textColor,smooth, domain=(0:40)] ({.15*cos(x)},{.15*sin(x)});
          \addplot [very thick,penColor] plot coordinates {(0,0) (.766,.643)}; %% sector
          \addplot [very thick,penColor] plot coordinates {(0,0) (1,0)}; %% sector
          \addplot [very thick, penColor, smooth, domain=(0:40)] ({cos(x)},{sin(x)}); %% sector
          \node at (axis cs:.15,.07) [anchor=west] {$\theta$};
          \node at (axis cs:.5,0) [anchor=north] {$1$};
        \end{axis}
\end{tikzpicture} & 
\begin{tikzpicture}
	\begin{axis}[clip=false,
            xmin=-.1,xmax=1.1,ymin=-.1,ymax=1.1,
            axis lines=center,
            ticks=none,
            unit vector ratio*=1 1 1,
            xlabel=$x$, ylabel=$y$,
            every axis y label/.style={at=(current axis.above origin),anchor=south},
            every axis x label/.style={at=(current axis.right of origin),anchor=west},
          ]        
          \addplot [very thick,penColor,fill=fill1] plot coordinates {(0,0) (1,.839)}\closedcycle; %% triangle          
          \addplot [very thick, penColor2, smooth, domain=(-.1:1.671)] ({cos(deg(x))},{sin(deg(x))});
          \addplot [very thick, penColor] plot coordinates {(0,0) (1,.839)}; %% 40 degrees
          \addplot [textColor] plot coordinates {(.766,0) (.766,.643)}; %% 40 degrees
          \addplot [very thick, penColor] plot coordinates {(1,0) (1,.839)}; %% 40 degrees          
          \addplot [textColor,smooth, domain=(0:40)] ({.15*cos(x)},{.15*sin(x)});
          \node at (axis cs:.15,.07) [anchor=west] {$\theta$};
          \node at (axis cs:.5,0) [anchor=north] {$1$};
          \node at (axis cs:1,.42) [anchor=west] {$\tan(\theta)$};
        \end{axis}
\end{tikzpicture}\\
Triangle $A$ & Sector & Triangle $B$ \\
\end{tabular}

From our diagrams above we see that
\[
\text{Area of Triangle $A$} \le \text{Area of Sector} \le \text{Area of Triangle $B$}
\]
and computing these areas we find
\[
\frac{\cos(\theta)\sin(\theta)}{2} \le \frac{\theta}{2} \le \frac{\tan(\theta)}{2}.
\]
Multiplying through by $2$, and recalling that $\tan(\theta) =
\frac{\sin(\theta)}{\cos(\theta)}$ we obtain
\[
\cos(\theta)\sin(\theta) \le \theta \le \frac{\sin(\theta)}{\cos(\theta)}.
\]
Dividing through by $\sin(\theta)$ and taking the reciprocals
(reversing the inequalities), we find
\[
\cos(\theta) \le \frac{\sin(\theta)}{\theta} \le \frac{1}{\cos(\theta)}.
\]
Note, $\cos(-\theta) = \cos(\theta)$ and $\frac{\sin(-\theta)}{-\theta} =
\frac{\sin(\theta)}{\theta}$, so these inequalities hold for all $\theta\in
\left(\frac{-\pi}{2}, \frac{\pi}{2}\right)$.  Additionally, we know
\[
\lim_{\theta \to 0}\cos(\theta) = 1 = \lim_{\theta\to 0}\frac{1}{\cos(\theta)},
\]
and so we conclude by the Squeeze Theorem, $\lim_{\theta \to
  0}\frac{\sin(\theta)}{\theta} = 1$.
\end{example}





\end{document}
