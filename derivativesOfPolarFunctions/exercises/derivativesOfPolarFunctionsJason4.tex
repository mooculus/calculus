\documentclass{ximera}

\newcommand{\RR}{\mathbb R}
\renewcommand{\d}{\,d}
\newcommand{\dd}[2][]{\frac{d #1}{d #2}}
\renewcommand{\l}{\ell}
\newcommand{\ddx}{\frac{d}{dx}}
\newcommand{\dfn}{\textbf}
\newcommand{\eval}[1]{\bigg[ #1 \bigg]}


\author{Jason Miller}
\license{Creative Commons 3.0 By-bC}


\outcome{}

\begin{document}
\begin{exercise}


Consider the polar curve $r=1+\sin(\theta)$. The graph is show below. 




\begin{image}  
  \begin{tikzpicture}  
    \begin{axis}[  
        xmin=-2.5,  
        xmax=2.5,  
        ymin=-1.5,  
        ymax=2.5,  
        axis lines=center,  
        xlabel=$x$,  
        ylabel=$y$,  
        every axis y label/.style={at=(current axis.above origin),anchor=south},  
        every axis x label/.style={at=(current axis.right of origin),anchor=west},  
      ]  
      \addplot[data cs=polar,blue,domain=0:360,samples=360,smooth, thick] (x,{1+sin(x)});
            \end{axis}  
  \end{tikzpicture}  
\end{image} 

We want to find all the horizontal and vertical tangent lines to this curve.

First we find the $\theta$ values where $r=1+\sin(\theta)$ has a horizontal tangent line and give the equation of the line. 

Since our curve can be generated by letting $\theta$ vary from over the interval $[0, 2\pi)$, we can assume the $\theta$ values lie in $[0, 2\pi)$. List the $\theta$ values in order from smallest to largest below: 


When $\theta=\answer{\frac{\pi}{2}}$ the equation of the tangent line is $\answer{y=2  }$.

When $\theta=\answer{\frac{7\pi}{6}}$ the equation of the tangent line is $\answer{y=-\frac{1}{4}}$ and when $\theta=\answer{ \frac{11\pi}{6}}$ the equation of 
the tangent line is $\answer{ y=-\frac{1}{4}}$. 




\begin{hint}

Consider the formulas for changing from polar coordinates to Cartesian coordinates:

\begin{align*}
x&=r\cos(\theta) \\
y&=r\sin(\theta)
\end{align*}

Using the equation of our curve $r=\cos(\theta)$, we can substitute for $r$ to obtain:

\begin{align*}
x&=\answer{(1+\sin(\theta))\cos(\theta)} \\
y&=\answer{(1+\sin(\theta))\sin(\theta) }
\end{align*}

Note that we have expressed both the $x$ and $y$ coordinates of the points of the curve in terms of functions of a single parameter $\theta$.

Since we have a parametric description of our curve in terms of $\theta$, 
we can use the chain rule to express $\dd[y]{x}$ in terms of $\theta$ to get $\dd[y]{x}=\frac{ dy/d\theta}{dx/d\theta}$.

Calculating we get

\begin{align*}
\dd[x]{\theta}&=\answer{ -\sin(\theta)+\cos^2(\theta)-\sin^2(\theta)  } \\
\dd[y]{\theta}&=\answer{ \cos(\theta) + 2\sin(\theta)\cos(\theta)   }
\end{align*} 

Thus we find $\dd[y]{x}=\answer{\frac{ \cos(\theta) + 2\sin(\theta)\cos(\theta) }{-\sin(\theta)+\cos^2(\theta)-\sin^2(\theta) }}$. 

The tangent line is horizontal when the slope is $0$. Therefore we need to find where $\dd[y]{x}$ is equal to $0$.

Setting the numerator of $\dd[y]{x}$ equal to $0$ gives us $\cos(\theta) + 2\sin(\theta)\cos(\theta)=0$. 

We can pull a $\cos(\theta)$ out of each term to get $\answer{\cos(\theta)(1+2\sin(\theta))=0 }$. 

Thus we have $\cos(\theta)=0$ or $1+2\sin(\theta)=0$. 


(Notice that our curve $r=\cos(\theta)$ can be generated by letting $\theta$ vary over the interval $[0, 2\pi)$, we solve these equations for $\theta$ values in  $[0, 2\pi)$).

Below list the $\theta$ values in order from smallest to largest.  

When $\cos(\theta)=0$, we get $\theta=\answer{\frac{\pi}{2}}$ and $\theta=\answer{\frac{3\pi}{2}}$.  

When $1+2\sin(\theta)=0$, we get $\theta=\answer{ \frac{7\pi}{6}}$ and $\theta=\answer{\frac{11\pi}{6}}$.

Notice that when $\theta=\frac{3\pi}{2}$ then the denominator of $\dd[y]{x}$ is $0$ as well. This corresponds to the cusp (sharp corner) on the graph at the origin. Recall that the tangent line at a given point is supposed to be the best linear approximation of the graph at the point. Our curve at $(0,0)$ does not look more and more like a line as we zoom in on the origin. 

\end{hint}

\begin{exercise}



Now we want to find the $\theta$ values where $r=\cos(\theta)$ has a vertical tangent line and give the equation of the line. 

Again assume the $\theta$ values lie in $[0, 2\pi)$. List the $\theta$ values in order below: 

When $\theta=\answer{\frac{\pi}{6}}$ the equation of the tangent line is $\answer{x=\frac{3\sqrt{3}}{4} }$ 

and when $\theta=\answer{ \frac{5\pi}{6}}$ the equation of the tangent line is $\answer{x=\frac{-3\sqrt{3}}{4}}$. 





\begin{hint}


In the hint for exercise 1 we derived a parametric description of our curve in terms of $\theta$: 

\begin{align*}
\dd[x]{\theta}&=\answer{  -\sin(\theta)+\cos^2(\theta)-\sin^2(\theta) } \\
\dd[y]{\theta}&=\answer{ \cos(\theta) + 2\sin(\theta)\cos(\theta)    }
\end{align*} 

Where we calculated $\dd[y]{x}=\answer{ \frac{ \cos(\theta) + 2\sin(\theta)\cos(\theta) }{-\sin(\theta)+\cos^2(\theta)-\sin^2(\theta) }}$. 

The tangent line is vertical when the slope becomes infinite. We can check for this by looking for where the denominator (but not the numerator) of $\dd[y]{x}$ is equal to $0$.

Setting the denominator equal to $0$ gives us $ -\sin(\theta)+\cos^2(\theta)-\sin^2(\theta)=0$.  

Since two of the terms involve $\sin(\theta)$ we can convert the cosine term by writing $\cos^2(\theta)=1-\sin^2(\theta)$. 

We then get $-2\sin^2(\theta)-\sin(\theta)+1=0$. 

This is a quadratic polynomial in $\sin(\theta)$ so we can try to factor it. We obtain $(2\sin(\theta)-1)(\sin(\theta)+1)=0$. 


Setting $2\sin(\theta)-1=0$ gives us $\theta=\answer{ \frac{\pi}{6}}$ and $\theta=\answer{\frac{5\pi}{6}}$. 


and setting $\sin(\theta)+1=0$ gives us $\theta=\answer{ \frac{3\pi}{2}}$. 

(Above list the $\theta$ values in order from smallest to largest and note we take $\theta$ to lie in the interval $[0, 2\pi)$).


\end{hint}

\end{exercise}
\end{exercise}
\end{document}
