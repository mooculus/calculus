\documentclass{ximera}

\newcommand{\RR}{\mathbb R}
\renewcommand{\d}{\,d}
\newcommand{\dd}[2][]{\frac{d #1}{d #2}}
\renewcommand{\l}{\ell}
\newcommand{\ddx}{\frac{d}{dx}}
\newcommand{\dfn}{\textbf}
\newcommand{\eval}[1]{\bigg[ #1 \bigg]}


\outcome{Determine if a series converges using the alternating series test.}

\title[Dig-In:]{The alternating series test}
\author{Tom Needham}

\begin{document}
\begin{abstract}
Alternating series are series whose terms alternate in sign between positive and negative. There is a powerful convergence test for alternating series. 
\end{abstract}
\maketitle

Many of the series convergence tests that have been introduced so far are stated with the assumption that all terms in the series are nonnegative. Indeed, this condition is assumed in the Integral Test, Ratio Test, Root Test, Comparison Test and Limit Comparison Test. In this section, we study series whose terms are not assumed to be strictly positive. In particular, we are interested in series whose terms alternate between positive and negative (aptly named \emph{alternating series}). It turns out that there is a powerful test for determining that a series of this form converges. 

\begin{definition}
Let $\{a_n\}_{n=1}^\infty$ be a sequence of positive numbers. An \dfn{alternating series} is
a series of the form 
\[
\sum_{n=1}^\infty (-1)^n a_n = -a_1 + a_2 - a_3 + a_4 - \cdots
\]
or of the form
$$
\sum_{n=1}^\infty (-1)^{n+1}a_n = a_1 - a_2 + a_3 - a_4 + \cdots.
$$

As usual, this definition can be modified to include series whose indexing starts somewhere other than $n=1$. 
\end{definition}

\begin{example}
The geometric series
$$
\sum_{n=0}^\infty \left(-\frac{2}{3}\right)^n = \sum_{n=0}^\infty \left(-1\right)^n \left(\frac{2}{3}\right)^n = 1 - \frac{2}{3} + \frac{4}{9} - \frac{8}{27} + \cdots
$$
is alternating. In general, a geometric series with ratio $r < 0$ is alternating, since
$$
\sum_{n=0}^\infty a r^n = \sum_{n=0}^\infty (-1)^n a  |r|^n.
$$
\end{example}

\begin{example}
Recall the harmonic series
$$
\sum_{n=1}^\infty \frac{1}{n} = 1 + \frac{1}{2} + \frac{1}{3} + \frac{1}{4} + \cdots,
$$
which is perhaps the simplest example of a divergent series whose terms approach zero as $n$ approaches $\infty$. 
A similarly important example is the \dfn{alternating harmonic series}
$$
\sum_{n=1}^\infty \frac{(-1)^{n+1}}{n} = 1 - \frac{1}{2} + \frac{1}{3} - \frac{1}{4} + \cdots.
$$
The terms of this series, of course, still approach zero, and their absolute values are monotone decreasing. Because the series is alternating, it turns out that this is enough to guarantee that it converges. This is formalized in the following theorem.
\end{example}

\begin{theorem}[Alternating Series Test]\index{alternating series test}
Let $\{a_n\}$ be a positive, nonincreasing sequence where
$\lim_{n\to\infty}a_n=0$. Then
\[
\sum_{n=1}^\infty (-1)^{n}a_n \qquad \text{and}\qquad \sum_{n=1}^\infty (-1)^{n+1}a_n 
\]
converge.
\end{theorem}

Compared to our convergence tests for series with strictly positive terms, this test is strikingly simple. Let us examine why it might be true by considering the partial sums of the alternating harmonic series. The first few partial sums with odd index are given by 
\begin{align*}
s_1 &= 1 \\
s_3 &= 1 - \frac{1}{2} + \frac{1}{3} = s_1 - \left(\frac{1}{2} - \frac{1}{3}\right) = \frac{5}{6} \\
s_5 &= s_3 - \frac{1}{4} + \frac{1}{5} = s_3 - \left( \frac{1}{4} - \frac{1}{5}\right) = \frac{47}{60}.
\end{align*}
Note that a general odd partial sum is of the form
$$
s_{2n+1} = s_{2n-1} - \left(\frac{1}{2n} - \frac{1}{2n+1}\right),
$$
and the quantity in the parentheses is positive. We conclude that:

\begin{question}
The sequence $\{s_1,s_3,s_5,\ldots\}$ of odd partial sums defined above is 
\begin{prompt}
        \begin{multipleChoice}
          \choice{increasing}
          \choice[correct]{decreasing}
        \end{multipleChoice}
 \end{prompt}
 \end{question}
        
        
%Observe that the sequence $\{s_1,s_3,s_5,\ldots\}$ is decreasing. Indeed, a general odd partial sum is of the form
%$$
%s_{2n+1} = s_{2n-1} - \left(\frac{1}{2n} - \frac{1}{2n+1}\right),
%$$
%and the quantity in the parentheses is positive. 

Moreover, the sequence of odd partial sums is bounded below by zero, since
$$
s_{2n+1} = \left(1 - \frac{1}{2}\right) + \left(\frac{1}{3} - \frac{1}{4}\right) + \cdots + \left(\frac{1}{2n-1}-\frac{1}{2n}\right) + \frac{1}{2n+1},
$$
and each quantity in parentheses is positive. 

\begin{question}
We conclude that the sequence $\{s_1,s_3,s_5,\ldots\}$ of odd partial must converge to a finite limit by applying
\begin{prompt}
        \begin{multipleChoice}
          \choice{The Fundamental Theorem of Calculus.}
          \choice[correct]{The Monotone Convergence Theorem.}
          \choice{The Ratio Test.}
        \end{multipleChoice}
 \end{prompt}
 \end{question}
 
%The Monotone Convergence Theorem therefore implies that the sequence $\{s_1,s_3,s_5,\cdots\}$ of odd partial sums has a finite limit. 

Next consider the sequence
\begin{align*}
s_2 &= 1 - \frac{1}{2} = \frac{1}{2} \\
s_4 &= 1- \frac{1}{2} + \frac{1}{3} - \frac{1}{4} = \frac{7}{12} \\
s_6 &= 1 - \frac{1}{2} + \frac{1}{3} - \frac{1}{4} + \frac{1}{5} - \frac{1}{6} = \frac{37}{60} \\
&\vdots
\end{align*}
of even partial sums.

\begin{question}
The sequence $\{s_2,s_4,s_6,\ldots\}$ of even partial sums defined above is 
\begin{prompt}
        \begin{multipleChoice}
          \choice[correct]{increasing and bounded, and therefore converges by the Monotone Convergence Theorem.}
          \choice{decreasing and bounded, and therefore converges by the Monotone Convergence Theorem.}
        \end{multipleChoice}
 \end{prompt}
 \end{question}
 
Finally, we use
$$
\lim_{n \rightarrow \infty} \left(s_{2n+1} - s_{2n} \right) = \lim_{n \rightarrow \infty} \frac{1}{2n+1} = 0
$$
and the fact that the limits of the sequences $\{s_{2n+1}\}$ and $\{s_{2n}\}$ are finite to conclude that
$$
\lim_{n \rightarrow \infty} s_{2n+1} = \lim_{n \rightarrow \infty} s_{2n} = \lim_{n\rightarrow \infty} s_n.
$$
It follows that the alternating harmonic series converges. 

With slight modification, the argument given above can be used to prove that the Alternating Series Test holds in general.


\begin{example}
Does the series $\sum_{n=1}^\infty (-1)^n\frac{n}{n^2+1}$ converge?

The terms of the sequence $\left\{n/(n^2+1)\right\}$ are positive and nonincreasing, so we can apply the Alternating Series Test. Since
$$
\lim_{n \rightarrow \infty} \frac{n}{n^2+1} = 0,
$$
the Alternating Series Test implies that the series converges.
\end{example}


    \begin{question}
      Does the alternating series test apply to the series
      \[
      \sum_{n=1}^\infty (-1)^{n+1}\frac{|\sin n|}{n^2}?
      \]
      \begin{prompt}
        \begin{multipleChoice}
          \choice{yes}
          \choice[correct]{no}
        \end{multipleChoice}
        \begin{feedback}
          The underlying sequence is $\seq{a_n} = |\sin n|/n$. This
          sequence is positive and approaches $0$ as
          $n\to\infty$. However, it is not a decreasing sequence; the
          value of $|\sin n|$ oscillates between $0$ and $1$ as
          $n\to\infty$. We cannot remove a finite number of terms to
          make $\seq{a_n}$ decreasing, therefore we cannot apply the
          alternating series test.
	  
          Keep in mind that this does not mean we conclude the series
          diverges; in fact, it does converge. We are just unable to
          conclude this based on the alternating series test.
        \end{feedback}
      \end{prompt}
    \end{question}






\end{document}
