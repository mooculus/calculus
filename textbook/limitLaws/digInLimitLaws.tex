\documentclass{ximera}

\outcome{Learn things.}

\newcommand{\RR}{\mathbb R}
\renewcommand{\d}{\,d}
\newcommand{\dd}[2][]{\frac{d #1}{d #2}}
\renewcommand{\l}{\ell}
\newcommand{\ddx}{\frac{d}{dx}}
\newcommand{\dfn}{\textbf}
\newcommand{\eval}[1]{\bigg[ #1 \bigg]}

\title{Limit laws}
\begin{document}
\begin{abstract}
ABSTRACT
\end{abstract}
\maketitle

We learned previously that just estimating limits by plugging in
nearby values does not always give the correct answer.  We need some
theorems that tell us how to quickly find limits, at least in the most
common cases.  In this section, we present a handful of tools to
compute many limits without explicitly working with the definition of
limit. Each of these could be proved directly as we did in the
previous section.

\begin{theorem}[Limit Laws]\index{limit laws}\label{theorem:limit-laws}
Suppose that $\lim_{x\to a}f(x)=L$, $\lim_{x\to a}g(x)=M$, $k$ is some
constant, $b$ is a nonzero constant, and $n$ is a positive integer.
\begin{description}
\item[\textbf{Limit of a Constant}] $\lim_{x\to a} k = k$.
\item[\textbf{Limit of $\boldsymbol{x}$}] $\lim_{x\to a}x =a$.
\item[\textbf{Limit of $\boldsymbol{\sin(x)}$}] $\lim_{x\to a} \sin(x) = \sin(a)$.
\item[\textbf{Limit of $\boldsymbol{b^x}$}] $\lim_{x\to a} b^x = b^a$.
\item[\textbf{Constant Multiple Law}] $\lim_{x\to a} kf(x) = k\lim_{x\to a}f(x)=kL$.
\item[\textbf{Sum/Difference Law}] $\lim_{x\to a} (f(x) \pm g(x)) = \lim_{x\to a}f(x) \pm \lim_{x\to a}g(x)=L \pm M$.  
\item[\textbf{Product Law}] $\lim_{x\to a} (f(x)g(x)) = \lim_{x\to a}f(x)\cdot\lim_{x\to a}g(x)=LM$. 
\item[\textbf{Quotient Law}] $\lim_{x\to a} \frac{f(x)}{g(x)} =
  \frac{\lim_{x\to a}f(x)}{\lim_{x\to a}g(x)}=\frac{L}{M}$, if $M\ne0$.
\item[\textbf{Power Law}] $\lim_{x\to a} f(x)^n = \left(\lim_{x\to a}f(x)\right)^n=L^n$. 
\item[\textbf{Root Law}] $\lim_{x\to a} \sqrt[n]{f(x)}= \sqrt[n]{\lim_{x\to  a}f(x)}=\sqrt[n]{L}$
  provided that if $n$ is even, then $f(x)\ge 0$ near $a$.
\item[\textbf{Composition Law}] If $\lim_{x\to a}g(x)=M$ and
  $\lim_{x\to M}f(x) = f(M)$, then $\lim_{x\to a} f(g(x)) = f(M)$.
\end{description}
\label{thm:limit laws}
\end{theorem}


\begin{example}
Find $\lim_{x\to1}(x-2)$.
\begin{explanation}
We do this ``by the book'' and explicitly use our limit laws. First write
\[
\lim_{x\to1}(x-2) = \lim_{x\to1}(x)-\lim_{x\to1}(2)
\]
What limit law was used in this step? $\answer[given]{\text{Sum/Difference Law}}$\\
\[
=\lim_{x\to1}(x)-2
\]
What limit law was used in this step? $\answer[given]{\text{Limit of a Constant}}$\\
\[
=1-2
\]
What limit law was used in this step? $\answer[given]{\text{Limit of $x$}}$\\
\[
=-1.
\]
Hence we se that $\lim_{x\to1}(x-2) = 1$.
%%%BADBAD This should probably be drop down from above
%%How do I make text answers?
\end{explanation}
\end{example}

This first example might seem trivial, but remember how we tricked
ourselves in the previous section.  The limit laws allow us to know
that as long as we follow them, we are actually getting the right
answer for the limit.

\begin{example}
Compute $\lim_{x\to 1}\frac{x^2-3x+5}{x-2}$. 
\begin{explanation}
Using limit laws, 
\begin{align*}
\lim_{x\to 1}\frac{x^2-3x+5}{x-2}&=
\frac{\lim_{x\to 1}(x^2-3x+5)}{\lim_{x\to1}(x-2)}  && \text{(assuming $\lim_{x\to1}(x-2) \neq 0$)} \\
&=\frac{\lim_{x\to 1}(x^2)-\lim_{x\to1}(3x)+\lim_{x\to1}(5)}{\lim_{x\to1}(x)-\lim_{x\to1}(2)} \\
&=\frac{\left(\lim_{x\to 1}x\right)^2-3(\lim_{x\to1}x)+5}{\lim_{x\to1}(x)-2} \\
&=\frac{1^2-3\cdot1+5}{1-2} \\
&=\frac{1-3+5}{-1} = -3.
\end{align*}
\textit{On your own}: Go through each step in this example an determine which limit laws were used.  It is important to make sure you are only using real limit laws and not making up your own rules!

%Can we use color in the book.  For example,  \text{\ \ (assuming $\lim_{x\to1}(x-2) \neq 0$)}  should be a different color.
%% BADBAD For the print book, we *could* but probably don't want to, since printing would be hard.
%% however, it would be good to do this for the online version...
\end{explanation}
\end{example}


\end{document}
