\documentclass{ximera}

\newcommand{\RR}{\mathbb R}
\renewcommand{\d}{\,d}
\newcommand{\dd}[2][]{\frac{d #1}{d #2}}
\renewcommand{\l}{\ell}
\newcommand{\ddx}{\frac{d}{dx}}
\newcommand{\dfn}{\textbf}
\newcommand{\eval}[1]{\bigg[ #1 \bigg]}


\author{Jim Fowler}
\license{Creative Commons 4.0 By-SA}

\begin{document}

\begin{exercise}
  Use the method of Lagrange multipliers to find the largest value of 
$$
f(x,y) = x + xy
$$
subject to the constraint $x^2 + y^2 = 1$.

The largest value is $\answer{\frac{3 \sqrt{3}}{4}}$.

\begin{hint}
We set $g(x,y) = x^2 + y^2$, and we set up the equations $\nabla f = \lambda \cdot \nabla g$, i.e.
,
\begin{align*}
\frac{\partial f}{\partial x} = 1 + y &= \lambda \cdot 2x = \lambda \cdot \frac{\partial g}{\partial x}\\
\frac{\partial f}{\partial y} = x &= \lambda \cdot 2y = \lambda \cdot \frac{\partial g}{\partial y
}.
\end{align*}
\end{hint}

\begin{hint}
Multiplying the first equation by $y$ and the second equation by $x$, we find
\begin{align*}
y + y^2 &= \lambda 2xy \\
x^2 &= \lambda 2xy
\end{align*}
\end{hint}

\begin{hint}
And so $y + y^2 = x^2$.  But by the constraint, $x^2 = 1 - y^2$, so $y + y^2 = 1 - y^2$.  Rearranging, we get
$$
2 y^2 + y - 1 = 0,
$$
so by the quadratic equation
$$
y = \frac{-1 \pm \sqrt{1 + 8 \cdot 1}}{4} = \frac{-1 \pm 3}{4}
$$
so $y = 1/2$ or $y = -1$.
\end{hint}

\begin{hint}
If $y = -1$, then $x = 0$, so $f(x,y) = 0$.  If $y = 1/2$, then $x = \pm \frac{\sqrt{3}}{2}$, so i
f we take $x$ positive,
$$
f(x,y) = \frac{\sqrt{3}}{2} + \frac{\sqrt{3}}{2} \cdot \frac{1}{2} = \frac{3 \sqrt{3}}{4}
$$
is the largest value that $f$ achieves; if we take $x =
-\frac{\sqrt{3}}{2}$ and $y = 1/2$, we get the smallest value that $f$
achieves.  Many students make a mistake by thinking $0$ was the
smallest value.
\end{hint}

\end{exercise}
\end{document}
