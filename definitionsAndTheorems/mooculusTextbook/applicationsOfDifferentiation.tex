\chapter{Applications of Differentiation}



\section{L'H\^{o}pital's Rule}


Derivatives allow us to take problems that were once difficult to
solve and convert them to problems that are easier to solve. Let us
consider l'H\^{o}pital's rule:

\marginnote[1in]{L'H\^opital's rule applies even when $\lim_{x\to a}f(x) =
  \pm \infty$ and $\lim_{x\to a}g(x) = \mp \infty$. See Example~\ref{example:xlnx infty}.}
\begin{mainTheorem}[L'H\^opital's Rule]\index{l'H\^opital's Rule} 
Let $f(x)$ and $g(x)$ be functions that are differentiable near $a$.  If
\[
\lim_{x \to a} f(x) = \lim_{x \to a}g(x) = 0 \qquad \text{or} \pm \infty,
\]
and $\lim_{x \to a} \frac{f'(x)}{g'(x)}$ exists, and $g'(x) \neq 0$
for all $x$ near $a$, then 
\[
\lim_{x \to a} \frac{f(x)}{g(x)} = \lim_{x \to a} \frac{f'(x)}{g'(x)}.
\]
\end{mainTheorem}
This theorem is somewhat difficult to prove, in part because it
incorporates so many different possibilities, so we will not prove it
here. 

\break

L'H\^{o}pital's rule allows us to investigate limits of
\textit{indeterminate form}.

\begin{definition}[List of Indeterminate Forms]\index{indeterminate form}\hfil
\begin{itemize}
\item[\textbf{0/0}] This refers to a limit of the form $\lim_{x\to a}
  \frac{f(x)}{g(x)}$ where $f(x)\to 0$ and $g(x)\to 0$ as $x\to a$.
\item[\textbf{$\pmb\infty$/$\pmb\infty$}] This refers to a limit of the form $\lim_{x\to a}
  \frac{f(x)}{g(x)}$ where $f(x)\to \infty$ and $g(x)\to \infty$ as $x\to a$.
\item[\textbf{0\,$\pmb{\cdot\infty}$}] This refers to a limit of the form $\lim_{x\to a}
  \left(f(x)\cdot g(x)\right)$ where $f(x)\to 0$ and $g(x)\to \infty$ as $x\to a$.
\item[\textbf{$\pmb\infty$--$\pmb\infty$}] This refers to a limit of the form $\lim_{x\to a}\left(
  f(x)-g(x)\right)$ where $f(x)\to \infty$ and $g(x)\to \infty$ as $x\to a$.

\item[\textbf{1$^{\pmb\infty}$}] This refers to a limit of the form $\lim_{x\to a}
  f(x)^{g(x)}$ where $f(x)\to 1$ and $g(x)\to \infty$ as $x\to a$.
\item[\textbf{0$^\text{0}$}] This refers to a limit of the form $\lim_{x\to a}
  f(x)^{g(x)}$ where $f(x)\to 0$ and $g(x)\to 0$ as $x\to a$.
\item[\textbf{$\pmb\infty^\text{0}$}] This refers to a limit of the form $\lim_{x\to a}
  f(x)^{g(x)}$ where $f(x)\to \infty$ and $g(x)\to 0$ as $x\to a$.
\end{itemize}
In each of these cases, the value of the limit is \textbf{not} immediately
obvious. Hence, a careful analysis is required!
\end{definition}

Our first example is the computation of a limit that was somewhat
difficult before, see Example~\ref{example:sinx/x}. Note, this is an
example of the indeterminate form $0/0$.

\begin{example}[0/0]\label{example:sinx/x-lhopital}
Compute
\[
\lim_{x\to 0} \frac{\sin(x)}{x}.
\]
\end{example}

\begin{solution}
Set $f(x) = \sin(x)$ and $g(x) = x$.  Since both $f(x)$ and $g(x)$ are
differentiable functions at $0$, and 
\[
\lim_{x \to 0} f(x) = \lim_{x \to 0}g(x) = 0,
\]
this situation is ripe for l'H\^opital's Rule. Now
\[
f'(x) = \cos(x) \qquad\text{and}\qquad g'(x) = 1.
\] 
L'H\^opital's rule tells us that 
\[
\lim_{x \to 0} \frac{\sin(x)}{x} = \lim_{x \to 0} \frac{\cos(x)}{1} = 1.
\]
\end{solution}


\begin{marginfigure}[-1in]
\begin{tikzpicture}
	\begin{axis}[
            xmin=-1.6,xmax=1.6,ymin=-1.5,ymax=1.5,
            axis lines=center,
            xtick={-1.57, 0, 1.57},
            xticklabels={$-\pi/2$, $0$, $\pi/2$},
            ytick={-1,1},
            %ticks=none,
            %width=3in,
            %height=2in,
            unit vector ratio*=1 1 1,
            xlabel=$x$, ylabel=$y$,
            every axis y label/.style={at=(current axis.above origin),anchor=south},
            every axis x label/.style={at=(current axis.right of origin),anchor=west},
          ]        
          \addplot [very thick, penColor, samples=100,smooth, domain=(-1.6:1.6)] {sin(deg(x))};
          \addplot [very thick, penColor2] {x};
          \node at (axis cs:1,.6) [penColor] {$f(x)$};
          \node at (axis cs:-1,-1.2) [penColor2] {$g(x)$};
        \end{axis}
\end{tikzpicture}
\caption{A plot of $f(x)=\sin(x)$ and $g(x) = x$. Note how the tangent
  lines for each curve are coincident at $x=0$.}
\label{example:sinx and x}
\end{marginfigure}


From this example, we gain an intuitive feeling for why l'H\^opital's
rule is true: If two functions are both $0$ when $x=a$, and if their
tangent lines have the same slope, then the functions coincide as $x$
approaches $a$. See Figure~\ref{example:sinx and x}. 




Our next set of examples will run through the remaining indeterminate
forms one is likely to encounter.

\begin{example}[$\pmb\infty$/$\pmb\infty$] Compute 
\[
\lim_{x\to \pi/2+} \frac{\sec(x)}{\tan(x)}.
\]
\end{example}

\begin{solution}
Set $f(x) = \sec(x)$ and $g(x) = \tan(x)$. Both $f(x)$ and $g(x)$
are differentiable near $\pi/2$. Additionally,
\[
\lim_{x \to \pi/2+} f(x) = \lim_{x \to \pi/2+}g(x) = -\infty.
\]
This situation is ripe for l'H\^opital's Rule. Now 
\[
f'(x) = \sec(x)\tan(x) \qquad\text{and}\qquad g'(x) = \sec^2(x).
\]
L'H\^opital's rule tells us that 
\[
\lim_{x\to \pi/2+} \frac{\sec(x)}{\tan(x)} = \lim_{x\to \pi/2+}
\frac{\sec(x)\tan(x)}{\sec^2(x)} = \lim_{x\to \pi/2+} \sin(x) =
1.
\]
\end{solution}



\begin{example}[0\,$\pmb{\cdot\infty}$]\label{example:xlnx infty} 
Compute 
\[
\lim_{x\to 0+} x\ln x.
\]
\end{example}

\begin{solution}
This doesn't appear to be suitable for l'H\^opital's Rule. As $x$
approaches zero, $\ln x$ goes to $-\infty$, so the product looks like
\[
(\text{something very small})\cdot (\text{something very large and
  negative}).
\] 
This product could be anything---a careful analysis is required.
Write
\[
x\ln x = \frac{\ln x}{x^{-1}}.
\]
Set $f(x) = \ln(x)$ and $g(x) = x^{-1}$.  Since both functions are differentiable near zero and 
\[
\lim_{x\to 0+} \ln(x) = -\infty\qquad\text{and}\qquad \lim_{x\to 0+} x^{-1} = \infty,
\]
we may apply l'H\^opital's rule. Write
\[
f'(x) = x^{-1}\qquad \text{and}\qquad g'(x) = -x^{-2},
\]
so
\[
\lim_{x\to 0+} x\ln x = \lim_{x\to 0+} \frac{\ln x}{x^{-1}} = \lim_{x\to 0+} \frac{x^{-1}}{-x^{-2}} =\lim_{x\to 0+} -x = 0.
\]
One way to interpret this is that since $\lim_{x\to 0^+}x\ln x = 0$,
the function $x$ approaches zero much faster than $\ln x$ approaches
$-\infty$.
\end{solution}

\subsection*{Indeterminate Forms Involving Subtraction}

There are two basic cases here, we'll do an example of each.

\begin{example}[$\pmb\infty$--$\pmb\infty$]
Compute
\[
\lim_{x\to 0} \left(\cot(x) - \csc(x)\right).
\]
\end{example}

\begin{solution}
Here we simply need to write each term as a fraction,
\begin{align*}
\lim_{x\to 0} \left(\cot(x) - \csc(x)\right) &= \lim_{x\to 0} \left(\frac{\cos(x)}{\sin(x)} - \frac{1}{\sin(x)}\right)\\
&= \lim_{x\to 0} \frac{\cos(x)-1}{\sin(x)} 
\end{align*}
Setting $f(x) = \cos(x)-1$ and $g(x)=\sin(x)$, both functions are differentiable near zero and 
\[
\lim_{x\to 0}(\cos(x)-1)=\lim_{x\to 0}\sin(x) = 0.
\]
We may now apply l'H\^opital's rule. Write
\[
f'(x) = -\sin(x)\qquad \text{and}\qquad g'(x) = \cos(x),
\]
so
\[
\lim_{x\to 0} \left(\cot(x) - \csc(x)\right) = \lim_{x\to 0} \frac{\cos(x)-1}{\sin(x)} = \lim_{x\to 0} \frac{-\sin(x)}{\cos(x)} =0.
\]
\end{solution}


Sometimes one must be slightly more clever. 

\begin{example}[$\pmb\infty$--$\pmb\infty$]
Compute
\[
\lim_{x\to\infty}\left(\sqrt{x^2+x}-x\right).
\]
\end{example}

\begin{solution}
Again, this doesn't appear to be suitable for l'H\^opital's Rule. A bit of algebraic manipulation will help. Write
\begin{align*}
\lim_{x\to\infty}\left(\sqrt{x^2+x}-x\right) &= \lim_{x\to\infty}\left(x\left(\sqrt{1+1/x}-1\right)\right)\\
&=\lim_{x\to\infty}\frac{\sqrt{1+1/x}-1}{x^{-1}}
\end{align*}
Now set $f(x) = \sqrt{1+1/x}-1$, $g(x) = x^{-1}$. Since both
  functions are differentiable for large values of $x$ and 
\[
\lim_{x\to\infty} (\sqrt{1+1/x}-1) = \lim_{x\to\infty}x^{-1} = 0, 
\]
we may apply l'H\^opital's rule. Write
\[
f'(x) = (1/2)(1+1/x)^{-1/2}\cdot(-x^{-2}) \qquad \text{and}\qquad g'(x) = -x^{-2}
\]
so
\begin{align*}
\lim_{x\to\infty}\left(\sqrt{x^2+x}-x\right) &= \lim_{x\to\infty}\frac{\sqrt{1+1/x}-1}{x^{-1}} \\
&= \lim_{x\to\infty}\frac{(1/2)(1+1/x)^{-1/2}\cdot(-x^{-2})}{-x^{-2}} \\
&= \lim_{x\to\infty} \frac{1}{2\sqrt{1+1/x}}\\
&= \frac{1}{2}.
\end{align*}
\end{solution}


\subsection*{Exponential Indeterminate Forms}

There is a standard trick for dealing with the indeterminate forms
\[
1^\infty,\qquad 0^0,\qquad \infty^0.
\]
Given $u(x)$ and $v(x)$ such that
\[
\lim_{x\to a}u(x)^{v(x)}
\]
falls into one of the categories described above, rewrite as
\[
\lim_{x\to a}e^{v(x)\ln(u(x))}
\]
and then examine the limit of the exponent
\[
\lim_{x\to a} v(x)\ln(u(x)) = \lim_{x\to a} \frac{\ln(u(x))}{v(x)^{-1}}
\]
using l'H\^opital's rule.  Since these forms are all very similar, we
will only give a single example.


\begin{example}[1$^{\pmb\infty}$]
Compute
\[
\lim_{x\to \infty}\left(1 + \frac{1}{x}\right)^x.
\]
\end{example}

\begin{solution}
Write
\[
\lim_{x\to \infty}\left(1 + \frac{1}{x}\right)^x = \lim_{x\to \infty}e^{x\ln\left(1 + \frac{1}{x}\right)}.
\]
So now look at the limit of the exponent
\[
\lim_{x\to\infty} x\ln\left(1 + \frac{1}{x}\right) = \lim_{x\to\infty} \frac{\ln\left(1 + \frac{1}{x}\right)}{x^{-1}}.
\]
Setting $f(x) = \ln\left(1 + \frac{1}{x}\right)$ and $g(x) = x^{-1}$,
both functions are differentiable for large values of $x$ and
\[
\lim_{x\to \infty}\ln\left(1 + \frac{1}{x}\right)=\lim_{x\to \infty}x^{-1} = 0.
\]
We may now apply l'H\^opital's rule. Write
\[
f'(x) = \frac{-x^{-2}}{1 + \frac{1}{x}}\qquad\text{and}\qquad g'(x) = -x^{-2},
\]
so
\begin{align*}
\lim_{x\to\infty} \frac{\ln\left(1 + \frac{1}{x}\right)}{x^{-1}} &= \lim_{x\to\infty} \frac{\frac{-x^{-2}}{1 + \frac{1}{x}}}{-x^{-2}} \\
&=\lim_{x\to\infty} \frac{1}{1 + \frac{1}{x}}\\
&=1.
\end{align*}
Hence, 
\[
\lim_{x\to \infty}\left(1 + \frac{1}{x}\right)^x = \lim_{x\to \infty}e^{x\ln\left(1 + \frac{1}{x}\right)} =e^{1} = e.
\]
\end{solution}











% Most from Keisler
\begin{exercises}

\noindent Compute the limits.

\twocol

\begin{exercise} $\lim_{x\to 0} {\cos x -1\over \sin x}$
\begin{answer} $0$
\end{answer}\end{exercise}

\begin{exercise} $\lim_{x\to \infty} {e^x\over x^3}$
\begin{answer} $\infty$
\end{answer}\end{exercise}

\begin{exercise} $\lim_{x\to \infty} \sqrt{x^2+x}-\sqrt{x^2-x}$
\begin{answer} $1$
\end{answer}\end{exercise}

\begin{exercise} $\lim_{x\to \infty} {\ln x\over x}$
\begin{answer} $0$
\end{answer}\end{exercise}

\begin{exercise} $\lim_{x\to \infty} {\ln x\over \sqrt{x}}$
\begin{answer} $0$
\end{answer}\end{exercise}

\begin{exercise} $\lim_{x\to\infty} {e^x + e^{-x}\over e^x -e^{-x}}$
\begin{answer} 1
\end{answer}\end{exercise}

\begin{exercise} $\lim_{x\to0}{\sqrt{9+x}-3\over x}$
\begin{answer} $1/6$
\end{answer}\end{exercise}

\begin{exercise} $\lim_{t\to1+}{(1/t)-1\over t^2-2t+1}$
\begin{answer} $-\infty$
\end{answer}\end{exercise}

\begin{exercise} $\lim_{x\to2}{2-\sqrt{x+2}\over 4-x^2}$
\begin{answer} $1/16$
\end{answer}\end{exercise}

\begin{exercise} $\lim_{t\to\infty}{t+5-2/t-1/t^3\over 3t+12-1/t^2}$
\begin{answer} $1/3$
\end{answer}\end{exercise}

\begin{exercise} $\lim_{y\to\infty}{\sqrt{y+1}+\sqrt{y-1}\over y}$
\begin{answer} $0$
\end{answer}\end{exercise}

\begin{exercise} $\lim_{x\to1}\frac{\sqrt{x}-1}{\sqrt[3]{x}-1}$
\begin{answer} $3/2$
\end{answer}\end{exercise}

\begin{exercise} $\lim_{x\to0}{(1-x)^{1/4}-1\over x}$
\begin{answer} $-1/4$
\end{answer}\end{exercise}

\begin{exercise} $\lim_{t\to 0}{\left(t+{1\over t}\right)((4-t)^{3/2}-8)}$
\begin{answer} $-3$
\end{answer}\end{exercise}

\begin{exercise} $\lim_{t\to 0+}\left({1\over t}+{1\over\sqrt{t}}\right)
(\sqrt{t+1}-1)$
\begin{answer} $1/2$
\end{answer}\end{exercise}

\begin{exercise} $\lim_{x\to 0}{x^2\over\sqrt{2x+1}-1}$
\begin{answer} $0$
\end{answer}\end{exercise}

\begin{exercise} $\lim_{u\to 1}{(u-1)^3\over (1/u)-u^2+3/u-3}$
\begin{answer} $0$
\end{answer}\end{exercise}

\begin{exercise} $\lim_{x\to 0}{2+(1/x)\over 3-(2/x)}$
\begin{answer} $-1/2$
\end{answer}\end{exercise}

\begin{exercise} $\lim_{x\to 0+}{1+5/\sqrt{x}\over 2+1/\sqrt{x}}$
\begin{answer} $5$
\end{answer}\end{exercise}

\begin{exercise} $\lim_{x\to 0+}{3+x^{-1/2}+x^{-1}\over 2+4x^{-1/2}}$
\begin{answer} $\infty$
\end{answer}\end{exercise}

\begin{exercise} $\lim_{x\to\infty}{x+x^{1/2}+x^{1/3}\over x^{2/3}+x^{1/4}}$
\begin{answer} $\infty$
\end{answer}\end{exercise}

\begin{exercise} $\lim_{t\to\infty}
{1-\sqrt{t\over t+1}\over 2-\sqrt{4t+1\over t+2}}$
\begin{answer} $2/7$
\end{answer}\end{exercise}

\begin{exercise} $\lim_{t\to\infty}{1-{t\over t-1}\over 1-\sqrt{t\over t-1}}$
\begin{answer} $2$
\end{answer}\end{exercise}

\begin{exercise} $\lim_{x\to-\infty}{x+x^{-1}\over 1+\sqrt{1-x}}$
\begin{answer} $-\infty$
\end{answer}\end{exercise}



\begin{exercise} $\lim_{x\to\pi/2}{\cos x\over (\pi/2)-x}$
\begin{answer} $1$
\end{answer}\end{exercise}

\begin{exercise} $\lim_{x\to0}{e^x-1\over x}$
\begin{answer} $1$
\end{answer}\end{exercise}

\begin{exercise} $\lim_{x\to0}{x^2\over e^x-x-1}$
\begin{answer} $2$
\end{answer}\end{exercise}

\begin{exercise} $\lim_{x\to1}{\ln x\over x-1}$
\begin{answer} $1$
\end{answer}\end{exercise}

\begin{exercise} $\lim_{x\to0}{\ln(x^2+1)\over x}$
\begin{answer} $0$
\end{answer}\end{exercise}

\begin{exercise} $\lim_{x\to1}{x\ln x\over x^2-1}$
\begin{answer} $1/2$
\end{answer}\end{exercise}

\begin{exercise} $\lim_{x\to0}{\sin(2x)\over\ln(x+1)}$
\begin{answer} $2$
\end{answer}\end{exercise}

\begin{exercise} $\lim_{x\to1}{x^{1/4}-1\over x}$
\begin{answer} $0$
\end{answer}\end{exercise}

\begin{exercise} $\lim_{x\to1+}{\sqrt{x}\over x-1}$
\begin{answer} $\infty$
\end{answer}\end{exercise}

\begin{exercise} $\lim_{x\to1}{\sqrt{x}-1\over x-1}$
\begin{answer} $1/2$
\end{answer}\end{exercise}

\begin{exercise} $\lim_{x\to\infty}{x^{-1}+x^{-1/2}\over x+x^{-1/2}}$
\begin{answer} $0$
\end{answer}\end{exercise}

\begin{exercise} $\lim_{x\to\infty}{x+x^{-2}\over 2x+x^{-2}}$
\begin{answer} $1/2$
\end{answer}\end{exercise}

\begin{exercise} $\lim_{x\to\infty}{5+x^{-1}\over 1+2x^{-1}}$
\begin{answer} $5$
\end{answer}\end{exercise}

\begin{exercise} $\lim_{x\to\infty}{4x\over\sqrt{2x^2+1}}$
\begin{answer} $2\sqrt2$
\end{answer}\end{exercise}

\begin{exercise} $\lim_{x\to0}{3x^2+x+2\over x-4}$
\begin{answer} $-1/2$
\end{answer}\end{exercise}

\begin{exercise} $\lim_{x\to0}{\sqrt{x+1}-1\over \sqrt{x+4}-2}$
\begin{answer} $2$
\end{answer}\end{exercise}

\begin{exercise} $\lim_{x\to0}{\sqrt{x+1}-1\over \sqrt{x+2}-2}$
\begin{answer} $0$
\end{answer}\end{exercise}

\begin{exercise} $\lim_{x\to0+}{\sqrt{x+1}+1\over\sqrt{x+1}-1}$
\begin{answer} $\infty$
\end{answer}\end{exercise}

\begin{exercise} $\lim_{x\to0}{\sqrt{x^2+1}-1\over\sqrt{x+1}-1}$
\begin{answer} $0$
\end{answer}\end{exercise}

\begin{exercise} $\lim_{x\to\infty}{(x+5)\left({1\over 2x}+{1\over x+2}\right)}$
\begin{answer} $3/2$
\end{answer}\end{exercise}

\begin{exercise} $\lim_{x\to0+}{(x+5)\left({1\over 2x}+{1\over x+2}\right)}$
\begin{answer} $\infty$
\end{answer}\end{exercise}

\begin{exercise} $\lim_{x\to1}{(x+5)\left({1\over 2x}+{1\over x+2}\right)}$
\begin{answer} $5$
\end{answer}\end{exercise}

\begin{exercise} $\lim_{x\to2}{x^3-6x-2\over x^3+4}$
\begin{answer} $-1/2$
\end{answer}\end{exercise}

\begin{exercise} $\lim_{x\to2}{x^3-6x-2\over x^3-4x}$
\begin{answer} does not exist
\end{answer}\end{exercise}

\begin{exercise} $\lim_{x\to1+}{x^3+4x+8\over 2x^3-2}$
\begin{answer} $\infty$
\end{answer}\end{exercise}
\endtwocol
\end{exercises}








\section{The Derivative as a Rate}

The world is constantly changing around us. To simplify matters we
will only consider change in one dimension. This means that if we
think of a ball being tossed in the air, we will consider its vertical
movement separately from its lateral and forward movement.  To
understand how things change, we need to understand the \textit{rate}
of change. Let's start out with some rather basic ideas.

\begin{definition}\index{average rate of change}
Given a function $f(x)$, the \textbf{average rate of change} over the
interval $[a, a+\Delta x]$ is given by
\[
\frac{f(a+\Delta x) - f(a)}{\Delta x}.
\]
\end{definition}
\begin{marginfigure}
\begin{tikzpicture}
	\begin{axis}[
            xmin=0,xmax=12,ymin=0,ymax=600,
            axis lines=center,
            ytick={100,200,300,400,500,600},
            xtick={1,...,12},
            grid=both,
            grid style={dashed, gridColor},
            xlabel=$t$, ylabel=$d$,
            every axis y label/.style={at=(current axis.above origin),anchor=south},
            every axis x label/.style={at=(current axis.right of origin),anchor=west},
          ]                  
          
          \addplot[mark=none,penColor,very thick] coordinates { 

            (0,0) (1.5, 50.)  (4.5, 270)  (6.5, 270.) (10.5, 540.)  (12.,600.)        
          };
        \end{axis}
\end{tikzpicture}
\caption{Here we see a plot of the distance traveled on a $600$ mile road trip.}
\label{figure:road-trip}
\end{marginfigure}

\begin{example}
Suppose you drive a car on a $600$ mile road trip. Your distance from
home is recorded by the plot shown in
Figure~\ref{figure:road-trip}. What was your average velocity during
hours $4$--$8$ of your trip?
\end{example}

\begin{solution}
Examining Figure~\ref{figure:road-trip}, we see that we were around
$240$ miles from home at hour $4$, and $360$ miles from home at hour
$8$. Hence our average velocity was
\[
\frac{360-240}{8-4} = \frac{120}{4} = 30~\text{miles per hour.}
\]
\end{solution}

Of course if you look at Figure~\ref{figure:road-trip} closely, you
see that sometimes we were driving faster and other times we were
driving slower. To get more information, we need to know the
\textit{instantaneous rate of change}.

\begin{definition}\index{instantaneous rate of change}
Given a function, the \textbf{instantaneous rate of change} at $x=a$ is given by
\[
\left.\ddx f(x) \right|_{x=a}.
\] 
\end{definition}

\begin{example} 
Again suppose, you drive a car 600 mile road trip. Your distance from
home is recorded by the plot shown in
Figure~\ref{figure:road-trip}. What was your instantaneous velocity
$8$ hours into your trip?
\end{example}
\begin{solution}
Since the instantaneous rate of change is measured by the derivative,
we need to find the slope of the tangent line to the curve. At $7$
hours, the curve is growing at an essentially constant rate. In fact,
the growth rate seems to be constant from $(7,300)$ to
$(10,500)$. This gives us an instantaneous growth rate at hour $8$ of
about $200/3 \approx 67$ miles per hour.
\end{solution}




\subsection*{Physical Applications}

In physical applications, we are often concerned about
\textit{position}, \textit{velocity}, \textit{speed},
\textit{acceleration}.
\begin{align*}
p(t) &= \text{position with respect to time.}\\
v(t) &= p'(t) = \text{velocity with respect to time.}\\
s(t) &= |v(t)| = \text{speed, the absolute value of velocity.}\\
a(t) &=v'(t) = \text{acceleration with respect to time.}
\end{align*}

Let's see an example.

\begin{example}
The Mostar bridge in Bosnia is $25$ meters above the river
Neretva. For fun, you decided to dive off the bridge. Your position
$t$ seconds after jumping off is
\[
p(t) = -4.9t^2 + 25.
\]
When do you hit the water? What is your instantaneous velocity as you
enter the water?  What is your average velocity during your dive?
\end{example}
\begin{marginfigure}
\begin{tikzpicture}
	\begin{axis}[
            xmin=0,xmax=3,ymin=0,ymax=30,
            axis lines=center,
            xlabel=$t$, ylabel=$p$,
            every axis y label/.style={at=(current axis.above origin),anchor=south},
            every axis x label/.style={at=(current axis.right of origin),anchor=west},
          ]        
          \addplot [very thick, penColor,smooth] {-4.9*x^2+25};
        \end{axis}
\end{tikzpicture}
\caption{Here we see a plot of $p(t) = -4.9t^2 + 25$. Note, time is on
  the $t$-axis and vertical height is on the $p$-axis.}
\end{marginfigure}
\begin{solution}
To find when you hit the water, you must solve
\[
-4.9t^2 + 25 = 0
\]
Write
\begin{align*}
-4.9t^2 &= -25 \\
t^2 &\approx 5.1 \\ 
t &\approx 2.26.
\end{align*}
Hence after approximately $2.26$ seconds, you gracefully enter the
river.

Your instantaneous velocity is given by $p'(t)$. Write
\[
p'(t) = -9.8t,
\]
so your instantaneous velocity when you enter the water is
approximately $-9.8\cdot 2.26\approx -22$ meters per second.

Finally, your average velocity during your dive is given by
\[
\frac{p(2.26) -p(0)}{2.26} \approx \frac{0-25}{2.26} =
-11.06~\text{meters per second}.
\]
\end{solution}





\subsection*{Biological Applications}

In biological applications, we are often concerned with how animals
and plants grow, though there are numerous other applications too.

\begin{example}
A certain bacterium divides into two cells every 20 minutes. The
initial population of a culture is 120 cells. Find a formula for the
population.  What is the average growth rate during the first 4 hours?
What is the instantaneous growth rate of the population at 4 hours?
What rate is the population growing at 20 hours?
\end{example}
\begin{marginfigure}
\begin{tikzpicture}
	\begin{axis}[
            xmin=0,xmax=5,ymin=0,ymax=1000000,
            axis lines=center,
            xlabel=$t$, ylabel=$p$,
            every axis y label/.style={at=(current axis.above origin),anchor=south},
            every axis x label/.style={at=(current axis.right of origin),anchor=west},restrict y to domain=0:10000000,
          ]        
          \addplot [very thick, penColor, smooth] {120*2^(3*x)};
        \end{axis}
\end{tikzpicture}
\caption{Here we see a plot of $p(t) = 120\cdot 2^{3t}$. Note, time is on
  the $t$-axis and population is on the $p$-axis.}
\end{marginfigure}


\begin{solution}
Since we start with $120$ cells, and this population doubles every $20$
minutes, then the population doubles three times an hour. So the
formula for the population is
\[
p(t) = 120\cdot 2^{3t}
\]
where $t$ is time measured in hours.

Now, the average growth rate during the first $4$ hours is given by
\[
\frac{p(4)-p(0)}{4} =\frac{491520-120}{4} = 122850~\text{cells per hour.}
\]

We compute the instantaneous growth rate of the population with
\[
p'(t) = \ln(2)\cdot 360\cdot 2^{3t}.
\]
So $p'(4) \approx 1022087$ cells per hour. Note how fast $p(t)$ is
growing, this is why it is important to stop bacterial infections
fast!
\end{solution}


%% \subsection*{Economic Applications}

%% There are three main players when dealing with very basic economic
%% questions: The \textit{revenue}, the amount of money acquired from
%% selling a product; the \textit{cost}, the amount of money it takes to
%% produce an item; and the \textit{profit}, the revenue minus the cost.
%% \begin{align*}
%% r(x) &= \text{revenue from selling $x$ items.}\\
%% c(x) &= \text{cost of producing $x$ items.}\\
%% p(x) &= r(x)-c(x) = \text{the profit from selling $x$ items}.
%% \end{align*}
%% The rate of change in these functions is denoted by the word \textit{marginal}, hence 
%% \begin{align*}
%% \dd[r]{x} &= \text{the marginal revenue, how the revenue is changing when selling $x$ items.}\\
%% \dd[c]{x} &= \text{the marginal cost, how the cost is changing when selling $x$ items.}
%% \end{align*}

%% Let's see an example.

%% \begin{example}
%% A manufacturer sells backpacks at the price of \$$40$ each. The cost function for producing $x$ backpacks is
%% \[
%% c(x) = \frac{2100}{x} + 22x.
%% \]
%% Express profit as a function of backpacks sold. What is the average
%% profit after $10$ backpacks are sold? What is the marginal profit when
%% $10$ backpacks are sold?
%% \end{example}
%% \begin{marginfigure}
%% \begin{tikzpicture}
%% 	\begin{axis}[
%%             xmin=1,xmax=30,ymin=-2100,ymax=500,
%%             axis lines=center,
%%             xlabel=$x$, ylabel=$p$,
%%             every axis y label/.style={at=(current axis.above origin),anchor=south},
%%             every axis x label/.style={at=(current axis.right of origin),anchor=west},restrict y to domain=-2500:500,
%%           ]        
%%           \addplot [very thick, penColor,smooth, samples=100,domain=(1:30)] {-2100/x + 18*x};
%%         \end{axis}
%% \end{tikzpicture}
%% \caption{Here we see a plot of $p(x) = 18x - \frac{2100}{x}$. Note,
%%   this graph shows that while initial production costs are high (and
%%   hence profit is low) but this can be reduced as more backpacks are
%%   produced.}
%% \end{marginfigure}
%% \begin{solution}
%% The profit is given by
%% \[
%% p(x) = r(x) - c(x)  = 40x - \left(\frac{2100}{x} + 22x\right).
%% \]
%% So the average growth rate of the profit after 10 backpacks are sold is
%% \[
%% \frac{p(10)-p(1)}{10} = 205.20~\text{dollars per backpack sold}.
%% \]
%% On the other hand, the marginal profit when $10$ backpacks are sold is
%% \[
%% \left.\ddx p(x) \right|_{x=10} = \left.\left(18 + \frac{2100}{x^2}\right) \right|_{x=10} = 39~\text{dollars per backpack sold}.
%% \]

%% \end{solution}




\begin{exercises}

\noindent Exercises related to physical applications:


\begin{exercise}
The position of a particle in meters is given by $1/t^3$ where is $t$
is measured in seconds. What is the acceleration of the particle after
$4$ seconds?
\begin{answer}
$3/256$ m/s$^2$
\end{answer}
\end{exercise}

\begin{exercise}
On the Earth, the position of a ball dropped from a height of 100
meters is given by
\[
-4.9t^2+100,\qquad\text{(ignoring air resistance)}
\]
where time is in seconds.  On the Moon, the position of a ball dropped
from a height of 100 meters is given by
\[
-0.8t^2+100,
\]
where time is in seconds.  How long does it take the ball to hit the
ground on the Earth? What is the speed immediately before it hits the
ground? How long does it take the ball to hit the ground on the Moon?
What is the speed immediately before it hits the ground?
\begin{answer}
on the Earth: $\approx 4.5$ s, $\approx 44$ m/s; on the Moon: $\approx
11.2$ s, $\approx 18$ m/s
\end{answer}
\end{exercise}

\begin{exercise}
A $10$ gallon jug is filled with water. If a valve can drain the jug
in 15 minutes, Torricelli's Law tells us that the volume of water in the jug is given by
\[
V(t) = 10\left(1-t/15\right)^2 \qquad\text{where}\qquad 0\le t\le 14. 
\]
What is the average rate that water flows out (change in volume) from
5 to 10 minutes? What is the instantaneous rate that water flows out at
7 minutes?
\begin{answer}
average rate: $\approx -0.67$ gal/min; instantaneous rate: $\approx-0.71$
gal/min
\end{answer}
\end{exercise}

\begin{exercise}
Starting at rest, the position of a car is given by $p(t) = 1.4t^2$ m,
where $t$ is time in seconds.  How many seconds does it take the car to reach
$96$ km/hr? What is the car's average velocity (in km/h) on that time period?
\begin{answer}
$\approx 9.5$ s; $\approx 48$ km/h.
\end{answer}
\end{exercise}

\noindent Exercises related to biological applications:

\begin{exercise}
A certain bacterium triples its population every 15 minutes. The
initial population of a culture is 300 cells. Find a formula for the
population after $t$ hours. 
\begin{answer}
$p(t) = 300\cdot3^{4t}$
\end{answer}
\end{exercise}

\begin{exercise}
The blood alcohol content of man starts at $0.18$ mg/ml. It is metabolized by the body over time, and after $t$ hours, it is given by
\[
c(t) = .18e^{-0.15 t}.
\]
What rate is the man metabolizing alcohol at after $2$ hours?
\begin{answer}
$\approx -.02$ mg/ml per hour
\end{answer}
\end{exercise}

\begin{exercise}
The area of mold on a square piece of bread that is $10$ cm per side
is modeled by
\[
a(t) = \frac{90}{1+150e^{-1.8 t}}~\text{cm}^2
\]
where $t$ is time measured in days. What rate is the mold growing after 3 days? After 10 days?
\begin{answer}
$\approx 39$ cm/day; $\approx 0$ cm/day
\end{answer}
\end{exercise}


%% \noindent Exercises related to economic applications:

%% \begin{exercise}
%% A certain item cost
%% \[
%% c(x) = 17x + 6500
%% \]
%% dollars to produce. If each 
%% \begin{answer}
%% \end{answer}
%% \end{exercise}

%% \begin{exercise}
%% 9
%% \begin{answer}
%% \end{answer}
%% \end{exercise}

%% \begin{exercise}
%% 10
%% \begin{answer}
%% \end{answer}
%% \end{exercise}

\end{exercises}







\section{Related Rates Problems}

Suppose we have two variables $x$ and $y$ which are both changing with
respect to time.  A \textit{related rates} problem is a problem where
we know one rate at a given instant, and wish to find the other.  If
$y$ is written in terms of $x$, and we are given $\dd[x]{t}$, then it
is easy to find $\dd[y]{t}$ using the chain rule:
\[
\dd[y]{t}=y'(x(t))\cdot x'(t).
\]
In many cases, particularly the interesting ones, our functions will
be related in some other way. Nevertheless, in each case we'll use the
same strategy:

\begin{guidelinesForRelatedRates}\hfil
\begin{itemize}
\item[\textbf{Draw a picture.}] If possible, draw a schematic picture with all the relevant information. 
\item[\textbf{Find an equation.}] We want an equation that relates all relevant functions. 
\item[\textbf{Differentiate the equation.}] Here we will often use
  implicit differentiation.
\item[\textbf{Evaluate the equation at the desired values.} ] The known values
  should let you solve for the relevant rate.
\end{itemize}
\end{guidelinesForRelatedRates}
Let's see a concrete example. 

\begin{example}
\label{exam:receding airplane}
A plane is flying directly away from you at $500$ mph at an altitude of
$3$ miles.  How fast is the plane's distance from you increasing at the
moment when the plane is flying over a point on the ground $4$ miles
from you?
\end{example}

\begin{solution}
We'll use our general strategy to solve this problem. To start,
\textbf{draw a picture}.

\begin{tikzpicture}
\draw[penColor2, dashed, very thick] (0,0) -- (5,4);
%\draw[penColor, dashed, very thick] (0,0) -- (0,4);
\draw[penColor, dashed, very thick] (5,0) -- (5,4);
\draw[penColor, dashed, very thick] (0,0) -- (5,0);
\draw[->,penColor, very thick] (1,4) -- (6,4);
\draw [penColor, fill] (5,4) circle [radius=.07];
\node [left,penColor] at (0,0) {\scalebox{3} \Ladiesroom};
\node [right,penColor] at (6,4) {\scalebox{3}{\ding{40}}};
\node [right,penColor] at (5,2) {$3$ miles};
\node [above,penColor] at (3,4) {$p'(t) = 500$ mph};
\node [above,penColor] at (5,4) {$p(t)$};
\node [below,penColor] at (2.5,0) {$4$ miles};
\node [left,penColor2] at (2.4,2) {$s(t)$ miles};
\end{tikzpicture}

Next we need to \textbf{find an equation}. By the Pythagorean Theorem
we know that
\[
p^2+3^2=s^2.
\] 
Now we \textbf{differentiate the equation}. Write
\[
2p(t)p'(t)  = 2s(t) s'(t).
\] 
Now we'll \textbf{evaluate the equation at the desired values}.  We
are interested in the time at which $p(t)=4$ and $p'(t) =
500$. Additionally, at this time we know that $4^2+9=s^2$, so
$s(t)=5$.  Putting together all the information we get
\[
2(4)(500)=2(5)s'(t),
\]
thus $s'(t)=400$ mph.
\end{solution}





\begin{example}
You are inflating a spherical balloon at the rate of 7 cm${}^3$/sec.  How
fast is its radius increasing when the radius is 4 cm?
\end{example}

\begin{solution}
To start, \textbf{draw a picture}.

\begin{tikzpicture}
%\draw[penColor!50!background,very thick] (0,0) ellipse (2 and 1);
\draw[very thick,penColor!20!background] (2,0) arc (0:180:2 and .7);% top half of ellipse
\draw [penColor, very thick] (0,0) circle [radius=2];
\draw[penColor2, dashed, very thick] (0,0) -- (2,0);
\node [below,penColor2] at (1,0) {$r=4$ cm};
\draw[very thick,penColor] (-2,0) arc (180:360:2 and .7);% bottom half of ellipse
\node [penColor,left] at (-1.5,1.42) {$\dd[V]{t} = 7$ cm$^3$/sec};
\node [penColor, right] at (1.5,-1.42) {$V = \frac{4\pi r^3}{3}$ cm$^3$};
\end{tikzpicture}

Next we need to \textbf{find an equation}.  Thinking of the variables
$r$ and $V$ as functions of time, they are related by the equation
\[
V(t)=\frac{4\pi (r(t))^3}{3}.
\]

Now we need to \textbf{differentiate the equation}.  Taking the
derivative of both sides gives 
\[
\dd[V]{t}=4\pi (r(t))^2\cdot r'(t).
\]  
Finally we \textbf{evaluate the equation at the desired values}. Set
$r(t)= 4$ cm and $\dd[V]{t}$ = 7 cm$^3$/sec. Write 
\begin{align*}
7 &=4\pi 4^2r'(t),\\
r'(t) &=7/(64\pi)~\text{cm/sec}.
\end{align*}
\end{solution}

\begin{example} Water is poured into a conical container at the rate of 10
cm${}^3$/sec.  The cone points directly down, and it has a height of
30 cm and a base radius of 10 cm.  How fast is the water level rising
when the water is 4 cm deep?
\end{example}

\begin{solution}
To start, \textbf{draw a picture}.

\begin{tikzpicture}
\draw[penColor,very thick] (0,4) ellipse (4 and 1);
\draw[very thick,penColor!20!background] (2,2) arc (0:180:2 and .5);% top half of ellipse
\draw[very thick,penColor] (-2,2) arc (180:360:2 and .5);% bottom half of ellipse
\draw[penColor, very thick] (3.97,3.85) -- (0,0);
\draw[penColor, very thick] (-3.97,3.85) -- (0,0);
\draw[penColor, very thick] (0,4) -- (4,4);
\draw[penColor!50!background, very thick] (0,2) -- (2,2);
\draw[->,line width=0.4cm, penColor!20!background] (0,6) -- (0,4.25);
\draw[dashed, penColor2, very thick] (2.1,0) -- (2.1,2);
\draw[dashed, penColor, very thick] (-4.1,0) -- (-4.1,4);
\node[right, penColor] at (.4,5.6) {$\dd[V]{t} = 10$ cm$^3$/sec};
\node[below, penColor] at (2,4) {$10$ cm};
\node[above, penColor] at (1,2) {$r$ cm};
\node[right, penColor2] at (2.1,1) {$h(t) = 4$ cm};
\node[left, penColor] at (-4.1,2) {$30$ cm};
\end{tikzpicture}

Note, no attempt was made to draw this picture to scale, rather we
want all of the relevant information to be available to the
mathematician.

Now we need to \textbf{find an equation}. The formula for the volume of a cone tells us that 
\[
V = \frac{\pi}{3} r^2 h.
\]

Now we must \textbf{differentiate the equation}. We should use implicit differentiation, and treat each of the variables as functions of $t$. Write
\begin{equation}\label{equation:cone/water}
\dd[V]{t} = \frac{\pi}{3}\left(2rh \dd[r]{t} + r^2 \dd[h]{t}\right).
\end{equation}

At this point we \textbf{evaluate the equation at the desired values}.
At first something seems to be wrong, we do not know $\dd[r]{t}$.
However, the dimensions of the cone of water must have the same
proportions as those of the container.  That is, because of similar
triangles, 
\[
\frac{r}{h}=\frac{10}{30} \qquad\text{so}\qquad r={h/3}.
\]  
In particular, we see that when $h = 4$, $r=4/3$ and 
\[
\dd[r]{t} = \frac{1}{3}\cdot \dd[h]{t}.
\]
Now we can \textbf{evaluate the equation at the desired
  values}. Starting with Equation~\ref{equation:cone/water}, we plug
in $\dd[V]{t} = 10$, $r = 4/3$, $\dd[r]{t} = \frac{1}{3}\cdot \dd[h]{t}$
and $h=4$. Write
\begin{align*}
10 &= \frac{\pi}{3}\left(2\cdot \frac{4}{3}\cdot 4 \cdot\frac{1}{3}\cdot\dd[h]{t} + \left(\frac{4}{3}\right)^2 \dd[h]{t}\right)\\
10 &= \frac{\pi}{3}\left(\frac{32}{9}\dd[h]{t} + \frac{16}{9} \dd[h]{t}\right)\\
10 &= \frac{16\pi}{9}\dd[h]{t}\\
\frac{90}{16\pi} &= \dd[h]{t}.
\end{align*}
Thus, $\dd[h]{t}=\frac{90}{16\pi}$ cm/sec.
\end{solution}

\begin{example}
A swing consists of a board at the end of a $10$ ft long rope.  Think
of the board as a point $P$ at the end of the rope, and let $Q$ be the
point of attachment at the other end.  Suppose that the swing is
directly below $Q$ at time $t=0$, and is being pushed by someone who
walks at 6 ft/sec from left to right.  What is the angular speed of
the rope in deg/sec after 1 sec?
\end{example}

\begin{solution}
To start, \textbf{draw a picture}.

\begin{tikzpicture}[scale=1.3]
\draw[penColor!50!background, very thick] (0,3) -- (0,-1);
\draw[penColor, very thick] (0,3) -- (2.12,-.12);
\draw [penColor!50!background, very thick] (-2.12,-.12) arc [radius=3, start angle=225, end angle= 315];
\draw [penColor2, very thick] (0,2.3) arc [radius=.7, start angle=270, end angle= 305];
\draw[->, penColor, very thick] (0,-.12) -- (2,-.12);

\node[penColor] at (2.12,-.12) {\scalebox{3} \Ladiesroom};
\node[penColor,right] at (2.3,-.12) {$P$};
\node[penColor,right,above] at (0,3) {$Q$};
\node[penColor,right] at (1.06,1.5) {$10$ ft};
\node[penColor,above] at (1,-.12) {$\dd[x]{t} = 6$ ft/sec};
\node[penColor2,right,above] at (.3,2) {$\theta$};
\end{tikzpicture}

Now we must \textbf{find an equation}. From the right triangle in our
picture, we see
\[
\sin(\theta)=x/10.
\]
We can now \textbf{differentiate the equation}. Taking derivatives we obtain 
\[
\cos(\theta)\cdot \theta'(t)=0.1 x'(t).
\]
Now we can \textbf{evaluate the equation at the desired values}.  When
$t=1$ sec, the person was pushed by someone who walks $6$
ft/sec. Hence we have a $6-8-10$ right triangle, with $x'(t) = 6$, and
$\cos\theta=8/10$. Thus
\[
(8/10) \theta'(t) =6/10,
\]
and so  $\theta'(t)=6/8=3/4$ rad/sec, or approximately $43$ deg/sec.
\end{solution} 



We have seen that sometimes there are apparently more than two
variables that change with time, but as long as you know the rates of
change of all but one of them you can find the rate of change of the
remaining one.  As in the case when there are just two variables, take
the derivative of both sides of the equation relating all of the
variables, and then substitute all of the known values and solve for
the unknown rate.



\begin{example}
A road running north to south crosses a road going east to west at the
point $P$.  Cyclist $A$ is riding north along the first road, and cyclist $B$ is
riding east along the second road.  At a particular time, cyclist $A$ is $3$
kilometers to the north of $P$ and traveling at $20$ km/hr, while cyclist
$B$ is $4$ kilometers to the east of $P$ and traveling at $15$ km/hr.
How fast is the distance between the two cyclists changing?
\end{example}

\begin{solution}
We start the same way we always do, we \textbf{draw a picture}.

\begin{tikzpicture}
\draw[->,penColor!50!background, very thick] (-1,0) -- (4,0);
\draw[->,penColor!50!background, very thick] (0,-1) -- (0,4);
\draw[->,penColor, very thick] (0,3) -- (0,4);
\draw[->,penColor, very thick] (3,0) -- (4,0);
\draw [penColor, fill] (0,0) circle [radius=.07];
\draw [penColor, fill] (3,0) circle [radius=.07];
\draw [penColor, fill] (0,3) circle [radius=.07];
\draw[dashed,penColor2, very thick] (3,0) -- (0,3);

\node[penColor,rotate=90,right] at (.5,3) {\scalebox{-2} \Bicycle};
\node[penColor,right] at (0,.2) {$P$};
\node[penColor,left] at (-.3,3) {$a'(t) = 20$ km/hr};
\node[penColor,left] at (0,1.5) {$3$ km};
\node[penColor,below] at (1.5,0) {$4$ km};
\node[penColor,below] at (4,0) {$b'(t)= 15$ km/hr};
\node[penColor2,above] at (1.6,1.6) {$c(t)$};
\node[penColor,right,above] at (3.5,0) {\scalebox{-2}[2] \Bicycle};
\end{tikzpicture}

Here $a(t)$ is the distance of cyclist $A$ north of $P$ at time $t$,
and $b(t)$ the distance of cyclist $B$ east of $P$ at time $t$, and
$c(t)$ is the distance from cyclist $A$ to cyclist $B$ at time $t$.

We must \textbf{find an equation}.  By the Pythagorean Theorem,
\[
c(t)^2=a(t)^2+b(t)^2.
\] 
Now we can \textbf{differentiate the equation}. Taking derivatives we get 
\[
2c(t)c'(t)=2a(t)a'(t)+2b(t)b'(t).
\]
Now we can  \textbf{evaluate the equation at the desired values}.
We know that $a(t) = 3$, $a'(t) = 20$, $b(t) = 4$ and $b'(t) = 15$. Hence 
by the Pythagorean Theorem, $c(t) = 5$. So 
\[
2\cdot 5 \cdot c'(t) = 2 \cdot 3\cdot 20 + 2 \cdot 4 \cdot 15
\]
solving for $c'(t)$ we find $c'(t) = 24$ km/hr.
\end{solution}



\begin{exercises}

\begin{exercise}
A cylindrical tank standing upright (with one circular base on the
ground) has radius 20 cm.  How fast does the water level in the
tank drop when the water is being drained at 25 cm${}^3$/sec?
\begin{answer} $1/(16\pi)$ cm/s
\end{answer}\end{exercise}

\begin{exercise}
A cylindrical tank standing upright (with one circular base on the
ground) has radius 1 meter.  How fast does the water level in the
tank drop when the water is being drained at 3 liters per second?
\begin{answer} $3/(1000\pi)$ meters/second
\end{answer}\end{exercise}

\begin{exercise} A ladder 13 meters long rests on horizontal ground and leans
against a vertical wall.  The foot of the ladder is pulled away from
the wall at the rate of 0.6 m/sec.  How fast is the top sliding down
the wall when the foot of the ladder is 5 m from the wall?
\begin{answer} $1/4$ m/s
\end{answer}\end{exercise}

\begin{exercise} A ladder 13 meters long rests on horizontal ground and leans
against a vertical wall. The top of the ladder is being pulled up the
wall at $0.1$ meters per second.
How fast is the foot of the ladder approaching 
the wall when the foot of the ladder is 5 m from the wall?
\begin{answer} $6/25$ m/s
\end{answer}\end{exercise}

\begin{exercise}
A rotating beacon is located 2 miles out in the water.  Let $A$ be the
point on the shore that is closest to the beacon.  As the beacon rotates at
10 rev/min, the beam of light sweeps down the shore once each time it revolves.
Assume that the shore is straight.  How fast is the point where the beam
hits the shore moving at an instant when the beam is lighting up a point 2
miles along the shore from the point $A$?
\begin{answer} $80\pi$ mi/min
\end{answer}\end{exercise}

\begin{exercise}
A baseball diamond is a square 90 ft on a side.  A player runs from first
base to second base at 15 ft/sec.  At what rate is the player's distance
from third base decreasing when she is half way from first to second base?
\begin{answer} $3\sqrt5$ ft/s
\end{answer}\end{exercise}

\begin{exercise} Sand is poured onto a surface at 15 cm${}^3$/sec, forming a
conical pile whose base diameter is always equal to its altitude.  How
fast is the altitude of the pile increasing when the pile is 3 cm
high?
\begin{answer} $20/(3\pi)$ cm/s
\end{answer}\end{exercise}

\begin{exercise}
A boat is pulled in to a dock by a rope with one end attached to the front
of the boat and the other end passing through a ring attached to the dock
at a point 5 ft higher than the front of the boat.  The rope is being
pulled through the ring at the rate of 0.6 ft/sec.  How fast is the boat
approaching the dock when 13 ft of rope are out?
\begin{answer} $13/20$ ft/s
\end{answer}\end{exercise}

\begin{exercise}
A balloon is at a height of 50 meters, and is rising at the constant rate
of 5 m/sec.  A bicyclist passes beneath it, traveling in a
straight line at the constant speed of 10 m/sec.  How fast is the distance
between the bicyclist and the balloon increasing 2 seconds later?
\begin{answer} $5\sqrt{10}/2$ m/s
\end{answer}\end{exercise}

\begin{exercise} A pyramid-shaped vat has square cross-section and stands on its
tip.  The dimensions at the top are 2 m $\times$ 2 m, and the depth is
5 m.  If water is flowing into the vat at 3 m${}^3$/min, how fast is
the water level rising when the depth of water (at the deepest point)
is 4 m?  Note: the volume of any ``conical'' shape (including
pyramids) is $(1/3)(\hbox{height})(\hbox{area of base})$.
\begin{answer} $75/64$ m/min
\end{answer}\end{exercise}

%% \begin{exercise}
%% The sun is rising at the rate of $1/4$ deg/min, and appears to be
%% climbing into the sky perpendicular to the
%% horizon, as depicted in figure~\xrefn{fig:sunrise sunset}.
%% How fast is the shadow of a 200 meter building
%% shrinking at the moment when the shadow is 500 meters long? 
%% \begin{answer} $145\pi/72$ m/s
%% \end{answer}\end{exercise}

%% \begin{exercise} The sun is setting at the rate of $1/4$ deg/min, and appears
%% to be dropping perpendicular to the horizon, as depicted in
%% figure~\xrefn{fig:sunrise sunset}. How fast is the shadow of a 25
%% meter wall lengthening at the moment when the shadow is 50 meters long?
%% \begin{answer} $25\pi/144$ m/min
%% \end{answer}\end{exercise}

%% %% BADBAD
%% %% \font\miscsymbols miscsymbols10 scaled 2000
%% %% %\begin{psfigure}{2.45in}{1.25in}{124.figME.ps}
%% %% \figure
%% %% \vbox{\beginpicture
%% %% \normalgraphs
%% %% \sevenpoint
%% %% \setcoordinatesystem units <0.3truecm,0.3truecm>
%% %% \setplotarea x from -10 to 10, y from 0 to 8
%% %% \axis bottom shiftedto y=0 /
%% %% \setdashes\setlinear
%% %% \plot 5 0 -8 8 /
%% %% \setsolid
%% %% \setplotsymbol ({\tenrm.}) 
%% %% \plot 0 0 0 3 /
%% %% \put {\miscsymbols k} at -9 8.6
%% %% \endpicture}
%% %% \figrdef{fig:sunrise sunset}
%% %% \endfigure{Sunrise or sunset.}



%% \begin{exercise}
%% The trough shown in figure~\xrefn{fig:trough}
%% is constructed by fastening together three
%% slabs of wood of dimensions 10 ft $\times$ 1 ft, and then attaching the
%% construction to a wooden wall at each end.  The angle $\theta$ was
%% originally $30^\circ$, but because of poor construction the sides are
%% collapsing.  The trough is full of water.  At what rate (in ft${}^3$/sec) 
%% is 
%% the water spilling out over the top of
%% the trough if the sides have each fallen to an angle of $45^\circ$, and are
%% collapsing at the rate of $1^\circ$ per second?
%% \begin{answer} $\pi\sqrt2/36$ ft$^3$/s
%% \end{answer}\end{exercise}

%% %%BADBAD
%% %\begin{psfigure}{2.5in}{1.5in}{124.figMF.ps}
%% %% \figure
%% %% \vbox{\beginpicture
%% %% \normalgraphs
%% %% \sevenpoint
%% %% \setcoordinatesystem units <0.9truecm,0.9truecm>
%% %% \setplotarea x from -1 to 12, y from 0 to 4
%% %% \setlinear
%% %% \plot -0.5 0.866 0 0 1 0 1.5 0.866 -0.5 0.866 /
%% %% \plot 1 0 10.4 3.42 10.9 4.29 8.9 4.29 -0.5 0.866 /
%% %% \plot 10.9 4.29 1.5 0.866 /
%% %% \put {$\theta$} [b] <0pt,3pt> at -0.15 0.3
%% %% \put {$\theta$} [b] <0pt,3pt> at 1.15 0.3
%% %% \put {$1$} [t] <0pt,-4pt> at 0.5 0
%% %% \put {$1$} [tr] <-2pt,-2pt> at -0.25 0.433
%% %% \put {$1$} [tl] <2pt,-2pt> at 10.65 3.853
%% %% \put {$10$} [tl] <2pt,-2pt> at 5.7 1.71
%% %% \circulararc 30 degrees from 0 0.3 center at 0 0
%% %% \circulararc -30 degrees from 1 0.3 center at 1 0
%% %% \setdashes <2pt>
%% %% \plot 0 0 9.4 3.42 10.4 3.42 / 
%% %% \plot 9.4 3.42 8.9 4.29 / 
%% %% \plot 0 0 0 0.866 /
%% %% \plot 1 0 1 0.866 /
%% %% \endpicture}
%% %% \figrdef{fig:trough}
%% %% \endfigure{Trough.}

\begin{exercise}
A woman 5 ft tall walks at the rate of 3.5 ft/sec away from a streetlight
that is 12 ft above the ground.  At what rate is the tip of her shadow
moving?  At what rate is her shadow lengthening?
\begin{answer} tip: 6 ft/s, length: $5/2$ ft/s
\end{answer}\end{exercise}

\begin{exercise} A man 1.8 meters tall walks at the rate of 1 meter per
second toward a streetlight that is 4 meters above the ground.  At
what rate is the tip of his shadow moving?  At what rate is his shadow
shortening?
\begin{answer} tip: $20/11$ m/s, length: $9/11$ m/s
\end{answer}\end{exercise}

\begin{exercise}
A police helicopter is flying at 150 mph at a constant altitude of 0.5 mile
above a straight road.  The pilot uses radar to determine that an oncoming
car is at a distance of exactly 1 mile from the helicopter, and that this
distance is decreasing at 190 mph.  Find the speed of the car.
\begin{answer} $380/\sqrt3-150\approx 69.4$ mph
\end{answer}\end{exercise}

\begin{exercise} A police helicopter is flying at 200 kilometers per hour at
a constant altitude of 1 km above a straight road.  The pilot uses
radar to determine that an oncoming car is at a distance of exactly 2
kilometers from the helicopter, and that this distance is decreasing at 250
kph.  Find the speed of the car.
\begin{answer} $500/\sqrt3-200\approx 88.7$ km/hr
\end{answer}\end{exercise}

%% %BADBAD
%% %\font\miscsymbols miscsymbols10 scaled 2000
%% \begin{exercise}
%% A light shines from the top of a pole 20 m high. A ball is falling 10
%% meters from the pole, casting a shadow on a building 30 meters away,
%% as shown in figure~\xrefn{fig:falling ball}.
%% When the ball is 25 meters from the ground it is falling at 6 meters
%% per second. How fast is its shadow moving?
%% \begin{answer} 18 m/s
%% \end{answer}\end{exercise}

%% %% BADBAD
%% %% \figure
%% %% \vbox{\beginpicture
%% %% \normalgraphs
%% %% \ninepoint
%% %% \setcoordinatesystem units <.7truecm,.7truecm>
%% %% \setplotarea x from 0 to 4.5, y from 0 to 4
%% %% \axis bottom /
%% %% \setlinear
%% %% \linethickness1pt
%% %% \putrule from 0 0 to 0 2
%% %% \put {$\bullet$} at 1 2.5
%% %% \put {\miscsymbols k} at 0.05 2
%% %% \putrule from 3 0 to 3 4
%% %% \putrule from 3 4 to 4.5 4
%% %% \setdashes <2pt>
%% %% \plot 0 2 3 3.5 /
%% %% \endpicture}
%% %% \figrdef{fig:falling ball}
%% %% \endfigure{Falling ball.}

\begin{exercise} 
A road running in a northwest direction crosses a road going east to
west at a $120^\circ$ at a point $P$. Car $A$ is driving northwesterly
along the first road, and car $B$ is driving east along the second
road. At a particular time car $A$ is $10$ kilometers to the northwest
of P and traveling at $80$ km/hr, while car $B$ is $15$ kilometers to
the east of $P$ and traveling at $100$ km/hr. How fast is the distance
between the two cars changing? Hint, recall the law of cosines:
$c^2=a^2+b^2-2ab\cos\theta$.
\begin{answer} $136\sqrt{475}/19\approx 156$ km/hr
\end{answer}\end{exercise}

\begin{exercise}
A road running north to south crosses a road going east to west at the
point $P$. Car A is 300 meters north of $P$, car B is 400 meters east
of $P$, both cars are going at constant speed toward $P$, and the two
cars will collide in 10 seconds. How fast is the distance between the
two cars changing?
\begin{answer} $-50$ m/s
\end{answer}\end{exercise}

\begin{exercise}
A road running north to south crosses a road going east to west at the
point $P$. Eight seconds ago car $A$ started from rest at $P$ and has
been driving north, picking up speed at the steady rate of $5$
m/sec${}^2$. Six seconds after car $A$ started, car $B$ passed $P$
moving east at constant speed 60 m/sec. How fast is the distance
between the two cars changing?
\begin{answer} $68$ m/s
\end{answer}\end{exercise}

\begin{exercise} 
Suppose a car is driving north along a road at $80$ km/hr and an
airplane is flying east at speed $200$ km/hr. Their paths crossed at a
point $P$. At a certain time, the car is 10 kilometers north of $P$
and the airplane is 15 kilometers to the east of $P$ at an altitude of
2 km. How fast is the distance between car and airplane changing?
\begin{answer} $3800/\sqrt{329}\approx 210$ km/hr 
\end{answer}\end{exercise}

%%BADBAD
%% \figure
%% \vbox{\beginpicture
%% \normalgraphs
%% \sevenpoint
%% \setcoordinatesystem units <0.3truecm,0.3truecm>
%% \setplotarea x from 0 to 8, y from -2 to 10
%% \setlinear
%% \plot 6 3 0 0 8 -2 /
%% \setdashes <2pt>
%% \plot 2.5 1.25 5 8.75 /
%% \plot 5 -1.25 5 8.75 /
%% \multiput {$\bullet$} at 2.5 1.25 5 8.75 /
%% \put {$A$} [br] <-2pt,2pt> at 2.5 1.25
%% \put {$B$} [b] <0pt,3pt> at 5 8.75
%% \put {$c(t)$} [br] <-2pt,2pt> at 3.75 5
%% \setplotsymbol ({\tenrm.}) 
%% \setsolid
%% \arrow <5pt> [.25, 1] from 2.5 1.25 to 4 2
%% \arrow <5pt> [.25, 1] from 5 8.75 to 7 8.25
%% \endpicture}
%% \figrdef{fig:car and airplane}
%% \endfigure{Car and airplane.}



\begin{exercise} 
Suppose a car is driving north along a road at $80$ km/hr and an
airplane is flying east at speed $200$ km/hr. Their paths crossed at a
point $P$. At a certain time, the car is 10 kilometers north of $P$
and the airplane is 15 kilometers to the east of $P$ at an altitude of
2 km---gaining altitude at 10 km/hr. How fast is the distance between
car and airplane changing?
\begin{answer} \hbox{$820/\sqrt{329}+150\sqrt{57}/\sqrt{47}\approx 210$ km/hr}
\end{answer}\end{exercise}

\begin{exercise}
A light shines from the top of a pole 20 m high.  An object is dropped from
the same height from a point 10 m away, so that its height at time $t$
seconds is $h(t)=20-9.8t^2/2$.  How fast is the object's shadow
moving on the ground one second later?
\begin{answer} $4000/49$ m/s
\end{answer}\end{exercise}

%% \begin{exercise}
%% The two blades of a pair of scissors are fastened at the point $A$ as
%% shown in figure~\xrefn{fig:scissors}.  Let
%% $a$ denote the distance from $A$ to the tip of the blade (the point $B$).
%% Let $\beta$ denote the angle at the tip of the blade that is formed by the
%% line $\overline{AB}$ and the bottom edge of the blade, line
%% $\overline{BC}$, and let $\theta$ denote the angle between
%% $\overline{AB}$ and the horizontal.
%% Suppose that a piece of paper is cut in such a way that the center
%% of the scissors at $A$ is fixed, and the paper is also fixed.  As the
%% blades are closed (i.e., the angle $\theta$ in the diagram is decreased),
%% the distance $x$ between $A$ and $C$ increases, cutting the paper.
%% \begin{enumerate}
%% \item{\bf a.} Express $x$ in terms of $a$, $\theta$, and $\beta$.

%% \item{\bf b.} Express $dx/dt$ in terms of $a$,
%% $\theta$, $\beta$, and $d\theta/dt$.

%% \item{\bf c.} Suppose that the distance $a$ is 20 cm, and the
%% angle $\beta$ is $5^\circ$.  Further suppose that $\theta$ is
%% decreasing at 50
%% deg/sec.  At the instant when $\theta=30^\circ$, find the rate (in
%% cm/sec) at which the paper is being cut.
%% \end{enumerate}
%% \begin{answer} (a) $x=a\cos\theta-a\sin\theta\cot(\theta+\beta)=
%% \hbox{$a\sin\beta/\sin(\theta+\beta$), (c) $\dot x\approx 3.79$ cm/s}$
%% \end{answer}\end{exercise}


%%BADBAD
%% \figure
%% \vbox{\beginpicture
%% \normalgraphs
%% \sevenpoint
%% \setcoordinatesystem units <0.7truecm,0.7truecm>
%% \setplotarea x from -3 to 6, y from -3.5 to 3.5
%% \setlinear
%% \plot 0.3 0 6 3.5 2 0 0.3 0 /
%% \setdashes <2pt>
%% \put {\beginpicture
%% \setplotarea x from -1 to 1, y from -0.33 to 0.33
%% \startrotation by 0.866 -0.5 about 0 0
%% \ellipticalarc  axes ratio 3:1  360 degrees from 1 0 center at 0 0
%% \stoprotation\endpicture} at -1 1
%% \put {\beginpicture
%% \setplotarea x from -1 to 1, y from -0.33 to 0.33
%% \startrotation by 0.866 0.5 about 0 0
%% \ellipticalarc  axes ratio 3:1  360 degrees from 1 0 center at 0 0
%% \stoprotation\endpicture} at -1 -1
%% \put {\beginpicture
%% \setplotarea x from -1 to 1, y from -0.33 to 0.33
%% \startrotation by 0.866 0.5 about 0 0
%% \ellipticalarc  axes ratio 3:1  245 degrees from 1 0.58 center at 0 0
%% \stoprotation\endpicture} at -1 -1
%% \put {\beginpicture
%% \setplotarea x from -1 to 1, y from -0.33 to 0.33
%% \startrotation by 0.866 -0.5 about 0 0
%% \ellipticalarc  axes ratio 3:1  -245 degrees from 1 -0.58 center at 0 0
%% \stoprotation\endpicture} at -1 1
%% %\multiput {$+$} at 0 0 0.27 0.9 -1 -1 /
%% \setquadratic
%% \plot 0.27 0.9 3 2.5 6 3.5 /
%% \plot 0.27 -0.9 3 -2.5 6 -3.5 /
%% \setlinear
%% \plot 0.3 0 6 -3.5 2 0 /
%% \plot 6 -3.5 6 3.5 /
%% \plot 2 0 6 0 /
%% \put {$B$} [l] <4pt,0pt> at 6 3.5
%% \put {$A$} [r] <-3pt,0pt> at 0.3 0
%% \put {$C$} [br] <0pt,3pt> at 2 0
%% \put {$\theta$} at 1 0.2
%% \endpicture}
%% \figrdef{fig:scissors}
%% \endfigure{Scissors.}

%\epsfbox{124.figMG.ps}

\end{exercises}









