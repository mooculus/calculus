\documentclass{ximera}

\newcommand{\RR}{\mathbb R}
\renewcommand{\d}{\,d}
\newcommand{\dd}[2][]{\frac{d #1}{d #2}}
\renewcommand{\l}{\ell}
\newcommand{\ddx}{\frac{d}{dx}}
\newcommand{\dfn}{\textbf}
\newcommand{\eval}[1]{\bigg[ #1 \bigg]}


\title[Dig-In:]{Falling objects}

\begin{document}
\begin{abstract}
  We study a special type of differential equation.
\end{abstract}
\maketitle

Remember, a \dfn{differential equation} is simply an equation with a derivative in it like this:
\[
f'(x) = k f(x).
\]
When a mathematician solves a differential equation, they are finding
a \textit{function} that satisfies the equation.


Recall that the acceleration due to gravity is about $-9.8$
m/s$^2$. Since the first derivative of the function giving the
velocity of an object gives the acceleration of the object and the
second derivative of a function giving the position of a falling
object gives the acceleration, we have the differential equations
\begin{align*}
v'(t) &=  -9.8,\\
p''(t) &=  -9.8.
\end{align*}
From these simple equation, we can derive equations for the velocity of
the object and for the position using antiderivatives.


\begin{example}
A ball is tossed into the air with an initial velocity of $15$
m/s. What is the velocity of the ball after 1 second? How about after
2 seconds?
\begin{explanation}
Knowing that the acceleration due to gravity is $-9.8$ m/s$^2$, we write
\[
v'(t) = \answer[given]{-9.8}.
\]
To solve this differential equation, take the antiderivative of both sides
\begin{align*}
\int v'(t) \d t &= \int \answer[given]{-9.8} \d t\\
v(t) &= \answer[given]{-9.8 t} + C.
\end{align*}
Here $C$ represents the initial velocity of the ball. Since it is
tossed up with an initial velocity of $15$ m/s, 
\[
\answer[given]{15} = v(0) = -9.8\cdot 0 + C,
\]
and we see that $C=\answer[given]{15}$. Hence $v(t) = -9.8t + 15$. Now when $t=1$,
$v(1) = 5.2$ m/s, and the ball is rising, and at $t=2$, $v(2) = -4.6$ m/s,
and the ball is falling.
\end{explanation}
\end{example}

Now let's do a similar problem, but instead of finding the velocity,
we will find the position.

\begin{example}
A ball is tossed into the air with an initial velocity of $15$ m/s
from a height of 2 meters. When does the ball hit the ground?
\begin{explanation}
Knowing that the acceleration due to gravity is $-9.8$ m/s$^2$, we write
\[
p''(t) = \answer[given]{-9.8}.
\]
Start by taking the antiderivative of both sides of the equation
\begin{align*}
\int p''(t) \d t &= \int \answer[given]{-9.8} \d t\\
p'(t) &= \answer[given]{-9.8 t} + C.
\end{align*}
Here $C$ represents the initial velocity of the ball. Since it is
tossed up with an initial velocity of $15$ m/s, $C = 15$ and 
\[
p'(t) = -9.8t + 15.
\]
Now let's take the antiderivative again. 
\begin{align*}
\int p'(t) \d t &= \int \answer[given]{-9.8t +15} \d t\\
p(t) &= \answer[given]{\frac{-9.8t^2}{2} + 15t} + D.
\end{align*}
Since we know the initial height was $2$ meters, write
\[
2 = p(0) =  \frac{-9.8\cdot 0^2}{2} + 15\cdot 0 + D.
\]
Hence $p(t) = \frac{-9.8t^2}{2} + 15t + 2$. We need to know when the
ball hits the ground, this is when $p(t)=0$. Solving the equation
\[
\frac{-9.8t^2}{2} + 15t + 2 = 0
\]
we find two solutions $t\approx -0.1$ and $t\approx 3.2$. Discarding
the negative solution, we see the ball will hit the ground after
approximately $3.2$ seconds.
\end{explanation}
\end{example}

The power of calculus is that it frees us from rote memorization of
formulas and enables us to derive what we need.

\end{document}
