\documentclass{ximera}

\newcommand{\RR}{\mathbb R}
\renewcommand{\d}{\,d}
\newcommand{\dd}[2][]{\frac{d #1}{d #2}}
\renewcommand{\l}{\ell}
\newcommand{\ddx}{\frac{d}{dx}}
\newcommand{\dfn}{\textbf}
\newcommand{\eval}[1]{\bigg[ #1 \bigg]}


\author{Jim Talamo}
\license{Creative Commons 3.0 By-NC}


\outcome{Answer conceptual questions about sequences}
\outcome{Understand the relationship between the terms of a sequence, its sequence of partial sums, and its sequence of remainders}

\begin{document}

\begin{exercise}

Suppose that $a_n = \frac{n^2}{2^n}$ for $n \geq 1$. Circle all that apply for each sequence listed below.

$\bullet$ The sequence $\{a_n\}_{n=1}$ is eventually
\begin{selectAll}
\choice{increasing}
\choice[correct]{decreasing}
\choice[correct]{bounded above}
\choice[correct]{bounded below}
\end{selectAll}

$\bullet$ The sequence $\{s_n\}_{n=1}$ is
\begin{selectAll}
\choice[correct]{increasing}
\choice{decreasing}
\choice[correct]{bounded above}
\choice[correct]{bounded below}
\end{selectAll}

$\bullet$ The sequence $\{r_n\}_{n=1}$ is
\begin{selectAll}
\choice{not defined}
\choice{increasing}
\choice[correct]{decreasing}
\choice[correct]{bounded above}
\choice[correct]{bounded below}
\end{selectAll}

\begin{hint}
Answering this question requires synthesizing many important facts we have introduced so far.

\begin{itemize}
\item $\{a_n\}_{n=1}$ is clearly eventually decreasing and is both bounded below and above since $\lim_{n \to \infty} \frac{n^2}{2^n} = 0$ by growth rates.  
\item Since $a_n \geq 0$ for all $n$, we have that $\{s_n\}_{n=1}$ is increasing.  Hence, $\{s_n\}_{n=1}$ is bounded below.  Furthermore, note that $\sum_{k=1}^{\infty} \frac{k^2}{2^k}$ converges by the ratio test.  Hence, $\lim_{n \to \infty} s_n$ exists, so $\{s_n\}_{n=1}$ must also be bounded above.
\item Since $\sum_{k=1}^{\infty} \frac{k^2}{2^k}$ converges, $\{r_n\}_{n=1}$ is defined.   Since we have that $\sum_{k=1}^{\infty} \frac{1}{k^2} = s_n +r_n$ (meaning that no matter which $n$ we choose, the sum of $s_n$ and $r_n$ is the same), $\{r_n\}_{n=1}$ must be decreasing since $s_n$ is increasing.  Hence, $\{r_n\}_{n=1}$ is bounded above.  Also, since $\sum_{k=1}^{\infty} \frac{k^2}{2^k}$ converges, $\lim_{n \to \infty} r_n=0$, so $\{r_n\}_{n=1}$ must also be bounded below.
\end{itemize}

\end{hint}

\begin{exercise}
Which of the sequences has a limit?
\begin{selectAll}
\choice[correct]{$\{a_n\}_{n=1}$}
\choice[correct]{$\{s_n\}_{n=1}$}
\choice[correct]{$\{r_n\}_{n=1}$}
\end{selectAll}
\end{exercise}

\end{exercise}
\end{document}