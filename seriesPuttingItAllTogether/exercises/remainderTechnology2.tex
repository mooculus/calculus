\documentclass{ximera}

\newcommand{\RR}{\mathbb R}
\renewcommand{\d}{\,d}
\newcommand{\dd}[2][]{\frac{d #1}{d #2}}
\renewcommand{\l}{\ell}
\newcommand{\ddx}{\frac{d}{dx}}
\newcommand{\dfn}{\textbf}
\newcommand{\eval}[1]{\bigg[ #1 \bigg]}


\author{Jim Talamo}
\license{Creative Commons 3.0 By-NC}


\outcome{Understand the relationship between the sequence of remainders and the convergence of the series.}

\begin{document}

\begin{exercise}

Which test or tests below could be applied to determine whether $\sum_{k=1}^{\infty} \frac{k^3}{2^k}$ converges or diverges?

\begin{selectAll}
\choice[correct]{The integral test}
\choice[correct]{The ratio test}
\choice[correct]{The root test}
\end{selectAll}

Which test would be easier to use to determine whether $\sum_{k=1}^{\infty} \frac{k^3}{2^k}$ converges or diverges?

\begin{multipleChoice}
\choice{The integral test}
\choice[correct]{The ratio test or root test}
\end{multipleChoice}

\begin{exercise}
Note that Ratio and Root test do not come with nice error estimates,
but integral test does!  Using the integral test remainder estimates,
calculate the value of $\sum_{k=1}^{\infty} \frac{k^3}{2^k}$ to within
$0.0001$ of its exact value.  Use technology to perform all necessary
computations.

%N=1000

To within $0.0001$, we find that $\sum_{k=1}^{\infty}
\frac{k^3}{2^k} \approx \answer[tolerance=.0001]{26}$
\begin{hint}
This integral would require three applications of integration by
parts, so it could be done by hand.  However, the point of this
exercise is to use technology, so take advantage of it!  By using
technology to solve $\int_N^{N+1} \frac{1}{x^{4/3}} \d x =0.0001$, we
find that $N = \answer{28}$ (remember to round up).  The error
estimates can now be used to find a range of possible values for the
series.
\end{hint}

\end{exercise}
\end{exercise}
\end{document}
