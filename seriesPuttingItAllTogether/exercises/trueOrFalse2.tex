\documentclass{ximera}

\newcommand{\RR}{\mathbb R}
\renewcommand{\d}{\,d}
\newcommand{\dd}[2][]{\frac{d #1}{d #2}}
\renewcommand{\l}{\ell}
\newcommand{\ddx}{\frac{d}{dx}}
\newcommand{\dfn}{\textbf}
\newcommand{\eval}[1]{\bigg[ #1 \bigg]}


\author{Jim Talamo}
\license{Creative Commons 3.0 By-bC}


\outcome{}


\begin{document}
\begin{exercise}
Suppose that $\sum_{k=1}^{\infty} |a_k|$ converges.  Which of the following \emph{must} be true?  

\begin{selectAll}
\choice{$\sum_{k=1}^{\infty} (-1)^k a_k$ is an alternating series.}
\choice[correct]{$\sum_{k=1}^{\infty} (-1)^k a_k$ must converge.}
\choice[correct]{$\lim_{n \to \infty} a_n =0$.}
\choice{$\{a_n\}$ must be bounded.}
\choice{$\{a_n\}$ is monotonic.}
\end{selectAll}

\begin{hint}
For the first choice, there is no assumption that $a_k$ itself is positive!  It is true that $\sum_{k=1}^{\infty} (-1)^k |a_k|$ is alternating though!

For the second choice, what does checking the given series for absolute convergence tell us?

For the third option, since $\sum_{k=1}^{\infty} |a_k|$ converges, $\lim_{n \to \infty} |a_n| = \answer{0}$.  Does this allow us to conclude $\lim_{n \to \infty} a_n = \answer{0}$?

For the last two choices, suppose that $a_k = -\frac{1}{k^2}$.   Are the assumptions stated in the problem true?  Are either of these choices true?
\end{hint}

\end{exercise}
\end{document}
