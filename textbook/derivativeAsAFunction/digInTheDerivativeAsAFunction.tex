\documentclass{ximera}

\newcommand{\RR}{\mathbb R}
\renewcommand{\d}{\,d}
\newcommand{\dd}[2][]{\frac{d #1}{d #2}}
\renewcommand{\l}{\ell}
\newcommand{\ddx}{\frac{d}{dx}}
\newcommand{\dfn}{\textbf}
\newcommand{\eval}[1]{\bigg[ #1 \bigg]}


\outcome{}

\title[Dig-in:]{The derivative as a function}

\begin{document}
\begin{abstract}
Here we study the derivative of a function, as a function, in its own
right.
\end{abstract}
\maketitle

\section{The derivative of a function, as a function}


We know that to find the derivative of a function at a point $x=a$ we
write
\[
f'(a) = \lim_{h\to 0}\frac{f(a+h)-f(a)}{h}.
\]
However, if we replace the given number $a$ with a varible $x$, we now
have
\[
f'(x) = \lim_{h\to 0}\frac{f(x+h)-f(x)}{h}.
\]
This tells us the instantaneous rate of change at any given point $x$.
\begin{warning}
  The notation
  \[
  \text{$f'(a)$ means take the derivative of $f$ first, then evaluate
    at $x=a$.}
  \]
  In other words, given $f$ a function of $x$
  \[
  f'(a) = \eval{\ddx f(x)}_{x=a}.
  \]
\end{warning}
Given a function $f$ from the real numbers to the real numbers, the
derivative $f'$ is also a function from the real numbers to the real
numbers. Understanding the relationship between the \textit{functions}
$f$ and $f'$ helps us understand any situation (real or imagined)
involving changing values. 

\begin{question}
  Let $f(x) = 3x+2$. What is $f'(-1)$?
  \begin{multipleChoice}
    \choice{$f'(-1) = 0$ because $f'(3)$ is a number, and a number cooresponds to a horizontal line, which has a slope of zero.}
    \choice{$f'(-1) = 3$ because $y=f(x)$ is a line with slope $3$.}
    \choice{We cannot solve this problem yet.}
  \end{multipleChoice}
\end{question}


\begin{example}
  Here we see the graph of $f'$. 
  \begin{image}
    \begin{tikzpicture}
      \begin{axis}[
          xmin=-2,xmax=2,ymin=-8,ymax=8,
          axis lines=center,
          ticks=none,
          width=6in,
          height=3in,
          every axis y label/.style={at=(current axis.above origin),anchor=south},
          every axis x label/.style={at=(current axis.right of origin),anchor=west},
        ]        
        \addplot [very thick,dashed, penColor,smooth, domain=(-2:2)] {x^3+.3*x^2-2*x)};
        \addplot [very thick,penColor,smooth, domain=(-2:2)] {3*x^2+2*.3*x-2)};
      \end{axis}
    \end{tikzpicture}
  \end{image}
  Describe $y=f(x)$ when $f'$ is positive. Describe $y=f(x)$ when $f'$
  is negative.
  \begin{explanation}
    When $f'$ is positive, $y=f(x)$ is \wordChoice{\choice{positive}\choice[correct]{increasing}\choice{negtive}\choice{decreasing}}.
    When $f'$ is negative, $y=f(x)$ is \wordChoice{\choice{positive}\choice{increasing}\choice{negtive}\choice[correct]{decreasing}}
  \end{explanation}
  \begin{question}
    Which of the following graphs could be $y = f(x)$?
     \begin{multipleChoice}
       \choice{\begin{tikzpicture}[framed,scale=1,baseline=20ex]
           \begin{axis}[
               xmin=-2,xmax=2,ymin=-8,ymax=8,
               axis lines=center,
               ticks=none,
               width=6in,
               height=3in,
               every axis y label/.style={at=(current axis.above origin),anchor=south},
               every axis x label/.style={at=(current axis.right of origin),anchor=west},
             ]        
             \addplot [very thick,penColor,smooth, domain=(-2:2)] {3*x^2+2*.3*x-2)};
           \end{axis}
       \end{tikzpicture}}
       \choice{\begin{tikzpicture}[framed,scale=1,baseline=20ex]
           \begin{axis}[
               xmin=-2,xmax=2,ymin=-8,ymax=8,
               axis lines=center,
               ticks=none,
               width=6in,
               height=3in,
               every axis y label/.style={at=(current axis.above origin),anchor=south},
               every axis x label/.style={at=(current axis.right of origin),anchor=west},
             ]        
             \addplot [very thick,penColor,smooth, domain=(-2:2)] {x^3+.3*x^2-2*x)};
           \end{axis}
       \end{tikzpicture}}
       \choice{\begin{tikzpicture}[framed,scale=1,baseline=20ex]
           \begin{axis}[
               xmin=-2,xmax=2,ymin=-8,ymax=8,
               axis lines=center,
               ticks=none,
               width=6in,
               height=3in,
               every axis y label/.style={at=(current axis.above origin),anchor=south},
               every axis x label/.style={at=(current axis.right of origin),anchor=west},
             ]        
             \addplot [very thick,penColor,smooth, domain=(-2:2)] {6*x+2*.3)};
           \end{axis}
       \end{tikzpicture}}
     \end{multipleChoice}
  \end{question}
\end{example}



\section{The derivative as a function of functions}

While writing $f'$ is viewing the derivative of $f$ as a function in
its own right, the derivatve itself
\[
\ddx
\]
is in fact a function that maps functions to functions,
\begin{align*}
  x^2 &\overset{\ddx}{\to} 2x\\
  f(x) &\overset{\ddx}{\to} f'(x).
\end{align*}

\begin{question}
  As a function, is
  \[
  \ddx
  \]
  one-to-one?
  \begin{multipleChoice}
    \choice{yes}
    \choice[correct]{no}
  \end{multipleChoice}
  \begin{feedback}
    Many different functions share the same derivative since the
    derivative recordes only the slope of the tangent line and not
    the value, or height.
  \end{feedback}
\end{question}

\end{document}

