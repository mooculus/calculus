\documentclass{ximera}

\outcome{Calculate limits using the limit laws.}

\newcommand{\RR}{\mathbb R}
\renewcommand{\d}{\,d}
\newcommand{\dd}[2][]{\frac{d #1}{d #2}}
\renewcommand{\l}{\ell}
\newcommand{\ddx}{\frac{d}{dx}}
\newcommand{\dfn}{\textbf}
\newcommand{\eval}[1]{\bigg[ #1 \bigg]}

\title[Dig-In:]{Limit Laws}
\begin{document}
\begin{abstract}
We give basic laws for working with limits. 
\end{abstract}
\maketitle

%This section comes second.

In this section, we present a handful of rules called the \textit{Limit Laws}
that allow us to find limits of various combinations of functions.

\begin{theorem}[Limit Laws]\index{limit laws}\label{theorem:limit-laws}
Suppose that $\lim_{x\to a}f(x)=L$, $\lim_{x\to a}g(x)=M$.
\begin{description}
%\item[\textbf{Constant Multiple Law}] $\lim_{x\to a} kf(x) = k\lim_{x\to a}f(x)=kL$.
\item[Sum/Difference Law] $\lim_{x\to a} (f(x) \pm g(x)) =
  \lim_{x\to a}f(x) \pm \lim_{x\to a}g(x)=L \pm M$.
\item[Product Law]  $\lim_{x\to a} (f(x)g(x)) = \lim_{x\to
  a}f(x)\cdot\lim_{x\to a}g(x)=LM$.
\item[Quotient Law]  $\lim_{x\to a} \frac{f(x)}{g(x)} =
  \frac{\lim_{x\to a}f(x)}{\lim_{x\to a}g(x)}=\frac{L}{M}$, if
  $M\ne0$.
\end{description}
\label{thm:limit laws}
\end{theorem}

\begin{question}
  True or false: If $f$ and $g$ are continuous functions on an
  interval $I$, then $f\pm g$ is continuous on $I$.
  \begin{multipleChoice}
    \choice[correct]{True}
    \choice{False}
  \end{multipleChoice}
\end{question}

\begin{question}
  True or false: If $f$ and $g$ are continuous functions on an
  interval $I$, then $f/g$ is continuous on $I$.
  \begin{multipleChoice}
    \choice{True}
    \choice[correct]{False}
  \end{multipleChoice}
  \begin{feedback}
    In this case, $f/g$ will not be continuous for $x$ where $g(x) =
    0$.
  \end{feedback}
\end{question}


\begin{example}
  Compute the following limit using limit laws:
  \[
  \lim_{x\to 1}(5x^2+3x-2)
  \]
\begin{explanation}
  Well, get out your pencil and write:
  \begin{align*}
    \lim_{x\to 1} (5x^2+3x-2) &= \lim_{x\to 1} 5x^2 + \lim_{x\to 1} \answer[given]{3x} - \lim_{x\to 1}2 && \text{Sum/Difference Law}\\
    &= 5\lim_{x\to 1} x^2 + 3\lim_{x\to 1} x - \lim_{x\to 1}\answer[given]{2} && \text{Product Law}\\
    &= 5(1)^2 + 3(1) - 2 =\answer[given]{6} && \text{Continuity of $x^k$ and $k$}
  \end{align*}
\end{explanation}  
\end{example}

We can generalize the example above to get the following theorems.

\begin{theorem}
  All polynomial functions, meaning functions of the form
  \[
  f(x) = a_nx^n + a_{n-1}x^{n-1} + \dots + a_1 x + a_0
  \]
  where $n$ is a whole number and each $a_i$ is a real number, are
  continuous for all real numbers.
\end{theorem}

\begin{theorem} 
  Let $f$ and $g$ be polynomials.  Then a rational function, meaning an
  expression of the form
  \[
  h=\frac{f}{g}
  \]
  is continuous for all real numbers except where $g(x)=0$.  That is,
  rational functions are continuous wherever they are defined.
\begin{explanation}
      Let $a$ be a real number such that $g(a)\neq 0$.  Then, since
      $g(x)$ is continuous at $a$, $\lim_{x\to a} g(x) \neq 0$.
      Therefore,
      \begin{align*}
        \lim_{x \to a} h(x) &= \lim_{x\to a} \frac{f(x)}{g(x)} &&\\
        &= \frac{\lim_{x\to a} f(x)}{ \lim_{x\to a} g(x)} && \text{\wordChoice{\choice{Sum/Difference Law}\choice{Product Law}\choice[correct]{Quotient Law}}}\\
        &= \frac{f(a)}{g(a)}=h(a) && \text{Continuity of polynomials}
      \end{align*}
\end{explanation}
\end{theorem}


Now, we give basic rules for how limits interact with composition
of functions.

\begin{theorem}[Limit Laws: Composition]
  If $f(x)$ is continuous at $x = \lim_{x\to a} g(x)$, then
  \[
  \lim_{x\to a} f(g(x)) = f(\lim_{x\to a} g(x)).
  \]
\end{theorem}

Because the limit of a continuous function is the same as the function
value, we can now pass limits inside continuous functions.

\begin{corollary}[Continuity of Composite Functions]
If $g(x)$ is continuous at $a$, then $f(g(x))$ is continuous at $a$.
\end{corollary}

\begin{example}
  Compute the following limit using limit laws:
  \[
  \lim_{x \to 0} \sqrt{\cos(x)}
  \]
  \begin{explanation}
    Write with me:
\begin{align*}
  \lim_{x \to 0} \sqrt{\cos(x)} &= 
  \sqrt{\lim_{x \to 0} (\cos(x))} &&\text{by continuity of $x^k$, assuming $\lim_{x \to 0} \cos(x) >0$}\\
  &= \sqrt{\cos(0)} = \sqrt{1} = 1 &&\text{by continuity of $\cos(x)$}\\
\end{align*}
\end{explanation}
\end{example}

Many of the Limit Laws and theorems about continuity in this section
might seem like they should be obvious.  You may be wondering why we
spent an entire section on these theorems.  The answer is that these
theorems will tell you exactly when it is easy to find the value of a
limit, and exactly what to do in those cases.

The most important thing to learn from this section is whether the
limit laws can be applied for a certain problem, and when we need to
do something more interesting.  We will begin discussing those more
interesting cases in the next section.  For now, we end this section
with a question:

\section{A list of questions}

Let's try this out.

\begin{question}
  Can the following limit can be directly computed by theorems given
  in this section?
  \[
  \lim_{x\to 2}\frac{x^2+3x+2}{x+2} 
  \]
  \begin{multipleChoice}
    \choice[correct]{yes}
    \choice{no}
  \end{multipleChoice}
  \begin{question}
    Compute
    \[
    \lim_{x\to 2}\frac{x^2+3x+2}{x+2} =\answer{3}
    \]
    \begin{feedback}
      Since $f(x)=\frac{x^2+3x+2}{x+2}$ is a rational function, and
      the denominator does not equal $0$, we see that $f(x)$ is
      continuous at $x=2$.  Thus, to find this limit, it suffices to
      plug $2$ into $f(x)$.
    \end{feedback}
  \end{question}
\end{question}


\begin{question}
  Can the following limit can be directly computed by theorems given
  in this section?
  \[
  \lim_{x\to 2}\frac{x^2-3x+2}{x-2}
  \]
  \begin{multipleChoice}
    \choice{yes}
    \choice[correct]{no}
  \end{multipleChoice}
  \begin{feedback}
    $f(x) = \frac{x^2-3x+2}{x-2}$ is a rational function, but the
    denominator $x-2$ equals $0$ when $x=2$. None of our current
    theorems address the situation when the denominator of a fraction
    approaches $0$.
  \end{feedback}
\end{question}


\begin{question}
  Can the following limit can be directly computed by theorems given
  in this section?
  \[
  \lim_{x\to 0} x\sin(1/x)
  \]
  \begin{multipleChoice}
    \choice{yes}
    \choice[correct]{no}
  \end{multipleChoice}
  \begin{feedback}
    If we are trying to use limit laws to compute this limit, we would
    first have to use the Product Law to say that
    \[
    \lim_{x\to 0}x\sin(1/x)= \lim_{x\to 0} x \cdot \lim_{x\to 0} \sin(1/x).
    \]
    We are only allowed to use this law if both limits exist, so we
    must check this first.  We know from continuity that
    \[
    \lim_{x\to  0}x=0.
    \]
    However, we also know that $\sin(1/x)$ oscillates ``wildly'' as
    $x$ approaches $0$, and so the limit
    \[
    \lim_{x\to 0} \sin(1/x)
    \]does not exist.  Therefore, we cannot use the
    Product Law.
  \end{feedback}
\end{question}


\begin{question}
  Can the following limit can be directly computed by theorems given
  in this section?
  \[
  \lim_{x\to 0} \cot(x^3)
  \]
  \begin{multipleChoice}
    \choice{yes}
    \choice[correct]{no}
  \end{multipleChoice}
  \begin{feedback}
    Notice that
    \[
    \cot(x^3) = \frac{\cos(x^3)}{\sin(x^3)}.
    \]
    If we are trying to use limit laws to compute this limit, we would
    like to use the Quotient Law to say that
    \[
    \lim_{x\to 0} \frac{\cos(x^3)}{\sin(x^3)} = \frac{\lim_{x\to 0}
      \cos(x^3)}{\lim_{x\to 0} \sin(x^3)}.
    \]
    We are only allowed to use this law if both limits exist and the
    denominator is not $0$. We suspect that the limit on on the
    denominator might equal $0$, so we check this limit.
    \begin{align*}
      \lim_{x\to 0} \sin(x^3) &= \sin(\lim_{x\to 0}x^3)\\
      &=\sin(0) \\
      &=0.
  \end{align*}
  This means that the denominator is zero and hence we cannot use the
  Quotient Law.
  \end{feedback}
\end{question}


\begin{question}
  Can the following limit can be directly computed by theorems given
  in this section?
  \[
  \lim_{x\to 0}\sec^2(e^x-1)
  \]
  \begin{multipleChoice}
    \choice[correct]{yes}
    \choice{no}
  \end{multipleChoice}
  \begin{question}
    Compute
    \[
    \lim_{x\to 0}\sec^2(e^x-1) =\answer{1}
    \]
    \begin{feedback}
      Notice that
      \[
      \lim_{x\to 0} \sec^2(e^x-1) = \lim_{x\to 0} \frac{1}{\cos^2(e^x-1)}.
      \]
      If we are trying to use Limit Laws to compute this limit, we
      would now have to use the Quotient Law to say that
      \[
      \lim_{x\to 0} \frac{1}{\cos^2(e^x-1)} = \frac{ \lim_{x\to 0}1}{
        \lim_{x\to 0}\cos^2(e^x-1)}.
      \]
      We are only allowed to use this law if both limits exist and the
      denominator is not $0$.  Let's check the denominator and numerator
      separately. First we'll compute the limit of the denominator:
      \begin{align*}
        \lim_{x\to 0}\cos^2(e^x-1) &= \cos^2(\lim_{x\to 0}(e^x-1))\\
        &=\cos^2(\lim_{x\to 0}(e^x)-\lim_{x\to 0}(1))\\
        &=\cos^2(1-1)\\
        &= \cos^2(0)\\
        &=1
      \end{align*}
      Therefore, the limit in the denominator exists and does not
      equal $0$. We can use the Quotient Law, so we will compute the limit of the numerator:
      \[
      \lim_{x\to 0}1=1
      \]
      Hence
      \[
      \frac{ \lim_{x\to 0}1}{ \lim_{x\to 0}\cos^2(e^x-1)} =
      \frac{1}{1}=1
      \]
    \end{feedback}
  \end{question}
\end{question}


\begin{question}
  Can the following limit can be directly computed by theorems given
  in this section?
  \[
  \lim_{x\to 1}{(x-1)\cdot \csc(\ln(x))}
  \]
  \begin{multipleChoice}
    \choice{yes}
    \choice[correct]{no}
  \end{multipleChoice}
  \begin{feedback}
    If we are trying to use limit laws to compute this limit, we would have to use the Product Law to say that
    \[
    \lim_{x\to 1}{(x-1)\cdot \csc(\ln(x))}= \lim_{x\to 1}{(x-1)\cdot \lim_{x\to 1}\csc(\ln(x))}.
    \]
    We are only allowed to use this law if both limits exist.  Let's
    check each limit separately.
    \begin{align*}
      \lim_{x\to 1} (x-1) &= \lim_{x\to 1} (x)-\lim_{x\to 1}(1)\\
      &=1-1\\
      &=0.
    \end{align*}
   So this limit exists. Now we check the other factor.
   Notice that
   \[
   \lim_{x\to 1}\csc(\ln(x)) = \frac{1}{\sin(\ln(x))}.
   \]
   If we are trying to use limit laws to compute this limit, we would now have to use the Quotient Law to say that
   \[
   \frac{1}{\sin(\ln(x))} = \frac{\lim_{x\to 1}1}{\lim_{x\to
       1}\sin(\ln(x))}.
   \]
   We are only allowed to use this law if both limits exist and the
   denominator does not equal $0$.  The limit in the numerator definitely
   exists, so lets check the limit in the denominator.
   \begin{align*}
   \lim_{x\to 1}\sin(\ln(x)) &= \sin(\lim_{x\to 1}\ln(x))\\
   &=\sin(\ln(1))\\
   &= \sin(0)\\
   &= 0
  \end{align*}
  Since the denominator is $0$, we cannot apply the Quotient Law.
  \end{feedback}
\end{question}

\begin{question}
  Can the following limit can be directly computed by theorems given
  in this section?
  \[
  \lim_{x\to 0} x\ln x
  \]
  \begin{multipleChoice}
    \choice{yes}
    \choice[correct]{no}
  \end{multipleChoice}
  \begin{feedback}
  If we are trying to use limit laws to compute this limit, we would
  have to use the Product Law to say that
  \[
  \lim_{x\to 0} x\ln x =\lim_{x\to 0} x \cdot \lim_{x\to 0}\ln x.
  \]
  We are only allowed to use this law if both limits exist.  We know
  $\lim_{x\to 0} x = 0$, but what about $\lim_{x\to 0}\ln x$?  We do
  not know how to find $\lim_{x\to 0}\ln x$ using limit laws because $0$
  is not in the domain of $\ln x$.
  \end{feedback}
\end{question}


\begin{question}
  Can the following limit can be directly computed by theorems given
  in this section?
  \[
  \lim_{x\to0}\frac{2^x-1}{3^{x-1}}
  \]
  \begin{multipleChoice}
    \choice[correct]{yes}
    \choice{no}
  \end{multipleChoice}
  \begin{question}
    Compute
    \[
    \lim_{x\to0}\frac{2^x-1}{3^{x-1}} =\answer{0}
    \]
    \begin{feedback}
      If we are trying to use limit laws to compute this limit, we
      would have to use the Quotient Law to say that
      \[
      \lim_{x\to 0}\frac{2^x-1}{3^{x-1}} = \frac{\lim_{x\to
          0}(2^x-1)}{\lim_{x\to 0}(3^{x-1})}.
      \]
      We are only allowed to use this law if both limits exist and the
      denominator does not equal $0$.  Let's check each limit
      separately, starting with the denominator
      \begin{align*}
        \lim_{x\to 0}(3^{x-1}) &=3^{\lim_{x\to0}(x-1)}\\
        &=3^{-1}\\
        &=\frac{1}{3}
      \end{align*}

      On the other hand the limit in the numerator is
      \begin{align*}
        \lim_{x\to 0}(2^x-1) &=\lim_{x\to0}(2^x)-\lim_{x\to0}(1)\\
        &=1-1\\
        &=0
      \end{align*}
      The limits in both the numerator and denominator exist and the
      limit in the denominator does not equal $0$, so we can use the
      Quotient Law.  We find:
      \[
        \frac{\lim_{x\to 0}(2^x-1)}{\lim_{x\to 0}(3^{x-1})}
        =\frac{0}{\frac{1}{3}}=0.
        \]
    \end{feedback}
  \end{question}
\end{question}


\begin{question}
  Can the following limit can be directly computed by theorems given
  in this section?
  \[
  \lim_{x\to 0}(1+x)^{1/x}
  \]
  \begin{multipleChoice}
    \choice{yes}
    \choice[correct]{no}
  \end{multipleChoice}
  \begin{feedback}
  We do not have any limit laws for functions of the form $f(x)^{g(x)}$, so we cannot compute this limit.
  \end{feedback}
\end{question}

\end{document}
