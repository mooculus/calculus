\documentclass{ximera}

\newcommand{\RR}{\mathbb R}
\renewcommand{\d}{\,d}
\newcommand{\dd}[2][]{\frac{d #1}{d #2}}
\renewcommand{\l}{\ell}
\newcommand{\ddx}{\frac{d}{dx}}
\newcommand{\dfn}{\textbf}
\newcommand{\eval}[1]{\bigg[ #1 \bigg]}


\author{Jim Fowler}

\outcome{Apply the constant multiple and sum rule for vector-valued derivatives.}

\begin{document}

\begin{exercise}
  Suppose $\vec{p} : \R \to \R^3$ and $\vec{q} : \R \to \R^3$ are
  vector-valued functions.

  Further assume $\vec{p}'(0) = \vector{1,1,1}$ and $\vec{q}'(0) =
  \vector{1,2,3}$.

  Find real numbers $a = \answer{2}$ and $b = \answer{-1}$ so that the
  vector-valued function
  \[
    \vec{r}(t) = a \cdot \vec{p}(t) + b\cdot \vec{q}(t)
  \]
  satisfies $\vec{r}'(0) = \vector{1,0,-1}$.

  \begin{hint}
    By the constant multiple and sum rules for vector-valued derivative, $\vec{r}'(t) = a \cdot \vec{p}'(t) + b \cdot \vec{q}'(t)$.
  \end{hint}

  \begin{hint}
    Consequently, we want $\vec{r}'(0) = \vector{1,0,-1}$ to equal $a \cdot \vec{p}'(0) + b \cdot \vec{q}'(0)$.
  \end{hint}

  \begin{hint}
    But $a \cdot \vec{p}'(0) + b \cdot \vec{q}'(0)$ equals $a \cdot \vector{1,1,1} + b \cdot \vector{1,2,3}$.
  \end{hint}

  \begin{hint}
    So we wish to solve $a \cdot \vector{1,1,1} + b \cdot \vector{1,2,3} = \vector{1,0,-1}$.
  \end{hint}

  \begin{hint}
    This is a system of three equations in two unknowns, namely that $a + b = 1$ and $a + 2b = 0$ and $a + 3b = -1$.
  \end{hint}

  \begin{hint}
    Subtracting the second equation from the first reveals that $b = -1$.
  \end{hint}

  \begin{hint}
    Then since $a + b = -1$, it must be that $a = 2$.
  \end{hint}
  
\end{exercise}
\end{document}
