\documentclass{ximera}
\newcommand{\RR}{\mathbb R}
\renewcommand{\d}{\,d}
\newcommand{\dd}[2][]{\frac{d #1}{d #2}}
\renewcommand{\l}{\ell}
\newcommand{\ddx}{\frac{d}{dx}}
\newcommand{\dfn}{\textbf}
\newcommand{\eval}[1]{\bigg[ #1 \bigg]}


\author{}
\license{Creative Commons 3.0 By-NC}
\outcome{}
\begin{document}

\begin{exercise}

  At this point in the course there are very few types of expressions involving square roots that we can integrate.  The common types of integrals that arise in these sections include integrals whose integrand involves:
  \begin{itemize}
    \item
      square roots of linear functions of $x$
      
    \item
      square roots where the expression underneath the square root is a perfect square
    
    \item
      a $u$-substitution.
  \end{itemize}
  
  The next three problems deal with these.
  

\begin{problem}[Square roots of linear expressions in $x$]
  Evaluate $\int \sqrt{4x + 5}$ by filling out the following argument:
  \begin{enumerate}
    \item Let $u = \answer{4x + 5}$, then $\d u = \answer{4} \d x$.
    \item Making these substitutions into the integral gives: $\int \sqrt{4x + 5} \d x = \int \answer{\frac{\sqrt{u}}{4}} \d u$.
    \item After computing the antiderivative in $u$ and substituting, we find:
    \[
      \int \sqrt{4x + 5} \d x = \answer{\frac{1}{6}(4x+5)^{3/2}} + C
    \]
  \end{enumerate}
\end{problem}

\begin{problem}[Perfect squares]
  Expressions of the form $ax^2 + bx + c$ can be written as a sum or difference of squares.
  The following problem reviews how to do this.
  
  Write $x^2 + 6x - 2$ as a sum or difference of squares.
  
  \subsection{Procedure}
  \begin{enumerate}
    \item Take half of the coefficient in front of $x$
    \item Square it and add and subtract it from the expression

      The result so far is: $x^2 + 6x + 9 - 9 - 2$.
      
    \item The first three terms can be be rewritten as 
      \[
        x^2 + 6x + 9 = (x + 3)^2
      \]
      so the expression can be written as
      \[
        x^2 + 6x - 2 = (x+3)^2 - 11.
      \]
  \end{enumerate}
  
 
    Following this example, write $x^2 + 8x + 3$ as a sum or difference of squares:
    \[
      x^2 + 8x + 3 = (x + \answer{4})^2 + \answer{-13}
    \]  
\end{problem}

\begin{problem}
  An integral that will arise in the context of a specific type of problem we will study is
  \[
    \int \sqrt{1 + \left(x - \frac{1}{4x} \right)^2} \d x, \text{ $x > 0$}
  \]
  This is the actual form you will get when answer this type of problem.
  You will be expected to simplify the integrand then compute the antiderivative!
  
  \begin{enumerate}
    \item The expression in the integrand can be simplified.
      Simplify this completely and in your final answer, write all powers of $x$ in the denominator as negative powers of $x$ in the numerator.
      $1 + \left(x - \frac{1}{4x}\right)^2 = \answer{x^2 + \frac{1}{2} + \frac{1}{16} x^{-2}}$
    \item Simplify the expression $\left( x + \frac{1}{4x} \right)^2$
    \[
       \left( x + \frac{1}{4x} \right)^2 = \answer{x^2 + \frac{1}{2} + \frac{1}{16x^2}}
    \]
    \item From this you should see that the expression under the square root is actually a perfect square!
      Using this and the fact that $x > 0$, compute the antiderivative:
      \[
        \int \sqrt{1 + \left( x - \frac{1}{4x} \right)^2} \d x = \answer{\frac{1}{2}x^2 + \frac{1}{4} \ln(x)} + C
      \]
  \end{enumerate}
\end{problem}


\end{exercise}
\end{document}