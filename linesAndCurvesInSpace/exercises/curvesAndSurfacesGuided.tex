\documentclass{ximera}

\newcommand{\RR}{\mathbb R}
\renewcommand{\d}{\,d}
\newcommand{\dd}[2][]{\frac{d #1}{d #2}}
\renewcommand{\l}{\ell}
\newcommand{\ddx}{\frac{d}{dx}}
\newcommand{\dfn}{\textbf}
\newcommand{\eval}[1]{\bigg[ #1 \bigg]}


\author{Jim Talamo}

\license{Creative Commons 3.0 By-SA}
\outcome{Given formulas for a parametric curve, identify the corresponding plot.}

\begin{document}

\begin{exercise}
Given a parameterization of a curve $\mathcal{C}$ and a Cartesian representation $f(x,y,z)=0$ of a surface $\mathcal{S}$, we can ask whether the curve lies on the surface, intersects the surface at finitely many points, or does not intersect the surface.

In order to determine the nature of the intersection of the curve and the surface, denote $\vec{r} (t) = \vector{x(t), y(t) , z(t)}$; that is, there is a common $t$-value for which the $x$, $y$, and $z$-coordinate of any point $(x,y,z)$ on the curve can be found by evaluating $\vec{r}(t)$. We now can substitute into the equation $f(x,y,z)=0$ that gives the surface.

\begin{itemize}
\item If the equation above holds for \emph{all} $t$-values, this means that every point on the curve $\mathcal{C}$ must also lie on the surface $\mathcal{S}$. Hence, the curve $\mathcal{C}$ lies on $\mathcal{S}$.
\item If the equation above holds for finitely many $t$-values, this means that the curve $\mathcal{C}$ intersects the surface at precisely  the $(x,y,z)$ values associated with each of those $t$-values. Thus, the curve $\mathcal{C}$ intersects $\mathcal{S}$ at finitely many points $(x,y,z)$ and these points may be found by evaluating $\vec{r}(t)$ at those $t$-values.
\item If the equation above holds for \emph{no} $t$-values, this means that no point on the curve $\mathcal{C}$ lies on the surface $\mathcal{S}$. Hence, the curve $\mathcal{C}$ and the surface $\mathcal{S}$ do not intersect.
\end{itemize}

Let's try this out.

Suppose that $\mathcal{C}$ is traced out by $\vec{l}(t) = \vector{2t+3, 4t,3-t}$ and the surface $\mathcal{S}$ is described by the equation

 \[\mathcal{S} = \left\{(x,y,z)\in \R^3 ~ \bigg| ~ x+2y-3z=4 \right\}.\]
 
We want to determine whether the curve lies on the surface, intersects the surface at finitely many points, or never intersects the surface. 

To begin, note from the parameterization of $\mathcal{C}$, we find that 

\begin{align*}
x(t)&= \answer{2t+3} \\
y(t)&= \answer{4t} \\
z(t)&= \answer{3-t} \\
\end{align*}

We now must substitute these into the equation that describes $\mathcal{S}$.

\[
[x(t)]+2[y(t)]-3[z(t)] = \left[\answer{2t+3} \right] + 2 \left[  \answer{4t}  \right] -3 \left[ \answer{3-t} \right] = \answer{13t-6}.
\]

In order for the point to lie on the surface, we need the simplified quantity above to equal $4$. Since this happens for \wordChoice{\choice{all}\choice[correct]{some}\choice{no}} $t$-values, the curve \wordChoice{\choice{ lies on the surface}\choice[correct]{intersects the surface at finitely many points}\choice{does not intersect the surface}}.

\begin{exercise}
We find that there is only one $t$-value for which the curve intersects the surface, and this occurs at $t= \answer{\frac{10}{13}}$.  We can find the point $(x,y,z)$ where the curve and surface intersect by evaluating $\vec{r}(t)$ at this $t$-value.

\[
\vec{r}\left(\frac{10}{13} \right) = \vector{\answer{\frac{59}{13}},\answer{\frac{40}{13}},\answer{\frac{29}{13}}},
\]

\begin{feedback}[correct]
Hence, the curve and surface intersect at $(x,y,z) = \left(\answer{\frac{59}{13}},\answer{\frac{40}{13}},\answer{\frac{29}{13}}\right)$.
\end{feedback}
\end{exercise}


\end{exercise}

\end{document}
