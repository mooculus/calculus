\documentclass{ximera}

\newcommand{\RR}{\mathbb R}
\renewcommand{\d}{\,d}
\newcommand{\dd}[2][]{\frac{d #1}{d #2}}
\renewcommand{\l}{\ell}
\newcommand{\ddx}{\frac{d}{dx}}
\newcommand{\dfn}{\textbf}
\newcommand{\eval}[1]{\bigg[ #1 \bigg]}


\outcome{Understand the definition of the derivative at a point.}
\outcome{Compute the derivative of a function at a point.}

\author{Nela Lakos \and Kyle Parsons}

\begin{document}
\begin{exercise}

The graph of a function $p$ and a point in its domain, $b$, are shown in the figure below.

\begin{image}
  \begin{tikzpicture}
    \begin{axis}[
        xmin=-.3,xmax=6.8,ymin=-.3,ymax=5.3,
        clip=false,
        unit vector ratio*=1 1 1,
        axis lines=center,
        grid = major,
        ytick={0,1,...,5},
    xtick={0,1,...,6},
        xlabel=$x$, ylabel=$y$,
        every axis y label/.style={at=(current axis.above origin),anchor=south},
        every axis x label/.style={at=(current axis.right of origin),anchor=west},
      ]
      \addplot[very thick,penColor,domain=-0.0496:6.0496,samples=50] plot{-x*(x-6)/2};
      
      \draw[thick,red] (axis cs:4.3,0) -- (axis cs:4.3,5/2);
      \draw[thick,decorate,decoration={brace,amplitude=10pt},xshift=-24pt] 
      	(axis cs:4.3,0) -- (axis cs:4.3,{4.3*(6-4.3)/2})
		node[midway,xshift=-18pt]{$C$};
	  \draw[thick,dotted] (axis cs:4.3,{4.3*(6-4.3)/2}) -- +(-24pt,0);
      
      
      \draw[thick,cyan] (axis cs:4.3,5/2) -- (axis cs:4.3,{4.3*(6-4.3)/2});
      \draw[thick,decorate,decoration=brace,xshift=-2pt] 
      	(axis cs:4.3,5/2) -- (axis cs:4.3,{4.3*(6-4.3)/2}) 
		node[midway,xshift=-9pt] {$A$};
      
      \draw[thick,red] (axis cs:5,0) -- (axis cs:5,{5*(6-5)/2});
      \draw[thick,decorate,decoration={brace,mirror,aspect=1/3},xshift=35pt] 
      	(axis cs:5,0) -- (axis cs:5,{5*(6-5)/2}) 
		node[pos=1/3,xshift=9pt]{$D$};
	  \draw[thick,dotted] (axis cs:5,5/2) -- +(35pt,0);
	
	  \draw[thick,decorate,decoration={brace,mirror},yshift=-2pt]
	  	(axis cs:4.3,5/2) -- (axis cs:5,5/2) node[midway,yshift=-9pt]{$B$};
		
	  \node[] at (axis cs:1,4.5) {$y=p(x)$};
	  
	  \addplot[thick,only marks,mark=*] coordinates{(5,{5*(6-5)/2})};
	  \node[shift={(20pt,9pt)}] at (axis cs:5,{5*(6-5)/2}) {$(5,p(5))$};
	  
	  \addplot[thick,only marks,mark=*] coordinates{(4.3,{4.3*(6-4.3)/2})};
	  \node[shift={(20pt,5pt)}] at (axis cs:4.3,{4.3*(6-4.3)/2}) {$(b,p(b))$};
	  
	  \node[shift={(0pt,-8pt)}] at (axis cs:4.3,0) {$b$};
      
      \end{axis}`
  \end{tikzpicture}
\end{image}

Match the lengths $A$, $B$, $C$, and $D$, marked in the graph, with the following expressions:

$p(b)-p(5)$ is the length $\answer{A}$.

$5-b$ is the length $\answer{B}$.

$p(5)$ is the length $\answer{D}$.

$p(b)$ is the length $\answer{C}$.

\begin{exercise}

The quotient $\frac{p(b)-p(5)}{b-5}$ is the slope of the \wordChoice{\choice{tangent}\choice{cosecant}\choice[correct]{secant}} line through the points $\left(\answer{b},\answer{p(b)}\right)$ and $\left(5,\answer{p(5)}\right)$.

\begin{exercise}

Provided it exists, the limit $\lim_{t\to5}\frac{p(t)-p(5)}{t-5}$ is the slope of the \wordChoice{\choice[correct]{tangent}\choice{cosecant}\choice{secant}} line to the curve $y=p(t)$ at the point $\left(\answer{5},\answer{p(5)}\right)$.

\end{exercise}
\end{exercise}
\end{exercise}
\end{document}