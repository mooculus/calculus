\documentclass{ximera}

\newcommand{\RR}{\mathbb R}
\renewcommand{\d}{\,d}
\newcommand{\dd}[2][]{\frac{d #1}{d #2}}
\renewcommand{\l}{\ell}
\newcommand{\ddx}{\frac{d}{dx}}
\newcommand{\dfn}{\textbf}
\newcommand{\eval}[1]{\bigg[ #1 \bigg]}


\author{Jim Talamo}
\license{Creative Commons 3.0 By-NC}

%This exercise mimics the other guided exercise remainderIntegralTestGuided.tex but only discusses the upper bound for error

\outcome{Understand the integral test remainder estimates.}

\begin{document}

\begin{exercise}

Consider the series $\sum_{k=1}^{\infty} \frac{2k}{(k^2+1)^2}$.  

Does the integral test apply?

\begin{multipleChoice}
\choice[correct]{Yes.}
\choice{No.}
\end{multipleChoice}

What does the integral test tell us?
\begin{multipleChoice}
\choice{The series converges to $\frac{1}{2}$.}
\choice[correct]{The series converges, but we do not know its value yet.}
\choice{The series diverges.}
\end{multipleChoice}

\begin{feedback}
Note that $\frac{2x}{(x^2+1)^2}$ is positive, decreasing(check for yourself!), and continuous for all $x \geq 1$, and

\[
\int_1^\infty \frac{2x}{(x^2+1)^2} \d x = \lim_{b \to \infty} \eval{-\frac{1}{x^2+1}}_1^b = \lim_{b \to \infty} \eval{-\frac{1}{b^2+1}+\frac{1}{2}}.
\]
Hence, the improper integral converges, so $\sum_{k=1}^{\infty}  \frac{2k}{(k^2+1)^2}$ converges by the integral test.  

Suppose that we want to estimate the value of the series.  Since we do not have a way to find an explicit formula for $s_n=\sum_{k=1}^n \frac{2k}{(k^2+1)^2}$, we need to turn to the remainder estimate that comes with the integral test.

\begin{theorem}[Integral Remainder Upper Bound Estimate]
If $f(x)$ is a function that is positive, increasing, and continuous for $x \geq n_0$,  and $f(n) = a_n$ for every $n \geq n_0$, then 

\[
 r_n \leq \int_{n}^{\infty} f(x) \d x \qquad \textrm{ for all } n \geq n_0,
\]
where $r_n = \sum_{k=n+1}^{\infty} a_k$.
\end{theorem}

\end{feedback}


\begin{exercise}
Suppose we want to approximate $\sum_{k=1}^{\infty} \frac{2k}{(k^2+1)^2}$ as best as we can by using $\sum_{k=1}^{15} \frac{2k}{(k^2+1)^2}$.  

First, we can determine an upper bound for the error from the inequality $ r_n \leq \int_{n}^{\infty} f(x) \d x$. 

We can mimic the calculation done earlier

\[
\int_n^\infty \frac{2x}{(x^2+1)^2} \d x = \lim_{b \to \infty} \eval{\answer{-\frac{1}{x^2+1}}}_n^b = \lim_{b \to \infty} \eval{-\frac{1}{b^2+1}+\frac{1}{n^2+1}} =\answer{\frac{1}{n^2+1}}. 
\]

Since we want to approximate $\sum_{k=1}^{\infty} \frac{2k}{(k^2+1)^2}$ by using $\sum_{k=1}^{15} \frac{2k}{(k^2+1)^2}$, we should use $n=\answer{15}$.

\begin{exercise}
This gives us that 
\[
r_{15} = error \leq\frac{1}{\left( \answer{15}\right)^2+1}, 
\]

which to four decimal places is $\answer[tolerance=.0001]{.0044}$.  

This tells us that $\sum_{k=1}^{15} \frac{2k}{(k^2+1)^2}$ will be no more than $\answer[tolerance=.0001]{.0044}$ off from $\sum_{k=1}^{\infty} \frac{2k}{(k^2+1)^2}$.


Now, we can use technology to compute that to four decimal places, $\sum_{k=1}^{15} \frac{2k}{(k^2+1)^2} = \answer[tolerance=.0002]{.7901}$. 

\end{exercise}
\end{exercise}


\begin{exercise}
Now suppose that we want to ensure that we can approximate $\sum_{k=1}^{\infty} \frac{2k}{(k^2+1)^2}$ to within $.001$ of its exact value.  

We first find $N$ so $\sum_{k=1}^{N} \frac{2k}{(k^2+1)^2}$ is within $.001$ of $\sum_{k=1}^{\infty} \frac{2k}{(k^2+1)^2}$.  To do so, we will use the upper bound for error; that is, we will set $r_N \leq \int_N^\infty \frac{2x}{(x^2+1)^2} \d x \leq .001$.

From our previous work, we have found that

\[
 \int_N^\infty \frac{2x}{(x^2+1)^2} \d x= \answer{\frac{1}{N^2+1}},
\]

so we just have to solve the inequality below.

\begin{align*}
 \int_N^\infty \frac{2x}{(x^2+1)^2} \d x =  \answer{\frac{1}{N^2+1}} &\leq .001 \\
N^2+1 &\geq 1000 \\
N \geq \sqrt{\answer{999}}
\end{align*}

Using technology, we find $N \geq 31.6$, so we use $N=\answer{32}$.

Our approximation is thus $\sum_{k=1}^{32} \frac{2k}{(k^2+1)^2} = \answer[tolerance=.002]{.7933}$

\end{exercise}
\end{exercise}
\end{document}