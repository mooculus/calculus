\documentclass{ximera}

\newcommand{\RR}{\mathbb R}
\renewcommand{\d}{\,d}
\newcommand{\dd}[2][]{\frac{d #1}{d #2}}
\renewcommand{\l}{\ell}
\newcommand{\ddx}{\frac{d}{dx}}
\newcommand{\dfn}{\textbf}
\newcommand{\eval}[1]{\bigg[ #1 \bigg]}


\author{Bart Snapp}

\begin{document}
If $\vec{n}$ is a normal vector to a mirror, and $\vec{v}$
represents a light beam, then the reflected light beam $\vec{r}$ is given by:
\[
\vec{r}=\vec{v} - 2\proj_{\vec{n}}(\vec{v})
\]

\begin{problem}
  Consider the line $x+2y=4$. State a vector normal to the line at the
  point $(2,1)$.
  \begin{prompt}
    \[
    \vec{n} = \vector{\answer{1},\answer{2}}
    \]
  \end{prompt}

  \vfill
  
\end{problem}


\begin{problem}
  Consider the line $x + 2y = 4$ along with vector $\vec{v}$:
  \begin{image}[6in]
    \begin{tikzpicture}
        \begin{axis}[
            xmin=-2,xmax=6,ymin=-2,ymax=4,
            clip=true,
            domain=-2:6,
            axis lines=center,
            %ticks=none,
            unit vector ratio*=1 1 1,
            xlabel=$x$, ylabel=$y$,
            ytick={-2,-1,...,4},
	    xtick={-2,-1,...,6},
	    grid=both,
            grid style={line width=.1pt, draw=gray!50},major grid style={line width=.2pt,draw=black},
            minor tick num=4,
            every axis y label/.style={at=(current axis.above origin),anchor=south},
            every axis x label/.style={at=(current axis.right of origin),anchor=west},
          ]
          \addplot[ultra thick,penColor] {-x/2+2};
          %\addplot[very thick,penColor2,->] plot coordinates {(2,1) (3,3)};
          \addplot[ultra thick,penColor3,->] plot coordinates {(1,3) (2,1)};
          %\addplot[very thick,dashed,penColor3,->] plot coordinates {(2,1) (4,1)};
          \node[left,penColor3] at (axis cs: 1.5,2) {$\vec{v}$};
        \end{axis}
  \end{tikzpicture}
    \end{image}
    Sketch the vector normal vector you found above. Label this normal
    vector $\vec{n}$.
    \begin{prompt}
      \begin{multipleChoice}
        \choice[correct]{I've drawn the vector.}
      \end{multipleChoice}
      \begin{feedback}
          \begin{image}[6in]
    \begin{tikzpicture}
        \begin{axis}[
            xmin=-2,xmax=6,ymin=-2,ymax=4,
            clip=true,
            domain=-2:6,
            axis lines=center,
            %ticks=none,
            unit vector ratio*=1 1 1,
            xlabel=$x$, ylabel=$y$,
            ytick={-2,-1,...,4},
	    xtick={-2,-1,...,6},
	    grid=both,
            grid style={line width=.1pt, draw=gray!50},major grid style={line width=.2pt,draw=black},
            minor tick num=4,
            every axis y label/.style={at=(current axis.above origin),anchor=south},
            every axis x label/.style={at=(current axis.right of origin),anchor=west},
          ]
          \addplot[ultra thick,penColor] {-x/2+2};
          \addplot[ultra thick,penColor2,->] plot coordinates {(2,1) (3,3)};
          \addplot[ultra thick,penColor3,->] plot coordinates {(1,3) (2,1)};
                    \node[left,penColor3] at (axis cs: 1.5,2) {$\vec{v}$};
          \node[above right, penColor2] at (axis cs: 3,3) {$\vec{n}$};
        \end{axis}
  \end{tikzpicture}
    \end{image}
      \end{feedback}
    \end{prompt}
  \end{problem}

  \begin{problem}
    Now draw and label vectors on the graph above to show that:
    \[
    \vec{r}=\vec{v} - 2\proj_{\vec{n}}(\vec{v})
    \]
    \begin{prompt}
      \begin{multipleChoice}
        \choice[correct]{I've drawn the vectors.}
      \end{multipleChoice}
      \begin{feedback}
         \begin{image}[6in]
    \begin{tikzpicture}
        \begin{axis}[
            xmin=-2,xmax=6,ymin=-2,ymax=4,
            clip=true,
            domain=-2:6,
            axis lines=center,
            %ticks=none,
            unit vector ratio*=1 1 1,
            xlabel=$x$, ylabel=$y$,
            ytick={-2,-1,...,4},
	    xtick={-2,-1,...,6},
	    grid=both,
            grid style={line width=.1pt, draw=gray!50},major grid style={line width=.2pt,draw=black},
            minor tick num=4,
            every axis y label/.style={at=(current axis.above origin),anchor=south},
            every axis x label/.style={at=(current axis.right of origin),anchor=west},
          ]
          \addplot[ultra thick,penColor] {-x/2+2};
          \addplot[ultra thick,penColor2,->] plot coordinates {(2,1) (3,3)};
          \addplot[ultra thick,penColor2,dashed] plot coordinates {(2,1) (1,-1)};
          \addplot[ultra thick,penColor3,->] plot coordinates {(1,3) (2,1)};
          \addplot[ultra thick,penColor3,->] plot coordinates {(2,1) (3,-1)};
          \addplot[ultra thick,penColor4,->] plot coordinates {(2,1) (21/5,7/5)};

          \addplot[ultra thick,black,->] plot coordinates {(2,1) (7/5,-1/5)};
          \addplot[ultra thick,black,->] plot coordinates {(18/5,1/5) (3,-1)};
          \addplot[ultra thick,black,->] plot coordinates {(21/5,7/5) (18/5,1/5)};
          
          \node[left,penColor3] at (axis cs: 1.5,2) {$\vec{v}$};
          \node[above left,penColor4] at (axis cs: 16/5,6/5) {$\vec{r}$};
          \node[above right, penColor2] at (axis cs: 3,3) {$\vec{n}$};
          \node[left, black] at (axis cs: 9/5,3/5) {$\proj_{\vec{n}}(\vec{v})$};
          
        \end{axis}
  \end{tikzpicture}
         \end{image}
      \end{feedback}
    \end{prompt}
  \end{problem}

  

  
  \begin{problem}
    Use the dot product to compute: 
    \begin{enumerate}
    \item The angle between $\vec{n}$ and $\vec{v}$.
    \item The angle between $\vec{n}$ and $\vec{r}$.
    \end{enumerate}
    You may leave your answer in terms of $\arccos$. \textbf{Explain}
    why your answers make perfect sense.
    \begin{prompt}
      \begin{enumerate}
      \item The angle between $\vec{n}$ and $\vec{v}$ is $\answer{\arccos(-3/5)}$.
      \item The angle between $\vec{n}$ and $\vec{r}$ is $\answer{\arccos(3/5)}$.
      \end{enumerate}
      \begin{problem}
        From our picture above, we see that that the angle between
        $\vec{n}$ and $\vec{v}$ is $\pi/2$ greater than the angle
        between $\vec{n}$ and $\vec{r}$. Hence the numbers within the
        $\arccos$ are the same except for their sign.
      \end{problem}
    \end{prompt}

    \vfill
    
  \end{problem}
  
\end{document}
