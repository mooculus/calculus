\documentclass{ximera}

\newcommand{\RR}{\mathbb R}
\renewcommand{\d}{\,d}
\newcommand{\dd}[2][]{\frac{d #1}{d #2}}
\renewcommand{\l}{\ell}
\newcommand{\ddx}{\frac{d}{dx}}
\newcommand{\dfn}{\textbf}
\newcommand{\eval}[1]{\bigg[ #1 \bigg]}


\outcome{Use limits to find the slope of the tangent line at a point.}
\outcome{Understand the definition of the derivative at a point.}
\outcome{Compute the derivative of a function at a point.}
\outcome{Write the equation of the tangent line to a graph at a given point.}
\outcome{Recognize and distinguish between secant and tangent lines.}

\author{Nela Lakos \and Kyle Parsons}

\begin{document}
\begin{exercise}

The graph of the function $f(x) = \frac{10}{x+2}$ is given below.

\begin{image}
  \begin{tikzpicture}
    \begin{axis}[
        xmin=-0.2,xmax=8.2,ymin=-0.2,ymax=6.2,
        clip=false,
        unit vector ratio*=1 1 1,
        axis lines=center,
        grid = major,
        ytick={0,1,...,6},
    xtick={0,1,...,8},
        xlabel=$x$, ylabel=$y$,
        every axis y label/.style={at=(current axis.above origin),anchor=south},
        every axis x label/.style={at=(current axis.right of origin),anchor=west},
      ]
		\addplot[very thick,penColor,domain=-0.2:8.2,samples=50] plot{10/(x+2)};
		
		\addplot[only marks,mark=*] coordinates{(0,5) (3,2)};
		\node at (axis cs:0,5) [above right] {$A$};
		\node at (axis cs:3,2) [above right] {$B$};
		
		\addplot[very thick,red,domain=-0.2:5.2] plot{5-x};
		
		\node at (axis cs:4,4.5) {$y=f(x)$};

      \end{axis}`
  \end{tikzpicture}
\end{image}

Let $A$ and $B$ be two points on the graph of $f$ as depicted above.
\[
A  = \left(0,\answer{5}\right) \quad B = \left(3,\answer{2}\right)
\]

\begin{exercise}

The slope, $m_{\text{sec}}$, of the secant line through the points $A$ and $B$ is
\[
m_{\text{sec}} = \answer{-1}.
\]

\begin{exercise}

The figure below depicts the tangent line to the curve $y=f(x)$ at the point B.

\begin{image}
  \begin{tikzpicture}
    \begin{axis}[
        xmin=-0.2,xmax=8.2,ymin=-0.2,ymax=6.2,
        clip=false,
        unit vector ratio*=1 1 1,
        axis lines=center,
        grid = major,
        ytick={0,1,...,6},
    xtick={0,1,...,8},
        xlabel=$x$, ylabel=$y$,
        every axis y label/.style={at=(current axis.above origin),anchor=south},
        every axis x label/.style={at=(current axis.right of origin),anchor=west},
      ]
		\addplot[very thick,penColor,domain=-0.2:8.2,samples=50] plot{10/(x+2)};
		
		\addplot[only marks,mark=*] coordinates{(3,2)};
		\node at (axis cs:3,2) [above right] {$B$};
		
		\addplot[very thick,red,domain=-0.2:8.2] plot{-2/5*(x-3)+2};
		
		\node at (axis cs:4,4.5) {$y=f(x)$};

      \end{axis}`
  \end{tikzpicture}
\end{image}

The slope of the tangent line to the curve $y=f(x)$ at the point $(a,f(a))$ is given by
\[
m_{\text{tan}} = \lim_{x\to a}\frac{f(x)-f(a)}{x-a},
\]
provided the limit exists.

Using the definition above, the slope of the tangent line to $y=f(x)$ at the point $B$ is
\[
m_{\text{tan}} = \answer{\frac{-2}{5}}.
\]

\begin{exercise}

The equation of the tangent line to $y=f(x)$ at the point $B$ is
\[
y=\answer{-\frac{2}{5}(x-3)+2}.
\]

\end{exercise}
\end{exercise}
\end{exercise}
\end{exercise}
\end{document}