\documentclass{ximera}
\newcommand{\RR}{\mathbb R}
\renewcommand{\d}{\,d}
\newcommand{\dd}[2][]{\frac{d #1}{d #2}}
\renewcommand{\l}{\ell}
\newcommand{\ddx}{\frac{d}{dx}}
\newcommand{\dfn}{\textbf}
\newcommand{\eval}[1]{\bigg[ #1 \bigg]}

\author{Steven Gubkin \and Nela Lakos}
\license{Creative Commons 3.0 By-NC}

\outcome{Use limits to find the slope of the tangent line at a point.}
\outcome{Understand the definition of the derivative at a point.}
\outcome{Compute the derivative of a function at a point.}
\outcome{Determine whether a piecewise function is differentiable.}


\begin{document}
\begin{exercise}

Consider 
\[
f(x) = \begin{cases} 
	x^2 &\text{if $x \leq 1$,}\\
	 3x-2 &\text{if  $x > 1$.}
\end{cases}
\]


Determine if $f$ is differentiable at $x=4$, and if it is, find the value of its derivative there.

\begin{prompt}
	If $f$ is differentiable, give its value below.  Otherwise, write "DNE".
	
	$f'(4) = \answer{3}$
\end{prompt}

\end{exercise}
\begin{exercise}

Consider 
\[
f(x) = \begin{cases} 
	x^2 &\text{if $x \leq 1$,}\\
	 3x-2 &\text{if  $x > 1$.}
\end{cases}
\]


Determine if $f$ is differentiable at $x=1$, and if it is, find the value of its derivative there.

Using the limit definition to find the derivative of $f$ we find
\[
f'(1)= \lim_{x\to1}\frac{f(\answer{x})-1}{\answer{x-1}}\\
\]

We cannot find an expression for $f(x)$, because as $x$ approaches $1$, it can be bigger or smaller than $1$, and $f(x)$ has different expression for $x\le1$ and $x>1$.
So, we have to find two one-sided limits separately.
\begin{align*}
 \lim_{x\to1^{-}}\frac{f(\answer{x})-1}{x-1}&= \lim_{x\to1^{-}}\frac{(\answer{x^2})-1}{x-1}\\
 &= \lim_{x\to1^{-}}\frac{(\answer{x+1})(x-1)}{x-1}\\
  &= \lim_{x\to1^{-}}(x+1)\\
  &=\answer{2}
       \end{align*}

On the other hand


\begin{align*}
 \lim_{x\to1^{+}}\frac{f(\answer{x})-1}{x-1}&= \lim_{x\to1^{+}}\frac{(\answer{3x-2})-1}{x-1}\\
 &=\lim_{x\to1^{+}}\frac{\answer{3x-3}}{x-1}\\
 &=\lim_{x\to1^{+}}\frac{3(\answer{x-1})}{x-1}\\
  &=\answer{3}
       \end{align*}
The two one-sided limits are different, and therfore
\[
f'(1)= {\answer{DNE}}\\
\]

\end{exercise}
\end{document}
