\documentclass{ximera}

\newcommand{\RR}{\mathbb R}
\renewcommand{\d}{\,d}
\newcommand{\dd}[2][]{\frac{d #1}{d #2}}
\renewcommand{\l}{\ell}
\newcommand{\ddx}{\frac{d}{dx}}
\newcommand{\dfn}{\textbf}
\newcommand{\eval}[1]{\bigg[ #1 \bigg]}


\author{Jim Talamo \and Bart Snapp}

\outcome{View a vector valued function as a position function.}
\outcome{Differentiate and integrate vector-valued functions.}
\outcome{For a position vector function of time, interpret the derivative as velocity.}
\outcome{For a position vector function of time, interpret the second derivative as acceleration.}

\begin{document}
\begin{exercise}
  The starting position of a particle is given by
  \[
  \vec{p}(0) =\vector{0,0,0}
  \]
  and its acceleration is described by the vector-valued function
  $\vec{a} (t) = \vector{-3,0,-6t}$ for $t \geq 0$. Suppose the
  initial velocity is given by $\vec{v}(0) =\vector{60,10,75}$.
  Find:
\begin{itemize}
\item The velocity function: $\vec{v}(t) =\vector{\answer{-3t+60},\answer{10},\answer{-3t^2+75}}$
\item The speed function: $\vec{s}(t) = \answer{9t^4-441t^2-360t+9325}$ 
\item The position function: $\vec{p}(t) = \vector{\answer{-t^3+60t},\answer{10t},\answer{-t^3+75t}}$
\item The maximum $z$-height of the particle when $t \geq 0$:
  \[
  \text{maximum height:}\quad \answer{250}
  \]
\item The total distance the particle travels from its starting
  location once it hits the ground.
  \[
  \text{total distance:}\quad \answer{25\sqrt{39}}
  \]
\end{itemize}
\end{exercise}
\end{document}
