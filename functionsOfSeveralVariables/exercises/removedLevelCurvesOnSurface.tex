\begin{example}
  Suppose that $F(x,y) = \sqrt{1-\frac{x^2}{9}-\frac{y^2}{4}}$.  Sketch the 
  level curves of $F$ for
  $c=0$, $0.2$, $0.4$, $0.6$, $0.8$ and $1$.
  \begin{explanation}
    First, notice that the domain of $F$ is 
    $\{ (x,y) \in \R^2 :1-\frac{x^2}{9}-\frac{y^2}{4} \geq0 \}$, or the set of 
    all points $(x,y)$ which satisfy $1-\frac{x^2}{9}-\frac{y^2}{4} \geq 0$, and 
    the range of $F$ is $[0,1]$.  It's particularly important to notice that all 
    of the values of $c$ we will use to find level curves are in the range 
    of $F$.

    Each of our level curves will be of the form
      \begin{align*}
        c &= \sqrt{1-\frac{x^2}{9}-\frac{y^2}{4}}\\
        c^2 &= \answer[given]{1-\frac{x^2}{9}-\frac{y^2}{4}}.
      \end{align*}
      Now we substitute all of our values for $c$ and
      plot each of the following implicit functions.
    \begin{align*}
      0^2   &= 1-\frac{x^2}{9}-\frac{y^2}{4}, \\
      0.2^2 &= 1-\frac{x^2}{9}-\frac{y^2}{4}, \\
      0.4^2 &= 1-\frac{x^2}{9}-\frac{y^2}{4}, \\
      0.6^2 &= 1-\frac{x^2}{9}-\frac{y^2}{4}, \\
      0.8^2 &= 1-\frac{x^2}{9}-\frac{y^2}{4}, \\
        1^2 &= 1-\frac{x^2}{9}-\frac{y^2}{4}.   
    \end{align*}
    To make your sketch, either plot these implicit functions with your favorite
    graphing device, or recognize that they are ellipses.
    \begin{onlineOnly}
      As a gesture of friendship, we have included a graph of these
      level curves.
      \[
      \graph[xmin=-5, xmax=5, ymin=-2.3, ymax=2.3]{0^2 = 1-\frac{x^2}{9}-\frac{y^2}{4}, 0.2^2 = 1-\frac{x^2}{9}-\frac{y^2}{4}, 0.4^2 = 1-\frac{x^2}{9}-\frac{y^2}{4},0.6^2 = 1-\frac{x^2}{9}-\frac{y^2}{4},0.8^2 = 1-\frac{x^2}{9}-\frac{y^2}{4},1^2 = 1-\frac{x^2}{9}-\frac{y^2}{4}}
      \]
    \end{onlineOnly}
    Below, we evaluate $F$ on our level curves and plot the resulting 
    curves on the surface $F$.
    \begin{image}
      \begin{tikzpicture}
        \begin{axis}%
          [tick label style={font=\scriptsize},axis on top,
	    axis lines=center,
	    view={135}{25},
	    name=myplot,
	    %xtick=\empty,
	    %ytick=\empty,
	    %ztick=\empty,
	    ymin=-3.1,ymax=3.1,
	    xmin=-3.1,xmax=3.1,
	    zmin=-.1, zmax=2.1,
	    every axis x label/.style={at={(axis cs:\pgfkeysvalueof{/pgfplots/xmax},0,0)},xshift=-3pt,yshift=-3pt},
	    xlabel={\scriptsize $x$},
	    every axis y label/.style={at={(axis cs:0,\pgfkeysvalueof{/pgfplots/ymax},0)},xshift=5pt,yshift=-2pt},
	    ylabel={\scriptsize $y$},
	    every axis z label/.style={at={(axis cs:0,0,\pgfkeysvalueof{/pgfplots/zmax})},xshift=0pt,yshift=4pt},
	    zlabel={\scriptsize $z$},colormap/cool
	  ]
          \addplot3[domain=0:180,mesh,y domain=0:180,samples y=30,very thin,z buffer=sort,samples=30,] ({3*cos(x)*cos(y)},{2*sin(x)*cos(y)},{sin(y)});
          
          \addplot3 [very thick,penColor, smooth,domain=-40:170,samples=60,samples y=0] ({3*(cos(x))},{2*(sin(x))},0);

          \addplot3 [very thick,penColor, smooth,domain=-40:170,samples=60,samples y=0] ({2.93*(cos(x))},{1.96*(sin(x))},.2);
          
          \addplot3 [very thick,penColor, smooth,domain=-40:170,samples=60,samples y=0] ({2.75*(cos(x))},{1.83*(sin(x))},.4);
          
          \addplot3 [very thick,penColor, smooth,domain=-40:170,samples=60,samples y=0] ({2.4*(cos(x))},{1.6*(sin(x))},.6);
          
          \addplot3 [very thick,penColor, smooth,domain=-40:170,samples=60,samples y=0] ({1.8*(cos(x))},{1.2*(sin(x))},.8);
          
          %\filldraw [penColor] (axis cs: 0,0,1) circle (1pt);          
        \end{axis}
      \end{tikzpicture}
    \end{image}
    Notice how the difference between consecutive $c$ values is always $0.2$, so 
    we can use the closeness of the level curves on the $(x,y)$-plane to determine 
    how the surface is changing. Near the level curves of $c=0$ and $c=0.2$ we 
    can both predict (from our sketch of just the level curves) as well as see (on our 
    graph of the curves on the surface) that $F$ indeed is growing quickly.
  \end{explanation}
\end{example}

