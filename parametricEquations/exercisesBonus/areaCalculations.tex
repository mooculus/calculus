\documentclass{ximera}

\newcommand{\RR}{\mathbb R}
\renewcommand{\d}{\,d}
\newcommand{\dd}[2][]{\frac{d #1}{d #2}}
\renewcommand{\l}{\ell}
\newcommand{\ddx}{\frac{d}{dx}}
\newcommand{\dfn}{\textbf}
\newcommand{\eval}[1]{\bigg[ #1 \bigg]}


%\outcome{Find tangent lines to parametric curves}
\author{Jim Talamo}

\begin{document}
\begin{exercise}
Recall that the curve $C$ is described by the parametric equations $x(t)=t^2$ and $y=t$ for $t \geq 0$.  Take $t$ to be measured in seconds.  From the Desmos worksheet you created, you can make some qualitative observations about the area of $R(a)$ and the rate at which the area is changing.  To make more quantitative observations, we first need to find the area.

For any time $t=a$, an integral with respect to $y$ that gives the area is

\[
\textrm{Area}(R(a)) = \int_{y= 0}^{y = \answer{a}} \answer{y^2-2ay+a^2} \d y
\]
(type your answer in terms of $a$).

\begin{hint}
To find the upper limit of integration, use the parametric equations to find $y$ when $t=a$.  To find the integrand, recall that the area is found by computing the integral $\int_{y=c}^{y=d} \left(x_{right}-x_{left} \right) \d y$.
\end{hint}

Evaluating this integral gives

\[
\textrm{Area}(R(a)) = \answer{\frac{1}{3}a^3}
\]

\begin{exercise}
Go back to your Desmos worksheet and set $a=2$, then give responses to the following.

\begin{itemize}
\item When $a=2$, the area of $R(a)$ is $\answer{\frac{8}{3}}$.
\item When $a=2$, the rate at which the area of $R(a)$ is changing is $\answer{4} \unit{units^2/sec}$.
\end{itemize}

\end{exercise}

\end{exercise}
\end{document}
