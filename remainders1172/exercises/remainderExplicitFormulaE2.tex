\documentclass{ximera}

\newcommand{\RR}{\mathbb R}
\renewcommand{\d}{\,d}
\newcommand{\dd}[2][]{\frac{d #1}{d #2}}
\renewcommand{\l}{\ell}
\newcommand{\ddx}{\frac{d}{dx}}
\newcommand{\dfn}{\textbf}
\newcommand{\eval}[1]{\bigg[ #1 \bigg]}


\author{Jim Talamo}
\license{Creative Commons 3.0 By-NC}


\outcome{Understand the relationship between sequences, sequences of partial sums, and remainders.}

\begin{document}

\begin{exercise}

Consider the series $\sum_{k=0}^{\infty} 4\left(\frac{3}{4}\right)^k$.  Notice that the series \wordChoice{\choice[correct]{converges, so we may define a sequence of remainders}\choice{diverges, so we cannot define a sequence of remainders}}.

\begin{exercise}
An explicit formula for $r_n$ is given by

\[
r_n = \answer{16\left(\frac{3}{4}\right)^{n+1}}.
\]

\begin{hint}
In fact, we can use the results for geometric series to show that  $\sum_{k=0}^{\infty} 4\left(\frac{3}{4}\right)^k = \answer{16}$ and that $s_n = \answer{16-16\left(\frac{3}{4}\right)^{n+1}}$ .

From the relationship 

\[
\sum_{k=1}^{\infty} 4\left(\frac{3}{4}\right)^k = s_n+r_n,
\]

we find that an explicit formula for $r_n$ by using the result that when $|r|<1$, we have that $\sum_{k=0}^n ar^k = \frac{a-ar^{n+1}}{1-r}$.  
\end{hint}

%%%%%%%%%%%%%%%%%%%%%%%%%%%%%%%%%%%%%%%%%%%%%%%%%%%%%%%%%%%%%%%
\begin{exercise}
To four decimal places, what error do we expect if we use $\sum_{k=1}^{10} 4\left(\frac{3}{4}\right)^k $ to approximate $\sum_{k=1}^{\infty} 4\left(\frac{3}{4}\right)^k$? 

\[
error \approx \answer[tolerance=.0002]{.6758}
\] 


\begin{hint}
Note that we can answer this by noting that we are really trying to use \wordChoice{\choice{$s_{9}$}\choice[correct]{$s_{10}$}\choice{$s_{11}$}} to approximate the infinite series, so the error in this approximation will be given by \wordChoice{\choice{$r_{9}$}\choice[correct]{$r_{10}$}\choice{$r_{11}$}}.

Since we have the result $r_n = 16\left(\frac{3}{4}\right)^{n+1}$, we notice that the error made using this approximation is $\answer[tolerance=.0002]{.6758}$ (to four decimal places).

\end{hint}

\end{exercise}
%%%%%%%%%%%%%%%%%%%%%%%%%%%%%%%%%%%%%%%%%%%%%%%%%%%%%%%%%%%%%%%
\begin{exercise}
What is the smallest value for $N$ so $\sum_{k=1}^{N} 4\left(\frac{3}{4}\right)^k $ is accurate to within $.05$ of the exact value of the series $\sum_{k=1}^{\infty} 4\left(\frac{3}{4}\right)^k $?  To three decimal places, what is $\sum_{k=0}^N 4\left(\frac{3}{4}\right)^k$ for this value of $N$?

\[
N = \answer[tolerance=.0001]{20} \qquad \textrm{ and } \qquad \sum_{k=1}^N a_k = \answer[tolerance=.001]{15.962}
\]

Make sure to compare your approximation with the actual value of the series once you've finished!


\begin{hint}
We can achieve this by finding $N$ so the error is no more than $.05$.  Mathematically, we find this by setting $r_N \leq .05$.

\begin{align*}
r_N & \leq .05 \\
16\left(\frac{3}{4}\right)^{N+1} & \leq .05 \\
\left(\frac{3}{4}\right)^{N+1} & \leq \frac{.05}{16} \\
(N+1) \ln\left(\frac{3}{4}\right) &\leq \ln\left(\frac{.05}{16}\right) \\ 
(N+1) &\geq \frac{\ln(\frac{.05}{16})}{\ln\left(\frac{3}{4}\right) } \qquad \textrm{(note the inequality sign changes since $\ln\left(\frac{3}{4}\right)<0$)} \\ 
N &\geq \frac{\ln(\frac{.05}{16})}{\ln\left(\frac{3}{4}\right) } -1
\end{align*}

Using technology, we find that $N \approx \answer{19.05}$ to two decimal places, meaning that the smallest value of $N$ we need is $N= \answer{20}.$
\end{hint}


\end{exercise}
%%%%%%%%%%%%%%%%%%%%%%%%%%%%%%%%%%%%%%%%%%%%%%%%%%%%%%%%%%%%%%%



\end{exercise}




\end{exercise}
\end{document}