\documentclass{ximera}

\newcommand{\RR}{\mathbb R}
\renewcommand{\d}{\,d}
\newcommand{\dd}[2][]{\frac{d #1}{d #2}}
\renewcommand{\l}{\ell}
\newcommand{\ddx}{\frac{d}{dx}}
\newcommand{\dfn}{\textbf}
\newcommand{\eval}[1]{\bigg[ #1 \bigg]}


\author{Bart Snapp}

\outcome{Compute tangent planes.}
\outcome{}

\title[Dig-In:]{Tangent planes}

\begin{document}
\begin{abstract}
  We find tangent planes.
\end{abstract}
\maketitle



\subsection{Finding tangent planes}

In your earlier calculus courses, you often found tangent lines to
curves. To do this, you were given a function $f$, a point $x=a$, and
then you produced
\[
y = f'(a)\cdot (x-a) + f(a)
\]
as your tangent line. Using our new notation, we would write this same equation as
\[
T_a(x)=f(a)+f^{(1)}(a)\cdot (x-a)
\]


Now, for a function $F:\R^2\to\R$ which is differentiable at a vector $\vec{a}=\vector{a_1, a_2}$ we would like to use partial derivatives to find the tangent
  plane. 

We know from the definition above that our tangent plane is of the form:
\[
T_a(\vec{x})=c_0+ c_1 (x_1-a_1)+ c_2 (x_2-a_2)
\]

To find the constants, we use the following criteria about the tangent plane:

\[
F(\vec{a}) =T_a(\vec{a})
\]
\[
 F^{(1,0)}(\vec{a})=T_a^{(1,0)}(\vec{a})
\]
\[
F^{(0,1)}(\vec{a})= T_a^{(0,1)}(\vec{a})
\]
 That is, the tangent plane at $\vec{a}$ touches the surface at $\vec{a}$ and the partial derivatives all agree at $\vec{a}$.  Saying the tangent plane has the same partial derivatives as the function is another way of saying that tangent plane has to contain the tangent lines we can find using the partial derivatives.

**INSERT IMAGE**\\
It is now an easy computation to find the constants $c_0, c_1,$ and $c_2$.
\[
F(\vec{a})=T_a(\vec{a})=c_0+ c_1 (a_1-a_1)+ c_2 (a_2-a_2) =\answer[given]{c_0}
\]
\[
F^{(1,0)}(\vec{a})= T_a^{(1,0)}(\vec{a})=\answer[given]{c_1}
\]
\[
F^{(0,1)}(\vec{a})= T_a^{(0,1)}(\vec{a})=\answer[given]{c_2}
\]

Therefore, we now have the equation of the tangent plane 
\[
T_a(\vec{x})=F(\vec{a})+ F^{(1,0)}(\vec{a}) (x_1-a_1)+ F^{(0,1)}(\vec{a}) (x_2-a_2)
\]
We can rewrite this equation to look more like the equation from single variable calculus if we use the dot product.
\[
T_a(\vec{x})=F(\vec{a})+ \vector{F^{(1,0)}(\vec{a}), F^{(0,1)}(\vec{a})} \cdot (\vec{x}-\vec{a})
\]

\begin{question}
  Find a tangent plane to $F(x,y) = 3\cos(x)\sin(y)$ at $(x,y) =
  (\pi/3,\pi/6)$.
  \begin{prompt}
    \[
    z = \answer{(-3\sqrt{3}/4)\cdot (x-\pi/3) + (3\sqrt{3}/4)\cdot (y-\pi/6) + 3/4}
    \]
  \end{prompt}
\end{question}


We can also find tangent planes of surfaces that are defined
parameterically.
\begin{example}
Consider the parameterization of the unit sphere,
centered at the origin where $0\le \theta< 2\pi$ and $0\le\phi\le \pi$:
\begin{align*}
  x(\theta,\phi) &= \cos(\theta)\sin(\phi)\\
  y(\theta,\phi) &= \sin(\theta)\sin(\phi)\\
  z(\theta,\phi) &= \cos(\phi)
\end{align*}
Find a plane tangent to the sphere when $\theta = \pi/4$ and $\phi =
\pi/3$.
\begin{explanation}
  Here we'll use the parametric formula for a plane:
  \[
  \vec{L}(s,t) = \vec{p}+ s \vec{v} + t\cdot \vec{w}
  \]
  The the point $\vec{p}$ (denoted by a vector) is:
  \[
  \vec{p} = \vector{\answer[given]{\sqrt{3/8}},\answer[given]{\sqrt{3/8}},\answer[given]{1/2}}
  \]
  The vector $\vec{v}$ is given by
  \begin{align*}
    \vec{v} &=\eval{\pp{\theta}\vector{x(\theta,\phi),y(\theta,\phi),z(\theta,\phi)}}_{\substack{\theta = \pi/4\\ \phi = \pi/3}}\\
      &= \eval{\vector{\answer[given]{-\sin(\theta)\sin(\phi)},\answer[given]{\cos(\theta)\sin(\phi)},\answer[given]{0}}}_{\substack{\theta = \pi/4\\ \phi = \pi/3}}\\
      &= \vector{-\sqrt{3/8},\sqrt{3/8},0}
  \end{align*}
  And vector $\vec{w}$ is given by
  \begin{align*}
    \vec{w} &=\eval{\pp{\phi}\vector{x(\theta,\phi),y(\theta,\phi),z(\theta,\phi)}}_{\substack{\theta = \pi/4\\ \phi = \pi/3}}\\
      &= \eval{\vector{\answer[given]{\cos(\theta)\cos(\phi)},\answer[given]{\sin(\theta)\cos(\phi)},\answer[given]{-\sin(\phi)}}}_{\substack{\theta = \pi/4\\ \phi = \pi/3}}\\
      &= \vector{1/\sqrt{8},1/\sqrt{8},-\sqrt{3}/2}
  \end{align*}
  Hence our desired plane is given by:
  \[
  \vec{L}(s,t) = \begin{bmatrix}
    \answer[given]{\sqrt{3/8}-s\sqrt{3/8}+ t/\sqrt{8}}\\
    \answer[given]{\sqrt{3/8} +s\sqrt{3/8} + t/\sqrt{8}}\\
    \answer[given]{1/2 -t\sqrt{3}/2}
  \end{bmatrix}
  \]
\end{explanation}
\end{example}



\end{document}
