\documentclass{ximera}

\newcommand{\RR}{\mathbb R}
\renewcommand{\d}{\,d}
\newcommand{\dd}[2][]{\frac{d #1}{d #2}}
\renewcommand{\l}{\ell}
\newcommand{\ddx}{\frac{d}{dx}}
\newcommand{\dfn}{\textbf}
\newcommand{\eval}[1]{\bigg[ #1 \bigg]}


\author{Jim Talamo}

\outcome{Compute the length of a parametric curve.}

\begin{document}
\begin{exercise}
 
Suppose that the curve $C$ in the $xy$-plane is traced out by the vector-valued function 

\[
\vec{r}(t) = \vector{2t^{3/2},4t+1}, 0 \leq t \leq 4.
\] 

In order to determine if the curve is parameterized by arclength, we could check either if

\begin{itemize}
\item $\int_0^s \left|\vec{r}'(t)\right| \d s = s$
\item $\left|\vec{r}'(t)\right|=1$ for all $t$.
\end{itemize}

If either of these holds, then the curve uses arclength as a parameter.  Furthermore, in order to establish the first result, we would have to compute $\left|\vec{r}'(t)\right|$ anyways, so let's take this approach.

We calculate that $\vec{r}'(t) = \vector{\answer{3t^{1/2}},\answer{4}}$, and hence $\left|\vec{r}'(t)\right| = \answer{\sqrt{9t+16}}$.

Since $\left|\vec{r}'(t)\right| \neq 1$ for all $t$, the curve \wordChoice{\choice{does}\choice[correct]{does not}} use arclength as a parameter.

\begin{exercise}
In order to find a description $\vec{p}(s)$ that does use arclength as a parameter, we can take the following steps.

\begin{itemize}
\item[1.] Find $s$ in terms of $t$ by computing $s = \int_0^t \left|\vec{r}'(\tau)\right| \d \tau$.

Since $\left|\vec{r}'(t)\right| = \sqrt{9t+16}$, $\left|\vec{r}'(\tau)\right| = \sqrt{9\tau+16}$, and thus

\begin{align*}
s &= \int_0^t \left|\vec{r}'(\tau)\right| \d \tau \\
&= \int_0^t \sqrt{9\tau+16} \d \tau
\end{align*}

Note that we need to find the antiderivative $\int \sqrt{9x+16} \d x$ to proceed, and 

\[
\int \sqrt{9x+16} \d x = \answer{\frac{2}{27} (9x+16)^{3/2}} +C.
\] 

Thus, we find that $s = \answer{\frac{2}{27} (9t+16)^{3/2} - \frac{128}{27}}$.

\begin{hint}
We can calculate $\int \sqrt{9x+16} \d x$ with the substitution $u=9x+16$.  Don't forget to evaluate the antiderivative at $t=0$ when you find $s$.

\end{hint}

\item[2.] Solve for $t$ in terms of $s$.

This requires some careful algebra, after which we find

\[
t = \answer{\frac{\left(\frac{27}{2}s+64\right)^{2/3}-16}{9}}
\]

\begin{hint}
Let's work through the start of the algebra one step at a time.

\begin{align*}
s &= \frac{2}{27} (9t+16)^{3/2} - \frac{128}{27} \\
s +\frac{128}{27} &=  \frac{2}{27} (9t+16)^{3/2} \\
\frac{27}{2}s +64 &= (9t+16)^{3/2}& (\textrm{ multiply both sides by } \frac{27}{2} \textrm{ and simplify. } ) \\
\left(\frac{27}{2}s +64\right)^{\answer{2/3}} &= 9t+16 \\
\end{align*}

From here, there's not too much more work necessary to solve for $t$.
\end{hint}

\item[3.] We can now find the parameterization $\vec{p}(s)$ by substituting the expression above for $t$ in the original parameterization. 

\begin{align*}
\vec{r}(t) &= \vector{2t^{3/2},4t+1} , 0 \leq t \leq 4 \\
\vec{p}(s) &= \vector{2\left(\answer{\frac{\left(\frac{27}{2}s+64\right)^{2/3}-16}{9} } \right)^{3/2},4\left( \answer{\frac{\left(\frac{27}{2}s+64\right)^{2/3}-16}{9}} \right)+1}
\end{align*} 

Note that we also need to transform the domain as well.  Since the original domain is $0 \leq t \leq 4$, we have $0 \leq \frac{\left(\frac{27}{2}s+64\right)^{2/3}-16}{9} \leq 4$, or 

\[
\textrm{ Domain in } s: ~ 0 \leq s \leq \answer{\frac{2}{27}\left((52)^{3/2}-64\right)}
\]
\end{itemize}

\begin{feedback}[correct]
Note that the arclength parameterization here is quite painful.  In practice, the best we can hope for is that this parameterization is painful, but possible.  Since finding $s$ in terms of $t$ in general requires that we evaluate an integral that has a square root, we often will not be able to compute the necessary antiderivative explicitly.  Even if we can, we still have to solve for $t$ in terms of $s$.  

In general, arclength is not generally a useful parameter to use explicitly, but it is a fundamentally important one to use when formulating definitions that have physical meaning, as a later exercise will explore.  A common theme in results that require this is to take the definition in terms of arclength and establish a computational result in terms of the parameter you want to use.  This will be the subject of a later problem in this assignment.

\end{feedback}
\end{exercise}
 \end{exercise}
\end{document}
