\documentclass{ximera}

\newcommand{\RR}{\mathbb R}
\renewcommand{\d}{\,d}
\newcommand{\dd}[2][]{\frac{d #1}{d #2}}
\renewcommand{\l}{\ell}
\newcommand{\ddx}{\frac{d}{dx}}
\newcommand{\dfn}{\textbf}
\newcommand{\eval}[1]{\bigg[ #1 \bigg]}


\outcome{}

\title[Dig In:]{The derivative of sine}

\begin{document}
\begin{abstract}
  Some abstrct
\end{abstract}
\maketitle

It is now time to visit our two friends who concern themselves
periodically with triangles and circles.

\begin{theorem}[The Derivative of sin(\textit{x})]\index{derivative!of sine}\label{theorem:deriv sin}
\[
\ddx \sin(x) = \cos(x).
\]
\end{theorem}

\begin{align*}
\lim_{h\to 0}\frac{\cos(h)-1}{h} &= \lim_{h\to 0}\left(\frac{\cos(h)-1}{h}\cdot\frac{\cos(h)+1}{\cos(h)+1}\right)\\
&=\lim_{h\to 0}\frac{\cos^2(h)-1}{h(\cos(h)+1)}\\
&=\lim_{h\to 0}\frac{-\sin^2(h)}{h(\cos(h)+1)}\\
&=-\lim_{h\to 0}\left(\frac{\sin(h)}{h}\cdot\frac{\sin(h)}{(\cos(h)+1)}\right)\\
&= -1 \cdot \frac{0}{2} = 0.
\end{align*}

\begin{proof}
Using the definition of the derivative, write
\begin{align*}
\ddx \sin(x) &= \lim_{h\to0} \frac{\sin(x+h)-\sin(x)}{h} \\
&= \lim_{h\to0} \frac{\sin(x)\cos(h)+\sin(h)\cos(x)-\sin(x)}{h}  & \text{Trig Identity.}\\
&= \lim_{h\to0} \left(\frac{\sin(x)\cos(h)-\sin(x)}{h} + \frac{\sin(h)\cos{x}}{h} \right)\\
&=\lim_{h\to0} \left(\sin (x)\frac{\cos(h) - 1}{h}+\cos(x)\frac{\sin(h)}{h}\right) \\
&=\sin(x) \cdot 0 + \cos(x) \cdot 1 = \cos x. & \text{See Example~\ref{example:sinx/x}.}
\end{align*}
\end{proof}

Consider the following geometric interpretation of the derivative of
$\sin(\theta)$.  
%\begin{figure}

\begin{image}
\begin{tikzpicture}
	\begin{axis}[
            xmin=-.1,xmax=1.1,ymin=-.1,ymax=1.1,
            axis lines=center,
            ticks=none,
            width=5in,
            unit vector ratio*=1 1 1,
            xlabel=$x$, ylabel=$y$,
            every axis y label/.style={at=(current axis.above origin),anchor=south},
            every axis x label/.style={at=(current axis.right of origin),anchor=west},
          ]        
          \addplot [very thick, textColor!30!background, smooth, domain=(-.2:.2+pi/2)] ({cos(deg(x))},{sin(deg(x))});
          \addplot [textColor,very thick] plot coordinates {(0,0) (.766,.643)}; %% 40 degrees
          \addplot [textColor,very thick] plot coordinates {(0,0) (.766,0)}; %% bottom
          \addplot [very thick, penColor2!30!background] {(x-.766)*(-.766/.643)+.643};
          \addplot [textColor,dashed] plot coordinates {(0,0) (.766-.196,.643+1-.766)}; %% 40+16.98 degrees          

          %% \addplot [textColor!20!background] plot coordinates {(.766,.643) (1,.839)}; %% hyp
          %% \addplot [textColor!20!background] plot coordinates {(1,.643) (1,.839)}; %% side
          %% \addplot [textColor!20!background] plot coordinates {(.766,.643) (1,.643)}; %% bottom
          %% \addplot [textColor!20!background,smooth, domain=(0:40)] ({.05*cos(x)+.766},{.05*sin(x)+.643}); %% angle
          %% \node at (axis cs:.84,.670) [textColor!20!background] {\footnotesize$\theta$};
          
          %% \addplot [textColor!20!background] plot coordinates {(.766,.643) (.766,.839)}; %% side
          %% \addplot [textColor!20!background] plot coordinates {(.766,.839) (1,.839)}; %% bottom
          %% \addplot [textColor!20!background,smooth, domain=(180:220)] ({.05*cos(x)+1},{.05*sin(x)+.839}); %% angle
          %% \node at (axis cs:.926,.812) [textColor!20!background] {\footnotesize$\theta$};
          
          \draw[rotate around={30:(.5,.5)}] (.7,.7) rectangle (.25,.25);

          %\draw[textColor, rotate around={45:(.5,.5)}] (.5,.5) rectangle (.2,.2);

          \addplot [penColor4,very thick] plot coordinates {(.766,.643) (.766,.643+1-.766)}; %% side
          \addplot [textColor,very thick] plot coordinates {(.766,.643+1-.766) (.766-.196,.643+1-.766)}; %% top
          \addplot [textColor,smooth, domain=(90:130)] ({.05*cos(x)+.766},{.05*sin(x)+.643}); %% angle
          \addplot [very thick, textColor] plot coordinates {(.766-.196,.643+1-.766) (.766,.643)}; %% hyp
          \node at (axis cs:.739,.717) [textColor] {\footnotesize$\theta$};
          
          \node at (axis cs:.668,.877) [anchor=south] {\footnotesize$h\sin(\theta)$};
          \node at (axis cs:.766,.76) [anchor=west] {\footnotesize$h\cos(\theta)$};
          \node at (axis cs:.65,.78) [anchor=west] {\footnotesize$\approx h$};

          \addplot [very thick, penColor] plot coordinates {(.766,0) (.766,.643)}; %% sin theta          
          
          \addplot [textColor, smooth, domain=(0:40)] ({.15*cos(x)},{.15*sin(x)});
          \addplot [textColor, smooth, domain=(40:56.90)] ({.17*cos(x)},{.17*sin(x)});
          \addplot [textColor, smooth, domain=(40:56.90)] ({.185*cos(x)},{.185*sin(x)});
          \node at (axis cs:.15,.07) [anchor=west] {$\theta$};
          \node at (axis cs:.15,.17) {$h$};
          \node at (axis cs:.766,.322) [anchor=east] {$\sin(\theta)$};
          \node at (axis cs:.383,0) [anchor=north] {$\cos(\theta)$};
        \end{axis}
\end{tikzpicture}
\end{image}
%\label{figure:geo-interp sinx/x}
%\end{figure}

Here we see that increasing $\theta$ by a ``small amount'' $h$,
increases $\sin(\theta)$ by approximately $h\cos(\theta)$. Hence,
\[
\frac{\Delta y}{\Delta \theta}\approx \frac{h\cos(\theta)}{h} =
\cos(\theta).
\]

With this said, the derivative of a function measures the slope of the
plot of a function.  If we examine the graphs of the sine and cosine
side by side, it should be that the latter appears to accurately
describe the slope of the former, and indeed this is true, see
Figure~\ref{figure:sin/cos}.
\begin{image}
\begin{tikzpicture}
	\begin{axis}[
            xmin=-6.75,xmax=6.75,ymin=-1.5,ymax=1.5,
            axis lines=center,
            xtick={-6.28, -4.71, -3.14, -1.57, 0, 1.57, 3.142, 4.71, 6.28},
            xticklabels={$-2\pi$,$-3\pi/2$,$-\pi$, $-\pi/2$, $0$, $\pi/2$, $\pi$, $3\pi/2$, $2\pi$},
            ytick={-1,1},
            %ticks=none,
            width=9in,
            height=2in,
            unit vector ratio*=1 1 1,
            xlabel=$x$, ylabel=$y$,
            every axis y label/.style={at=(current axis.above origin),anchor=south},
            every axis x label/.style={at=(current axis.right of origin),anchor=west},
          ]        
          \addplot [very thick, penColor, samples=100,smooth, domain=(-6.75:6.75)] {sin(deg(x))};
          \addplot [very thick, penColor2, samples=100,smooth, domain=(-6.75:6.75)] {cos(deg(x))};
          \node at (axis cs:3.14,.75) [penColor] {$f(x)$};
          \node at (axis cs:-1.57,.75) [penColor2] {$f'(x)$};
        \end{axis}
\end{tikzpicture}
%% \caption{Here we see a plot of $f(x)=\sin(x)$ and its derivative
%%   $f'(x)=\cos(x)$. One can readily see that $\cos(x)$ is positive when
%%   $\sin(x)$ is increasing, and that $\cos(x)$ is negative when
%%   $\sin(x)$ is decreasing.}
%% \label{figure:sin/co
%%  s}
\end{image}

Of course, now that we know the derivative of the sine, we can compute
derivatives of more complicated functions involving sine.

\end{document}
