\documentclass{ximera}

\newcommand{\RR}{\mathbb R}
\renewcommand{\d}{\,d}
\newcommand{\dd}[2][]{\frac{d #1}{d #2}}
\renewcommand{\l}{\ell}
\newcommand{\ddx}{\frac{d}{dx}}
\newcommand{\dfn}{\textbf}
\newcommand{\eval}[1]{\bigg[ #1 \bigg]}


\outcome{Use the first derivative to determine whether a function is increasing or decreasing.}
\outcome{Identify the relationships between the function and its first and second derivatives.}
\outcome{Sketch a graph of the second derivative, given the original function.}
\outcome{Sketch a graph of the original function, given the graph of its first and second derivatives.}
\outcome{State the relationship between concavity and the second derivative.}

\title[Dig-In:]{Concavity}

\begin{document}
\begin{abstract}
  Here we examine what the second derivative tells us about the
  geometry of functions.
\end{abstract}
\maketitle

The graphs of two functions, $f$ and $g$, both increasing on the given interval, are given below.
\begin{image}
  \begin{tikzpicture}
    \draw [penColor,  ultra thick,domain=180:270] plot ({2*cos(\x)+4},{9-2*sin(\x)});
    \draw [penColor, ultra thick,domain=270:360] plot ({2*cos(\x)+8}, {2*sin(\x)+11});
    \draw [color=red,thick, domain=1.92:2.2] plot({\x},{cot(193)*(\x-(2*cos(193)+4))+9-2*sin(193)});
    \draw [color=red, thick, domain=2.3:3.1] plot({\x},{cot(225)*(\x-(2*cos(225)+4))+9-2*sin(225)});    
    \draw [color=red, thick, domain=3.19:4.2] plot({\x},{cot(260)*(\x-(2*cos(260)+4))+9-2*sin(260)});
    \draw [color=red, thick, domain=7.8:9.01] plot({\x},{-cot(280)*(\x-(2*cos(280)+8))+11+2*sin(280)});
    \draw [color=red, thick, domain=9.05:9.79] plot({\x},{-cot(315)*(\x-(2*cos(315)+8))+11+2*sin(315)});
    \draw [color=red, thick, domain=9.82:10.08] plot({\x},{-cot(347)*(\x-(2*cos(347)+8))+11+2*sin(347)});
    \node at (3,7.5) [text width=5cm] {
      The function $f$ is increasing, while the rate itself is decreasing.
      In this case the curve  $y=f(x)$is \dfn{concave down}.};
    \node at (9,7.5) [text width=5cm] {
      The function $g$ is increasing, while the rate itself is increasing.
      In this case the curve  $y=g(x)$is \dfn{concave up}.};
  \end{tikzpicture}
\end{image}


\begin{definition} Let $f$ be a  function differentiable on an open interval $I$.\\
 We say that the graph of  $f$ is \textbf{concave up} on $I$ if  $f'$, the derivative of $f$, is \textbf{increasing} on $I$.\\
We say that the graph of  $f$ is \textbf{concave down} on $I$ if $f'$, the derivative of $f$, is \textbf{decreasing} on $I$.
\end{definition}

We know that the sign of the derivative tells us whether a function is
increasing or decreasing at some point. Likewise, the sign of the
second derivative $f''(x)$ tells us whether $f'(x)$ is increasing or
decreasing at $x$. 
If we are trying to understand the shape of the graph of a function,
knowing where it is concave up and concave down helps us to get a more
accurate picture. This is  summarized in a single theorem.


\begin{theorem}[Test for Concavity]\index{concavity test} Let $I$ be an open interval.


\begin{enumerate}
\item If $f''(x)>0$ for all $x$ in $I$, then the graph of $f$ is  concave up on $I$.
\item If $f''(x)<0$ for all $x$ in $I$, then the graph of $f$ is  concave down on $I$.
\end{enumerate}
\end{theorem}
We summarize the consequences of this theorem  in the table below:


\begin{image}
  \begin{tikzpicture}
    \draw (0,0) -- (0,12);
    \draw (0,0) -- (12,0);
    \draw (6,0) -- (6,12);
    \draw (0,6) -- (12,6);
    \draw (12,0) -- (12,12);
    \draw (0,12) -- (12,12);
    
    \node at (-1.3,9) {\Large$0<f''(x)$};
    \node at (-1.3,3) {\Large$f''(x)<0$};
    \node at (3,12.4) {\Large$f'(x)<0$};
    \node at (9,12.4) {\Large$0<f'(x)$};
    
    \draw [penColor,ultra thick,domain=180:270] plot ({2*cos(\x)+4}, {2*sin(\x)+11});
    \draw [penColor,ultra thick,domain=270:360] plot ({2*cos(\x)+8}, {2*sin(\x)+11});
    \draw [penColor,ultra thick,domain=0:90] plot ({2*cos(\x)+2}, {2*sin(\x)+3});
    \draw [penColor,ultra thick,domain=180:90] plot ({2*cos(\x)+10}, {2*sin(\x)+3});

    \node at (3,7.5) [text width=5cm] {\large
      The function $f$ is decreasing, while the rate itself is increasing.
      In this case the curve $y=f(x)$ is \dfn{concave up}.};

    \node at (9,7.5) [text width=5cm] {\large
     The function $f$ is increasing, while the rate itself is increasing.
      In this case the curve $y=f(x)$ is \dfn{concave up}.};

    \node at (3,1.5) [text width=5cm] {\large
      The function $f$  is decreasing, while the rate itself is decreasing.
      In this case the curve  $y=f(x)$ is \dfn{concave down}.};

    \node at (9,1.5) [text width=5cm] {\large
     The function $f$ is increasing, while the rate itself is decreasing.
      In this case the curve $y=f(x)$ is \dfn{concave down}.};
  \end{tikzpicture}
\end{image}


\begin{example}
  Let $f$ be a continuous function and suppose that:
  \begin{itemize}
  \item $f'(x) > 0$ for $-1< x<1$.
  \item $f'(x) < 0$ for $-2< x<-1$ and $1<x<2$.
  \item $f''(x) > 0$ for $-2<x<0$ and $1<x< 2$.
  \item $f''(x) < 0$ for $0<x< 1$.  
  \end{itemize}
  Sketch a possible graph of $f$.
  \begin{explanation}
    Start by marking points in the domain where the derivative changes sign and indicate
    intervals where $f$ is increasing and intervals $f$ is
    decreasing. The function $f$ has a negative derivative from $-2$
    to $x=\answer[given]{-1}$. This means that $f$ is
    \wordChoice{\choice{increasing}\choice[correct]{decreasing}} on
    this interval. The function $f$ has a positive derivative from
    $x=\answer[given]{-1}$ to $x=\answer[given]{1}$. This means that
    $f$ is
    \wordChoice{\choice[correct]{increasing}\choice{decreasing}} on
    this interval. Finally, The function $f$ has a negative derivative
    from $x=\answer[given]{1}$ to $2$. This means that $f$ is
    \wordChoice{\choice{increasing}\choice[correct]{decreasing}} on
    this interval.
  \begin{image}
    \begin{tikzpicture}
    \begin{axis}[
        xmin=-2,xmax=2,ymin=-2,ymax=2,
        axis lines=center,
        width=6in,
        height=3in,
        every axis y label/.style={at=(current axis.above origin),anchor=south},
        every axis x label/.style={at=(current axis.right of origin),anchor=west},
      ]
      \addplot [dashed, penColor2] plot coordinates {(-1,-2) (-1,2)}; %% Critical points
      \addplot [dashed, penColor2] plot coordinates {(1,-2) (1,2)}; %% Critical points

      \addplot [->, line width=10, penColor!10!background] plot coordinates {(-1+.2,-2+.2) (1-.2,2-.2)};
      \addplot [->, line width=10, penColor!10!background] plot coordinates {(-2+.2,2-.2) (-1-.2,-2+.2)};
      \addplot [->, line width=10, penColor!10!background] plot coordinates {(1+.2,2-.2) (2-.2,-2+.2)}; 
      
      %\addplot [very thick,penColor,smooth, domain=(-2:2)] {x^3+x^2-2*x)};
    \end{axis}
  \end{tikzpicture}
  \end{image}
  Now we should sketch the concavity: \wordChoice{\choice[correct]{concave up}\choice{concave down}} when the second
  derivative is positive, \wordChoice{\choice{concave up}\choice[correct]{concave down}} when the second derivative is
  negative.
    \begin{image}
    \begin{tikzpicture}
    \begin{axis}[
        xmin=-2,xmax=2,ymin=-2,ymax=2,
        axis lines=center,
        width=6in,
        height=3in,
        every axis y label/.style={at=(current axis.above origin),anchor=south},
        every axis x label/.style={at=(current axis.right of origin),anchor=west},
      ]
      \addplot [dashed, penColor2] plot coordinates {(-1,-2) (-1,2)}; %% Critical points
      \addplot [dashed, penColor2] plot coordinates {(1,-2) (1,2)}; %% Critical points

      %\addplot [->, line width=10, penColor!10!background] plot coordinates {(-1+.2,-2+.2) (1-.2,2-.2)};
      %\addplot [->, line width=10, penColor!10!background] plot coordinates {(-2+.2,2-.2) (-1-.2,-2+.2)};
      %\addplot [->, line width=10, penColor!10!background] plot coordinates {(1+.2,2-.2) (2-.2,-2+.2)};

      \addplot [penColor3!20!background,line width=10,domain=180:270] ({-1.1+.7*cos(x)}, {1.2+2*sin(x)});
      \addplot [penColor3!20!background,line width=10,domain=270:360] ({-.9+.7*cos(x)}, {-.1+.7*sin(x)});
      \addplot [penColor3!20!background,line width=10,domain=180:90] ({.9+.7*cos(x)}, {.1+.7*sin(x)});
      \addplot [penColor3!20!background,line width=10,domain=180:265] ({1.9+.7*cos(x)}, {.9+ 2*sin(x)});

      \addplot [->, line width=10, penColor3!20!background] plot coordinates {(-1.1,-.1-.7) (-1,-.1-.7)};
      \addplot [->, line width=10, penColor3!20!background] plot coordinates {(.9,.1+.7) (1,.1+.7)};
      
      \addplot [->, line width=10, penColor3!20!background] plot coordinates {(-.9+.7,-.2) (-.9+.8,.2)};
      \addplot [->, line width=10, penColor3!20!background] plot coordinates {(1.8,-1.1) (2,-1.2)};
      
      %\addplot [very thick,penColor,smooth, domain=(-2:2)] {x^3+x^2-2*x)};
    \end{axis}
  \end{tikzpicture}
    \end{image}
    Finally, we can sketch our curve:
        \begin{image}
    \begin{tikzpicture}
    \begin{axis}[
        xmin=-2,xmax=2,ymin=-2,ymax=2,
        axis lines=center,
        width=6in,
        height=3in,
        every axis y label/.style={at=(current axis.above origin),anchor=south},
        every axis x label/.style={at=(current axis.right of origin),anchor=west},
      ]
      \addplot [dashed, penColor2] plot coordinates {(-1,-2) (-1,2)}; %% Critical points
      \addplot [dashed, penColor2] plot coordinates {(1,-2) (1,2)}; %% Critical points

      \addplot [penColor,ultra thick,domain=-2:1,smooth] {(-x^3+3*x)*.5};
      \addplot [penColor,ultra thick,domain=1:2,smooth] {(-(x-3)^3+3*(x-3))*.5};
    \end{axis}
  \end{tikzpicture}
  \end{image}
  \end{explanation}
\end{example}

\end{document}
