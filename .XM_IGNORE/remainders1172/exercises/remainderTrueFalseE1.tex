\documentclass{ximera}

\newcommand{\RR}{\mathbb R}
\renewcommand{\d}{\,d}
\newcommand{\dd}[2][]{\frac{d #1}{d #2}}
\renewcommand{\l}{\ell}
\newcommand{\ddx}{\frac{d}{dx}}
\newcommand{\dfn}{\textbf}
\newcommand{\eval}[1]{\bigg[ #1 \bigg]}


\author{Jim Talamo}
\license{Creative Commons 3.0 By-NC}


\outcome{Understand the relationship between the sequence of remainders and the convergence of the series.}

\begin{document}

\begin{exercise}

Recall that if $\sum_{k=1}^{\infty} a_k$, we define $s_n = \sum_{k=1}^n a_k$.  When $\sum_{k=1}^{\infty} a_k$ converges, we define $r_n = \sum_{k=n+1}^n a_k$. 

Select all of the following statements that must be true.

\begin{selectAll}
\choice[correct]{If $s_3 = 3$ and $r_3=2$, then $\sum_{k=1}^{\infty} a_k$ \emph{must} converge to $5$.}
\choice[correct]{If $s_n = \frac{n}{n+1}$, then $r_n =\frac{1}{n+1}$.}
\choice{If $s_n = n$, then it is possible that $\lim_{n \to \infty} r_n =0$, where $r_n = \sum_{k=n+1}^{\infty} a_k$}.
\end{selectAll}

\begin{hint}
Think about each of the statements conceptually.

\begin{itemize}
\item We have $s_3=a_1+a_2+a_3 = 3$, and $r_3 = a_4+a_5+a_6+\ldots$ (notice that the series represented by $r_n$ starts with $a_{n+1}$!).  Since $\sum_{k=1}^{\infty} a_k = a_1+a_2+a_3+a_4+a_5 + \ldots$, what can you conclude>

\item We can compute $\lim_{n \to \infty} s_n$ and have a formula for $s_n$, so this can be used to find $r_n$.

\item If $s_n=n$, then $\sum_{k=1}^{\infty} a_k$ \wordChoice{\choice{converges}\choice[correct]{diverges}}.  If $\lim_{n \to \infty} r_n =0$, then there would be a value for $N$ so $r_n \leq 1$ for all $n \geq N$.  But this means that $s_n \leq N+1$ for all $n \geq N$, which is a contradiction.  Hence, we see that statement  (c) is \wordChoice{\choice{true}\choice[correct]{false}}.
\end{itemize}
\end{hint}
\end{exercise}
\end{document}