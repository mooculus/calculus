\documentclass[handout]{ximera}
%handout:  for handout version with no solutions or instructor notes
%handout,instructornotes:  for instructor version with just problems and notes, no solutions
%noinstructornotes:  shows only problem and solutions

%% handout
%% space
%% newpage
%% numbers
%% nooutcomes

%I added the commands here so that I would't have to keep looking them up
%\newcommand{\RR}{\mathbb R}
%\renewcommand{\d}{\,d}
%\newcommand{\dd}[2][]{\frac{d #1}{d #2}}
%\renewcommand{\l}{\ell}
%\newcommand{\ddx}{\frac{d}{dx}}
%\everymath{\displaystyle}
%\newcommand{\dfn}{\textbf}
%\newcommand{\eval}[1]{\bigg[ #1 \bigg]}

%\begin{image}
%\includegraphics[trim= 170 420 250 180]{Figure1.pdf}
%\end{image}

%add a ``.'' below when used in a specific directory.

%\newcommand{\RR}{\mathbb R}
\renewcommand{\d}{\,d}
\newcommand{\dd}[2][]{\frac{d #1}{d #2}}
\renewcommand{\l}{\ell}
\newcommand{\ddx}{\frac{d}{dx}}
\newcommand{\dfn}{\textbf}
\newcommand{\eval}[1]{\bigg[ #1 \bigg]}

\newcommand{\RR}{\mathbb R}
\renewcommand{\d}{\,d}
\newcommand{\dd}[2][]{\frac{d #1}{d #2}}
\renewcommand{\l}{\ell}
\newcommand{\ddx}{\frac{d}{dx}}
\newcommand{\dfn}{\textbf}
\newcommand{\eval}[1]{\bigg[ #1 \bigg]}




\author{Tom Needham}

\outcome{Determine a parametric equation describing a curve.}
\outcome{Sketch a parametrically-defined curve.}
\outcome{Determine the equation of the tangent line to a parametrically-defined curve.}

\title[Collaborate:]{Parametric Equations}

\begin{document}
\begin{abstract}
\end{abstract}
\maketitle

\section{Discussion Questions}

\begin{problem}
Consider the parametric curve defined by 
\begin{align*}
x(t) &= e^{2t} + 3 \\
y(t) &= 1-e^t,
\end{align*}
where $t$ runs over all real numbers. A student claims that the plot of this parametric curve is equal to the graph of the parabola $x = (1-y)^2 + 3$. Is the student correct?
\begin{solution}
solution
\end{solution}
\end{problem}

\begin{problem}
Sketch a plot of the parametric function
\begin{align*}
x(t) &= 2 \sin (t) +2\\
y(t) &= 2 \cos (t),
\end{align*}
where $0 \leq t \leq 2\pi$. Indicate the orientation of the parametric curve on your graph.
\end{problem}


\section{Group Work}

\begin{problem}
For each example, find an equation in the variables $x$ and $y$ whose graph is the same as the graph of the parametrically-defined curve (this procedure is frequently referred to as ``eliminating the parameter").

\begin{itemize}
\item[I.] $x(t) = 2t+3$, $y(t)=2t^2-4$, $-\infty < t < \infty$  
\item[II.] $x(t) = 3 \cos (2t)$, $y(t) = 2 \sin(2t)$, $0 \leq t \leq \frac{\pi}{2}$
\end{itemize}
\begin{solution}
solution.
\end{solution}
\end{problem}

\begin{problem}
\begin{itemize}
\item[I.] Consider the parametric function given by $x(t)=t^3$, $y(t)=t^2$, $-\infty <  t < \infty$. Find the equation of the tangent line to the plot of the parametric curve at the point corresponding to $t=1$ by ``eliminating the parameter". 

\item[II.] Consider the parametric function given by $x(t)=t^3+t$, $y(t)=t^4-3t^2$, $-\infty <  t < \infty$. A student is asked to find the equation of the tangent line to the plot of the parametric curve at the point corresponding to $t=1$. The student attempts to do so by elimating the parameter, but is unable find a Cartesian representation of the curve. They conclude that it is impossible to find the equation of the tangent line. Is the student correct? If not, find the equation of the tangent line.

\item[III.] Using any method you like, find the equation of the tangent line to the plot of the parametric function $x(t)=t^2$, $y(t)=1+2t$, $-2 \leq t \leq 2$ at the point $(x,y)=(1,-1)$. 
\end{itemize}
\end{problem}

\begin{problem}
Find the Cartesian coordinates of all points on the parametric curve
\begin{align*}
x(t) &= t \ln (t) - 1,\\
y(t) &= \frac{t^3}{3}-2t + 4,
\end{align*}
where $0< t \leq 10$, where the tangent line to the curve is either horizontal or vertical.
\end{problem}
\end{document}
