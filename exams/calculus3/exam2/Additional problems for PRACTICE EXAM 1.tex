\documentclass{ximera}
\newcommand{\RR}{\mathbb R}
\renewcommand{\d}{\,d}
\newcommand{\dd}[2][]{\frac{d #1}{d #2}}
\renewcommand{\l}{\ell}
\newcommand{\ddx}{\frac{d}{dx}}
\newcommand{\dfn}{\textbf}
\newcommand{\eval}[1]{\bigg[ #1 \bigg]}

\begin{document}


\begin{problem} \textbf{Compute} the following limit, or \textbf{explain} why the limit does not exist.
  \[
  \lim_{\vector{x,y}\to \vector{0,0}} \frac{2x y + xy^2}{x y}
  \]
  
  \vfill
\end{problem}

\textbf{For problems 2--4,} consider the following contour plot for a
  differentiable function $F : \R^2 \to \R$.  The $x$-axis is
  horizontal and the $y$-axis is vertical.  Assume nothing too
  complicated happens between the contour lines.
\begin{image}[6in]
\includegraphics{contourPlot.png}
\end{image}

\begin{problem}
\textbf{On the plot above, draw} your best guess as to what the
\textbf{gradient vector} looks like at the point $A$.

\vfill

\end{problem}

\begin{problem}
  Do you think $\frac{\partial F}{\partial y}$ is \textbf{positive} or \textbf{negative} at the point $B$?  Why?


\vfill

\end{problem}




\begin{problem}
The point $D$ is at $(-4,3)$ and the point $E$ is at $(4,-2)$. Let
$\vec{v} = \grad F(-4,3)$ and let $\vec{w} = \grad F(4,-2)$.  Do you
think $|\vec{v}|$ or $|\vec{w}|$ is \textbf{larger?}  Why?


\vfill

\end{problem}
\textbf{Problem 5.} Suppose that $\vec{u}$ and $\vec{v}$ are nonzero three dimensional vectors.  \textbf{Show} that $\vec{u}$ and $\vec{v}$ are parallel if and only if $\vec{u} \cross \vec{v} = \vec{v} \cross \vec{u}$.

\vspace{80mm}

\textbf{Problem 6.}The curve $\mathcal{C}$ traced out by a vector-valued function $\vec{p}(s), 0 \leq s \leq 11$ and the points $P_0$ and $P_1$ associated to $\vec{p}(0)$ and $\vec{p}(11)$, respectively, are shown below.  Given that $\vec{p}(t)$ \underline{uses arclength as a parameter}, answer the following questions.  Make sure you \textbf{justify} your responses.


\begin{center}
\resizebox {6cm} {!} { 
    \begin{tikzpicture}
      \begin{axis}%
        [
	  xmin=-5,xmax=4.5,
          ymin=-4.5,ymax=4.5,
          xlabel=$x$,ylabel=$y$,
          axis lines=center,
          every axis y label/.style={at=(current axis.above origin),anchor=south},
          every axis x label/.style={at=(current axis.right of origin),anchor=west},
          clip=false,
	  grid =major,
          width=8cm,
          height=5cm,
          xtick={-4,-3,...,4},
          ytick={-5,-4,...,4},
	]
        \addplot[line join =bevel,penColor,ultra thick] coordinates{
          (-4,-4) (-4,0)(0,3) (2,3) 
        };
        \addplot[color=penColor,fill=penColor,only marks,mark=*] coordinates{(-4,-4)};  %% closed hole
        \addplot[color=penColor,fill=penColor,only marks,mark=*] coordinates{(2,3)};  %% closed hole
        %        \addplot[color=penColor,fill=penColor,only marks,mark=*] coordinates{(-1,-2)};  %% closed hole
        \node[penColor,right] at (axis cs: -4,-4.3) {\large $P_1$};
        \node[penColor,right] at (axis cs: 2,3) {\large $P_0$};
                
      \end{axis}
    \end{tikzpicture}}
\end{center}

\begin{itemize}
\item[A.] \textbf{Find} $s_0$ so $\vec{p}(s_0) = \vector{-4,-2}$.

\vspace{20mm}

\item[B.]  \textbf{Find} $\vec{p}(5)$.

\vspace{20mm}

\item[C.] \textbf{Find} $\vec{p}'(5)$.

\end{itemize}
\vspace{15mm}

\textbf{Problem 7.}\textbf{Multiselect}

\vspace{3mm}

\textbf{Directions:} A perfect response to this question is worth 10 points.  You will be penalized 2 points for each incorrect choice that you circle and each correct choice that you do not circle, but you cannot score below a 0.  Thus, the possible scores for this question are 0, 2, 4, 6, 8, or 10.  

\vspace{3mm}


\textbf{Problem}:  For the statements that follow, $\vec{u}$, $\vec{v}$, and $\vec{w}$ are nonzero three dimensional vectors.  The symbol $\dotp$ refers to the vector dot product, and $\cross$ refers to the vector cross product.  

\vspace{3mm}

\textbf{Circle} all of the following statements that \textbf{must} be true.  No justification is necessary.

\vspace{3mm}
\begin{itemize}
\item[I.] $\Mag{v}^2 = \vec{v} \dotp \vec{v}$ .   \\[3ex]
\item[II.] If $\Proj{u}{v}$ is nonzero and $\vec{u}$ and $\Proj{u}{v}$ are parallel, then $\vec{u}$ and $\vec{v}$ are parallel.  \\[3ex]
\item[III.] If $\vec{u}$ and$ \vec{v}$ are parallel, then $\Proj{w}{u} = \Proj{w}{v}$ .     \\ [3 ex]
\item[IV.] $\Mag{v} \leq  \left|\Proj{u}{v}\right|$.  \\[3ex]
\item[V.]  If $\vec{u}$ and $\vec{v}$ are orthogonal, then $\left|\vec{u}  \cross \vec{v} \right| = \Mag{u}\Mag{v}$.   \\ [3 ex]
\item[VI.] $\vec{u} \dotp \left(\vec{v} \cross \vec{w}\right) = \left(\vec{u} \dotp \vec{v}\right) \cross \vec{w}$  \\[3ex]
%$\vec{u} \dotp \vec{v} = \vec{v} \dotp \vec{u}$ \\[ 3ex]
\item[VII.] $\vec{u} \dotp \left(\vec{u} \cross \vec{v}\right) = 0 $. \\[3ex]
\item[VIII.]  $\vec{u} = \proj_{\veci} \vec{u} + \proj_{\vecj} \vec{u}+\proj_{\veck} \vec{u}$ \\ [3 ex]
\item[IX.] If $\big| \vec{r} \, '(t) \big| = 1$ for all $t$, and $\vec{r}(0) =\vec{0}$, then $\big|\vec{r} (t)\big| = t$ for all $t$.\\[3ex]
\item[X.] If $\big| \vec{r} (t) \big| = 4$ for all $t$, then $\vec{r} \, ' (t) = \vec{0}$.

\vspace{3mm}



\end{itemize}

\textbf{Problem 8.}Suppose that $\vec{r}(t) = \vector{t^3,4+3t^2}$ is defined for all $t$ and let $\mathcal{C}$ be the curve traced out by $\vec{r}(t)$ for $t\geq 0$.

\item[I.] \textbf{Find} all times $t$ at which $\vec{r}(t)$ and $\vec{r} \, ' (t)$ are orthogonal. 

\vspace{35mm}

\item[II.]  \textbf{Find} a vector $\vec{w}$ of length $17$ that is parallel to the tangent line to $\mathcal{C}$ at the point where $t=1$.  

\vspace{35mm}


\item[III.] \textbf{Explain} whether $\vec{r} (t)$ uses arclength as a parameter.  If it does not, find a vector-valued function $\vec{p}(s)$ that traces out $\mathcal{C}$ and uses arclength as a parameter.



\textbf{Problem 9.} \textbf{Short Answer}


\item[I.]  Consider the vector-valued function $\vec{r}(t) = \vector{\frac{(t+4)^2}{e^t+3},\sqrt{1-2t},2+\ln(t+5)}$.  

\begin{itemize}
\item[A.]  \textbf{State} the domain of $\vec{r}(t)$.  No justification is necessary.

\vspace{35mm}

\item[B.]  \textbf{Explain} why the point $(0,3,2)$ lies on the curve traced out by $\vec{r}(t)$, then \textbf{find} a parametric representation of the tangent line to the curve traced out by $\vec{r}(t)$ at $(0,3,2)$.

\vspace{40mm}

\end{itemize}

\item[II.]  Suppose $\vec{u}=8\veci+5\vecj -10\veck$ and $\vec{v}= 2\veci+2\vecj-\veck$. 

\textbf{Find} a vector $\vec{P}$ parallel to $\vec{v}$ and a vector $\vec{N}$ orthogonal to $\vec{v}$ so that  $\vec{u} = \vec{P} + \vec{N}$.  

\vspace{65mm}

\textbf{Problem 10.}
\item[IV.] \textbf{Determine} whether the lines traced out by $\vec{r}_1(t) = \vector{2,1,0} t+\vector{-1,3,2}$ and $\vec{r}_2(T) = \vector{1,0,1} T+\vector{-1,4,1}$ are parallel, intersect each other, or are skew\footnote{Recall that two lines are \emph{skew} if they are nonparallel and nonintersecting.} and \textbf{explain} your response.  

\vspace{75mm}

\item[V.] \textbf{Explain} whether the conditional statement below is true.  

\begin{quote}
``If $\uvec{v}$ is a unit vector in $\R^3$, then for every vector $\vec{u}$ in $\R^3$, $\vec{u} \dotp \uvec{v} = \scal_{\uvec{v}}{\vec{u}}$.'' 
\end{quote}

If it is false, give an explicit counterexample by finding vectors $\vec{u}$ and $\uvec{v}$ for which the hypothesis holds but the conclusion does not.

\newpage
\textbf{Problem 11 } Consider the vector-valued function $\vec{r}(t) =\vector{\cos(t),2\sin(t), \sin(t)}$, $0 \leq t \leq 2\pi$ and the plane defined by the equation $ax+3y-5z = 0$.

\item[I.]  If $a=1$, \textbf{determine} whether the curve traced out by $\vec{r}(t)$ intersects the plane and \textbf{give} the coordinates $(x,y,z)$ of all points of intersection.

\vspace{100mm}

\item[II.] \textbf{Find} a value for $a$ so the curve traced out by $\vec{r}(t)$ never intersects the plane or \textbf{explain} why no such $a$-value exists.

\newpage
\textbf{Problem 12. }\textbf{Circle} the contour plot below that shows the level curves for the plane $z=x+y-1$.

\begin{center}
\resizebox {3.5cm} {!} { 
    \begin{tikzpicture}
      \begin{axis}%
        [
	  xmin=-1.5,xmax=1.5,
          ymin=-1.5,ymax=1.5,
          xlabel=$x$,ylabel=$y$,
          axis lines=center,
          every axis y label/.style={at=(current axis.above origin),anchor=south},
          every axis x label/.style={at=(current axis.right of origin),anchor=west},
          clip=false,
	  grid =major,
          width=8cm,
          height=5cm,
          xtick={-2,-1,...,2},
          ytick={-2,-1,1,2},
	]
            	\addplot [draw=penColor,very thick,smooth,domain=-1.5:-.5] {-2-x};
            	\addplot [draw=penColor,very thick,smooth,domain=-1.5:.5] {-1-x};
            	\addplot [draw=penColor,very thick,smooth,domain=-1.5:1.5] {-x};
            	\addplot [draw=penColor,very thick,smooth,domain=-.5:1.5] {1-x};
            	\addplot [draw=penColor,very thick,smooth,domain=.5:1.5] {2-x};
        %        \addplot[color=penColor,fill=penColor,only marks,mark=*] coordinates{(-1,-2)};  %% closed hole
                
      \end{axis}
    \end{tikzpicture}} 
    \qquad
\resizebox {3.5cm} {!} { 
    \begin{tikzpicture}
      \begin{axis}%
        [
	  xmin=-1.5,xmax=1.5,
          ymin=-1.5,ymax=1.5,
          xlabel=$x$,ylabel=$y$,
          axis lines=center,
          every axis y label/.style={at=(current axis.above origin),anchor=south},
          every axis x label/.style={at=(current axis.right of origin),anchor=west},
          clip=false,
	  grid =major,
          width=8cm,
          height=5cm,
          xtick={-2,-1,...,2},
          ytick={-2,-1,1,2},
	]
            	\addplot [draw=penColor,very thick,smooth,domain=-1.5:-.5] {2+x};
            	\addplot [draw=penColor,very thick,smooth,domain=-1.5:.5] {1+x};
            	\addplot [draw=penColor,very thick,smooth,domain=-1.5:1.5] {x};
            	\addplot [draw=penColor,very thick,smooth,domain=-.5:1.5] {-1+x};
            	\addplot [draw=penColor,very thick,smooth,domain=.5:1.5] {-2+x};
        %        \addplot[color=penColor,fill=penColor,only marks,mark=*] coordinates{(-1,-2)};  %% closed hole
                
      \end{axis}
    \end{tikzpicture}} 
\qquad
\resizebox {3.5cm} {!} { 
    \begin{tikzpicture}
      \begin{axis}%
        [
	  xmin=-1.5,xmax=1.5,
          ymin=-1.5,ymax=1.5,
          xlabel=$x$,ylabel=$y$,
          axis lines=center,
          every axis y label/.style={at=(current axis.above origin),anchor=south},
          every axis x label/.style={at=(current axis.right of origin),anchor=west},
          clip=false,
	  grid =major,
          width=8cm,
          height=5cm,
          xtick={-2,-1,...,2},
          ytick={-2,-1,1,2},
	]
            	\addplot [draw=penColor,very thick,smooth,domain=-1.5:-.75] {3+2*x};
            	\addplot [draw=penColor,very thick,smooth,domain=-1.5:0] {1.5+2*x};
            	\addplot [draw=penColor,very thick,smooth,domain=-.75:.75] {2*x};
            	\addplot [draw=penColor,very thick,smooth,domain=-0:1.5] {-1.5+2*x};
            	\addplot [draw=penColor,very thick,smooth,domain=.75:1.5] {-3+2*x};
        %        \addplot[color=penColor,fill=penColor,only marks,mark=*] coordinates{(-1,-2)};  %% closed hole
                
      \end{axis}
    \end{tikzpicture}} 
\qquad
\resizebox {3.5cm} {!} { 
    \begin{tikzpicture}
      \begin{axis}%
        [
	  xmin=-1.5,xmax=1.5,
          ymin=-1.5,ymax=1.5,
          xlabel=$x$,ylabel=$y$,
          axis lines=center,
          every axis y label/.style={at=(current axis.above origin),anchor=south},
          every axis x label/.style={at=(current axis.right of origin),anchor=west},
          clip=false,
	  grid =major,
          width=8cm,
          height=5cm,
          xtick={-2,-1,...,2},
          ytick={-2,-1,1,2},
	]
            	\addplot [draw=penColor,very thick,smooth,domain=-1.5:-.75] {-3-2*x};
            	\addplot [draw=penColor,very thick,smooth,domain=-1.5:0] {-1.5-2*x};
            	\addplot [draw=penColor,very thick,smooth,domain=-.75:.75] {-2*x};
            	\addplot [draw=penColor,very thick,smooth,domain=-0:1.5] {1.5-2*x};
            	\addplot [draw=penColor,very thick,smooth,domain=.75:1.5] {3-2*x};
        %        \addplot[color=penColor,fill=penColor,only marks,mark=*] coordinates{(-1,-2)};  %% closed hole
                
      \end{axis}
    \end{tikzpicture}} 




\end{center}

\textbf{Problem 13.} \textbf{State} the domain and range of $f(x,y) = \sqrt{4x^4+y^2+25}$.  No justification is necessary.

\vspace{35mm}



\vspace{50mm}
\newpage
\textbf{Problem 14.} (Limits) 

\textbf{Directions:}  For each part of the problem, \textbf{determine} whether the indicated limit exists or does not exist.  

\begin{itemize}
\item If a limit exists, \textbf{calculate} its value.  
\item If a limit does not exist, make sure that you provide clear justification to support your response.  

If your justification involves analyzing the function along different paths, you \textbf{must clearly state} the paths. Notation like $\Lim{(x,y)}{(a,b)^+} f(x,y)$ will be considered ambiguous and will result in a loss of points.
\end{itemize}

Parts I and II of this problem are unrelated.

\item[I.] $\Lim{(x,y)}{(0,0)} \frac{x^3y-xy^3}{x^4+y^4}$.

\vspace{25mm}

\item[II.] Let $f(x,y) =  \begin{cases} 2xy , & x+y \neq 5 \\[2 ex]  x^2y , & x+y = 5 \end{cases} $ . 

\item[A.] $\Lim{(x,y)}{(1,2)}f(x,y)$.

\vspace{25mm}

\item[B.] $\Lim{(x,y)}{(3,2)}f(x,y)$.



 



\end{document}
