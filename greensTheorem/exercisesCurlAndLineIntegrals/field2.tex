\documentclass{ximera}

\newcommand{\RR}{\mathbb R}
\renewcommand{\d}{\,d}
\newcommand{\dd}[2][]{\frac{d #1}{d #2}}
\renewcommand{\l}{\ell}
\newcommand{\ddx}{\frac{d}{dx}}
\newcommand{\dfn}{\textbf}
\newcommand{\eval}[1]{\bigg[ #1 \bigg]}


\outcome{Understand how curl measures local rotation in two dimensions.}
\outcome{Understand that curl need not measure gobal rotation.}

\begin{document}
\begin{exercise}
  Below we have plotted a discrete ``sampling'' of a vector field:
  \begin{image}
    \includegraphics{field2.png}
  \end{image}
  Let $C$ be a circle of radius $3$ centered at the origin drawn in a
  counterclockwise fashion.  What conclusions seem to be true?
  \begin{selectAll}
    \choice[correct]{This is a gradient field.}
    \choice{This is not a gradient field.}
    \choice{This field has positive curl.}
    \choice{This field has negative curl.}
    \choice{$\oint_C \vec{F}\dotp\d\vec{p} < 0$}
    \choice[correct]{$\oint_C \vec{F}\dotp\d\vec{p} = 0$}
    \choice{$\oint_C \vec{F}\dotp\d\vec{p} > 0$}
  \end{selectAll}
  \begin{hint}
    Note that the radius of the circle is irreverent. 
  \end{hint}
\end{exercise}
\end{document}
