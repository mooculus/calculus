\documentclass{ximera}

\newcommand{\RR}{\mathbb R}
\renewcommand{\d}{\,d}
\newcommand{\dd}[2][]{\frac{d #1}{d #2}}
\renewcommand{\l}{\ell}
\newcommand{\ddx}{\frac{d}{dx}}
\newcommand{\dfn}{\textbf}
\newcommand{\eval}[1]{\bigg[ #1 \bigg]}


%\outcome{Find tangent lines to parametric curves}
\author{Jason Miller}

\begin{document}
\begin{exercise}

Express the following polar coordinates in Cartesian coordinates: 

If $\left(r,\theta\right) = \left( 6, \frac{3\pi}{2}\right)$, then in Cartesian coordinates $(x,y) = \left( \answer{0}, \answer{-6 }\right)$.

If $\left(r,\theta\right) = \left(3, \frac{7 \pi}{4}\right)$, then in Cartesian coordinates $(x,y) = \left( \answer{\frac{3\sqrt{2}}{2}}, \answer{\frac{-3\sqrt{2}}{2}} \right)$

If $\left(r,\theta\right) =\left( -5, \frac{4\pi}{3} \right)$, then in Cartesian coordinates $(x,y) = \left( \answer{ \frac{5}{2}}, \answer{\frac{5\sqrt{3}}{2}} \right)$

If $\left(r,\theta\right) =\left( -10, \frac{-5\pi}{3} \right)$, then in Cartesian coordinates $(x,y) = \left( \answer{ -5 },  \answer{ \frac{-10 \sqrt{3}}{2} }     \right)$

\begin{hint}
Use the relationships $x=r\cos(\theta)$ and $y=r\sin(\theta)$.
\end{hint}

\end{exercise}

\begin{exercise}

Express the following Cartesian coordinates in polar coordinates.  In your answer, use $r>0$ and $0 \leq \theta < 2\pi$. 


If $(x,y) =\left( 3, 4 \right)$, then in polar coordinates $(r,\theta) =\left( \answer{ 5}, \answer{  \arctan\left( \frac{4}{3}\right) } \right) $. 

If $(x,y) =\left(-7, 3 \right)$, then in polar coordinates $(r,\theta) =\left( \answer{ \sqrt{58} } , \answer{  \arctan\left( \frac{-3}{7} \right) + \pi    } \right)$. 

If $(x,y) =\left(  3  ,  -8   \right)$, then in polar coordinates $(r,\theta) =\left (  \answer{ \sqrt{73}  }, \answer{\arctan\left( \frac{-8}{3}\right) + 2\pi } \right)$




\end{exercise}
\end{document}