\documentclass{ximera}
\newcommand{\RR}{\mathbb R}
\renewcommand{\d}{\,d}
\newcommand{\dd}[2][]{\frac{d #1}{d #2}}
\renewcommand{\l}{\ell}
\newcommand{\ddx}{\frac{d}{dx}}
\newcommand{\dfn}{\textbf}
\newcommand{\eval}[1]{\bigg[ #1 \bigg]}

\author{Bart Snapp}
\license{Creative Commons 3.0 By-NC}
\outcome{Set up and evaluate a volume integral using the Washer Method}
\begin{document}
\begin{exercise}
  The cross-section of a glass can be modeled by the function $r(x) =
  \frac{x^4}{3}$, with units in centimeters:
  \begin{image}
    \begin{tikzpicture}[
        declare function = {f(\x) = (1/3)* pow(\x,4);} ]
      \begin{axis}[
          xmin =-4,xmax=4,ymax=23,ymin=-.2,
          axis lines=center, xlabel=$x$, ylabel=$y$,
          every axis y label/.style={at=(current axis.above origin),anchor=south},
          every axis x label/.style={at=(current axis.right of origin),anchor=west},
          axis on top,
        ]
        \addplot [draw=none,fill=fill1!50!white,domain=-2.65:2.65, smooth] {.8*sqrt(2.65^2-x^2)+16.8} \closedcycle;
        \addplot [draw=none,fill=fill1,domain=-2.65:2.65, smooth] {-.8*sqrt(2.65^2-x^2)+16.8} \closedcycle; 
        \addplot [draw=none,fill=white,domain=-2.7:2.7, smooth] {f(x)} \closedcycle;
        \addplot [ultra thick,penColor, smooth,domain=-2.75:2.75] {f(x)};

        \draw[penColor,very thick] (axis cs:0,16.8) ellipse (265 and 20);
        \draw[penColor,very thick] (axis cs:0,19) ellipse (275 and 20);

        \node[black] at (axis cs:2.5,3) {$y=\frac{x^4}{3}$};       
      \end{axis}
    \end{tikzpicture}
  \end{image}
  Follow the steps below to determine at what height would one need to place a mark indicating
  $250\unit{ml}$ of fluid:
  
  To start, we should be looking at the following cross-section:
    \begin{image}
    \begin{tikzpicture}[
        declare function = {f(\x) = (1/3)* pow(\x,4);} ]
      \begin{axis}[
          xmin =-4,xmax=4,ymax=23,ymin=-.2,
          axis lines=center, xlabel=$x$, ylabel=$y$,
          every axis y label/.style={at=(current axis.above origin),anchor=south},
          every axis x label/.style={at=(current axis.right of origin),anchor=west},
          axis on top,
        ]
        
        
        \addplot [ultra thick,penColor, smooth,domain=-2.75:2.75] {f(x)};

        \draw[penColor,very thick,fill=fill1] (axis cs:0,16.2) ellipse (265 and 20);
        \draw[penColor,very thick,fill=fill1] (axis cs:0,16.8) ellipse (265 and 20);
        
        \draw[penColor,very thick] (axis cs:0,19) ellipse (275 and 20);
        
        \draw[decoration={brace,mirror,raise=.1cm},decorate,thin] (axis cs:2.7,15.9)--(axis cs:2.7,16.9);  
        \node[anchor=west] at (axis cs:2.9,16.5) {$\Delta y$};

        \node[black] at (axis cs:2.5,3) {$y=\frac{x^4}{3}$};       
      \end{axis}
    \end{tikzpicture}
    \end{image}
    The radius of the disk, given any value of $y$ is given by
    $\answer[given]{(3y)^{1/4}}$. Hence the volume of the
    infinitesimal disk is
    \[
    \d V = \pi \left(\answer[given]{(3y)^{1/4}}\right)^2 \d y
    \]
   
   
   \begin{exercise}
   Summing these all together via integration we find:
    \begin{align*}
      \int_0^h \pi ((3y)^{1/4})^2 \d y &= \int_0^h \pi\answer[given]{3^{1/2} y^{1/2}} \d y\\
      &= \eval{\answer[given]{\frac{2\pi y^{3/2}}{\sqrt{3}}}}_0^h \textrm{ by evaluating the antiderivative.} \\
      &= \answer[given]{\frac{2\pi h^{3/2}}{\sqrt{3}}} 
    \end{align*}
    
    \begin{exercise}
    
    Now that we have a formula for volume, we need to see when it is
    equal to $250$. Write with me:
    \begin{align*}
       \frac{2\pi h^{3/2}}{\sqrt{3}} &= 250\\
        h^{3/2} &=  \answer{\frac{\sqrt{3} \cdot 125}{\pi}},\\
      h &= \answer{\left(\frac{\sqrt{3} \cdot 125}{\pi}\right)^{2/3}}
    \end{align*}
    
    Using a calculator to approximate this height to 2 decimal places, we see that we should put our mark at approximately $16.8$ centimeters.

\end{exercise}
\end{exercise}
\end{exercise}


\end{document}
