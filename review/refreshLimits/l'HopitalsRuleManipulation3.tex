\documentclass{ximera}

\newcommand{\RR}{\mathbb R}
\renewcommand{\d}{\,d}
\newcommand{\dd}[2][]{\frac{d #1}{d #2}}
\renewcommand{\l}{\ell}
\newcommand{\ddx}{\frac{d}{dx}}
\newcommand{\dfn}{\textbf}
\newcommand{\eval}[1]{\bigg[ #1 \bigg]}

\author{Jim Talamo}
\title[Refresh:]{ L'H\^{o}pital's Rule}

\begin{document}
\begin{abstract}
Review algebra with logarithms.
\end{abstract}
\maketitle

\begin{problem}
For many improper integrals involving rational functions, limits of differences of logarithms arise.  The algebraic properties, listed below, are often vital to covert the indeterminate form into a form for which the limit can be evaluated by inspection or L'H\^{o}pital's Rule can be applied. 

\textbf{Algebraic Properties of Logarithms}
\begin{itemize}
\item $\ln(ab) = \ln(a) + \ln(b)$ \\
\item $\ln\left(\frac{a}{b}\right) = \ln(a)-\ln(b)$\\
\item $\ln(a^b) = b \ln(a)$
\end{itemize}

\begin{example}
Compute $\lim_{x \to \infty} \left[ \ln(3x^2+1) - 2 \ln(x) \right]$.

\begin{explanation}
Direct evaluation gives the form $\infty - \infty$, which is an indeterminate form. We cannot use L'H\^{o}pital's Rule yet, but we can use the properties of logarithms.

\begin{align*}
\lim_{x \to \infty} \left[ \ln(3x^2+1) - 2 \ln(x)\right] &= \lim_{x \to \infty}\left[ \ln(3x^2+1) - \ln(x^2) \right]\\
&= \lim_{x \to \infty} \ln\left(\frac{3x^2+1}{x^2}\right) \\
\end{align*} 

Note that by inspection

\[
 \lim_{x \to \infty} \frac{3x^2+1}{x^2} = 3,
\]

so we have that 

\[
 \lim_{x \to \infty} \ln\left(\frac{3x^2+1}{x^2}\right)  = \ln(3).
\]
\end{explanation}
\end{example}

Try your hand at these.  If a limit does not exist, use $\infty$ or $-\infty$ as appropriate or write ``DNE'' otherwise.

\begin{exercise}
\[
\lim_{x \to \infty} \left[\ln(x^2+2x) - \ln(x^3+2) \right]= \answer{-\infty}  
\]

\end{exercise}

\begin{exercise}
\[
\lim_{x \to \infty} \left[3\ln(x+2) - 2\ln(2x-1) \right] = \answer{\infty}  
\]
\end{exercise}

\end{problem}
\end{document}
