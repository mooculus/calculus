\documentclass{ximera}

\newcommand{\RR}{\mathbb R}
\renewcommand{\d}{\,d}
\newcommand{\dd}[2][]{\frac{d #1}{d #2}}
\renewcommand{\l}{\ell}
\newcommand{\ddx}{\frac{d}{dx}}
\newcommand{\dfn}{\textbf}
\newcommand{\eval}[1]{\bigg[ #1 \bigg]}



\outcome{Use integral notation for both antiderivatives and definite integrals.}
\outcome{Compute definite integrals using geometry.}
\outcome{Compute definite integrals using the properties of integrals.}
%\outcome{Justify the properties of definite integrals using algebra or geometry.}
%\outcome{Understand how Riemann sums are used to find exact area.}
\outcome{Define net area.}
%\outcome{Approximate net area.}
\outcome{Split the area under a curve into several pieces to aid with calculations.}
\outcome{Use symmetry to calculate definite integrals.}
\outcome{Explain geometrically why symmetry of a function simplifies calculation of some definite integrals.}

\author{Nela Lakos \and Kyle Parsons}

\begin{document}
\begin{exercise}

Evaluate the following integrals.  Use geometry and properties of the definite integral.
\begin{hint}
Use  geometry, properties of definite integrals and/or symmetry.

Recall: $\int_{-a}^a (any \hspace{0.05in}odd\hspace{0.05in} function)(x) \d x =0$!
\end{hint}
\begin{align*}
\int_{-2}^2 \cos(x)\sin(x) \d x &= \answer{0}\\
\int_{-2}^2 (xe^{x^2} + 14x^5 + \pi x + \pi) \d x &= \answer{4\pi}\\
\int_{-2}^{20} \pi x \d x &= \answer{198\pi}\\
\int_{-3}^3 \sqrt{9-x^2} \d x &= \answer{\frac{9\pi}{2}}\\
\int_{-3}^0 \sqrt{9-x^2} \d x &= \answer{\frac{9\pi}{4}}\\
\int_4^7 \sqrt{9-(x-4)^2} \d x &= \answer{\frac{9\pi}{4}}\\
\int_{10}^{13} \sqrt{9-(x-10)^2} \d x &= \answer{\frac{9\pi}{4}}
\end{align*}

\end{exercise}
\end{document}