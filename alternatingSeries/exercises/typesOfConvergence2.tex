\documentclass{ximera}

\newcommand{\RR}{\mathbb R}
\renewcommand{\d}{\,d}
\newcommand{\dd}[2][]{\frac{d #1}{d #2}}
\renewcommand{\l}{\ell}
\newcommand{\ddx}{\frac{d}{dx}}
\newcommand{\dfn}{\textbf}
\newcommand{\eval}[1]{\bigg[ #1 \bigg]}


\author{Jim Talamo}
\license{Creative Commons 3.0 By-bC}


\outcome{}


\begin{document}
\begin{exercise}

Consider the series:

\[
\sum_{k=1}^{\infty} \frac{(-1)^k}{k^p}
\] 

Select the option below that describes all of the $p$-values for which the series converges:

\begin{multipleChoice}
\choice{$p \geq1$}
\choice{$p > 1$}
\choice{$p =1$}
\choice{$0 < p \leq 1$}
\choice{$0 < p < 1$}
\choice[correct]{$p >0$}
\choice{There are no $p$-values for which the series converges.}
\choice{The series converges for all $p$-values.}
\end{multipleChoice}

Select the option below that describes all of the $p$-values for which the series converges \emph{absolutely}:

\begin{multipleChoice}
\choice{$p \geq1$}
\choice[correct]{$p > 1$}
\choice{$p =1$}
\choice{$0 < p \leq 1$}
\choice{$0 < p < 1$}
\choice{$p >0$}
\choice{There are no $p$-values for which the series converges absolutely.}
\choice{The series converges absolutely for all $p$-values.}
\end{multipleChoice}

Select the option below that describes all of the $p$-values for which the series converges \emph{conditionally}:

\begin{multipleChoice}
\choice{$p \geq1$}
\choice{$p > 1$}
\choice{$p =1$}
\choice[correct]{$0 < p \leq 1$}
\choice{$0 < p < 1$}
\choice{$p >0$}
\choice{There are no $p$-values for which the series converges conditionally.}
\choice{The series converges conditionallyfor all $p$-values.}
\end{multipleChoice}

\begin{hint}
The alternating series test can be used for the first part.  For the remaining two parts, remember that we say a series $\sum_{k=1}^{\infty} a_k$:

\begin{itemize}
\item \emph{converges absolutely} if $\sum_{k=1}^{\infty} |a_k|$ converges.
\item  \emph{converges conditionally} if $\sum_{k=1}^{\infty} a_k$ converges but $\sum_{k=1}^{\infty} |a_k|$ diverges.
\end{itemize}

Use this as well as the results for $p$-series.
\end{hint}

\end{exercise}
\end{document}