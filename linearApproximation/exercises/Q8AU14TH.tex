\documentclass{ximera}

\newcommand{\RR}{\mathbb R}
\renewcommand{\d}{\,d}
\newcommand{\dd}[2][]{\frac{d #1}{d #2}}
\renewcommand{\l}{\ell}
\newcommand{\ddx}{\frac{d}{dx}}
\newcommand{\dfn}{\textbf}
\newcommand{\eval}[1]{\bigg[ #1 \bigg]}


\begin{document}
\outcome{Compute differentials.}
\outcome{Find the linear approximation to a function at a point and use it to approximate the function value.}
The total number of people, $N$, who have contracted a common cold by a time $t$ days after its outbreak on an island is given by
\[
N=N(t)=\frac{200000}{1+100e^{-t/10}},\quad t\ge0.
\]
\begin{exercise}
Express the relationship between a small change in $t$ and the corresponding change in $N$ in the form $dN=N'(t)\d t$.
\[
dN=\answer{\frac{200000 e^{t/10}}{(e^{t/10}+100)^2}}\d t
\]
\begin{exercise}
Suppose $45$ days have passed after the outbreak. Use your answer to estimate the number of people that will fall sick the next day (round your answer to the nearest whole number): $\answer{499}$
\begin{exercise}
Suppose $100$ days have passed after the outbreak. Use the first part to estimate the number of people that will fall sick the next day (round your answer to the nearest whole number): $\answer{9}$
\begin{exercise}
Find the function $\ell_{45}(t)$ whose graph is tangent to $N$ at $t=45$.
\[
\ell_{45}(t)=\answer{\frac{200000 e^{9/2} \left(10 (t-35)+e^{9/2}\right)}{\left(100+e^{9/2}\right)^2}}
\]
\begin{exercise}
Find the function $\ell_{100}(t)$ whose graph is tangent to $N$ at $t=100$.
\[
\ell_{100}(t)=\answer{\frac{200000 e^{10} \left(10 (t-90)+e^{10}\right)}{\left(100+e^{10}\right)^2}}
\]
\end{exercise}
\end{exercise}
\end{exercise}
\end{exercise}
\end{document}
