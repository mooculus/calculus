\documentclass{ximera}

\newcommand{\RR}{\mathbb R}
\renewcommand{\d}{\,d}
\newcommand{\dd}[2][]{\frac{d #1}{d #2}}
\renewcommand{\l}{\ell}
\newcommand{\ddx}{\frac{d}{dx}}
\newcommand{\dfn}{\textbf}
\newcommand{\eval}[1]{\bigg[ #1 \bigg]}


\outcome{Application of Intermediate Value Theorem.}

\author{Nela Lakos }


\begin{document}
\begin{exercise}
On January morning,  at 5 am, the temperature of air at Columbus airport  was $-2^{\circ}$F.
At noon the temperature was $4^{\circ}$F.
Use the Intermediate Theorem to show that at some time that morning the temperature was  exactly $0^{\circ}$F.


Let's solve the problem in steps.

STEP 1

In order to use the Intermediate Value Theorem, we have to define a  suitable function.

Define a function $T$, where $T(t)$,  $5\le t\le 12$, gives the temperature of air at the Columbus airport at  time $t$, measured in $^{\circ}$F.

It is reasonable to assume that $T$ is continuous on its domain, $[5,12]$.

STEP 2

\begin{exercise}

Then,
\[
T(5)=\answer{-2},
T(12)=\answer{4}.
\]

\end{exercise}
\begin{exercise}
 Let $L=0$. Select all the following statements that are correct.

\begin{selectAll}
\choice{Since $T(-2)< L<T(4)$ , the IVT implies that there is a number $c$ in $\left(-2,4\right)$ such that $T(c)=L$.}
\choice[correct]{Since  $T(5) < L<T(12)$, the IVT implies that there is a number $c$ in $\left(5,12\right)$ such that $f(c)=L$.}
\choice{Since $-2< L< 4$ , the IVT implies that there is a number $c$ in $\left(-2,4\right)$ such that $T(c)=L$.}
\choice[correct]{Since $-2< L< 4$, the IVT implies that there is a number $c$ in $\left(5,12\right)$ such that $f(c)=L$.}
\end{selectAll}

\end{exercise}
When we apply the Intermediate Value Theorem to $T$ on the interval $\left[5,12\right]$ with $L=0$, we are guaranteed the existence of at least one point $c$ such that.
\[
T(c)=\answer{0}
\]
\end{exercise}

\end{document}