\documentclass{ximera}

\newcommand{\RR}{\mathbb R}
\renewcommand{\d}{\,d}
\newcommand{\dd}[2][]{\frac{d #1}{d #2}}
\renewcommand{\l}{\ell}
\newcommand{\ddx}{\frac{d}{dx}}
\newcommand{\dfn}{\textbf}
\newcommand{\eval}[1]{\bigg[ #1 \bigg]}


\title{Logarithmic Functions}

\begin{document}
\begin{abstract}
	Logarithmic functions are inverses of exponential functions.
\end{abstract}

\maketitle

The exponential functions with base greater than one are all increasing functions:

\begin{image}
\begin{tikzpicture}
	\begin{axis}[
            domain=-2:2,
            xmin=-2, xmax=2,
            ymin=-2, ymax=4,
            axis lines =middle, xlabel=$x$, ylabel=$y$,
            every axis y label/.style={at=(current axis.above origin),anchor=south},
            every axis x label/.style={at=(current axis.right of origin),anchor=west},
          ]
	  \addplot [very thick, penColor, smooth] {2^x};
          \addplot [very thick, penColor2, smooth] {3^x)};
          \addplot [very thick, penColor4, smooth] {4^x)};

          \node at (axis cs: 1.5, 2.5 ) [penColor,anchor=west] {$2^x$};
          \node at (axis cs:1, 2.8) [penColor2,anchor=west] {$3^x$};
          \node at (axis cs:0.6, 3.5 ) [penColor4,anchor=west] {$4^x$};

        \end{axis}
\end{tikzpicture}
\end{image} 

The exponential functions with base greater less than one are all decreasing functions:

\begin{image}
\begin{tikzpicture}
	\begin{axis}[
            domain=-2:2,
            xmin=-2, xmax=2,
            ymin=-2, ymax=4,
            axis lines =middle, xlabel=$x$, ylabel=$y$,
            every axis y label/.style={at=(current axis.above origin),anchor=south},
            every axis x label/.style={at=(current axis.right of origin),anchor=west},
          ]
	  \addplot [very thick, penColor, smooth] {1.5^-x};
          \addplot [very thick, penColor2, smooth] {2^-x)};
          \addplot [very thick, penColor4, smooth] {4^-x)};

          \node at (axis cs: -1.7, 1 ) [penColor,anchor=west] {$\left( \frac{2}{3}\right)^x$};
          \node at (axis cs:-1.6, 3.3) [penColor2,anchor=west] {$\left(\frac{1}{2}\right)^x$};
          \node at (axis cs:-0.85, 3.5 ) [penColor4,anchor=west] {$\left(\frac{1}{4}\right)^x$};

        \end{axis}
\end{tikzpicture}
\end{image} 

The exponential function with base $1$ is just the constant function $1$, so it is both increasing and decreasing (This is a funny thing about how mathematicians use words.  The definition of increasing and decreasing both technically allow the possibility that the function is constant).

As a result of this we have 

\begin{theorem}
	If $a$ is a positive real number which is not equal to $1$, then the exponential function of base $a$, $y = a^x$, is one-to-one on entire real domain.  Thus it has an inverse.  We call the inverse of $a^x$ the\textbf{ logarithm} with base $a$, and write it as $\log_a$.  The inverse of the natural exponential function $y=e^x$ is called the \textbf{natural logarithm}, and is usually written $\ln$ (although $\log_e$ and just $\log$ are also acceptable).
\end{theorem}



\end{document}