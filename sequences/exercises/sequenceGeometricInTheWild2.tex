\documentclass{ximera}

\newcommand{\RR}{\mathbb R}
\renewcommand{\d}{\,d}
\newcommand{\dd}[2][]{\frac{d #1}{d #2}}
\renewcommand{\l}{\ell}
\newcommand{\ddx}{\frac{d}{dx}}
\newcommand{\dfn}{\textbf}
\newcommand{\eval}[1]{\bigg[ #1 \bigg]}


\author{Jim Talamo and Bart Snapp}
\license{Creative Commons 3.0 By-bC}


\outcome{}


\begin{document}
\begin{exercise}

  Suppose you have $100$ dollars in your bank account and you earn
  $2.25\%$ interest per year. Let $m_n$ be the amount of money in your
  account after $n$ years. 
  
  Determine the amount of money you have in your account after years 1, 2, 3, and 4 to 2 decimal places:
  
  \begin{align*}
m_1&= \answer[tolerance=.03]{102.25} &m_2&= \answer[tolerance=.03]{104.55} &m_3&= \answer[tolerance=.03]{106.90} &m_4&= \answer[tolerance=.03]{109.31} &
\end{align*}
  
\begin{exercise}
This series is:
\begin{multipleChoice}
\choice{arithmetic.}
\choice[correct]{geometric.}
\choice{neither arithmetic nor geometric.}
\end{multipleChoice}

\begin{hint}
    After one year, your bank account has $m_1 = 1.0225\cdot 100$ dollars
    in it. To find the amount in each successive year, you multiply
    again by $1.0225$. Hence this is a geometric sequence!
\end{hint}

This is a geometric sequence and can be given by both an explicit and recursive formula:

\begin{exercise}
The explicit formula is $m_n = \answer[given]{100 \cdot (1.0225)^n}$ for $n \geq 0$. 
\end{exercise}

\begin{exercise}
The recursive formula $m_0 = 100$ and $m_n =
    \answer[given]{1.0225}\cdot m_{n-1}$.
\end{exercise}
\end{exercise}
\end{exercise}
\end{document}
