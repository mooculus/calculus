\documentclass{ximera}

\newcommand{\RR}{\mathbb R}
\renewcommand{\d}{\,d}
\newcommand{\dd}[2][]{\frac{d #1}{d #2}}
\renewcommand{\l}{\ell}
\newcommand{\ddx}{\frac{d}{dx}}
\newcommand{\dfn}{\textbf}
\newcommand{\eval}[1]{\bigg[ #1 \bigg]}


\outcome{Know the graphs and properties of ``famous'' functions.}
\outcome{Understand the definition of a rational function.}

\title[Dig-In:]{Rational functions}


\begin{document}
\begin{abstract}
  Rational functions are functions defined by fractions of
  polynomials.
\end{abstract}
\maketitle


\section{What are rational functions?}

\begin{definition}
  A \dfn{rational function} in the variable $x$ is a function the form
  \[
  f(x) = \frac{p(x)}{q(x)}
  \]
  where $p$ and $q$ are polynomial functions. The domain of a rational
  function is all real numbers except for where the denominator is
  equal to zero.
\end{definition}

\begin{question}
  Which of the following are rational functions?
  \begin{selectAll}
    \choice[correct]{$f(x) = 0$}
    \choice[correct]{$f(x) = \frac{3x+1}{x^2-4x+5}$}
    \choice{$f(x)=e^x$}
    \choice{$f(x)=\frac{\sin(x)}{\cos(x)}$}
    \choice[correct]{$f(x) = -4x^{-3}+5x^{-1}+7-18x^2$}
    \choice{$f(x) = x^{1/2}-x +8$}
    \choice{$f(x)=\frac{\sqrt{x}}{x^3-x}$}
  \end{selectAll}
  \begin{feedback}
    All polynomials can be thought of as rational functions.
  \end{feedback}
\end{question}



\section{What can the graphs look like?}

There is a somewhat wide variation in the graphs of rational
functions.

\begin{example}
    Here we see the the graphs of four rational functions.
\begin{image}
  \begin{tabular}{cc}
    \begin{tikzpicture}
      \begin{axis}[
          xmin=-30,xmax=30,
            ymin=-30,ymax=30,
            domain=-2:2,
            width=2.5in,
            axis lines =middle, xlabel=$x$, ylabel=$y$,
            every axis y label/.style={at=(current axis.above origin),anchor=south},
            every axis x label/.style={at=(current axis.right of origin),anchor=west},
        ]
	\addplot [very thick, penColor, smooth, samples=100, domain=-30:-2.2] {(x^2-3*x+2)/(x+2)};
       	\addplot [very thick, penColor, smooth, samples=100, domain=-1.8:30] {(x^2-3*x+2)/(x+2)};

        \node at (axis cs:10,15) [penColor,anchor=west] {$A$};          
      \end{axis}
    \end{tikzpicture}
    &
    \begin{tikzpicture}
	\begin{axis}[
            xmin=-2,xmax=4,
            ymin=-3,ymax=3,
            width=2.5in,
            axis lines =middle, xlabel=$x$, ylabel=$y$,
            every axis y label/.style={at=(current axis.above origin),anchor=south},
            every axis x label/.style={at=(current axis.right of origin),anchor=west},
          ]
	  \addplot [very thick, penColor2, domain=-2:.9] {1/(x-1)};
          \addplot [very thick, penColor2, domain=1.1:4] {1/(x-1)};
          \addplot[color=penColor2,fill=background,only marks,mark=*] coordinates{(2,1)};  %% open hole
          \node at (axis cs:2.5,1.3) [penColor2] {$B$};
        \end{axis}
    \end{tikzpicture}
        \\
    \begin{tikzpicture}
      \begin{axis}[
          xmin=-1,xmax=5,
          ymin=-30,ymax=30,
          width=2.5in,
          axis lines =middle, xlabel=$x$, ylabel=$y$,
          every axis y label/.style={at=(current axis.above origin),anchor=south},
          every axis x label/.style={at=(current axis.right of origin),anchor=west},
        ]
        \addplot [very thick, penColor3, smooth, samples=100, domain=-1:.95] {(x+2)/(x^2-3*x+2)};
        \addplot [very thick, penColor3, smooth, samples=100, domain=1.1:1.9]  {(x+2)/(x^2-3*x+2)};
        \addplot [very thick, penColor3, smooth, samples=100, domain=2.1:5]  {(x+2)/(x^2-3*x+2)};
        \node at (axis cs:3,7) [penColor3] {$C$};
      \end{axis}
    \end{tikzpicture}
    &
 \begin{tikzpicture}
      \begin{axis}[
          xmin=-2,xmax=4,
          ymin=-3,ymax=3,
          domain=-2:4,
          width=2.5in,
          axis lines =middle, xlabel=$x$, ylabel=$y$,
          every axis y label/.style={at=(current axis.above origin),anchor=south},
          every axis x label/.style={at=(current axis.right of origin),anchor=west},
        ]
	\addplot [very thick, penColor4] {x-1};
        \addplot[color=penColor4,fill=background,only marks,mark=*] coordinates{(2,1)};  %% open hole
        \node at (axis cs:3,1.5) [penColor4] {$D$};
      \end{axis}
    \end{tikzpicture}
  \end{tabular}
\end{image}
Match the curves $A$, $B$, $C$, and $D$ with the functions
  \begin{align*}
    &\frac{x^2-3x+2}{x-2}, &&\frac{x^2-3x+2}{x+2}, \\
    &\frac{x-2}{x^2-3x+2}, &&\frac{x+2}{x^2-3x+2}.
  \end{align*}
\begin{explanation}
  Consider $\frac{x^2-3x+2}{x-2}$. This function is undefined only at
  $x=2$. Of the curves that we see above, $\answer[given]{D}$ is
  undefined exactly at $x=2$.

  Now consider $\frac{x^2-3x+2}{x+2}$. This function is undefined only
  at $x=-2$. The only function above that undefined exactly at $x=-2$
  is curve $\answer[given]{A}$.

  Now consider $\frac{x-2}{x^2-3x+2}$. This function is undefined at
  the roots of
  \[
  x^2-3x+2 = (x-2)(x-1).
  \]
  Hence it is undefined at $x=2$ and $x=1$. It looks like both curves
  $B$ and $C$ would work. Distinguishing between these two curves is
  easy enough if we evaluate at $x=-2$. Check it out.
  \begin{align*}
    \eval{\frac{x-2}{x^2-3x+2}}_{x=-2} &= \frac{-2-2}{(-2)^2-3(-2)+2}\\
    &= \frac{-4}{4+6+2}\\
    &=\frac{-4}{12}.
  \end{align*}
  Since this is negative, we see that $\frac{x-2}{x^2-3x+2}$
  corresponds to curve $\answer[given]{B}$.

  Finally, it must be the case that curve $\answer[given]{C}$
  corresponds to $\frac{x+2}{x^2-3x+2}$. We should note that if this
  function is evaluated at $x=-2$, the output is zero, and this
  corroborates our work above.
\end{explanation}
\end{example}


%% \section{Connections to polynomials}
%% $\frac{x^2-3x+2}{x-2} \ne x-1$

\end{document}
