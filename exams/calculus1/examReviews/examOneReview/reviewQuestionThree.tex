\documentclass{ximera}
\newcommand{\RR}{\mathbb R}
\renewcommand{\d}{\,d}
\newcommand{\dd}[2][]{\frac{d #1}{d #2}}
\renewcommand{\l}{\ell}
\newcommand{\ddx}{\frac{d}{dx}}
\newcommand{\dfn}{\textbf}
\newcommand{\eval}[1]{\bigg[ #1 \bigg]}

\outcome{Calculate limits using the limit laws.}
\author{Tom Dinitz and Nela Lakos}
\license{Creative Commons 3.0 By-NC}
\begin{document}

\begin{exercise}

  Let
  \[
  f(x) =
  \begin{cases}
    \frac{x^2-x-12}{x+3} &\text{if $x<4$ and $x\ne -3$}\\
    5 &\text{if $x=-3$}\\
    \frac{x}{x-4} &\text{if $x>4$}
  \end{cases}
  \]
  (i) Determine if the following limits exist. If they do not, say `DNE'. Note: You may not use a table of values, a graph, or L'Hospital's Rule to justify your answer.

  (a) $\lim_{x\to -3} f(x)=\begin{prompt}\answer{-7}\end{prompt}$

  (b) $\lim_{x\to 4^-} f(x)=\begin{prompt}\answer{0}\end{prompt}$

  (c) $\lim_{x\to 4^+} f(x)=\begin{prompt}\answer{\infty}\end{prompt}$

  (d) $\lim_{x\to 4} f(x)=\begin{prompt}\answer[format=string]{DNE}\end{prompt}$

  (ii) Find all vertical asymptotes of f: $x=\begin{prompt}\answer{4}\end{prompt}$

  (iii) Find all horizontal asymptotes of f: $y=\begin{prompt}\answer{1}\end{prompt}$

  (iv) List the (largest) intervals of continuity of f: $\begin{prompt}\left(\answer{-\infty},\answer{-3}\right),\left(\answer{-3},\answer{4}\right),\left(\answer{4},\answer{\infty}\right)\end{prompt}$

\end{exercise}
\end{document}
