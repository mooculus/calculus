\documentclass{ximera}

\newcommand{\RR}{\mathbb R}
\renewcommand{\d}{\,d}
\newcommand{\dd}[2][]{\frac{d #1}{d #2}}
\renewcommand{\l}{\ell}
\newcommand{\ddx}{\frac{d}{dx}}
\newcommand{\dfn}{\textbf}
\newcommand{\eval}[1]{\bigg[ #1 \bigg]}


\outcome{}

\title[Break-Ground:]{Geometry and substitution}

\begin{document}
\begin{abstract}
Two students consider substitution geometrically.
\end{abstract}
\maketitle

\begin{dialogue}
\item[Devyn] Riley! We should be able to figure some integrals
  geometrically using transformations of functions.
\item[Riley] That sounds like a cool idea.  Maybe, since we know the
  graph of $f(x) = \sqrt{1-x^2}$ is a semicircle, we get an ellipse
  defined on $[-2,2]$ just by stretching the graph of $f$ by a factor
  of $2$ horizontally.  The equation of this ellipse would be
  \[
  g(x) =\sqrt{1-(\frac{x}{2})^2}
  \]
\item[Devyn] Exactly!  So since we know that
  \[
  \int_{-1}^1 \sqrt{1-x^2} \d x = \frac{\pi}{2}
  \]
  geometrically\dots
\item[Riley] And we know that the area under $g$ from $[-2,2]$ is
  twice the under $f$\dots
\item[Devyn and Riley] We must have
  \[
  \int_{-2}^2 \sqrt{1-(x/2)^2} \d x =\pi !
  \]
\item[Devyn and Riley] Jinx!
\item[Devyn] It is kind of like we just stretched out our whole
  coordinate system, and that helped us solve an integral.
\item[Riley] In this case, everything got stretched out by a constant
  factor of $2$ in the horizontal direction.  I wonder if we could
  ever say anything useful about cases where we stretch the $x$-axis
  by a different amount at each point?
\item[Devyn] Whao, that is a wild thought.  That seems really hard.
  Since derivatives measure how much a function stretches a little
  piece of the domain, maybe the derivative will come into play here?
\item[Riley] Hmmmm, but I do not see exactly how.  Maybe we should ask
  our TA about this?
\end{dialogue}

%%Q stretch ellipse just like above

\begin{question}
Say we know that $\int_1^4 f(x) dx = 5$.  Then, using this transformation idea, we can evaluate $\int_a^b f(3x+1)$ if $a= \answer{0}$ and $b=\answer{1}$.  The value of the integral on this interval is $\answer{5/3}$.
\end{question}

%% \begin{xarmaBoost}
%%   Write down at least \textbf{five} questions for this lecture. After
%%   you have your questions, label them as ``Level 1,'' ``Level 2,'' or
%%   ``Level 3'' where:
%% \begin{description}
%% \item[Level 1] Means you know the answer, or know exactly how to do
%%   this problem.
%% \item[Level 2] Means you think you know how to do the problem.
%% \item[Level 3] Means you have no idea how to do the problem.
%% \end{description}
%% \begin{freeResponse}
%% \end{freeResponse}
%% \end{xarmaBoost}



\end{document}
