\documentclass{ximera}

\newcommand{\RR}{\mathbb R}
\renewcommand{\d}{\,d}
\newcommand{\dd}[2][]{\frac{d #1}{d #2}}
\renewcommand{\l}{\ell}
\newcommand{\ddx}{\frac{d}{dx}}
\newcommand{\dfn}{\textbf}
\newcommand{\eval}[1]{\bigg[ #1 \bigg]}


%\outcome{Given a velocity function, calculate displacement and distance traveled.}
%\outcome{Given a velocity function, find the position function.}
%\outcome{Given an acceleration function, find the velocity function.}
%\outcome{Understand the difference between displacement and distance traveled.}
%\outcome{Understand the relationship between position, velocity and acceleration.}

\author{Nela Lakos \and Kyle Parsons}

\begin{document}
\begin{exercise}

The velocity of an object moving along a straight line is given by the function
\[
v(t) = 
\begin{cases}
t-2 & 0\leq t\leq4\\
2\cos\left(\frac{\pi t}{2}\right) & 4<t\leq8\\
\end{cases}.
\]
[$v$ is measured in m/s and $t$ in seconds.]


The velocity attains its maximum at $t=\answer{4}$ s  and $t=\answer{8}$ s  (answer from left to right).

The velocity is zero at $t=\answer{2}$ s, $t=\answer{5}$ s, and $t=\answer{7}$ s  (answer from left to right).

The total distance the object travels on the interval $[0,4]$ is $\answer{4}$ m.

The total distance the object travels on the interval $[0,8]$ is $\answer{4+\frac{16}{\pi}}$ m.

The displacement of the object on the interval $[0,8]$ is $\answer{0}$ m.

\end{exercise}
\end{document}