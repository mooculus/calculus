\documentclass{ximera}

\newcommand{\RR}{\mathbb R}
\renewcommand{\d}{\,d}
\newcommand{\dd}[2][]{\frac{d #1}{d #2}}
\renewcommand{\l}{\ell}
\newcommand{\ddx}{\frac{d}{dx}}
\newcommand{\dfn}{\textbf}
\newcommand{\eval}[1]{\bigg[ #1 \bigg]}


\author{Bart Snapp}

\outcome{Compute the flux across a surface using a surface integral.}

\begin{document}
\begin{exercise}
  Compute the flux of the field $\vec{F}= \vector{x,y,z}$ across the
  outwardly oriented surface bounded above by $\vec{P}(u,v) =
  \vector{u,v,\sqrt{9-u^2-v^2}}$ and below by the plane $z=0$.
  \[
  \text{Flux} = \answer{54\pi}
  \]
  \begin{hint}
    You will need to set-up two integrals, one for the top, and one
    for the bottom.
  \end{hint}
  \begin{hint}
    For the bottom, $\uvec{n} = \vector{0,0,-1}$. 
  \end{hint}
  \begin{hint}
    \[
    \vec{F}\dotp\uvec{n}\d S = -z \d S
    \]
  \end{hint}
  \begin{hint}
    However, since $z=0$, no flux is contributed.
  \end{hint}
  \begin{hint}
    For the top, you could immediately switch to spherical coordinates.
  \end{hint}
  \begin{hint}
    The top is now parameterized by
    \[
    \vec{P}(\theta,\phi)=\vector{3\cos(\theta)\sin(\phi),3\sin(\theta)\sin(\phi),3\cos(\phi)}
    \]
    for $0\le \theta<2\pi$ and $0\le \phi\le \pi/2$.
  \end{hint}
  \begin{hint}
    Computing, and dotting with $\vec{F}$ we find
    \[
    \vec{F}\dotp\left(\pp[\vec{P}]{\phi}\cross\pp[\vec{P}]{\theta}\right) = 27\sin(\phi)
    \]
  \end{hint}
  \begin{hint}
    Now,
    \[
    \iint_R \vec{F}\dotp\uvec{n}\d S = \int_0^{2\pi}\int_0^{\pi/2} 27\sin(\phi) \d\phi\d\theta
    \]
  \end{hint}
\end{exercise}
\end{document}
