\documentclass{ximera}

\newcommand{\RR}{\mathbb R}
\renewcommand{\d}{\,d}
\newcommand{\dd}[2][]{\frac{d #1}{d #2}}
\renewcommand{\l}{\ell}
\newcommand{\ddx}{\frac{d}{dx}}
\newcommand{\dfn}{\textbf}
\newcommand{\eval}[1]{\bigg[ #1 \bigg]}


\author{David Guichard \and Neal Koblitz \and H. Jerome Keisler \and Albert Scheller \and Barry Balof \and Mike Wills \and Matthew Carr\and Nela Lakos}
\license{CC-By-SA-NC}

\outcome{Recognize a function of several variables as a contour plot.}

\acknowledgement{https://www.whitman.edu/mathematics/multivariable/}

\begin{document}
\begin{exercise}
Let $F(x,y)=(x-y)^2$. Determine the equations and shapes of the cross-sections when $x=0$, $y=0$, $x=y$, and describe the level curves.

\begin{prompt}
\[
\text{When $x=0$, the cross-sections are given by } z=\answer{y^2}
\]
\end{prompt}
\begin{prompt}
\[
\text{When $y=0$, the cross-sections are given by } z=\answer{x^2}
\]
\end{prompt}
\begin{prompt}
\[
\text{When $x=y$, the cross-sections are given by } z=\answer{0}
\]
\end{prompt}

The level curves are best described as:
\begin{hint}
All the level curves will satisfy the equation $(x-y)^2=c$, for some $c\ge0$.
\end{hint}
\begin{hint}
Therefore, a level curve corresponding to an output $c$ is given by



$x-y= \sqrt{c}$ or  $x-y= -\sqrt{c}$,
which is equivalent to

$y=x- \sqrt{c}$ or  $y=x+\sqrt{c}$.

\end{hint}
\begin{hint}

 Four level curves, corresponding to output values $c=0$, $c=1$, $c=4$, and $c=9$, are shown in the picture below.
\begin{image}
    \begin{tikzpicture}
      \begin{axis}[
          xmin=-4,xmax=4,
            ymin=-4,ymax=4,
            domain=-2:2,
            width=2.5in,
           axis lines =center, xlabel=$x$,xtick={-3,-2,-1,1,2,3}, ylabel=$y$, ytick={-3,-2,-1,1,2,3},
            every axis y label/.style={at=(current axis.above origin),anchor=south},
            every axis x label/.style={at=(current axis.right of origin),anchor=west},
        ]
	\addplot [ thick, penColor, smooth, samples=100, domain=-4:4
	] {x};
       	\addplot [ thick, penColor2, smooth, samples=100, domain=-4:4] {x+1};
  	\addplot [ thick, penColor2, smooth, samples=100, domain=-4:4] {x-1};
	 	\addplot [ thick, penColor3, smooth, samples=100, domain=-4:4] {x+2};
  	\addplot [ thick, penColor3, smooth, samples=100, domain=-4:4] {x-2};
        \node at (axis cs:0.9,3.2) [penColor3,anchor=west] {$4$};  
         \node at (axis cs:3,1.3) [penColor3,anchor=west] {$4$};   
         \node at (axis cs:1.6,2.9) [penColor2,anchor=west] {$1$};    
          \node at (axis cs:2.6,1.9) [penColor2,anchor=west] {$1$};  
            \node at (axis cs:2.4,2.6) [penColor,anchor=west] {$0$};  
              \node at (axis cs:3.4,0.7) [penColor4,anchor=west] {$9$};  
            \node at (axis cs:0.3,3.57) [penColor4,anchor=west] {$9$};      
            	 	\addplot [ thick, penColor4, smooth, samples=100, domain=-4:4] {x+3};
  	\addplot [ thick, penColor4, smooth, samples=100, domain=-4:4] {x-3};
  
      \end{axis}
    \end{tikzpicture}
    \end{image}
\end{hint}
\begin{multipleChoice}
\choice{diamonds}
\choice{squares}
\choice{circles}
\choice[correct]{lines of slope $1$}
\choice{hyperbolas}
\end{multipleChoice}

\end{exercise}
\end{document}
