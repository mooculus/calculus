\documentclass{ximera}

\newcommand{\RR}{\mathbb R}
\renewcommand{\d}{\,d}
\newcommand{\dd}[2][]{\frac{d #1}{d #2}}
\renewcommand{\l}{\ell}
\newcommand{\ddx}{\frac{d}{dx}}
\newcommand{\dfn}{\textbf}
\newcommand{\eval}[1]{\bigg[ #1 \bigg]}


\outcome{Discuss what it means for a limit to equal $\infty$.}
\outcome{Understand the relationship between limits and vertical asymptotes.}
\outcome{Evaluate the limit as $x$ approaches a point where there is a vertical asymptote.}
\outcome{Recognize when a limit is indicating there is a vertical asymptote.}

\author{Nela Lakos \and Kyle Parsons}

\begin{document}
\begin{exercise}

Let $f$ be defined by
\[
f(x) = 
\begin{cases} 
\frac{x-2}{x^2-3x+2} & \text{if } x\neq 2 \text{ and } x\neq 1 \\
C & \text{if } x=2
   \end{cases}
\]
where $C$ is some constant.

Determine if the following limits exist and if they do, compute them analytically using the limit laws and techniques for computing limits.  Possible answers are any number, $\infty$, $-\infty$, and $DNE$.

\begin{exercise}
\[
\lim_{x\to1^-}f(x) = \answer{-\infty}
\]
\begin{exercise}
\[
\lim_{x\to2}f(x) = \answer{1}
\]
\end{exercise}
\end{exercise}

\begin{exercise}
In order for $f$ to be continuous at 2, we must have
\[
\lim_{x\to\answer{2}}f(x) = f\left(\answer{2}\right) = C
\]
And so we know that $C=\answer{1}$ if $f$ is continuous.
\end{exercise}

\begin{exercise}
By what we calculated above, the only vertical asymptote of $f$ is $x=\answer1$.
\end{exercise}

\begin{exercise}
\begin{align*}
\lim_{x\to-\infty}f(x) &= \answer{0}\\
\lim_{x\to\infty}f(x) &= \answer{0}
\end{align*}
So the only horizontal asymptote of $f$ is $y=\answer{0}$.
\end{exercise}

\end{exercise}
\end{document}