\documentclass{ximera}

\newcommand{\RR}{\mathbb R}
\renewcommand{\d}{\,d}
\newcommand{\dd}[2][]{\frac{d #1}{d #2}}
\renewcommand{\l}{\ell}
\newcommand{\ddx}{\frac{d}{dx}}
\newcommand{\dfn}{\textbf}
\newcommand{\eval}[1]{\bigg[ #1 \bigg]}


\title[Dig-In:]{Derivatives of inverse functions}

\begin{document}
\begin{abstract}
\end{abstract}
\maketitle


\section{The derivative of the natural logarithm}

Geometrically, there is a close relationship between the plots of
$e^x$ and $\ln(x)$, they are reflections of each other over the line
$y=x$:
\begin{image}
\begin{tikzpicture}
	\begin{axis}[
            xmin=-6,xmax=6,ymin=-6,ymax=6,
            axis lines=center,
            xlabel=$x$, ylabel=$y$,
            every axis y label/.style={at=(current axis.above origin),anchor=south},
            every axis x label/.style={at=(current axis.right of origin),anchor=west},
          ]        
          \addplot [very thick, penColor, smooth, domain=(-6:6)] {e^x};
          \addplot [very thick, penColor2, samples=100, smooth, domain=(.002:6)] {ln(x)};
          \addplot [dashed, textColor, domain=(-6:6)] {x};
          \node at (axis cs:-2,1) [penColor] {$e^x$};
          \node at (axis cs:1,-2) [penColor2] {$\ln(x)$};
        \end{axis}
\end{tikzpicture}
%% \caption{A plot of $e^x$ and $\ln(x)$. Since they are inverse
%%   functions, they are reflections of each other across the line $y=x$.}
%% \label{plot:e^x lnx}
\end{image}
One may suspect that we can use the fact that $\ddx e^x = e^x$, to
deduce the derivative of $\ln(x)$.  We will use implicit
differentiation to exploit this relationship computationally.

\begin{theorem}[The Derivative of the Natural Logrithm]\index{derivative!of the natural logarithm}
\[
\ddx \ln(x) = \frac{1}{x}.
\]
\begin{explanation}
Recall
\[
\ln(x) = y \qquad\Leftrightarrow\qquad e^y = x.
\]
Hence
\begin{align*}
e^y &= x\\
\ddx e^y &= \ddx x &\text{Differentiate both sides.}\\
e^y \dd[y]{x} &= 1 &\text{Implicit differentiation.}\\
\dd[y]{x} &= \frac{1}{e^y} = \frac{1}{x}.
\end{align*}
Since $y=\ln(x)$, $\ddx \ln(x) = \frac{1}{x}$.
\end{explanation}
\end{theorem}

From this fact, we can deduce another:

\begin{theorem}[The derivative of an exponential function]
  \[
  \ddx a^x = a^x\cdot \ln(a).
  \]
  \begin{explanation}
    Here we need to be slightly sneaky. Note
    \[
    a^x = e^{\ln(a^x)} = e^{x\ln(a)}.
    \]
    So we may write
    \begin{align*}
      \ddx a^x &= \ddx e^{x\ln(a)}\\
      &= e^{x\ln(a)} \cdot \ln(a)\\
      &= a^x\cdot \ln(a).
    \end{align*}
  \end{explanation}
\end{theorem}





\section{The derivatives of inverse trigonometric functions}

What is the derivative of the arcsine? Since this is an inverse
function, we can find its derivative by using implicit
differentiation and the Inverse Function Theorem, Theorem~\ref{theorem:IFT}.


\begin{theorem}[The Derivative of arcsin(\textit{y})]\index{derivative!of arcsine}
\[
\dd{y} \arcsin(y) = \frac{1}{\sqrt{1-y^2}}.
\]
\begin{explanation} 
To start, note that the Inverse Function Theorem,
Theorem~\ref{theorem:IFT} assures us that this derivative actually
exists.  Recall
\[
\arcsin(y) = \theta \qquad\Rightarrow\qquad \sin(\theta) = y.
\]
Hence
\begin{align*}
\sin(\theta) &= y\\
\dd{y} \sin(\theta) &= \dd{y} y \\
\cos(\theta) \frac{d\theta}{dy} &= 1 \\
\frac{d\theta}{dy} &= \frac{1}{\cos(\theta)}.
\end{align*}
At this point, we would like $\cos(\theta)$ written in terms of $y$. Since
\[
\cos^2(\theta)+\sin^2(\theta) =1
\]
and $\sin(\theta) = y$, we may write
\begin{align*}
\cos^2(\theta)+y^2 &=1\\
\cos^2(\theta) &=1-y^2\\
\cos(\theta) &= \pm \sqrt{1-y^2}.
\end{align*}
Since $\theta=\arcsin(y)$ we know that $-\pi/2\le \theta\le \pi/2$, and the cosine of
an angle in this interval is always positive. Thus
$\cos(\theta)=\sqrt{1-y^2}$ and 
\[
\dd{y} \arcsin(y) = \frac{1}{\sqrt{1-y^2}}.
\]
\end{explanation}
\end{theorem}


We can do something similar with arccosine. 

\begin{theorem}[The Derivative of arccos(\textit{y})]\index{derivative!of arccosine}
\[
\dd{y} \arccos(y) = \frac{-1}{\sqrt{1-y^2}}.
\]
\begin{explanation} 
To start, note that the Inverse Function Theorem,
Theorem~\ref{theorem:IFT} assures us that this derivative actually
exists.  Recall
\[
\arccos(y) = \theta \qquad\Rightarrow\qquad \cos(\theta) = y.
\]
Hence
\begin{align*}
\cos(\theta) &= y\\
\dd{y} \cos(\theta) &= \dd{y} y \\
-\sin(\theta) \frac{d\theta}{dy} &= 1 \\
\frac{d\theta}{dy} &= \frac{-1}{\sin(\theta)}.
\end{align*}
At this point, we would like $\sin(\theta)$ written in terms of $y$. Since
\[
\cos^2(\theta)+\sin^2(\theta) =1
\]
and $\cos(\theta) = y$, we may write
\begin{align*}
y^2+\sin^2(\theta) &=1\\
\sin^2(\theta) &=1-y^2\\
\sin(\theta) &= \pm \sqrt{1-y^2}.
\end{align*}
Since $\theta=\arccos(y)$ we know that $0\le \theta\le \pi$, and the sine of
an angle in this interval is always positive. Thus
$\sin(\theta)=\sqrt{1-y^2}$ and 
\[
\dd{y} \arccos(y) = \frac{-1}{\sqrt{1-y^2}}.
\]
\end{explanation}
\end{theorem}


Finally, let's look at arctangent.

\begin{theorem}[The Derivative of arctan(\textit{y})]\index{derivative!of arctangent}
\[
\dd{y} \arctan(y) = \frac{1}{1+y^2}.
\]
\begin{explanation} 
To start, note that the Inverse Function Theorem,
Theorem~\ref{theorem:IFT} assures us that this derivative actually
exists.  Recall
\[
\arctan(y) = \theta \qquad\Rightarrow\qquad \tan(\theta) = y.
\]
Hence
\begin{align*}
\tan(\theta) &= y\\
\dd{y} \tan(\theta) &= \dd{y} y \\
\sec^2(\theta) \frac{d\theta}{dy} &= 1 \\
\frac{d\theta}{dy} &= \frac{1}{\sec^2(\theta)}.
\end{align*}
At this point, we would like $\sec^2(\theta)$ written in terms of $y$. Recall
\[
\sec^2(\theta) = 1+\tan^2(\theta)
\]
and $\tan(\theta) = y$, we may write $\sec^2(\theta)=1+y^2$. Hence
\[
\dd{y} \arctan(y) = \frac{1}{1+y^2}.
\]
\end{explanation}
\end{theorem}


We leave it to you, the reader, to investigate the derivatives of
arcsecant, arccosecant, and arccotangent. However, as a gesture of
friendship, we now present you with a list of derivative formulas for
inverse trigonometric functions.

\begin{theorem}[The Derivatives of Inverse Trigonometric Functions] \hfil
\begin{itemize}
\item $\dd{y} \arcsin(y) = \frac{1}{\sqrt{1-y^2}}$.
\item $\dd{y} \arccos(y) = \frac{-1}{\sqrt{1-y^2}}$.
\item $\dd{y} \arctan(y) = \frac{1}{1+y^2}$.
\item $\dd{y} \arcsec(y) = \frac{1}{|y|\sqrt{y^2-1}}$ for $|y|>1$.
\item $\dd{y} \arccsc(y) = \frac{-1}{|y|\sqrt{y^2-1}}$ for $|y|>1$.
\item $\dd{y} \arccot(y) = \frac{-1}{1+y^2}$.
\end{itemize}
\end{theorem}








\end{document}
