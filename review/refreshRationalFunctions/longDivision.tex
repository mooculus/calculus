\documentclass{ximera}

\newcommand{\RR}{\mathbb R}
\renewcommand{\d}{\,d}
\newcommand{\dd}[2][]{\frac{d #1}{d #2}}
\renewcommand{\l}{\ell}
\newcommand{\ddx}{\frac{d}{dx}}
\newcommand{\dfn}{\textbf}
\newcommand{\eval}[1]{\bigg[ #1 \bigg]}



\author{Jim Talamo}
\license{Creative Commons 3.0 By-NC}
\outcome{}
%\newcommand\Mydiv[2]{%
%$\strut#1$\kern.25em\smash{\raise.3ex\hbox{$\big)$}}$\mkern-8mu
%        \overline{\enspace\strut#2}$}

\begin{document}

\begin{exercise}

The integration technique of partial fractions is a useful technique
when evaluating antiderivatives of rational expressions when the
degree of the numerator is less than the degree of the denominator.

In order to use this�� technique, it is important to be able to
perform polynomial long division when the degree of the numerator is
greater than the degree of the denominator.����

Here is a reminder of the procedure.

\begin{itemize}
\item[Step 1:] Make sure the polynomial is written in descending order.  Fill in a $0$ if there is a ``missing'' power of $x$.
\item[Step 2:] Divide the term with the highest power inside the division symbol by the term with the highest power outside the division symbol.
\item[Step 3:] Multiply the answer obtained in the previous step by the polynomial in front of the division symbol.
\item[Step 4:] Subtract and bring down the next term.
\item[Step 5:] Repeat Steps 2, 3, and 4 until there are no more terms to bring down.
\item[Step 6:] The term remaining after the last subtract step is the remainder and must be written as a fraction, whose denominator is the divisor, in the final answer.
\end{itemize}

For some worked examples, please use the link at the end of the  \link[problem.]{http://www.mesacc.edu/~scotz47781/mat120/notes/divide_poly/long_division/long_division.html}.  Once you feel comfortable, use polynomial long division to answer the following questions.

\begin{itemize}
\item $\frac{2x+3}{x+1} = \answer{2}+\frac{\answer{1}}{x+1}$
\item $\frac{2x^3-5x^2+5x+5}{2x+1} = \answer{x^2-3x+4}+\frac{\answer{1}}{2x+1}$.
\item $\frac{2x^4+x-8}{x^2+2}=\answer{2x^2-4}+\frac{\answer{x}}{x^2+2}$.
\end{itemize}

\end{exercise}
\end{document}
