\documentclass{ximera}

\newcommand{\RR}{\mathbb R}
\renewcommand{\d}{\,d}
\newcommand{\dd}[2][]{\frac{d #1}{d #2}}
\renewcommand{\l}{\ell}
\newcommand{\ddx}{\frac{d}{dx}}
\newcommand{\dfn}{\textbf}
\newcommand{\eval}[1]{\bigg[ #1 \bigg]}


\outcome{Application of Intermediate Value Theorem.}

\author{Nela Lakos }


\begin{document}
\begin{exercise}

Use the Intermediate Theorem to show that the equation
\[
x^2=\sqrt{x+1}
\]
has a root  (a solution) in the interval $(1,2)$.

Let's solve the problem in steps.

REMARK:
Notice that  the equation above is equivalent to the equation 
\[
x^2-\sqrt{x+1}=0
\]
STEP 1

 In order to use the Intermediate Value Theorem, we have to define a  suitable function.

Let
\[
f(x)=x^2-\sqrt{x+1}
\]
where

\[
\text{Domain of }  f=[\answer{1},\answer{2}].
\]
Then $f$ is continuous on its domain.


STEP 2

\begin{exercise}

Find the values.
\[
f(1)=\answer{1-\sqrt{2}},
f(2)=\answer{4-\sqrt{3}}.
\]

\end{exercise}
\begin{exercise}
Let
\[
L=\answer{0}
\]
Select all the following statements that are correct.

\begin{selectAll}
\choice[correct]{$f(1)< L$ }
\choice{$f(1)> L$ }
\choice{$f(2)< L$ }
\choice[correct]{$f(2)> L$ }
\end{selectAll}
\end{exercise}
\begin{exercise}
Select all the following statements that are correct.

\begin{selectAll}
\choice{Since $f(2)< L<f(1)$ , the IVT implies that there is a number $c$ in $\left(1,2\right)$ such that $f(c)=L$.}
\choice{Since  $f(1)<  L<f(2)$, the IVT implies that there is a number $c$ in $\left(f(1),f(2)\right)$ such that $f(c)=L$.}
\choice{Since $1< L< 2$ , the IVT implies that there is a number $c$ in $\left(1,2\right)$ such that $f(c)=L$.}
\choice[correct]{Since $f(1)< L< f(2)$, the IVT implies that there is a number $c$ in $\left(1,2\right)$ such that $f(c)=L$.}
\end{selectAll}

\end{exercise}
When we apply the Intermediate Value Theorem to $f$ on the interval $\left[1,2\right]$ with $L=0$, we are guaranteed the existence of at least one point $c$ such that.
\[
f(c)=\answer{0}
\]
The number $c$ is a solution, or a root, of the original equation.
\end{exercise}

\end{document}