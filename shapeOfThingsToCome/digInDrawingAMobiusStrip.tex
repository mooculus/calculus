\documentclass{ximera}

\newcommand{\RR}{\mathbb R}
\renewcommand{\d}{\,d}
\newcommand{\dd}[2][]{\frac{d #1}{d #2}}
\renewcommand{\l}{\ell}
\newcommand{\ddx}{\frac{d}{dx}}
\newcommand{\dfn}{\textbf}
\newcommand{\eval}[1]{\bigg[ #1 \bigg]}


\author{Bart Snapp}
\outcome{Understand the parametric formula for a Mobius strip.}
\title[Dig-In:]{Drawing a M\"obius strip}

\begin{document}
\begin{abstract}
  Learn how to draw a M\"obius strip.
\end{abstract}
\maketitle

Recall the parametric formula for a  \link[M\"{o}bius
  strip]{https://en.wikipedia.org/wiki/Mobius_strip}:
\[
\vec{M}(\theta,t) =
\begin{bmatrix}
  (1 + t \cos(\theta/2))\cos(\theta)\\
  (1 + t \cos(\theta/2))\sin(\theta)\\
  t \sin(\theta/2)
\end{bmatrix}
\]
for $0\le \theta< 2\pi$ and $-1\le t\le 1$.
\begin{onlineOnly}
  You can play with this parameterization below:
  \begin{center}
    \geogebra{KP2skSqe}{800}{600} %% https://ggbm.at/KP2skSqe
  \end{center}
\end{onlineOnly}

Let me show you how to draw a M\"obius strip yourself. Get out a sheet
of paper, and play along---it will be fun! Start by drawing two
parallel line segments:
\begin{image}
  \begin{tikzpicture}
    \draw[ultra thick, penColor] (2,-1)--(2,1);
    \draw[ultra thick, penColor] (-2,-1)--(-2,1);

    %% \draw[ultra thick, penColor] (-2,1) arc (229.05:257.82:9);
    %% \draw[ultra thick, penColor] (-2,-1) arc (282.18:294.78:9);
    %% \draw[ultra thick, penColor] (.13,-.25) arc (296.57:310.95:9);
    
    %% \draw[ultra thick, penColor] (2,1) arc (60.02:119.98:4);

    %% \draw[ultra thick, dashed, penColor] (2,-1) arc (36.26:77.86:2.48);
    %% \draw[ultra thick, dashed, penColor] (-.52,-.04) arc (102.14:143.74:2.48);

    %% \draw[ultra thick, penColor] (.52,-.04) arc (77.86:102.14:2.48);
  \end{tikzpicture}
\end{image}
Next, draw an ``x'' in the middle:
\begin{image}
  \begin{tikzpicture}
    \draw[ultra thick, penColor] (2,-1)--(2,1);
    \draw[ultra thick, penColor] (-2,-1)--(-2,1);

    \draw[ultra thick, penColor] (-2,1) arc (229.05:257.82:9);
    \draw[ultra thick, penColor] (-2,-1) arc (282.18:294.78:9);
    \draw[ultra thick, penColor] (.13,-.25) arc (296.57:310.95:9);
    
    %% \draw[ultra thick, penColor] (2,1) arc (60.02:119.98:4);

    %% \draw[ultra thick, dashed, penColor] (2,-1) arc (36.26:77.86:2.48);
    %% \draw[ultra thick, dashed, penColor] (-.52,-.04) arc (102.14:143.74:2.48);

    %% \draw[ultra thick, penColor] (.52,-.04) arc (77.86:102.14:2.48);
  \end{tikzpicture}
\end{image}
Finally, add the curves in the back:
\begin{image}
  \begin{tikzpicture}
    \draw[ultra thick, penColor] (2,-1)--(2,1);
    \draw[ultra thick, penColor] (-2,-1)--(-2,1);

    \draw[ultra thick, penColor] (-2,1) arc (229.05:257.82:9);
    \draw[ultra thick, penColor] (-2,-1) arc (282.18:294.78:9);
    \draw[ultra thick, penColor] (.13,-.25) arc (296.57:310.95:9);
    
    \draw[ultra thick, penColor] (2,1) arc (60.02:119.98:4);

    \draw[ultra thick, dashed, penColor] (2,-1) arc (36.26:77.86:2.48);
    \draw[ultra thick, dashed, penColor] (-.52,-.04) arc (102.14:143.74:2.48);

    \draw[ultra thick, penColor] (.52,-.04) arc (77.86:102.14:2.48);
  \end{tikzpicture}
\end{image}
Now you can impress your friends and enemies alike with your drawing
prowess.

\end{document}
