\documentclass{ximera}

\newcommand{\RR}{\mathbb R}
\renewcommand{\d}{\,d}
\newcommand{\dd}[2][]{\frac{d #1}{d #2}}
\renewcommand{\l}{\ell}
\newcommand{\ddx}{\frac{d}{dx}}
\newcommand{\dfn}{\textbf}
\newcommand{\eval}[1]{\bigg[ #1 \bigg]}


\outcome{Solve optimization problems by finding the appropriate extreme values.}

\title[Break-Ground:]{Optimization}

\begin{document}
\begin{abstract}
Two young mathematicians discuss optimization from an abstract point
of view.
\end{abstract}
\maketitle

Check out this dialogue between two calculus students:

\begin{dialogue}
\item[Devyn] Riley, what do you think is the maximum value of
  \[
  f(x) = \frac{10}{x^2-2.8x+3}?
  \]
\item[Riley] Where did that function come from?
\item[Devyn] It's just some, um, random function.
\item[Riley] Wait, does this have to do with coffee?
\item[Devyn] Um, uh, no?
\item[Riley] Well what interval are we on?
\item[Devyn] Let's say $[0,10]$, I mean there's no way I could possibly drink ten cups of coff\dots
\item[Riley] I knew this was about coffee.
\end{dialogue}

Here Devyn has made a function, that is supposed to record Devyn's
``well-being'' with respect to the number of cups of coffee consumed
in one day.

\begin{problem}
  Graph Devyn's function. Where do you estimate the maximum on the
  interval $[0,10]$ to be?
  \begin{prompt}
    The maximum is at $x=\answer[tolerance=.2]{1.4}$
  \end{prompt}
\end{problem}

\begin{problem}
  If you wanted to argue that this is the (global) maximum value on
  $[0,10]$ without plotting, what arguments could you use?
  \begin{freeResponse}
  \end{freeResponse}
 \end{problem}

%% \begin{xarmaBoost}
%%   Write down at least \textbf{five} questions for this lecture. After
%%   you have your questions, label them as ``Level 1,'' ``Level 2,'' or
%%   ``Level 3'' where:
%% \begin{description}
%% \item[Level 1] Means you know the answer, or know exactly how to do
%%   this problem.
%% \item[Level 2] Means you think you know how to do the problem.
%% \item[Level 3] Means you have no idea how to do the problem.
%% \end{description}
%% \begin{freeResponse}
%% \end{freeResponse}
%% \end{xarmaBoost}


\end{document}
