\documentclass{ximera}

\newcommand{\RR}{\mathbb R}
\renewcommand{\d}{\,d}
\newcommand{\dd}[2][]{\frac{d #1}{d #2}}
\renewcommand{\l}{\ell}
\newcommand{\ddx}{\frac{d}{dx}}
\newcommand{\dfn}{\textbf}
\newcommand{\eval}[1]{\bigg[ #1 \bigg]}


\outcome{Sketch a parametric curve.}
\outcome{Eliminate a parameters of a parametric equation.}
\outcome{Represent a graph with parametric equations.}


\title[Dig-In:]{Parametric equations}

\begin{document}
\begin{abstract}
  We discuss the basics of parametric curves.
\end{abstract}        
\maketitle

\section{The idea of parametric equations}

Think back to when you first learned how to graph a function. I'm
pretty sure you used a so-called ``T-chart,'' and if $y = x^2$, I bet it
looked something like this:
\[
\begin{array}{c|c}
  x & y = x^2\\\hline
  0 & 0 \\
  1 & 1\\
  -1 & 1\\
  2 & 4\\
  -2 & 4
\end{array}
\]
With a parametric plot, both $x$ and $y$ are now functions of a third
parameter, we'll call it $t$, often thought of as time:
\[
\begin{array}{c|c|c}
  t  & x = 2-3t & y = -1+4t\\\hline
  0  & 2  & -1 \\
  1  & -1 & 3 \\
  2  & -4 & 7 \\
  3  & -7 & 11\\
  4  & -10& 15
\end{array}
\]
If $x=t$, then there isn't much difference between a parametric plot
and a regular plot.  On the other hand, with parametric functions, we
can generate plots that fail the vertical line test! Check out this
graph of
\begin{align*}
x(t) &= \cos(t)\\
y(t) &= \sin(t)
\end{align*}
as $t$ runs from $0$ to $2\pi$:
\begin{image}
  \begin{tikzpicture}
    \begin{axis}[
        xmin=-1.2,xmax=1.2,ymin=-1.2,ymax=1.2,
        axis lines=center,
        xlabel=$x$, ylabel=$y$,
        unit vector ratio*=1 1 1,
        every axis y label/.style={at=(current axis.above origin),anchor=south},
        every axis x label/.style={at=(current axis.right of origin),anchor=west},
          ]        
      \addplot [very thick, penColor, smooth, domain=(0:360)] ({cos(x)},{sin(x)});
    \end{axis}
\end{tikzpicture}
\end{image}

\begin{question}
  Do the parametric equations
  \begin{align*}
    x(t) &= \cos(t)\\
    y(t) &= \sin(t)
  \end{align*}
  as $t$ runs from $0$ to $2\pi$ define a function of $t$?
  \begin{multipleChoice}
    \choice{No, because the graph does not pass the vertical line
      test.}
    \choice[correct]{Yes, it is a function of $t$, because for each input
      $t$, there is exactly one output value, an ordered pair.}
  \end{multipleChoice}
  \begin{feedback}
    For the graph of a circle, $y$ is \textbf{not} a function of
    $x$. However, it can be a function of $t$ that maps
    \begin{align*}
    \R &\to \R\times \R\\
    t &\mapsto (x,y)
    \end{align*}
    where the domain is $[0,2\pi)$ and the elements of the range
      consist of ordered pairs.
  \end{feedback}
\end{question}


\section{Famous parametric equations}

We'll discuss some basic parametric equations.


\subsection{Circles}

The \dfn{standard form for a circle} centered at a point $(a,b)$ with
radius $r$ is given by
\[
(x-a)^2 + (y-b)^2 = r^2.
\]
One problem with the standard form for a circle is that it is somewhat
difficult to find points on the circle. A parametric equation
representing a circle solves this problem.

\begin{example}
  Give a parametric equation representing the circle
  \[
  (x-1)^2 + (y-2)^2 = 3^2
  \]
  and explain why your answer is correct.
  \begin{explanation}
    This is the circle of radius $3$ centered at the point
    $(1,2)$. Here we set
    \begin{align*}
      x(t) &= \answer[given]{1} + \answer[given]{3}\cos(t)\\
      y(t) &= \answer[given]{2} + \answer[given]{3}\sin(t)
    \end{align*}
    as $t$ runs from $0$ to $2\pi$.  To see that our answer is
    correct, we can ``plug'' it back into the implicit equation
    for the circle. Write with me:
    \begin{align*}
      (x-1)^2 + (y-2)^2 &= (x(t)-1)^2 + (y(t)-2)^2\\
      &= (1 + 3\cos(t)-1)^2 + (2 + 3\sin(t)-2)^2\\
      &= (3\cos(t))^2 + (3\sin(t))^2\\
      &= 3^2\cos^2(t) + 3^2\sin^2(t)\\
      &= 3^2(\cos^2(t) + \sin^2(t))\\
      &= 3^2
    \end{align*}
    by the Pythagorean identity.\index{Pythagorean identity} Since our
    functions $(x(t),y(t))$ satisfy the form of the circle, our
    solution is correct.
  \end{explanation}
\end{example}


\begin{question}
  One way to think about parametric formulas for circles is to
  imagine
  \begin{align*}
    x(t) &= \cos(t)\\
    y(t) &= \sin(t) 
  \end{align*}
  as ``drawing'' the unit circle as $t$ changes. Make a table showing
  how the circle is being plotted as $t$ runs from $0$ to $2\pi$:
  \begin{prompt}
    \[
    \begin{array}{c|c|c}
      t      & x(t) = \cos(t) & y(t) = \sin(t)\\ \hline
      0      & \answer{1}              & \answer{0}\\
      \pi/4  & \answer{\sqrt{2}/2}     & \answer{\sqrt{2}/2}\\
      \pi/2  & \answer{0}              & \answer{1}\\
      3\pi/4 & \answer{-\sqrt{2}/2}    & \answer{\sqrt{2}/2}\\
      \pi    & \answer{-1}             & \answer{0}\\
      5\pi/4 & \answer{-\sqrt{2}/2}    & \answer{-\sqrt{2}/2}\\
      3\pi/2 & \answer{0}              & \answer{-1}\\
      7\pi/4 & \answer{\sqrt{2}/2}     & \answer{-\sqrt{2}/2}\\
      2\pi   & \answer{1}              & \answer{0}
    \end{array}
    \]
  \end{prompt}
  \begin{question}
    Is the circle ``drawn'' in a clockwise or counterclockwise fashion?
    \begin{prompt}
      \begin{multipleChoice}
        \choice{clockwise}
        \choice[correct]{counterclockwise}
      \end{multipleChoice}
    \end{prompt}
    \begin{question}
      One day while trying to graph a unit circle, you accidentally
      write down
      \begin{align*}
        x(t) &= \sin(t)\\
        y(t) &= \cos(t) 
      \end{align*}
      What happens now? Do you still get a circle? How is this different
      from what we did in the previous question?
      \begin{prompt}
        \begin{multipleChoice}
        \choice{you still plot a unit circle in a counterclockwise fashion, with the same starting and ending points}
        \choice{you plot a unit circle but in a clockwise fashion, with the same starting and ending points}
        \choice{you still plot a unit circle in a counterclockwise fashion, but the starting and ending points are different}
        \choice[correct]{you plot a unit circle but in a clockwise fashion, but the starting and ending points are different}
        \choice{this no longer plots a circle}
        
        \end{multipleChoice}
      \end{prompt}
  \end{question}
\end{question}
\end{question}

In mathematics, when parameterizing closed curves (like circles), the
convention is to draw them in a ``counterclockwise'' direction. This
is called the \dfn{positive orientation}.

\begin{warning}
  If you parameterize your closed curves in a clockwise direction, you
  may find your ``answers'' are off by a factor of $-1$.
\end{warning}

\begin{question}
What should $x(t)$ and $y(t)$ be to parameterize the circle
\[
(x-a)^2 + (y-b)^2 = r^2.
\]
in a counterclockwise fashion, with $t=0$ corresponding to $(1,0)$?
\begin{prompt}
  \begin{align*}
    x(t) &= \answer{a} + \answer{r}\cdot \cos(t)\\
    y(t) &= \answer{b} + \answer{r}\cdot \sin(t)
  \end{align*}
\end{prompt}
\end{question}



\subsection{Lines}

Suppose you want a parametric equation for a line that goes through
the point $(2,1)$ with a certain direction:
\begin{image}
  \begin{tikzpicture}
	\begin{axis}[
            domain=(-1:2),
            clip=false,
            axis lines=center,
            %ticks=none,
            unit vector ratio*=1 1 1,
            xlabel=$x$, ylabel=$y$,
            ytick={-2,-1,...,7},
	    %yticklabels={$0.5$,$1$,$1.5$,$2$},
	    xtick={-2,-1,...,10},
	    %xticklabels={$0.5$,$1$,$1.5$,$2$},
	    grid = major,
            every axis y label/.style={at=(current axis.above origin),anchor=south},
            every axis x label/.style={at=(current axis.right of origin),anchor=west},
          ]
          \addplot[very thick,penColor2,->,>=stealth'] plot coordinates {(2,1) (6,4)};
          \addplot[color=penColor,fill=penColor,only marks,mark=*] coordinates{(2,1)};  %% closed hole
          \addplot[color=penColor,dashed] ({2+4*x},{1+3*x});
        \end{axis}
\end{tikzpicture}
\end{image}
where the $\mathbf{direction}$ is imagined as starting at $(0,0)$ and going
to the point $(u,v)$:
\begin{image}
  \begin{tikzpicture}
	\begin{axis}[
            domain=(-1:2),
            clip=false,
            axis lines=center,
            %ticks=none,
            unit vector ratio*=1 1 1,
            xlabel=$x$, ylabel=$y$,
            ytick={-2,-1,...,7},
	    %yticklabels={$0.5$,$1$,$1.5$,$2$},
	    xtick={-2,-1,...,10},
	    %xticklabels={$0.5$,$1$,$1.5$,$2$},
	    grid = major,
            every axis y label/.style={at=(current axis.above origin),anchor=south},
            every axis x label/.style={at=(current axis.right of origin),anchor=west},
          ]
          \addplot[very thick,penColor2!50!white,->,>=stealth'] plot coordinates {(2,1) (6,4)};
          \addplot[very thick,penColor2,->,>=stealth'] plot coordinates {(0,0) (4,3)};
          \addplot[color=penColor,fill=penColor,only marks,mark=*] coordinates{(2,1)};  %% closed hole
          \addplot[color=penColor,dashed] ({2+4*x},{1+3*x});
        \end{axis}
\end{tikzpicture}
\end{image}
So to understand the direction of the arrow above, we need to move it
back to the origin.

\begin{question}
  In the graph above, what is the direction of the arrow?
  \begin{prompt}
    The direction is given by $(u,v) = \left(\answer{4},\answer{3}\right)$.
  \end{prompt}
\end{question}

At this point we can give a very useful representation for a line:
\begin{align*}
  \l(t) &= \mathbf{point} + t\cdot \mathbf{direction}\\
  &= (a,b) + t \cdot(u,v)
\end{align*}
where the $\mathbf{direction}$ is imagined as starting at $(0,0)$ and going
to the point $(u,v)$. Another way of writing this is
\[
\l(t) = \left\{\begin{aligned}
  x(t) &= a + t\cdot u\\
  y(t) &= b + t\cdot v
\end{aligned}\right.
\]
\begin{question}
  Suppose you have the parametric formula for a line
  \[
  \l(t) = (2,1) + t \cdot(4,3)
  \]
  What is $\l(0)$?
  \begin{prompt}
    \[
    \l(0) = \left(\answer{2},\answer{1}\right)
    \]
  \end{prompt}
  \begin{feedback}
    We're at the starting point!
  \end{feedback}
  \begin{question}
     What is $\l(1)$?
  \begin{prompt}
    \[
    \l(1) = \left(\answer{6},\answer{4}\right)
    \]
  \end{prompt}
  \begin{feedback}
    We've moved exactly the ``direction'' from the  point!
  \end{feedback}
  \end{question}
\end{question}


\subsection{Other equations and other plots}

One thing that can be confusing about parametric plots is that there
can be multiple representations of the same plot:

\begin{question}
  Which of the following parametric equations draw the circle $(x-1)^2
  + (y-2)^2 = 3^2$?
  \begin{selectAll}
    \choice[correct]{$x(t) = 1 + 3\cos(t)$ and $y(t) = 2 + 3\sin(t)$ for $0 \le t \le 2\pi$}
    \choice[correct]{$x(t) = 1 + 3\cos(t)$ and $y(t) = 2 + 3\sin(t)$ for $2\pi \le t \le 4\pi$}
    \choice[correct]{$x(t) = 1 + 3\sin(t)$ and $y(t) = 2 + 3\cos(t)$ for $0 \le t \le 2\pi$}
    \choice{$x(t) = 2 + 3\sin(t)$ and $y(t) = 1 + 3\cos(t)$ for $0 \le t \le 2\pi$}
    \choice{$x(t) = 1 + 3\cos(e^t)$ and $y(t) = 2 + 3\sin(e^t)$ for $0 \le t \le \ln(2\pi)$}
    \choice[correct]{$x(t) = 1 + 3\cos(e^t)$ and $y(t) = 2 + 3\sin(e^t)$ for $\ln(2\pi) \le t \le \ln(4\pi)$}
  \end{selectAll}
\end{question}

\begin{question}
  Which of the following parametric equations draw the line $y=x$?
  \begin{selectAll}
    \choice[correct]{$x(t) = t$ and $y(t) = t$ for $-\infty < t < \infty$}
    \choice[correct]{$x(t) = t-5$ and $y(t) = t-5$ for $-\infty < t < \infty$}
    \choice{$x(t) = \sin(t)$ and $y(t) = \sin(t)$ for $-\infty < t < \infty$}
    \choice[correct]{$x(t) = \tan(t)$ and $y(t) = \tan(t)$ for $-\pi/2 < t < \pi/2$}
    \choice{$x(t) = t^2$ and $y(t) = t^2$ for $-\infty < t < \infty$}
    \choice[correct]{$x(t) = t^3$ and $y(t) = t^3$ for $-\infty < t < \infty$}
  \end{selectAll}
\end{question}


Parametric plots allow us to make some pretty crazy plots.



\begin{question}
Can you give a parametric formula for this cool spiral that starts at
the origin and runs to $(8\pi,0)$?
\begin{image}
\begin{tikzpicture}
	\begin{axis}[
            %xmin=-25,xmax=25,ymin=-25,ymax=25,
            width=3in,
            clip=false,
            axis lines=center,
            %ticks=none,
            unit vector ratio*=1 1 1,
            xlabel=$x$, ylabel=$y$,
            every axis y label/.style={at=(current axis.above origin),anchor=south},
            every axis x label/.style={at=(current axis.right of origin),anchor=west},
          ]      
          \addplot [very thick, penColor, smooth,samples=100,domain=(0:8*pi)] ({x*cos(deg(x))},{x*sin(deg(x))});
        \end{axis}
\end{tikzpicture}
\end{image}
\begin{prompt}
  \begin{align*}
    x(t) = \answer{t}\cdot \cos(t)\\
    y(t) = \answer{t}\cdot \sin(t)
  \end{align*}
\end{prompt}
\end{question}

One important class of parametric curves are \dfn{Lissajous figures}. These are curves of the form
\begin{align*}
  x(t) &= A \sin(at + \delta)\\
  y(t) &= B\sin(bt)
\end{align*}
Here is a plot of a Lissajous curve where $A=B=1$, $\delta = \pi/2$,
$a=5$ and $b=4$
\begin{image}
\begin{tikzpicture}
	\begin{axis}[
            %xmin=-25,xmax=25,ymin=-25,ymax=25,
            width=3in,
            clip=false,
            axis lines=center,
            %ticks=none,
            unit vector ratio*=1 1 1,
            xlabel=$x$, ylabel=$y$,
            every axis y label/.style={at=(current axis.above origin),anchor=south},
            every axis x label/.style={at=(current axis.right of origin),anchor=west},
          ]      
          \addplot [very thick, penColor, smooth,samples=100,domain=(0:2*pi)] ({cos(deg(5*x))},{sin(deg(4*x))});
        \end{axis}
\end{tikzpicture}
\end{image}
These figures come up a lot in electrical engineering. Do yourself a
favor and play around with Lissajous figures for differing values of
$\delta$, $a$ and $b$.





\section{Converting to parametric equations}

If you are given
\[
y = f(x)
\]
it is really easy to convert this to a parametric function, just write
\begin{align*}
  x(t) &= t\\
  y(t) &= f(t).
\end{align*}

\begin{question}
  Can you use the technique described immediately above to express $y
  = e^x$ as a parametric function?
  \begin{prompt}
    \begin{align*}
      x(t) &= \answer{t}\\
      y(t) &= \answer{e^t}
    \end{align*}
  \end{prompt}
\end{question}



\section{Converting from parametric equations}

On the other hand, if you are given a parametric function, to express
$y$ as function of $x$ can be much more difficult. Here are the basic
strategies to try:
\begin{itemize}
\item Solve for $t$.
\item Solve for a function of $t$.
\item Use a trigonometric identity.
\end{itemize}
In each case the process that we are using is called
\dfn{elimination of a parameter}.

We'll give several examples of how one actually \textit{eliminates a
  parameter}.

\subsection{Solving for the variable}

In the first example, we'll solve for $t$.
\begin{example}
  Let
  \begin{align*}
    x(t) &= -4 t^2\\
    y(t) &= 3t-1.
  \end{align*}
    Eliminate a parameter to express this curve purely in terms of $x$ and $y$.
  \begin{explanation}
    Here we will solve for $t$. Since it is easier, we will solve for
    $t$ in this equation:
    \begin{align*}
      y &= 3t-1\\
      y+1 &= \answer[given]{3t}\\
      \frac{y+1}{3} & = \answer[given]{t}.
    \end{align*}
    Now plug this into $x(t) = -4 t^2$, and write
    \[
    x = \answer[given]{-4 \left(\frac{y+1}{3}\right)^2}.
    \]
  \end{explanation}
\end{example}

\subsection{Solving for a common function}

In our next example we'll solve for a function of $t$ that is common
to both $x(t)$ and $y(t)$.

\begin{example}
  Let
  \begin{align*}
    x(t) &= -5e^t\\
    y(t) &= e^{3t}.
  \end{align*}
    Eliminate a parameter to express this curve purely in terms of $x$
    and $y$.
    \begin{explanation}
      Here we will solve for a function of $t$. Write
      \begin{align*}
        x &= -5e^t\\
        \frac{x}{-5} &= e^t
      \end{align*}
      Now we can rewrite $y(t)$ as
      \begin{align*}
        y(t) &= e^{3t}\\
        y(t) &= \left(e^{t}\right)^{\answer[given]{3}}.
      \end{align*}
      Now we see that
      \[
      y = \answer[given]{\left(\frac{x}{-5}\right)^3}.
      \]
  \end{explanation}
\end{example}


\subsection{Solving for related functions}

In our final example, we will use a trigonometric identity.

\begin{example}
  Let
  \begin{align*}
    x(t) &= 3\cos(t)\\
    y(t) &= 1+4\sin(t).
  \end{align*}
  Eliminate a parameter to express this curve purely in terms of $x$ and $y$.
  \begin{explanation}
    The basic idea is to try an use the Pythagorean identity:
    \[
    \cos^2(t) + \sin^2(t) = 1
    \]
    So first isolate cosine and sine, and square the equations. With $x(t)$ we have
    \begin{align*}
      x &= 3 \cos(t) \\
      \answer[given]{\frac{x}{3}} &= \cos(t)\\
      \answer[given]{\left(\frac{x}{3}\right)^2} &= \cos^2(t)
    \end{align*}
    and with $y(t)$ we have
    \begin{align*}
      y &= 1 + 4\sin(t)\\
      y-1 &= 4\sin(t)\\
      \frac{y-1}{4} &= \sin(t)\\
      \left(\frac{y-1}{4}\right)^2 &= \sin^2(t).
    \end{align*}
    Plugging this back into the Pythagorean identity, we see:
    \begin{align*}
      \cos^2(t) + \sin^2(t) &= 1\\
      \left(\frac{x}{3}\right)^2 + \left(\frac{y-1}{4}\right)^2 &= 1.
    \end{align*}
  \end{explanation}
\end{example}


\end{document}
