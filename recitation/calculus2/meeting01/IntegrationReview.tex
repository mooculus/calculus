\documentclass[handout,hints]{ximera}
%handout:  for handout version with no solutions or instructor notes
%handout,instructornotes:  for instructor version with just problems and notes, no solutions
%noinstructornotes:  shows only problem and solutions

%% handout
%% space
%% newpage
%% numbers
%% nooutcomes

%I added the commands here so that I would't have to keep looking them up
%\newcommand{\RR}{\mathbb R}
%\renewcommand{\d}{\,d}
%\newcommand{\dd}[2][]{\frac{d #1}{d #2}}
%\renewcommand{\l}{\ell}
%\newcommand{\ddx}{\frac{d}{dx}}
%\everymath{\displaystyle}
%\newcommand{\dfn}{\textbf}
%\newcommand{\eval}[1]{\bigg[ #1 \bigg]}

%\begin{image}
%\includegraphics[trim= 170 420 250 180]{Figure1.pdf}
%\end{image}

%add a ``.'' below when used in a specific directory.

\newcommand{\RR}{\mathbb R}
\renewcommand{\d}{\,d}
\newcommand{\dd}[2][]{\frac{d #1}{d #2}}
\renewcommand{\l}{\ell}
\newcommand{\ddx}{\frac{d}{dx}}
\newcommand{\dfn}{\textbf}
\newcommand{\eval}[1]{\bigg[ #1 \bigg]}




\author{Tom Needham and Jim Talamo}

\outcome{Recall basic rules and techniques of integration.}

\title[]{A Review of Integration}

\begin{document}
\begin{abstract}
\end{abstract}
\maketitle

\vspace{-0.9in}

\section{Discussion Questions}

\begin{problem}
Classify each of the following expressions as a number, a function of $t$, a function of $x$ or none of the above.


\begin{tabular}{llll}
I. $\frac{\d}{\d t} \int_0^t e^{x^2} \d x$ \hspace{0.2in} & II. $\int e^{x^2} \d x$ \hspace{0.2in} & III. $\int_0^2 e^{x^2} \d x$ \hspace{0.2in}
IV. $\frac{d}{dx} \int_0^9 e^{x^2} \d x$
\end{tabular}

\end{problem}

\begin{problem}
Being able to compute antiderivatives of functions quickly is an important skill to develop.  Compute the following indefinite integrals by modifying the basic antiderivative rules.  Try not to use a $u$-substitution.

\begin{center}
\begin{tabular}{lll} 
I. $\int \cos(2x) \d x$ \hspace{.7in} & II. $\int e^{x/2} \d x$ \hspace{.7in} & III. $\int (2x+1)^3 \d x$
\end{tabular}
\end{center}

\end{problem}

\begin{problem}
Consider the indefinite integral $\int \frac{1}{3x} \d x$. Student A calculates the integral using basic antiderivative rules as
$$
\int \frac{1}{3x} \d x = \frac{1}{3} \ln |3x| + C.
$$
Student B calculates the integral by moving the constant to the front as 
$$
\int \frac{1}{3x} \d x = \frac{1}{3} \int \frac{1}{x} \d x = \frac{1}{3} \ln |x| + C.
$$
Is either student incorrect?  If so, explain the faulty logic in the incorrect student's solution.  If both students are correct, how can their answers be rectified?

\end{problem}

\section{Group Work}

\begin{problem}
Consider the integral $\int_0^4  6x\sqrt{x^2+9} \d x$.  There are common approaches to evaluating this.

\begin{itemize}
\item[I.] Use a substitution to show that:

\[\int  6x\sqrt{x^2+9} \d x = 2(x^2+9)^{3/2} +C.\]  
Then, use the Fundamental Theorem on Calculus to evaluate $\int_0^4  6x\sqrt{x^2+9} \d x.$ 

\item[II.] Make the substitution $u=x^2+9$ and show that:

\[\int_{x=0}^{x=4}  6x\sqrt{x^2+9} \d x = \int_{u=9}^{u=25} 3\sqrt{u}.\]  Then, evaluate the integral in $u$ by using the Fundamental Theorem of Calculus.
\end{itemize}


\end{problem}

\begin{problem}
Calculate the following integrals.

\begin{tabular}{lll}
I.  $\int \left(\sqrt{y}+1\right)^2 \d y$ \hspace{.5in} & II. $\int_0^{\sqrt{\ln(2)}} x e^{x^2} \d x$ \hspace{.5in} & III. $\int \frac{2x+1}{x^2+4} \d x$ \hspace{.05in}
\end{tabular}

\end{problem}



\begin{problem}
Find a function $f(x)$ satisfying each of the given condition(s).

$$
\begin{array}{lll}
\mathrm{I.} \; \int f(x) \d x = 2x \sqrt{1+ x^2} + C & \hspace{.2in} & \mathrm{II.} \; \int xf(x) \d x = x e^x + C \\
\mathrm{III.} \; \frac{\d}{\d t} \int_0^t f(x) \d x = t e^t & \hspace{.2in} & \mathrm{IV.} \; f'(x) = 3x^2 \mbox{ and } f(0)=2
\end{array}
$$
\end{problem}



\begin{problem}
A student calculates an indefinite integral as follows:
$$
\int \frac{5}{x^2 + 9} \d x = 5\ln (x^2 + 9) + C.
$$
Is the student correct? First explain your answer without actually calculating the antiderivative. Then, if the student is incorrect, calculate the correct antiderivative.
\end{problem}

\begin{problem}
The figure below shows a plot of a function $y = f(x)$. On the axes below, draw representations of 


\begin{tabular}{ll}
I. $\int f(x) \d x$ \hspace{0.4in} & II. $\int_{-2}^2 f(x) \d x$\\
\resizebox {6cm} {!} { 
          \begin{tikzpicture}
	    \begin{axis}[
            domain=-3:3,
            xmin=-3, xmax=3,
            ymin=-2, ymax=6
         ,
            axis lines =middle, xlabel=$x$, ylabel=$y$,
            every axis y label/.style={at=(current axis.above origin),anchor=south},
            every axis x label/.style={at=(current axis.right of origin),anchor=west},
          ]
	  \addplot [very thick, penColor, smooth] {x^2};
        
        \end{axis}
\end{tikzpicture}
%% \caption{A plot of $f(x)=x^2$ and $f^{-1}(x) = \sqrt{x}$. While
%%   $f(x)=x^2$ is not one-to-one on $\RR$, it is one-to-one on
%%   $[0,\infty)$.}
%% \label{plot:fxn and inverse x^2}
}
  &
\resizebox {6cm} {!} { 
          \begin{tikzpicture}
	    \begin{axis}[
            domain=-3:3,
            xmin=-3, xmax=3,
            ymin=-2, ymax=6
         ,
            axis lines =middle, xlabel=$x$, ylabel=$y$,
            every axis y label/.style={at=(current axis.above origin),anchor=south},
            every axis x label/.style={at=(current axis.right of origin),anchor=west},
          ]
	  \addplot [very thick, penColor, smooth] {x^2};
        
        \end{axis}
\end{tikzpicture}
%% \caption{A plot of $f(x)=x^2$ and $f^{-1}(x) = \sqrt{x}$. While
%%   $f(x)=x^2$ is not one-to-one on $\RR$, it is one-to-one on
%%   $[0,\infty)$.}
%% \label{plot:fxn and inverse x^2}
}

\end{tabular}

\end{problem}

\begin{problem}
(Multiselect)

Circle all of the statements below that are true.  Make sure you understand the reasoning behind each response.

\begin{itemize}
\item[I.] $\displaystyle \int \dfrac{2}{3+4x} \, dx = 2\ln|3+4x|+C$

\item[II.] $\displaystyle \int 4xe^{2x} \, dx = 2xe^{2x}-e^{2x}+C$

\item[III.] The Fundamental Theorem of Calculus can be applied directly to find $ \int_0^2 \frac{2x}{x^2-1} \d x$.

\item[IV.] To evaluate $ \int_0^1 2x(x^2+1)^2 \d x$, set $u=x^2+1$.  Then. $\d u = 2x \d x$ so:
\[ \int_0^1 2x(x^2+1)^2 \d x = \int_0^1 u^2 \d u = \eval{\frac{1}{3} u^3}_0^1 = \frac{1}{3}\]
\end{itemize}

\end{problem}
\end{document}
\documentclass[handout,hints]{ximera}

\newcommand{\RR}{\mathbb R}
\renewcommand{\d}{\,d}
\newcommand{\dd}[2][]{\frac{d #1}{d #2}}
\renewcommand{\l}{\ell}
\newcommand{\ddx}{\frac{d}{dx}}
\newcommand{\dfn}{\textbf}
\newcommand{\eval}[1]{\bigg[ #1 \bigg]}


\usepackage{paralist}

\author{Tom Needham}

\outcome{Recall basic rules and techniques of integration.}

\title[]{Review of Integration Techniques}

\begin{document}
\begin{abstract}
\end{abstract}
\maketitle

\vspace{-0.9in}

\section{Discussion Questions}

\begin{problem}
Classify each of the following expressions as a number, a function of $t$, a function of $x$ or none of the above.

\begin{center}
\begin{inparaenum}
\item $\frac{\d}{\d t} \int_0^t e^{x^2} \d x$ \hspace{0.2in}
\item $\int e^{x^2} \d x$ \hspace{0.2in}
\item $\int_0^2 e^{x^2} \d x$ \hspace{0.2in}
\item $\frac{d}{dx} \int_0^9 e^{x^2} \d x$
\end{inparaenum}
\end{center}
\end{problem}

\begin{problem}
Compute the following indefinite integrals via basic antiderivative rules, without explicitly using $u$-substitution.

\begin{center}
\begin{inparaenum}
\item $\int \cos(2x) \d x$ \hspace{0.4in}
\item $\int e^{x/2} \d x$
\end{inparaenum}
\end{center}

\end{problem}

\begin{problem}
Consider the indefinite integral $\int \frac{1}{3x} \d x$. Student A calculates the integral using basic antiderivative rules as
$$
\int \frac{1}{3x} \d x = \frac{1}{3} \ln |3x| + C.
$$
Student B calculates the integral by moving the constant to the front as 
$$
\int \frac{1}{3x} \d x = \frac{1}{3} \int \frac{1}{x} \d x = \frac{1}{3} \ln |x| + C.
$$
Which student is correct?
\end{problem}


\section{Group Work}


\begin{problem}
Calculate the following integrals.


\begin{inparaenum}
\item $\int \left(\sqrt{y}+1\right)^2 \d y$ \hspace{.05in}
\item $\int_0^{\sqrt{\ln(2)}} x e^{x^2} \d x$ \hspace{.05in}
\item $\int \frac{2x+1}{x^2+4} \d x$ \hspace{.05in}
\item $\int_{-\pi/8}^{\pi/8} \frac{\cos^2(2t) + 1}{\cos^2(2t)} \d t$ 
\end{inparaenum}

\end{problem}



\begin{problem}
Find a function $f(x)$ satisfying each of the given condition(s).

$$
\begin{array}{lll}
\mathrm{(a)} \; \int f(x) \d x = 2x \sqrt{1+ x^2} + C & \hspace{.2in} & \mathrm{(b)} \; \int f(x) \d x = x e^x + C \\
\mathrm{(c)} \; \frac{\d}{\d t} \int_0^t f(x) \d x = t e^t & \hspace{.2in} & \mathrm{(d)} \; f'(x) = 3x^2 \mbox{ and } f(0)=2
\end{array}
$$
\end{problem}



\begin{problem}
A student calculates an indefinite integral as follows:
$$
\int \frac{1}{3x^2 + 5} \d x = \ln (3x^2 + 5) + C.
$$
Is the student correct? First explain your answer without actually calculating the antiderivative. Then, if the student is incorrect, calculate the correct antiderivative.
\end{problem}

\begin{problem}
The figure below shows a plot of a function $y = f(x)$. On the same axes, draw representations of 

\begin{center}
\begin{inparaenum}
\item $\int f(x) \d x$ \hspace{0.4in}
\item $\int_{-2}^2 f(x) \d x$
\end{inparaenum}
\end{center}

\begin{image}
          \begin{tikzpicture}
	    \begin{axis}[
            domain=-3:3,
            xmin=-3, xmax=3,
            ymin=-2, ymax=6
         ,
            axis lines =middle, xlabel=$x$, ylabel=$y$,
            every axis y label/.style={at=(current axis.above origin),anchor=south},
            every axis x label/.style={at=(current axis.right of origin),anchor=west},
          ]
	  \addplot [very thick, penColor, smooth] {x^2};
        
        \end{axis}
\end{tikzpicture}
%% \caption{A plot of $f(x)=x^2$ and $f^{-1}(x) = \sqrt{x}$. While
%%   $f(x)=x^2$ is not one-to-one on $\RR$, it is one-to-one on
%%   $[0,\infty)$.}
%% \label{plot:fxn and inverse x^2}
\end{image}
\end{problem}


\end{document}
