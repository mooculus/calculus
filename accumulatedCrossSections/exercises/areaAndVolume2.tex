\documentclass{ximera}

\newcommand{\RR}{\mathbb R}
\renewcommand{\d}{\,d}
\newcommand{\dd}[2][]{\frac{d #1}{d #2}}
\renewcommand{\l}{\ell}
\newcommand{\ddx}{\frac{d}{dx}}
\newcommand{\dfn}{\textbf}
\newcommand{\eval}[1]{\bigg[ #1 \bigg]}


\author{Jim Talamo}
\license{Creative Commons 3.0 By-NC}


\outcome{Set up an integral that gives the area of a region}
\outcome{Set up an integral that gives the volume of a solid whose cross sections are the region}

\begin{document}


The region $R$ is bounded by $y=3-x^2$ and $y=x+1$. 





\begin{exercise}
By integrating with respect to $y$, how many integrals are needed to express the area of $R$? $\answer{2}$

\begin{exercise}
The area of the region can be found by evaluating:

\[
	A = \int_1^{\answer{2}} \answer{y-1+\sqrt{3-y}} \d y  +\int_{\answer{2}}^{\answer{3}} \answer{2\sqrt{3-y}} \d y 
\]
	
\end{exercise}
\end{exercise}


\begin{exercise}
The base of a certain solid is the region $R$.  Cross sections through the solid taken parallel to the $y$-axis are semicircles. 


To express the volume of a solid using a definite integral, we should:
\begin{multipleChoice}
\choice[correct]{integrate with respect to $x$.}
\choice{integrate with respect to $y$.}
\end{multipleChoice}

An integral that expresses the volume of the solid is:

\[
	V= \int_{x=\answer{-2}}^{x=\answer{1}}\answer{\frac{\pi}{8}(2-x-x^2)^2} \d x
\]
		
		



\end{exercise}
\end{document}
