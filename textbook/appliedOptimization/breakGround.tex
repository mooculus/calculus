\documentclass{ximera}

\newcommand{\RR}{\mathbb R}
\renewcommand{\d}{\,d}
\newcommand{\dd}[2][]{\frac{d #1}{d #2}}
\renewcommand{\l}{\ell}
\newcommand{\ddx}{\frac{d}{dx}}
\newcommand{\dfn}{\textbf}
\newcommand{\eval}[1]{\bigg[ #1 \bigg]}


\outcome{Interpret an optimization problem as the procedure used to
  make a system or design as effective or functional as possible.}

\outcome{Set up an optimization problem by identifying the objective
  function and appropriate constraints.}

\title[Break-Ground:]{Volumes of aluminium cans}

\begin{document}
\begin{abstract}
Two young mathematicians discuss optimizing aluminum cans.
\end{abstract}
\maketitle

Check out this dialogue between two calculus students (based on a true
story):

\begin{dialogue}
\item[Devyn] Riley, have you ever noticed aluminum cans?
\item[Riley] So very recyclable! 
\item[Devyn] I know! But I've also noticed that there are some that
  are short and fat, and others that are tall and skinny, and yet they
  have the same volume!
\item[Riley] So very observant! 
\item[Devyn] This got me wondering, if we want to make a can with
  volume $V$, what shape of can uses the least aluminum?
\item[Riley] Ah! OK so the volume of a cylindrical can is given by
  \[
  V = \pi \cdot r^2 \cdot h
  \]
  where $r$ is the radius of the cylinder and $h$ is the height of the
  cylinder. Also the surface area is given by
  \begin{align*}
    A &= \underbrace{\pi \cdot r^2}_{\text{bottom}} + \underbrace{2\cdot\pi \cdot r\cdot h}_{\text{sides}} + \underbrace{\pi \cdot r^2}_{\text{top}}\\
    &= 2\cdot \pi \cdot r^2 + 2\cdot\pi \cdot r\cdot h.    
  \end{align*}
  So somehow we want to minimize the surface area, because that's the
  amount of aluminum used, but keep the volume constant.
\item[Devyn] Whoa, we have way too many letters here.
\item[Riley] Yeah, somehow we need only one variable. Yikes. Too many letters.
\end{dialogue}

\begin{problem}
  Suppose we wish to construct an aluminum can with volume $V$ that
  uses the least amount of aluminum. In the context above, what do we
  want to minimize?
  \begin{multipleChoice}
    \choice[correct]{$A$}
    \choice{$V$}
    \choice{$h$}
    \choice{$r$}
  \end{multipleChoice}
\end{problem}

\begin{problem}
  In the context above, what should be considered a constant?
  \begin{multipleChoice}
    \choice{$A$}
    \choice[correct]{$V$}
    \choice{$h$}
    \choice{$r$}
  \end{multipleChoice}
\end{problem}

When doing this problems, we need to reduce the problem to a single
variable.

\begin{problem}
  If we set $h$ to be the variable, express $A$ as a function of $h$.
  \begin{hint}
    We know that $V$ is a constant and that
    \[
    V = \pi \cdot r^2 \cdot h
    \]
    so $h=\answer{V/(\pi r^2)}$.
    Now substitute this expression for $h$ in the equation
    \[
  A = 2\cdot \pi \cdot r^2 + 2\cdot\pi \cdot r \cdot h
    \]
  \end{hint}
  \begin{prompt}
    \[
    A = \answer{2\cdot \pi \cdot r^2 + 2\cdot\pi \cdot r\cdot V/(\pi r^2)}
    \]
  \end{prompt}
\end{problem}

\begin{problem}
  If we set $r$ to be the variable, express $A$ as a function of $r$.
  \begin{hint}
    We know that $V$ is a constant and that
    \[
    V = \pi \cdot r^2 \cdot h
    \]
    so $r=\answer{\sqrt{V/(\pi h)}}$.
    Now substitute this expression for $h$ in the equation
    \[
    A = 2\cdot \pi \cdot r^2 + 2\cdot\pi \cdot r \cdot h
    \]
  \end{hint}
  \begin{prompt}
    \[
    A = \answer{2\cdot \pi \cdot V/(\pi h) + 2\cdot\pi \cdot \sqrt{V/(\pi h)} \cdot h}
    \]
  \end{prompt}
\end{problem}

The key point here is that we've reduced (one way or another) this
function of two variables to one variable. This will be a key step for
nearly every problem in this next section.

%% \begin{xarmaBoost}
%%   Write down at least \textbf{five} questions for this lecture. After
%%   you have your questions, label them as ``Level 1,'' ``Level 2,'' or
%%   ``Level 3'' where:
%% \begin{description}
%% \item[Level 1] Means you know the answer, or know exactly how to do
%%   this problem.
%% \item[Level 2] Means you think you know how to do the problem.
%% \item[Level 3] Means you have no idea how to do the problem.
%% \end{description}
%% \begin{freeResponse}
%% \end{freeResponse}
%% \end{xarmaBoost}


\end{document}
