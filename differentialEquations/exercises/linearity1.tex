\documentclass{ximera}
\newcommand{\RR}{\mathbb R}
\renewcommand{\d}{\,d}
\newcommand{\dd}[2][]{\frac{d #1}{d #2}}
\renewcommand{\l}{\ell}
\newcommand{\ddx}{\frac{d}{dx}}
\newcommand{\dfn}{\textbf}
\newcommand{\eval}[1]{\bigg[ #1 \bigg]}

\author{Jim Talamo}
\license{Creative Commons 3.0 By-NC}
\outcome{Verify solutions to differential equations}
\begin{document}


\begin{exercise}
Select all of the following differential equations below that are \emph{linear}

\begin{selectAll}
\choice[correct]{$\frac{dy}{dx}+xy=3$}
\choice{$y\frac{dy}{dx}+y=3x$}
\choice[correct]{$\frac{dy}{dx}+2x=e^x$}
\choice{$\frac{dy}{dx}+2x=e^y$}
\choice[correct]{$x^2\frac{d^2y}{dx^2}+2xy=\arctan(2x^2)$}
\choice{$\frac{dy}{dx}+\sqrt{y}=3x$}
\end{selectAll}

\begin{hint}
Recall that an $n$-th order differential equation written in the form $f(x,y(x),y'(x), \ldots y^{(n)}(x))=0$ is \emph{linear} if $f$ is linear in $y$ and its derivatives. 

For first order equations, this means an equation is linear if it can be written in the form:

\[
\frac{dy}{dx}+p(x)y(x)=q(x)
\]


For second order equations, this means an equation is linear if it can be written in the form:

\[
\frac{d^2y}{dx^2}+p(x)\frac{dy}{dx}+q(x)y(x)=r(x)
\]

Note that the functions of $x$ that serve as the ``coefficients'' of $y$ and its derivatives can be anything!
\end{hint}
\end{exercise}

\end{document}