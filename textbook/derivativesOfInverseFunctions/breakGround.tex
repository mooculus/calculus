\documentclass{ximera}

\newcommand{\RR}{\mathbb R}
\renewcommand{\d}{\,d}
\newcommand{\dd}[2][]{\frac{d #1}{d #2}}
\renewcommand{\l}{\ell}
\newcommand{\ddx}{\frac{d}{dx}}
\newcommand{\dfn}{\textbf}
\newcommand{\eval}[1]{\bigg[ #1 \bigg]}


\outcome{}

\title[Break-Ground:]{Derivatives of inverse functions}

\begin{document}
\begin{abstract}
Two young mathematicians discuss the derivative of inverse functions.
\end{abstract}
\maketitle

Check out this dialogue between two calculus students:

\begin{dialogue}
\item[Devyn] Riley, I have a calculus question.
\item[Riley] Hit me with it.
\item[Devyn] What's the derivative of $\arctan(x)$?
\item[Riley] Hmmm\dots we haven't done it yet in our class.
\item[Devyn] I know! But maybe we can figure it out.
\item[Riley] Well
  \[
  \arctan(x) = \tan^{-1}(x)
  \]
  and now we can use the chain rule to take its derivative
  \begin{align*}
    \ddx \tan^{-1}(x) &= -\tan^{-2}(x)\sec^2(x)\\
    &= -\frac{\cos^2x}{\sin^2x}\cdot \frac{1}{\cos^2x}\\
    &= \frac{-1}{\sin^2x}.
  \end{align*}
\item[Devyn] But is this right?
\end{dialogue}

% This gets at what the notation sin^{-1} x means, and what the inverse function theorem is saying

% Contrast it with a geometric interpretation that since the inverse is the fxn flipped over the line y=x, we should have this info once we know f'

% becaue we want to actually have two answers that both seem reasonable and demand that they student resolve the contradiction in mathematics

%% \begin{xarmaBoost}
%%   Write down at least \textbf{five} questions for this lecture. After
%%   you have your questions, label them as ``Level 1,'' ``Level 2,'' or
%%   ``Level 3'' where:
%% \begin{description}
%% \item[Level 1] Means you know the answer, or know exactly how to do
%%   this problem.
%% \item[Level 2] Means you think you know how to do the problem.
%% \item[Level 3] Means you have no idea how to do the problem.
%% \end{description}
%% \begin{freeResponse}
%% \end{freeResponse}
%% \end{xarmaBoost}


\end{document}
