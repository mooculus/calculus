\documentclass{ximera}

\newcommand{\RR}{\mathbb R}
\renewcommand{\d}{\,d}
\newcommand{\dd}[2][]{\frac{d #1}{d #2}}
\renewcommand{\l}{\ell}
\newcommand{\ddx}{\frac{d}{dx}}
\newcommand{\dfn}{\textbf}
\newcommand{\eval}[1]{\bigg[ #1 \bigg]}


\author{Jim Talamo}

\outcome{Add and subtract vectors.}

\begin{document}
\begin{exercise}
Suppose that $\vec{u} = \vector{-1,2,2}$ and $\vec{v} = \vector{-2,0,1}$.  Find a vector $\vec{w}$ of magnitude $4$ that is parallel to $2\vec{u}-3\vec{v}$.

\[
\vec{w} = \vector{\answer{\frac{16}{\sqrt{33}} }, \answer{ \frac{16}{\sqrt{33}} } , \answer{ \frac{-4}{\sqrt{33}} }}
\]

\begin{hint}
To begin, let's find a vector in the same direction as $2\vec{u}-3\vec{v}$.  Using the rules of addition and scalar multiplication, we find:

\[
2\vec{u}-3\vec{v} = \vector{\answer{4} , \answer{4} , \answer{-1}}
\]

How should we proceed?

\begin{multipleChoice}
\choice{Multiply this result by $4$; that is, $\vec{w} = \vector{16 , 16, -4}$.}
\choice[correct]{Find the magnitude of $\vector{4, 4, -1}$ and scale it appropriately if necessary.}
\end{multipleChoice}

We compute:

\[
\left|2\vec{u}-3\vec{v}\right| = \sqrt{\left( \answer{4} \right)^2+\left(  \answer{4} \right)^2+\left(  \answer{-1}\right)^2 }  = \sqrt{\answer{33}}
\]
(type the components in the order of $2\vec{u}-3\vec{v}$)

A unit vector in the direction of $\vec{w}$ is thus $\frac{2\vec{u}-3\vec{v}}{\left|2\vec{u}-3\vec{v}\right|}$, so:

\[
\uvec{w} = \vector{\answer{\frac{4}{\sqrt{33}} }, \answer{ \frac{4}{\sqrt{33}} } , \answer{ \frac{-1}{\sqrt{33}} }}
\]

and  $\vec{w} = \answer{4} \uvec{w}$.
\end{hint}

\end{exercise}
\end{document}
