\documentclass{ximera}

%\newcommand{\RR}{\mathbb R}
\renewcommand{\d}{\,d}
\newcommand{\dd}[2][]{\frac{d #1}{d #2}}
\renewcommand{\l}{\ell}
\newcommand{\ddx}{\frac{d}{dx}}
\newcommand{\dfn}{\textbf}
\newcommand{\eval}[1]{\bigg[ #1 \bigg]}

\newcommand{\RR}{\mathbb R}
\renewcommand{\d}{\,d}
\newcommand{\dd}[2][]{\frac{d #1}{d #2}}
\renewcommand{\l}{\ell}
\newcommand{\ddx}{\frac{d}{dx}}
\newcommand{\dfn}{\textbf}
\newcommand{\eval}[1]{\bigg[ #1 \bigg]}


\author{}
\license{Creative Commons 3.0 By-NC}
\outcome{}
\begin{document}

\begin{exercise}

In this exercise, we evaluate several definite integrals using basic antiderivative rules, the change of variables technique  and the Fundamental Theorem of Calculus. 

\begin{problem} Calculate:
\[
\int_0^{1/2} \frac{1}{\sqrt{1-x^2}} \d x =\answer{\frac{\pi }{6}}
\]
\end{problem}


\begin{problem} Calculate:
\[
\int_0^1 \frac{1}{x^2 + 1} \d x = \answer{\frac{\pi }{4}}
\]
\end{problem}



\begin{problem} Calculate:
\[
\int_0^1 \frac{1}{x + 1} \d x = \answer{\ln(2)}
\]
\end{problem}


\begin{problem} Calculate:
\[
\int_0^{1/2} \frac{4}{\sqrt{1 - x^2}} \d x = \answer{\frac{2\pi }{3}}
\]
\end{problem}


\begin{problem} Calculate:
\[
\int_0^{1/2} \frac{-1}{\sqrt{1 - x^2}} \d x = \answer{\frac{-\pi }{6}}
\]
\end{problem}


\begin{problem} Calculate:
\[
\int_2^3 \sec^2 \theta  \d \theta = \answer{\sin(1) \sec(2) \sec(3)}
\]
\end{problem}



\begin{problem} Calculate:
\[
\int_2^3 \sec(\theta)\cdot \tan(\theta)  \d \theta = \answer{\sec(3)-\sec(2)}
\]
\end{problem}



\begin{problem} Calculate:
\[
\int_2^3 3\sec^2\theta  \d \theta = \answer{3 \sin (1) \sec (2) \sec (3)}
\]
\end{problem}


\begin{problem} Calculate:
\[
\int_0^1 e^{-x/2} \d x = \answer{2-\frac{2}{\sqrt{e}}}
\]
\end{problem}



\begin{problem} Calculate:
\[
\int_0^1 2e^{4x} \d x = \answer{\frac{1}{2} (e^4-1)}
\]
\end{problem}




\begin{problem} Calculate:
\[
\int_0^3 \cos(3x) \d x = \answer{\frac{\sin (9)}{3}}
\]
\end{problem}


\begin{problem} Calculate:
\[
\int_0^{1/2} \cos(\pi x) \d x = \answer{\frac{1}{\pi }}
\]
\end{problem}




\begin{problem} Calculate:
\[
\int_1^2 3\sin(3x) \d x = \answer{\cos (3)-\cos (6)}
\]
\end{problem}



\begin{problem} Calculate:
\[
\int_1^4 \pi \sin(- x) \d x = \answer{\pi  (\cos (4)-\cos (1))}
\]
\end{problem}



\begin{problem} Calculate:
\[
\int_3^5 \frac{\sin(\pi x)}{\pi} \d x = \answer{0}
\]
\end{problem}



\begin{problem} Calculate:
\[
\int_1^2 \cos(7 x) \d x= \answer{\frac{1}{7} (\sin (14)-\sin (7))}
\]
\end{problem}



\begin{problem} Calculate:
\[
\int_3^4 \frac{\cos(2 x)}{3} \d x=\answer{\frac{1}{6} (\sin (8)-\sin (6))}
\]
\end{problem}


\begin{problem} Calculate:
\[
\int_0^{1/2} \pi \cos(x/\pi) \d x = \answer{\pi ^2 \sin (\frac{1}{2 \pi })}
\]
\end{problem}


\begin{problem} Calculate:
\[
\int_0^1 \frac{\sin(x/\pi)}{\pi} \d x = \answer{1-\cos \left(\frac{1}{\pi }\right)}
\]
\end{problem}



\begin{problem} Calculate:
\[
\int_0^2 \pi\sin(x/\pi) \d x =\answer{2 \pi ^2 \sin ^2\left(\frac{1}{\pi }\right)}
\]
\end{problem}



\begin{problem} Calculate:
\[
\int_0^1 \frac{3}{(2x+1)} \d x = \answer{\frac{3 \ln (3)}{2}}
\]
\end{problem}


\begin{problem} Calculate:
\[
\int_0^1 \frac{1}{5-3x} \d x \answer{\frac{\ln(5/2)}{3}}
\]
\end{problem}

\begin{problem} Calculate:
\[
\int_1^2 \frac{x^3}{1 + x^4} \d x = \answer{\frac{1}{4} \ln(\frac{17}{2})}
\]
\end{problem}



\begin{problem} Calculate:
\[
\int_0^1 \frac{2+ e^{2x}}{4x + e^{2x}} \d x = \answer{\frac{1}{2} \ln(4+e^2)}
\]
\end{problem}



\begin{problem} Calculate:
\[
\int_0^1 \frac{1}{1 + x} \d x =\answer{\ln(2)}
\]
\end{problem}


\begin{problem} Calculate:
\[
\int_3^6 \frac{x^2+4x^3}{x^3+3x^4} \d x =\answer{\frac{1}{3} \ln(\frac{76}{5})}
\]
\end{problem}



\begin{problem} Calculate:
\[
\int_2^5 \frac{1}{-3-3x} \d x =\answer{\frac{-\ln(2)}{3}}
\]
\end{problem}

\end{exercise}
\end{document}