\documentclass{ximera}

\newcommand{\RR}{\mathbb R}
\renewcommand{\d}{\,d}
\newcommand{\dd}[2][]{\frac{d #1}{d #2}}
\renewcommand{\l}{\ell}
\newcommand{\ddx}{\frac{d}{dx}}
\newcommand{\dfn}{\textbf}
\newcommand{\eval}[1]{\bigg[ #1 \bigg]}


\author{Jim Talamo \and Bart Snapp}

\outcome{Compute the length of a parametric curve.}

\begin{document}
\begin{exercise}
  A curve $C$ in space is described by the vector-valued function:
  \[
  \vec{p}(t) = \vector{t^2-1,2t,2t^2+2}
  \]
  Find a unit vector in the first octant that is orthogonal to both
  $\vec{p}(0)$ and $\vec{p}'(0)$:
  \[
  \uvec{u} =\vector{\answer{2/\sqrt{5}},\answer{0},\answer{1/\sqrt{5}}}
  \]
  \begin{exercise}
    Find the point $(x,y,z)$ where $\vec{p}(t)$ is orthogonal to $\vec{p}'(t)$:
    \[
    (x,y,z) = \left(\answer{-1},\answer{0},\answer{2}\right)
    \]
    \begin{exercise}
      Does $\vec{p}$ use arc length of as a parameter?
      \begin{multipleChoice}
        \choice{yes}
        \choice[correct]{no}
      \end{multipleChoice}
      \begin{feedback}
        Just check to see if $|\vec{p}'(t)| = 1$.
      \end{feedback}
      \begin{exercise}
        Find all points where the tangent line to the curve at $t=0$
        intersects the surface $2x+y-z^2=0$.
        \[
        (x,y,z) = \left(\answer{8},\answer{6},\answer{20}\right)
        \]
        \end{exercise}
    \end{exercise}
  \end{exercise}
\end{exercise}
\end{document}
