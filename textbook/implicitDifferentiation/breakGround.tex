\documentclass{ximera}

\newcommand{\RR}{\mathbb R}
\renewcommand{\d}{\,d}
\newcommand{\dd}[2][]{\frac{d #1}{d #2}}
\renewcommand{\l}{\ell}
\newcommand{\ddx}{\frac{d}{dx}}
\newcommand{\dfn}{\textbf}
\newcommand{\eval}[1]{\bigg[ #1 \bigg]}


\outcome{}

\title[Break-Ground:]{Different plots}


\begin{document}
\begin{abstract}
  Two young mathematicians discuss differential equations.
\end{abstract}
\maketitle

% Thinking about unit circle

% Locally, even nonfunctions look like functions.

% $x^2 = y^2$ has two intersecting tangent lines

% Race cars on figure eight track, which spectators die?

% How do these differ from differentiating parametric equations?

% When does implicit differentiation work?

% Robot arm interactive

Check out this dialogue between two calculus students (based on a true
story):

\begin{dialogue}
\item[Devyn] Riley, I think we've been too explicit with each
  other. We should try to be more implicit.
\item[Riley] I. Um. Don't really\dots
\item[Devyn] Oh! I mean when plotting things!
\item[Riley] Ok, but I still have no idea what you are talking about.
\item[Devyn] Remember when we first learned the equation of a line, and the ``standard form'' was
  \[
  ax+by = c
  \]
  or something, which is totally useless for graphing. Also a circle is
  \[
  x^2 + y^2 = r^2
  \]
  or something, and this isn't even $y$ as a function of
  $x$. Nevertheless, sometimes we plot these things.
\item[Riley] Ah, so if you have
  \[
  y = mx + b \rightleftaarow \text{where $y$ is \textbf{explicitly} a function of $x$}
  \]
  or you have
    \[
  ax + by = c  \rightleftaarow \text{here $y$ is \textbf{implicitly} a function of $x$.}
  \]
\end{dialogue}

%% \begin{xarmaBoost}
%%   Write down at least \textbf{five} questions for this lecture. After
%%   you have your questions, label them as ``Level 1,'' ``Level 2,'' or
%%   ``Level 3'' where:
%% \begin{description}
%% \item[Level 1] Means you know the answer, or know exactly how to do
%%   this problem.
%% \item[Level 2] Means you think you know how to do the problem.
%% \item[Level 3] Means you have no idea how to do the problem.
%% \end{description}
%% \begin{freeResponse}
%% \end{freeResponse}
%% \end{xarmaBoost}



\end{document}
