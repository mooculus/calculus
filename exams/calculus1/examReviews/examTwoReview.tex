\documentclass{ximera}
\newcommand{\RR}{\mathbb R}
\renewcommand{\d}{\,d}
\newcommand{\dd}[2][]{\frac{d #1}{d #2}}
\renewcommand{\l}{\ell}
\newcommand{\ddx}{\frac{d}{dx}}
\newcommand{\dfn}{\textbf}
\newcommand{\eval}[1]{\bigg[ #1 \bigg]}

\author{Tom Dinitz and Nela Lakos}
\license{Creative Commons 3.0 By-NC}
\title{Midterm 2 Review}

\begin{document}
\begin{abstract}
Review questions for MIDTERM 2.
\end{abstract}
\maketitle
%Exercise 1

\begin{exercise}
(a) Suppose $s(t)$ is the position (in feet) of an object moving along a line at time $t\geq 0$ seconds. 

What is the average velocity between times $t=a$ and $t=b$?
\begin{prompt}
\begin{multipleChoice}
\choice{$s(b)-s(a)$ $\frac{\text{ft}}{\text{second}}$}
\choice[correct]{$\frac{s(b)-s(a)}{b-a}$ $\frac{\text{ft}}{\text{second}}$}
\choice{$\frac{b-a}{s(b)-s(a)}$ $\frac{\text{ft}}{\text{second}}$}
\choice{$b-a$ $\frac{\text{ft}}{\text{second}}$}
\end{multipleChoice}
\end{prompt}

(b) The table gives $s(t)$ over a three second interval.

\[
\begin{array}{|c|c|c|c|c|c|c|c|}
\hline
t& 0& 0.5& 1& 1.5& 2& 2.5& 3\\ \hline
s(t)& 0 & 22 & 32 & 48 & 54 & 64 & 74 \\ \hline
\end{array}
\]

The average velocity over the interval $[1,3]$ is $v_{av}=\answer{21}$ $\frac{\text{ft}}{\text{second}}$

(c) Select all reasonable approximations for the instantaneous velocity at time $t=1$ second:
\begin{selectAll}
\choice[correct]{32 $\frac{\text{ft}}{\text{second}}$}
\choice{10 $\frac{\text{ft}}{\text{second}}$}
\choice[correct]{26 $\frac{\text{ft}}{\text{second}}$}
\choice{-5 $\frac{\text{ft}}{\text{second}}$}
\choice{48 $\frac{\text{ft}}{\text{second}}$}
\end{selectAll}
\end{exercise}

%Exercise 2
\begin{exercise}
(a) Fill in the blanks: $f'(x)=\lim_{h\to\answer{0}}\frac{f(\answer{x+h})-f(\answer{x})}{\answer{h}}$

(b) Let $f(x)=\frac{1}{x-4}$. Using the equation in part (a), compute:

$f'(x)=\lim_{h\to\answer{0}}\frac{\frac{1}{\answer{(x+h)-4}}-\frac{1}{\answer{x-4}}}{h}=\answer{-(x-4)^{-2}}$

\end{exercise}

%Exercise 3
\begin{exercise}
The graph of a function $f$ is given below

\begin{image}
\begin{tikzpicture}
    \begin{axis}[
            xmin=0, xmax=4.5, ymin=0,ymax=4.5,
            unit vector ratio*=1 1 1,
            axis lines =middle, xlabel=$x$, ylabel=$y$,
            every axis y label/.style={at=(current axis.above origin),anchor=south},
            every axis x label/.style={at=(current axis.right of origin),anchor=west},
            xtick={0,...,4}, ytick={0,...,4},
            grid=major,width=4in,
            grid style={dashed, gridColor},
          ]
	  \addplot[very thick, samples=200, color=penColor, smooth, domain=(0:4.5)] {2*(x)^(1/2)};
\end{axis}
\end{tikzpicture}
\end{image}

(1) (a) Select the figure which has a secant line drawn through the points $(0,f(0))$ and $(4,f(4))$.

\begin{multipleChoice}
\choice{
\begin{image}
\begin{tikzpicture}
    \begin{axis}[
            xmin=0, xmax=4.5, ymin=0,ymax=4.5,
            unit vector ratio*=1 1 1,
            axis lines =middle, xlabel=$x$, ylabel=$y$,
            every axis y label/.style={at=(current axis.above origin),anchor=south},
            every axis x label/.style={at=(current axis.right of origin),anchor=west},
            xtick={0,...,4}, ytick={0,...,4},
            grid=major,width=4in,
            grid style={dashed, gridColor},
          ]
        \addplot[very thick, samples=200, color=penColor, smooth, domain=(0:4.5)] {2*(x)^(1/2)};
        \addplot[very thick, color=red, smooth, domain=(0:3.5)] {x+1};
   \end{axis}
\end{tikzpicture}
\end{image}
}

\choice[correct]{
\begin{image}
\begin{tikzpicture}
    \begin{axis}[
            xmin=0, xmax=4.5, ymin=0,ymax=4.5,
            unit vector ratio*=1 1 1,
            axis lines =middle, xlabel=$x$, ylabel=$y$,
            every axis y label/.style={at=(current axis.above origin),anchor=south},
            every axis x label/.style={at=(current axis.right of origin),anchor=west},
            xtick={0,...,4}, ytick={0,...,4},
            grid=major,width=4in,
            grid style={dashed, gridColor},
          ]
        \addplot[very thick, samples=200, color=penColor, smooth, domain=(0:4.5)] {2*(x)^(1/2)};
	\addplot[very thick, color=penColor, color=red] plot coordinates {(0,0) (4.5,4.5)};
   \end{axis}
\end{tikzpicture}
\end{image}
}

\choice{
\begin{image}
\begin{tikzpicture}
    \begin{axis}[
            xmin=0, xmax=4.5, ymin=0,ymax=4.5,
            unit vector ratio*=1 1 1,
            axis lines =middle, xlabel=$x$, ylabel=$y$,
            every axis y label/.style={at=(current axis.above origin),anchor=south},
            every axis x label/.style={at=(current axis.right of origin),anchor=west},
            xtick={0,...,4}, ytick={0,...,4},
            grid=major,width=4in,
            grid style={dashed, gridColor},
          ]
        \addplot[very thick, samples=200, color=penColor, smooth, domain=(0:4.5)] {2*(x)^(1/2)
};
        \addplot[very thick, color=red, smooth, domain=(0:4.5)] {.5*x+2};
   \end{axis}
\end{tikzpicture}
\end{image}
}
\end{multipleChoice}

(b) Select the figure which has the line tangent to the curve at $x=1$

\begin{multipleChoice}
\choice[correct]{
\begin{image}
\begin{tikzpicture}
    \begin{axis}[
            xmin=0, xmax=4.5, ymin=0,ymax=4.5,
            unit vector ratio*=1 1 1,
            axis lines =middle, xlabel=$x$, ylabel=$y$,
            every axis y label/.style={at=(current axis.above origin),anchor=south},
            every axis x label/.style={at=(current axis.right of origin),anchor=west},
            xtick={0,...,4}, ytick={0,...,4},
            grid=major,width=4in,
            grid style={dashed, gridColor},
          ]
        \addplot[very thick, samples=200, color=penColor, smooth, domain=(0:4.5)] {2*(x)^(1/2)};
        \addplot[very thick, color=red, smooth, domain=(0:3.5)] {x+1};
   \end{axis}
\end{tikzpicture}
\end{image}
}

\choice{
\begin{image}
\begin{tikzpicture}
    \begin{axis}[
            xmin=0, xmax=4.5, ymin=0,ymax=4.5,
            unit vector ratio*=1 1 1,
            axis lines =middle, xlabel=$x$, ylabel=$y$,
            every axis y label/.style={at=(current axis.above origin),anchor=south},
            every axis x label/.style={at=(current axis.right of origin),anchor=west},
            xtick={0,...,4}, ytick={0,...,4},
            grid=major,width=4in,
            grid style={dashed, gridColor},
          ]
        \addplot[very thick, samples=200, color=penColor, smooth, domain=(0:4.5)] {2*(x)^(1/2)};
	\addplot[very thick, color=penColor, color=red] plot coordinates {(0,0) (4.5,4.5)};
   \end{axis}
\end{tikzpicture}
\end{image}
}

\choice{
\begin{image}
\begin{tikzpicture}
    \begin{axis}[
            xmin=0, xmax=4.5, ymin=0,ymax=4.5,
            unit vector ratio*=1 1 1,
            axis lines =middle, xlabel=$x$, ylabel=$y$,
            every axis y label/.style={at=(current axis.above origin),anchor=south},
            every axis x label/.style={at=(current axis.right of origin),anchor=west},
            xtick={0,...,4}, ytick={0,...,4},
            grid=major,width=4in,
            grid style={dashed, gridColor},
          ]
        \addplot[very thick, samples=200, color=penColor, smooth, domain=(0:4.5)] {2*(x)^(1/2)
};
        \addplot[very thick, color=red, smooth, domain=(0:4.5)] {.5*x+2};
   \end{axis}
\end{tikzpicture}
\end{image}
}
\end{multipleChoice}

(2) Choose the correct statement:

(i) \begin{multipleChoice}
\choice{$f(1)=f(3)$}
\choice{$f(1)>f(3)$}
\choice[correct]{$f(1)<f(3)$} 
\end{multipleChoice}

(ii) \begin{multipleChoice}
\choice{$f'(1)=f'(3)$}
\choice[correct]{$f'(1)>f'(3)$} 
\choice{$f'(1)<f'(3)$} 
\end{multipleChoice}

(iii) Select the best approximation of $f'(3)$
\begin{multipleChoice}
\choice{$f'(3)\approx 2$}
\choice{$f'(3)\approx -2$}
\choice{$f'(3)\approx 1$}
\choice{$f'(3)\approx -1$}
\choice[correct]{$f'(3)\approx \frac{1}{2}$}
\choice{$f'(3)\approx -\frac{1}{2}$}  
\end{multipleChoice}
\end{exercise}

%Exercise 4
\begin{exercise}
Let $f$ be a function such that $f(4)=3$ and $f'(4)=-6$.

(a) Find the limit or say it does not exist (DNE).

$\lim_{x\to 4} \frac{f(x)-f(4)}{x-4}=\answer{-6}$

(b) An equation of the tangent line to the curve $y=f(x)$ at the point where $x=4$ is given by $y=\answer{-6x+27}$
\end{exercise}

%Exercise 5
\begin{exercise}
(i) Given $$e^{xy}=e+x-y,$$ use implicit differentiation to find $\frac{dy}{dx}=\answer{\frac{1-ye^{xy}}{1+xe^{xy}}}$

(ii) Given y, select $\frac{dy}{dx}$:

(a) $y=5\sin(x)\arctan\left(\frac{x}{x+1}\right)$
\begin{multipleChoice}
\choice{$5\cos(x)\frac{(x+1)^{-1}-x(x+1)^{-2}}{1+\frac{x^2}{(x+1)^2}}$}
\choice[correct]{$5\sin(x)\frac{(x+1)^{-1}-x(x+1)^{-2}}{1+\frac{x^2}{(x+1)^2}}+5\cos(x)\arctan(\frac{x}{x+1})$}
\choice{$5\sin(x)\frac{1}{1+\frac{x^2}{(x+1)^2}}+5\cos(x)\arctan(\frac{x}{x+1})$}
\end{multipleChoice}
(b) $y=\frac{1}{\sqrt{\tan(x^2)+5}}$
\begin{multipleChoice}
\choice[correct]{$\frac{-x\sec^2(x^2)}{(\tan(x^2)+5)^{1.5}}$}
\choice{$\frac{-1}{(\tan(x^2)+5)^{1.5}}$}
\choice{$\frac{-x\sec^2(2x)}{(\tan(x^2)+5)^{1.5}}$}
\end{multipleChoice}

(c) $y=xe^{ax}+\arcsin(ax)$
\begin{multipleChoice}
\choice{$e^{ax}+axe^{ax}+a\arccos(ax)$}
\choice{$ae^{ax}+\frac{a}{\sqrt{1-a^2x^2}}$}
\choice[correct]{$e^{ax}+axe^{ax}+\frac{a}{\sqrt{1-a^2x^2}}$}
\end{multipleChoice}

(d) $y(x)=\left(\sin(x)\right)^{ax}$
\begin{multipleChoice}
\choice[correct]{$\sin(x)^{ax}(ax\cot(x)+a\ln(\sin(x)))$}
\choice{$ax\sin(x)^{ax-1}\cos{x}$}
\choice{$a\sin(x)^{ax}\ln(\sin(x))$}
\end{multipleChoice}

(ii) Find the following values or state the value is undefined (DNE):

(a) $\sin\left(\frac{17\pi}{6}\right)=\frac{1}{\answer{2}}$

(b) $\arcsin\left(\frac{17\pi}{6}\right)=\answer[format=string]{DNE}$

(c) $\arcsin\left(\frac{-\sqrt{3}}{2}\right)=\frac{\pi}{\answer{-3}}$

(d) $e^{-2\ln(3)}=\frac{1}{\answer{9}}$

(e) $\ln\left(-e^2\right)=\answer[format=string]{DNE}$

(f) $\ln\left(e^{2t}\right)=2\answer{t}$

(g) $10^{3\log_{10}(4)}=4^{\answer{3}}$

(h) $\ln(1)=\answer{0}$

(i) $\arctan\left(-\sqrt{3}\right)=\frac{\pi}{\answer{-3}}$

(j) $\left[\ddx\left(\cot(x)\right)\right]_{\frac{\pi}{3}}=\answer{-\frac{4}{3}}$

(k) $\left[\ddx\left(\arctan(x)\right)\right]_{-\sqrt{3}}=\answer{\frac{1}{4}}$
\end{exercise}

\begin{exercise}
A table of values for $f(x)$ and $f'(x)$, along with a graph of a function $g(x)$ is shown below.
\[
\begin{array}{c|c|c}
x & f(x)&f'(x)\\ \hline
1 & 2 & 4\\ \hline
2 & 3 & 5\\ \hline
3 & 4 & 1\\ \hline
\end{array}
\]
\begin{image}

\begin{tikzpicture}
    \begin{axis}[
            xmin=0,
            xmax=6,
            ymax=5,
	    ymin=-1,
            samples=100,
            axis lines =middle, xlabel=$x$, ylabel=$y$,
            ytick={1,2,3,4},
            every axis y label/.style={at=(current axis.above origin),anchor=south},
            every axis x label/.style={at=(current axis.right of origin),anchor=west},
	    grid=major, width=4in,
            grid style={dashed, gridColor},
          ]
          \addplot [very thick, penColor, smooth, domain=(0:1)] {3*x};
          \addplot [very thick, penColor, smooth, domain=(1:4)] {4-x};
      \addplot [very thick, penColor, smooth, domain=(4:6)] {(x-4)^2};
        \end{axis}
\end{tikzpicture}
\end{image}

(i)  $\ddx\left[g(x)\right]_{x=1}$=\begin{multipleChoice}
\choice{$\frac{13}{3}$}
\choice{$\frac{4}{3}$}
\choice{12}
\choice{4}
\choice[correct]{DNE}
\choice{None of the above}
\end{multipleChoice}

(ii)  $\ddx\left[f(g(x))\right]_{x=3}$=\begin{multipleChoice}
\choice{-1}
\choice[correct]{-4}
\choice{$\frac{1}{4}$}
\choice{4}
\choice{DNE}
\choice{None of the above}
\end{multipleChoice}

\[
\begin{array}{c|c|c}
x & f(x)&f'(x)\\ \hline
1 & 2 & 4\\ \hline
2 & 3 & 5\\ \hline
3 & 4 & 1\\ \hline
\end{array}
\]
\begin{image}

\begin{tikzpicture}
    \begin{axis}[
            xmin=0,
            xmax=6,
            ymax=5,
            ymin=-1,
            samples=100,
            axis lines =middle, xlabel=$x$, ylabel=$y$,
            ytick={1,2,3,4},
            every axis y label/.style={at=(current axis.above origin),anchor=south},
            every axis x label/.style={at=(current axis.right of origin),anchor=west},
            grid=major, width=4in,
            grid style={dashed, gridColor},
          ]
          \addplot [very thick, penColor, smooth, domain=(0:1)] {3*x};
          \addplot [very thick, penColor, smooth, domain=(1:4)] {4-x};
      \addplot [very thick, penColor, smooth, domain=(4:6)] {(x-4)^2};
        \end{axis}
\end{tikzpicture}
\end{image}

(iii)  $\ddx\left[g(f(x))\right]_{x=2}$=\begin{multipleChoice}
\choice{-1}
\choice[correct]{-5}
\choice{$\frac{1}{5}$}
\choice{0}
\choice{DNE}
\choice{None of the above}
\end{multipleChoice}

(iv) $f^{-1}(3)=$ \begin{multipleChoice}
\choice{4}
\choice{1}
\choice{$\frac{1}{3}$}
\choice[correct]{2}
\choice{DNE}
\choice{None of the above}
\end{multipleChoice}

\[
\begin{array}{c|c|c}
x & f(x)&f'(x)\\ \hline
1 & 2 & 4\\ \hline
2 & 3 & 5\\ \hline
3 & 4 & 1\\ \hline
\end{array}
\]
\begin{image}

\begin{tikzpicture}
    \begin{axis}[
            xmin=0,
            xmax=6,
            ymax=5,
            ymin=-1,
            samples=100,
            axis lines =middle, xlabel=$x$, ylabel=$y$,
            ytick={1,2,3,4},
            every axis y label/.style={at=(current axis.above origin),anchor=south},
            every axis x label/.style={at=(current axis.right of origin),anchor=west},
            grid=major, width=4in,
            grid style={dashed, gridColor},
          ]
          \addplot [very thick, penColor, smooth, domain=(0:1)] {3*x};
          \addplot [very thick, penColor, smooth, domain=(1:4)] {4-x};
      \addplot [very thick, penColor, smooth, domain=(4:6)] {(x-4)^2};
        \end{axis}
\end{tikzpicture}
\end{image}

(v)  $\ddx\left[f^{-1}(x)\right]_{x=3}$=\begin{multipleChoice}
\choice{$\frac{1}{4}$}
\choice[correct]{$\frac{1}{5}$}
\choice{2}
\choice{4}
\choice{DNE}
\choice{None of the above}
\end{multipleChoice}
\end{exercise}

%Exercise 7
\begin{exercise}
The function $f$ is defined by $f(x)=\frac{x}{\sqrt{x^2-9}}$.\\

Answer the following question about the function $f$.\\

(a) The domain of $f$ is $(\answer{-\infty},\answer{-3})\cup (\answer{3},\answer{\infty})$

(b) The function $f$ is \begin{multipleChoice}
\choice[correct]{odd}
\choice{even}
\choice{neither}
\end{multipleChoice}

(c) Select all horizontal asymptotes
\begin{selectAll}
\choice{y=0}
\choice[correct]{y=1}
\choice{x=1}
\choice[correct]{y=-1}
\choice{There are no horizontal asymptotes}
\end{selectAll}

(d) Select all vertical asymptotes
\begin{selectAll}
\choice[correct]{x=-3}
\choice{y=1}
\choice[correct]{x=3}
\choice{y=3}
\choice{There are no vertical asymptotes}
\end{selectAll}  

(e) $f'(x)=\answer{\frac{-9}{(x^2-9)^{1.5}}}$

(f) $f''(x)=\answer{\frac{27x}{(x^2-9)^{2.5}}}$

(g) Select the correct statement for each interval:

\begin{tabular}{|c|c|c|}\hline
On this interval & f is & f is \\
$(-\infty,-3)$ & \wordChoice{\choice{inceasing}\choice[correct]{decreasing}\choice{not defined}}& \wordChoice{\choice{concave up}\choice{not defined}\choice[correct]{cocave down}}\\ \hline
$(-3,3)$ & \wordChoice{\choice{inceasing}\choice{decreasing}\choice[correct]{not defined}}& \wordChoice{\choice{concave up}\choice[correct]{not defined}\choice{cocave down}}\\ \hline
$(3,\infty)$ & \wordChoice{\choice{inceasing}\choice[correct]{decreasing}\choice{not defined}} & \wordChoice{\choice[correct]{concave up}\choice{not defined}\choice{cocave down}}\\\hline 
\end{tabular}

(h) Select the correct graph for $f$:
\begin{multipleChoice}
\choice{
\begin{image}
\begin{tikzpicture}
    \begin{axis}[
            xmin=-6, xmax=6, ymin=-3,ymax=3,
            unit vector ratio*=1 1 1,
            axis lines =middle, xlabel=$x$, ylabel=$y$,
            every axis y label/.style={at=(current axis.above origin),anchor=south},
            every axis x label/.style={at=(current axis.right of origin),anchor=west},
            xtick={-6,...,6}, ytick={-6,...,6},
            grid=major,width=4in,
            grid style={dashed, gridColor},
          ]
        \addplot[very thick, color=penColor, smooth, domain=(-6:-3)] {-x/sqrt(x^2-9)};
        \addplot[very thick, color=penColor, smooth, domain=(3:6)] {-x/sqrt(x^2-9)};
    \end{axis}
\end{tikzpicture}
\end{image}
}
\choice{
\begin{image}
\begin{tikzpicture}
    \begin{axis}[
            xmin=-6, xmax=6, ymin=-3,ymax=3,
            unit vector ratio*=1 1 1,
            axis lines =middle, xlabel=$x$, ylabel=$y$,
            every axis y label/.style={at=(current axis.above origin),anchor=south},
            every axis x label/.style={at=(current axis.right of origin),anchor=west},
            xtick={-6,...,6}, ytick={-6,...,6},
            grid=major,width=4in,
            grid style={dashed, gridColor},
          ]
        \addplot[very thick, color=penColor, smooth, domain=(-3:3)] {x/sqrt(9-x^2)};
    \end{axis}
\end{tikzpicture}
\end{image}
}
\choice[correct]{
\begin{image}
\begin{tikzpicture}
    \begin{axis}[
            xmin=-6, xmax=6, ymin=-3,ymax=3,
            unit vector ratio*=1 1 1,
            axis lines =middle, xlabel=$x$, ylabel=$y$,
            every axis y label/.style={at=(current axis.above origin),anchor=south},
            every axis x label/.style={at=(current axis.right of origin),anchor=west},
            xtick={-6,...,6}, ytick={-6,...,6},
            grid=major,width=4in,
            grid style={dashed, gridColor},
          ]
        \addplot[very thick, color=penColor, smooth, domain=(-6:-3)] {x/sqrt(x^2-9)};
        \addplot[very thick, color=penColor, smooth, domain=(3:6)] {x/sqrt(x^2-9)};
    \end{axis}
\end{tikzpicture}
\end{image}
}
\choice{\begin{image}
\begin{tikzpicture}
    \begin{axis}[
            xmin=-6, xmax=6, ymin=-3,ymax=3,
            unit vector ratio*=1 1 1,
            axis lines =middle, xlabel=$x$, ylabel=$y$,
            every axis y label/.style={at=(current axis.above origin),anchor=south},
            every axis x label/.style={at=(current axis.right of origin),anchor=west},
            xtick={-6,...,6}, ytick={-6,...,6},
            grid=major,width=4in,
            grid style={dashed, gridColor},
          ]
        \addplot[very thick, color=penColor, smooth, domain=(-6:-3)] {x/sqrt(x^2-9)};
        \addplot[very thick, color=penColor, smooth, domain=(3:6)] {-x/sqrt(x^2-9)};
    \end{axis}
\end{tikzpicture}
\end{image}
}
\end{multipleChoice}


(i) Is the function $f$ one-to-one:
\begin{multipleChoice}
\choice[correct]{Yes}
\choice{No}
\end{multipleChoice}
\end{exercise}

%Exercise 8
\begin{exercise}
The graph of $f'$ (the derivative of $f$) on the interval $(-6,7)$ is shown in the figure.

\begin{image}
\begin{tikzpicture}
    \begin{axis}[
            xmin=-6, xmax=7, ymin=-4,ymax=3,
            unit vector ratio*=1 1 1,
            axis lines =middle, xlabel=$x$, ylabel=$y$,
            every axis y label/.style={at=(current axis.above origin),anchor=south},
            every axis x label/.style={at=(current axis.right of origin),anchor=west},
            xtick={-6,...,7}, ytick={-6,...,6},
          ]
        \addplot[very thick, color=penColor, smooth, domain=(-6:0)] {x+2};
        \addplot[very thick, color=penColor, smooth, domain=(0:4)] {(2-x^2/8};
	\addplot[very thick, color=penColor, smooth, domain=(4:7)] {1-(x-5)^2};
    \end{axis}
\end{tikzpicture}
\end{image}



Use the given graph of $f'$ to answer the following questions about f:

(a) On what interval(s) is $f$ decreasing?\\

ANSWER:$(\answer{-6},\answer{-2})$ and $(\answer{6},\answer{7})$\\

(b) List the x- coordinates of all critical points of $f$ (in ascending order). \\

ANSWER: $x=\answer{-2},\answer{4},\answer{6}$\\

(c) List the x-coordinates of all critical points of $f$ that correspond to local maxima?\\

 ANSWER: $x=\answer{6}$\\

(d)  List the x-coordinates of all critical points of $f$ that correspond to neither local maxima nor local minima? \\

ANSWER: $x=\answer{4}$\\

(e) On what intervals is $f$ concave up?\\

  ANSWER:$(\answer{-6},0)$ and $(\answer{4},\answer{5})$\\

(f) List the x-coordinates of  all inflection points of $f$ (in ascending order). \\

ANSWER $x=\answer{0},\answer{4},\answer{5}$\\
\end{exercise}

%Exercise 9
\begin{exercise}
Given that $f(1)=2$, $f'(1)=3$, and $f''(1)=-1$, find the following values or state `cannot be determined' (CBD):

(a) $\left(\ddx\frac{(x+5)f'(x)}{f(x)}\right)_{x=1}=\answer{-15}$

(b) $\lim_{x\to 1}f(x)=\answer{2}$

(c) $\left(\ddx f^{-1}(x)\right)_{x=2}=\answer{\frac{1}{3}}$

(d) $\lim_{h\to 0}\frac{f(1+h)-f(1)}{h}=\answer{3}$

(e) $\lim_{h\to 0} \frac{f'(1+h)-f'(1)}{h}=\answer{-1}$

(f) $\lim_{x \to 1} \frac{f'(x)-f'(1)}{x-1}=\answer{-1}$

\end{exercise}

%Exercise 10
\begin{exercise}
Simplify!\\
Given that $f(x)=\ln(\sec(x)+\tan(x))$, $f'(x) = \answer{\sec(x)}$
\end{exercise}

%Exercise 11

\begin{exercise}
The position, $s(t)$, of an object moving along a horizontal line is given by $s(t)=3\sin\left(\frac{\pi}{4}t\right)$, $t\geq0$, where $s$ is measured in feet and $t$ in seconds.\\

Complete the statements below.\\

(a) The position of the particle at time $t=2$ is $\answer{3}$ ft.

(b) The \underline{average velocity}, $v_{av}$, of the object over the interval $[0,t]$ is

$\answer{\frac{3\sin(\frac{\pi}{4}t)}{t}}$ $\frac{\text{ft}}{\text{s}}$.\\

(c) Using the expression found in part (b), evaluate $\lim_{t\to0^+} v_{av} = \answer{\frac{3\pi}{4}}$ $\frac{\text{ft}}{\text{s}}$

(e) What does the limit in part (c) represent?
\begin{multipleChoice}
\choice{The position at $t=0$}
\choice{The acceleration at $t=0$}
\choice[correct]{The instantaneous velocity at $t=0$}
\end{multipleChoice}

(f) Find the velocity of the particle: $v(t)=\answer{\frac{3\pi}{4}\cos(\frac{\pi}{4}t)}$ $\frac{\text{ft}}{\text{s}}$

(g) Find the acceleartion of the particle: $a(t)=\answer{-\frac{3\pi^2}{16}\sin(\frac{\pi}{4}t)}$ $\frac{\text{ft}}{\text{s}^2}$

\end{exercise}

%Exercise 12
\begin{exercise}
The figure below shows the graphs of $f$, $f'$, and another function $g$.
\begin{image}
\begin{tikzpicture}
    \begin{axis}[
            xmin=-3, xmax=3, ymin=-1.3,ymax=1.3,
            unit vector ratio*=1 1 1,
            axis lines =middle, xlabel=$x$, ylabel=$y$,
            every axis y label/.style={at=(current axis.above origin),anchor=south},
            every axis x label/.style={at=(current axis.right of origin),anchor=west},
            xtick={-3,...,3}, ytick={-1,0,1},
          ]
        \addplot[ultra thick, color=penColor, smooth, domain=(-3:3)] {-2*x*e^(-x^2)} node [pos=0.75, below right] {A};
        \addplot[thick, color=penColor, smooth, domain=(-3:3)] {sin(x*180/pi)} node [pos=.9, above right] {B};
        \addplot[very thick, color=penColor, smooth, domain=(-3:3)] {e^(-x^2)} node [pos=.55, above right] {C};
    \end{axis}
\end{tikzpicture}
\end{image}

Which curve is which: $f=\answer[format=string]{c}$; $f'=\answer[format=string]{a}$; $g=\answer[format=string]{b}$
\end{exercise}
%Exercise 13

\begin{exercise}
The (entire) graph of a function $f$ is shown in the figure below.

\begin{image}
\begin{tikzpicture}
    \begin{axis}[
            xmin=0, xmax=5, ymin=-1,ymax=2.2,
            unit vector ratio*=1 1 1,
            axis lines =middle, xlabel=$x$, ylabel=$y$,
            every axis y label/.style={at=(current axis.above origin),anchor=south},
            every axis x label/.style={at=(current axis.right of origin),anchor=west},
            xtick={0,...,5}, ytick={-1,...,2},
          ]
        \addplot[ultra thick, color=penColor, smooth, domain=(0:1)] {(x-1)^2};
        \addplot[ultra thick, color=penColor, smooth, domain=(1:2)] {2*(x-1)^2};
        \addplot[ultra thick, color=penColor, smooth, domain=(2:5)] {4-x};
	\addplot [color=penColor,fill=penColor,only marks,mark=*] coordinates{(0,1)};
	\addplot [color=penColor,fill=penColor,only marks,mark=*] coordinates{(5,-1)};
    \end{axis}
\end{tikzpicture}
\end{image}

Give answers for the following, or write `DNE'.

%(a) Find the x-coordinate of all points where $f$ attains its absolute maximum: $x=\answer{2}$

%(b) Find the x-coordinate of all points where $f$ attains its absolute minimum: $x=\answer{5}$

(a) Find the x-coordiantes of all points in the interval $(0,5)$ where $f$ has a local minimum. \\

ANSWER: $x=\answer{1}$\\

(b) Find the x-coordinate of all points in the interval $(0,5)$ where $f$ has a local maximum.\\

 ANSWER: $x=\answer{2}$\\

(c) Find all critical points in the interval $(0,5)$ (in ascending order). \\

ANSWER: $x=\answer{1},\answer{2}$\\
\end{exercise}

%Exercise 14
\begin{exercise}
Assume that a function $f$ is continuous on its domain, $(-1,6)$. The graph of $f'$, the derivative of $f$, is shown in the figure below.

\begin{image}
\begin{tikzpicture}
    \begin{axis}[
            xmin=-1, xmax=6, ymin=-2.5,ymax=1.5,
            unit vector ratio*=1 1 1,
            axis lines =middle, xlabel=$x$, ylabel=$y$,
            every axis y label/.style={at=(current axis.above origin),anchor=south},
            every axis x label/.style={at=(current axis.right of origin),anchor=west},
            xtick={-1,...,6}, ytick={-2,...,1},
          ]
        \addplot[ultra thick, color=penColor, smooth, domain=(-1:1)] {-2*x^2};
        \addplot[ultra thick, color=penColor, smooth, samples=500, domain=(1:3)] {-2*2^(-1/3)*(3-x)^(1/3)};
	\addplot[ultra thick, color=penColor, smooth, samples=100, domain=(3:5)] {2^(-1/3)*(x-3)^(1/3)};
        \addplot[ultra thick, color=penColor, smooth, domain=(5:6)] {x-6};
	\addplot[ultra thick, color=penColor, smooth]plot coordinates{(3,-.1) (3,0)};
	\addplot [color=penColor,fill=background,only marks,mark=*] coordinates{(-1,-2)};
	\addplot [color=penColor,fill=background,only marks,mark=*] coordinates{(5,1)};
	\addplot [color=penColor,fill=background, only marks, mark=*] coordinates{(5,-1)};
        \addplot[color=penColor,fill=background,only marks, mark=*] coordinates{(6,0)};
    \end{axis}
\end{tikzpicture}
\end{image}

(a) Write the x-coordinates of all critical points of $f$ (or write NONE), in ascending order.\\

 ANSWER: $x=\answer{0},\answer{3},\answer{5}$\\

(b) Write the x-coordinates of all local maxima of $f$ (or write NONE).\\ 

ANSWER: $x=\answer{5}$\\

(c) Write the x-coordinates of all local minima of $f$ (or write NONE).\\

 ANSWER: $x=\answer{3}$\\

(d)Find the interval(s) on which $f$ is increasing.\\ 

ANSWER: $(\answer{3},\answer{5})$\\

(e) Find the interval(s) on which $f$ is concave down. \\

ANSWER: $(\answer{0},\answer{1})$\\

(f) Write the x-coordinates of all inflection points (or write NONE), in ascending order.\\

 ANSWER: $x=\answer{0},\answer{1}$\\
\end{exercise}

%Exercise 15
\begin{exercise}
A curve is given by the equation $$\tan(3y+x)=x^2-1$$.

(i) (a) Use implicit differentiation to find the derivative.\\

 $\frac{dy}{dx}=\answer{\frac{2x}{3\sec^2(3y+x)}-\frac{1}{3}}$

(b) Check (algebraically), that the point $(1,-\frac{1}{3})$ lies on the curve.

(c) Compute $\frac{dy}{dx}$ at the point $(1,-\frac{1}{3})$.\\

ANSWER: $\left[\frac{dy}{dx}\right]_{x=1, y=-\frac{1}{3}}=\answer{\frac{1}{3}}$
 
(ii) Part of the curve $\tan(3y+x)=x^2-1$ that contains the point $(1,-\frac{1}{3})$ is shown in the figure below:

\begin{image}
\begin{tikzpicture}
    \begin{axis}[
            xmin=-1, xmax=2, ymin=-.7,ymax=.7,
            unit vector ratio*=1 1 1,
            axis lines =middle, xlabel=$x$, ylabel=$y$,
            every axis y label/.style={at=(current axis.above origin),anchor=south,font=\tiny},
            every axis x label/.style={at=(current axis.right of origin),anchor=west,font=\tiny},
            xtick={-1,...,2}, ytick={-.66,-.33,0,.33,.66}, 
	    yticklabels={$-\frac{2}{3}$,$-\frac{1}{3}$,$0$,$\frac{1}{3}$,$\frac{2}{3}$},
          ]
     \addplot[very thick, color=penColor, smooth, domain=(-1:2)] {1/3*(rad(atan(x^2-1))-x)};
     \addplot[color=penColor,fill=penColor,only marks,mark=*] coordinates{(1,-1/3)};
        \end{axis}
\end{tikzpicture}
\end{image}

(a) Find an explicit expression for $y$, $y=y(x)$, represented by the graph above.\\

 $y(x)=\answer{\frac{1}{3}(\arctan(x^2-1)-x)}$

(b) Using the explicit expression in part (a), find $\frac{dy}{dx}$.\\

$\frac{dy}{dx}=\answer{\frac{1}{3}(\frac{2x}{(x^2-1)^2+1}-1)}$

(c) Evaluate the derivative.\\
 $\left[\frac{dy}{dx}\right]_{x=1}=\answer{\frac{1}{3}}$

%(d) Select the graph with the line tangent to the curve at the point $(1,-\frac{1}{3})$

%INSERT GRAPHS

(d) Write an equation of the tangent line to the graph  of $y=y(x)$ at the point where $x=1$. \\

ANSWER: $y=\answer{\frac{1}{3}(x-1)-\frac{1}{3}}$
\end{exercise}

%Exercise 16

%Exercise 17
\begin{exercise}
A function $f'$ (the derivative of $f$) is given by $f'(x)=(x-5)^3$. Answer the following, or write  "DNE" (does not exist).

(a) List all interval(s) on which $f$ is increasing.\\


ANSWER: $(\answer{5},\answer{\infty})$\\


(b) List x-coordinates of all points where $f$ has a local maximum. \\

ANSWER: $x=\answer[format=string]{DNE}$\\

(c)  List x-coordinates of all points where $f$ has a local minimum.\\

ANSWER: $x=\answer{5}$\\

(d) Find  $f''(x)$.\\

ANSWER: $f''(x)=\answer{3(x-5)^2}$\\

(e) List all interval(s) on which $f$ is concave up.\\

ANSWER:  $(\answer{-\infty},\answer{\infty})$\\

(f) List x-coordinates of all inflection points of $f$.\\

 ANSWER: $x=\answer[format=string]{DNE}$\\

\end{exercise}

%Exercise 18
\begin{exercise}
The function $s(t)=\frac{12}{t+1}$, measured in meters, gives the position of an object moving along a line at time $t\geq 0$, measured in seconds.

(a) Is the function $s$ one-to-one? 
\begin{multipleChoice}
\choice[correct]{Yes}
\choice{No}
\end{multipleChoice}

(b) Find the expression for inverse of $s$, $s^{-1}(t)$.\\

ANSWER: $s^{-1}(t)=\answer{\frac{12}{t}-1}$\\

(c) Find $s^{-1}(5)$.\\

ANSWER: $s^{-1}(5) =\answer{\frac{7}{5}}$\\

(d) Interpret your answer to part (c), by selecting the correct answer.
\begin{multipleChoice}
\choice{$s^{-1}(5)$ gives the speed of the object at time 5}
\choice{$s^{-1}(5)$ gives the position of the object at time 5}
\choice[correct]{$s^{-1}(5)$ gives the time when the object was at position 5}
\choice{$s^{-1}(5)$ gives the time when the object had speed 5}
\end{multipleChoice}

(e) Find the average velocity between times $t=0$ and $t=2$.\\

ANSWER: $v_{av}=\answer{-4}$ $\frac{\text{m}}{\text{s}}$\\

(f) Find an expression for the instantaneous velocity at time $t>0$.\\

ANSWER:  $v(t)=\answer{-12(t+1)^{-2}}$ $\frac{\text{m}}{\text{s}}$\\

(g) Find an expression for the instantaneous acceleration at time $t>0$.\\

ANSWER: $a(t)=\answer{24(t+1)^{-3}}$ $\frac{\text{m}}{\text{s}^2}$\\

(h) Is the velocity increasing or decreasing for $t>0$? Select the correct answer.\\
\begin{multipleChoice}
\choice[correct]{Increasing}
\choice{Decreasing}
\end{multipleChoice}

(i) Is the speed increasing or decreasing for $t>0$? Select the correct answer.\\
\begin{multipleChoice}
\choice{Increasing}
\choice[correct]{Decreasing}
\end{multipleChoice}
\end{exercise}

%Exercise 20
\begin{exercise}
A ladder $10$ ft long rests against a vertical wall. If the bottom of the ladder slides away from the wal at a speed of $2$ ft/s, how fast is the angle between the top of the ladder and the wall changing when the angle is $\frac{\pi}{3}$ radians?  $\answer{\frac{2}{5}}$ $\frac{\text{rad}}{\text{s}}$
\end{exercise}

%Exercise 21
\begin{exercise}
The position function (in feet) for an object at time $t$ (in minutes), $0\leq t\leq 10$, is given by $s(t)=-40t^2+160t+480$.

(a) Find the velocity $v(t)$ at any time $t$.\\

ANSWER: $v(t)=\answer{-80t+160}$ $\frac{\text{ft}}{\text{min}}$

(b) Find the acceleration $a(t)$ at any time $t$.\\

ANSWER: $a(t)=\answer{-80}$ $\frac{\text{ft}}{\text{min}^2}$

(c) At what time is the object furthest from the origin in the positive direction?\\

ANSWER: At the time $t=\answer{2}$ min.\\

(d) What are the velocity and acceleration at that time?\\

ANSWER:  $v=\answer{0}$ $\frac{\text{ft}}{\text{min}}$, $a=\answer{-80} \frac{\text{ft}}{\text{min}^2}$\\

(e) At what (positive) time is the object at the origin?\\

ANSWER: At the time  $t=\answer{6}$ min.\\

(f) What are the velocity and acceleration at that time?\\

ANSWER: $v=\answer{-320}$ $\frac{\text{ft}}{\text{min}}$, $a=\answer{-80}$ $\frac{\text{ft}}{\text{min}^2}$

(g) Find the time interval(s) when the velocity is decreasing.\\

ANSWER:  $(\answer{0},\answer{10})$\\
\end{exercise}

%Exercise 22
\begin{exercise}
The (entire) graph of a function $f$ is given in the figure below:

\begin{image}
 
\begin{tikzpicture}
    \begin{axis}[
            xmin=-10, xmax=10, ymin=-6,ymax=6,
            unit vector ratio*=1 1 1,
            axis lines =middle, xlabel=$x$, ylabel=$y$,
            every axis y label/.style={at=(current axis.above origin),anchor=south},
            every axis x label/.style={at=(current axis.right of origin),anchor=west},
            xtick={-10,-8,-6,-4,-2,0,2,4,6,8,10}, ytick={-6,-4,-2,0,2,4,6},
          ]
          \addplot[color=penColor,very thick] plot coordinates
                  {(-8,0) (0,-4)};
      \addplot[color=penColor,very thick] plot coordinates
          {(3,2) (8,0)};
      \addplot[very thick, color=penColor, smooth, domain=(0:3)] {6-(2*x/3)^2};
           \addplot[color=penColor,fill=background,only marks,mark=*] coordinates{(-8,0)};  %% open hole
          \addplot[color=penColor,fill=background,only marks,mark=*] coordinates{(0,-4)};  %% open hole
      \addplot[color=penColor,fill=background,only marks,mark=*] coordinates{(0,6)};
      \addplot[color=penColor,fill=background,only marks,mark=*] coordinates{(3,2)};
      \addplot[color=penColor,fill=penColor,only marks,mark=*] coordinates{(3,0)};
      \addplot[color=penColor,fill=background,only marks,mark=*] coordinates{(8,0)};
        \end{axis}
\end{tikzpicture}
\end{image}

(a) Find the value.\\
 $\ddx\left[f^{-1}(x)\right]_{x=-3}=\answer{-2}$\\

(b) List  the x-coordinates of all critical points of $f$.\\

ANSWER: $x=\answer{3}$\\

(c) List the x-coordinates of all local minima of $f$, or say `none'.\\

ANSWER: $x=\answer{3}$\\

(d) List the x-coordinates of all local maxima in of $f$, or say `none'.\\ 

$x=\answer[format=string]{none}$\\

%(e) List the x-coordinates of all absolute maxima of $f$, or say `none': $x=\answer[format=string]{none}$

%(f) List the x-coordinates of all absolute minima of $f$, or say `none': $x=\answer[format=string]{none}$

(g) Select the correct graph of $f'$.
\begin{multipleChoice}
\choice{
\begin{image}
\begin{tikzpicture}
    \begin{axis}[
            xmin=-8, xmax=8, ymin=-2,ymax=2,
            unit vector ratio*=1 1 1,
            axis lines =middle, xlabel=$x$, ylabel=$y$,
            every axis y label/.style={at=(current axis.above origin),anchor=south},
            every axis x label/.style={at=(current axis.right of origin),anchor=west},
            xtick={-8,-6,-4,-2,0,2,4,6,8}, ytick={-1,0,1},
          ]
          \addplot[color=penColor,very thick] plot coordinates
                  {(-8,-1/2) (0,-1/2)};
      \addplot[color=penColor,very thick] plot coordinates
          {(3,-2/5) (8,-2/5)};
      \addplot[very thick, color=penColor, smooth, domain=(0:3)] {-2*x/9};
          \addplot[color=penColor,fill=penColor,only marks,mark=*] coordinates{(-8,-1/2)};  
%% closed hole
          \addplot[color=penColor,fill=penColor,only marks,mark=*] coordinates{(0,-1/2)};  %
% open hole
      \addplot[color=penColor,fill=background,only marks,mark=*] coordinates{(0,0)};
      \addplot[color=penColor,fill=penColor,only marks,mark=*] coordinates{(3,-2/3)};
      \addplot[color=penColor,fill=background,only marks,mark=*] coordinates{(3,-2/5)};
      \addplot[color=penColor,fill=penColor,only marks,mark=*] coordinates{(8,-2/5)};
        \end{axis}
\end{tikzpicture}
\end{image}
}

\choice[correct]{
\begin{image}
\begin{tikzpicture}
    \begin{axis}[
            xmin=-8, xmax=8, ymin=-2,ymax=2,
            unit vector ratio*=1 1 1,
            axis lines =middle, xlabel=$x$, ylabel=$y$,
            every axis y label/.style={at=(current axis.above origin),anchor=south},
            every axis x label/.style={at=(current axis.right of origin),anchor=west},
            xtick={-8,-6,-4,-2,0,2,4,6,8}, ytick={-1,0,1},
          ]
          \addplot[color=penColor,very thick] plot coordinates
                  {(-8,-1/2) (0,-1/2)};
      \addplot[color=penColor,very thick] plot coordinates
          {(3,-2/5) (8,-2/5)};
      \addplot[very thick, color=penColor, smooth, domain=(0:3)] {-2*x/9};
          \addplot[color=penColor,fill=background,only marks,mark=*] coordinates{(-8,-1/2)};  

%% closed hole
          \addplot[color=penColor,fill=background,only marks,mark=*] coordinates{(0,-1/2)};  %

% open hole
      \addplot[color=penColor,fill=background,only marks,mark=*] coordinates{(0,0)};
      \addplot[color=penColor,fill=background,only marks,mark=*] coordinates{(3,-2/3)};
      \addplot[color=penColor,fill=background,only marks,mark=*] coordinates{(3,-2/5)};
      \addplot[color=penColor,fill=background,only marks,mark=*] coordinates{(8,-2/5)};
        \end{axis}
\end{tikzpicture}
\end{image}
}

\choice{
\begin{image}
\begin{tikzpicture}
    \begin{axis}[
            xmin=-8, xmax=8, ymin=-2,ymax=2,
            unit vector ratio*=1 1 1,
            axis lines =middle, xlabel=$x$, ylabel=$y$,
            every axis y label/.style={at=(current axis.above origin),anchor=south},
            every axis x label/.style={at=(current axis.right of origin),anchor=west},
            xtick={-8,-6,-4,-2,0,2,4,6,8}, ytick={-1,0,1},
          ]
          \addplot[color=penColor,very thick] plot coordinates
                  {(-8,1/2) (0,1/2)};
      \addplot[color=penColor,very thick] plot coordinates
          {(3,2/5) (8,2/5)};
      \addplot[very thick, color=penColor, smooth, domain=(0:3)] {2*x/9};
          \addplot[color=penColor,fill=background,only marks,mark=*] coordinates{(-8,1/2)};  %% closed hole
          \addplot[color=penColor,fill=background,only marks,mark=*] coordinates{(0,1/2)};  %% open hole
      \addplot[color=penColor,fill=background,only marks,mark=*] coordinates{(0,0)};
      \addplot[color=penColor,fill=background,only marks,mark=*] coordinates{(3,2/3)};
      \addplot[color=penColor,fill=background,only marks,mark=*] coordinates{(3,2/5)};
      \addplot[color=penColor,fill=background,only marks,mark=*] coordinates{(8,2/5)};
        \end{axis}
\end{tikzpicture}
\end{image}
}

\choice{
\begin{image}
\begin{tikzpicture}
    \begin{axis}[
            xmin=-8, xmax=8, ymin=-2,ymax=2,
            unit vector ratio*=1 1 1,
            axis lines =middle, xlabel=$x$, ylabel=$y$,
            every axis y label/.style={at=(current axis.above origin),anchor=south},
            every axis x label/.style={at=(current axis.right of origin),anchor=west},
            xtick={-8,-6,-4,-2,0,2,4,6,8}, ytick={-1,0,1},
          ]
          \addplot[color=penColor,very thick] plot coordinates
                  {(-8,-1) (0,-1)};
      \addplot[color=penColor,very thick] plot coordinates
          {(3,-4/5) (8,-4/5)};
      \addplot[very thick, color=penColor, smooth, domain=(0:3)] {-4*x/9};
          \addplot[color=penColor,fill=background,only marks,mark=*] coordinates{(-8,-1)};
%% closed hole
          \addplot[color=penColor,fill=background,only marks,mark=*] coordinates{(0,1)};  %
% open hole
      \addplot[color=penColor,fill=background,only marks,mark=*] coordinates{(0,0)};
      \addplot[color=penColor,fill=background,only marks,mark=*] coordinates{(3,-4/3)};
      \addplot[color=penColor,fill=background,only marks,mark=*] coordinates{(3,-4/5)};
      \addplot[color=penColor,fill=background,only marks,mark=*] coordinates{(8,-4/5)};
        \end{axis}
\end{tikzpicture}
\end{image}
}
\end{multipleChoice}
\end{exercise}

%Exercise 23
\begin{exercise}
(i) Let $f(x)=x^4-8x^2+10$. 

 List the x-coordinates of all critical points of $f$ in ascending order.\\

ANSWER: $x=\answer{-2},\answer{0},\answer{2}$\\

%(b) Find the absolute maximum of $f$ on the interval $[-3,1]$: $\answer{19}$

%(c) Find the absolute minimum of $f$ on the interval $[-3,1]$: $\answer{-6}$

(ii) Let $f(x)=x\ln(x)$

(a) Find the domain of the function $f$.\\

ANSWER: $(\answer{0},\answer{\infty})$

(b)  List the x-coordinates of all  critical points of $f$ in ascending order.\\

ANSWER: $x=\answer{\frac{1}{e}}$

%(c) List the interval(s) on which $f$ is increasing.$(\answer{\frac{1}{e}},\answer{\infty})$

%(d) Find the interval(s) on which $f$ is decreasing: $(\answer{0},\answer{e^{-1}})$

%(e) Find the absolute minimum value of $f$ on its domain or say that it does not exist (DNE): $\answer{-\frac{1}{e}}$

%(f) Find the absolute maximum value of $f$ on its domain or say that it does not exist (DNE): $\answer[format=string]{DNE}$

%(g) Find all inflection points of $f$, or, if non exist, say DNE: $x=\answer[format=string]{DNE}$

%(i) Find all interval(s) on which f is concave up: $(\answer{0},\answer{\infty})$

(iii) Let $f(x)=x\sqrt{6-x^2}$

(a) Find the domain of the function $f$.\\

ANSWER: $[\answer{-\sqrt{6}},\answer{\sqrt{6}}]$\\

(b) State the interval(s) of continuity of $f$.\\

ANSWER: $[\answer{-\sqrt{6}},\answer{\sqrt{6}}]$\\

(c) List the x-coordinates of all  critical points of $f$ in ascending order.\\

ANSWER: $\answer{-\sqrt{3}},\answer{\sqrt{3}}$\\

%(d) Find the absolute maximum of $f$ on its domain: $\answer{3}$

%(e) Find the absolute minimum of $f$ on its domain: $\answer{-3}$
\end{exercise}

%Exercise 24
\begin{exercise}
A water tank is to be drained for cleaning. There are $V$ liters of water left in the tank $t$ minutes after the draining began, where $V=42(60-t)^2$

(a) Find the average rate at which water drains during the first $10$ minutes.\\

ANSWER:  $\answer{-4620}$ $\frac{\text{liters}}{\text{min}}$\\

(b) Find the rate at which the volume of water is changing $10$ minutes after the draining began.\\ 

ANSWER: $\answer{-4200}$ $\frac{\text{liters}}{\text{min}}$

(c) Compute the derivative.\\

 $\frac{d}{dt}\left(\frac{\frac{dV}{dt}(t)}{V(t)}\right)=\answer{-2(60-t)^{-2}}$\\

(d) What are the units of the answer to part (c)?

(e) Find the rate of the rate at which the volume of water is changing 10 minutes after draining begins.

ANSWER: $\answer{84}$ $\frac{\text{liters}}{\text{min}^2}$\\

(f) Is the rate at which the volume of the water is changing increasing or decreasing (during the draining)? 
\begin{multipleChoice}
\choice[correct]{Increasing}
\choice{Decreasing}
\end{multipleChoice}

(g) Assume that the tank has the shape of a rectangular box $7$m long, $6$m wide, and $5$m high. What is the rate of change of the water depth when the water depth is $3$m? (HINT: 1 liter = .001 $m^3$) \\

ANSWER: $\answer{\frac{-\sqrt{30}}{50}}$ $\frac{\text{m}}{\text{min}}$

\end{exercise}

%Exercise 23
\begin{exercise}
We are inflating a spherical balloon at a rate of 3 $cm^3/sec$. 

(a) At what rate is the radius increasing when the radius is $6$ cm?\\

ANSWER:  $\answer{\frac{1}{48\pi}}$ $\frac{\text{cm}}{\text{sec}}$

(b) At what rate is the surface area increasing at that moment?\\

ANSWER:  $\answer{1}$ $\frac{\text{cm}^2}{\text{sec}}$
\end{exercise}

%Exercise 24
\begin{exercise}
Two cars leave an intersection. One heads west at $30$ mi/hr. The other leaves 30 minutes later, heading north at $40$ mi/hr. How fast is the distance between them changing $1$ hr after the first car left the intersection?\\

ANSWER:  $\answer{\frac{170}{\sqrt{13}}}$ $\frac{\text{mi}}{\text{hr}}$

\end{exercise}

\end{document}
%%%%%%
\begin{exercise}
Draw a possible graph of $f$ given that it satisfies all of the following conditions.

(a) Domain of $f=(-\infty,-2)\cup (-2,\infty)$,

(b) f is continuous on its domain and differentiable all all points in the domain except at $x=6$

(c) $f(2)=0, f(6)=4$,

(d) $\lim_{x\to -2} f(x)=-8$, $\lim_{x\to-\infty}f(x)=1$, $\lim_{x\to\infty}f(x)=1$,

(e) $f'(x)<0$ on $(-\infty,-2)$, and on $(6,\infty)$,

(f) $f'(x)>0$ on $(-2,6)$,

(g) $f''(x)<0$ on $(-\infty,-2)$ and on $(-2,2)$,

(h) $f''(x)>0$ on $(2,6)$ and on $(6,\infty)$

Once you've finished, select `Done', and compare your answer with the one shown

\begin{multipleChoice}
\choice[correct]{Done}
\end{multipleChoice}
\begin{exercise}
\begin{image}
\begin{tikzpicture}
    \begin{axis}[
            xmin=-8,xmax=10,ymin=-10,ymax=6,
            samples=100,
            axis lines =middle, xlabel=$x$, ylabel=$y$,
	    xtick={-8,-6,-4,-2,0,2,4,6,8,10},ytick={-8,-6,-4,-2,0,2,4,6},
            every axis y label/.style={at=(current axis.above origin),anchor=south},
            every axis x label/.style={at=(current axis.right of origin),anchor=west}
          ]
          \addplot [very thick, penColor, smooth, domain=(-8:-2)] {1+9*1/(4*x+7)};
	  \addplot [very thick, penColor, smooth, domain=(-2:2)] {-1/2*(x-2)^2};
	  \addplot [very thick, penColor, smooth, domain=(2:6)] {1/4*(x-2)^2};
	  \addplot [very thick, penColor, smooth, domain=(6:10)] {1+6/(2*(x-5))};
	  \addplot [color=penColor,fill=penColor, only marks, mark=*] coordinates{(2,0)};
 	  \addplot [color=penColor,fill=background,only marks,mark=*] coordinates{(-2,-8)};
	  \addplot [textColor, dashed] plot coordinates {(-8,1) (10,1)};
        \end{axis}
\end{tikzpicture}
\end{image}
\end{exercise}
%%%%%%
