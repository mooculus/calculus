\documentclass{ximera}

\newcommand{\RR}{\mathbb R}
\renewcommand{\d}{\,d}
\newcommand{\dd}[2][]{\frac{d #1}{d #2}}
\renewcommand{\l}{\ell}
\newcommand{\ddx}{\frac{d}{dx}}
\newcommand{\dfn}{\textbf}
\newcommand{\eval}[1]{\bigg[ #1 \bigg]}


\author{Gregory Hartman \and Matthew Carr}
\license{Creative Commons 3.0 By-NC}
\acknowledgement{https://github.com/APEXCalculus}

\outcome{Compute basic antiderivatives.}
\outcome{Use integral notation for both antiderivatives and definite integrals.}
\outcome{Undo the Chain Rule.}
\outcome{Calculate indefinite integrals (antiderivatives) using basic substitution.}

\begin{document}
\begin{exercise}

Find the indefinite integral of $\frac{\ln(x^3)}{x}$ with respect to $x$.

\[
\int \frac{\ln(x^3)}{x}\d x=\answer{\frac{1}{6}(\ln(x^3))^2}+C
\]
\begin{hint}
Using log properties, write the integral as $\int 3\frac{\ln(x)}{x}\d x$ then use $u=\ln(x)$.
\end{hint}
\begin{hint}
Another way: let $u=ln(x^3)$ at the beginning without using log properties. 
\end{hint}
\begin{hint}
Try solving it both ways and verify that both answers are the same. 
\end{hint}
\end{exercise}
\end{document}
