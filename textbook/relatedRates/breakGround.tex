\documentclass{ximera}

\newcommand{\RR}{\mathbb R}
\renewcommand{\d}{\,d}
\newcommand{\dd}[2][]{\frac{d #1}{d #2}}
\renewcommand{\l}{\ell}
\newcommand{\ddx}{\frac{d}{dx}}
\newcommand{\dfn}{\textbf}
\newcommand{\eval}[1]{\bigg[ #1 \bigg]}


\outcome{}


\title[Break-Ground:]{A changing circle}

\begin{document}
\begin{abstract}
Here we see a dialog where two calculus students discuss a circle that
is changing.
\end{abstract}
\maketitle

Check out this dialog between two calculus students (based on a true
story):

\begin{itemize}
  \item[\textbf{Devyn}] Riley, I've been thinking about calculus.   
  \item[\textbf{Riley}] YOLO.
  \item[\textbf{Devyn}] Consider a circle of some radius $r$.
  \item[\textbf{Riley}] What else would it be?
  \item[\textbf{Devyn}] Its perimeter?
  \item[\textbf{Riley}] $P=2\cdot\pi \cdot r$ baby.
  \item[\textbf{Devyn}] Its area?
  \item[\textbf{Riley}] You know it's $A=\pi\cdot r^2$.
  \item[\textbf{Devyn}] Right, but here's what's bugging me: If I know
    $r'$, what is $P'$? What's $A'$?
  \item[\textbf{Riley}] Oooh. Ouch. Hmmm. I wanna say it's
    \[
    A' = 2\cdot\pi \cdot r' \qquad\text{and}\qquad  A' = \pi (r')^2 
    \]
    but I'm not sure that is right.
  \item[\textbf{Devyn}] Yeah\dots me too. But I'm not sure that's
    right either. Are we forgetting something?
\end{itemize}
  
\begin{problem}
  Do you think our young mathematicians above are correct?
  \begin{multipleChoice}
    \choice{Yes.}
    \choice[correct]{No.}
    \choice{There is no way to tell.}
  \end{multipleChoice}
\end{problem}

\begin{problem}
  Set $r(t)=3\cdot t$. What is $r'(t)$ when $r=15$?
  \begin{prompt}
    $r'(t)=\answer{3}$
  \end{prompt}
\end{problem}

\begin{problem}
  Set $r=3\cdot t$. Now $P(t) = 2\cdot \pi\cdot 3\cdot t$. What is
  $P'(t)$ when $r=15$?
  \begin{prompt}
    $P'(t)=\answer{2*pi*3}$
  \end{prompt}
\end{problem}

\begin{problem}
  Set $r=3\cdot t$. Now $A(t) = \pi\cdot (3\cdot t)^2$. What is
  $A'(t)$ when $r=15$?
  \begin{prompt}
    $A'(t)=\answer{2*pi*15*3}$
  \end{prompt}
\end{problem}

\begin{problem}
  What, if anything, did our two young mathematicians forget about above?
  \begin{freeResponse}
 They forgot the chain rule.
  \end{freeResponse}
\end{problem}



\begin{xarmaBoost}
  Write down at least \textbf{five} questions for this lecture. After
  you have your questions, label them as ``Level 1,'' ``Level 2,'' or
  ``Level 3'' where:
\begin{description}
\item[Level 1] Means you know the answer, or know exactly how to do
  this problem.
\item[Level 2] Means you think you know how to do the problem, or will
  soon learn how to do the problem.
\item[Level 3] Means you have no idea how to do the problem.
\end{description}
\begin{freeResponse}
\end{freeResponse}
\end{xarmaBoost}


\end{document}
