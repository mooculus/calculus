\documentclass{ximera}

\newcommand{\RR}{\mathbb R}
\renewcommand{\d}{\,d}
\newcommand{\dd}[2][]{\frac{d #1}{d #2}}
\renewcommand{\l}{\ell}
\newcommand{\ddx}{\frac{d}{dx}}
\newcommand{\dfn}{\textbf}
\newcommand{\eval}[1]{\bigg[ #1 \bigg]}


\author{Jim Talamo}
\license{Creative Commons 3.0 By-NC}


\outcome{Understand the integral test remainder estimates.}

\begin{document}

\begin{exercise}

Consider the series $\sum_{k=1}^{\infty} \frac{2k}{(k^2+1)^2}$.  

Does the integral test apply?

\begin{multipleChoice}
\choice[correct]{Yes.}
\choice{No.}
\end{multipleChoice}

What does the integral test tell us?
\begin{multipleChoice}
\choice{The series converges to $\frac{1}{2}$.}
\choice[correct]{The series converges, but we do not know its value yet.}
\choice{The series diverges.}
\end{multipleChoice}

\begin{feedback}
Note that $\frac{2x}{(x^2+1)^2}$ is positive, decreasing(check for yourself!), and continuous for all $x \geq 1$, and

\[
\int_1^\infty \frac{2x}{(x^2+1)^2} \d x = \lim_{b \to \infty} \eval{-\frac{1}{x^2+1}}_1^b = \lim_{b \to \infty} \eval{-\frac{1}{b^2+1}+\frac{1}{2}}.
\]
Hence, the improper integral converges, so $\sum_{k=1}^{\infty}  \frac{2k}{(k^2+1)^2}$ converges by the integral test.  

Suppose that we want to estimate the value of the series.  Since we do not have a way to find an explicit formula for $s_n=\sum_{k=1}^n \frac{2k}{(k^2+1)^2}$, we need to turn to the remainder estimate that comes with the integral test.

\begin{theorem}[Integral Remainder Estimates]
If $f(x)$ is a function that is positive, increasing, and continuous for $x \geq n_0$,  and $f(n) = a_n$ for every $n \geq n_0$, then 

\[
\int_{n+1}^{\infty} f(x) \d x \leq  r_n \leq \int_{n}^{\infty} f(x) \d x \qquad \textrm{ for all } n \geq n_0,
\]
where $r_n = \sum_{k=n+1}^{\infty} a_k$.
\end{theorem}

\end{feedback}


\begin{exercise}
Suppose we want to approximate $\sum_{k=1}^{\infty} \frac{2k}{(k^2+1)^2}$ as best as we can by using $\sum_{k=1}^{15} \frac{2k}{(k^2+1)^2}$.  

How will we find our approximation?

\begin{multipleChoice}
\choice{Our approximation is $\sum_{k=1}^{15} \frac{2k}{(k^2+1)^2}$.}
\choice[correct]{Our approximation will be made from $\sum_{k=1}^{N} \frac{2k}{(k^2+1)^2}$ and the upper and lower bounds for the error.}
\end{multipleChoice}

\textbf{First, we can determine an upper bound for the error from the inequality $ r_n \leq \int_{n}^{\infty} f(x) \d x$. } 

We can mimic the calculation done earlier

\[
\int_n^\infty \frac{2x}{(x^2+1)^2} \d x = \lim_{b \to \infty} \eval{\answer{-\frac{1}{x^2+1}}}_n^b = \lim_{b \to \infty} \eval{-\frac{1}{b^2+1}+\frac{1}{n^2+1}} =\answer{\frac{1}{n^2+1}}. 
\]

Since we want to approximate $\sum_{k=1}^{\infty} \frac{2k}{(k^2+1)^2}$ by using $\sum_{k=1}^{15} \frac{2k}{(k^2+1)^2}$, we should use $n=\answer{15}$, and doing so gives us that

\[
error \leq \answer{\frac{1}{15^2+1}} \approx .0044
\]

This tells us that $\sum_{k=1}^{15} \frac{2k}{(k^2+1)^2}$ will be no more than $.0044$ off from $\sum_{k=1}^{\infty} \frac{2k}{(k^2+1)^2}$.

\textbf{Secondly, we can determine a lower bound for the error from the inequality $ \int_{n+1}^{\infty} f(x) \d x  \leq r_n $. } 

We can mimic the calculation done earlier

\[
\int_{n+1}^\infty \frac{2x}{(x^2+1)^2} \d x = \lim_{b \to \infty} \eval{\answer{-\frac{1}{x^2+1}}}_{n+1}^b = \lim_{b \to \infty} \eval{-\frac{1}{b^2+1}+\frac{1}{(n+1)^2+1}} =\answer{\frac{1}{(n+1)^2+1}}. 
\]

Since we want to approximate $\sum_{k=1}^{\infty} \frac{2k}{(k^2+1)^2}$ by using $\sum_{k=1}^{15} \frac{2k}{(k^2+1)^2}$, we should use $n=\answer{15}$, and doing so gives us that

\[
error \geq \answer{\frac{1}{16^2+1}} \approx .0039
\]

This tells us that $\sum_{k=1}^{15} \frac{2k}{(k^2+1)^2}$ will be at least $.0039$ off from the exact value of  $\sum_{k=1}^{\infty} \frac{2k}{(k^2+1)^2}$.

\begin{exercise}
Now, we can use technology to compute that to four decimal places, $\sum_{k=1}^{15} \frac{2k}{(k^2+1)^2} = \answer[tolerance=.0002]{.7901}$.  This allows us to determine bounds for the value of $\sum_{k=1}^{\infty} \frac{2k}{(k^2+1)^2}$.

\[
\textrm{appx} + \textrm{lower error bound}  \leq \sum_{k=1}^{\infty} \frac{2k}{(k^2+1)^2}  \leq \textrm{appx} + \textrm{upper error bound} 
\]
\begin{align*}
.7901 +.0039 &\leq \sum_{k=1}^{\infty} \frac{2k}{(k^2+1)^2} \leq .7901 +.0044 \\
.7939 &\leq  \sum_{k=1}^{\infty} \frac{2k}{(k^2+1)^2} \leq .7945 
\end{align*}

\begin{exercise}
Now suppose that we want to ensure that we can approximate $\sum_{k=1}^{\infty} \frac{2k}{(k^2+1)^2}$ to within $.001$ of its exact value.  

To explore how the error  bounds sharpen our estimate, suppose that we find $N$ so $\sum_{k=1}^{N} \frac{2k}{(k^2+1)^2}$ is within $.001$ of $\sum_{k=1}^{\infty} \frac{2k}{(k^2+1)^2}$.  To do so, we will use the upper bound for error; that is, we will set $r_N \leq \int_N^\infty \frac{2x}{(x^2+1)^2} \d x \leq .001$.

From our previous work, we have found that

\[
 \int_N^\infty \frac{2x}{(x^2+1)^2} \d x= \answer{\frac{1}{N^2+1}},
\]

so we just have to solve the inequality below.

\begin{align*}
 \int_N^\infty \frac{2x}{(x^2+1)^2} \d x =  \answer{\frac{1}{N^2+1}} &\leq .001 \\
N^2+1 &\geq 1000 \\
N \geq \sqrt{\answer{999}}
\end{align*}

Using technology, we find $N \geq 31.6$, so we use $N=32$.

Our approximation is thus $\sum_{k=1}^{32} \frac{2k}{(k^2+1)^2} = \answer[tolerance=.002]{.7933}$

\begin{exercise}
Of course, there's a much more efficient way to estimate by using both error bounds.  Note that since the terms in the series are all positive, this estimate is must be an \wordChoice{\choice{overestimate}\choice[correct]{underestimate}}.  Had we used $N=32$ and repeat the construction in the previous part to find new error bounds for the actual value of the series, which gives us a much better idea of the value of the series!

The lower bound for the error
\[
\textrm{appx} + \textrm{lower error bound}  \leq \sum_{k=1}^{\infty} \frac{2k}{(k^2+1)^2}  \leq \textrm{appx} + \textrm{upper error bound} 
\]
\begin{align*}
.7933 +.0009 &\leq \sum_{k=1}^{\infty} \frac{2k}{(k^2+1)^2} \leq .7933 +.001 \\
.7942 &\leq  \sum_{k=1}^{\infty} \frac{2k}{(k^2+1)^2} \leq .7943 
\end{align*}

A much more computationally efficient is to require that the difference between the upper estimate for the series and the lower estimate be no more than $.001$; that is note that for every $n$,

\[
\int_{n+1}^\infty \frac{2x}{(x^2+1)^2} \d x  +s_n \leq \sum_{k=1}^{\infty} a_k \leq \int_{n}^\infty \frac{2x}{(x^2+1)^2} \d x  +s_n.
\]
 
If the difference between the left and right sides of the inequality is less than $.001$, then the infinite series must be within $.001$ of its minimum possible value (the lefthand side) and its maximum possible value (the righthand side).
 
 Now, note that

\begin{align*}
RHS-LHS &\leq .001 \\
\left( \int_{n}^\infty \frac{2x}{(x^2+1)^2} \d x  +s_n\right) - \left( \int_{n+1}^\infty \frac{2x}{(x^2+1)^2} \d x  +s_n \right) &\leq .001 \\
 \int_{n}^\infty \frac{2x}{(x^2+1)^2} \d x  - \int_{n+1}^\infty \frac{2x}{(x^2+1)^2} \d x   &\leq .001 \\
  \int_{n}^\infty \frac{2x}{(x^2+1)^2} \d x  +  \int_{\infty}^{n+1} \frac{2x}{(x^2+1)^2} \d x   &\leq .001 \\
    \int_{n}^{n+1} \frac{2x}{(x^2+1)^2}  &\leq .001 \\
\end{align*} 

Noting that $\int_N^{N+1}  \frac{2x}{(x^2+1)^2} \d x = \answer{\frac{1}{(N+1)^2+1} - \frac{1}{N^2+1}}$, we can find that the smallest integer value for $N$ for which $\int_N^{N+1}  \frac{2x}{(x^2+1)^2} \d x \leq .001$.  We simply must solve

\[
\frac{1}{(N+1)^2+1} - \frac{1}{N^2+1} \leq .001,
\]

and technology shows us that $N=13.06$, so we take is $N= \answer{14}$.  This is much better than the previous value found of $N=32$!
 
 We now know that the actual value for the series must be between $s_{14} + \int_{15}^\infty \frac{2x}{(x^2+1)^2} \d x = \answer[tolerance=.002]{.7934}$ and $s_{14} + \int_{14}^\infty \frac{2x}{(x^2+1)^2} \d x = \answer[tolerance=.002]{.7939}$

\end{exercise}
\end{exercise}
\end{exercise}








\end{exercise}
\end{exercise}
\end{document}