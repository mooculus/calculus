\documentclass{ximera}

\newcommand{\RR}{\mathbb R}
\renewcommand{\d}{\,d}
\newcommand{\dd}[2][]{\frac{d #1}{d #2}}
\renewcommand{\l}{\ell}
\newcommand{\ddx}{\frac{d}{dx}}
\newcommand{\dfn}{\textbf}
\newcommand{\eval}[1]{\bigg[ #1 \bigg]}


\author{Jim Fowler and Bart Snapp}

\outcome{Compute integrals in the plane by invoking an appropriate choice of Stokes' theorem}

\begin{document}

Let $C$ be the circle of radius $3$ centered at the origin, i.e.,
\[
  C = \{ (x,y) \in \R^2 : x^2 + y^2 = 3^2 \}
\]
and assume $C$ is oriented in the counterclockwise direction.

Further let $H$ be the ``northern semicircle'' meaning
\[
  H = \{ (x,y) \in C : y \geq 0 \}
\]
oriented compatibly with $C$.

\begin{exercise}
  Compute $\displaystyle\oint_C \left( 2x\d x - 2x \d y\right) = \answer{-18\pi}$

  \begin{hint}
    Let $D$ be the disk of radius $3$ centered at the origin, i.e.,
    \[
      D = \{ (x,y) \in \R^2 : x^2 + y^2 \leq 3^2 \}.
    \]
  \end{hint}
  \begin{hint}
    Then apply Green's theorem to deduce
    \[
      \oint_C \left( 2x\d x - 2x \d y\right) =
      \iint_D  - 2 \d x \d y
    \]
    which is $-2$ times the area of $D$.
  \end{hint}
  \begin{hint}
    The area of $D$ is $\pi \cdot 3^2$.
  \end{hint}
  \begin{hint}
    So the integral equals $-2 \cdot \pi \cdot 3^2 = -18\pi$.
  \end{hint}  
\end{exercise}

\begin{exercise}
  Compute $\displaystyle\oint_C \left( (1+2y)\d x + (2x-2y)\d y\right) = \answer{0}$.

  \begin{hint}
    Note that the field is $\vector{1+2y,2x-2y}$.
  \end{hint}

  \begin{hint}
    In other words, if $F : \R^2 \to \R$ is the function $F(x,y) = 2xy + x - y^2$, then $\grad F = \vector{1+2y, 2x-2y}$.
  \end{hint}

  \begin{hint}
    So by the fundamental theorem of line integrals, this integral vanishes.
  \end{hint}
\end{exercise}

\begin{exercise}
  $\displaystyle\int_H \left( (1+2y)\d x + (2x-2y)\d y\right) = \answer{-6}$
  
  \begin{hint}
    Note that the field is $\vector{1+2y,2x-2y}$. 
  \end{hint}
  
  \begin{hint}
    In other words, if $F : \R^2 \to \R$ is the function $F(x,y) = 2xy + x - y^2$, then $\grad F = \vector{1+2y, 2x-2y}$.
  \end{hint}
  
  \begin{hint}
    So by the fundamental theorem of line integrals, this integral equals the difference of the potential function $F$ at the endpoints.
  \end{hint}
  
  \begin{hint}
    The endpoints $\partial H$ consist of the points $(3,0)$ and $(-3,0)$.
  \end{hint}
    
  \begin{hint}
    So the integral equals $F(-3,0) - F(3,0) = -6$.
  \end{hint}
\end{exercise}

\begin{exercise}
  $\displaystyle\int_H \left( 2x\d x - 2x \d y\right) = \answer{-9\pi}$

  \begin{hint}
    Since $H$ doesn't bound a region, we can't apply Green's theorem directly.
  \end{hint}

  \begin{hint}
    One method to set $\vec{p}(t) = \left( 3 \cos t, 3 \sin t \right)$ which, for $0 \leq t \leq \pi$, traces out $H$.  Then the given integral can be appropriately transformed, e.g., $dx = -3 \sin t \d t$ and so forth.
  \end{hint}

  \begin{hint}
    There are other methods, though: consider the curve $\ell$ which is the straight line segment from $(3,0)$ to $(-3,0)$.  Then $H - \ell$ bounds the upper half-disk, with area $\frac{9}{2}\pi$.
  \end{hint}

  \begin{hint}
    Note that $\int_\ell \left( 2x\d x - 2x \d y\right) = \int_{x=3}^{-3} 2 x \d x = 0$.
  \end{hint}
  
  \begin{hint}
    Consequently the given integral equals the circulation around $H - \ell$, which, bounding the upper half-disk, is, by Green's theorem, equal to $-2 \cdot \frac{9}{2}\pi$.
  \end{hint}
\end{exercise}

\end{document}
