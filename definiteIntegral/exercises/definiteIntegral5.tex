\documentclass{ximera}

\newcommand{\RR}{\mathbb R}
\renewcommand{\d}{\,d}
\newcommand{\dd}[2][]{\frac{d #1}{d #2}}
\renewcommand{\l}{\ell}
\newcommand{\ddx}{\frac{d}{dx}}
\newcommand{\dfn}{\textbf}
\newcommand{\eval}[1]{\bigg[ #1 \bigg]}



%\outcome{Use integral notation for both antiderivatives and definite integrals.}
\outcome{Compute definite integrals using geometry.}
%\outcome{Compute definite integrals using the properties of integrals.}
%\outcome{Justify the properties of definite integrals using algebra or geometry.}
\outcome{Understand how Riemann sums are used to find exact area.}
%\outcome{Define net area.}
%\outcome{Approximate net area.}
%\outcome{Split the area under a curve into several pieces to aid with calculations.}
\outcome{Use symmetry to calculate definite integrals.}
\outcome{Explain geometrically why symmetry of a function simplifies calculation of some definite integrals.}

\author{Nela Lakos \and Kyle Parsons}

\begin{document}
\begin{exercise}

Evaluate the following limits of Riemann sums on the indicated intervals.
\begin{hint}
Recall: \text{On }$ [-2,20]$, $\lim_{n\to\infty}\sum_{k=1}^n x^* \Delta x=\int_{-2}^{20}x \d x$.
This integral gives you the net area of the region between the curve $y=x$ and the $x-$axis.

\begin{image}
  \begin{tikzpicture}
    \begin{axis}[
        xmin=-2.3,xmax=20.3,ymin=-2.3,ymax=20.3,
        clip=true,
        unit vector ratio*=1 1 1,
        axis lines=center,
        grid = major,
        ytick={2,4,...,20},
        xtick={2,4,...,20},
        xlabel=$x$, ylabel=$y$,
        every axis y label/.style={at=(current axis.above origin),anchor=south},
        every axis x label/.style={at=(current axis.right of origin),anchor=west},
        compat = 1.5.1
      ]
   
  \fill [fill=penColor,fill opacity=0.5] (axis cs:-2,0) -- (axis cs:-2,-2) -- (axis cs:0,0) -- (axis cs:20,20)-- (axis cs:20,0) -- cycle;
     
      \draw [thick,penColor] (axis cs:-2,-2) -- (axis cs:0,0) -- (axis cs:20,20);
              \node at (axis cs:9.5,14.5) {$f(x)$};
      \end{axis}`
  \end{tikzpicture}
\end{image}
\end{hint}
\begin{hint}
Express the limit as a definite integral. Then use properties of definite integrals, symmetry and geometry.
\end{hint}
\begin{align*}
\text{On } [-2,20],\,\lim_{n\to\infty}\sum_{k=1}^n x^* \Delta x &= \answer{198}\\
\text{On } [-2,2],\,\lim_{n\to\infty}\sum_{k=1}^n 6(x^*)^3 \Delta x &= \answer{0}\\
\text{On } [-7,7],\,\lim_{n\to\infty}\sum_{k=1}^n 6\sin(x^*) \Delta x &= \answer{0}\\
\text{On } [-7,7],\,\lim_{n\to\infty}\sum_{k=1}^n 6(x^*)^2\sin(x^*) \Delta x &= \answer{0}\\
\text{On } [4,7],\,\lim_{n\to\infty}\sum_{k=1}^n \sqrt{9-(x^*-4)^2} \Delta x &= \answer{\frac{9\pi}{4}}
\end{align*}

\end{exercise}
\end{document}