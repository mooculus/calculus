\documentclass{ximera}
\newcommand{\RR}{\mathbb R}
\renewcommand{\d}{\,d}
\newcommand{\dd}[2][]{\frac{d #1}{d #2}}
\renewcommand{\l}{\ell}
\newcommand{\ddx}{\frac{d}{dx}}
\newcommand{\dfn}{\textbf}
\newcommand{\eval}[1]{\bigg[ #1 \bigg]}

\author{Steven Gubkin}
\license{Creative Commons 3.0 By-NC}
\outcome{ Define a critical point.}
\outcome{ Find critical points.}
\outcome{ Define absolute maximum and absolute minimum.}
\outcome{ Find the absolute max or min of a continuous function on a closed interval.}
\outcome{ Define local maximum and local minimum.}
\outcome{ Compare and contrast local and absolute maxima and minima.}
\outcome{ Identify situations in which an absolute maximum or minimum is guaranteed.}
\outcome{ Classify critical points.}
\outcome{ State the First Derivative Test.}
\outcome{ Apply the First Derivative Test.}
\outcome{ State the Second Derivative Test.}
\outcome{ Apply the Second Derivative Test.}
\begin{document}

\begin{exercise}

The function $f(x) = x^4-4x^3+16x-3$ has two critical points. If we
call these critical points $a$ and $b$, and order them such that $a <
b$, then

$$
a = \answer{-1}
$$

$$
b=\answer{2}
$$

At $x=a$, the second derivative test
\begin{multipleChoice}
\choice{Indicates a local maxima}
\choice[correct]{Indicates a local minima}
\choice{Fails, but the First derivative test indicates a local max}
\choice{Fails, but the First derivative test indicates a local min}
\choice{Fails, and the First derivative test indicates that it is not a local extrema}
\end{multipleChoice}

At $x=b$, the second derivative test
\begin{multipleChoice}
\choice{Indicates a local maxima}
\choice{Indicates a local minima}
\choice{Fails, but the First derivative test indicates a local max}
\choice{Fails, but the First derivative test indicates a local min}
\choice[correct]{Fails, and the First derivative test indicates that it is not a local extrema}
\end{multipleChoice}

\end{exercise}
\end{document}

