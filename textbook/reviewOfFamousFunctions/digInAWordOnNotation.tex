\documentclass{ximera}

\newcommand{\RR}{\mathbb R}
\renewcommand{\d}{\,d}
\newcommand{\dd}[2][]{\frac{d #1}{d #2}}
\renewcommand{\l}{\ell}
\newcommand{\ddx}{\frac{d}{dx}}
\newcommand{\dfn}{\textbf}
\newcommand{\eval}[1]{\bigg[ #1 \bigg]}


\title{A word on notation}

\begin{document}
\begin{abstract}
  We discuss the notation used for functions.
\end{abstract}
\maketitle

Given a function $f$, we have a way of writing an inverse of $f$,
assuming it exists. Given a point $x$, 
\[
f^{-1}(x) = \text{$y$ such that $y = f(x)$, should it exist.}
\]
On the other hand, given $x$
\[
f(x)^{-1} = \frac{1}{f(x)}.
\]
\begin{warning}
It is not usually the case that 
\[
f^{-1}(x) = f(x)^{-1}.
\]
\end{warning}

This confusing notation is often exacerbated by the fact that 
\[
\sin^2(x) = (\sin(x))^2=\sin(x)\cdot \sin(x)\qquad \text{but} \qquad \sin^{-1}(x)
\ne(\sin(x))^{-1}.
\]

\begin{warning}
  Note that 
  \[
  \sin^{-1}(x)=\arcsin(x)\qquad\text{but}\qquad (\sin(x))^{-1} = \frac
  {1}{\sin(x)}.
  \]
  In the case of trigonometric functions, this confusion can be avoided
  by using the notation $\arcsin$ and so on for other trigonometric
  functions.
\end{warning}
\end{document}



