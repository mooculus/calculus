\documentclass{ximera}

\newcommand{\RR}{\mathbb R}
\renewcommand{\d}{\,d}
\newcommand{\dd}[2][]{\frac{d #1}{d #2}}
\renewcommand{\l}{\ell}
\newcommand{\ddx}{\frac{d}{dx}}
\newcommand{\dfn}{\textbf}
\newcommand{\eval}[1]{\bigg[ #1 \bigg]}


\author{Jim Talamo}
\license{Creative Commons 3.0 By-bC}
%Example from stewart

\outcome{}


\begin{document}

\begin{exercise}
The following exercise is meant to be done using technology to perform calculations.

Consider the function $f(x) = \frac{\sin(\tan(x))-\tan(\sin(x))}{\arcsin(\arctan(x))-\arctan(\arcsin(x))}$.

Use a calculator or a computer to complete the table below.  Round your answer to four decimal places.

\begin{tabular}{llll}
$f(.1) = \answer[tolerance=.001]{.9821}$ & $f(.01) = \answer[tolerance=.001]{2.0000}$ & $f(.001) = \answer[tolerance=.001]{3.3333}$ & $f(.0001) = \answer{3.3333}$ 
\end{tabular}

Try some other values near $x=0$.  What do you see?

Now, use a calculator or computational software of your choice to graph $y=f(x)$ (A good option is Desmos).  Does it look like $\lim_{x \to 0} f(x)$ exists?

\begin{multipleChoice}
\choice[correct]{I have read the above and drawn the graph.}
\choice{I have not drawn the graph yet.}
\end{multipleChoice}

\begin{exercise}
Use technology to compute $f'(x)$.  The result should be quite unpleasant.  Thankfully, technology is well-equipped to handle Taylor polynomials.  Using a program of your choice, write down the second degree Taylor Polynomial $p_2(x)$ centered at $x=0$ for $f(x)$.

\[
p_2(x) = \answer{1}+\answer{0}x+\answer{\frac{5}{3}}x^2
\]

From this, we see that $\lim_{x \to 0} f(x) = \answer{1}$.
\end{exercise}
\end{exercise}
\end{document}
