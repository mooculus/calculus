\documentclass{ximera}

\newcommand{\RR}{\mathbb R}
\renewcommand{\d}{\,d}
\newcommand{\dd}[2][]{\frac{d #1}{d #2}}
\renewcommand{\l}{\ell}
\newcommand{\ddx}{\frac{d}{dx}}
\newcommand{\dfn}{\textbf}
\newcommand{\eval}[1]{\bigg[ #1 \bigg]}


\title[Dig-In:]{Instantaneous velocity}

\begin{document}
\begin{abstract}
\end{abstract}
\maketitle

When one computes average velocity, we look at 
\[
\frac{\text{change in displacement}}{\text{change in time}}.
\]
To obtain the (instantaneous) velocity, we want the change in time to
``go to'' zero. By this point we should know that ``go to'' is a
buzz-word for a \textit{limit}. The change in time is often given as
an interval whose length goes to zero.  However, intervals must always
be written
\[
[a,b] \qquad\text{where $a < b$.}
\]


Given $I = [a, a+h]$, we see that $h$ cannot be
negative, or else it violates the notation for intervals. Hence, if we
want smaller, and smaller, intervals around a point $a$, and we want
$h$ to be able to be negative, we write
\[
I_h = 
\begin{cases}
  [a+h,a]  & \text{if $h<0$}, \\ %% note this is MORE correct than std books
  [a,a+h]  & \text{if $0<h$}.     %% in the content section, we can explain this in detail
\end{cases}
\]
\todo{An interactive for this interval would be nice.}
Regardless of the value of $h$, the average velocity is computed by
\[
\frac{\text{change in displacement}}{\text{change in time}} =
\frac{s(t+h)-s(t)}{h}.
\]

Let's put this together by working an example.

\begin{example}
The \textit{MOOCulus Team} recently took a road trip from Columbus
Ohio to Urbana-Champaign Illinois in the \textit{MOOCulus-Mobile}. The
position of the MOOCulus-Mobile is roughly modeled by
\[
s(t) = 36t^2 - 4.8t^3 \qquad\text{(miles West of Columbus)} %% note the model is wrong
\]
on the interval $[0,5]$, where $t$ is measured in hours. The average
velocity on the interval $[0,5]$ is \answer[given]{60} miles per hour. On the
other hand, consider the interval
\[
I_h = 
\begin{cases}
  [1+h,1]  & \text{if $h<0$}, \\ %% note this is MORE correct than std books
  [1,1+h]  & \text{if $0<h$}.     %% in the content section, we can explain this in detail
\end{cases}
\]
When $h = 0.1$, the average velocity is \answer[given]{59.712}. On the other
hand, when $h=-0.1$, the average velocity is \answer[given]{55.392}.
\end{example}

Now let's change gears a bit. Displacement and velocity are good
metaphors for concepts in calculus. 

\end{document}
