\documentclass{ximera}

\newcommand{\RR}{\mathbb R}
\renewcommand{\d}{\,d}
\newcommand{\dd}[2][]{\frac{d #1}{d #2}}
\renewcommand{\l}{\ell}
\newcommand{\ddx}{\frac{d}{dx}}
\newcommand{\dfn}{\textbf}
\newcommand{\eval}[1]{\bigg[ #1 \bigg]}


\outcome{Define linear approximation as an application of the tangent to a curve.}
\outcome{Find the linear approximation to a function at a point and use it to approximate the function value.}
%\outcome{Identify when a linear approximation can be used.}
%\outcome{Label a graph with the appropriate quantities used in linear approximation.}
%\outcome{Find the error of a linear approximation.}
%\outcome{Compute differentials.}
%\outcome{Use the second derivative to discuss whether the linear approximation over or underestimates the actual function value.}
%\outcome{Contrast the notation and meaning of \d{y} versus \Delta y.}
%\outcome{Understand that the error shrinks faster than the displacement in the input.}
%\outcome{Justify the chain rule via the composition of linear approximations.}

\author{Nela Lakos \and Kyle Parsons \and Bobby Ramsey}

\begin{document}
\begin{exercise}

Consider the function $f(x) = e^{2x}$.

The linear approximation of $f$ at $a=0$ is 
\[
L(x) = \answer{1+2x }.
\]

Using this linear approximation, approximate the value of $e$.
\begin{hint}
We have to express $e$ as $f(x)$, for some value of $x$.
\end{hint}
\begin{hint}
It follows that $e=e^1=e^{2(\frac{1}{2})}=f\left(\frac{1}{2}\right)$.
\end{hint}
\begin{hint}
It follows that $e=f\left(\frac{1}{2}\right)\approx L(\frac{1}{2})$.
\end{hint}
\[
e \approx \answer{2}.
\]

This is an
\wordChoice{\choice[correct]{underestimate}\choice{overestimate}}
because the graph of $f$ is \wordChoice{\choice[correct]{Concave Up}\choice{Concave Down}}
on the interval $\left( \answer{0}, \answer{0.5} \right)$.

\end{exercise}
\end{document}