\documentclass{ximera}

\author{Bart Snapp}

\begin{document}

\textbf{For problems 11--13,} use the table below describing the
derivatives of a differentiable function $F:\R^2\to\R$:
\[\def\arraystretch{1.5}
\begin{array}{|l||c|c|c|c|}\hline
  \text{point} & \grad F          & \frac{\partial^2 F}{\partial x^2} & \frac{\partial^2 F}{\partial y^2}  & \frac{\partial^2 F}{\partial x\partial y}\\\hline\hline
  (-1,4)        & \vector{0,0}     &        4                        &             ??              &  5 \\
  (2,3)        & \vector{0,0}   &       ??                       &            2                  &  6      \\
  (1,2)        & \vector{??,??}   &        -2                       &            -3                &  ?? \\
  \hline
\end{array}
\]
\begin{problem}
  What is the \textbf{smallest} $\frac{\partial^2 F}{\partial y^2}$
  could be that would guarentee that $F$ has a \textbf{local extremum} at
  $(-1,4)$?
  \begin{prompt}
    \[
    \answer{25/4}<\frac{\partial^2 F}{\partial y}
    \]
  \end{prompt}
  \vfill
\end{problem}

\begin{problem}
  Supposing that $F$ has a local extremum at $(-1,4)$, is this extremum a \textbf{maximum} or a \textbf{minimum}?
  \begin{prompt}
    \begin{multipleChoice}
      \choice{Maximum}
      \choice[correct]{Minimum}
    \end{multipleChoice}
  \end{prompt}
  \vfill
\end{problem}


\begin{problem}
  What is the \textbf{largest} $\frac{\partial^2 F}{\partial x^2}$ could be that
  would guarentee that $F$ has a \textbf{saddle point} at $(2,3)$?
  \begin{prompt}
    \[
    \frac{\partial^2 F}{\partial x^2} < \answer{0}
    \]
  \end{prompt}
\vfill
\end{problem}


%% \begin{problem}
%%   Let $G(x,y) = F(x,y) +3x-2y$
%% \end{problem}


\end{document}
