\documentclass{ximera}

\newcommand{\RR}{\mathbb R}
\renewcommand{\d}{\,d}
\newcommand{\dd}[2][]{\frac{d #1}{d #2}}
\renewcommand{\l}{\ell}
\newcommand{\ddx}{\frac{d}{dx}}
\newcommand{\dfn}{\textbf}
\newcommand{\eval}[1]{\bigg[ #1 \bigg]}


\title{Exponential and Logarithmic Functions}

\begin{document}
\begin{abstract}
	These kinds of functions enjoy many interesting properties
\end{abstract}

\maketitle

\begin{definition}
	If $a$ is a real number, and $n$ is a positive integer, $a^n$ is defined to be $a \cdot a \cdot ... \cdot a$, where there are $n$ factors of $a$ being multiplied together.  In the expression $a^n$, $a$ is called the \textbf{base} and $n$ is called the \textbf{exponent}.
\end{definition}

\begin{question}
	The base of $7^9$ is $\answer{7}$
\end{question}

\begin{question}
	The exponent of $4^{10}$ is $\answer{10}$
\end{question}

\begin{question}
	$2^4 = \answer{16}$
\end{question}

\begin{question}
	$100^1= \answer{100}$
\end{question}

There are a couple cool properties of exponents:

\begin{question}
	\begin{hint}
		$2^4 \cdot 2^3 = (2 \cdot 2\cdot 2 \cdot 2) \cdot  (2 \cdot 2\cdot 2) = 2^7 $
	\end{hint}

	$2^4 \cdot 2^3 = 2^{\answer{7}}$
\end{question}

The same reasoning leads to the following rule of exponents:

\begin{theorem}
	If $a$ is a real number, and $n$ and $m$ are positive integers, then $a^n \cdot a^m = a^{n+m}$.
\end{theorem}

The reason for this rule is just that $a^n$ has $n$ factors of $a$, and $a^m$ has $m$ factors of $a$, so their product has $n+m$ factors of $a$.

\begin{question}
	Which of the following are equal to $5^4$?
	\begin{selectAll}
		\choice[correct]{$5 \cdot 5^3$}
		\choice{$4 \cdot 4 \cdot 4 \cdot 4 \cdot 4$}
		\choice{$20$}
		\choice[correct]{$5^2 \cdot 5^2$}
		\choice[correct]{$(5^2)^2$}
	\end{selectAll}
\end{question}

\begin{question}
	Which of the following are equal to $(7^3)^2$?
	\begin{selectAll}
		\choice{$7^5$}
		\choice[correct]{$(7^3) \cdot (7^3)$}
		\choice{$7^6$}
		\choice{$7\cdot 7 \cdot 7 \cdot 7 \cdot 7$}
	\end{selectAll}
\end{question}

\begin{theorem}
	If $a$ is a real number, and $n$ and $m$ are positive integers, then $(a^n)^m = a^{nm}$.
\end{theorem}



This makes sense, since $(a^n)^m$ means you have $m$ factors of $a^n$.  Each of those has $n$ factors of $a$, so you have $m$ groups of $n$ factors of $a$, for a total of $nm$ factors of $a$.

\begin{question}
	Which of the following statements is true?
	\begin{multipleChoice}
		\choice{$(2^3)^{100} = 2^{(3^{100})}$}
		\choice{$(2^3)^{100} > 2^{(3^{100})}$}
		\choice[correct]{$(2^3)^{100} < 2^{(3^{100})}$}
	\end{multipleChoice}
\end{question}

So far we have only talked about exponentials where the exponent is a positive integer.  We do not yet have any meaning for an expression like $5^{2.6}$.  What could that expression mean?  $2.6$ factors of $5$ does not make much sense.  The same could be said for negative exponents.  $5^{-3}$ does not make sense when interpreted as negative $3$ factors of $5$.  It turns out that it is useful to define exponents for these kinds of numbers so that the two rules we learned still hold true.

\begin{question}
	We are going to try to find a reasonable value for $a^0$ which satisfies all of the rules we have learned so far.
	
If we still want $a^{n+m} = a^n \cdot a^m$, then we must have $a^{0+m} = a^0\cdot a^m$.  It follows that $a^m = a^0 \cdot a^m$.  So what must the value of $a^0$ be?

\[a^0 = \answer{1}\]
%\feedback{No matter what the base, $a^0$ is always $1$, since this is the only value which makes the sum %law hold. Note that this is an aesthetic choice.  We had no good definition of the value of the exponential %expression before, but we are making a choice about our definitions which makes a certain pattern continue %to hold.} 

\end{question}

This justifies the following definition:

\begin{definition}
For any positive real number $a$, we define $a^0 = 1$.
\end{definition}

\begin{question}
	Now we are going to try to figure out what a reasonable definition of a negative exponent would be.   Let us try to assign a nice value to $2^{-3}$. 
	
	Let $x = 2^{-3}$.

	Then $2^3 x = 2^{-3} 2^3$
	
	so $2^3 x = \answer{1}$
	
	Thus $x = \frac{1}{\answer{2^3}}$.
\end{question}

This justifies the following definition:

\begin{definition}
	If $a$ is a positive real number and $m$ is a positive integer, we define $a^{-m} = \frac{1}{a^m}$
\end{definition}

\begin{question}
	Which of the following are true?  Mark all that apply.
	\begin{selectAll}
		\choice[correct]{$4^7 = (\frac{1}{4})^{-7}$}
		\choice{$(-a)^{-b} = a^b$}
		\choice{If $n<m$ then $a^n < a^m$}
		\choice[correct]{$(\frac{1}{2})^3 < (\frac{1}{2})^2$}
		\choice[correct]{$\frac{a^n}{a^m} = a^{n-m}$}
	\end{selectAll}
\end{question}

\begin{question}
Now we are going to try to figure out what a reasonable definition of a rational exponent would be.  Let us try to assign a value to $7^\frac{1}{2}$ which is consistent with the rules for exponents we have learned so far.

Let $x = 7^\frac{1}{2}$.

Then $x^2 = 7^\frac{1}{2} \cdot 7^\frac{1}{2}$.

Assuming that the exponents still add, we have

$x^2 =  \answer{7}$.

so $x = \answer{\sqrt{7}}$
\end{question}


This line of thought justifies the following definition:

\begin{definition}
If $a$ is a positive real number, and $n$ and $m$ are positive integers, then we define $a^{\frac{n}{m}}$ to be $\sqrt[m]{a^n}$.
\end{definition}

\begin{question}
	 $125^{-\frac{2}{3}} = \answer{1/25}$
\end{question}

We now have a definition for $a^r$ where $a$ is any positive real number, and $r$ is any rational number.  We still do not have a definition of exponentiation when the exponent is an irrational number.  For instance, at this stage $2^\pi$ a meaningless expression.  

We will solve the problem of extending the definition of exponentiation to real exponents in many different ways in this course.  We cannot do so until we have covered limits.  We will find a definition using limits, another using differential equations,  another using definite integrals, and another using power series.

While we cannot truly make sense of an expression like $2^\pi$ right now, we can at least get an idea.  $\pi = 3.1415926...$.  We can interpret $2^3 = 8 $,  $2^{3.1} = \sqrt[10]{2^{31}} \approx 8.57$, $2^{3.14} = \sqrt[100]{2^{314}} \approx 8.81$, ...
When we keep going, we get numbers closer and closer to $8.82497782708...$.  

In essence, we define the value of the exponential function for irrational exponents by approximating those irrational exponents by rational ones.  This is the idea of taking a limit, which is a big part of what calculus is all about.

Since the exponential functions are now defined for all real numbers, we can look at some graphs of them.

\begin{question}
	The graphs of $4$ functions are pictured below.  Match them with their formulas.
	
	\begin{image}
\begin{tikzpicture}
	\begin{axis}[
            domain=-2:2,
            xmin=-2, xmax=2,
            ymin=-2, ymax=4,
            axis lines =middle, xlabel=$x$, ylabel=$y$,
            every axis y label/.style={at=(current axis.above origin),anchor=south},
            every axis x label/.style={at=(current axis.right of origin),anchor=west},
          ]
	  \addplot [very thick, penColor, smooth] {3^x};
          \addplot [very thick, penColor2, smooth] {2^x)};
          \addplot [very thick, penColor3, smooth] {(1/2)^x)};
          \addplot [very thick, penColor4, smooth] {(1/3)^x)};




          \node at (axis cs:-1.5, 2 ) [penColor3,anchor=west] {$A$};
          \node at (axis cs:-1, 3.5 ) [penColor4,anchor=west] {$B$};
          \node at (axis cs:0.8, 3.5 ) [penColor,anchor=west] {$C$};
          \node at (axis cs:1.2, 2 ) [penColor2,anchor=west] {$D$};

        \end{axis}
\end{tikzpicture}
\end{image} 

Write the capital letter associated with the graph of each function in the appropriate box.

\begin{align*}
3^x &=  \answer{C}\\
\left(\frac{1}{2}\right)^{-x} &= \answer{D}\\
\left(\frac{1}{3}\right)^{x} &=  \answer{B}\\
 2^{-x} &= \answer{A}
\end{align*}

\end{question}

\begin{summary}
	For each positive real number $a$, we now have (sort of) defined a function $f(x) = a^x$, the exponential function with base $a$, which is defined for all real numbers $x$.  For positive integers, it agrees with the repeated multiplication definition, and for non-positive integers and rationals it is defined in a way that it continues to enjoy the nice algebraic properties ($a^{b+c} = a^b \cdot a^c$ and $(a^b)^c = a^{bc}$) we expect of it.  For irrational numbers we do not yet have a solid definition, but we will learn some in the next chapters.  Basically the values at the irrationals are determined by approximating the irrationals with rationals.  
\end{summary}

\end{document}