\documentclass{ximera}

\newcommand{\RR}{\mathbb R}
\renewcommand{\d}{\,d}
\newcommand{\dd}[2][]{\frac{d #1}{d #2}}
\renewcommand{\l}{\ell}
\newcommand{\ddx}{\frac{d}{dx}}
\newcommand{\dfn}{\textbf}
\newcommand{\eval}[1]{\bigg[ #1 \bigg]}


\begin{document}
\begin{exercise}

\outcome{Discuss what it means for a limit to equal $\infty$.}
\outcome{Understand the relationship between limits and vertical asymptotes.}
\outcome{Evaluate the limit as $x$ approaches a point where there is a vertical asymptote.}
\outcome{Recognize when a limit is indicating there is a vertical asymptote.}

Let 
\[
f(x)=\begin{cases}
\frac{x^2-x-12}{x+3} & \text{if $x<4$ and $x\ne -3$}\\
5 & \text{if $x=-3$}\\
\frac{x}{x-4} & \text{if $x>4$}\ldotp
\end{cases}
\]
\begin{enumerate}[label=\bf{(\Roman*)},align=left]
\item Determine if the following limits exist. If they exist, compute them analytically using the limit laws and techniques for computing limits. If a limit does not exist, writen `DNE' and explain why. 

\noindent[Do not use a table of values, a graph, or L'h\^opital's rule to justify your answer, for instance.]
\begin{enumerate}[label=\bf{(\alph*)}]
\item $\lim_{x\to -3}f(x)=\answer{-7}$
\item $\lim_{x\to 4^{-}}f(x)=\answer{0}$
\item $\lim_{x\to 4^+}f(x)=\answer{\infty}$ %% the limit exists in the extended reals, at least. Based on other exercises we have, I decided $\infty$ was better than DNE 
\item $\lim_{x\to 4}f(x)=\answer{DNE}$
\end{enumerate}
\item List the vertical asymptotes of $f$ from least to greatest: $\answer{4}$
\end{enumerate}

\end{exercise}
\end{document}

