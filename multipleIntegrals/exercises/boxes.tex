\documentclass{ximera}

\newcommand{\RR}{\mathbb R}
\renewcommand{\d}{\,d}
\newcommand{\dd}[2][]{\frac{d #1}{d #2}}
\renewcommand{\l}{\ell}
\newcommand{\ddx}{\frac{d}{dx}}
\newcommand{\dfn}{\textbf}
\newcommand{\eval}[1]{\bigg[ #1 \bigg]}


\author{Jim Fowler}

\outcome{Integrate a function defined on a rectangle}

\begin{document}

\begin{exercise}
  Let:
  \begin{align*}
    B &= \{(x,y):0\le x\le 5, 2\le y\le 3\}\\
    L &= \{(x,y):0\le x\le 2, 2\le y\le 3\}\\
    R &= \{(x,y):2\le x\le 5, 2\le y\le 3\}
  \end{align*}
  The area of $L$ is $\answer{2}$ square units, so $\iint_L 1
  \d A = \answer{2}$.  What is $\iint_L 2 \d A$?
  \begin{prompt}
    \[
      \iint_L 2 \d A = \answer{4}.
    \]
  \end{prompt}    

  Similarly the area of $R$ is $\answer{3}$ square units, so $\iint_R 1 \d A = \answer{3}$.  
  What is $\iint_R 3 \d A$?
  \begin{prompt}
    \[
      \iint_R 3 \d A = \answer{9}.
    \]
  \end{prompt}
  Now suppose $F : \R^2 \to \R$ is the function given by the rule
  \[
    F(x,y) = \begin{cases}
      2 & \mbox{ if $x \leq 2$} \\
      3 & \mbox{ otherwise.}
    \end{cases}
    \]
    Note that $\iint_B F \d A = \iint_L F \d A + \iint_R F \d A$.
    What is $\iint_B F \d A$?
    \begin{prompt}
    \[
    \iint_B F \d A = \answer{13}.
    \]
    \end{prompt}
    
\end{exercise}

\end{document}
