\documentclass{ximera}

\newcommand{\RR}{\mathbb R}
\renewcommand{\d}{\,d}
\newcommand{\dd}[2][]{\frac{d #1}{d #2}}
\renewcommand{\l}{\ell}
\newcommand{\ddx}{\frac{d}{dx}}
\newcommand{\dfn}{\textbf}
\newcommand{\eval}[1]{\bigg[ #1 \bigg]}


\outcome{Understand the graph of a function}
%\outcome{Understand what information the second derivative gives concerning concavity of a function.}
%\outcome{Interpret limits as giving information about functions.}
%\outcome{Determine how the graph of a function looks based on an analytic description of the function.}

\author{Nela Lakos \and Kyle Parsons}

\begin{document}
\begin{exercise}

 The graph of a function $g$ is given below.

\begin{image}
  \begin{tikzpicture}
    \begin{axis}[
        xmin=-1.3,xmax=6.3,ymin=-2.3,ymax=2.3,
        clip=true,
        unit vector ratio*=1 1 1,
        axis lines=center,
        grid = major,
        ytick={-10,-9,...,10},
    	xtick={-10,-9,...,10},
        xlabel=$x$, ylabel=$y$,
        every axis y label/.style={at=(current axis.above origin),anchor=south},
        every axis x label/.style={at=(current axis.right of origin),anchor=west},
      ]
      \addplot[very thick,penColor,domain=-1:1] plot{-2*x^3};
      \addplot[very thick,penColor,domain=1:3] plot{sqrt(6-2*x)};
      \addplot[very thick,penColor,domain=3:5] plot{sqrt((x-3)/2)};
      \addplot[very thick,penColor,domain=5:6] plot{(6-x)^2};
      
      \addplot[color=penColor,fill=white,only marks,mark=*] coordinates{(-1,2) (1,-2)  (6,0)};
        \addplot[color=penColor,fill=blue,only marks,mark=*] coordinates{ (1,2) };
      
      \node at (axis cs:2.5,-1.5) {$y=g(x)$};
      \end{axis}`
  \end{tikzpicture}
\end{image}
Based on the graph of $g$, answer  the question below.\\

$g$ has critical points at the $x$ values (from left to right) $\answer{0}$, $\answer{1}$, $\answer{3}$ and $\answer{5}$.


$g$ has critical points where $g'(x)=0$ at the $x=\answer{0}$.\\ 


$g$ has critical points where $g'(x)$ is \textbf{undefined} at the $x$ values (from left to right) $\answer{1}$, $\answer{3}$ and $\answer{5}$.


$g$ has a local max at $x=\answer{5}$ .

$g$ has a local min at $x=\answer{3}$.

$g$ is decreasing and concave down on $\left(\answer{0},\answer{1}\right)$ and  $\left(\answer{1},\answer{3}\right)$.

$g$ is increasing and concave down on $\left(\answer{3},\answer{5}\right)$.

$g$ is decreasing and concave up on $\left(\answer{-1},\answer{0}\right)$ and  $\left(\answer{5},\answer{6}\right)$.

$g$ has an inflection points at the $x$ values (from left to right) $\answer{0}$ and $\answer{5}$.

\end{exercise}
\end{document}
