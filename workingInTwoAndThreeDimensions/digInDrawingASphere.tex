\documentclass{ximera}

\newcommand{\RR}{\mathbb R}
\renewcommand{\d}{\,d}
\newcommand{\dd}[2][]{\frac{d #1}{d #2}}
\renewcommand{\l}{\ell}
\newcommand{\ddx}{\frac{d}{dx}}
\newcommand{\dfn}{\textbf}
\newcommand{\eval}[1]{\bigg[ #1 \bigg]}


\author{Bart Snapp}

\outcome{Give the equation for a sphere or ball.}

\title[Dig-In:]{Drawing a sphere}

\begin{document}
\begin{abstract}
  Learn how to draw a sphere.
\end{abstract}
\maketitle

A key challenge in mathematics is converting formulas and equations
into ideas. We want to get to the point that when you see something
like
\[
S = \{(x,y,z):(x-1)^2+(y-2)^2+(z-3)^2=16\}
\]
you say to yourself, ``Hey, that's a sphere of radius $4$ centered at
the point $(1,2,3)$. Let's state this generally.
\begin{theorem}
  The set of points on the outside of a sphere of radius $r$
  centered at $(a,b,c)$ are given by the set:
  \[
  S = \{(x,y,z):(x-a)^2+(y-b)^2+(z-c)^2=r^2\}
  \]
  The set of points forming the filled-in ball of radius $r$ centered
  at $(a,b,c)$ are given by the set:
  \[
  B = \{(x,y,z):(x-a)^2+(y-b)^2+(z-c)^2\le r^2\}
  \]
\end{theorem}
Above we give implicit formulas. Sometimes it is better to have
parametric formulas.
\begin{theorem}
  The parametric formula for a sphere of radius $r$ centered at
  $(a,b,c)$ is give by:
  \begin{align*}
    x(\phi,\theta) &=r\cdot\cos(\theta)\sin(\phi)\\
    y(\phi,\theta) &=r\cdot\sin(\theta)\sin(\phi)\\
    z(\phi,\theta) &=r\cdot\cos(\phi)
  \end{align*}
  for $0\le \phi\le \pi$ and $0\le\theta< 2\pi$.

  If you want a parametric formula for the filled-in ball of radius
  $r$, you write:
  \begin{align*}
    x(\rho,\phi,\theta) &=\rho\cdot\cos(\theta)\sin(\phi)\\
    y(\rho,\phi,\theta) &=\rho\cdot\sin(\theta)\sin(\phi)\\
    z(\rho,\phi,\theta) &=\rho\cdot\cos(\phi)
  \end{align*}
  where  $0\le \phi\le \pi$, $0\le\theta< 2\pi$, and $0\le \rho\le r$.
\end{theorem}



One thing to note above is that in some sense, the parametric formula
for the sphere ``draws'' the sphere. Now let me show you how to draw a
sphere yourself. Start by drawing a set of axes.
\begin{image}
  \begin{tikzpicture}  
    \begin{axis}[  
        xmin=-1.2,  
        xmax=1.2,  
        ymin=-1.2,  
        ymax=1.2,
        unit vector ratio=1 1 1,
        axis lines=center,
        ticks=none,
        xlabel=$y$,  
        ylabel=$z$,  
        every axis y label/.style={at=(current axis.above origin),anchor=south},  
        every axis x label/.style={at=(current axis.right of origin),anchor=west},  
      ]  
      \addplot [->] coordinates {(.7,.5) (-.7,-.5)};
      %\addplot [ultra thick, penColor,domain=0:360,smooth] ({cos(x)},{sin(x)});
      %\addplot [ultra thick, penColor,domain=180:360,smooth] ({cos(x)},{.4*sin(x)});
      %\addplot [ultra thick, dashed, penColor,domain=0:180,smooth] ({cos(x)},{.4*sin(x)});
      \node at (axis cs: -.75,-.52) {$x$};
    \end{axis}  
  \end{tikzpicture}  
\end{image}
Now draw a circle in the $(y,z)$-plane:
\begin{image}
  \begin{tikzpicture}  
    \begin{axis}[  
        xmin=-1.2,  
        xmax=1.2,  
        ymin=-1.2,  
        ymax=1.2,
        unit vector ratio=1 1 1,
        axis lines=center,
        ticks=none,
        xlabel=$y$,  
        ylabel=$z$,  
        every axis y label/.style={at=(current axis.above origin),anchor=south},  
        every axis x label/.style={at=(current axis.right of origin),anchor=west},  
      ]  
      \addplot [->] coordinates {(.7,.5) (-.7,-.5)};
      \addplot [ultra thick, penColor,domain=0:360,smooth] ({cos(x)},{sin(x)});
      %\addplot [ultra thick, penColor,domain=180:360,smooth] ({cos(x)},{.4*sin(x)});
      %\addplot [ultra thick, dashed, penColor,domain=0:180,smooth] ({cos(x)},{.4*sin(x)});
      \node at (axis cs: -.75,-.52) {$x$};
    \end{axis}  
  \end{tikzpicture}  
\end{image}
Now draw an ellipse, dashing the part at the ``back'' of the sphere:
\begin{image}
  \begin{tikzpicture}  
    \begin{axis}[  
        xmin=-1.2,  
        xmax=1.2,  
        ymin=-1.2,  
        ymax=1.2,
        unit vector ratio=1 1 1,
        axis lines=center,
        ticks=none,
        xlabel=$y$,  
        ylabel=$z$,  
        every axis y label/.style={at=(current axis.above origin),anchor=south},  
        every axis x label/.style={at=(current axis.right of origin),anchor=west},  
      ]  
      \addplot [->] coordinates {(.7,.5) (-.7,-.5)};
      \addplot [ultra thick, penColor,domain=0:360,smooth] ({cos(x)},{sin(x)});
      \addplot [ultra thick, penColor,domain=180:360,smooth] ({cos(x)},{.4*sin(x)});
      \addplot [ultra thick, dashed, penColor,domain=0:180,smooth] ({cos(x)},{.4*sin(x)});
      \node at (axis cs: -.75,-.52) {$x$};
    \end{axis}  
  \end{tikzpicture}  
\end{image}
And voli\`a, we have a sphere! Now let me tell you something: People who like mathematics really like
asking questions like the following:

\begin{example}
  Verify that the two descriptions
  \[
  S = \{(x,y,z):(x-a)^2+(y-b)^2+(z-c)^2=r^2\}
  \]
  and
  \begin{align*}
    x(\phi,\theta) &=r\cdot\cos(\theta)\sin(\phi)\\
    y(\phi,\theta) &=r\cdot\sin(\theta)\sin(\phi)\\
    z(\phi,\theta) &=r\cdot\cos(\phi)
  \end{align*}
  where $0\le \phi\le \pi$ and $0\le \theta\le 2\pi$ describe the same geometric set.
  \begin{explanation}
    FILL IN - NEEDS TO BE WRITTEN
  \end{explanation}
\end{example}


\end{document}
