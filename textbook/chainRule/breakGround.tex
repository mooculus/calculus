\documentclass{ximera}

\newcommand{\RR}{\mathbb R}
\renewcommand{\d}{\,d}
\newcommand{\dd}[2][]{\frac{d #1}{d #2}}
\renewcommand{\l}{\ell}
\newcommand{\ddx}{\frac{d}{dx}}
\newcommand{\dfn}{\textbf}
\newcommand{\eval}[1]{\bigg[ #1 \bigg]}


\outcome{}

% BADBAD

\title[Break-Ground:]{Chain rule}

\begin{document}
\begin{abstract}
A dialogue where students discuss the chain rule.
\end{abstract}
\maketitle

\begin{dialogue}
\item[Devyn] Remember when we proved that the derivative of $\sin$ is $\cos$
\item[Riley] Ya, I think I was awake for that lecture.  We had to draw a bunch of weird pictures to get some limit?
\item[Devyn]  Ya.  I was not quite convinced by all of that.  Seemed a little bit too hand-wavy for me.  I did some experiments on my calculator to double check myself, and I do not get the same answer.
\item[Riley] Are you seriously questioning that $\frac{d}{dx} \sin(x) = \cos(x)$?  I mean hundreds of thousands of mathematicians must have checked this. 
\item[Devyn] Just hear me out.  The derivative of $\sin(x)$ at $x =0$ should be approximated by $\frac{\sin(0.001) - \sin(0)}{ 0.001 - 0}$, right?
\item[Riley] Ya.  That is the slope of a secant line, which should be close to the slope of the tangent line.
\item[Devyn] And if $\frac{d}{dx} \sin(x) = \cos(x)$, then this should be pretty close to $\cos(0) = 1$ right?
\item[Riley] For sure.  This is that limit $\lim_{x \to 0} \frac{\sin(x)}{x} = 1$ that came up in the proof that the derivative of $\sin$ is $\cos$.
\item[Devyn] The problem is, when I put this in my calculator I get $\frac{\sin(0.001)}{0.001} \approx 0.017453$.
\item[Riley] Whao.  That cannot be right.  Let me try it on my calculator...  I get $0.99999983$, which seems pretty close to $1$.  Maybe your calculator is in degree mode instead of radian mode?  I know my TA is always warning us that we need it to be in radian mode for calculus.
\item[Devyn] Ha, you are right.  I get the same answer as you now.  Does this mean that the derivative of sine in degrees is different from the derivative of sine in radians?
\item[Riley] I guess so, but isn't it the same function?
\item[Devyn] Well, to convert from degrees to radians, we have to multiply the degree measurement by $\frac{\pi}{180}$
\item[Riley] and $\frac{\pi}{180} \approx 0.017453$!
\item[Devyn] Hmm.  So it seems like changing units affects the derivative of a function by scaling the slope.  That sort of makes sense.
\item[Riley] I wonder if there are any rules of differentiation which might shed more light on this. 
\end{dialogue}

\begin{question}

\end{question}


%% \begin{xarmaBoost}
%%   Write down at least \textbf{five} questions for this lecture. After
%%   you have your questions, label them as ``Level 1,'' ``Level 2,'' or
%%   ``Level 3'' where:
%% \begin{description}
%% \item[Level 1] Means you know the answer, or know exactly how to do
%%   this problem.
%% \item[Level 2] Means you think you know how to do the problem.
%% \item[Level 3] Means you have no idea how to do the problem.
%% \end{description}
%% \begin{freeResponse}
%% \end{freeResponse}
%% \end{xarmaBoost}



\end{document}
