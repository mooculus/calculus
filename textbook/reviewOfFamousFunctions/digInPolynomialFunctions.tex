\documentclass{ximera}

\newcommand{\RR}{\mathbb R}
\renewcommand{\d}{\,d}
\newcommand{\dd}[2][]{\frac{d #1}{d #2}}
\renewcommand{\l}{\ell}
\newcommand{\ddx}{\frac{d}{dx}}
\newcommand{\dfn}{\textbf}
\newcommand{\eval}[1]{\bigg[ #1 \bigg]}


\title[Dig-In:]{Polynomial functions}


\begin{document}
\begin{abstract}
  Polynomials are some of our favorite functions. 
\end{abstract}
\maketitle


The functions you are most familiar with are probably polynomial
functions.

\section{What are polynomial functions?}

\begin{definition}
  A \dfn{polynomial function} in the variable $x$ is a function
  of the form
  \[
  f(x) = a_nx^n + a_{n-1}x^{n-1} + \dots + a_1 x + a_0
  \]
  where the $a_i$'s are all constants (called the \dfn{coefficents})
  and $n$ is a whole number (called the \dfn{degree} when $n\ne
  0$). The domain of a polynomial function is $(-\infty,\infty)$.
\end{definition}

\begin{question}
  Which of the following are polynomial functions?
  \begin{multipleChoice}
    \choice[correct]{$f(x) = 0$}
    \choice[correct]{$f(x) = -9$}
    \choice[correct]{$f(x) = 3x+1$}
    \choice{$f(x) = x^{1/2}-x +8$}
    \choice{$f(x) = \frac{x^2 - 3x + 2}{x-2}$}
    \choice[correct]{$f(x) = x^7-32x^6-\pi x^3+45/84$}
  \end{multipleChoice}
\end{question}

The phrase above ``in the variable $x$'' can acutally change.
\[
y^2-4y +1
\]
is a polynomial in $y$, and
\[
\sin^2(x) + \sin(x) -3 
\]
is a polynomial in $\sin(x)$.


\section{What can the graphs look like?}

Fun fact:

\begin{theorem}[The Fundamental Theorem of Algebra]
  Every polynomial of the form
  \[
  a_n x^n + a_{n-1} x^{n-1} + \dots + a_1 x + a_0
  \]
  where the $a_i$'s are real (or even complex!) numbers has exactly
  $n$ (possibly repeated) complex roots.
\end{theorem}

Remember, a \dfn{root} is where a polynomial is zero. The theorem
above is a deep fact of mathematics. The great mathematician Gauss
(spelled Gau\ss\ for fancy people) proved the theorem in 1799 for his
doctoral thesis. 

The upshot as far as we are concerned is that when we plot a
polynomial of degree $n$, its graph will cross the $x$-axis at most
$n$ times.





\section{Connections to inverse functions}



x^even vs x^odd and simple nth roots



\end{document}
