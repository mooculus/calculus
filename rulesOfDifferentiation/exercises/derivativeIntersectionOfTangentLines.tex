\documentclass{ximera}

\newcommand{\RR}{\mathbb R}
\renewcommand{\d}{\,d}
\newcommand{\dd}[2][]{\frac{d #1}{d #2}}
\renewcommand{\l}{\ell}
\newcommand{\ddx}{\frac{d}{dx}}
\newcommand{\dfn}{\textbf}
\newcommand{\eval}[1]{\bigg[ #1 \bigg]}


\outcome{Equations of Tangent lines.}
\outcome{Computing Intersections of lines.}

\author{Nela Lakos}

\begin{document}
\begin{exercise}

The graph of the function  $f$, defined by $f(x)=\sin{x}$ and  two points on the graph of $f$, $P$ and $Q$, are given in the figure below.


\begin{image}
\begin{tikzpicture}
	\begin{axis}[
            xmin=-3.35,xmax=3.35,ymin=-1.8,ymax=1.8,
            axis lines=center,
            xtick={ -3.14, -1.57, 0, 1.57, 3.142},
            xticklabels={$-\pi$, $-\pi/2$, $0$, $\pi/2$, $\pi$},
            ytick={-1,1},
            %ticks=none,
            width=6in,
            height=3in,
            unit vector ratio*=1 1 1,
            xlabel=$x$, ylabel=$y$,
            every axis y label/.style={at=(current axis.above origin),anchor=south},
            every axis x label/.style={at=(current axis.right of origin),anchor=west},
          ]        
        
          \addplot [very thick, penColor, samples=100,smooth, domain=(-3.35:3.35)] {sin(deg(x))};
          
          \addplot[color=penColor,fill=penColor,only marks,mark=*] coordinates{(0,0)};  %% closed hole          
          \addplot[color=penColor,fill=penColor,only marks,mark=*] coordinates{(3.142,0)};  %% closed hole          
          \node at (axis cs:-2.14,.75) [penColor] {$y=f(x)=\sin(x)$};
           \node at (axis cs:0.13,-0.13) [penColor] {$P$};
            \node at (axis cs:3.23,0.16) [penColor] {$Q$};
        \end{axis}
\end{tikzpicture}
%% \caption{The function $\sin(\theta)$ takes on all values between $-1$
%%   and $1$ exactly once on the interval $[-\pi/2,\pi/2]$. If we
%%   restrict $\sin(\theta)$ to this interval, then this restricted
%%   function has an inverse.}
%% \label{figure:sin-restricted}
%% \end{figure*}
\end{image}

\begin{enumerate}
\item  Find an equation of  the tangent line to the curve $y=f(x)$ at the point $P$.


First, the coordinates of the point $P$ are
\[
P=(0,\answer{0}).
\]
In oder to find to find an equation of this tangent line, we need to determine the slope, $m$.
\[
m=f'(\answer{0})=\answer{1},
\]
since 
\[
f'(x)=\answer{\cos x}.
\]

Therefore, the equation of the  tangent line to the curve $y=f(x)$ at the point $P$ is given by
\[
y=\answer{1}x.
\]
\item  Find an equation of  the tangent line to the curve $y=f(x)$ at the point $Q$.


First, the coordinates of the point $Q$ are
\[
Q=(\pi,\answer{0}).
\]


In oder to find to find an equation of this tangent line, we need to determine the slope, $m$.
\[
m=f'(\answer{\pi})=\answer{-1}.
\]
Therefore, the equation of the  tangent line to the curve $y=f(x)$ at the point $Q$ is given by
\[
y=\answer{-1}(x-\pi).
\]
\item The two tangent lines, as defined in part (a) and part (b),  are shown in the figure below.

 Find the point of intersection of these two lines or show that such point does NOT exist.


\begin{image}
\begin{tikzpicture}
	\begin{axis}[
            xmin=-3.35,xmax=3.35,ymin=-1.8,ymax=1.8,
            axis lines=center,
            xtick={ -3.14, -1.57, 0, 1.57, 3.142},
            xticklabels={$-\pi$, $-\pi/2$, $0$, $\pi/2$, $\pi$},
            ytick={-1,1},
            %ticks=none,
            width=6in,
            height=3in,
            unit vector ratio*=1 1 1,
            xlabel=$x$, ylabel=$y$,
            every axis y label/.style={at=(current axis.above origin),anchor=south},
            every axis x label/.style={at=(current axis.right of origin),anchor=west},
          ]        
        
          \addplot [very thick, penColor, samples=100,smooth, domain=(-3.35:3.35)] {sin(deg(x))};
           \addplot [very thick, penColor2, samples=100,smooth, domain=(-3.35:3.35)] {x};
             \addplot [very thick, penColor2, samples=100,smooth, domain=(-3.35:3.35)] {3.14-x};
          
          \addplot[color=penColor,fill=penColor,only marks,mark=*] coordinates{(0,0)};  %% closed hole          
          \addplot[color=penColor,fill=penColor,only marks,mark=*] coordinates{(3.142,0)};  %% closed hole      
           \addplot[color=penColor2,fill=penColor,only marks,mark=*] coordinates{(3.142/2,3.142/2)};  %% closed hole       
          \node at (axis cs:-2.14,.75) [penColor] {$y=f(x)=\sin(x)$};
           \node at (axis cs:0.13,-0.13) [penColor] {$P$};
            \node at (axis cs:3.23,0.16) [penColor] {$Q$};
        \end{axis}
\end{tikzpicture}
%% \caption{The function $\sin(\theta)$ takes on all values between $-1$
%%   and $1$ exactly once on the interval $[-\pi/2,\pi/2]$. If we
%%   restrict $\sin(\theta)$ to this interval, then this restricted
%%   function has an inverse.}
%% \label{figure:sin-restricted}
%% \end{figure*}
\end{image}


In order to find the point of intersection of these two tangent lines, we have to solve the following system of equations:
\[
y=\answer{1}x
\]
\[
y=\answer{-1}(x-\pi).
\]
The solution of this system of equations and, therefore, the point of intersection of the two tangent lines is the point
\[
(\answer{\pi/2},\answer{\pi/2}).
\]

\end{enumerate}




\end{exercise}
\end{document}