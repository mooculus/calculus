\documentclass{ximera}
\newcommand{\RR}{\mathbb R}
\renewcommand{\d}{\,d}
\newcommand{\dd}[2][]{\frac{d #1}{d #2}}
\renewcommand{\l}{\ell}
\newcommand{\ddx}{\frac{d}{dx}}
\newcommand{\dfn}{\textbf}
\newcommand{\eval}[1]{\bigg[ #1 \bigg]}

\author{Steven Gubkin}
\license{Creative Commons 3.0 By-NC}
\begin{document}
\begin{exercise}

Let $f$ be a function.  Consider the function $A(t) = \frac{1}{t} \int_0^t f(x)\d x $, which gives the average of $f$ on the interval $[0,t]$.  

Suppose that $A$ has a local maximum at $t=b$, then $A'(b) = \answer{0}$ (answer with a number).  Now, the formula for $A'(t)$ is
\begin{align*}
A'(t) &= \answer{\frac{f(t)}{t}} - \answer{\frac{1}{t^2}}\int_0^t\answer{f(x)}\d x\\
&= \frac{1}{t}\left(\answer{f(t)-A(t)}\right)
\end{align*}
Now since $A'(b) = \answer{0}$ (answer with a number), we can conclude that $A(b) = \answer{f(b)}$ (answer with an expression in terms of $f$).

Now suppose that $f(t) = A(t)$ for all $t >0$, then $A'(t)=\answer{0}$ for all $t>0$ and so $A$ must be constant, that is, there exists $c$ such that
\[
A(t) = \frac{1}{t}\int_0^t f(x) \d x = \answer{c}.
\]
So
\[
\int_0^t f(x) \d x = \answer{tc}.
\]
Differentiating both sides with respect to $t$ we find that
\[
\answer{f(t)} = \answer{c},
\]
that is, $f$ is constant.

Suppose $f(x)  = x(1-x)$.  For what value of $t$ is $A(t)$ maximized?
\[
t = \answer{\frac{3}{4}}
\]

\end{exercise}
\end{document}