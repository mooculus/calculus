\documentclass{ximera}

\newcommand{\RR}{\mathbb R}
\renewcommand{\d}{\,d}
\newcommand{\dd}[2][]{\frac{d #1}{d #2}}
\renewcommand{\l}{\ell}
\newcommand{\ddx}{\frac{d}{dx}}
\newcommand{\dfn}{\textbf}
\newcommand{\eval}[1]{\bigg[ #1 \bigg]}


%\outcome{Find tangent lines to parametric curves}
\author{Jim Talamo }

\begin{document}
\begin{exercise}

Express the parabola $y= x^2$ using polar coordinates:

\[
r= \answer{\sec(\theta)\tan(\theta)}
\]

\begin{hint}
Start by using:

\begin{multipleChoice}
\choice[correct]{$x = r\cos(\theta), y= r \sin(\theta)$}
\choice{$x = r\sin(\theta), y= r \cos(\theta)$}
\end{multipleChoice}

We thus find:

\begin{align*}
y &= x^2 \\
\left( \answer{r \sin(\theta)} \right) &= \left( \answer{r \cos(\theta)} \right)^2
\end{align*}

Now, solve for $r$ by moving all terms with an $r$ to the lefthand side:

\[
\answer{r \sin(\theta) - r^2 \cos^2(\theta)} =0
\]

Factor out an $r$:

\[
r \answer{ \sin(\theta) -r  \cos^2(\theta)} = 0
\]
So, either $r=0$ or $ \sin(\theta) -r  \cos^2(\theta) = 0$.  We can solve for $r$ for the latter equation:

\begin{align*}
\sin(\theta) -r  \cos^2(\theta) &= 0 \\
r &= \answer{\sec(\theta)\tan(\theta)}
\end{align*}
Note that there is a $\theta$ value for which $r=0$; in fact, when $\theta = \answer{0}$, $r=0$, so this does include the case when $r=0$.

\end{hint}

\end{exercise}
\end{document}