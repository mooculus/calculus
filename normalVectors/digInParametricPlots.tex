\documentclass{ximera}

\newcommand{\RR}{\mathbb R}
\renewcommand{\d}{\,d}
\newcommand{\dd}[2][]{\frac{d #1}{d #2}}
\renewcommand{\l}{\ell}
\newcommand{\ddx}{\frac{d}{dx}}
\newcommand{\dfn}{\textbf}
\newcommand{\eval}[1]{\bigg[ #1 \bigg]}


\title[Dig-In:]{Parametric plots}

\author{Bart Snapp}

\begin{document}
\begin{abstract}
  Tangent and normal vectors can help us make interesting parametric
  plots.
\end{abstract}
\maketitle

\section{A sine curve on a circle}

Suppose you wish to draw a sine curve on a circle like this:
\begin{image}
  \begin{tikzpicture}
    \begin{axis}[
        xmin=-4.1,xmax=4.1,ymin=-4.1,ymax=4.1,
        unit vector ratio*=1 1 1,
        axis lines={none},
      ]
      \addplot [ultra thick, black, smooth, samples=100,domain=(0:360)] ({cos(x)*(3)},{sin(x)*(3))});
      
      \addplot [ultra thick, penColor, smooth, samples=100,domain=(0:360)] ({cos(x)*(3-sin(4*x))},{sin(x)*(3-sin(4*x))});
    \end{axis}
  \end{tikzpicture}
\end{image}

How do you do this? Well, a general method for placing one curve along
another is to use unit tangent and unit normal vectors!
\begin{example}
  Plot the curve $y= \sin(5x)$ ``wrapped'' around a circle of radius $3$.
  \begin{explanation}
    To do this, start by setting
    \[
    \vec{c}(t) = \vector{3\cos(t),\answer[given]{3\sin(t)}}
    \]
    This will produce our circle of radius $3$.
    Now we need to add our sine curve. To do this, we compute the unit tangent vector
    \begin{align*}
      \utan(t) &= \frac{\vec{c}'(t)}{|\vec{c}'(t)|}\\
      &=\vector{\answer[given]{-\sin(t)},\answer[given]{\cos(t)}}
    \end{align*}
    and the unit normal vector:
    \begin{align*}
      \unormal(t) &= \frac{\utan'(t)}{|\utan'(t)|}\\
      &=\vector{\answer[given]{-\cos(t)},\answer[given]{-\sin(t)}}
    \end{align*}
    We've plotted our circle of radius $3$ with some unit tangent and unit
    normal vectors for your viewing pleasure:
    \begin{image}
      \begin{tikzpicture}
	\begin{axis}[
            xmin=-4.1,xmax=4.1,ymin=-4.1,ymax=4.1,
            axis lines=center,
            clip=false,
            unit vector ratio*=1 1 1,
            xlabel=$x$, ylabel=$y$,
            every axis y label/.style={at=(current axis.above origin),anchor=south},
            every axis x label/.style={at=(current axis.right of origin),anchor=west},
          ]
          \addplot [ultra thick, black, smooth, samples=100,domain=(0:360)] ({cos(x)*(3))},{sin(x)*(3)});
          
          \foreach \x in {0,40,...,320}
          {
          \addplot[penColor,->,ultra thick] plot coordinates {
            (
            {cos(\x)*(3)},
            {sin(\x)*(3)}
            )
            (
            { -sin(\x)+3*cos(\x)},
            {cos(\x)+3*sin(\x)}
            )
          };
          }
          \foreach \x in {0,40,...,320}
          {
          \addplot[penColor2,->,ultra thick] plot coordinates {
            (
            {cos(\x)*(3)},
            {sin(\x)*(3)}
            )
            (
            { -cos(\x)+3*cos(\x)},
            {-sin(\x)+3*sin(\x)}
            )
          };
          }
          
          \node[penColor,left] at (axis cs: -1.4,2.8) {$\utan(t)$};
          \node[penColor2,left] at (axis cs: 1.5,-1.5) {$\unormal(t)$};
        \end{axis}
      \end{tikzpicture}
    \end{image}
    To add the sine curve, quite literally, add it in:
    \[
    \vec{f}(t) = \vec{c}(t) + \unormal(t)\cdot \sin(5t)
    \]
    where $\vec{c}(t)$ draws the circle, and $\unormal(t)\cdot
    \sin(5t)$ draws the sine curve. Note, breaking $\vec{f}$ in to components we have:
    \begin{align*}
      x(t) &= \answer{3\cos(t)-\sin(5t)\cos(t)}\\
      y(t) &= \answer{3\sin(t)-\sin(5t)\sin(t)}
    \end{align*}
    \begin{onlineOnly}
      We can confirm our construction by making a graph:
      \[
      \graph{(3\cos(t)-\sin(5t)\cos(t),3\sin(t)-\sin(5t)\sin(t))}
      \]
    \end{onlineOnly}
  \end{explanation}
\end{example}

\section{Thickening a curve}

Suppose you have a vector-valued function $\vec{f}:\R\to\R^2$ that
defines a curve in space, and you want to build a parameterized
surface that looks like a ``thickened'' version of the curve.  In
other words, we want to convert a curve like
\begin{image}
  \includegraphics{curve.jpg}
\end{image}
%% \begin{sageOutput}
%% x(t) = sin(t)
%% y(t) = sin(2*t + pi/5)
%% z(t) = sin(3*t + pi/7)
%% f(t) = (x(t),y(t),z(t))
%% parametric_plot3d(f(t), (t,0,2*pi))
%%   \end{sageOutput}
into a thickened ``tube'' like
\begin{image}
  \includegraphics{tube.jpg}
\end{image}
%% \begin{sageOutput}
%% var('s')
%% x(t) = sin(t)
%% y(t) = sin(2*t + pi/5)
%% z(t) = sin(3*t + pi/7)
%% f(t) = (x(t),y(t),z(t))
%% df=derivative(f,t)
%% ut = df / df.norm()
%% ddf = derivative(ut,t)
%% n = ddf / ddf.norm()
%% bn = n.cross_product(ut)
%% thickness = 0.10
%% parametric_plot3d(f(t) + (n * cos(s) + bn * sin(s)) * thickness, (t,0,2*pi), (s,0,2*pi), plot_points=[100,100])
%%   \end{sageOutput}
%% \begin{image}
%%   \begin{tikzpicture}
%%     \begin{axis}[tick label style={font=\scriptsize},axis on top,view={-30}{30},no markers,zmax=1,axis lines=center,
%%         ymax=1.5,ymin=-1.5,clip=false,
%%         xmax=1.5,xmin=-1.5,
%%         every axis x label/.style={at={(axis cs:\pgfkeysvalueof{/pgfplots/xmax},0,0)},xshift=-3pt,yshift=-3pt},
%% 	xlabel={\scriptsize $x$},
%% 	every axis y label/.style={at={(axis cs:0,\pgfkeysvalueof{/pgfplots/ymax},0)},xshift=0pt,yshift=-5pt},
%% 	ylabel={\scriptsize $y$},
%% 	every axis z label/.style={at={(axis cs:0,0,\pgfkeysvalueof{/pgfplots/zmax})},xshift=0pt,yshift=4pt},
%% 	zlabel={\scriptsize $z$}]
%%       \addplot3+[domain=-0:360,samples=200,samples y=0,ultra thick](sin(x),{sin(2*x+36},{sin(3*x+25)});
%%     \end{axis}
%%   \end{tikzpicture}
%% \end{image}

To plot a ``tube'' around a vector-valued function $\vec{f}$, we need
three handy vectors:
\begin{itemize}
\item The \dfn{unit tangent vector}:
  \[
  \utan(t) = \frac{\vec{f}'(t)}{|\vec{f}'(t)|}
  \]
\item The \dfn{unit normal vector}:
  \[
  \unormal(t) = \frac{\vec{t}'(t)}{|\vec{t}'(t)|}
  \]
\item The \dfn{unit binormal vector}:
  \[
  \ubinormal(t) = \utan(t) \cross \unormal(t)
  \]
\end{itemize}

Let's see these vectors in action with our next example.

\begin{example}
  You have been given a curve in space, say
  \[
  \vec{f}(t) = \langle t^2-2, t^2-2, 1-t\rangle.
  \]
  We want to make a tube around it of radius $0.5$.
  \begin{explanation}
    Here is the idea, we want to take our curve and attach two
    orthogonal vectors to every point
    \begin{image}
      \includegraphics{paraArrows.jpg}
    \end{image}
    We'll start by computing the
    \wordChoice{\choice[correct]{unit tangent}\choice{unit normal}\choice{unit binormal}}
    vectors. Write with me
    \[
    \utan(t) = \frac{\vec{f}'(t)}{|\vec{f}'(t)|}
    \]
    From these vectors we can obtain the \wordChoice{\choice{unit
        tangent}\choice[correct]{unit normal}\choice{unit binormal}}
    vectors by differentiating:
    \[
    \unormal(t) = \frac{\utan'(t)}{|\utan'(t)|}
    \]
    Finally we can obtain the \wordChoice{\choice{unit
        tangent}\choice{unit normal}\choice[correct]{unit binormal}}
    vectors via the cross product:
    \[
    \ubinormal(t) = \utan(t) \cross \unormal(t)
    \]
    Looking again at
    \begin{image}
      \includegraphics{paraArrows.jpg}
    \end{image}
    we see the unit normal vectors in red, and the unit binormal
    vectors in blue. So we now can plot our tube where $s$ runs from
    $0$ to $2\pi$:
    \begin{align*}
      \vec{F}(s,t) = \vec{f}(t) &+ 0.5 \unormal(\answer[given]{t})\cos(\answer[given]{s})\\
      &+ 0.5 \ubinormal(\answer[given]{t})\sin(\answer[given]{s})
    \end{align*}
    \begin{onlineOnly}
      We can check our work with \link[SAGE]{http://www.sagemath.org/}.
  \begin{sageCell}
var('s t')
x(t) = t^2-2
y(t) = t^2-2
z(t) = 1-t
f(t) = (x(t),y(t),z(t))
df=derivative(f,t)
ut = df / df.norm()
ddf = derivative(ut,t)
n = ddf / ddf.norm()
bn = ut.cross_product(n)
parametric_plot3d(f(t) + 0.5*n*cos(s) + 0.5*bn*sin(s), (t,-3,3), (s,0,2*pi))
  \end{sageCell}
    \end{onlineOnly}
  \end{explanation}
\end{example}


\end{document}
