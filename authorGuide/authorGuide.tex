\documentclass{article}


\usepackage{amsmath,amssymb}

\newcommand{\RR}{\mathbb R}
\renewcommand{\d}{\,d}
\newcommand{\dd}[2][]{\frac{d #1}{d #2}}
\renewcommand{\l}{\ell}
\newcommand{\ddx}{\frac{d}{dx}}
\everymath{\displaystyle}
\newcommand{\dfn}{\textbf}
\newcommand{\eval}[1]{\bigg[ #1 \bigg]}





\newcommand{\mooculus}{\textsf{\textbf{MOOC}\textnormal{\textsf{ULUS}}}}

\begin{document}

\section*{Author guide for \mooculus}

\subsection*{Basic guidelines}

The first letter of the first word of a title is capitalized, then all
other words are lowercase, except for proper nouns. For example:
\begin{quote}
Introduction to Newton's method
\end{quote}
No puncutation is used at the end, except for perhaps a question mark.


The abstract is a one-sentence description of the activity. It is
intended to give the instructor an idea of what the activity is about.


Do not hack the \LaTeX\ document to make it appear in a customized
way. Do not add vertical space, boxes around forumlas, etc. This
should be done by the Ximera conversion. Hence, if some special
formatting is needed, it should be done at the level of the
ximera.cls.

\subsection*{Ftructure guidelines}

Each lecture corresponds in \mooculus to three distinct sections:
\begin{description}
\item[Launch] An introduction to the topic.
\item[Learn] A detailed discussion of the topic.
\item[Practice] Practice problems.
\end{description}

\paragraph{Launch}

The launch should somehow present a ``mystery'' for the students to
solve.  Moreover, it should lead them to the right path. Ideally if we
had a bright student working on the launch, they might even develop
the techniques from the lesson to solve the problem.  By the end of
the learn phase, the mystery is solved!

Typically the launch will be between 5 and 20 \textbf{Problems}.

Each launch needs to end with a \textbf{Xarma Boost} asking the students to generate their
own questions.


\paragraph{Learn}


The learn section is most like a traditional textbook. However, some key differences:


Every definition and every theorem should have a \textbf{Question}
following it to check for student comprehension.


The ``mystery'' presented in the Launch must be solved here. 



\paragraph{Practice}

This section is a list of \textbf{Problems}. Though maybe we want to
ask \textbf{exercises} and \textbf{explorations} here too.



\subsection*{\LaTeX\ Macros}

\renewcommand{\arraystretch}{2}
\begin{tabular*}{1.0\textwidth}{lll}
\hline
Command & Example & Typset \\
\hline
%\verb|\fxn| & \verb|\fxn{f}(x)| & $\fxn{f}(x)$\\
\verb|\RR| & \verb|f:\RR\to\RR| & $f:\RR\to\RR$\\ 
\verb|\ddx| & \verb|\ddx f(x)| & $\ddx f(x)$\\
\verb|\dd[_]{_}| & \verb|\dd[y]{x}| & $\dd[y]{x}$ \\
\verb|\d | & \verb|\int f(x) \d x| & $\int f(x) \d x$\\
\verb|\l| & \verb|\l(x) = mx+b| & $\l(x) = mx +b$\\
\verb|\dfn| & \verb|we define \dfn{this}| & we define \dfn{this}\\
\verb|\eval| & \verb|\eval{f(x)}_a^b| & $\eval{f(x)}_a^b$
\end{tabular*}



\end{document}
