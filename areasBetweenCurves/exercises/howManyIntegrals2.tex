\documentclass{ximera}

\newcommand{\RR}{\mathbb R}
\renewcommand{\d}{\,d}
\newcommand{\dd}[2][]{\frac{d #1}{d #2}}
\renewcommand{\l}{\ell}
\newcommand{\ddx}{\frac{d}{dx}}
\newcommand{\dfn}{\textbf}
\newcommand{\eval}[1]{\bigg[ #1 \bigg]}


\author{Jim Talamo and Alex Beckwith}
\license{Creative Commons 3.0 By-NC}


\outcome{Set up an integral with respect to $x$.}

\begin{document}
\begin{exercise}
	Consider the region bounded by $y=\frac{1}{x}, y=\frac{2}{x}, y=x$, and $y=6x$.
	\begin{image}
	\begin{tikzpicture}
		\begin{axis}[
			domain=-2:4, ymax=4.5,xmax=3, ymin=-1, xmin=-0.5,
			axis lines =center, xlabel=$x$, ylabel=$y$,
            		every axis y label/.style={at=(current axis.above origin),anchor=south},
            		every axis x label/.style={at=(current axis.right of origin),anchor=west},
            		axis on top,
            		]
                      
            	\addplot [draw=penColor,very thick,smooth] {6*x};
            	\addplot [draw=penColor,very thick,smooth] {x};
		\addplot [domain=0.3:2,draw=penColor2,very thick,smooth] {1/x};
		\addplot [domain=0.3:2,draw=penColor2,very thick,smooth] {2/x};
                       
            	\addplot [name path=A,domain=6^(-1/2):3^(-1/2),draw=none] {6*x};   
            	\addplot [name path=B,domain=6^(-1/2):3^(-1/2),draw=none] {1/x};
		\addplot [name path=C,domain=3^(-1/1.9):1,draw=none] {1/x};
		\addplot [name path=D,domain=3^(-1/1.9):1,draw=none] {2/x};
		\addplot [name path=E,domain=0.99:2^(1/2),draw=none] {x};
		\addplot [name path=F,domain=0.99:2^(1/2),draw=none] {2/x};
            	\addplot [fillp] fill between[of=A and B];
		\addplot [fillp] fill between[of=C and D];
		\addplot [fillp] fill between[of=E and F];
		
		\node at (axis cs:1,3.75) [penColor] {$y=6x$};
            	\node at (axis cs:2.5,3) [penColor] {$y=x$};
		\node at (axis cs:2.3,0.5) [penColor2] {$y=\frac{1}{x}$};
            	\node at (axis cs:2,1.45) [penColor2] {$y=\frac{2}{x}$};
                      
            	\end{axis}
	\end{tikzpicture}
	\end{image}
	
	How many integrals with respect to $x$ are needed to compute this area? $\answer{3}$
		
	How many integrals with respect to $y$ are needed to compute this area? $\answer{3}$
	
\end{exercise}
\begin{remark}
You'll learn a way to compute this area using only one integral in a course in multivariable calculus!
\end{remark}

\end{document}
