%\documentclass{ximera}
%
%\newcommand{\RR}{\mathbb R}
\renewcommand{\d}{\,d}
\newcommand{\dd}[2][]{\frac{d #1}{d #2}}
\renewcommand{\l}{\ell}
\newcommand{\ddx}{\frac{d}{dx}}
\newcommand{\dfn}{\textbf}
\newcommand{\eval}[1]{\bigg[ #1 \bigg]}

%
%\title[Refresh:]{Factorials}
%
%\begin{document}
%\begin{abstract}
%  Remember our facts about factorials.
%\end{abstract}
%\maketitle
%
%\begin{problem}
%  The Ratio Test is a useful convergence test that is very useful when
%  trying to determine convergence or divergence of series whose
%  summands involve products of​ polynomials, factorials, and
%  exponential functions.  
%
%
%  A fundamental skill when using this test is to be able to simplify
%  ratios of expressions involving these types of terms.
%
%  This assignment will review some of the algebraic skills required
%  when using this test.
%  \begin{multipleChoice}
%    \choice[correct]{I understand.}
%    \choice{I do not understand.}
%  \end{multipleChoice}
%\end{problem}
%
%
%\section{Ratios of exponentials}
%
%
%\begin{problem}
%  Ratios of exponential terms also simplify nicely. Suppose $a_k=2^k$
%  and $b_k=3^{2k}$. Simplify:
%  \[
%  \frac{a_{k+1}}{a_k} = \answer{2}
%  \]
%  \[
%  \frac{b_{k+1}}{b_k}=\answer{9}
%  \]
%\end{problem}
%
%
%\begin{problem}
%  Suppose $a_k$ is a sequence whose $k$th term is given​ by:
%  \[
%  a_k=4^{k^2}
%  \]
%  Note: The $k^2$ is in the exponent. Simplify
%  \[
%  \frac{a_{k+1}}{a_k} = \answer{4^{2k+1}}
%  \]
%\end{problem}
%
%\section{Factorials}
%
%\begin{problem}
%  Given an integer $n$, the notation​ ``$n!$'' is defined as​ follows:
%  \[
%  n! = n(n-1)(n-2)\dots (3)(2)(1)
%  \]
%  For instance, $3!=3\cdot2\cdot 1 = 6$. Compute:
%  \begin{align*}
%    4! &= \answer{24}\\
%    6! &= \answer{720}\\
%    \frac{6!}{4!} &= \answer{30}
%  \end{align*}
%  \begin{problem}
%    How did you compute $\frac{6!}{4!}$ in the last problem? You could
%    certainly find $6!$ and $4!$ separately, then​ divide, but there is
%    a nicer way to do​ this:
%    \begin{align*}
%      \frac{6!}{4!} &= \frac{6\cdot5\cdot4\cdot3\cdot2\cdot1}{4\cdot3\cdot2\cdot1}\\
%      &=\frac{6\cdot5\cdot\not{4}\cdot\not{3}\cdot\not{2}\cdot\not{1}}{\not{4}\cdot\not{3}\cdot\not{2}\cdot\not{1}}\\
%      &= 6\cdot 5.
%    \end{align*}
%    Ratios of factorials are always easiest to compute by canceling
%    like​ terms! Compute the following, simplify your final answers.
%    \begin{align*}
%      \frac{5!}{4!} &= \answer{5}\\
%      \frac{6!}{3!\cdot 3!} &= \answer{20}\\
%      \frac{k!}{(k+2)!}  &=\answer{\frac{1}{(k+2)(k+1)}}
%    \end{align*}
%  \end{problem}
%\end{problem}
%
%\begin{problem}
%  ​Often, factorials arise in​ sequences, and it is important to
%  understand how to write out various terms.   Suppose $a_k$ is a
%  sequence whose $k$th term is given​ by:
%  \[
%  a_k = (2k+1)!
%  \]
%  \begin{enumerate}
%  \item $a_2 = \answer{120}$
%  \item Find an expression for $a_{k+1}$. Express your answer in the form $(ck+d)!$.
%    \[
%    a_{k+1} = \left(\answer{2} \cdot k + \answer{3}\right)!
%    \]
%  \item Calculate and simplify:
%    \[
%    \frac{a_{k+1}}{a_k} = \answer{(2k+3)(2k+2)}
%    \]
%  \end{enumerate}
%\end{problem}
%
%\begin{problem}
%  Let $a_k$ be a sequence whose $k$th term is given​ by:
%  \[
%  a_k = \frac{(2k)!}{(k!)^2}
%  \]
%  Calculate and simplify:
%  \begin{align*}
%    \frac{a_{k+1}}{a_k} &= \answer{\frac{4k+2}{k+1}}\\
%    \lim_{k\to\infty} \frac{a_{k+1}}{a_k} &= \answer{4}
%  \end{align*}
%\end{problem}
%
%
%\end{document}
