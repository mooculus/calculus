\documentclass{ximera}
\newcommand{\RR}{\mathbb R}
\renewcommand{\d}{\,d}
\newcommand{\dd}[2][]{\frac{d #1}{d #2}}
\renewcommand{\l}{\ell}
\newcommand{\ddx}{\frac{d}{dx}}
\newcommand{\dfn}{\textbf}
\newcommand{\eval}[1]{\bigg[ #1 \bigg]}

\author{Jim Talamo and Alex Beckwith}
\license{Creative Commons 3.0 By-NC}
\outcome{Set up a volume integral using the Washer Method}
\begin{document}
\begin{exercise}

	Let $R$ be the region in the $xy$-plane bounded by $y=0$, $y=\ln x$, $y=2$, and $x=0$. This exercise will walk you through setting up an integral using the Washer Method that will give the volume of the solid generated when $R$ is revolved about the line $x=-1$.
            \begin{image}
            \begin{tikzpicture}
            	\begin{axis}[
            		domain=-2.5:8.5, ymax=2.5,xmax=8.4, ymin=-0.5, xmin=-2.4,
            		axis lines =center, xlabel=$x$, ylabel=$y$,
            		every axis y label/.style={at=(current axis.above origin),anchor=south},
            		every axis x label/.style={at=(current axis.right of origin),anchor=west},
            		axis on top,
            		]
                      
            	\addplot [draw=penColor,very thick,smooth] {2};
            	\addplot [draw=penColor2,very thick,smooth] {ln(x)};
		\addplot [draw=penColor3,very thick,smooth] {0};
		\addplot [draw=penColor4,very thick,smooth] coordinates {(1,-4)(1,12)};
		\addplot [draw=penColor5,very thick,dotted] coordinates {(-1,-0.5)(-1,2.5)};
                       
            	\addplot [name path=A,domain=1:7.4,draw=none] {2};   
            	\addplot [name path=B,domain=1:7.4,draw=none] {ln(x)};
            	\addplot [fillp] fill between[of=A and B];
	                
            	\node at (axis cs:5,1.2) [penColor2] {$y=\ln(x)$};
		\node at (axis cs:4.5,2.15) [penColor] {$y=2$};
		\node at (axis cs:1.8,2.3) [penColor4] {$x=1$};
            	\end{axis}
            \end{tikzpicture}
            \end{image}

\begin{exercise}

Since we are using the Washer Method, the slices must be:
\begin{multipleChoice}
\choice{parallel}
\choice[correct]{perpendicular}
\end{multipleChoice}
to the axis of rotation.  

Slices that are perpendicular to the axis of rotation $x=-1$ are:
\begin{multipleChoice}
\choice{vertical}
\choice[correct]{horizontal}
\end{multipleChoice}

Since the slices are horizontal, we must: 
\begin{multipleChoice}
\choice{integrate with respect to $x$.}
\choice[correct]{integrate with respect to $y$.}
\end{multipleChoice}

Since we must integrate with respect to $y$, we will use the result:

\[V = \int_{y=c}^{y=d} \pi\left(R^2-r^2\right) \d y \]

to set up the volume.  We must now find the limits of integration as express the outer radius $R$ and the inner radius $r$ in terms of the variable of integration $y$. 

\begin{exercise}
The limits of integration are: $c = \answer{0}$ and $d = \answer{2}$. 
\end{exercise}

\begin{exercise}
For the curve given by $y=\ln(x)$, we can solve for $x$ to find $x= \answer{e^y}$.
\end{exercise}

\begin{exercise}

We thus have a helpful version of the picture of the region $R$ below:

  \begin{image}
            \begin{tikzpicture}
            	\begin{axis}[
            		domain=-2.5:8.5, ymax=2.5,xmax=8.4, ymin=-0.5, xmin=-2.4,
            		axis lines =center, xlabel=$x$, ylabel=$y$,
            		every axis y label/.style={at=(current axis.above origin),anchor=south},
            		every axis x label/.style={at=(current axis.right of origin),anchor=west},
            		axis on top,
            		]
                      
            	\addplot [draw=penColor,very thick,smooth] {2};
            	\addplot [draw=penColor2,very thick,smooth] {ln(x)};
		\addplot [draw=penColor3,very thick,smooth] {0};
		\addplot [draw=penColor4,very thick,smooth] coordinates {(1,-4)(1,12)};
		\addplot [draw=penColor5,very thick,dotted] coordinates {(-1,-0.5)(-1,2.5)};
                       
            	\addplot [name path=A,domain=1:7.4,draw=none] {2};   
            	\addplot [name path=B,domain=1:7.4,draw=none] {ln(x)};
            	\addplot [fillp] fill between[of=A and B];
	                
            	\node at (axis cs:5,1.2) [penColor2] {$x=e^y$};
		\node at (axis cs:4.5,2.15) [penColor] {$y=2$};
		\node at (axis cs:1.8,2.3) [penColor4] {$x=1$};
	
		\addplot [draw=penColor, fill = gray!50] plot coordinates {(1,1.63) (1,1.71) (5,1.71) (5,1.63) (1,1.63)};
          
          %Draw R and r
          \addplot [draw=black!30!red,very thick] coordinates {(-1,1.2)(exp(1.2),1.2)};
          \node at (axis cs:1.8,1.32) [black!30!red] {$R$};
          
	 \addplot [draw=black!30!blue,very thick] coordinates {(-1,.9)(1,.9)};

	 \node at (axis cs:.4,.75)  [black!30!blue]  {$r$};
                      
                  	\end{axis}
            \end{tikzpicture}
  \end{image}
            
 We see from the picture that both $R$ and $r$ are:
 \begin{multipleChoice}
 \choice{vertical distances}
 \choice[correct]{horizontal distances}
 \end{multipleChoice}           
            
\begin{exercise}
Since $R$ is the distance from the axis of rotation to the outer curve, and this is a horizontal distance, we find $R = x_{right}-x_{left}$.
\begin{multipleChoice}
 \choice[correct]{$x_{right} = e^y$}
 \choice{$x_{right} = 1$}
  \choice{$x_{right} = -1$}
\end{multipleChoice}       

\begin{multipleChoice}
 \choice{$x_{left} = e^y$}
 \choice{$x_{left} = 1$}
  \choice[correct]{$x_{left} = -1$}
\end{multipleChoice}   

So, $R= \answer{e^y-(-1)}$.
 \end{exercise}
 
 \begin{exercise}

Since $r$ is the distance from the axis of rotation to the inner curve, and this is a horizontal distance, we find $r = x_{right}-x_{left}$.
\begin{multipleChoice}
 \choice{$x_{right} = e^y$}
 \choice[correct]{$x_{right} = 1$}
  \choice{$x_{right} = -1$}
\end{multipleChoice}  
 
 \begin{multipleChoice}
 \choice{$x_{left} = e^y$}
 \choice{$x_{left} = 1$}
  \choice[correct]{$x_{left} = -1$}
\end{multipleChoice} 

So, $r= \answer{1-(-1)}$.

\end{exercise}            
            
\begin{exercise}

Using \[V = \int_{y=c}^{y=d} \pi\left(R^2-r^2\right) \d y, \] we find that an integral that gives the volume of the solid of revolution is:            
	\[
	V= \int_{y=\answer{0}}^{y=\answer{2}}
	\answer{\pi \left((e^y-(-1))^2-(2)^2\right)}\d y
	\]
\end{exercise}
\end{exercise}
\end{exercise}
\end{exercise}
\end{document}