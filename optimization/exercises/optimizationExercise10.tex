\documentclass{ximera}

\newcommand{\RR}{\mathbb R}
\renewcommand{\d}{\,d}
\newcommand{\dd}[2][]{\frac{d #1}{d #2}}
\renewcommand{\l}{\ell}
\newcommand{\ddx}{\frac{d}{dx}}
\newcommand{\dfn}{\textbf}
\newcommand{\eval}[1]{\bigg[ #1 \bigg]}


%\outcome{Describe the goals of optimization problems generally.}
%\outcome{Find all local maximums and minimums using the First and Second Derivative tests.}
%\outcome{Identify when we can find an absolute maximum or minimum on an open interval.}
%\outcome{Contrast optimization on open and closed intervals.}
%\outcome{Describe the objective function and constraints in a given optimization problem.}
%\outcome{Solve optimization problems by finding the appropriate extreme values.}

\author{Nela Lakos \and Kyle Parsons}

\begin{document}
\begin{exercise}

Consider the function $f$ given by
\[
f(x) = 
\begin{cases}
4-x^2, & \text{ if } x>0\\
4-\frac{x^2}{4}, & \text{ if } x\leq 0
\end{cases}.
\]
Consider a rectangle with sides parallel to the axes, having two vertices on the $x$-axis, and the other two vertices on the graph of $f$ above the $x$-axis.

\begin{image}
 \begin{tikzpicture}
    \begin{axis}[
        xmin=-4.3,xmax=2.3,ymin=-0.3,ymax=4.3,
        clip=false,
        unit vector ratio*=1 1 1,
        axis lines=center,
        grid = major,
        ytick={-21,-20,...,20},
        xtick={-21,-20,...,20},
        xlabel=$x$, ylabel=$y$,
        y tick label style={anchor=west},
        every axis y label/.style={at=(current axis.above origin),anchor=south},
        every axis x label/.style={at=(current axis.right of origin),anchor=west},
      ]
      \addplot[very thick,penColor,domain=-4.2:0,samples=50] plot{4-x^2/4};
      \addplot[very thick,penColor,domain=0:2.1,samples=50]  plot{4-x^2};
      
      \pgfmathsetmacro{\x}{1.3};
      
      \coordinate (a) at (axis cs:\x,0);
      \coordinate (b) at (axis cs:\x,{4-\x^2});
      \coordinate (c) at (axis cs:{-2*\x},{4-\x^2});
      \coordinate (d) at (axis cs:{-2*\x},0);
      
      \filldraw[thick,penColor2,fill opacity=0.3] (a) -- (b) -- (c) -- (d) -- cycle;
      
      \node at (a) [below] {$l$};
       
      \node at (axis cs:-3.5,3.5) {$y = f(x)$};
      \end{axis}`
  \end{tikzpicture}
\end{image}

The area of the rectangle as a function of $l$ is (see picture above)
\[
A(l) = \answer{3l(4-l^2)}.
\]
The domain of the function $A$ is $\left[\answer{0},\answer{2}\right]$.

The width of the rectangle with the maximum area is $\answer{2\sqrt{3}}$ and its height is $\answer{\frac{8}{3}}$.

\end{exercise}
\end{document}