\documentclass{ximera}

\newcommand{\RR}{\mathbb R}
\renewcommand{\d}{\,d}
\newcommand{\dd}[2][]{\frac{d #1}{d #2}}
\renewcommand{\l}{\ell}
\newcommand{\ddx}{\frac{d}{dx}}
\newcommand{\dfn}{\textbf}
\newcommand{\eval}[1]{\bigg[ #1 \bigg]}


\author{Jim Talamo}
\license{Creative Commons 3.0 By-bC}


\outcome{}


\begin{document}
\begin{exercise}

Select all of the following series that converge.  

\begin{selectAll}
\choice{$\sum_{k=1}^{\infty} (-1)^k \frac{k+2}{3k+1}$}
\choice[correct]{$\sum_{k=3}^{\infty} \frac{2^{3k+1}}{9^k}$}
\choice{$\sum_{k=1}^{\infty} \frac{4}{\sqrt{k}}$}
\choice[correct]{$\sum_{k=2}^{\infty} \frac{4^k+3k}{5^k}$}
\choice[correct]{$\sum_{k=1}^{\infty} \frac{8^k+k^2}{k!}$}
\end{selectAll}

\begin{hint}
We say that a convergence test is \emph{applicable} if the assumptions of the test are met.  We say that a test is \emph{conclusive} if it can be used to determine whether a series converges or diverges.

%%%%%%%%%%%  FOR FIRST SERIES %%%%%%%%%%%
\begin{question}
For the series $\sum_{k=1}^{\infty} (-1)^k \frac{k+2}{3k+1}$, note that:

\[
\lim_{n \to \infty} \frac{n+2}{3n+1} = \answer{\frac{1}{3}}
\]

Hence:

\begin{multipleChoice}
\choice[correct]{$\lim_{n \to \infty} (-1)^n \frac{n+2}{3n+1} = \frac{1}{3}$}
\choice{$\lim_{n \to \infty} (-1)^n \frac{n+2}{3n+1} = -\frac{1}{3}$}
\choice{$\lim_{n \to \infty} (-1)^n \frac{n+2}{3n+1} = \infty$}
\choice{$\lim_{n \to \infty} (-1)^n \frac{n+2}{3n+1} = -\infty$}
\choice{$\lim_{n \to \infty} (-1)^n \frac{n+2}{3n+1}$ does not exist.}
\end{multipleChoice}

Thus, the divergence test is conclusive.
\end{question}

%%%%%%%%%%%  FOR SECOND SERIES %%%%%%%%%%%


\begin{question}
For the series $\sum_{k=3}^{\infty} \frac{2^{3k+1}}{9^k}$, is the series geometric or a $p$-series?

\begin{multipleChoice}
\choice[correct]{The series is a geometric series.}
\choice{The series is a $p$-series.}
\choice{The series is neither a geometric series nor a $p$-series.}
\end{multipleChoice}

After some manipulation, we can write $ \frac{2^{3k+1}}{9^k} = \answer{2} \cdot \left(\answer{\frac{8}{9}}\right)^k$.

Thus, the series $\sum_{k=3}^{\infty} \frac{2^{3k+1}}{9^k}$ is geometric and it:

\begin{multipleChoice}
\choice[correct]{converges.}
\choice{diverges.}
\end{multipleChoice}
\end{question}

%%%%%%%%%%%  FOR THIRD SERIES %%%%%%%%%%%
\begin{question}
For the series $\sum_{k=2}^{\infty} \frac{4}{\sqrt{k}}$, is the series geometric or a $p$-series?

\begin{multipleChoice}
\choice{The series is a geometric series.}
\choice[correct]{The series is a $p$-series.}
\choice{The series is neither a geometric series nor a $p$-series.}
\end{multipleChoice}

Here, $p= \answer{\frac{1}{2}}$. 
\end{question}



%%%%%%%%%%%  FOR FOURTH SERIES %%%%%%%%%%%

\begin{question}
For the series $\sum_{k=2}^{\infty} \frac{4^k+3k}{5^k}$, note that the summand is positive.  A solution can be given using either the Ratio Test, the Root Test, or the Limit Comparison Test without too much difficulty.

\end{question}


%%%%%%%%%%%  FOR FIFTH SERIES %%%%%%%%%%%
\begin{question}

Because there is a factorial in the summand, the best option is:

\begin{multipleChoice}
\choice{Divergence Test}
\choice{Comparison Test}
\choice{Limit Comparison Test}
\choice[correct]{Ratio Test}
\choice{Root Test}
\end{multipleChoice}

The limit to evaluate is:

\[
L = \lim_{n \to \infty} \frac{a_{n+1}}{a_n} = \lim_{n \to \infty} \frac{\answer{ \frac{8^{n+1}+(n+1)^2}{(n+1)!}}}{\answer{ \frac{8^n+n^2}{n!}}}
\]

\begin{question}
Writing this out in a more convenient form, we find:

\[
L = \lim_{n \to \infty} \frac{8^{n+1}+(n+1)^2}{8^n+n^2} \cdot \frac{n!}{(n+1)!} = \answer{8} \cdot \answer{0} = \answer{0}
\]
(For the first limit, factor out the dominant term in the numerator and the dominant term in the denominator)

\end{question}
\end{question}
\end{hint}







\begin{quote}
You should be able to give a detailed solution for each of your choices.  Such a solution should include:

\begin{itemize}
\item What test you chose.
\item Why you are allowed to use that test.
\item The computations required for the conclusions of the test.
\item An explanation why the series either converges or diverges by the test you chose.
\end{itemize}

\end{quote}


\end{exercise}
\end{document}
