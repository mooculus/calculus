\documentclass{ximera}

\newcommand{\RR}{\mathbb R}
\renewcommand{\d}{\,d}
\newcommand{\dd}[2][]{\frac{d #1}{d #2}}
\renewcommand{\l}{\ell}
\newcommand{\ddx}{\frac{d}{dx}}
\newcommand{\dfn}{\textbf}
\newcommand{\eval}[1]{\bigg[ #1 \bigg]}


\outcome{Compute average velocity}

\author{Nela Lakos \and Kyle Parsons}

\begin{document}
\begin{exercise}

An oil tank is to be drained for cleaning.  There are $V$ gallons of oil left in the tank $t$ minutes after the draining has begun, where 
\[
V(t) = 45(60-t)^2.
\]

The average rate at which the oil drains in the time interval $\left[0,15\right]$ is
\[
AR = \answer{-4725}\text{gal/min}.
\]

The average rate at which the oil drains in the time interval $\left[10,15\right]$ is
\[
AR = \answer{-4275}\text{gal/min}.
\]

The rate at which the oil drains 15 minutes after draining has begun is
\[
R = \answer{-4050}\text{gal/min}.
\]

The average rate at which the oil drains during the time interval $\left[15,15+h\right]$ for $0<h<1$ or the interval $\left[15+h,15\right]$ for $-1<h<0$ is
\[
AR(h) = \answer{-4050 + 45h}\text{gal/min}.
\]

The limit as $h$ goes to zero of the above average rate is
\[
\lim_{h\to0}AR(h) = \answer{-4050}\text{gal/min}.
\]

\end{exercise}
\end{document}