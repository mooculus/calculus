\documentclass{ximera}

\newcommand{\RR}{\mathbb R}
\renewcommand{\d}{\,d}
\newcommand{\dd}[2][]{\frac{d #1}{d #2}}
\renewcommand{\l}{\ell}
\newcommand{\ddx}{\frac{d}{dx}}
\newcommand{\dfn}{\textbf}
\newcommand{\eval}[1]{\bigg[ #1 \bigg]}


\title[Refresh:]{Partial Fraction Decomposition}
\author{Jim Talamo}
\outcome{Practice partial fraction decomposition.}
\begin{document}
\begin{abstract}
 Refresh on partial fraction decomposition.
\end{abstract}
%\maketitle

\begin{example}
Calculate $\int \frac{-2x^3+9}{x^4+9x^2} \d x$.  

\begin{explanation}
We first note that 

\[
\frac{-2x^3+9}{x^4+9x^2} = \frac{-2x^3+9}{x^2(x^2+9)}.
\] 

The factor $x^2$ is a repeated linear factor, and the factor $x^2+9$ is an irreducible quadratic factor, which leads us to look for constants $A, B, C,$ and $D$ for which 

\[
\frac{-2x^3+9}{x^2(x^2+9)} = \frac{A}{x^2}+\frac{B}{x}+\frac{Cx+D}{x^2+9}.
\]

Multiplying both sides by $x^2(x^2+9)$ produces

\[
-2x^3+9 = A(x^2+9)+Bx(x^2+9)+(Cx+D)x^2.
\]

By setting $x=0$, we immediately find that $A=1$.

There are no more convenient $x$-values that allow us to find the other constants, so we substitute $A=1$ and simplify.

\begin{align*}
-2x^3+9 &= 1\cdot(x^2+9)+Bx(x^2+9)+(Cx+D)x^2 \\
-2x^3+9 &= x^2+9+Bx^3+9Bx+Cx^3+Dx^2 \\
-2x^3 &= x^2+Bx^3+9Bx+Cx^3+Dx^2
\end{align*}

Now, collect like powers of $x$.
\begin{align*}
-2x^3 &= (B+C)x^3 +(D+1)x^2 +9Bx
\end{align*}

These polynomials are only equal if the coefficients of each like power of $x$ are equal, which gives us a system of equations.

\begin{align*}
-2 & = B+C & \longleftarrow \textrm{from comparing the coefficients of } x^3 \\
0 & = D+1 & \longleftarrow  \textrm{from comparing the coefficients of } x^2 \\
0 & = B & \longleftarrow  \textrm{from comparing the coefficients of } x 
\end{align*}

From this, we find $B=0$, $C=-2$, and $D=-1$.  Hence,

\[
\frac{-2x^3+9}{x^2(x^2+9)} = \frac{1}{x^2}-\frac{2x+1}{x^2+9},
\]
and we can substitute this into the original integral.

\begin{align*}
\int \frac{-2x^3+9}{x^4+9x^2} \d x &=  \int \frac{1}{x^2}-\frac{2x+1}{x^2+9} \d x \\
&=  \int \frac{1}{x^2}-\frac{2x}{x^2+9} +\frac{1}{x^2+9} \d x \\
&=  -\frac{1}{x}-\ln\left(x^2+9\right) -\frac{1}{3} \arctan\left(\frac{x}{3}\right) +C \\
\end{align*}
where we have used the fact $\int \frac{1}{x^2+a^2} = \frac{1}{a} \arctan\left(\frac{x}{a}\right)+C$, 
\end{explanation}

\end{example}


\begin{exercise}
Partial fraction decomposition is a helpful technique when we integrate rational functions. 


Compute $\int \frac{2x^2-4x+8}{x^3+4x} \d x$.

\[
\int \frac{2x^2-4x+8}{x^3+4x} \d x = \answer{2 \ln|x|-2\arctan\left(\frac{x}{2}\right)+C}
\]
(Use $C$ for the constant of integration)
\end{exercise}

\end{document}
