\documentclass{ximera}

\outcome{Understand how the derivative of an inverse function relates to the original derivative.}

\newcommand{\RR}{\mathbb R}
\renewcommand{\d}{\,d}
\newcommand{\dd}[2][]{\frac{d #1}{d #2}}
\renewcommand{\l}{\ell}
\newcommand{\ddx}{\frac{d}{dx}}
\newcommand{\dfn}{\textbf}
\newcommand{\eval}[1]{\bigg[ #1 \bigg]}


\title[Dig-In:]{The Inverse Function Theorem}

\begin{document}
\begin{abstract}
  We see the theoretical underpinning of finding the derivative of an
  inverse function at a point.
\end{abstract}
\maketitle

There is one catch to all the explanations given above where we
computed derivatives of inverse functions. To write something like
\[
\ddx(e^y)=e^y\cdot y'
\]
we need to know that the function $y$ \textit{has} a derivative.  The
\textit{Inverse Function Theorem} guarantees this.

\begin{theorem}[Inverse Function Theorem]\index{Inverse Function Theorem}\label{theorem:IFT}
If $f$ is a differentiable function that is one-to-one near $a$ and
$f'(a) \neq 0$, then
\begin{enumerate}
\item $f^{-1}(x)$ is \textbf{defined} for $x$ near $b=f(a)$,
\item $f^{-1}(x)$ is \textbf{differentiable} near $b=f(a)$, 
\item last, but not least:
  \[
  \eval{\ddx f^{-1}(x)}_{x=b}  = \frac{1}{f'(a)}\qquad\text{where}\qquad b = f(a).
  \]
\end{enumerate}
\begin{explanation}
  We will only explain the last result. We know
  \[
  f(f^{-1}(x)) = x,
  \]
  and now we use the chain rule to
  write
  \begin{align*}
  \ddx f(f^{-1}(x)) &= f'(f^{-1}(x)) (f^{-1})'(x)\\
  &=1.
  \end{align*}
  Solving for $(f^{-1})'(x)$ we see
  \[
  (f^{-1})'(x) = \frac{1}{f'(f^{-1}(x))}.
  \]
  This is what we have written above.
\end{explanation}
\end{theorem}

It is worth giving one more piece of evidence for the formula above,
this time based on increments in function, $\Delta f$, and increments in variable, $\Delta x$. Consider this plot of a function $f$ and its inverse:
\begin{image}
\begin{tikzpicture}
	\begin{axis}[
            xmin=-.5,xmax=6,ymin=-.5,ymax=6,
            axis lines=center,
             unit vector ratio*=1 1 1,
              width=4in,
            xlabel=$x$, ylabel=$y$,
            every axis y label/.style={at=(current axis.above origin),anchor=south},
            every axis x label/.style={at=(current axis.right of origin),anchor=west},
          ]        
          \addplot [very thick, penColor, smooth, domain=(-.5:5)] {e^x};
          \addplot [very thick, penColor2, samples=100, smooth, domain=(.002:6)] {ln(x)};
          \addplot [dashed, textColor, domain=(-.5:6)] {x};
          \node at (axis cs:1.2,4) [penColor] {$f$};
          \node at (axis cs:4,1.6) [penColor2] {$f^{-1}$};

          \addplot [draw=black,->] plot coordinates {(1,e) (1.6,e)};
	  \addplot [draw=black,->] plot coordinates {(1.6,e) (1.6,1.8*e)};
          \addplot[color=penColor3,fill=penColor3,only marks,mark=*] coordinates{(1,e)};  %% closed hole            
          \node at (axis cs:1.25,e) [below] {$\Delta x$};
          \node at (axis cs:1.54,3.8) [right] {$\Delta f$};

          \addplot [draw=black,->] plot coordinates {(1.8*e,1) (1.8*e,1.6)};
	  \addplot [draw=black,->] plot coordinates {(e,1) (1.8*e,1)};
          \addplot[color=penColor3,fill=penColor3,only marks,mark=*] coordinates{(e,1)};  %% closed hole            
          \node at (axis cs:5.9,1.3) [left] {$\Delta f^{-1}$};
          \node at (axis cs:3.8,0.6) [above] {$\Delta x$};
        \end{axis}
\end{tikzpicture}
%% \caption{A plot of $e^x$ and $\ln(x)$. Since they are inverse
%%   functions, they are reflections of each other across the line $y=x$.}
%% \label{plot:e^x lnx}
\end{image}
Since the graph of the inverse of a function is the reflection of the graph of the function over
the line $y=x$, we see that the increments are ``switched'' when
reflected. Hence we see that
\[
\dfrac{\Delta f^{-1}}{\Delta{x}} = \dfrac{\Delta x}{\Delta f}=\dfrac{1}{\dfrac{\Delta f}{\Delta x}}.
\]
Taking the limit as $\Delta x$ goes to $0$, we can obtain the expression for the derivative of $f^{-1}$. \\
\[
\dfrac{d f^{-1}}{d{x}} = \dfrac{1}{\dfrac{d f}{d x}}
\]

The inverse function theorem gives us a recipe for computing the
derivatives of inverses of functions at points.

\begin{example}
  Let $f$ be a differentiable function that has an inverse. In the
  table below we give several values for both $f$ and $f'$:
  \[
  \begin{array}{|c|c|c|}\hline
    x & f  & f' \\ \hline \hline
    2 & 0  & 2  \\ \hline
    3 &1  &5 \\ \hline
    4 & 3 & 0  \\ \hline
  \end{array}
  \]
  Compute
  \[
  \ddx f^{-1}(x)\;\text{at $x=1$.}
  \]
  \begin{explanation}
    From the table above we see that
    \[
    1 = f(\answer [given]{3}).
    \]
    Hence, by the inverse function theorem
    \[
    \left(f^{-1}\right)' (1)= \frac{1}{f'(\answer[given]{3})} = \answer[given]{\frac{1}{5}}.
    \]
  \end{explanation}
\end{example}

If one example is good, two are better:

\begin{example}
  Let $f$ be a differentiable function that has an inverse. In the
  table below we give several values for both $f$ and $f'$:
  \[
  \begin{array}{|c|c|c|}\hline
    x & f  & f' \\ \hline \hline
    2 & 0  & 2  \\ \hline
    3 & 1  &5 \\ \hline
    4 & 3 & 0  \\ \hline
  \end{array}
  \]
  Compute
  \[
  \left(f^{-1}\right)'(0)
  \]
  \begin{explanation}
    Note,
    \[
    \left(f^{-1}\right)'(0) = \ddx f^{-1}(x)\;\text{at $x=0$.}
    \]
    From the table above we see that
    \[
    0 = f(\answer[given]{2}).
    \]
    Hence, by the inverse function theorem
    \[
    \left(f^{-1}\right)'(0) = \frac{1}{f'(\answer[given]{2})} = \answer[given]{\frac{1}{2}}.
    \]
  \end{explanation}
\end{example}

Finally, let's see an example where the theorem does not apply.

\begin{example}
  Let $f$ be a differentiable function that has an inverse. In the
  table below we give several values for both $f$ and $f'$:
  \[
  \begin{array}{|c|c|c|}\hline
    x & f  & f' \\ \hline \hline
    2 & 0  & 2  \\ \hline
    3 & 1  & 4 \\ \hline
    4 & 3 & 0  \\ \hline
  \end{array}
  \]
  Compute
  \[
  \eval{\ddx f^{-1}(x)}_{x=3}
  \]
  \begin{explanation}
    From the table above we see that
    \[
    3 = f(\answer[given]{4}).
    \]
    Ah! But here, $f'(\answer[given]{4}) = \answer[given]{0}$, so we have no guarantee that the
    inverse exists near the point $x=3$, but even if it did the inverse would not be differentiable there.
      \end{explanation}
\end{example}


\end{document}
