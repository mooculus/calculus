\documentclass{ximera}

\newcommand{\RR}{\mathbb R}
\renewcommand{\d}{\,d}
\newcommand{\dd}[2][]{\frac{d #1}{d #2}}
\renewcommand{\l}{\ell}
\newcommand{\ddx}{\frac{d}{dx}}
\newcommand{\dfn}{\textbf}
\newcommand{\eval}[1]{\bigg[ #1 \bigg]}


\author{Jason Miller}
\license{Creative Commons 3.0 By-NC}


\outcome{}


\begin{document}
\begin{exercise}
Consider the improper integral 
\[
\int_{0}^{\infty} \frac{1}{2x-1} \d x
\]

How many integrals is it necessary to split this improper integral into in order to evaluate it? $\answer{3}$


\begin{feedback}[correct]
 The integrand has a vertical asymptote at the point $x=\frac{1}{2}$. So we need to split the integral around this point. 

\[
\int_{0}^{{\frac{1}{2}}^{-}} \frac{1}{2x-1} \d x + \int_{{\frac{1}{2}}^{+}}^{\infty} \frac{1}{2x-1} \d x 
\]

The second integral is improper for two reasons. The integrand has a vertical asymptote at the lower bound $x=\frac{1}{2}$ and the interval is infinite. So we need to split this integral into two pieces. It doesn't matter where we split the integral, so we choice a convenient point like $x=1$. 

Splitting the second integral and using limits we have

\[
\int_{0}^{\infty} \frac{1}{2x-1} \d x= \lim_{N \to {\frac{1}{2}}^{-}} \int_{0}^{N} \frac{1}{2x-1} \d x +
\lim_{M \to {\frac{1}{2}}^{+}} \int_{M}^{1} \frac{1}{2x-1} \d x + \lim_{P \to \infty} \int_{1}^{P} \frac{1}{2x-1} \d x
\]

Notice that we use different variables for each of the three limits.

\end{feedback} 
\end{exercise}
\end{document}
