\documentclass[hints,nooutcomes,noauthor]{ximera}

\newcommand{\RR}{\mathbb R}
\renewcommand{\d}{\,d}
\newcommand{\dd}[2][]{\frac{d #1}{d #2}}
\renewcommand{\l}{\ell}
\newcommand{\ddx}{\frac{d}{dx}}
\newcommand{\dfn}{\textbf}
\newcommand{\eval}[1]{\bigg[ #1 \bigg]}


\author{Bart Snapp}

\outcome{.}
\outcome{Interpert implicit equations as geometric sets.}

\title[Collaborate:]{Geometric sets}

\begin{document}
\begin{abstract}
  We investigate geometric sets determined by implicit equations.
\end{abstract}
\maketitle

\textbf{Work in groups of 3--4, writing your answers on a separate
  sheet of paper.}

%% \begin{problem}
%%   Working in $\R^2$, give an implicit equation for a circle of radius
%%   $2$ centered at the origin.
%% \end{problem}


%% \begin{problem}
%%   Working in $\R^3$, give an implicit equation for a sphere of radius
%%   $2$ centered at the origin.
%% \end{problem}


%% \begin{problem}
%%   Explain the connection from your answers for the previous questions
%%   to the distance formula in $\R^n$.
%% \end{problem}


%% \begin{problem}
%%   Working in $\R^3$, give an implicit equation for a cylinder of
%%   infinite height whose intersection with the plane $z= -3$ is a circle
%%   of radius $2$ centered at the point $(-1,2,-3)$.
%% \end{problem}

%% \begin{problem}
%%   What sort of geometric object does one expect to get when looking at
%%   the solutions to a single equation in $\R^2$? Explain your reasoning.
%% \end{problem}

%% \begin{problem}
%%   What sort of geometric object does one expect to get when looking at
%%   the solutions to a single equation in $\R^3$? Explain your reasoning.
%% \end{problem}


%% \begin{problem}
%%   What sort of geometric object does one expect to get when looking at
%%   the solutions to a single equation in $\R^n$? Explain your reasoning.
%% \end{problem}


%% \begin{problem}
%%   What sort of geometric object does one expect to get when looking at
%%   the simultaneous solutions to two equations in $\R^2$? What about
%%   $\R^3$? What about $\R^n$? Explain your reasoning.
%% \end{problem}

%% \begin{problem}
%%   We claim that the equations
%%   \[
%%   \vec{p}(t)=\begin{cases}
%%     x(t) = 1+3\cos(t)\\
%%     y(t) = 2+3\sin(t)
%%   \end{cases}
%%   \]
%%   parameterize a circle of radius $3$ centered at the point $(1,2)$ as
%%   $t$ runs from $0$ to $2\pi$.
%%   \begin{enumerate}
%%     \item Write the implicit equation of a circle of radius $3$
%%       centered at $(1,2)$.
%%     \item Use your equation to confirm our claim that $\vec{p}(t)$
%%       parameterizes the desired circle.
%%     \item Let
%%       \[
%%       \vec{q}(t)=\begin{cases}
%%       x(t) = 1+3\cos(2t)\\
%%       y(t) = 2+3\sin(2t)
%%       \end{cases}
%%       \]
%%       and let $t$ run from $0$ to $\pi$. Compare and contrast
%%       $\vec{p}(t)$ and $\vec{q}(t)$.
%%   \end{enumerate}
%% \end{problem}


%% \begin{problem}
%%   We claim that the equations
%%   \[
%%   \vec{P}(\theta,\varphi)=\begin{cases}
%%     x(\theta,\varphi) = 3\cos(\theta)\sin(\varphi)\\
%%     y(\theta,\varphi) = 3\sin(\theta)\sin(\varphi)\\
%%     z(\theta,\varphi) = 3 \cos(\varphi)
%%   \end{cases}
%%   \]
%%   parameterize a sphere of radius $3$ centered at the origin as
%%   $\theta$ runs from $0$ to $2\pi$ and $\varphi$ runs from $0$ to $\pi$.
%%   \begin{enumerate}
%%     \item Write the implicit equation of a sphere of radius $3$
%%       centered at $(0,0,0)$.
%%     \item Use your equation to confirm our claim that $\vec{P}(\theta,\varphi)$
%%       parameterizes the desired sphere.
%%     \item Give values for $\theta$ and $\varphi$ corresponding to the
%%       following points: $(0,0,3)$, $(0,0,-3)$, $(3,0,0)$, and
%%       $(0,-3,0)$.
%%   \end{enumerate}
%% \end{problem}


\section{Geometry disguised as algebra}

\begin{problem}
  Consider the equations:
  \begin{align*}
    x+y &= 6\\
    x^2+y^2+z^2&=18
  \end{align*}
  Find a solution to these equations.
  \begin{hint}
    Guess-and-check is not a bad method for this problem.
  \end{hint}
\end{problem}

\begin{problem}
  Explain what the equations
  \begin{align*}
    x+y &= 6\\
    x^2+y^2+z^2&=18
  \end{align*}
  describe from a geometric point of view.
\end{problem}

\begin{problem}
  Use geometry to explain why the equations
  \begin{align*}
    x+y &= 6\\
    x^2+y^2+z^2&=18
  \end{align*}
  have exactly one solution where $x$, $y$, and $z$ are real numbers.
\end{problem}

%% \begin{problem}
%%   Consider the following parametric formula:
%%   \[
%%   \vec{p}(t)=\begin{cases}
%%     x(t) = \frac{\cos(t)+\sin(t)}{2}\\
%%     y(t) = \frac{-\cos(t)-\sin(t)}{2}\\
%%     z(t) = \frac{\cos(t)-\sin(t)}{\sqrt{2}}
%%   \end{cases}
%%   \]
%%   Without using any devices other than a pencil, paper, and your
%%   brain:
%%   \begin{enumerate}
%%   \item Does this plot a curve or a surface?
%%   \item Show this plot lives ``inside'' the set of points satisfying
%%     \[
%%     x+y=0
%%     \]
%%   \item Make your own sketch of this curve to the best of your
%%     ability. Try it a couple times. Draw it carefully.
%%   \item Give a mathematical explanation as to how you know your plot
%%     is correct.
%%   \end{enumerate}
%%   \begin{hint}
%%     Use either the distance formula or a familiar surface to help you. 
%%   \end{hint}
%% \end{problem}



\section{Thinking about a tetrahedron}


\begin{problem}
  Sketch the triangle in $\R^3$ whose vertices are the intersections
  of the plane
  \[
  20x + 15y + 12z = 60
  \]
  and the coordinate axes.
\end{problem}



\begin{problem}
  Compute the volume of the tetrahedron (triangular-based pyramid) of
  in $\R^3$ bounded by the planes $x=0$, $y=0$, $z=0$, and the plane
  $20x + 15y + 12z = 60$.
  \begin{hint}
    Recall that the volume of a cone (pyramid!) is given by:
    \[
    V = \left(\frac{1}{3}\right) (\text{area of base})(\text{height})
    \]
  \end{hint}
\end{problem}



\begin{problem}
  Compute the area of the triangle in $\R^3$ whose vertices are the intersections of the plane $20x + 15y + 12z = 60$ and the coordinate axes.
  \begin{hint}
    For now, use your old friend:
    \[
    A = \frac{bh}{2}
    \]
    and use calculus to minimize the distance between the line
    connecting $(3,0,0)$ to $(0,4,0)$, and the point $(0,0,5)$.
  \end{hint}
\end{problem}




\end{document}
