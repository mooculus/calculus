\documentclass{ximera}

\newcommand{\RR}{\mathbb R}
\renewcommand{\d}{\,d}
\newcommand{\dd}[2][]{\frac{d #1}{d #2}}
\renewcommand{\l}{\ell}
\newcommand{\ddx}{\frac{d}{dx}}
\newcommand{\dfn}{\textbf}
\newcommand{\eval}[1]{\bigg[ #1 \bigg]}


\outcome{}

% Synthesis of everything we've learned

% We've learned so much stuff!
% Can't we put this all together somehow?

% Let's make a concept map of everything we've learned so far.

% Holes, point discontinuities, review from limits at the beginning


\title[Break-Ground:]{Wanted: graphing proceedure}

\begin{document}
\begin{abstract}
Here we see a dialogue where two young mathematicians discussing how
to sketch the plots of functions.
\end{abstract}
\maketitle

Check out this dialogue between two calculus students (based on a true
story):

\begin{dialogue}
\item[Devyn] Riley, I've been thinking about the derivative. 
\item[Riley] It's all about change. It's some ``change-detector'' tool
  for math.
\item[Devyn] I know!  What's crazy is that you can use it as a tool
  for sniffing out dirt on functions.
\item[Riley] $f'$ tells us increasing or decreasing.
\item[Devyn] $f''$ tells us concavity.
\item[Riley] From just that we know all local maxs and mins.
\item[Devyn] And if we use limits, we can find asymptotes!
\item[Riley] You know, I'd like to make up a proceedure based on all
  these facts, that would tell me how to plot functions.
\item[Devyn] Me too! Let's get to work!
\end{dialogue}




%% \begin{xarmaBoost}
%%   Write down at least \textbf{five} questions for this lecture. After
%%   you have your questions, label them as ``Level 1,'' ``Level 2,'' or
%%   ``Level 3'' where:
%% \begin{description}
%% \item[Level 1] Means you know the answer, or know exactly how to do
%%   this problem.
%% \item[Level 2] Means you think you know how to do the problem.
%% \item[Level 3] Means you have no idea how to do the problem.
%% \end{description}
%% \begin{freeResponse}
%% \end{freeResponse}
%% \end{xarmaBoost}


\end{document}
