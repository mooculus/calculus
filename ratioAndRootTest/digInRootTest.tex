\documentclass{ximera}

\newcommand{\RR}{\mathbb R}
\renewcommand{\d}{\,d}
\newcommand{\dd}[2][]{\frac{d #1}{d #2}}
\renewcommand{\l}{\ell}
\newcommand{\ddx}{\frac{d}{dx}}
\newcommand{\dfn}{\textbf}
\newcommand{\eval}[1]{\bigg[ #1 \bigg]}

\author{Bart Snapp}
\outcome{Use the root test to determine if a series diverges or converges.}
\outcome{Determine when to use the ratio or root test.}

\title[Dig-In:]{The root test}

\begin{document}
\begin{abstract}
Some infinite series can be compared to geometric series.
\end{abstract}
\maketitle

We learned that the ratio test is a powerful technique based on the
concept of recognizing when a series is ``approximately'' geometric.
If
\[
\lim_{k \to \infty} \left|\frac{a_{k+1}}{a_k}\right| = L,
\]
then the ``tail'' of the series looks like a geometric series of ratio
$L$, and follows the same convergence and divergence behavior as a
geometric series when $L\neq 1$.  The \textit{root test} uses a similar idea in a
slightly different situation.
\begin{theorem}[The Root Test]\index{root test}
  If $\sum_{k=0}^\infty a_k$ is an infinite series, and $\lim_{k \to \infty} \sqrt[k]{|a_k|} = L$, then 
  \begin{itemize}
  \item $\sum_{k=0}^\infty a_k$ converges if $0 \leq L < 1$.
  \item $\sum_{k=0}^\infty a_k$ diverges if $L>1$ or is infinite.
  \item the test is inconclusive if $L=1$; the series could diverge or converge.
  \end{itemize}
\end{theorem}
Notice that the conclusion of the root test follows exactly the same
form as the ratio test.  It does so for exactly the same reason:
\begin{quote}
  If $\sqrt[k]{a_k} \approx L$ for large $k$ and $L \neq 1$, then $a_k \approx L^k$
  for large $k$, which says that the tail of $a_k$ behaves like
  a geometric series with ratio $L$.
\end{quote}
Again, we do not give a formal proof in this course (but if you are
interested, you can find a proof online!)

When using the ratio test, there are two other subjects we should keep in mind. 
The first subject is the rules of exponents.  If you can't remember these rules, 
pause here and look them up to refresh your memory!  The second subject is 
L'H\^{o}pital's rule, which we used to evaluate limits like
\[
\lim_{k \to \infty} \sqrt[k]{k} = 1.
\]
This limit, in particular, can appear frequently!


\begin{example}
  Consider 
  \[
  \sum_{k=4}^\infty \frac{k^5}{k^k}
  \]
  Discuss the convergence of this series.
  \begin{explanation}
    We will attempt to use the root test. Setting
    $a_k=\frac{k^5}{k^k}$. Write with me.
    \[
    \lim_{k \to \infty} \sqrt[k]{a_k} = \answer[given]{0}	
    \]
    So, the root test
	  \wordChoice{
	   \choice[correct]{says the series is convergent}
	   \choice{says the series is divergent}
	   \choice{gives no information in this case, but we know the series is convergent using the integral test}
	   \choice{gives no information in this case, but we know the series is divergent using the integral test}}.		
	  \begin{hint}
            \begin{align*}
	      \lim_{k \to \infty} \sqrt[k]{a_k} &= \lim_{k \to \infty} \sqrt[k]{\frac{k^5}{k^k}}\\
	      &=\lim_{k \to \infty} \frac{(k^{\frac{1}{k}})^5}{k}\\
	      &=0 \quad\text{since $\lim_{k \to \infty} k^\frac{1}{k} = 1$}
	    \end{align*}
	    So the series is convergent by the root test.
	  \end{hint}
  \end{explanation}
\end{example}

\begin{example}
  Consider 
  \[
  \sum_{k=3}^\infty \frac{2^k}{k^2}
  \]
  Discuss the convergence of this series.
  \begin{explanation}
    We will attempt to use the root test. Setting
    $a_k=\frac{2^k}{k^2}$. Write with me.
    \[
    \lim_{k \to \infty} \sqrt[k]{\frac{2^k}{k^2}} = \answer[given]{2}	
    \]
    So, the root test
	  \wordChoice{
	   \choice{says the series is convergent}
	   \choice[correct]{says the series is divergent}
	   \choice{gives no information in this case, but we know the series is convergent through some other method}
	   \choice{gives no information in this case, but we know the series is divergent through some other method}}.		
	  \begin{hint}
            \begin{align*}
	      \lim_{k \to \infty} \sqrt[k]{a_k} &= \lim_{k \to \infty} \sqrt[k]{\frac{2^k}{k^2}}\\
	      &=\lim_{k \to \infty} \frac{2}{(k^{1/k})^2}\\
	      &=2 \quad\text{since $\lim_{k \to \infty} k^\frac{1}{k} = 1$}
	    \end{align*}
	    So the series is divergent by the root test.
	  \end{hint}
  \end{explanation}
\end{example}

\begin{example}
  Consider 
  \[
  \sum_{k=1}^\infty \left(\frac{k^2-k}{k^2+k}\right)^k
  \]
  Discuss the convergence of this series.
  \begin{explanation}
    We will attempt to use the root test. Setting
    $a_k=\left(\frac{k^2-k}{k^2+k}\right)^k$. Write with me.
    \[
    \lim_{k \to \infty} \sqrt[k]{\left(\frac{k^2-k}{k^2+k}\right)^k} = \answer[given]{1}	
    \]
    So, the root test
	  \wordChoice{
	   \choice{says the series is convergent}
	   \choice{says the series is divergent}
	   \choice{gives no information in this case, but we know the series is convergent through some other method}
	   \choice[correct]{gives no information in this case, but we know the series is divergent through some other method}}.		
	  \begin{hint}
            \begin{align*}
	      \lim_{k \to \infty} \sqrt[k]{a_k} &= \lim_{k \to \infty} \sqrt[k]{\left(\frac{k^2-k}{k^2+k}\right)^k}\\
	      &=\lim_{k \to \infty}\frac{k^2-k}{k^2+k}\\
	      &=1
	    \end{align*}
	    So the root test gives no information.  However, we know this series diverges by the divergence test.
	  \end{hint}
  \end{explanation}
\end{example}


\begin{example}
  Consider 
  \[
  \sum_{k=1}^\infty \frac{1}{k^2}
  \]
  Discuss the convergence of this series.
  \begin{explanation}
    We will attempt to use the root test. Setting
    $a_k=\frac{1}{k^2}$. Write with me.
    \[
    \lim_{k \to \infty} \sqrt[k]{\frac{1}{k^2}} = \answer[given]{1}	
    \]
    So, the root test
	  \wordChoice{
	   \choice{says the series is convergent}
	   \choice{says the series is divergent}
	   \choice[correct]{gives no information in this case, but we know the series is convergent through some other method}
	   \choice{gives no information in this case, but we know the series is divergent through some other method}}.		
	  \begin{hint}
            \begin{align*}
	      \lim_{k \to \infty} \sqrt[k]{a_k} &= \lim_{k \to \infty} \sqrt[k]{\frac{1}{k^2}}\\
	      &=\lim_{k \to \infty}\left(k^{1/k}\right)^{-2}\\
	      &=1 \quad\text{since $\lim_{k \to \infty} k^\frac{1}{k} = 1$}
	    \end{align*}
	    So the root test gives no information.  However, we know
            this series converges by the $p$-test.
	  \end{hint}
  \end{explanation}
\end{example}

When analyzing a series for convergence or divergence, choosing which 
test to use is often the most difficult task we face.  Generally, the root test 
is most useful when you have a lot of powers and no factorials.  Anytime 
you see a factorial is a pretty good time to try the ratio test.  Of course, 
don't forget to use the divergence test first!







\end{document}


