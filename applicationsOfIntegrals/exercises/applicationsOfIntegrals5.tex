\documentclass{ximera}

\newcommand{\RR}{\mathbb R}
\renewcommand{\d}{\,d}
\newcommand{\dd}[2][]{\frac{d #1}{d #2}}
\renewcommand{\l}{\ell}
\newcommand{\ddx}{\frac{d}{dx}}
\newcommand{\dfn}{\textbf}
\newcommand{\eval}[1]{\bigg[ #1 \bigg]}


%\outcome{Given a velocity function, calculate displacement and distance traveled.}
%\outcome{Given a velocity function, find the position function.}
%\outcome{Given an acceleration function, find the velocity function.}
%\outcome{Understand the difference between displacement and distance traveled.}
%\outcome{Understand the relationship between position, velocity and acceleration.}

\author{Nela Lakos \and Kyle Parsons}

\begin{document}
\begin{exercise}

Consider a particle moving along a line with acceleration and initial velocity given by
\begin{align*}
a(t) &= 1-2t\\
v(0) &= 6
\end{align*}
for $0\leq t\leq4$.

The  velocity function of the particle as a function of $t$ is
\[
v(t) = \answer{6-t^2+t}.
\]

The total \emph{distance} that the particle travels on the interval $[0,4]$ is $\answer{\frac{49}{3}}$.

\end{exercise}
\end{document}