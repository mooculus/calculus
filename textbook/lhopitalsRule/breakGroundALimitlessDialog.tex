\documentclass{ximera}

\newcommand{\RR}{\mathbb R}
\renewcommand{\d}{\,d}
\newcommand{\dd}[2][]{\frac{d #1}{d #2}}
\renewcommand{\l}{\ell}
\newcommand{\ddx}{\frac{d}{dx}}
\newcommand{\dfn}{\textbf}
\newcommand{\eval}[1]{\bigg[ #1 \bigg]}


\outcome{}


\title[Break-Ground:]{A limitless dialog}

\begin{document}
\begin{abstract}
Here we see a dialog where limits are computed using derivatives.
\end{abstract}
\maketitle

Check out this dialog between two calculus students (based on a true
story):

\begin{itemize}
\item[\textbf{Devyn}] Yo Riley, guess what I did last night?
\item[\textbf{Riley}] What?
\item[\textbf{Devyn}] I was doing some calculus.
\item[\textbf{Riley}] That. Is. Awesome.
\item[\textbf{Devyn}] I know! Anyway, I noticed something kinda funny. I
  think you can sometimes take limits by taking the derivative of the
  numerator and the denominator.
\item[\textbf{Riley}] That's crazy.
\item[\textbf{Devyn}] I know! But check it:
  \begin{align*}
    \lim_{x\to 0} \frac{\sin(x)}{x} &= \lim_{x\to 0} \frac{\dd{x}\sin(x)}{\dd{x}x}\\
    &= \lim_{x\to 0} \frac{\cos(x)}{1}\\
    &=1.
  \end{align*}
  \item[\textbf{Riley}] Woah. That. Is. Awes\dots weird. Hmmm, but it seems like
    cheating. Wait, it doesn't always work, check this out:
    \[
    \lim_{x\to 0} \frac{x^2+1}{x+1} = 1,
    \]
    but
    \begin{align*}
      \lim_{x\to 0} \frac{\dd{x}\left(x^2+1\right)}{\dd{x}\left(x+1\right)} &=
      \lim_{x\to 0} \frac{2x}{1} \\
      &=2.
    \end{align*}
\end{itemize}

\begin{problem}
  Write down \textbf{five} examples where this ``trick'' works, and
  \textbf{five} examples when it doesn't work.
  \begin{freeResponse}
\end{freeResponse}
\end{problem}

\begin{problem}
  What is the pattern for when the ``trick'' works and when it does not work?
  \begin{freeResponse}
\end{freeResponse}
\end{problem}



\begin{xarmaBoost}
  Write down at least \textbf{five} questions for this lecture. After
  you have your questions, label them as ``Level 1,'' ``Level 2,'' or
  ``Level 3'' where:
\begin{description}
\item[Level 1] Means you know the answer, or know exactly how to do
  this problem.
\item[Level 2] Means you think you know how to do the problem, or will
  soon learn how to do the problem.
\item[Level 3] Means you have no idea how to do the problem.
\end{description}
\begin{freeResponse}
\end{freeResponse}
\end{xarmaBoost}


\end{document}
