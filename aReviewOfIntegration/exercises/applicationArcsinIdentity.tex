\documentclass{ximera}
\newcommand{\RR}{\mathbb R}
\renewcommand{\d}{\,d}
\newcommand{\dd}[2][]{\frac{d #1}{d #2}}
\renewcommand{\l}{\ell}
\newcommand{\ddx}{\frac{d}{dx}}
\newcommand{\dfn}{\textbf}
\newcommand{\eval}[1]{\bigg[ #1 \bigg]}

\author{Jim Talamo}
\license{Creative Commons 3.0 By-NC}
\outcome{Practice prerequisite skills in the context of a harder problem}
\outcome{Apply knowledge of substitution}
\begin{document}

(\emph{Source: The College Mathematics Journal} 32, 5, Nov 2001)

\begin{exercise}
The following exercise will provide an argument for the identity:

\[
2 \arcsin(\sqrt{x})-\arcsin(2x-1)= \frac{\pi}{2}
\]

Consider the indefinite integral: 

\[
I = \int \frac{1}{\sqrt{x-x^2}} \d x
\]


\begin{exercise}
One way to compute $I$ is to set $u=2x-1$ and write the integral $I$ as an integral with respect to $u$:
\[
\int \frac{1}{\sqrt{x-x^2}} \d x = \int\frac{1}{ \answer{\sqrt{1-u^2}}} \d u
\]
\begin{hint}
You will need to do some algebra!
\end{hint}

\begin{exercise}
Evaluate this integral in $u$ and reverse the substitution to write $\int \frac{1}{\sqrt{x-x^2}} \d x$ in terms of $x$:

(Use $+C$ for the constant of integration)
\[
I = \answer{\arcsin(2x-1)+C}
\]

\begin{hint}
You may find the result $\int \frac{1}{\sqrt{a^2-x^2}} \d x = \arcsin\left(\frac{x}{a}\right)$ helpful.
\end{hint}

\end{exercise}

\begin{exercise}
Another way to compute $I$ is to set $v=\sqrt{x}$ and write $I$ as an integral with respect to $v$.

\[
\int \frac{1}{\sqrt{x-x^2}} \d x  = \int\frac{2}{ \answer{\sqrt{1-v^2}}} \d v
\]
\begin{hint}
You will need to do some algebra! Note that by definition that $v\geq 0$, so $\sqrt{v^2} = v$ (recall that usually $\sqrt{v^2} = |v|$ because by convention, the square root is positive!)

If you work the substitution correctly, you should obtain:

\[
\int \frac{1}{\sqrt{x-x^2}} \d x  = \int\frac{\answer{2v}}{ \answer{\sqrt{v^2-v^4}}} \d v
\]

Factor a $v^2$ out of the denominator and simplify!
\end{hint}


\end{exercise}

\begin{exercise}
Evaluate this integral in $v$ and reverse the substitution to write $\int \frac{1}{\sqrt{x-x^2}} \d x$ in terms of $x$:

(Use $+C$ for the constant of integration)
\[
I = \answer{2\arcsin(\sqrt{x})+C}
\]

\end{exercise}
\end{exercise}

You have now computed $ \int \frac{1}{\sqrt{x-x^2}} \d x$ two different ways!  The antiderivatives $2\arcsin(\sqrt{x})$ and $\arcsin(2x-1)$ above can differ only by a constant.

Using this observation:
\[
2\arcsin(\sqrt{x})-\arcsin(2x-1) = \answer{\frac{\pi}{2}}
\] 

\begin{hint}
Since these antiderivatives differ by a constant, we have:
\[
2\arcsin(\sqrt{x})-\arcsin(2x-1) = C
\]
and this constant $C$ must be the same \emph{no matter what x-value we use} (this is precisely what it means for $C$ to be a constant!)

Set $x=1$ above and find $C$.

\end{hint}

\end{exercise}

\end{document}