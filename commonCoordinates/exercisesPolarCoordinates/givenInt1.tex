\documentclass{ximera}

\newcommand{\RR}{\mathbb R}
\renewcommand{\d}{\,d}
\newcommand{\dd}[2][]{\frac{d #1}{d #2}}
\renewcommand{\l}{\ell}
\newcommand{\ddx}{\frac{d}{dx}}
\newcommand{\dfn}{\textbf}
\newcommand{\eval}[1]{\bigg[ #1 \bigg]}


\author{Jim Fowler \and Bart Snapp}

\outcome{Work in polar coordinates.}
\outcome{Compute double integrals in polar coordinates.}

\begin{document}
\begin{exercise}
  Suppose $f:\R\to\R$ is a function with the property that:
  \[
  \int_0^4 f(t) \d t = \pi
  \]
  Let $C = \{(x,y): x^2+y^2 \le 4\}$. Compute:
  \[
  \iint_C f(x^2 + y^2) \d A
  \begin{prompt}
    = \answer{\pi^2}
  \end{prompt}
  \]
  \begin{hint}
    Start by converting to polar coordinates.
  \end{hint}
  \begin{hint}
    Write with me:
    \[
    \iint_C f(x^2 + y^2) \d A = \int_0^{\answer{2\pi}}\int_0^{\answer{2}} f(r^2) r \d r \d \theta
    \]
  \end{hint}
  \begin{hint}
    Now let $t = r^2$
  \end{hint}
  \begin{hint}
    So:
    \begin{align*}
      t &= \answer{r^2}\\
      \d t &= \answer{2 r} \d r\\
      \d r &= \frac{\d t}{\answer{2r}}
    \end{align*}
  \end{hint}
  \begin{hint}
    Now we have:
    \[
    \int_0^{\answer{2\pi}}\int_0^{\answer{2}} f(r^2) r \d r \d \theta = \int_0^{2\pi}\int_0^{\answer{4}} \frac{f(t)}{\answer{2}} \d t \d \theta
    \]
  \end{hint}
\end{exercise}
\end{document}
