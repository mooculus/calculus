\documentclass{ximera}

\newcommand{\RR}{\mathbb R}
\renewcommand{\d}{\,d}
\newcommand{\dd}[2][]{\frac{d #1}{d #2}}
\renewcommand{\l}{\ell}
\newcommand{\ddx}{\frac{d}{dx}}
\newcommand{\dfn}{\textbf}
\newcommand{\eval}[1]{\bigg[ #1 \bigg]}


\author{Bart Snapp}

\outcome{Use iterated integrals to compute multiple integrals.}
\outcome{Apply Fubini's Theorem.}

\begin{document}
\begin{exercise}
  Consider the region:
  \begin{image}
    \begin{tikzpicture}
      \begin{axis}[
          tick label style={font=\scriptsize},axis y line=middle,axis x line=middle,name=myplot,axis on top,%
	  ymin=-.5,ymax=1.1,%
	  xmin=-.5,xmax=1.1%
        ]
        \draw (axis cs: .5,.4) node {$R$}
	(axis cs: .5,.8) node [rotate=26] {\scriptsize $y=\sqrt{x}$}
	(axis cs: .75,.16) node [rotate=29] {\scriptsize $y=x^4$};
        
        \addplot [penColor,ultra thick, smooth,domain=0:1.05,samples=20] ({x},{x^4});
        \addplot [penColor,ultra thick, smooth,domain=0:1.05,samples=20] ({x^2},{x});
      \end{axis}
      
      \node [right] at (myplot.right of origin) {\scriptsize $x$};
      \node [above] at (myplot.above origin) {\scriptsize $y$};
    \end{tikzpicture}
  \end{image}
  Set-up two iterated integrals that will compute the area of $R$,
  one that integrates with respect to $y$ and then $x$, and another
  that integrates with respect to $x$ and then $y$.
  \begin{prompt}
    \[
    \int_{\answer{0}}^{\answer{1}} \int_{\answer{x^4}}^{\answer{\sqrt{x}}} \d y \d x
    \]
    \[
    \int_{\answer{0}}^{\answer{1}}\int_{\answer{y^2}}^{\answer{y^{1/4}}} \d x \d y
    \]
  \end{prompt}
\end{exercise}
\end{document}
