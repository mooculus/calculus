\documentclass[handout]{ximera}
%handout:  for handout version with no solutions or instructor notes
%handout,instructornotes:  for instructor version with just problems and notes, no solutions
%noinstructornotes:  shows only problem and solutions

%% handout
%% space
%% newpage
%% numbers
%% nooutcomes

%I added the commands here so that I would't have to keep looking them up
%\newcommand{\RR}{\mathbb R}
%\renewcommand{\d}{\,d}
%\newcommand{\dd}[2][]{\frac{d #1}{d #2}}
%\renewcommand{\l}{\ell}
%\newcommand{\ddx}{\frac{d}{dx}}
%\everymath{\displaystyle}
%\newcommand{\dfn}{\textbf}
%\newcommand{\eval}[1]{\bigg[ #1 \bigg]}

%\begin{image}
%\includegraphics[trim= 170 420 250 180]{Figure1.pdf}
%\end{image}

%add a ``.'' below when used in a specific directory.

\newcommand{\RR}{\mathbb R}
\renewcommand{\d}{\,d}
\newcommand{\dd}[2][]{\frac{d #1}{d #2}}
\renewcommand{\l}{\ell}
\newcommand{\ddx}{\frac{d}{dx}}
\newcommand{\dfn}{\textbf}
\newcommand{\eval}[1]{\bigg[ #1 \bigg]}




\author{Jim Talamo}


\title[]{The washer method}

\begin{document}
\begin{abstract}
\end{abstract}
\maketitle

\vspace{-0.9in}

\section{Discussion Questions}

\begin{problem}
Answer the following questions. 
\begin{enumerate}
\item[I.] The region bounded by $y=x$, $y=3x$, and $x=2$ is revolved about the line $x=5$. We wish to compute the volume of the resulting solid by integrating with respect to $x$. How many integrals are needed to compute the volume of the solid if the Washer Method is used?

\item[II.]  The region bounded by $y=\sqrt{x}$, $y=0$, $x=1$ and $x=4$ is revolved about the line $x=8$.  We wish to compute the volume of the resulting solid using the Washer Method. Which variable should we integrate with respect to in order to compute the volume?

\item[III.] The region bounded by $y=-e^x$, $y=0$, $x=1$ and $x=2$ is rotated about either a horizontal or a vertical line. Suppose that the volume of the resulting solid can be calculated by using the washer method and integrating with respect to $y$. Was the axis of rotation horizontal or vertical?
\end{enumerate}
\end{problem}

\begin{freeResponse}
I. Since we are revolving around a vertical line and integrating with respect to $x$, our slices need to be vertical as well. This means that we should use the Shell method.

II. Since we are revolving around a vertical line and using the Washer method, our slices must be horizontal. This means that we should use the Washer method.

III. Since we are integrating with respect to $y$, the slices used to form the shells must be horizontal. This means that the axis of rotation must have been horizontal.
\end{freeResponse}

\section{Group Work}

\begin{problem}
The region $R$ bounded by the curves $y=4-x^2$ and $y=2-x$ is shown below.

\begin{image}
 \begin{tikzpicture}
            	\begin{axis}[
            		domain=-2.5:8.5, ymax=4.5,xmax=3.2, ymin=-1.5, xmin=-2.2,
            		axis lines =center, xlabel=$x$, ylabel=$y$,
            		every axis y label/.style={at=(current axis.above origin),anchor=south},
            		every axis x label/.style={at=(current axis.right of origin),anchor=west},
            		axis on top,
            		]
                      
            	\addplot [draw=penColor,very thick,smooth] {4-x^2};
            	\addplot [draw=penColor2,very thick,smooth] {2-x};
                       
            	\addplot [name path=A,domain=-1:2,draw=none] {4-x^2};   
            	\addplot [name path=B,domain=-1:2,draw=none] {2-x};
            	\addplot [fillp] fill between[of=A and B];
	                
            	\node at (axis cs:5,1.2) [penColor2] {$y=4-x^2$};
		\node at (axis cs:4.5,2.15) [penColor] {$y=2-x$};

            	\end{axis}
            \end{tikzpicture}
\end{image}

\begin{itemize}
\item[I.] Calculate the volume of the solid of revolution formed when $R$ is revolved about the $x$-axis.
\item[II.] Set up, but do not evaluate, an integral or sum of integrals that would give the volume of the solid of revolution obtained when $R$ is revolved about the line $y=5$.
\item[III.] Set up, but do not evaluate, an integral or sum of integrals that would give the volume of the solid of revolution obtained when $R$ is revolved about the line $x=3$.
\item[IV.] The base of a solid is the region $R$.  Cross sections taken perpendicular to the $x$-axis are squares.  Set up, but do not evaluate, and integral or sum of integrals that would give the volume of the solid.
\end{itemize}
\end{problem}


\end{document}
