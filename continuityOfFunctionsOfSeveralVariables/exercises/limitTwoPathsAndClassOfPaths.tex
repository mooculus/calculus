\documentclass{ximera}

\author{Jim Talamo}

\newcommand{\RR}{\mathbb R}
\renewcommand{\d}{\,d}
\newcommand{\dd}[2][]{\frac{d #1}{d #2}}
\renewcommand{\l}{\ell}
\newcommand{\ddx}{\frac{d}{dx}}
\newcommand{\dfn}{\textbf}
\newcommand{\eval}[1]{\bigg[ #1 \bigg]}


\outcome{Evaluate limits of functions of several variables.}
\newcommand{\point}[1]{\left(#1\right)} %this allows \vector{ to be changed to \point{ with a quick find and replace
\newcommand{\pt}[1]{\mathbf{#1}} %this allows \vec{ to be changed to \pt{ with a quick find and replace
\newcommand{\Lim}[2]{\lim_{#1 \to #2}}

\begin{document}
\begin{exercise}
 
Consider the function $f(x,y) = \frac{2x^3y-4x^2y^2}{2x^4+y^4}$.

The domain of this function is \wordChoice{\choice{$\R$}\choice{$\left\{ z \in \R \big| z \neq 0 \right\}$}\choice{$\R^2$}\choice[correct]{$\left\{ (x,y) \in \R^2 \big| (x,y) \neq (0,0) \right\}$}}.

We now examine whether $\Lim{\vector{x,y}}{\vector{0,0}} f(x,y)$ exists.  By noting that the degree of each term in the numerator and denominator are the same, we can try to choose a type of path along which we can induce algebraic cancellation.

Which type of path would likely be a good choice?

\begin{multipleChoice}
\choice[correct]{lines}
\choice{planes}
\choice{parabolas}
\choice{circles}
\end{multipleChoice}

\begin{exercise}
Let's try to analyze along two specific lines.

\begin{itemize}
\item Along $x=0$, we find that 

\[
f(x,y) = f(0,y) =\frac{2\left(\answer{0}\right)^3y-4\left(\answer{0}\right)^2y^2}{2\left(\answer{0}\right)^4+y^4} = \answer{0}, \textrm{ for } y \neq 0.
\]

Thus, $f(x,y) \to \answer{0}$ as $\vector{x,y} \to \vector{0,0}$ along the path $x=0$.

\item Along $y=0$, we find that 

\[
f(x,y) = f\left(\answer{x},\answer{0}\right) = \answer{0} , \textrm{ for } x \neq 0.
\]

Thus, $f(x,y) \to \answer{0}$ as $\vector{x,y} \to \vector{0,0}$ along the path $y=0$.
\end{itemize}

Is this enough to conclude that $\Lim{\vector{x,y}}{\vector{0,0}}$ exists?

\begin{multipleChoice}
\choice{Yes}
\choice[correct]{No}
\end{multipleChoice}

\begin{exercise}
Note that \emph{if} $\Lim{\vector{x,y}}{\vector{0,0}} f(x,y)$ exists, the above shows that it \emph{must} be $0$; that is, along \emph{every} path, the function must tend to $0$.

Let's pick another explicit path, say $y=2x$.  

\[
f(x,y) = f\left(x,\answer{2x}\right) =\frac{2x^3\left(\answer{2x}\right)-4x^2\left(\answer{2x}\right)^2}{2x^4+\left(\answer{2x}\right)^4} = \frac{\answer{-12}\cdot x^4}{\answer{18}\cdot x^4} = -\frac{2}{3}, x \neq 0.
\]

Thus, $f(x,y) \to \answer{-\frac{2}{3}}$ as $\vector{x,y} \to \vector{0,0}$ along the path $y=2x$.

Is this enough to conclude that $\Lim{\vector{x,y}}{\vector{0,0}}$ exists or does not exist?

\begin{multipleChoice}
\choice{Yes; $\Lim{\vector{x,y}}{\vector{0,0}}$ exists.}
\choice[correct]{Yes; $\Lim{\vector{x,y}}{\vector{0,0}}$ does not exist.}
\choice{No}
\end{multipleChoice}

\end{exercise}
\end{exercise}
%%%%%%%%%%%%%%%%%%%%%%%%%%%%%
\begin{exercise}
Since we think that analyzing the outputs of the function along lines in its domain, we can require that $y=mx$ and see what happens.

Along $y=mx$, we have the following.  

\[
f(x,y) = f\left(x,\answer{mx}\right) =\frac{2x^3\left(\answer{mx}\right)-4x^2\left(\answer{mx}\right)^2}{2x^4+\left(\answer{mx}\right)^4} = \frac{\answer{2m-4m^2}\cdot x^4}{\answer{2+m^4}\cdot x^4} = \frac{\answer{2m-4m^2}}{2+m^4}, x \neq 0.
\]

Thus, $f(x,y) \to \frac{\answer{2m-4m^2}}{2+m^4}$ as $\vector{x,y} \to \vector{0,0}$ along the path $y=mx$.

Is this enough to conclude that $\Lim{\vector{x,y}}{\vector{0,0}}$ exists or does not exist?

\begin{multipleChoice}
\choice{Yes; $\Lim{\vector{x,y}}{\vector{0,0}}$ exists because $f(x,y)$ tends to a number along each path.}
\choice[correct]{Yes; $\Lim{\vector{x,y}}{\vector{0,0}}$ does not exist because $f(x,y)$ approaches a different value for different choices of $m$.}
\choice{No}
\end{multipleChoice}

\begin{exercise}
The results of both approaches are related.

\begin{itemize}
\item The path $y=0$ is obtained from $y=mx$ by setting $m = \answer{0}$.  
\item The path $x=0$ is is a vertical line, so we take $m \to \infty$.
\item The path $y=2x$ is obtained from $y=mx$ by setting $m = \answer{2}$.

Do the results of both agree?

One nice consequence of analyzing the function along paths of the form $y=mx$ is that this gives explicit information about how the outputs of the function depend on the choice of path.
\end{itemize}
\end{exercise}
\end{exercise}
 
\end{exercise}
\end{document}
