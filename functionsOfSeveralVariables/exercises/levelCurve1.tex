\documentclass{ximera}

\newcommand{\RR}{\mathbb R}
\renewcommand{\d}{\,d}
\newcommand{\dd}[2][]{\frac{d #1}{d #2}}
\renewcommand{\l}{\ell}
\newcommand{\ddx}{\frac{d}{dx}}
\newcommand{\dfn}{\textbf}
\newcommand{\eval}[1]{\bigg[ #1 \bigg]}


\author{Jim Talamo\and Nela Lakos}

\outcome{Define a function of several variables.}

\begin{document}
\begin{exercise}
  Consider $F(x,y) = 4x^2+2y^2+5$.
  \begin{itemize}
  \item The domain is \wordChoice{\choice{$\R$}  \choice[correct]{$\R^2$}  \choice{$z>5$}  \choice{$z \geq 5$}}.
  \item The range is \wordChoice{\choice{$\R$}  \choice{$\R^2$}  \choice{$z>5$}  \choice[correct]{$z \geq 5$}} .
  \end{itemize}
  
Give a description of the level curve that passes through $(1,2)$ in terms of $x$ and $y$.
  \begin{hint}
  The level curve that passes through the point $(1,2)$ is depicted in the figure below.
  
 \begin{image}
            \begin{tikzpicture}
            	\begin{axis}[
            		domain=-10:10, ymax=3.6,xmax=3.6, ymin=-3.6, xmin=-3.6,
            		axis lines =center, xlabel=$x$, ylabel=$y$, ytick={-3,-2,-1,1,2,3},
            		every axis y label/.style={at=(current axis.above origin),anchor=south},
            		every axis x label/.style={at=(current axis.right of origin),anchor=west},
            		axis on top,
            		]
                                       %ellipse
                  \addplot [draw=penColor,domain=-1.7:1.7,ultra thick,smooth] {sqrt(6- 2*x^2)};
                
		                \addplot [draw=penColor,domain=-1.8:-1.6,ultra thick,smooth,samples=800] {sqrt(6- 2*x^2)};
                \addplot [draw=penColor,domain=1.7:1.8,ultra thick,smooth,samples=800] {sqrt(6- 2*x^2)};
               
                  \addplot [draw=penColor,domain=-1.7:1.7,ultra thick,smooth] {-sqrt(6- 2*x^2)};
               
                
                  \addplot [draw=penColor,domain=-1.8:-1.7,ultra thick,smooth,samples=800] {-sqrt(6- 2*x^2)};
                  \addplot [draw=penColor,domain=1.7:1.8,ultra thick,smooth,samples=800] {-sqrt(6- 2*x^2)};
                   \addplot[color=penColor,fill=penColor,only marks,mark=*] coordinates{(1,2)}; %C
   	\node at (axis cs:-2.2,2.3) [penColor] {$F(x,y)=F(1,2)$};
	\node at (axis cs:1.5,2.2) [penColor] {$(1,2)$};
	      \end{axis}
            \end{tikzpicture}
            \end{image}

            
   
  \end{hint}
\begin{align*}
F(x,y) &= \answer{17} \\
4x^2+2y^2 &= \answer{12} 
\end{align*}

This level curve is \wordChoice{\choice{a line}\choice{a plane}\choice{a parabola}\choice{a circle}\choice[correct]{an ellipse}\choice{a hyperbola}}.

\begin{exercise}
Does the curve $C$ parameterized by $\vec{r}(t) = \vector{2t,t}$ lie on a level curve of $F(x,y)$?

\begin{multipleChoice}
\choice{Yes.}
\choice[correct]{No.}
\end{multipleChoice}

Give a parameterization of the curve on the surface that lies above $C$.

\[
\vec{R}(t) = \vector{\answer{2t},\answer{t},\answer{18t^2+5}}
\]

\begin{hint}
If $\vec{r}(t) = \vector{2t,t}$ lies on a level curve of $F(x,y)$, then $F(x,y)$ should be constant along $C$.  Along $C$, we have $x=\answer{2t}$ and $y=\answer{t}$ so $F(x(t),y(t)) = \answer{18t^2+5}$, which \wordChoice{\choice{is}\choice[correct]{is not}} constant.

To find the curve on the surface $z=F(x,y)$ above the curve, note that $x(t) = 2t$ and $y(t)=t$.  We can use the function $F(x,y)$ to write $z$ in terms of $t$.

\end{hint}

\end{exercise}

\end{exercise}
\end{document}
