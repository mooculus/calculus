\documentclass[handout]{ximera}
%handout:  for handout version with no solutions or instructor notes
%handout,instructornotes:  for instructor version with just problems and notes, no solutions
%noinstructornotes:  shows only problem and solutions

%% handout
%% space
%% newpage
%% numbers
%% nooutcomes

%I added the commands here so that I would't have to keep looking them up
%\newcommand{\RR}{\mathbb R}
%\renewcommand{\d}{\,d}
%\newcommand{\dd}[2][]{\frac{d #1}{d #2}}
%\renewcommand{\l}{\ell}
%\newcommand{\ddx}{\frac{d}{dx}}
%\everymath{\displaystyle}
%\newcommand{\dfn}{\textbf}
%\newcommand{\eval}[1]{\bigg[ #1 \bigg]}

%\begin{image}
%\includegraphics[trim= 170 420 250 180]{Figure1.pdf}
%\end{image}

%add a ``.'' below when used in a specific directory.

\newcommand{\RR}{\mathbb R}
\renewcommand{\d}{\,d}
\newcommand{\dd}[2][]{\frac{d #1}{d #2}}
\renewcommand{\l}{\ell}
\newcommand{\ddx}{\frac{d}{dx}}
\newcommand{\dfn}{\textbf}
\newcommand{\eval}[1]{\bigg[ #1 \bigg]}




\author{Tom Needham and Jim Talamo}

\outcome{Use integrals to compute areas of regions bounded between curves.}

\title[]{Regions Between Curves}

\begin{document}
\begin{abstract}
\end{abstract}
\maketitle

\vspace{-0.9in}

\section{Discussion Questions}

\begin{problem}
A student is asked to compute the area of the region bounded by the curves $y=\sin (x)$ and $y = 0$ between $x=0$ and $x=2\pi$. The student responds that the area is zero. Either explain why the student's answer is correct or explain a likely error in the student's reasoning and give the correct answer.
\begin{center}
\resizebox {6cm} {!} { 
          \begin{tikzpicture}
          
	    \begin{axis}[
            domain=-0.5:6.5,
            xmin=-0.5, xmax=6.5,
            ymin=-1.25, ymax=1.25
         ,
            axis lines =middle, xlabel=$x$, ylabel=$y$, yticklabels={,,}, xticklabels={,,},
            every axis y label/.style={at=(current axis.above origin),anchor=south},
            every axis x label/.style={at=(current axis.right of origin),anchor=west},
          ]
	  \addplot [draw=none,fill=fillp,domain=0:6.28, smooth] {sin(deg(x))} \closedcycle;
	  \addplot [very thick, penColor, smooth] {sin(deg(x))};
	  
        
        \end{axis}
\end{tikzpicture}}
\end{center}

\end{problem}

\begin{freeResponse}

\end{freeResponse}

\begin{problem}

The area between curves $y=f(x)$ and $y=g(x)$ can frequently be computed either as an integral with respect to $x$ or as an integral with respect to $y$. In each of the following figures, explain which strategy you would use to calculate the area of the shaded region.

\begin{tabular}{ll}
\resizebox {6cm} {!} { 

\begin{tikzpicture}
	\begin{axis}[
            domain=0:1.5, ymax=1.5,xmax=1.5, ymin=0, xmin=0,
            axis lines =center, xlabel=$x$, ylabel=$y$,
            every axis y label/.style={at=(current axis.above origin),anchor=south},
            every axis x label/.style={at=(current axis.right of origin),anchor=west},
            axis on top,
          ]
          \addplot [ fill = fillp, smooth, samples=100, domain=(0:1.5)] ({1-x^2},{x}) \closedcycle;
          \addplot [draw=none,fill=background,domain=0:1.5] {x} \closedcycle;   
          \addplot [very thick, penColor2, smooth, samples=100, domain=(0:1.5)] ({1-x^2},{x});
          \addplot [draw=penColor,very thick,smooth] {x};
          
          \node at (axis cs:1.2,.25) [penColor2] {$y=\sqrt{1-x}$};
          \node at (axis cs:1.2,.95) [penColor] {$y=x$};
        \end{axis}
\end{tikzpicture}
}
  &
\resizebox {6cm} {!} { 
         \begin{tikzpicture}
	\begin{axis}[
            domain=0:5.5, ymax=2.8,xmax=5.5, ymin=0, xmin=0,
            axis lines =center, xlabel=$x$, ylabel=$y$,
            every axis y label/.style={at=(current axis.above origin),anchor=south},
            every axis x label/.style={at=(current axis.right of origin),anchor=west},
            axis on top,
          ]
          \addplot [ fill = fillp, smooth, samples=100, domain=(0:2)] ({1+x^2},{x}) \closedcycle;
          \addplot [draw=none,fill=background,domain=0:5.2] {x-3} \closedcycle;   
          \addplot [very thick, penColor2, smooth, samples=100, domain=(0:3)] ({1+x^2},{x});
          \addplot [draw=penColor,very thick,smooth] {x-3};
          
          \node at (axis cs:2,1.5) [penColor2] {$y=\sqrt{x-1}$};
          \node at (axis cs:4.5,0.7) [penColor] {$y=x-3$};
        \end{axis}
\end{tikzpicture}
} 

\end{tabular}

\begin{center}
\resizebox {6cm} {!} {
\begin{tikzpicture}
		\begin{axis}[
			domain=-2:4, ymax=4.5,xmax=3, ymin=-1, xmin=-0.5,
			axis lines =center, xlabel=$x$, ylabel=$y$,
            		every axis y label/.style={at=(current axis.above origin),anchor=south},
            		every axis x label/.style={at=(current axis.right of origin),anchor=west},
            		axis on top,
            		]
                      
            	\addplot [draw=penColor,very thick,smooth] {6*x};
            	\addplot [draw=penColor,very thick,smooth] {x};
		\addplot [domain=0.3:2,draw=penColor2,very thick,smooth] {1/x};
		\addplot [domain=0.3:2,draw=penColor2,very thick,smooth] {2/x};
                       
            	\addplot [name path=A,domain=6^(-1/2):3^(-1/2),draw=none] {6*x};   
            	\addplot [name path=B,domain=6^(-1/2):3^(-1/2),draw=none] {1/x};
		\addplot [name path=C,domain=3^(-1/1.9):1,draw=none] {1/x};
		\addplot [name path=D,domain=3^(-1/1.9):1,draw=none] {2/x};
		\addplot [name path=E,domain=0.99:2^(1/2),draw=none] {x};
		\addplot [name path=F,domain=0.99:2^(1/2),draw=none] {2/x};
            	\addplot [fillp] fill between[of=A and B];
		\addplot [fillp] fill between[of=C and D];
		\addplot [fillp] fill between[of=E and F];
		
		\node at (axis cs:1,3.75) [penColor] {$y=6x$};
            	\node at (axis cs:2.5,3) [penColor] {$y=x$};
		\node at (axis cs:2.3,0.5) [penColor2] {$y=\frac{1}{x}$};
            	\node at (axis cs:2,1.45) [penColor2] {$y=\frac{2}{x}$};
                      
            	\end{axis}
	\end{tikzpicture}}
	\end{center}

\iffalse
\begin{image}
\begin{tikzpicture}
	\begin{axis}[
            domain=0:1.5, ymax=6,xmax=1.5,ymin=0, xmin=0,
            axis lines =left, xlabel=$x$, ylabel=$y$,
            xtick={0.5,1,2},
            %width=4in,
            %height=2in,
            %yticklabels={},
            %xticklabels={$1$, $2$},
            every axis y label/.style={at=(current axis.above origin),anchor=south},
            every axis x label/.style={at=(current axis.right of origin),anchor=west},
            axis on top,
          ]
          \addplot [draw=none,fill=fillp,domain=0.5:1] {-3*x^2+4*x+3} \closedcycle;
          \addplot [draw=none,fill=background,domain=0.5:1] {3*x^2-3*x +2} \closedcycle;
          \addplot [draw=penColor,very thick] {-3*x^2+4*x+3};
          \addplot [draw=penColor2,very thick] {3*x^2-3*x +2};
          \node at (axis cs:0.75,4.75) [penColor] {$y= -3x^2+4x+3$};
          \node at (axis cs:0.75,0.75) [penColor2] {$y=3x^2-3x+2$};
        \end{axis}
\end{tikzpicture}
\end{image} 
\fi
\end{problem}

\begin{freeResponse} 
\end{freeResponse}


\section{Group Work}

\begin{problem}
Let $R$ denote the region bounded by the curves $y=2-2x$, $y=-1$ and $x=0$. Set up an integral to (either with respect to $x$ or with respect to $y$) which can be used to calculate the area of $R$. Either evaluate your integral or determine another way to calculate the area of $R$. 
\end{problem}

\begin{freeResponse}

\end{freeResponse}

\begin{problem}

\begin{enumerate}
\item[I.] Find the area bounded between the curves $y=2-x^2$ and $y=x-1$.

\item[II.] Find the area bounded by the curves $y=e^{2x}$, $x=0$ and $x=3$. 
\end{enumerate}
\end{problem}

\begin{freeResponse}
\end{freeResponse}

\begin{problem}
The region $R$ bounded by $x^2+y^2=1$ and $y=x$ is shown below. 

\begin{center}
\resizebox {6cm} {!} {
            \begin{tikzpicture}
            	\begin{axis}[
            		domain=-1.2:1.2, ymax=1.2,xmax=1.2, ymin=-1.1, xmin=-1.2,
            		axis lines =center, xlabel=$x$, ylabel=$y$,
            		every axis y label/.style={at=(current axis.above origin),anchor=south},
            		every axis x label/.style={at=(current axis.right of origin),anchor=west},
            		axis on top,
            		]
                      
            	\addplot [draw=penColor,domain=-1:1/sqrt(2),very thick,smooth,samples=200] {sqrt(1-x^2)};
	        \addplot [draw=penColor,domain=-1:-1/sqrt(2),very thick,smooth,samples=200] {-sqrt(1-x^2)};
            	\addplot [draw=penColor2,very thick,smooth] {x};
	                            
            	\addplot [name path=A,domain=-1:1/sqrt(2),draw=none,samples=200] {sqrt(1-x^2)};   
            	\addplot [name path=B,domain=-1:-1/sqrt(2),draw=none,samples=200] {-sqrt(1-x^2)};
	        \addplot [name path=C,domain=-2:1,draw=none] {x};
            	\addplot [fillp] fill between[of=A and B];
                                   
            	\node at (axis cs:-.8,1) [penColor] {$x^2+y^2=1$};
            	\node at (axis cs:.7,.3) [penColor2] {$y=x$};
	    
	      \end{axis}
            \end{tikzpicture}}
            \end{center}
            
\begin{enumerate}
\item[I.] Set up (but do not evaluate!) an integral (or sum of integrals) with respect to $x$ which could be used to calculate the area of $R$.
\item[II.] Set up (but do not evaluate!) an integral (or sum of integrals) with respect to $y$ which could be used to calculate the area of $R$.
\end{enumerate}
\end{problem}

\begin{freeResponse}

\end{freeResponse}

\begin{problem}

\end{problem}

\begin{freeResponse}

\end{freeResponse}

\end{document}
