\documentclass{ximera}

\newcommand{\RR}{\mathbb R}
\renewcommand{\d}{\,d}
\newcommand{\dd}[2][]{\frac{d #1}{d #2}}
\renewcommand{\l}{\ell}
\newcommand{\ddx}{\frac{d}{dx}}
\newcommand{\dfn}{\textbf}
\newcommand{\eval}[1]{\bigg[ #1 \bigg]}


\begin{document}
\author{Nela Lakos}
\outcome{Classify critical points.}
\outcome{Apply the Second Derivative Test.}
\outcome{Find inflection points.}



\begin{exercise}

Let $f(x) = 20 + 8x^2 - x^4$.\\
(a) Find all the critical points of $f$ and classify them, i.e., for each critical point of $f$ decide whether it is a local minimum, local maximum or neither.
(b) Find all inflection points of $f$.

(a) First, we have to compute $f'(x)$.

$f'(x) = \answer{16x-4x^3}$.

Complete the statement below.

The  x-coordinates of all critical points of $f$ (from left to right) are $a=\answer{-2}$, $b=\answer{0}$, and $c=\answer{2}$.

%The absolute maximum of $f$ on the interval $[-1,3]$ is $\answer{36}$ and it occurs at $x=\answer{2}$.

%The absolute minimum of $f$ on the interval $[-1,3]$ is $\answer{11}$ and it occurs at $x=\answer{3}$.


\begin{exercise}
Now compute
\[
f''(x)=\answer{16-12x^2}
\]

Now, we will evaluate the second derivative at all the critical points of $f$ and apply the second derivative test to determine whether the function $f$ has a local extremum at any of those points.
\[
f''(a)=\answer{-32}
\]
\begin{exercise}
Because the value of $f''(a)$ is \wordChoice{\choice{positive}\choice[correct]{negative}\choice{zero}}, the function $f$ has  \wordChoice{\choice[correct]{a local maximum}\choice{a local minimum}\choice{neither a local maximum nor a local minimum}\choice{possibly a local maximum, a local minimum or neither}} at $x=a$ by the second derivative test.
\[
f''(b)=\answer{16}
\]
\begin{exercise}
Because the value of $f''(b)$ is \wordChoice{\choice[correct]{positive}\choice{negative}\choice{zero}}, the function $f$ has\wordChoice{\choice{a local maximum}\choice[correct]{a local minimum}\choice{neither a local maximum nor a local minimum}\choice{possibly a local maximum, a local minimum or neither}}  at $x=b$ by the second derivative test.

\[
f''(c)=\answer{-32}
\]
\begin{exercise}
Because the value of $f''(c)$ is \wordChoice{\choice{positive}\choice[correct]{negative}\choice{zero}}, the function $f$ has\wordChoice{\choice[correct]{a local maximum}\choice{a local minimum}\choice{neither a local maximum nor a local minimum}\choice{possibly a local maximum, a local minimum or neither}} at $x=c$  by the second derivative test.\\


(b) Complete the statement below.
\begin{exercise}

Since $f''(x)=\answer{16-12x^2}$, it follows that inflection point might occur at  (from left to right) $m=\answer{-\frac{2}{\sqrt{3}}}$ and $n=\answer{\frac{2}{\sqrt{3}}}$.\\

In order to check whether the function $f$ has an inflection point at either $x=m$ or $x=n$, we have to check the sign of $f''(x)$ on the intervals $(-\infty,m)$, $(m,n)$, and $(n,+\infty)$.

Since $a < m < b < n < c$, we can use the values $f''(a)$, $f''(b)$, and $f''(c)$ to determine the sign of $f''(x)$ on these intervals.
\begin{exercise}


 $f''(x)$ is \wordChoice{\choice{positive}\choice[correct]{negative}\choice{zero}} on the interval $(-\infty,m)$.\\
 
  $f''(x)$ is \wordChoice{\choice[correct]{positive}\choice{negative}\choice{zero}} on the interval $(m,n)$.\\
  
    $f''(x)$ is \wordChoice{\choice{positive}\choice[correct]{negative}\choice{zero}} on the interval  $(n,+\infty)$.\\
  \begin{exercise}
 Make a correct choice.\\
 
  The function $f$ \wordChoice{\choice[correct]{has}\choice{does not have}} an inflection point at $x=m$, because the sign of $f''(x)$\wordChoice{\choice[correct]{changes}\choice{does not change} } at $x=m$.\\
  The function $f$ \wordChoice{\choice[correct]{has}\choice{does not have}} an inflection point at $x=n$, because the sign of $f''(x)$\wordChoice{\choice[correct]{changes}\choice{does not change} } at $x=n$.\\
 
  \end{exercise}
 \end{exercise}
\end{exercise}
\end{exercise}
\end{exercise}
\end{exercise}
\end{exercise}
\end{exercise}
\end{document}
