\documentclass{ximera}

\newcommand{\RR}{\mathbb R}
\renewcommand{\d}{\,d}
\newcommand{\dd}[2][]{\frac{d #1}{d #2}}
\renewcommand{\l}{\ell}
\newcommand{\ddx}{\frac{d}{dx}}
\newcommand{\dfn}{\textbf}
\newcommand{\eval}[1]{\bigg[ #1 \bigg]}


\outcome{}


\title[Break-Ground:]{Stars, limits, and velocity}

\begin{document}
\begin{abstract}
Here we see a dialogue where two young mathematicians discuss stars,
limits, and instantaneous velocity.
\end{abstract}
\maketitle

Check out this dialogue between two calculus students (based on a true
story):

\begin{dialogue}
\item[Devyn] Riley, did you know I like looking at the stars at night?
\item[Riley] Stars are freaking awesome balls of nuclear fire whos
  light took thousands of years to reach us.
\item[Devyn] I know! But did you know that the best way to see a very
  dim star is to look \textbf{near} it but \textbf{not exactly at} it? It's
  because then you can use the ``rods'' in your eye, which work better
  in low light than the ``cones'' in your eyes.
\item[Riley] That's amazing! Hey! That reminds me of something\dots.
  How a GPS or a phone computes velocity!
\item[Devyn] What! How?
\item[Riley] To compute velocity from position look at
  \[
  \frac{\text{change in position}}{\text{change in time}}
  \]
\item[Devyn] And then we study this as the change in time gets closer
  and closer to zero.
\item[Riley] Just like with stars, we can study something by looking
  \textbf{near} a point, but \textbf{not exactly at} a point.
\item[Devyn] O.M.G.\ Life's a rich tapestry.
\item[Riley] Poet, you know it.
\end{dialogue}


\begin{problem}
  near vs closer and closer.
\end{problem}


\begin{problem}
  near vs closer and closer.
\end{problem}

%% \begin{xarmaBoost}
%%   Write down at least \textbf{five} questions for this lecture. After
%%   you have your questions, label them as ``Level 1,'' ``Level 2,'' or
%%   ``Level 3'' where:
%% \begin{description}
%% \item[Level 1] Means you know the answer, or know exactly how to do
%%   this problem.
%% \item[Level 2] Means you think you know how to do the problem.
%% \item[Level 3] Means you have no idea how to do the problem.
%% \end{description}
%% \begin{freeResponse}
%% \end{freeResponse}
%% \end{xarmaBoost}



\end{document}
