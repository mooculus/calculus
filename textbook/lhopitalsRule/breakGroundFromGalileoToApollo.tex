\documentclass{ximera}

\newcommand{\RR}{\mathbb R}
\renewcommand{\d}{\,d}
\newcommand{\dd}[2][]{\frac{d #1}{d #2}}
\renewcommand{\l}{\ell}
\newcommand{\ddx}{\frac{d}{dx}}
\newcommand{\dfn}{\textbf}
\newcommand{\eval}[1]{\bigg[ #1 \bigg]}


\outcome{}


\title[Break-Ground:]{From Galileo to Apollo}

\begin{document}

\begin{abstract}

\end{abstract}
\maketitle


In the 1600's, Galileo popularized the idea that falling bodies fell
at the same velocity regardless of their mass. While this is something
we know on an intellectual level, you also know from experience that
hammers fall quickly and feathers fall slowly. The reason for this is
\textit{air resistance}.  Let's take a minute to investiagte the
effects of air resistance. The velocity of a falling object with air
resistance can be modeled by
\[
v = \frac{mg(1-e^{-kt/m})}{k}\qquad\text{in meters per second}
\]
where $m$ is the mass of the object in kilograms, $g$ is the
acceleration due to gravity in meter per second-squared, $t$ is the
amount of time in seconds that the object is falling, and $k$ is a
positive constant that will vary per object and represents the
resistance caused by the atmosphere.


\begin{problem} %% Here we are asking the student to think about context
\end{problem}

\begin{xarmaBoost}
  Write down at least \textbf{five} questions for this lecture. After
  you have your questions, label them as ``Level 1,'' ``Level 2,'' or
  ``Level 3'' where:
\begin{description}
\item[Level 1] Means you know the answer, or know exactly how to do
  this problem.
\item[Level 2] Means you think you know how to do the problem, or will
  soon learn how to do the problem.
\item[Level 3] Means you have no idea how to do the problem.
\end{description}
\begin{freeResponse}
\end{freeResponse}
\end{xarmaBoost}

\end{document}
