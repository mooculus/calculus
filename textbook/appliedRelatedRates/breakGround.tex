\documentclass{ximera}

\newcommand{\RR}{\mathbb R}
\renewcommand{\d}{\,d}
\newcommand{\dd}[2][]{\frac{d #1}{d #2}}
\renewcommand{\l}{\ell}
\newcommand{\ddx}{\frac{d}{dx}}
\newcommand{\dfn}{\textbf}
\newcommand{\eval}[1]{\bigg[ #1 \bigg]}


\outcome{}

\title[Break-Ground:]{Another big mess}

\begin{document}
\begin{abstract}
Two young mathematicians 
\end{abstract}
\maketitle

Check out this dialogue between two calculus students (based on a true
story):

%% Right Triangle
%% Oil Spill

\begin{dialogue}
\item[Devyn] Hey Riley, do you know what I love?
\item[Riley] Calculus? 
\item[Devyn] And pancakes! I really like to pour a lot of maple syrup
  on my pancakes.
\item[Riley] Mmmmmmm. Calculus. 
\item[Devyn] I know! So suppose you are pouring maple syrup on your
  pancake at a constant rate. You see it forming a disk that's expanding.
\item[Riley] Mmmmmmm. Sweet calculus. Wait! How fast is the radius of
  the disk of syrup expanding?
\end{dialogue}

To solve the problem above, we will need to solve a related-rates
problem. However, for now let's see if we can simply reason through
some problems.

\begin{problem}
  When comparing the volume of the maple syrup to the radius of the
  disk it forms, why might someone say that
  \begin{quote}
    The volume is proportional to the area of the disk.
  \end{quote}
  \begin{multipleChoice}
    \choice{Because we sound smarter when we say ``proportional.''}
    \choice[correct]{Because $V = h\cdot A$ where $h$ is the height of the disk.}
    \choice{Because volume and area are related.}
    \choice{Trick question! Nobody would ever say this.}
  \end{multipleChoice}
\end{problem}


\item[Devyn] Well, if we take the area of the disk of maple syrup and
  multiply it by some height, say $h$, that will give us a rough
  approximation of how much syrup is on the pancake.
\item[Riley] Right, and the area of a disk is
  \[
  A = \pi \cdot r^2
  \]
\item[Devyn] So now we have some formula for volume
  \[
 V = A\cdot h = \pi \cdot r^2 \cdot h 
  \]
\end{dialogue}

Maybe something about, what is the formula for r, when the volume is constant!


Suppose we assume that the amount of oil is proportional to the surface area of the oil spill. 

\begin{problem}
\end{problem}

%% \begin{xarmaBoost}
%%   Write down at least \textbf{five} questions for this lecture. After
%%   you have your questions, label them as ``Level 1,'' ``Level 2,'' or
%%   ``Level 3'' where:
%% \begin{description}
%% \item[Level 1] Means you know the answer, or know exactly how to do
%%   this problem.
%% \item[Level 2] Means you think you know how to do the problem.
%% \item[Level 3] Means you have no idea how to do the problem.
%% \end{description}
%% \begin{freeResponse}
%% \end{freeResponse}
%% \end{xarmaBoost}



\end{document}
