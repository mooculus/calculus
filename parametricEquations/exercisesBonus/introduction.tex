\documentclass{ximera}

\newcommand{\RR}{\mathbb R}
\renewcommand{\d}{\,d}
\newcommand{\dd}[2][]{\frac{d #1}{d #2}}
\renewcommand{\l}{\ell}
\newcommand{\ddx}{\frac{d}{dx}}
\newcommand{\dfn}{\textbf}
\newcommand{\eval}[1]{\bigg[ #1 \bigg]}


%\outcome{Find tangent lines to parametric curves}
\author{Jim Talamo}

\begin{document}
\begin{exercise}
The following mini-project has been written to synthesize concepts from you calculus journey, including:

\begin{itemize}
\item Related rates
\item Tangent lines
\item Area between curves
\item Volumes of solids of revolution
\end{itemize}

Completing this assignment successfully will be worth an additional 5 points to be added to your midterm subtotal score.  These points should be considered purely as \emph{bonus}.

Consider the curve $C$ defined by the parametric equations $x(t) = t^2$ and $y(t) = t$ for $t \geq 0$.  Eliminate the parameter to find a description of the curve in terms of $x$ and $y$ only.

 A description of the curve in terms of $x$ and $y$ only is $y = \answer{\sqrt{x}}$.

\begin{exercise}
The slope $m_{tan}$ of the tangent line to $C$ when $t=a$ is:

\[
m_{tan} = \answer{\frac{1}{2a}}
\]
(give your answer in terms of $a$).

The point $(x,y)$ on the curve when $t=a$ is:

\[
(x,y) = \left(\answer{a^2},\answer{a}\right)
\]
(give your answer in terms of $a$).
\end{exercise}
 
 The equation of the tangent line to the curve when $t=a$ is:
 
 \[
x+ \answer{-2a}y= \answer{-a^2}
 \] 
 (give your answer in terms of $a$).
 
\end{exercise}
\end{document}
