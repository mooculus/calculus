\documentclass{ximera}

\newcommand{\RR}{\mathbb R}
\renewcommand{\d}{\,d}
\newcommand{\dd}[2][]{\frac{d #1}{d #2}}
\renewcommand{\l}{\ell}
\newcommand{\ddx}{\frac{d}{dx}}
\newcommand{\dfn}{\textbf}
\newcommand{\eval}[1]{\bigg[ #1 \bigg]}


\outcome{}

% BADBAD This is not done

% Harvard calculus left the mean value theorem out.

% How is this related to the "racetrack principle"?

\title[Break-Ground:]{Meaning of multiplication}

\begin{document}
\begin{abstract}
A dialogue where students discuss multiplication.
\end{abstract}
\maketitle

Check out this dialogue between two calculus students (based on a true story):

\begin{dialogue}
\item[Devyn] Hey Riley, I was reading about the history of mathematics.
\item[Riley] Really? Tell me about it!
\item[Devyn] Apparently, back in the day, mathematicians worried about writing things like:
  \[
  3\times 4 + 5
  \]
\item[Riley] Why? What's the matter here?
\item[Devyn] Well, they thought of $3\times 4$ as an
  \textbf{area}. But $5$ was thought of as a \textbf{length}. Apparently they worried whether it made sense to add ``areas'' and ``lengths.''
\item[Riley] Hmmm. We don't seem to worry about that now. I wonder why?
\end{dialogue}

\begin{problem}
  What are some ways to interpret $3\times 4$?
  \begin{freeResponse}
    Answer will vary.
  \end{freeResponse}
\end{problem}

%% \begin{xarmaBoost}
%%   Write down at least \textbf{five} questions for this lecture. After
%%   you have your questions, label them as ``Level 1,'' ``Level 2,'' or
%%   ``Level 3'' where:
%% \begin{description}
%% \item[Level 1] Means you know the answer, or know exactly how to do
%%   this problem.
%% \item[Level 2] Means you think you know how to do the problem.
%% \item[Level 3] Means you have no idea how to do the problem.
%% \end{description}
%% \begin{freeResponse}
%% \end{freeResponse}
%% \end{xarmaBoost}


\end{document}
