\documentclass{ximera}

\newcommand{\RR}{\mathbb R}
\renewcommand{\d}{\,d}
\newcommand{\dd}[2][]{\frac{d #1}{d #2}}
\renewcommand{\l}{\ell}
\newcommand{\ddx}{\frac{d}{dx}}
\newcommand{\dfn}{\textbf}
\newcommand{\eval}[1]{\bigg[ #1 \bigg]}


\outcome{}


\title[Break-Ground:]{Stars and functions}

\begin{document}
\begin{abstract}
Here we see a dialogue where two young mathematicians discuss stars
and functions.
\end{abstract}
\maketitle

Check out this dialogue between two calculus students (based on a true
story):

\begin{dialogue}
\item[Devyn] Riley, did you know I like looking at the stars at night?
\item[Riley] Stars are freaking awesome balls of nuclear fire whose
  light took thousands of years to reach us.
\item[Devyn] I know! But did you know that the best way to see a very
  dim star is to look \textbf{near} it but \textbf{not exactly at} it? It's
  because then you can use the ``rods'' in your eye, which work better
  in low light than the ``cones'' in your eyes.
\item[Riley] That's amazing! Hey, that reminds me of when we were talking about the two functions
  \[
  f(x) = \frac{x^2-3x+2}{x-2}\qquad\text{and}\qquad g(x)= x+1,
  \]
  which we now know are completely different functions.
\item[Devyn] Woah. How are you seeing a connection here?
\item[Riley] If we want to understand what is happening with the
  function
  \[
  f(x) = \frac{x^2-3x+2}{x-2},
  \]
  at $x=2$, we can't do it by setting $x=2$. Instead we need to look
  \textbf{near} $x=2$ but \textbf{not exactly at} $x=2$.
  \item[Devyn] Ah ha! Because if we are \textbf{not exactly at} $x=2$,
    then
    \[
    \frac{x^2-3x+2}{x-2} = x+1.
    \]
\end{dialogue}

% something about "closer and closer"

% reconcile two experiences: we say limits are about "closer and closer" but in practice we just compute algebraically.  How do you reconcile these two experiences?

%% \begin{xarmaBoost}
%%   Write down at least \textbf{five} questions for this lecture. After
%%   you have your questions, label them as ``Level 1,'' ``Level 2,'' or
%%   ``Level 3'' where:
%% \begin{description}
%% \item[Level 1] Means you know the answer, or know exactly how to do
%%   this problem.
%% \item[Level 2] Means you think you know how to do the problem.
%% \item[Level 3] Means you have no idea how to do the problem.
%% \end{description}
%% \begin{freeResponse}
%% \end{freeResponse}
%% \end{xarmaBoost}



\end{document}
