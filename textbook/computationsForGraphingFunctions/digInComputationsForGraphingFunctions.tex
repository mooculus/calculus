\documentclass{ximera}

\newcommand{\RR}{\mathbb R}
\renewcommand{\d}{\,d}
\newcommand{\dd}[2][]{\frac{d #1}{d #2}}
\renewcommand{\l}{\ell}
\newcommand{\ddx}{\frac{d}{dx}}
\newcommand{\dfn}{\textbf}
\newcommand{\eval}[1]{\bigg[ #1 \bigg]}


\outcome{Determine how the graph of a function looks without using a calculator.}

\title[Dig-In:]{Computations for graphing functions}

\begin{document}
\begin{abstract}
  We will give some general guidelines for sketching the plot of a
  function.
\end{abstract}
\maketitle

Let's get to the point. Here we use all of the tools we know to sketch
the plots of functions.


\begin{itemize}
\item Find the $y$-intercept, this is the point $(0,f(0))$. Place this
  point on your graph.
\item Find any vertical asymptotes, these are points $x=a$ where
  $f(x)$ goes to infinity as $x$ goes to (from the right, left, or
  both) $a$.
\item Compute $f'(x)$ and $f''(x)$.
\item Find the critical points, the points where $f'(x) = 0$ or
  $f'(x)$ is undefined.
\item Use the second derivative test to identify local extrema and/or
  find the intervals where your function is increasing/decreasing.
\item Find the candidates for inflection points, the points where
  $f''(x) = 0$ or $f''(x)$ is undefined.
\item Identify inflection points and concavity.
\item If possible, find the $x$-intercepts, the points where $f(x) =
  0$. Place these points on your graph.
\item Find horizontal asymptotes.
\item Determine an interval that shows all relevant behavior.
\end{itemize}
At this point you should be able to sketch the plot of your function.

\begin{example}
Sketch the plot of $2x^3-3x^2-12x$.
\begin{explanation}
Following our guidelines above, we start by
computing $f(0) = \answer[given]{0}$.  Hence we see that the $y$-intercept is
$(0,\answer[given]{0})$. Place this point on your plot.
\begin{image}
\begin{tikzpicture}
	\begin{axis}[
            domain=-2:4,
            xmin=-2,
            xmax=4,
            ymax=25,
            ymin=-25,
            axis lines =middle, xlabel=$x$, ylabel=$y$,
            every axis y label/.style={at=(current axis.above origin),anchor=south},
            every axis x label/.style={at=(current axis.right of origin),anchor=west}
          ]
         \addplot[color=penColor,fill=penColor,only marks,mark=*] coordinates{(0,0)};  %% closed hole
        \end{axis}
\end{tikzpicture}
%% \caption{We start by placing the point $(0,0)$.}
%\label{figure:CS1}
\end{image}

Note that there are no %% BADBAD can we do a dropdown here?
vertical asymptotes as our function is defined
for all  %% BADBAD can we do a dropdown here?
real numbers.  Now compute $f'(x)$ and $f''(x)$,
\[
f'(x) = \answer[given]{6x^2 -6x -12}\qquad\text{and}\qquad f''(x) = \answer[given]{12x-6}.
\]

The critical points are where $f'(x) = 0$, thus we need to solve $6x^2
-6x -12 = 0$ for $x$. Write
\begin{align*}
6x^2 -6x -12 &= 0 \\
x^2 - x -2 &=0\\
(x-2)(x+1) &=0.
\end{align*}
Thus
\[
f'(2) = 0\qquad\text{and}\qquad f'(-1) = 0.
\]
Mark the critical points $x=2$ and $x=-1$ on your plot. %%BADBAD either or answer would be nice
\begin{image}
\begin{tikzpicture}
	\begin{axis}[
            domain=-2:4,
            xmin=-2,
            xmax=4,
            ymax=25,
            ymin=-25,
            axis lines =middle, xlabel=$x$, ylabel=$y$,
            every axis y label/.style={at=(current axis.above origin),anchor=south},
            every axis x label/.style={at=(current axis.right of origin),anchor=west}
          ]
         \addplot [dashed, penColor2] plot coordinates {(-1,-25) (-1,25)}; 
         \addplot [dashed, penColor2] plot coordinates {(2,-25) (2,25)}; 
         \addplot[color=penColor,fill=penColor,only marks,mark=*] coordinates{(0,0)};  %% closed hole
        \end{axis}
\end{tikzpicture}
%\caption{Now we add the critical points $x=-1$ and $x=2$.}
%\label{figure:CS2}
\end{image}

Check the second derivative evaluated at the critical points. In this
case,
\[
f''(-1) = \answer[given]{-18} \qquad\text{and}\qquad f''(2) = \answer[given]{18},
\]
hence $x=-1$, corresponding to the point $(-1,7)$ is a local maximum
and $x=2$, corresponding to the point $(2,-20)$ is local minimum of
$f(x)$. Moreover, this tells us that our function is increasing on
$[-2,-1)$, decreasing on $(-1,2)$, and increasing on $(2,4]$. Identify
this on your plot.
\begin{image}
\begin{tikzpicture}
	\begin{axis}[
            axis on top=true,
            domain=-2:4,
            xmin=-2,
            xmax=4,
            ymax=25,
            ymin=-25,
            axis lines =middle, xlabel=$x$, ylabel=$y$,
            every axis y label/.style={at=(current axis.above origin),anchor=south},
            every axis x label/.style={at=(current axis.right of origin),anchor=west}
          ]
          \addplot [->, line width=10, penColor!10!background] plot coordinates {(-2,-25) (-1,7)}; 
          \addplot [->, line width=10, penColor!10!background] plot coordinates {(-1,7) (2,-20)}; 
          \addplot [->, line width=10, penColor!10!background] plot coordinates {(2,-20) (4,25)}; 
          \addplot [dashed, penColor2] plot coordinates {(-1,-25) (-1,25)}; 
          \addplot [dashed, penColor2] plot coordinates {(2,-25) (2,25)}; 
          \addplot [color=penColor,fill=penColor,only marks,mark=*] coordinates{(0,0)};  %% closed hole
          \addplot [color=penColor,fill=penColor,only marks,mark=*] coordinates{(-1,7)};  %% closed hole
          \addplot [color=penColor,fill=penColor,only marks,mark=*] coordinates{(2,-20)};  %% closed hole
          %\addplot [very thick, penColor, samples=100, smooth,domain=(-1.2:-.8)] {2*x^3-3*x^2-12*x};
          %\addplot [very thick, penColor, samples=100, smooth,domain=(1.8:2.2)] {2*x^3-3*x^2-12*x};
        \end{axis}
\end{tikzpicture}
%% \caption{We have identified the local extrema of $f(x)$ and where this
%%   function is increasing and decreasing.}
%% \label{figure:CS3}
\end{image}


The candidates for the inflection points are where $f''(x) = 0$, thus
we need to solve $12x-6=0$ for $x$.  Write
\begin{align*}
12x-6 &=0\\
x - 1/2 &=0\\
x &=1/2.
\end{align*}
Thus $f''(1/2) = \answer[given]{0}$. Checking points, $f''(0) = -6$ and $f''(1) = 6$.
Hence $x=1/2$ is an inflection point, with $f(x)$ concave down to the
left of $x=1/2$ and $f(x)$ concave up to the right of $x=1/2$. We can
add this information to our plot.
\begin{image}
\begin{tikzpicture}
	\begin{axis}[
            axis on top=true,
            domain=-2:4,
            xmin=-2,
            xmax=4,
            ymax=25,
            ymin=-25,
            axis lines =middle, xlabel=$x$, ylabel=$y$,
            every axis y label/.style={at=(current axis.above origin),anchor=south},
            every axis x label/.style={at=(current axis.right of origin),anchor=west}
          ]
          \addplot [->, line width=10, penColor!10!background] plot coordinates {(-2,-25) (-1,7)}; 
          \addplot [->, line width=10, penColor!10!background] plot coordinates {(-1,7) (2,-20)}; 
          \addplot [->, line width=10, penColor!10!background] plot coordinates {(2,-20) (4,25)}; 
          \addplot [dashed, penColor2] plot coordinates {(-1,-25) (-1,25)}; 
          \addplot [dashed, penColor2] plot coordinates {(2,-25) (2,25)}; 
          \addplot [dashed, penColor4] plot coordinates {(1/2,-25) (1/2,25)}; 
          \addplot [color=penColor,fill=penColor,only marks,mark=*] coordinates{(1/2,-6.5)};  %% closed hole
          \addplot [color=penColor,fill=penColor,only marks,mark=*] coordinates{(0,0)};  %% closed hole
          \addplot [color=penColor,fill=penColor,only marks,mark=*] coordinates{(-1,7)};  %% closed hole
          \addplot [color=penColor,fill=penColor,only marks,mark=*] coordinates{(2,-20)};  %% closed hole
          \addplot [very thick, penColor, samples=100, smooth,domain=(-1.5:3)] {2*x^3-3*x^2-12*x};
        \end{axis}
\end{tikzpicture}
%% \caption{We identify the inflection point and note that the curve is
%%   concave down when $x<1/2$ and concave up when $x>1/2$.}
%% \label{figure:CS4}
\end{image}
Finally, in this case, $f(x) =2x^3-3x^2-12x$, we can find the
$x$-intercepts. Write
\begin{align*}
2x^3-3x^2-12x &=0\\
x(2x^2 -3x -12) &=0.\\
\end{align*}
Using the quadratic formula, we see that the $x$-intercepts of $f(x)$ are
\[
x = 0, \qquad x= \frac{3-\sqrt{105}}{4}, \qquad x= \frac{3+\sqrt{105}}{4}.
\]
Since all of this behavior as described above occurs on the interval
$[-2,4]$, we now have a complete sketch of $f(x)$ on this interval,
see the figure below.
\begin{image}
\begin{tikzpicture}
	\begin{axis}[
            axis on top=true,
            domain=-2:4,
            xmin=-2,
            xmax=4,
            ymax=25,
            ymin=-25,
            axis lines =middle, xlabel=$x$, ylabel=$y$,
            every axis y label/.style={at=(current axis.above origin),anchor=south},
            every axis x label/.style={at=(current axis.right of origin),anchor=west}
          ]
          \addplot [->, line width=10, penColor!10!background] plot coordinates {(-2,-25) (-1,7)}; 
          \addplot [->, line width=10, penColor!10!background] plot coordinates {(-1,7) (2,-20)}; 
          \addplot [->, line width=10, penColor!10!background] plot coordinates {(2,-20) (4,25)}; 
          \addplot [dashed, penColor2] plot coordinates {(-1,-25) (-1,25)}; 
          \addplot [dashed, penColor2] plot coordinates {(2,-25) (2,25)}; 
          \addplot [dashed, penColor4] plot coordinates {(1/2,-25) (1/2,25)}; 
          \addplot [color=penColor,fill=penColor,only marks,mark=*] coordinates{(1/2,-6.5)};  %% closed hole
          \addplot [color=penColor,fill=penColor,only marks,mark=*] coordinates{(0,0)};  %% closed hole
          \addplot [color=penColor,fill=penColor,only marks,mark=*] coordinates{(-1,7)};  %% closed hole
          \addplot [color=penColor,fill=penColor,only marks,mark=*] coordinates{(2,-20)};  %% closed hole
          \addplot [color=penColor,fill=penColor,only marks,mark=*] coordinates{(-1.812,0)};  %% closed hole
          \addplot [color=penColor,fill=penColor,only marks,mark=*] coordinates{(3.312,0)};  %% closed hole
          \addplot [very thick, penColor, samples=100, smooth,domain=(-2:4)] {2*x^3-3*x^2-12*x};
        \end{axis}
\end{tikzpicture}
\end{image}
\end{explanation}
\end{example}
\end{document}
