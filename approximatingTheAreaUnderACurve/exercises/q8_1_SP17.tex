\documentclass{ximera}

\newcommand{\RR}{\mathbb R}
\renewcommand{\d}{\,d}
\newcommand{\dd}[2][]{\frac{d #1}{d #2}}
\renewcommand{\l}{\ell}
\newcommand{\ddx}{\frac{d}{dx}}
\newcommand{\dfn}{\textbf}
\newcommand{\eval}[1]{\bigg[ #1 \bigg]}


\begin{document}

Let $f(x)= 16 - x^{2}$ on the interval $[0,6]$.
\begin{image}
  \begin{tikzpicture}
    \begin{axis}[
            domain=0:6, xmin =-.2,xmax=6.2,ymax=20,ymin=-20,
            width=4in,
            height=3in,
            xtick={1,2,...,6},
            ytick={4,8,...,16},
            axis lines=center, xlabel=$x$, ylabel=$y$,
            every axis y label/.style={at=(current axis.above origin),anchor=south},
            every axis x label/.style={at=(current axis.right of origin),anchor=west},
            axis on top,
      ]
      \addplot [ultra thick,penColor,smooth,domain=0:6] {16-x^2};
      \node[penColor] at (axis cs:4,16) {$y=f(x)=16-x^2$};
    \end{axis}
  \end{tikzpicture}
\end{image}
We will calculate the \textbf{midpoint} Riemann sum with $n=3$ rectangles.
\begin{exercise}  
  First, compute the width of each of each of the $n$ rectangles:
  \[
  \Delta x = \answer{2}
  \]
  %% look at https://ximera.osu.edu/mooculus/calculus1/approximatingTheAreaUnderACurve/digInApproximatingAreaWithRectangles
  \begin{exercise}
  Select the picture showing the corresponding \textbf{midpoint} Riemann sum.
  \begin{multipleChoice}
  \choice{\begin{tikzpicture}[framed,scale=.75,baseline=14.5ex, 
      declare function = {f(\x) = 16-\x^2;}
      ]
    \begin{axis}[
            domain=0:6, xmin =-.2,xmax=6.2,ymax=20,ymin=-20,
            width=4in,
            height=3in,
            xtick={1,2,...,6},
            ytick={4,8,...,16},
            axis lines=center, xlabel=$x$, ylabel=$y$,
            every axis y label/.style={at=(current axis.above origin),anchor=south},
            every axis x label/.style={at=(current axis.right of origin),anchor=west},
            axis on top,
            y label style={at={(0.031, 1)}},
            x label style={at={(1,1/2)}}
          ]
          \foreach \rectnumber in {1,2,3}
                   {
                     \addplot [draw=penColor,fill=fillp] plot coordinates
                              {({2*(\rectnumber-1)},{f(2*(\rectnumber-1))})
                                ({2*(\rectnumber) },{f(2*(\rectnumber-1)) })} \closedcycle;
               };
                   \addplot [very thick,penColor, smooth] {f(x)};
        \end{axis}
\end{tikzpicture}
  }
    \choice{\begin{tikzpicture}[framed,scale=.75,baseline=14.5ex, 
      declare function = {f(\x) = 16-\x^2;}
      ]
    \begin{axis}[
            domain=0:6, xmin =-.2,xmax=6.2,ymax=20,ymin=-20,
            width=4in,
            height=3in,
            xtick={1,2,...,6},
            ytick={4,8,...,16},
            axis lines=center, xlabel=$x$, ylabel=$y$,
            every axis y label/.style={at=(current axis.above origin),anchor=south},
            every axis x label/.style={at=(current axis.right of origin),anchor=west},
            axis on top,
            y label style={at={(0.031, 1)}},
            x label style={at={(1,1/2)}}
          ]
          \foreach \rectnumber in {1,2,...,5}
                   {
                     \addplot [draw=penColor,fill=fillp] plot coordinates
                              {({(\rectnumber-1)*6/5},{f((\rectnumber-1)*6/5)})
                                ({(\rectnumber)*6/5},{f((\rectnumber-1)*6/5)})} \closedcycle;
               };
                   \addplot [very thick,penColor, smooth] {f(x)};
        \end{axis}
\end{tikzpicture}
  }
    \choice{\begin{tikzpicture}[framed,scale=.75,baseline=14.5ex, 
      declare function = {f(\x) = 16-\x^2;}
      ]
    \begin{axis}[
            domain=0:6, xmin =-.2,xmax=6.2,ymax=20,ymin=-20,
            width=4in,
            height=3in,
            xtick={1,2,...,6},
            ytick={4,8,...,16},
            axis lines=center, xlabel=$x$, ylabel=$y$,
            every axis y label/.style={at=(current axis.above origin),anchor=south},
            every axis x label/.style={at=(current axis.right of origin),anchor=west},
            axis on top,
            y label style={at={(0.031, 1)}},
            x label style={at={(1,1/2)}}
          ]
          \foreach \rectnumber in {1,2,3}
                   {
                     \addplot [draw=penColor,fill=fillp] plot coordinates
                              {({2*(\rectnumber-1)},{f(2*(\rectnumber))})
                                ({2*(\rectnumber) },{f(2*(\rectnumber)) })} \closedcycle;
               };
                   \addplot [very thick,penColor, smooth] {f(x)};
        \end{axis}
\end{tikzpicture}
  }
    \choice[correct]{\begin{tikzpicture}[framed,scale=.75,baseline=14.5ex, 
      declare function = {f(\x) = 16-\x^2;}
      ]
    \begin{axis}[
            domain=0:6, xmin =-.2,xmax=6.2,ymax=20,ymin=-20,
            width=4in,
            height=3in,
            xtick={1,2,...,6},
            ytick={4,8,...,16},
            axis lines=center, xlabel=$x$, ylabel=$y$,
            every axis y label/.style={at=(current axis.above origin),anchor=south},
            every axis x label/.style={at=(current axis.right of origin),anchor=west},
            axis on top,
            y label style={at={(0.031, 1)}},
            x label style={at={(1,1/2)}}
          ]
          \foreach \rectnumber in {1,2,3}
                   {
                     \addplot [draw=penColor,fill=fillp] plot coordinates
                              {({2*(\rectnumber-1)},{f(2*(\rectnumber)-1)})
                                ({2*(\rectnumber) },{f(2*(\rectnumber)-1) })} \closedcycle; %%\closedcycle shades
               };
                   \addplot [very thick,penColor, smooth] {f(x)};
        \end{axis}
\end{tikzpicture}
  }
      \choice{\begin{tikzpicture}[framed,scale=.75,baseline=14.5ex, 
      declare function = {f(\x) = 16-\x^2;}
      ]
    \begin{axis}[
            domain=0:6, xmin =-.2,xmax=6.2,ymax=20,ymin=-20,
            width=4in,
            height=3in,
            xtick={1,2,...,6},
            ytick={4,8,...,16},
            axis lines=center, xlabel=$x$, ylabel=$y$,
            every axis y label/.style={at=(current axis.above origin),anchor=south},
            every axis x label/.style={at=(current axis.right of origin),anchor=west},
            axis on top,
            y label style={at={(0.031, 1)}},
            x label style={at={(1,1/2)}}
          ]
          \foreach \rectnumber in {1,2,...,5}
                   {
                     \addplot [draw=penColor,fill=fillp] plot coordinates
                              {({(\rectnumber-1)*6/5},{f((\rectnumber-1)*6/5)})
                                ({(\rectnumber)*6/5},{f((\rectnumber)*6/5)})} \closedcycle;
               };
                   \addplot [very thick,penColor, smooth] {f(x)};
        \end{axis}
\end{tikzpicture}
  }

  \end{multipleChoice}
  \begin{exercise}
    Now fill-in the following table where $x_k$ denotes the $k$-th endpoint of a rectangle and $x^{*}_k$ denotes the midpoint of the rectangle in the interval $[x_{k-1},x_k]$ (you may need to make your own table for a written assessment!):
    \[
    \begin{array}{c|c|c|c}
      k &  x_k & x^{*}_k & f(x^{*}_k) \\ \hline  
      0 & \answer{0}   & \text{NA} & \text{NA}\\
      1 & \answer{2} & \answer{1}  & \answer{15}\\
      2 & \answer{4}  & \answer{3} & \answer{7}\\
      3 & \answer{6}  & \answer{5} & \answer{-9}\\
    \end{array}
\]
   \begin{exercise}
    Finally, calculate the \textbf{midpoint} Riemann sum with $n=3$ rectangles.
    \[
   \answer{26}
    \]
    \end{exercise}
  \end{exercise}
 \end{exercise}
\end{exercise}

  %%%SEE https://ximera.osu.edu/mooculus/calculus1/reviewOfFamousFunctions/digInTrigonometricFunctions

%% What arc on the unit circle corresponds to the restricted domain described above of cos(θ)cos⁡(θ)?

\end{document}
