\documentclass{ximera}

\newcommand{\RR}{\mathbb R}
\renewcommand{\d}{\,d}
\newcommand{\dd}[2][]{\frac{d #1}{d #2}}
\renewcommand{\l}{\ell}
\newcommand{\ddx}{\frac{d}{dx}}
\newcommand{\dfn}{\textbf}
\newcommand{\eval}[1]{\bigg[ #1 \bigg]}


\outcome{Understand what is meant by the form of a limit.}
\outcome{Distinguish between determinate and indeterminate forms.}
\outcome{Calculate limits of the form zero over zero.}

\begin{document}

\begin{exercise}
Let $g(x)=|x-1|$ and let $h(x)=\sqrt{x}-1$. Choose the correctly sketched and labeled graph of $g$ and $h$ on the interval $[-5,5]$.
\begin{multipleChoice}
\choice[correct]{\begin{tikzpicture}[framed,scale=.75,baseline=14.5ex]
    \begin{axis}[
            domain=-5:5, xmin =-5.2,xmax=5.2,ymax=5.2,ymin=-5.2,
            width=4in,
            height=4in,
            xtick={-5,-4,...,5},
            ytick={-5,-4,...,5},
            axis lines=center,
            xlabel=$x$, ylabel=$y$,
            every axis y label/.style={at=(current axis.above origin),anchor=south},
            every axis x label/.style={at=(current axis.right of origin),anchor=west},
            axis on top,
            y label style={at={(0.5, 1)}},
            x label style={at={(1,1/2)}},
      ]
      \addplot [ultra thick,penColor,smooth,domain=-5:1] {-x+1};
      \addplot [ultra thick,penColor,smooth,domain=1:5] {x-1};
      \addplot[ultra thick,penColor2,smooth,domain=.01:5,samples=100]{sqrt(x)-1};
      \addplot[color=penColor2,fill=penColor2,only marks,mark=*] coordinates{(0,-1)};
      \node[penColor] at (axis cs:2.4,4.5) {$g(x)=|x-1|$};
      \node[penColor2] at (axis cs:2.4,-2) {$f(x)=\sqrt{x}-1$};
    \end{axis}
  \end{tikzpicture}
}
\choice{\begin{tikzpicture}[framed,scale=.75,baseline=14.5ex]
    \begin{axis}[
            domain=-5:5, xmin =-5.2,xmax=5.2,ymax=5.2,ymin=-5.2,
            width=4in,
            height=4in,
            xtick={-5,-4,...,5},
            ytick={-5,-4,...,5},
            axis lines=center,
            xlabel=$x$, ylabel=$y$,
            every axis y label/.style={at=(current axis.above origin),anchor=south},
            every axis x label/.style={at=(current axis.right of origin),anchor=west},
            axis on top,
            y label style={at={(0.5, 1)}},
            x label style={at={(1,1/2)}},
      ]
      \addplot [ultra thick,penColor,smooth,domain=-5:-1] {-x-1};
      \addplot [ultra thick,penColor,smooth,domain=-1:5] {x+1};
      \addplot[ultra thick,penColor2,smooth,domain=0:5,samples=100]{sqrt(x)-1};
      \addplot[ultra thick,penColor2,smooth,domain=-5:0,samples=100]{sqrt(-x)-1};
    %%  \addplot[color=penColor2,fill=penColor2,only marks,mark=*] coordinates{(0,-1)};
      \node[penColor] at (axis cs:3,2) {$g(x)=|x-1|$};
      \node[penColor2] at (axis cs:2.4,-2) {$f(x)=\sqrt{x}-1$};
    \end{axis}
  \end{tikzpicture}
}
\choice{\begin{tikzpicture}[framed,scale=.75,baseline=14.5ex]
    \begin{axis}[
            domain=-5:5, xmin =-5.2,xmax=5.2,ymax=5.2,ymin=-5.2,
            width=4in,
            height=4in,
            xtick={-5,-4,...,5},
            ytick={-5,-4,...,5},
            axis lines=center,
            xlabel=$x$, ylabel=$y$,
            every axis y label/.style={at=(current axis.above origin),anchor=south},
            every axis x label/.style={at=(current axis.right of origin),anchor=west},
            axis on top,
            y label style={at={(0.5, 1)}},
            x label style={at={(1,1/2)}},
      ]
      \addplot [ultra thick,penColor,smooth,domain=-5:1] {-x+1};
      \addplot [ultra thick,penColor,smooth,domain=1:5] {x-1};
      \addplot[ultra thick,penColor2,smooth,domain=0:5,samples=100]{sqrt(x)-1};
      \addplot[ultra thick,penColor2,smooth,domain=-5:0,samples=100]{-sqrt(-x)-1};
    %%  \addplot[color=penColor2,fill=penColor2,only marks,mark=*] coordinates{(0,-1)};
      \node[penColor] at (axis cs:2.4,4) {$g(x)=|x-1|$};
      \node[penColor2] at (axis cs:2.4,-2) {$f(x)=\sqrt{x}-1$};
    \end{axis}
  \end{tikzpicture}
}
\choice{\begin{tikzpicture}[framed,scale=.75,baseline=14.5ex]
    \begin{axis}[
            domain=-5:5, xmin =-5.2,xmax=5.2,ymax=5.2,ymin=-5.2,
            width=4in,
            height=4in,
            xtick={-5,-4,...,5},
            ytick={-5,-4,...,5},
            axis lines=center,
            xlabel=$x$, ylabel=$y$,
            every axis y label/.style={at=(current axis.above origin),anchor=south},
            every axis x label/.style={at=(current axis.right of origin),anchor=west},
            axis on top,
            y label style={at={(0.5, 1)}},
            x label style={at={(1,1/2)}},
      ]
      \addplot [ultra thick,penColor,smooth,domain=-5:-1] {-x-1};
      \addplot [ultra thick,penColor,smooth,domain=-1:5] {x+1};
      \addplot[ultra thick,penColor2,smooth,domain=0:5,samples=100]{sqrt(x)-1};
      \addplot[color=penColor2,fill=penColor2,only marks,mark=*] coordinates{(0,-1)};
      \node[penColor] at (axis cs:3.5,2) {$g(x)=|x-1|$};
      \node[penColor2] at (axis cs:2.4,-2) {$f(x)=\sqrt{x}-1$};
    \end{axis}
  \end{tikzpicture}
}
\end{multipleChoice} 
\begin{exercise}
Let $f$ be a function defined on $(0,2)$ such that $h(x)\le f(x)\le
g(x)$ for all $x$ with $0<x<2$ except possibly at $x=1$. Then
\[
\lim_{x\to 1}f(x)= \answer{0}
\]
due to \wordChoice{\choice{limit laws}\choice{continuity}\choice{of the difference
    law}\choice{quotient law}\choice[correct]{the Squeeze Theorem}}.
\begin{exercise}
Now consider the quotient $\frac{e^{h(x)}}{\cos{g(x)}}$.  In particular, it is of the form \wordChoice{\choice{nonzero over zero} \choice[correct]{zero over zero}}.
\begin{exercise}
Multiplying by the conjugate, it follows that for all $x>1$,  
\[
\frac{h(x)}{g(x)}=\frac{1}{\answer{\sqrt{x}+1}}.
\]
\begin{exercise}
Now, evaluate the limit:
\[
\lim_{x\to 1^+}\frac{h(x)}{g(x)}=\answer{\frac{1}{2}}.
\]
\end{exercise}
\end{exercise}
\end{exercise}
\end{exercise}
\end{exercise}
\end{document}
