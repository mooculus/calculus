\documentclass{ximera}

\newcommand{\RR}{\mathbb R}
\renewcommand{\d}{\,d}
\newcommand{\dd}[2][]{\frac{d #1}{d #2}}
\renewcommand{\l}{\ell}
\newcommand{\ddx}{\frac{d}{dx}}
\newcommand{\dfn}{\textbf}
\newcommand{\eval}[1]{\bigg[ #1 \bigg]}


\outcome{Know the graphs and properties of ``famous'' functions.}

\author{Nela Lakos \and Kyle Parsons}

\begin{document}
\begin{exercise}

Use these six graphs (of inverse trig functions) when answering questions in exercises below.

\parbox{0.6\textwidth}{
\begin{image}
  \begin{tikzpicture}
    \begin{axis}[
        ymin=-3.1416,ymax=3.1416,xmin=-3,xmax=3,
        clip=true,
        unit vector ratio*=1 1 1,
        axis lines=center,
        grid = major,
        xtick={-10,-9,...,10},
    	ytick={-6.2831,-4.7123,...,6.29},
		yticklabels={$-2\pi$,$-\frac{3\pi}{2}$,$-\pi$,$-\frac{\pi}{2}$,0,$\frac{\pi}{2}$,$\pi$,$\frac{3\pi}{2}$,$2\pi$},
        xlabel=$x$, ylabel=$y$,
        every axis y label/.style={at=(current axis.above origin),anchor=south},
        every axis x label/.style={at=(current axis.right of origin),anchor=west},
      ]
      \addplot[very thick,penColor,domain=-1:1,samples=50] plot{pi/180*asin(x)};
      \node at (axis cs:2,2) {\Huge A};
      \end{axis}`
  \end{tikzpicture}
\end{image}}
\hfill 
\parbox{0.6\textwidth}{
\begin{image}
  \begin{tikzpicture}
    \begin{axis}[
        ymin=-3.1416,ymax=3.1416,xmin=-3,xmax=3,
        clip=true,
        unit vector ratio*=1 1 1,
        axis lines=center,
        grid = major,
        xtick={-10,-9,...,10},
    	ytick={-6.2831,-4.7123,...,6.29},
		yticklabels={$-2\pi$,$-\frac{3\pi}{2}$,$-\pi$,$-\frac{\pi}{2}$,0,$\frac{\pi}{2}$,$\pi$,$\frac{3\pi}{2}$,$2\pi$},
        xlabel=$x$, ylabel=$y$,
        every axis y label/.style={at=(current axis.above origin),anchor=south},
        every axis x label/.style={at=(current axis.right of origin),anchor=west},
      ]
      \addplot[very thick,penColor,domain=-3:3,samples=50] plot{pi/180*atan(x)};
      \node at (axis cs:2,2) {\Huge B};
      \end{axis}`
  \end{tikzpicture}
\end{image}}

\parbox{0.6\textwidth}{
\begin{image}
  \begin{tikzpicture}
    \begin{axis}[
        ymin=-3.1416,ymax=3.1416,xmin=-3,xmax=3,
        clip=true,
        unit vector ratio*=1 1 1,
        axis lines=center,
        grid = major,
        xtick={-10,-9,...,10},
    	ytick={-6.2831,-4.7123,...,6.29},
		yticklabels={$-2\pi$,$-\frac{3\pi}{2}$,$-\pi$,$-\frac{\pi}{2}$,0,$\frac{\pi}{2}$,$\pi$,$\frac{3\pi}{2}$,$2\pi$},
        xlabel=$x$, ylabel=$y$,
        every axis y label/.style={at=(current axis.above origin),anchor=south},
        every axis x label/.style={at=(current axis.right of origin),anchor=west},
      ]
      \addplot[very thick,penColor,domain=0:3,samples=50] plot{pi/180*atan(1/x)};
      \addplot[very thick,penColor,domain=-3:0,samples=50] plot{pi + pi/180*atan(1/x)};
      \node at (axis cs:2,2) {\Huge C};
      \end{axis}`
  \end{tikzpicture}
\end{image}} 
\hfill 
\parbox{0.6\textwidth}{
\begin{image}
  \begin{tikzpicture}
    \begin{axis}[
        ymin=-3.1416,ymax=3.1416,xmin=-3,xmax=3,
        clip=true,
        unit vector ratio*=1 1 1,
        axis lines=center,
        grid = major,
        xtick={-10,-9,...,10},
    	ytick={-6.2831,-4.7123,...,6.29},
		yticklabels={$-2\pi$,$-\frac{3\pi}{2}$,$-\pi$,$-\frac{\pi}{2}$,0,$\frac{\pi}{2}$,$\pi$,$\frac{3\pi}{2}$,$2\pi$},
        xlabel=$x$, ylabel=$y$,
        every axis y label/.style={at=(current axis.above origin),anchor=south},
        every axis x label/.style={at=(current axis.right of origin),anchor=west},
      ]
      \addplot[very thick,penColor,domain=-3:-1,samples=50] plot{pi/180*acos(1/x)};
      \addplot[very thick,penColor,domain=1:3,samples=50] plot{pi/180*acos(1/x)};
      \node at (axis cs:2,2) {\Huge D};
      \end{axis}`
  \end{tikzpicture}
\end{image}} 

\parbox{0.6\textwidth}{
\begin{image}
  \begin{tikzpicture}
    \begin{axis}[
        ymin=-3.1416,ymax=3.1416,xmin=-3,xmax=3,
        clip=true,
        unit vector ratio*=1 1 1,
        axis lines=center,
        grid = major,
        xtick={-10,-9,...,10},
    	ytick={-6.2831,-4.7123,...,6.29},
		yticklabels={$-2\pi$,$-\frac{3\pi}{2}$,$-\pi$,$-\frac{\pi}{2}$,0,$\frac{\pi}{2}$,$\pi$,$\frac{3\pi}{2}$,$2\pi$},
        xlabel=$x$, ylabel=$y$,
        every axis y label/.style={at=(current axis.above origin),anchor=south},
        every axis x label/.style={at=(current axis.right of origin),anchor=west},
      ]
      \addplot[very thick,penColor,domain=-1:1,samples=50] plot{pi/180*acos(x)};
      \node at (axis cs:2,2) {\Huge E};
      \end{axis}`
  \end{tikzpicture}
\end{image}} 
\hfill 
\parbox{0.6\textwidth}{
\begin{image}
  \begin{tikzpicture}
    \begin{axis}[
        ymin=-3.1416,ymax=3.1416,xmin=-3,xmax=3,
        clip=true,
        unit vector ratio*=1 1 1,
        axis lines=center,
        grid = major,
        xtick={-10,-9,...,10},
    	ytick={-6.2831,-4.7123,...,6.29},
		yticklabels={$-2\pi$,$-\frac{3\pi}{2}$,$-\pi$,$-\frac{\pi}{2}$,0,$\frac{\pi}{2}$,$\pi$,$\frac{3\pi}{2}$,$2\pi$},
        xlabel=$x$, ylabel=$y$,
        every axis y label/.style={at=(current axis.above origin),anchor=south},
        every axis x label/.style={at=(current axis.right of origin),anchor=west},
      ]
      \addplot[very thick,penColor,domain=-3:-1,samples=50] plot{pi/180*asin(1/x)};
      \addplot[very thick,penColor,domain=1:3,samples=50] plot{pi/180*asin(1/x)};
      \node at (axis cs:2,2) {\Huge F};
      \end{axis}`
  \end{tikzpicture}
\end{image}} 

\begin{exercise}
Circle all the functions $f$ for which the following statement is true.

The function $f$ is increasing on the interval $(0,1)$.
\begin{selectAll}
\choice[correct]{$f=A$}
\choice[correct]{$f=B$}
\choice{$f=C$}
\choice{$f=D$}
\choice{$f=E$}
\choice{$f=F$}
\end{selectAll}
\end{exercise}
\begin{exercise}
Circle all the functions $f$ for which the following statement is true.

The function $f$ is decreasing on its entire domain.

\begin{selectAll}
\choice{$f=A$}
\choice{$f=B$}
\choice[correct]{$f=C$}
\choice{$f=D$}
\choice[correct]{$f=E$}
\choice{$f=F$}
\end{selectAll}
\end{exercise}
\begin{exercise}
Circle all the functions $f$ for which the following statement is true.

$f'$, the derivative of $f$, is increasing on the interval $(0,1)$.

\begin{selectAll}
\choice[correct]{$f=A$}
\choice{$f=B$}
\choice[correct]{$f=C$}
\choice{$f=D$}
\choice{$f=E$}
\choice{$f=F$}
\end{selectAll}
\end{exercise}
\begin{exercise}

Match the  graphs above  to the functions given below.

\begin{align*}
\sin^{-1}(x) &= \answer{A}\\
\cos^{-1}(x) &= \answer{E}\\
\tan^{-1}(x) &= \answer{B}\\
\csc^{-1}(x) &= \answer{F}\\
\sec^{-1}(x) &= \answer{D}\\
\cot^{-1}(x) &= \answer{C}
\end{align*}

\end{exercise}
\end{exercise}
\end{document}
