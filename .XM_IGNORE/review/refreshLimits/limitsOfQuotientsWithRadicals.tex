\documentclass{ximera}

\newcommand{\RR}{\mathbb R}
\renewcommand{\d}{\,d}
\newcommand{\dd}[2][]{\frac{d #1}{d #2}}
\renewcommand{\l}{\ell}
\newcommand{\ddx}{\frac{d}{dx}}
\newcommand{\dfn}{\textbf}
\newcommand{\eval}[1]{\bigg[ #1 \bigg]}

\author{Jim Talamo}
\title[Refresh:]{Limits}

\begin{document}
\begin{abstract}
Recall results concerning infinite limits of rational functions.
\end{abstract}
\maketitle


Sometimes, quotients of functions involve radicals.  Being able to determine infinite limits of rational functions quickly is an important skill.  When evaluating these limits, all that is necessary is to consider the highest degree term in the numerator and the denominator.  As a reminder as to why (and how) this works, consider the following example.

\begin{example}
Compute $\lim_{x \to \infty} \frac{\sqrt{9x^2+1}}{2x+1}$.

\begin{explanation}
Note that in the numerator, the highest degree term in $9x^2+1$ is $9x^2$, so we expect $\sqrt{9x^2+1}$ to behave like $\sqrt{9x^2} = 3x$ as $x \to \infty$.  We can show this more formally as follows.

\begin{align*}
\lim_{x \to \infty} \frac{\sqrt{9x^2+1}}{2x+1} &= \lim_{x \to \infty}  \frac{\sqrt{x^2 \cdot \left(9+\frac{1}{x^2}\right)}}{2x+1} \\
&= \lim_{x \to \infty} \frac{\sqrt{x^2} \cdot \sqrt{9+\frac{1}{x^2}}}{2x+1}  \\
&= \lim_{x \to \infty} \frac{x \cdot \sqrt{9+\frac{1}{x^2}}}{x \cdot\left(1+\frac{1}{x}\right)} \\
&= \lim_{x \to \infty} \frac{\sqrt{9+\frac{1}{x^2}}}{\left(2+\frac{1}{x}\right)} \\
&= \frac{\sqrt{9}}{2} \\
&=\frac{3}{2}
\end{align*}

This is the same result we would have obtained by treating the numerator like $3x$ from the start.
\end{explanation}
\end{example}

Now, try your hand at the following.  If a limit is infinite, write ``$\infty$'' or ``$-\infty$''; note that Ximera will \emph{not} accept the syntax ``$+\infty$''. indicate this.  You should be able to evaluate these limits quickly as well as show the algebraic steps above if asked.

\begin{problem}
\[
\lim_{x \to \infty} \frac{\sqrt{4x^4+7}}{x^3-x+2} = \answer{0}
\]

\[
\lim_{x \to \infty} \frac{\sqrt{4+x}}{\sqrt{2x}+1} = \answer{\frac{1}{\sqrt{2}}}
\]

\end{problem}
\end{document}
