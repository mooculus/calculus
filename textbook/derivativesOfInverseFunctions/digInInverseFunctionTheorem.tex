\documentclass{ximera}

\newcommand{\RR}{\mathbb R}
\renewcommand{\d}{\,d}
\newcommand{\dd}[2][]{\frac{d #1}{d #2}}
\renewcommand{\l}{\ell}
\newcommand{\ddx}{\frac{d}{dx}}
\newcommand{\dfn}{\textbf}
\newcommand{\eval}[1]{\bigg[ #1 \bigg]}


\title[Dig-In:]{The inverse function theorem}

\begin{document}
\begin{abstract}
\end{abstract}
\maketitle


Geometrically, there is a close relationship between the plots of
$e^x$ and $\ln(x)$, they are reflections of each other over the line
$y=x$, see Figure~\ref{plot:e^x lnx}. One may suspect that we can use the fact
that $\ddx e^x = e^x$, to deduce the derivative of $\ln(x)$.  We will
use implicit differentiation to exploit this relationship
computationally.

\begin{image}
\begin{tikzpicture}
	\begin{axis}[
            xmin=-6,xmax=6,ymin=-6,ymax=6,
            axis lines=center,
            xlabel=$x$, ylabel=$y$,
            every axis y label/.style={at=(current axis.above origin),anchor=south},
            every axis x label/.style={at=(current axis.right of origin),anchor=west},
          ]        
          \addplot [very thick, penColor, smooth, domain=(-6:6)] {e^x};
          \addplot [very thick, penColor2, samples=100, smooth, domain=(.002:6)] {ln(x)};
          \addplot [dashed, textColor, domain=(-6:6)] {x};
          \node at (axis cs:-2,1) [penColor] {$e^x$};
          \node at (axis cs:1,-2) [penColor2] {$\ln(x)$};
        \end{axis}
\end{tikzpicture}
%% \caption{A plot of $e^x$ and $\ln(x)$. Since they are inverse
%%   functions, they are reflections of each other across the line $y=x$.}
%% \label{plot:e^x lnx}
\end{image}


\begin{theorem}[The Derivative of the Natural Logrithm]\index{derivative!of the natural logarithm}
\[
\ddx \ln(x) = \frac{1}{x}.
\]
\end{theorem}
\begin{proof}
Recall
\[
\ln(x) = y \qquad\Leftrightarrow\qquad e^y = x.
\]
Hence
\begin{align*}
e^y &= x\\
\ddx e^y &= \ddx x &\text{Differentiate both sides.}\\
e^y \dd[y]{x} &= 1 &\text{Implicit differentiation.}\\
\dd[y]{x} &= \frac{1}{e^y} = \frac{1}{x}.
\end{align*}
Since $y=\ln(x)$, $\ddx \ln(x) = \frac{1}{x}$.
\end{proof}

There is one catch to the proof given above. To write
$\ddx(e^y)=e^y\dd[y]{x}$ we need to know that the function $y$
\textit{has} a derivative. All we have shown is that \textit{if} it
has a derivative then that derivative must be $1/x$. The \textit{Inverse
Function Theorem} guarantees this.

\begin{theorem}[Inverse Function Theorem]\index{Inverse Function Theorem}\label{theorem:IFT}
If $f(x)$ is a differentiable function, and $f'(x)$ is continuous, and
$f'(a) \neq 0$, then
\begin{enumerate}
\item $f^{-1}(y)$ is defined for $y$ near $f(a)$,
\item $f^{-1}(y)$ is differentiable near $f(a)$, 
\item $\dd{y} f^{-1}(y)$ is continuous near $f(a)$, and
\item $\dd{y} f^{-1}(y)  = \displaystyle\frac{1}{f'(f^{-1}(y))}$.
\end{enumerate}
\end{theorem}

\end{document}
