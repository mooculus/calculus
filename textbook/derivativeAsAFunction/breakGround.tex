\documentclass{ximera}

\newcommand{\RR}{\mathbb R}
\renewcommand{\d}{\,d}
\newcommand{\dd}[2][]{\frac{d #1}{d #2}}
\renewcommand{\l}{\ell}
\newcommand{\ddx}{\frac{d}{dx}}
\newcommand{\dfn}{\textbf}
\newcommand{\eval}[1]{\bigg[ #1 \bigg]}


\outcome{Relate the derivative function to the derivative at a point.}
% Outcome: understanding variables (as bound or unbound)

\title[Break-Ground:]{Derivative as a function}

\begin{document}
\begin{abstract}
Two young mathematicians discuss derivatives as
functions.
\end{abstract}
\maketitle

Check out this dialogue between two calculus students (based on a true
story):

\begin{dialogue}
\item[Devyn] Riley, I might be a calculus genius.
\item[Riley] Yeah?  Explain this one to me.
\item[Devyn] Let me first ask you a question.  Say you have a function, like 
 $f(x) = x^2$, and you want to know $f'(3)$.  Do
  you plug in the number $3$ before or after you find the derivative?
\item[Riley] Hmmmm. Well, my next step is usually
  \[
  f'(3) = \lim_{h\to 0}\frac{f(3+h)-f(3)}{h}.
  \]
  So I guess before.
\item[Devyn] Aha!  I think you're wasting time. You see I write
  \[
  f'(x) = \lim_{h\to 0}\frac{f(x+h)-f(x)}{h}.
  \]
  and it means that I can look at the derivative of my function at
  \textit{any} point.  So, I plug in the $3$ {\em after} I've found the derivative.
\item[Riley] That does seem like a pretty genius move. But doesn't working
 with $x$, instead of numbers, make all of this more difficult?
\item[Devyn] Not at all. Let's do the problems both ways, at the same time:
  \[
 \underbrace{\begin{aligned}
    f'(3) &= \lim_{h\to 0}\frac{f(3+h)-f(3)}{h}\\
    &= \lim_{h\to 0}\frac{(3+h)^2-9}{h}\\
    &= \lim_{h\to 0}\frac{9+6h+h^2-9}{h}\\
    &= \lim_{h\to 0}\frac{6h+h^2}{h}\\
    &= \lim_{h\to 0}(6+h)\\
    &= 6.
  \end{aligned}}_{\text{plugging in}}
  \qquad
  \underbrace{\begin{aligned}
    f'(x) &= \lim_{h\to 0}\frac{f(x+h)-f(x)}{h}\\
    &= \lim_{h\to 0}\frac{(x+h)^2-x^2}{h}\\
    &= \lim_{h\to 0}\frac{x^2+2xh+h^2-x^2}{h}\\
    &= \lim_{h\to 0}\frac{2xh+h^2}{h}\\
    &= \lim_{h\to 0}(2x+h)\\
    &= 2x,\\
    \text{so }f'(3) &=6. 
  \end{aligned}}_{\text{working with $x$}}
  \]
  \item[Riley] Whoa. So now the derivative is a function. Wait, what's
    its domain? Its range?
\end{dialogue}


\begin{problem}
  Suppose you have a function $f$. Which of the following are true?
  \begin{selectAll}
    \choice{The domain of $f'$ is equal to the domain of $f$.}
    \choice{The range of $f'$ is equal to the range of $f$.}
    \choice[correct]{The domain of $f'$ is a subset of the real numbers.}
    \choice[correct]{The range of $f'$ is a subset of the real numbers.}
    \choice{The domain of $f'$ is functions from the real numbers to
      the real numbers.}
      \choice{The range of $f'$ is functions from the real numbers to
      the real numbers.}
  \end{selectAll}
\end{problem}

\begin{problem}
Find $g'(2)$ for $g(x) = x^2 + 1$ using both methods described above.
\[ g'(2) = \answer[given]{4} \]
\end{problem}


% Dialogue: this homework problem is so boring
% why?
% I'm doing the same thing over and over again
% Just replace the thing by a VARIABLE!
% Does that work?

% Names of variables are a potential point of confusion.
% Get students to understand that we're simultaneously evaluating the derivative at multiple points

% Continuity at a point, derivative at a point

% Embed the Khan Academy activity where students build the derivative by computing tangent lines at multiple points
% Build graph and connect dots with more computation

%% \begin{xarmaBoost}
%%   Write down at least \textbf{five} questions for this lecture. After
%%   you have your questions, label them as ``Level 1,'' ``Level 2,'' or
%%   ``Level 3'' where:
%% \begin{description}
%% \item[Level 1] Means you know the answer, or know exactly how to do
%%   this problem.
%% \item[Level 2] Means you think you know how to do the problem.
%% \item[Level 3] Means you have no idea how to do the problem.
%% \end{description}
%% \begin{freeResponse}
%% \end{freeResponse}
%% \end{xarmaBoost}


\end{document}
