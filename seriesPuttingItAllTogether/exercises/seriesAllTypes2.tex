\documentclass{ximera}

\newcommand{\RR}{\mathbb R}
\renewcommand{\d}{\,d}
\newcommand{\dd}[2][]{\frac{d #1}{d #2}}
\renewcommand{\l}{\ell}
\newcommand{\ddx}{\frac{d}{dx}}
\newcommand{\dfn}{\textbf}
\newcommand{\eval}[1]{\bigg[ #1 \bigg]}


\author{Jason Miller}
\license{Creative Commons 3.0 By-bC}


\outcome{}


\begin{document}

\begin{exercise}
For the following series, determine whether they converge or diverge by applying one of the standard tests. 


\begin{exercise}

\[
\sum^{\infty}_{n=1}
\left( \frac{6}{5^n} + \frac{9}{n} \right)
\]

\begin{multipleChoice}
\choice{converges}
\choice[correct]{diverges}
\end{multipleChoice}

\begin{feedback}[correct]
Use the comparison test and compare our series with the series $\sum^{\infty}_{n=1} \frac{1}{n}$. 
\end{feedback}
\end{exercise}


\begin{exercise}

\[
\sum^{\infty}_{k=1} \frac{5^k}{2^k k^5}
\]



\begin{multipleChoice}
\choice{converges}
\choice[correct]      {diverges}
\end{multipleChoice}

\begin{feedback}[correct]
Use root or ratio test. 
\end{feedback}
\end{exercise}


\begin{exercise}


\[
\sum^{\infty}_{k=1}  \cos\left( \frac{ 6k^3-1}{4k^4 + 5} \right)
\]



\begin{multipleChoice}
\choice{converges}
\choice[correct]{diverges}
\end{multipleChoice}

\begin{feedback}[correct]
Look at $\lim_{k \to \infty}  \cos\left( \frac{ 6k^3-1}{4k^4 + 5}\right)$. 
\end{feedback}
\end{exercise}




\begin{exercise}

\[
\sum^{\infty}_{k=1} \frac{(-1)^{k+1} k^2}{6+2k+k^3}
\]

\begin{multipleChoice}
\choice[correct]{converges}
\choice{diverges}
\end{multipleChoice}

\begin{feedback}[correct]
The series alternates in sign and the positive part of the summand $\frac{k^2}{6+2k+k^3}$ is positive and eventually decreasing. Thus we should use the alternating series test. 
\end{feedback}
\end{exercise}

\begin{exercise}

\[
\sum^{\infty}_{n=1} \frac{n^n}{n!}
\]


\begin{multipleChoice}
\choice{converges}
\choice[correct]{diverges}
\end{multipleChoice}


\begin{feedback}[correct]
Factorials suggest ratio test as a possibility. Even better is to look at $\lim_{n \to \infty} \frac{n^n}{n!}$ and use growth rates to determine the limit. 
\end{feedback}
\end{exercise}

\begin{exercise}

\[
\sum^{\infty}_{n=1} \frac{n^3+7}{\sqrt{n^6+4n^2+1}}
\]

\begin{multipleChoice}
\choice{converges}
\choice[correct]{diverges}
\end{multipleChoice}



\begin{feedback}[correct]
Look at the limit $\lim_{n \to \infty} \frac{n^3+7}{\sqrt{n^6+4n^2+1}}$. What can you conclude?
\end{feedback}
\end{exercise}


\begin{exercise}


\[
\sum^{\infty}_{n=1} \frac{7n^2+n^3-6}{-5n^2+10+3n^6}
\]


\begin{multipleChoice}
\choice[correct]{converges}
\choice{diverges}
\end{multipleChoice}


\begin{feedback}[correct]
For very large $n$, the dominant term in the numerator is $n^3$ and the dominant term in the denominator is $3n^6$. Thus the general term 
behaves like $\frac{n^3}{3n^6}$ as $n$ goes to $\infty$. This suggest doing the limit comparison term and comparing with $\sum^{\infty}_{n=1} \frac{1}{n^4}$. 
\end{feedback}

\end{exercise}

%%%%%%%%%%%%%%%%%

\begin{quote}
You should be able to give a detailed solution for each of your choices.  Such a solution should include:

\begin{itemize}
\item What test you chose.
\item Why you are allowed to use that test.
\item The computations required for the conclusions of the test.
\item An explanation why the series either converges or diverges by the test you chose.
\end{itemize}

\end{quote}
\end{exercise}

\end{document}
