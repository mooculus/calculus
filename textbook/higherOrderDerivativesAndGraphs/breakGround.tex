\documentclass{ximera}

\newcommand{\RR}{\mathbb R}
\renewcommand{\d}{\,d}
\newcommand{\dd}[2][]{\frac{d #1}{d #2}}
\renewcommand{\l}{\ell}
\newcommand{\ddx}{\frac{d}{dx}}
\newcommand{\dfn}{\textbf}
\newcommand{\eval}[1]{\bigg[ #1 \bigg]}


\outcome{Second derivatives}


\title[Break-Ground:]{Rates of rates}

\begin{document}
\begin{abstract}
Two young mathematicians discuss what happens when they differentiate
multiple times.
\end{abstract}
\maketitle

Check out this dialogue between two calculus students (based on a true
story):

\begin{dialogue}
  \item[Devyn] Riley, I have a pressing question: What does the
    derivative of derivative mean?
  \item[Riley] Hmmm, well if we imagine that a function gives the
    position of an object with respect to time,
    \[
    s(t) = \text{position with respect to time}
    \]
    then the derivative is the change in position, so that should be
    velocity. Meaning
    \[
    s'(t) = \text{velocity with respect to time}.
    \]
  \item[Devyn] Ah! Right! A derivative is something like a
    ``speedometer'' for the growth of a function.
  \item[Riley] Ooooh! So if we want to know the derivative of
    velocity, we want to know the change in velocity with respect to
    time, and that is acceleration! So we have
    \[
    s''(t) = \text{acceleration with respect to time}.
    \]
  \item[Devyn] Hmmm what does $s'''(t)$ mean?
    \item[Riley] Good question! I wonder if we can recover $s(t)$ if
      we know $s'(t)$, and/or $s''(t)$?
\end{dialogue}

\begin{problem}
  Suppose that $s$ represents the position of a ball tossed at time
  $t=0$. Recalling that the acceleration for $t>0$ is only due to
  gravity, and knowing that the acceleration due to gravity is
  $-9.8~\mathrm{m}/\mathrm{s}^2$, what is $s''(t)$?
  \begin{prompt}
  \[
  s''(t) = \answer{-9.8}
  \]
  \end{prompt}
\end{problem}


%% \begin{xarmaBoost}
%%   Write down at least \textbf{five} questions for this lecture. After
%%   you have your questions, label them as ``Level 1,'' ``Level 2,'' or
%%   ``Level 3'' where:
%% \begin{description}
%% \item[Level 1] Means you know the answer, or know exactly how to do
%%   this problem.
%% \item[Level 2] Means you think you know how to do the problem.
%% \item[Level 3] Means you have no idea how to do the problem.
%% \end{description}
%% \begin{freeResponse}
%% \end{freeResponse}
%% \end{xarmaBoost}



\end{document}
