\documentclass{ximera}

\newcommand{\RR}{\mathbb R}
\renewcommand{\d}{\,d}
\newcommand{\dd}[2][]{\frac{d #1}{d #2}}
\renewcommand{\l}{\ell}
\newcommand{\ddx}{\frac{d}{dx}}
\newcommand{\dfn}{\textbf}
\newcommand{\eval}[1]{\bigg[ #1 \bigg]}


\title[Dig-In:]{Continuity}
\begin{document}
\begin{abstract}
Using continuity to find limits
\end{abstract}
\maketitle

%This is the first dig in in this section

We learned previously that just estimating limits by plugging in nearby values
does not always give the correct answer.  We need some theorems that
tell us how to quickly find limits, at least in some special cases.  

Limits are simple to compute when they can be found by plugging the
value into the function.  That is, when
\[
\lim_{x\to c}f(x) = f(c).
\]
We call this property \textit{continuity}.

\begin{definition}\index{continuous}
  A function $f$ is \dfn{continuous at a point} $a$ if
  \[
  \lim_{x\to a}f(x) = f(a).
  \]
\end{definition}


It is very important to note that saying a function $f$ is continuous
at a point $a$ is really making three statements:
\begin{enumerate}
\item $f(a)$ is defined.  That is, $a$ is in the domain of $f$.
\item $\lim_{x\to a}f(x)$ exists.
\item $\lim_{x\to a} f(x) = f(a)$.
\end{enumerate}

The first two of these statements are implied by the third statement.

\begin{example}
Find the discontinuities (the points $x$ where a function is not
continuous) for the function described below:
\begin{image}
\begin{tikzpicture}
	\begin{axis}[
            domain=0:10,
            ymax=5,
            ymin=0,
            samples=100,
            axis lines =middle, xlabel=$x$, ylabel=$y$,
            every axis y label/.style={at=(current axis.above origin),anchor=south},
            every axis x label/.style={at=(current axis.right of origin),anchor=west}
          ]
	  \addplot [very thick, penColor, smooth, domain=(4:10)] {3 + sin(deg(x*2))/(x-1)};
          \addplot [very thick, penColor, smooth, domain=(0:4)] {1};
          \addplot[color=penColor,fill=background,only marks,mark=*] coordinates{(4,3.30)};  %% open hole
          \addplot[color=penColor,fill=background,only marks,mark=*] coordinates{(6,2.893)};  %% open hole
          \addplot[color=penColor,fill=penColor,only marks,mark=*] coordinates{(4,1)};  %% closed hole
          \addplot[color=penColor,fill=penColor,only marks,mark=*] coordinates{(6,2)};  %% closed hole
        \end{axis}
\end{tikzpicture}
%% \caption{A plot of a function with discontinuities at $x=4$ and $x=6$.}
%% \label{plot:discontinuous-function}
\end{image}

List the $x$-values of the points of discontinuity.  $\answer[given]{4,6}$ %%BADBAD Need list of answers
\end{example}

\begin {explanation}
From the plot above we see that $\lim_{x\to 4} f(x)$ does not exist
because
\[
\lim_{x\to 4-}f(x) = 1\qquad\text{and}\qquad \lim_{x\to 4+}f(x) \approx 3.5
\]
Hence $\lim_{x\to 4} f(x) \ne f(4)$, and so $f(x)$ is not
continuous at $x=4$.

We also see that $\lim_{x\to 6} f(x) \approx 3$ while $f(6) =
2$. Hence $\lim_{x\to 6} f(x)$ does not exist, and so $f$ is not
continuous at $x=6$.
\end{explanation}

Building from the definition of \textit{continuity at a point}, we can
now define what it means for a function to be \textit{continuous} on
an interval.

\begin{definition}
  A function $f$ is \dfn{continuous on the open interval $I$} if $\lim_{x\to a}
  f(x) = f(a)$ for all $a$ in $I$.
\end{definition}

Graphically, a function is continuous on an interval $I$ if you can
draw the function on that interval without any breaks in the graph.
This is often referred to as being able to draw the graph ``without
picking up your pencil.''

Many of the functions we are most familiar with are continuous on
their domain.

\begin{theorem}[Continuity of Famous Functions]\index{continuity of famous functions}\label{theorem:continuity}
The following functions are continuous on the given intervals for $k$ a real number and $b$ a positive real number:
\[
\begin{array}{l|l}
  \text{Function} & \text{Domain}\\\hline
f(x)=k   & -\infty < x < \infty\\
f(x)=x   & -\infty <x <\infty\\
%We can get polynomials and rationals with the limit laws
f(x)=x^k &  \begin{minipage}{2in}\[
  -\infty<x<\infty \](Note: When $k$ is such that $x^k$ is an even root of x, $0< x <\infty$)\end{minipage}\\ %% at zero is a problem
f(x)=b^x &  -\infty <x <\infty\\
f(x)=\log_b(x) & 0<x<\infty\\ 
f(x)=\sin(x) & -\infty <x <\infty\\
f(x)=\cos(x) & -\infty <x <\infty
\end{array}
\]
\end{theorem}

This theorem can be summarized by saying that the functions listed above are continuous wherever they are defined, that is, on their natural domain. 

We have a small problem, though.  For functions such as $\sqrt{x}$, the natural domain is $0\leq x <\infty$.  This is not an open interval.  What does it mean to say that $\sqrt{x}$ is continuous at 0 when $\sqrt{x}$ is not defined for $x<0$?  We need a new definition:

\begin{definition}
  A function $f$ is \dfn{left continuous} at a point $a$ if $\lim_{x\to a^-}
  f(x) = f(a)$.   A function $f$ is \dfn{right continuous} at a point $a$ if $\lim_{x\to a^+}
  f(x) = f(a)$.  
\end{definition}

Now we can say that a function is continuous at a left endpoint of an interval if it is right continuous there, and a function is continuous at the right endpoint of an interval if it is left continuous there. This allows us to talk about continuity on closed intervals.  

\begin{example}
List the largest intervals of continuity for the function discribed below.  Make the assumption the function is defined on $[0,10]$.
\begin{image}
\begin{tikzpicture}
	\begin{axis}[
            domain=0:10,
            ymax=5,
            ymin=0,
            samples=100,
            axis lines =middle, xlabel=$x$, ylabel=$y$,
            every axis y label/.style={at=(current axis.above origin),anchor=south},
            every axis x label/.style={at=(current axis.right of origin),anchor=west}
          ]
	  \addplot [very thick, penColor, smooth, domain=(4:10)] {3 + sin(deg(x*2))/(x-1)};
          \addplot [very thick, penColor, smooth, domain=(0:4)] {1};
          \addplot[color=penColor,fill=background,only marks,mark=*] coordinates{(4,3.30)};  %% open hole
          \addplot[color=penColor,fill=background,only marks,mark=*] coordinates{(6,2.893)};  %% open hole
          \addplot[color=penColor,fill=penColor,only marks,mark=*] coordinates{(4,1)};  %% closed hole
          \addplot[color=penColor,fill=penColor,only marks,mark=*] coordinates{(6,2)};  %% closed hole
        \end{axis}
\end{tikzpicture}
%% \caption{A plot of a function with discontinuities at $x=4$ and $x=6$.}
%% \label{plot:discontinuous-function}
\end{image}

List the largest intervals of continuity for this function.  Separate your intervals with commas.  $\answer[given]{[0,4],(4,6),(6,10]}$ %%BADBAD Need list of answers

\begin{feedback}
Notice that our function is left continuous at $x=4$ so we can include 4 in the interval $[0,4]$. Four is not included in the interval (4,6) because our function is not right continuous at $x=4$.  Similarly, our function is neither right or left continuous at $x=6$, so 6 is not included in any intervals.  Our function is left continuous at $x=0$ and right continuous at $x=10$ so we included these endpoints in our intervals. 
\end{feedback}
\end{example}

For many of the functions we are most familiar with, we can use continuity to determine limit values.

\begin{example}
$\lim_{x\to 3} x^\pi = \answer[given]{3^\pi}$
  \begin{explanation}
   The function $f(x)=x^\pi$ is of the form $x^k$ for a real number $k$.  Therefore, $f(x)=x^\pi$ is continuous for all real values of $x$.  In particular, $f(x)$ is continuous at $x=3$.  Since $x^\pi$ is continuous at $3$, we know that $\lim_{x\to 3} f(x) = f(3)$.  That is, $\lim_{x\to 3} x^\pi = 3^\pi$
  \end{explanation}  
\end{example}

\end{document}

