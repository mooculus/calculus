\documentclass{ximera}

\newcommand{\RR}{\mathbb R}
\renewcommand{\d}{\,d}
\newcommand{\dd}[2][]{\frac{d #1}{d #2}}
\renewcommand{\l}{\ell}
\newcommand{\ddx}{\frac{d}{dx}}
\newcommand{\dfn}{\textbf}
\newcommand{\eval}[1]{\bigg[ #1 \bigg]}


\author{Bart Snapp}

\outcome{Give the equation for an elliptic paraboloid.}
\outcome{Give the equation for a hyperbolic paraboloid.}

\title[Dig-In:]{Drawing paraboloids}

\begin{document}
\begin{abstract}
  Learn how to draw an elliptic and a hyperbolic paraboloid.
\end{abstract}
\maketitle

At this point, there are two basic surfaces that you should get to know quite well:
\begin{itemize}
\item Elliptic paraboloids.\index{ellipic paraboloid}\index{paraboliod!ellipic}
\item Hyperbolic paraboloids.\index{hyperbolic parabolid}\index{paraboliod!hyperbolic}
\end{itemize}

For both of these surfaces, if they are sliced by a plane
perpendicular to the plane $z=0$, the cross-section looks like a
parabola, hence the name \index{paraboloid}\textit{paraboloid}.

\subsection{Drawing an elliptic paraboloid}

We'll start with elliptic paraboloid that opens up, like $z=
x^2+y^2$. You begin by drawing a set of axis:
\begin{image}
  \begin{tikzpicture}  
    \begin{axis}[  
        xmin=-1.2,  
        xmax=1.2,  
        ymin=-1.2,  
        ymax=1.2,
        unit vector ratio=1 1 1,
        axis lines=center,
        ticks=none,
        xlabel=$y$,  
        ylabel=$z$,  
        every axis y label/.style={at=(current axis.above origin),anchor=south},  
        every axis x label/.style={at=(current axis.right of origin),anchor=west},  
      ]  
      \addplot [->] coordinates {(.7,.5) (-.7,-.5)};
      \node at (axis cs: -.75,-.52) {$x$};
    \end{axis}  
  \end{tikzpicture}  
\end{image}
Now draw a parabola in the $(y,z)$-plane:
\begin{image}
  \begin{tikzpicture}  
    \begin{axis}[  
        xmin=-1.2,  
        xmax=1.2,  
        ymin=-1.2,  
        ymax=1.2,
        unit vector ratio=1 1 1,
        axis lines=center,
        ticks=none,
        xlabel=$y$,  
        ylabel=$z$,  
        every axis y label/.style={at=(current axis.above origin),anchor=south},  
        every axis x label/.style={at=(current axis.right of origin),anchor=west},  
      ]  
      \addplot [->] coordinates {(.7,.5) (-.7,-.5)};
      \addplot [ultra thick, penColor,smooth,domain=-.7:.7] {2*x^2};
      \node at (axis cs: -.75,-.52) {$x$};
    \end{axis}  
  \end{tikzpicture}  
\end{image}
Now add-in an ellipse:
\begin{image}
  \begin{tikzpicture}  
    \begin{axis}[  
        xmin=-1.2,  
        xmax=1.2,  
        ymin=-1.2,  
        ymax=1.2,
        unit vector ratio=1 1 1,
        axis lines=center,
        ticks=none,
        xlabel=$y$,  
        ylabel=$z$,  
        every axis y label/.style={at=(current axis.above origin),anchor=south},  
        every axis x label/.style={at=(current axis.right of origin),anchor=west},  
      ]  
      \addplot [->] coordinates {(.7,.5) (-.7,-.5)};
      \addplot [ultra thick, penColor,smooth,domain=-.7:.7] {2*x^2};
      \node at (axis cs: -.75,-.52) {$x$};
      \addplot [ultra thick, penColor,domain=0:360,smooth] ({.7*cos(x)},{.1*sin(x)+1});
    \end{axis}  
  \end{tikzpicture}  
\end{image}
To draw an elliptic parabolid opening down, like $z= -x^2-y^2$ you
draw something like this:
\begin{image}
  \begin{tikzpicture}  
    \begin{axis}[  
        xmin=-1.2,  
        xmax=1.2,  
        ymin=-1.2,  
        ymax=1.2,
        unit vector ratio=1 1 1,
        axis lines=center,
        ticks=none,
        xlabel=$y$,  
        ylabel=$z$,  
        every axis y label/.style={at=(current axis.above origin),anchor=south},  
        every axis x label/.style={at=(current axis.right of origin),anchor=west},  
      ]  
      \addplot [->] coordinates {(.7,.5) (-.7,-.5)};
      \addplot [ultra thick, penColor,smooth,domain=-.7:.7] {-2*x^2};
      \node at (axis cs: -.75,-.52) {$x$};
      \addplot [ultra thick,dashed,penColor,domain=0:180,smooth] ({.7*cos(x)},{.1*sin(x)-1});
      \addplot [ultra thick,penColor,domain=180:360,smooth] ({.7*cos(x)},{.1*sin(x)-1});
    \end{axis}  
  \end{tikzpicture}  
\end{image}








\subsection{Drawing a hyperbolic paraboloid}

The hyperbolic parabolid often gives people more trouble. Let's teach
you the tricks of the trade. Let's first draw
\[
z = x^2-y^2
\]
Note here, as the absolute value of $x$ increases, the $z$-values
increase. As the abosolute value of $y$ increase, the $z$-values
decrease. We'll see this in our drawing. You begin by drawing a set of
axis:
\begin{image}
  \begin{tikzpicture}  
    \begin{axis}[
        clip=false,
        xmin=-1.2,  
        xmax=1.2,  
        ymin=-1.2,  
        ymax=1.2,
        unit vector ratio=1 1 1,
        axis lines=center,
        ticks=none,
        xlabel=$y$,  
        ylabel=$z$,  
        every axis y label/.style={at=(current axis.above origin),anchor=south},  
        every axis x label/.style={at=(current axis.right of origin),anchor=west},  
      ]  
      \node at (axis cs: -.75,-.52) {$x$};
      \addplot [->] coordinates {(.7,.5) (-.7,-.5)};
      %% \addplot [ultra thick, penColor,smooth,domain=-1.3:.1] {-2*(x+.6)^2+.22};
      %% \addplot [ultra thick, penColor,smooth,domain=0:1.2] {-2*(x-.6)^2+.72};
      %% \addplot [ultra thick, penColor,smooth] coordinates {
      %%   (-1.3,-.76) (-0.6,-.48) (-.23,-.27) (-.15,-.18)};

      %% \addplot [ultra thick, penColor,smooth] coordinates {
      %%   (-.36,.1) (0,0) (.23,.1) (.43,.34) (.6,.72)};

      %% \addplot [ultra thick, penColor,smooth] coordinates {
      %%   (.1,-.77) (.28,-.36) (.52,-.15) (.8,-.08) (1.2,0)};
      
    \end{axis}  
  \end{tikzpicture}  
\end{image}
Now draw a parabola in a $(y,z)$-plane that contains some positive $x$-value.
\begin{image}
  \begin{tikzpicture}  
    \begin{axis}[
        clip=false,
        xmin=-1.2,  
        xmax=1.2,  
        ymin=-1.2,  
        ymax=1.2,
        unit vector ratio=1 1 1,
        axis lines=center,
        ticks=none,
        xlabel=$y$,  
        ylabel=$z$,  
        every axis y label/.style={at=(current axis.above origin),anchor=south},  
        every axis x label/.style={at=(current axis.right of origin),anchor=west},  
      ]  
      \node at (axis cs: -.75,-.52) {$x$};
      \addplot [->] coordinates {(.7,.5) (-.7,-.5)};
      \addplot [ultra thick, penColor,smooth,domain=-1.3:.1] {-2*(x+.6)^2+.22};
      %% \addplot [ultra thick, penColor,smooth,domain=0:1.2] {-2*(x-.6)^2+.72};
      %% \addplot [ultra thick, penColor,smooth] coordinates {
      %%   (-1.3,-.76) (-0.6,-.48) (-.23,-.27) (-.15,-.18)};

      %% \addplot [ultra thick, penColor,smooth] coordinates {
      %%   (-.36,.1) (0,0) (.23,.1) (.43,.34) (.6,.72)};

      %% \addplot [ultra thick, penColor,smooth] coordinates {
      %%   (.1,-.77) (.28,-.36) (.52,-.15) (.8,-.08) (1.2,0)};
      
    \end{axis}  
  \end{tikzpicture}  
\end{image}
Now draw a curvy line:
\begin{image}
  \begin{tikzpicture}  
    \begin{axis}[
        clip=false,
        xmin=-1.2,  
        xmax=1.2,  
        ymin=-1.2,  
        ymax=1.2,
        unit vector ratio=1 1 1,
        axis lines=center,
        ticks=none,
        xlabel=$y$,  
        ylabel=$z$,  
        every axis y label/.style={at=(current axis.above origin),anchor=south},  
        every axis x label/.style={at=(current axis.right of origin),anchor=west},  
      ]  
      \node at (axis cs: -.75,-.52) {$x$};
      \addplot [->] coordinates {(.7,.5) (-.7,-.5)};
      \addplot [ultra thick, penColor,smooth,domain=-1.3:.1] {-2*(x+.6)^2+.22};
      %% \addplot [ultra thick, penColor,smooth,domain=0:1.2] {-2*(x-.6)^2+.72};
      %% \addplot [ultra thick, penColor,smooth] coordinates {
      %%    (-1.3,-.76) (-0.6,-.48) (-.23,-.27) (-.15,-.18)};

      \addplot [ultra thick, penColor,smooth] coordinates {
        (-.36,.1) (0,0) (.23,.1) (.43,.34) (.6,.72)};

      %% \addplot [ultra thick, penColor,smooth] coordinates {
      %%   (.1,-.77) (.28,-.36) (.52,-.15) (.8,-.08) (1.2,0)};
      
    \end{axis}  
  \end{tikzpicture}  
\end{image}
And another parabola:
\begin{image}
  \begin{tikzpicture}  
    \begin{axis}[
        clip=false,
        xmin=-1.2,  
        xmax=1.2,  
        ymin=-1.2,  
        ymax=1.2,
        unit vector ratio=1 1 1,
        axis lines=center,
        ticks=none,
        xlabel=$y$,  
        ylabel=$z$,  
        every axis y label/.style={at=(current axis.above origin),anchor=south},  
        every axis x label/.style={at=(current axis.right of origin),anchor=west},  
      ]  
      \node at (axis cs: -.75,-.52) {$x$};
      \addplot [->] coordinates {(.7,.5) (-.7,-.5)};
      \addplot [ultra thick, penColor,smooth,domain=-1.3:.1] {-2*(x+.6)^2+.22};
      \addplot [ultra thick, penColor,smooth,domain=0:1.2] {-2*(x-.6)^2+.72};
      %% \addplot [ultra thick, penColor,smooth] coordinates {
      %%    (-1.3,-.76) (-0.6,-.48) (-.23,-.27) (-.15,-.18)};

      \addplot [ultra thick, penColor,smooth] coordinates {
        (-.36,.1) (0,0) (.23,.1) (.43,.34) (.6,.72)};

      %% \addplot [ultra thick, penColor,smooth] coordinates {
      %%   (.1,-.77) (.28,-.36) (.52,-.15) (.8,-.08) (1.2,0)};
      
    \end{axis}  
  \end{tikzpicture}  
\end{image}
Finally, add in the base:
\begin{image}
  \begin{tikzpicture}  
    \begin{axis}[
        clip=false,
        xmin=-1.2,  
        xmax=1.2,  
        ymin=-1.2,  
        ymax=1.2,
        unit vector ratio=1 1 1,
        axis lines=center,
        ticks=none,
        xlabel=$y$,  
        ylabel=$z$,  
        every axis y label/.style={at=(current axis.above origin),anchor=south},  
        every axis x label/.style={at=(current axis.right of origin),anchor=west},  
      ]  
      \node at (axis cs: -.75,-.52) {$x$};
      \addplot [->] coordinates {(.7,.5) (-.7,-.5)};
      \addplot [ultra thick, penColor,smooth,domain=-1.3:.1] {-2*(x+.6)^2+.22};
      \addplot [ultra thick, penColor,smooth,domain=0:1.2] {-2*(x-.6)^2+.72};
      \addplot [ultra thick, penColor,smooth] coordinates {
         (-1.3,-.76) (-0.6,-.48) (-.23,-.27) (-.15,-.18)};

      \addplot [ultra thick, penColor,smooth] coordinates {
        (-.36,.1) (0,0) (.23,.1) (.43,.34) (.6,.72)};

      \addplot [ultra thick, penColor,smooth] coordinates {
        (.1,-.77) (.28,-.36) (.52,-.15) (.8,-.08) (1.2,0)};
      
    \end{axis}  
  \end{tikzpicture}  
\end{image}
And you have a hyperbolic paraboloid!

\end{document}
