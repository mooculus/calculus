\documentclass{ximera}

\newcommand{\RR}{\mathbb R}
\renewcommand{\d}{\,d}
\newcommand{\dd}[2][]{\frac{d #1}{d #2}}
\renewcommand{\l}{\ell}
\newcommand{\ddx}{\frac{d}{dx}}
\newcommand{\dfn}{\textbf}
\newcommand{\eval}[1]{\bigg[ #1 \bigg]}


\title[Dig-In:]{Applied related rates}

\begin{document}
\begin{abstract}
\end{abstract}
\maketitle


Suppose we have two variables $x$ and $y$ which are both changing with
respect to time.  A \textit{related rates} problem is a problem where
we know one rate at a given instant, and wish to find the other.  If
$y$ is written in terms of $x$, and we are given $\dd[x]{t}$, then it
is easy to find $\dd[y]{t}$ using the chain rule:
\[
\dd[y]{t}=y'(x(t))\cdot x'(t).
\]
In many cases, particularly the interesting ones, our functions will
be related in some other way. Nevertheless, in each case we'll use the
same strategy:

REWRITE

\begin{example}
You are inflating a spherical balloon at the rate of 7 cm${}^3$/sec.  How
fast is its radius increasing when the radius is 4 cm?
\end{example}

\begin{solution}
To start, \textbf{draw a picture}.

\begin{tikzpicture}
%\draw[penColor!50!background,very thick] (0,0) ellipse (2 and 1);
\draw[very thick,penColor!20!background] (2,0) arc (0:180:2 and .7);% top half of ellipse
\draw [penColor, very thick] (0,0) circle [radius=2];
\draw[penColor2, dashed, very thick] (0,0) -- (2,0);
\node [below,penColor2] at (1,0) {$r=4$ cm};
\draw[very thick,penColor] (-2,0) arc (180:360:2 and .7);% bottom half of ellipse
\node [penColor,left] at (-1.5,1.42) {$\dd[V]{t} = 7$ cm$^3$/sec};
\node [penColor, right] at (1.5,-1.42) {$V = \frac{4\pi r^3}{3}$ cm$^3$};
\end{tikzpicture}

Next we need to \textbf{find an equation}.  Thinking of the variables
$r$ and $V$ as functions of time, they are related by the equation
\[
V(t)=\frac{4\pi (r(t))^3}{3}.
\]

Now we need to \textbf{differentiate the equation}.  Taking the
derivative of both sides gives 
\[
\dd[V]{t}=4\pi (r(t))^2\cdot r'(t).
\]  
Finally we \textbf{evaluate the equation at the desired values}. Set
$r(t)= 4$ cm and $\dd[V]{t}$ = 7 cm$^3$/sec. Write 
\begin{align*}
7 &=4\pi 4^2r'(t),\\
r'(t) &=7/(64\pi)~\text{cm/sec}.
\end{align*}
\end{solution}

\begin{example} Water is poured into a conical container at the rate of 10
cm${}^3$/sec.  The cone points directly down, and it has a height of
30 cm and a base radius of 10 cm.  How fast is the water level rising
when the water is 4 cm deep?
\end{example}

\begin{solution}
To start, \textbf{draw a picture}.

\begin{tikzpicture}
\draw[penColor,very thick] (0,4) ellipse (4 and 1);
\draw[very thick,penColor!20!background] (2,2) arc (0:180:2 and .5);% top half of ellipse
\draw[very thick,penColor] (-2,2) arc (180:360:2 and .5);% bottom half of ellipse
\draw[penColor, very thick] (3.97,3.85) -- (0,0);
\draw[penColor, very thick] (-3.97,3.85) -- (0,0);
\draw[penColor, very thick] (0,4) -- (4,4);
\draw[penColor!50!background, very thick] (0,2) -- (2,2);
\draw[->,line width=0.4cm, penColor!20!background] (0,6) -- (0,4.25);
\draw[dashed, penColor2, very thick] (2.1,0) -- (2.1,2);
\draw[dashed, penColor, very thick] (-4.1,0) -- (-4.1,4);
\node[right, penColor] at (.4,5.6) {$\dd[V]{t} = 10$ cm$^3$/sec};
\node[below, penColor] at (2,4) {$10$ cm};
\node[above, penColor] at (1,2) {$r$ cm};
\node[right, penColor2] at (2.1,1) {$h(t) = 4$ cm};
\node[left, penColor] at (-4.1,2) {$30$ cm};
\end{tikzpicture}

Note, no attempt was made to draw this picture to scale, rather we
want all of the relevant information to be available to the
mathematician.

Now we need to \textbf{find an equation}. The formula for the volume of a cone tells us that 
\[
V = \frac{\pi}{3} r^2 h.
\]

Now we must \textbf{differentiate the equation}. We should use implicit differentiation, and treat each of the variables as functions of $t$. Write
\begin{equation}\label{equation:cone/water}
\dd[V]{t} = \frac{\pi}{3}\left(2rh \dd[r]{t} + r^2 \dd[h]{t}\right).
\end{equation}

At this point we \textbf{evaluate the equation at the desired values}.
At first something seems to be wrong, we do not know $\dd[r]{t}$.
However, the dimensions of the cone of water must have the same
proportions as those of the container.  That is, because of similar
triangles, 
\[
\frac{r}{h}=\frac{10}{30} \qquad\text{so}\qquad r={h/3}.
\]  
In particular, we see that when $h = 4$, $r=4/3$ and 
\[
\dd[r]{t} = \frac{1}{3}\cdot \dd[h]{t}.
\]
Now we can \textbf{evaluate the equation at the desired
  values}. Starting with Equation~\ref{equation:cone/water}, we plug
in $\dd[V]{t} = 10$, $r = 4/3$, $\dd[r]{t} = \frac{1}{3}\cdot \dd[h]{t}$
and $h=4$. Write
\begin{align*}
10 &= \frac{\pi}{3}\left(2\cdot \frac{4}{3}\cdot 4 \cdot\frac{1}{3}\cdot\dd[h]{t} + \left(\frac{4}{3}\right)^2 \dd[h]{t}\right)\\
10 &= \frac{\pi}{3}\left(\frac{32}{9}\dd[h]{t} + \frac{16}{9} \dd[h]{t}\right)\\
10 &= \frac{16\pi}{9}\dd[h]{t}\\
\frac{90}{16\pi} &= \dd[h]{t}.
\end{align*}
Thus, $\dd[h]{t}=\frac{90}{16\pi}$ cm/sec.
\end{solution}



\end{document}
