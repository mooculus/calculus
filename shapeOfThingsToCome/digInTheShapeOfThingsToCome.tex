\documentclass{ximera}

\newcommand{\RR}{\mathbb R}
\renewcommand{\d}{\,d}
\newcommand{\dd}[2][]{\frac{d #1}{d #2}}
\renewcommand{\l}{\ell}
\newcommand{\ddx}{\frac{d}{dx}}
\newcommand{\dfn}{\textbf}
\newcommand{\eval}[1]{\bigg[ #1 \bigg]}


\title[Dig-In:]{The shape of things to come}

\begin{document}
\begin{abstract}
\end{abstract}
\maketitle


Derivatives and topology

%% \[
%% \vec{0} \lto \{\vector{a,b,c}:\R^3\to \R\} \lto \{F:\R^3 \to \R\} \lto^{\grad} \{\vec{F}: \R^3 \to \R^3\} \lto^{\curl} \vec{0}
%% \]
%% \[
%% \vec{0}\lto \{\vector{a,b,c}:\R^3\to \R^3\}\lto \{\vec{F}: \R^3 \to \R^3\}\lto^{\curl} \{\vec{F}:\R^3 \to \R^3\} \lto^{\divergence} 0
%% \]
\[
C^\infty(\R,\R)\lto^{\grad} C^\infty(\R,\R)
\]
\[
C^\infty(\R^2,\R) \lto^{\grad} C^\infty(\R^2,\R^2) \lto^{\curl} C^\infty(\R^2,\R)
\]
\[
C^\infty(\R^3,\R) \lto^{\grad} C^\infty(\R^3,\R^3) \lto^{\curl}C^\infty(\R^3,\R^3) \lto^{\divergence} C^\infty(\R^3,\R)
\]
\[
C^\infty(\R^4,\R) \lto^{\grad} C^\infty(\R^4,\R^4) \lto C^\infty(\R^4,\R^6) \lto C^\infty(\R^4,\R^4) \lto^{\divergence} C^\infty(\R^4,\R)
\]
\[
C^\infty(\R^5,\R) \lto^{\grad} C^\infty(\R^5,\R^5) \lto C^\infty(\R^5,\R^{10}) \lto C^\infty(\R^5,\R^{10}) \lto C^\infty (\R^5,\R^5)\lto^{\divergence} C^\infty(\R^5,\R)
\]
An attempt to show a general idea

\[
\int_a^b f'(x) \d x = f(b) - f(a)
\]
Line integral
\[
\int_{[a,b]} \d f = \int_{\partial[a,b]} f
\]
Stokes Theorem
\[
\int_{\partial S} \vec{F}(x,y,z)\dotp \d \vec{x} = \iint_S \left(\curl\vec{F}\right)\dotp \d \vec{S}
\]
Kelvin-Stokes Theorem
\[
\iint_S \left(\curl\vec{F}\right)\dotp \d \vec{S} = \int_{\partial S} \vec{F}(x,y,z)\dotp \d \vec{x}
\]
Divergence Theorem
\[
\iint_S \vec{F}\dotp \d \vec{A} = \iiint_R (\divergence \vec{F}) \d V
\]
\[
\int_R \d \vec{F} = \int_{\partial R} \vec{F}
\]

The accumulation of the rate of change of a function is determined by
the value of the function on the boundry. ((say something about FTC and how it is there the entire time))

\begin{quote}
  We shall not cease from exploration\\
  And the end of all our exploring \\
  Will be to arrive where we started \\
  And know the place for the first time.

  \hfill ---T.S.\ Eliot
\end{quote}





\end{document}
