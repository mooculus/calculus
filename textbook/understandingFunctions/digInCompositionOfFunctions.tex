\documentclass{ximera}

\newcommand{\RR}{\mathbb R}
\renewcommand{\d}{\,d}
\newcommand{\dd}[2][]{\frac{d #1}{d #2}}
\renewcommand{\l}{\ell}
\newcommand{\ddx}{\frac{d}{dx}}
\newcommand{\dfn}{\textbf}
\newcommand{\eval}[1]{\bigg[ #1 \bigg]}


\outcome{Find the domain and range of a function.}
\outcome{Distinguish between functions by considering their domains.}
\outcome{Perform basic operations and compositions on functions.}
\outcome{Work with piecewise defined functions.}
\outcome{Recognize different representations of the same function.}

\title{Compositions of functions}
\begin{document}
\begin{abstract}
  We discuss compositions of functions.
\end{abstract}
\maketitle


%% \section{Arithmetic on functions}

%% Functions can be combined using our standard operations: addition,
%% subtraction, multiplication, and division.

%% \begin{example}
%% Suppose we have: 
%% \begin{align*}
%%   f(x) &=x^2+4 \\ 
%%   g(x) &=\frac{1}{x} \\ 
%%   h(x) &=3+\sqrt{x}  
%% \end{align*}

%% Let's first note the domain of each function.  The domain of $f$ is
%% $(-\infty ,\infty )$, domain of $g$ is $(-\infty ,0)\cup (0,\infty )$,
%% domain of $h$ is $[0,\infty)$
%% \begin{enumerate}
%% \item Find $f+g$
  
%%   Solution: $f+g=x^2+4+\frac{1}{x}$  The domain is $(-\infty ,0)\cup (0,\infty )$
%% \item Find$g-h$ 

%%   Solution: $g-h=\frac{1}{x}-(3+\sqrt{x})=\frac{1}{x}-3-\sqrt{x}$ The domain is $(0,\infty )$
%% \item Find$f\cdot h$
  
%% Solution: $f\cdot g=\left( {{x}^{2}}+4 \right)\cdot \left( 3+\sqrt{x} \right)=3{{x}^{2}}+4\sqrt{x}+{{x}^{3/2}}+12$ The domain is$[0,\infty )$
%% \item Find$\frac{g}{h}$ 

%%   Solution:
%%   \begin{align*}
%%     \frac{g}{h} &=\frac{\left( \frac{1}{x} \right)}{3+\sqrt{x}}\\
%%     &=\frac{1}{x}\div \left( 3+\sqrt{x} \right)\\
%%     &=\frac{1}{x}\cdot \frac{1}{\left( 3+\sqrt{x} \right)}\\
%%     &=\frac{1}{x\left( 3+\sqrt{x} \right)}\\
%%     &=\frac{1}{3x+{{x}^{3/2}}}
%%   \end{align*}
%%   The domain is $(0,\infty )$
%% \end{enumerate}
%% \end{example}

%% We can generalize to say:
%% \begin{definition}
%% If $f$ is a function with domain $A$ and $g$ is a function with domain $B$, then we can define operations on these functions as:
%% \begin{align*}
%% (f+g)(x) &=f(x)+g(x) Domain A \cap B\\
%% (f-g)(x) &=f(x)-g(x)  Domain A \cap B\\
%% (fg)(x) &=f(x)g(x)  Domain A \cap B\\
%% \frac{f}{g}(x) = \frac{f(x)}{g(x)} Domain A \cap B except where g(x)=0
%% \end{align*}
%% \end{definition}


Given two functions, we can compose them. Let's give an example in a
``real context.''

\begin{example}
  Let
  \[
  g(m) = \text{the amount of gas one can buy with $m$ dollars,}
  \]
  and let
  \[
  f(g) = \text{how far one can drive with $g$ gallons of gas.}
  \]
  What does $f(g(m))$ represent in this setting?
  \begin{explanation}
    With $f(g(m))$ we first relate how far one can drive with
    $\answer[given]{g}$ gallons of gas, and this in turn is determined
    by how much money $\answer[given]{m}$ one has. Hence $f(g(m))$ represents how far
    one can drive with $\answer[given]{m}$ dollars.
  \end{explanation}
\end{example}

Composition of functions can be thought of as putting one function
inside another.  We use the notation
\[
(f\circ g)(x) = f(g(x)).
\]
\begin{warning}
  The composition $f\circ g$ only makes sense if
  \[
  \{\text{the range of $g$}\}
  \text{ is contained in or equal to }
  \{\text{the domain of $f$}\}
  \]
\end{warning}

\begin{example}
 Suppose we have
\begin{align*}
  f(x)&={{x}^{2}}+5x+4 &&\text{for $-\infty< x< \infty$,}\\
  g(x)&= x+7 &&\text{for $-\infty< x< \infty$.}\\
\end{align*}
Find $f(g(x))$ and state its domain.
\begin{explanation}
  The range of $g$ is $-\infty< x< \infty$, which is equal to the
  domain of $f$. This means the domain of $f\circ g$ is $-\infty< x<
  \infty$. Next, we substitute $x+7$ for each instance of $\answer[given]{x}$ found
  in
  \[
  f(x)={{x}^{2}}+5x+4
  \]
  and so
  \begin{align*}
  f(g(x)) &=f(x+7)\\
  &=\answer[given]{{{(x+7)}^{2}}+5(x+7)+4}.
  \end{align*}
\end{explanation}
\end{example}

Now let's try an example with a more restricted domain.

\begin{example}
 Suppose we have:
\begin{align*}
  f(x)&=x^2 &&\text{for $-\infty< x< \infty$,}\\
  g(x)&= \sqrt{x} &&\text{for $0\le x< \infty$.}\\
\end{align*}
Find $f(g(x))$ and state its domain.
\begin{explanation}
  The domain of $g$ is $0\le x< \infty$. From this we can see that the
  range of $g$ is $\answer[given]{0}\le x< \infty$. This is contained
  in the domain of $f$.

  This means that the domain of $f\circ g$ is $0\le x< \infty$.  Next,
  we substitute $\answer[given]{\sqrt{x}}$ for each instance of $x$
  found in
  \[
  f(x)={{x}^{2}}
  \]
  and so
  \begin{align*}
  f(g(x))=f(\sqrt{x})\\
  &=\left(\sqrt{x}\right)^2.
  \end{align*}
  We can summarize our results as a piecewise function, which
  looks somewhat interesting:
  \[
  (f\circ g)(x) = 
  \begin{cases}
    x & \text{if $0\le x < \infty$}\\
   \text{undefined} &\text{otherwise}. 
  \end{cases}
  \]
\end{explanation}
\end{example}


\begin{example}
 Suppose we have:
\begin{align*}
  f(x)&=\sqrt{x} &&\text{for $0\le x< \infty$,}\\
  g(x)&= x^2 &&\text{for $-\infty< x< \infty$.}
\end{align*}
Find $f(g(x))$ and state its domain.
\begin{explanation}
  While the domain of $g$ is $-\infty< x< \infty$, its range is only
  $0 \le x<\infty$. This is exactly the domain of $f$. This means that
  the domain of $f\circ g$ is $-\infty< x< \infty$. %%BADBAD Explain more
  Now we may substitute $\answer[given]{x^2}$ for each instance of
  $\answer[given]{x}$ found in
  \[
  f(x)=\sqrt{x}
  \]
  and so
  \begin{align*}
  f(g(x))&=f(x^2)\\
  &=\sqrt{x^2},\\
  &=|x|.
  \end{align*}
\end{explanation}
\end{example}

Compare and contrast the previous two examples.  We used the same
functions for each example, but composed them in different ways.  The resulting
compositions are not only different, they have different domains!



\end{document}
