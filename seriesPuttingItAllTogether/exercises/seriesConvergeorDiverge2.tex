\documentclass{ximera}

\newcommand{\RR}{\mathbb R}
\renewcommand{\d}{\,d}
\newcommand{\dd}[2][]{\frac{d #1}{d #2}}
\renewcommand{\l}{\ell}
\newcommand{\ddx}{\frac{d}{dx}}
\newcommand{\dfn}{\textbf}
\newcommand{\eval}[1]{\bigg[ #1 \bigg]}


\author{Jim Talamo}
\license{Creative Commons 3.0 By-bC}


\outcome{}


\begin{document}
\begin{exercise}

Select all of the following series that converge.

\begin{selectAll}
\choice{$\sum_{k=2}^{\infty} \frac{(\ln(k))^{10}}{k}$}
\choice{$\sum_{n=1}^{\infty} \frac{n!}{e^{n+3}+5n}$}
\choice{$\sum_{k=1}^{\infty} \ln\left(\frac{k+1}{k}\right)$}
\choice[correct]{$\sum_{k=1}^{\infty} 2^{2k-6}20^{-k}$}
\choice[correct]{$\sum_{n=1}^{\infty} \frac{n\cos(\pi n)}{\sqrt{n^3+2}}$}
\end{selectAll}

\begin{hint}
We say that a convergence test is \emph{applicable} if the assumptions of the test are met.  We say that a test is \emph{conclusive} if it can be used to determine whether a series converges or diverges.

%%%%%%%%%%%  FOR FIRST SERIES %%%%%%%%%%%
\begin{question}
For the series $\sum_{k=1}^{\infty} \frac{(\ln(k))^{10}}{k}$, is the series geometric or a $p$-series?

\begin{multipleChoice}
\choice{The series is a geometric series.}
\choice{The series is a $p$-series.}
\choice[correct]{The series is neither a geometric series nor a $p$-series.}
\end{multipleChoice}

 Which of the following tests are \emph{applicable}?

\begin{selectAll}
\choice[correct]{divergence test}
\choice[correct]{comparison test}
\choice[correct]{limit comparison test}
\choice[correct]{ratio test}
\choice[correct]{root test}
\choice{alternating series test}
\end{selectAll}


Our knowledge of growth rates tell us that for very large $k$

\begin{multipleChoice}
\choice[correct]{$k$ dominates any power of $\ln(k)$}
\choice{$\ln(k)$ dominates $k$}
\choice{$k$ dominates $\ln(k)$ but not $(\ln(x))^{10}$}
\end{multipleChoice}

Therefore we can conclude that the series 

\begin{multipleChoice}
\choice{diverges by the divergence test.}
\choice[correct]{could either converge or diverge} 
\end{multipleChoice}

Let's try the comparison test.  Notice that since $(\ln(k))^{10}>1$, we have
$\frac{(\ln(k))^{10}}{k}$ is  \wordChoice{\choice[correct]{greater than}\choice{less than}\choice{neither greater than nor less than}} $\frac{1}{k}$ for all integers $k$.  
Hence we can easily use the Comparison Test to conclude that the series
\[
\sum^{\infty}_{k=1} \frac{ (\ln(k))^{10}}{k}
\]

\begin{multipleChoice}
\choice[correct]{converges}
\choice{diverges}
\end{multipleChoice}


\end{question}

%%%%%%%%%%%  FOR SECOND SERIES %%%%%%%%%%%


\begin{question}
For the series $\sum_{n=1}^{\infty}  \frac{n!}{e^{n+3}+5n}$, is the series geometric or a $p$-series?

\begin{multipleChoice}
\choice{The series is a geometric series.}
\choice{The series is a $p$-series.}
\choice[correct]{The series is neither a geometric series nor a $p$-series.}
\end{multipleChoice}

Let's think about the other most useful convergence tests.  Which of the following tests are \emph{applicable}?

\begin{selectAll}
\choice[correct]{divergence test}
\choice[correct]{comparison test}
\choice[correct]{limit comparison test}
\choice[correct]{ratio test}
\choice[correct]{root test}
\choice{alternating series test}
\end{selectAll}

The dominant term in the denominator is $e^{n+3}$. Our knowledge of growth rates tells us 
that for large $n$,  $n!$ dominates $e^n$. 

This means that 
\[
\lim_{n \to \infty} \frac{n!}{e^{n+3}+5n}=\answer{ \infty}
\]

Thus the series $\sum^{\infty}_{n=1} \frac{n!}{e^{n+3}+5n}$
\wordChoice{
\choice{converges}
\choice[correct]{diverges}}

by the
\wordChoice{\choice[correct]{divergence test}
\choice{comparison test}
\choice{limit comparison test}
\choice{ratio test}
\choice{root test}
\choice{alternating series test}}
.


\end{question}

%%%%%%%%%%%  FOR THIRD SERIES %%%%%%%%%%%
\begin{question}
For the series $\sum_{k=1}^{\infty} \ln\left(\frac{k+1}{k}\right)$, note that:

\begin{multipleChoice}
\choice{The series is a geometric series.}
\choice{The series is a $p$-series.}
\choice[correct]{The series is telescoping.}
\choice{The series is none of the above}
\end{multipleChoice}

Recall that $\ln \left(\frac{a}{b}\right)=\ln(a)-\ln(b)$. Therefore $\ln\left(\frac{k+1}{k}\right)=\ln(k+1)-\ln(k)$. 

Thus we can rewrite our series as 
\[
\sum^{\infty}_{k=1} \ln(k+1)-\ln(k)
\]


Looking at the $N$th partial sum and simplifying we obtain 
\[
\sum^{N}_{k=1} \ln(k+1)-\ln(k)=\answer{\ln(N+1)}
\]

\begin{align*}
\sum^{\infty}_{k=1} \ln(k+1)-\ln(k)&=\lim_{N \to \infty} \sum^{N}_{k=1} \ln(k+1)-\ln(k) \\
&=\lim_{N \to \infty} \ln(N+1)=\answer{\infty}
\end{align*}

Thus our series

\[
\sum^{\infty}_{k=1} \ln\left(\frac{k+1}{k}\right)
\]

\begin{multipleChoice}
\choice{converges}
\choice[correct]{diverges}
\end{multipleChoice}


\end{question}

%%%%%%%%%%%  FOR FOURTH SERIES %%%%%%%%%%%
\begin{question}
For the series $\sum_{k=1}^{\infty} 2^{2k-6}20^{-k}$, is the series geometric or a $p$-series?

\begin{multipleChoice}
\choice[correct]{The series is a geometric series.}
\choice{The series is a $p$-series.}
\choice{The series is neither a geometric series nor a $p$-series.}
\end{multipleChoice}

Here the ratio $r= \answer{\frac{1}{5}}$ which is \wordChoice{\choice greater than \choice equal to \choice[correct] less than} $1$. 

Thus our series
\[
\sum_{k=1}^{\infty} 2^{2k-6}20^{-k}
\]

\begin{multipleChoice}
\choice[correct]{converges}
\choice{diverges}
\end{multipleChoice}

\end{question}

%%%%%%%%%%%  FOR FIFTH SERIES %%%%%%%%%%%
\begin{question}

For the series $\sum_{n=1}^{\infty} \frac{n\cos(\pi n)}{\sqrt{n^3+2}}$, is the series geometric or a $p$-series?

\begin{multipleChoice}
\choice{The series is a geometric series.}
\choice{The series is a $p$-series.}
\choice[correct]{The series is neither a geometric series nor a $p$-series.}
\end{multipleChoice}

Let's think about the other most useful convergence tests.  Which of the following tests are \emph{applicable}?

\begin{selectAll}
\choice[correct]{divergence test}
\choice{comparison test}
\choice{limit comparison test}
\choice{ratio test}
\choice{root test}
\choice[correct]{alternating series test}
\end{selectAll}

Note we can rewrite without using trig to get $\cos(\pi n)=\answer{(-1)^n}$. 

With alternating series, we can check for absolute convergence or try using the alternating series test. 

The positive portion of the summand is $\frac{n}{\sqrt{n^3+2}}$ and thus we expect that $\sum^{\infty}_{n=1}\frac{n}{\sqrt{n^3+2}}$ diverges. We would compare this series to the series $\sum^{\infty}_{n=1}\frac{1}{n^{\frac{1}{2}}}$ which is a $p$-series with $p=\frac{1}{2}$ and we would be able to show this series diverges.  Thus it would not be a good idea to check our original series for absolute convergence.

We use the alternating series test and thus conclude that

\[
\sum_{k=1}^{\infty} \frac{n \cos(\pi n)}{\sqrt{n^3+2}}
\]

\begin{multipleChoice}
\choice[correct]{converges}
\choice{diverges}
\end{multipleChoice}

\end{question}
\end{hint}







\begin{quote}
You should be able to give a detailed solution for each of your choices.  Such a solution should include:

\begin{itemize}
\item What test you chose.
\item Why you are allowed to use that test.
\item The computations required for the conclusions of the test.
\item An explanation why the series either converges or diverges by the test you chose.
\end{itemize}

\end{quote}


\end{exercise}
\end{document}
