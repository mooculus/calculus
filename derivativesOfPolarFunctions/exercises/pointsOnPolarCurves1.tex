\documentclass{ximera}

\newcommand{\RR}{\mathbb R}
\renewcommand{\d}{\,d}
\newcommand{\dd}[2][]{\frac{d #1}{d #2}}
\renewcommand{\l}{\ell}
\newcommand{\ddx}{\frac{d}{dx}}
\newcommand{\dfn}{\textbf}
\newcommand{\eval}[1]{\bigg[ #1 \bigg]}


\author{Jason Miller}
\license{Creative Commons 3.0 By-bC}


\outcome{}

\begin{document}
\begin{exercise}



Consider the polar curve $r=1+2\cos(\theta)$. The graph is shown below: 




\begin{image}  
  \begin{tikzpicture}  
    \begin{axis}[  
        xmin=-2.5,  
        xmax=3.5,  
        ymin=-2.5,  
        ymax=2.5,  
        axis lines=center,  
        xlabel=$x$,  
        ylabel=$y$,  
        every axis y label/.style={at=(current axis.above origin),anchor=south},  
        every axis x label/.style={at=(current axis.right of origin),anchor=west},  
      ]  
      \addplot[data cs=polar,penColor,domain=0:360,samples=360,smooth, thick] (x,{1+2*cos(x)});
       \addplot[only marks, mark=*] coordinates {(.634, -.366)};
       \draw (axis cs: .8,-.6) node { $P$};
            \end{axis}  
  \end{tikzpicture}  
\end{image} 

Choose the value of $\theta$ that corresponds to the point $P$. The curve can be traced out once by letting $\theta$ vary over the interval $[0, 2\pi)$.

\begin{multipleChoice}
\choice{$\frac{3\pi}{2}$}
\choice{$\frac{7\pi}{6}$}
\choice{$\frac{\pi}{3}$}
\choice[correct]{$\frac{5\pi}{6}$}
\choice{$\pi$}
\choice{$\frac{2\pi}{3 } $}
\choice{$\frac{6\pi }{5  }$}
\end{multipleChoice}


\begin{hint}
Remember that when $(r, \theta)$ and $(-r, \theta + \pi)$ refer to the same point so we can always interpret a point with negative radius as one with 
positive radius where we add $\pi$ to the angle. 
\end{hint}



\begin{exercise}

Consider the curve $r=2\cos(3\theta)$.


\begin{image}  
  \begin{tikzpicture}  
    \begin{axis}[  
        xmin=-2.5,  
        xmax=3.5,  
        ymin=-2.5,  
        ymax=2.5,  
        axis lines=center,  
        xlabel=$x$,  
        ylabel=$y$,  
        every axis y label/.style={at=(current axis.above origin),anchor=south},  
        every axis x label/.style={at=(current axis.right of origin),anchor=west},  
      ]  
      \addplot[data cs=polar,penColor,domain=0:360,samples=360,smooth, thick] (x,{2*cos(3*x)});
       \addplot[only marks, mark=*] coordinates {(-1, -1)};
       \draw (axis cs: -1.2,-.85) node { $D$};
        \addplot[only marks, mark=*] coordinates {(1.366, .366)};
        \draw (axis cs: 1.5,.6) node { $A$};
        \addplot[only marks, mark=*] coordinates {(-.366, 1.366)};
          \draw (axis cs: -.3,1.6) node { $B$};
          \addplot[only marks, mark=*] coordinates {(1.366, -.366)};
          \draw (axis cs: 1.36,-.6) node { $F$};
           \addplot[only marks, mark=*] coordinates {(-1, 1)};
            \draw (axis cs: -1.3,1) node { $C$};
            \addplot[only marks, mark=*] coordinates {(-.366, -1.366)};
           \draw (axis cs: -.23,-1.6) node { $E$};
            \end{axis}  
  \end{tikzpicture}  
\end{image} 

  Determine which of the labeled points corresponds to $\theta=\frac{\pi}{4}$.

\begin{multipleChoice}
\choice{$A$}
\choice{$B$}
\choice{$C$}
\choice[correct]{$D$}
\choice{$E$}
\choice{$F$}
\end{multipleChoice}

\begin{hint}
Remember that when $(r, \theta)$ and $(-r, \theta + \pi)$ refer to the same point so we can always interpret a point with negative radius as one with 
positive radius where we add $\pi$ to the angle. 
\end{hint}


\end{exercise}
\end{exercise}
\end{document}
