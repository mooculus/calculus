\documentclass{ximera}

\newcommand{\RR}{\mathbb R}
\renewcommand{\d}{\,d}
\newcommand{\dd}[2][]{\frac{d #1}{d #2}}
\renewcommand{\l}{\ell}
\newcommand{\ddx}{\frac{d}{dx}}
\newcommand{\dfn}{\textbf}
\newcommand{\eval}[1]{\bigg[ #1 \bigg]}


\outcome{Estimate partial derivatives from tables and graphs.}

\author{Jim Talamo}
% The contour plots were produced with the following Maple Commands

%with(plots);
%p := contourplot(4/((x-1)^2+(y-2)^2)-4/((x+3)^2+(y-1)^2)+6/(x^2+(y+2)^2), x = -4.5 .. 4.5, y = -4.5 .. 4.5, contours = [-3, -1, -1/2, 0, 1/2, 1, 1.5, 5], grid = [100, 100]);

%I will look into exporting the data from the plot to draw this in TikZ later

%Plan: I will ask this same question AFTER they learn about the properties of the gradient

\begin{document}
\begin{exercise}

This exercise gives an example of an important type of contour plot that arises in electrostatics.

The magnitude of force that a particle of charge $Q$ exerts on a particle of charge $q$ that is $r$ units away is given by \emph{Coulomb's Law}

\[
F(r) = \frac{kQq}{r^2},
\] 

where $k =8.99 \times 10^9\unit{N}\cdot\unit{m}^2\cdot\unit{C}^2$ is Coulomb's constant.

When several charges are placed near each other, the force that a test particle of charge $q$ experiences is determined by adding the forces that each charged particle exerts on it.  The \emph{electric field} is a ``vector field''; to each point $(x,y)$ in the $xy$-plane, the electric field associates the force vector experienced by the test charge $q$.

An interesting fact is that this electric field can actually be realized as the gradient of a certain function $V(x,y)$, called the \emph{electric potential} via the formula

\[
\vec{E}(x,y)=-\grad{V}(x,y),
\]
and a useful way to visualize electric fields to plot the level curves of the electric potential.  These level curves are often referred to as \emph{equipotential lines}.

The contour plot for the electric potential $V(x,y)$ are below shows equipotential lines for an electric field that results from a charge configuration with positive charges at $(1,2)$ and $(0,-2)$ and a negative charge at $(-3,1)$.

\begin{image}
\includegraphics[width=5in]{contours3.png}
\end{image}
Assume $V(x,y)$ either increases or decreases between contour lines and $V_x(x,y)$ and $V_y(x,y)$ are defined at all points in the picture.

Select all of the points where $V_x(x,y)=0$.

\begin{selectAll}
\choice[correct]{A}
\choice{B}
\choice{C}
\choice{D}
\choice{E}
\choice{F}
\end{selectAll}

\begin{hint}
Note that the contour plot shows the level curves of the function $V(x,y)$ in the $xy$-plane, and gives the value that the function takes along them.  For instance, the points $C$, $E$, and $F$ lie along the level curve associated to $V=3$.

An important observation is the following.

\begin{quote}
Since the values of a function do not change along a level curve, the tangent line to a level curve gives the direction in which the instantaneous rate of change of the function is $0$.
\end{quote}

$V_x(a,b)$ is the instantaneous rate of change of the function at $(a,b)$ in the $x$-direction.  Thus, to find the points at which $V_x=0$, we need to look for the locations where the tangent line to the level curve is in the $x$-direction.  This occurs at point $A$ only.

\end{hint}

%%%%%%%%%%%%%%%%%%%%%%%%%%%%%%%%%%%%%%%%%%%%%%%%%%%%%%
\begin{exercise}
Select all of the points where $V_y(x,y)=0$.

\begin{selectAll}
\choice{A}
\choice{B}
\choice[correct]{C}
\choice{D}
\choice[correct]{E}
\choice{F}
\end{selectAll}

\end{exercise}  
%%%%%%%%%%%%%%%%%%%%%%%%%%%%%%%%%%%%%%%%%%%%%%%%%%%%%%
\begin{exercise}
Select all of the points where $V_x(x,y)>0$.

\begin{selectAll}
\choice{A}
\choice[correct]{B}
\choice[correct]{C}
\choice{D}
\choice{E}
\choice{F}
\end{selectAll}

\begin{hint}
To find where $V_x>0$, we need to look for when the function increases in the $x$-direction.  Since the function either must increase or must decrease between contours, note

\begin{itemize}
\item We've already determined that $V_x =0$ at point $A$.
\item 
As $x$ increases from point $B$, we move from the contour curve along which $V=-1$ towards the one where $V=0$.  Thus, $V_x>0$ at $B$.
\item As $x$ increases from point $C$, we move from the contour curve along which $V=\answer{3}$ towards the one where $V=\answer{5}$.  Thus, \wordChoice{\choice{$V_x<0$}\choice{$V_x=0$}\choice[correct]{$V_x>0$}} at  $C$.
\item As $x$ increases from point $D$, we move from the contour curve along which $V=\answer{5}$ towards the one where $V=\answer{3}$.  Thus, \wordChoice{\choice[correct]{$V_x<0$}\choice{$V_x=0$}\choice{$V_x>0$}} at  $C$.
\end{itemize}

\end{hint}
\end{exercise}  
%%%%%%%%%%%%%%%%%%%%%%%%%%%%%%%%%%%%%%%%%%%%%%%%%%%%%%
\begin{exercise}
Select all of the points where $V_y(x,y)<0$.

\begin{selectAll}
\choice{A}
\choice{B}
\choice{C}
\choice{D}
\choice{E}
\choice[correct]{F}
\end{selectAll}

\end{exercise}  

\end{exercise}
\end{document}
