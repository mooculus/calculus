\documentclass{ximera}

\newcommand{\RR}{\mathbb R}
\renewcommand{\d}{\,d}
\newcommand{\dd}[2][]{\frac{d #1}{d #2}}
\renewcommand{\l}{\ell}
\newcommand{\ddx}{\frac{d}{dx}}
\newcommand{\dfn}{\textbf}
\newcommand{\eval}[1]{\bigg[ #1 \bigg]}


%\outcome{Understand the statement of the Extreme Value Theorem.}
\outcome{Understand the statement of the Mean Value Theorem.}
%\outcome{Sketch pictures to illustrate why the Mean Value Theorem is true.}
\outcome{Determine whether Rolle's Theorem or the Mean Value Theorem can be applied.}
\outcome{Find the values guaranteed by Rolle's Theorem or the Mean Value Theorem.}
%\outcome{Use the Mean Value Theorem to solve word problems.}
%\outcome{Compare and contrast the Intermediate Value Theorem, the Mean Value Theorem, and Rolle's Theorem.}
%\outcome{Identify calculus ideas which are consequences of the Mean Value Theorem.}
%\outcome{Use the Mean Value Theorem to bound the error in linear approximation.}

\author{Nela Lakos \and Kyle Parsons}

\begin{document}
\begin{exercise}

Consider the function $f(x) = \sqrt{x}$ on the interval $[0,16]$.

The equation for the linear approximation of $f$ at $a=4$ is
\[
L(x) = \answer{\frac{1}{4}(x-4)+2}.
\]

Using the linear approximation we can estimate $\sqrt{5}$ as 
\[
\sqrt{5} \approx \answer{2.25}.
\]

The Mean Value Theorem \wordChoice{\choice[correct]{does}\choice{does not}} apply to $f$ on $[0,16]$ and the point $c$ guaranteed to exist is $c=\answer{4}$. (Write \verb|DNE| if the Mean Value Theorem does not apply.)

\end{exercise}
\end{document}