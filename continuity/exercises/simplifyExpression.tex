\documentclass{ximera}

\newcommand{\RR}{\mathbb R}
\renewcommand{\d}{\,d}
\newcommand{\dd}[2][]{\frac{d #1}{d #2}}
\renewcommand{\l}{\ell}
\newcommand{\ddx}{\frac{d}{dx}}
\newcommand{\dfn}{\textbf}
\newcommand{\eval}[1]{\bigg[ #1 \bigg]}


\outcome{Simplify expression}

\author{Nela Lakos}
\begin{document}
\begin{exercise}

Simplify the  expression $$ \frac{\frac{4}{x-2}+1}{\frac{x}{x+3}-\frac{4}{x-2}}.$$ Show your work as indicated below.

First, we multiply the numerator and the denominator by  the lowest common denominator

	 \[
	 \frac{\frac{4}{x-2}+1}{\frac{x}{x+3}-\frac{4}{x-2}}= \frac{\frac{4}{x-2}+1}{\frac{x}{x+3}-\frac{4}{x-2}}\cdot \frac{(x+3)(\answer{x-2})}{(x+3)(\answer{x-2})},
	 \]
	and obtain that  
\[
\frac{\frac{4}{x-2}+1}{\frac{x}{x+3}-\frac{4}{x-2}}= \frac{4(\answer{x+3})+1(x+3)(\answer{x-2})}{x(\answer{x-2})-4(\answer{x+3})}.
\]
Then we simplify the numerator and the denominator.
\[
\frac{\frac{4}{x-2}+1}{\frac{x}{x+3}-\frac{4}{x-2}}= \frac{\answer{1}x^2+\answer{5}x+\answer{6}}{\answer{1}x^2+\answer{-6}x+\answer{-12}}.
\]
\end{exercise}
\end{document}
