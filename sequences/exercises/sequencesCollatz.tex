\documentclass{ximera}

\newcommand{\RR}{\mathbb R}
\renewcommand{\d}{\,d}
\newcommand{\dd}[2][]{\frac{d #1}{d #2}}
\renewcommand{\l}{\ell}
\newcommand{\ddx}{\frac{d}{dx}}
\newcommand{\dfn}{\textbf}
\newcommand{\eval}[1]{\bigg[ #1 \bigg]}


\author{Jim Talamo and Bart Snapp}
\license{Creative Commons 3.0 By-bC}


\outcome{}


\begin{document}
\begin{exercise}

To whet the appetite of the curious young mathematician further, we discuss an open problem in mathematics.

A particularly interesting example of a recursive sequence with which to amuse your friends---or distract
your enemies is as follows:

Let's start our sequence with $a_1 = 6$.  Subsequent terms are
  defined using the rule
  \[
  a_n =
  \begin{cases}
    a_{n-1} / 2 &\text{if $a_{n-1}$ is even,} \\
    3a_{n-1} + 1 &\text{if $a_{n-1}$ is odd.}
  \end{cases}
  \]
  Let's compute $a_2$.  Since $a_1$ is even, we follow the
  instructions in the first line, to find that:
  
  \[a_2 = a_1/2 = \answer[given]{3}\]
  
  \begin{exercise}
  To compute $a_3$, note that $a_2$ is odd so we
  follow the instruction in the second line:
  
  \[a_3 = 3 a_2 + 1 = 3 \cdot 3 + 1 = \answer[given]{10} \]
  
  \begin{exercise}
  Since $a_3$ is even, the first line applies, and:
  
  \[ a_4 = a_3 / 2 = 10 / 2 = \answer[given]{5} \]
  
  \begin{exercise}
Continuing in this way, we find:

\begin{align*}
a_5 & = \answer{16} & a_6 & = \answer{8} & a_7 & = \answer{4} & a_8 & = \answer{2} & a_9 & = \answer{1} & a_{10} & = \answer{4} & a_{11} & = \answer{2} & a_{12} & = \answer{1}
\end{align*}

\begin{exercise}
And it starts repeating!  Let's write down the start of this sequence:
  \[
  6,\, %1 
  3,\, %2
  10,\,  %3
  5,\,  %4
  16,\,  %5
  8,\,  %6
  4,\,  %7
  2,\,  %8
  1,\,  %9
  4,\, %10
  2,\, %11
  1,\, %12
  \overbrace{4,\, %10
    2,\, %11
    1,}^{\text{repeats}}\, %12
  4,\, %10
  \ldots
  \]
  
  \begin{exercise}
  By utilizing the above pattern, what is $a_{425}$?  
  
  \[
  a_{425} = \answer{2}
  \]
    
  \end{exercise}
  What if we had started with a number other than six?  What if we set
  $a_1 = 25$ but then we used the same rule?  In that case, since
  $a_1$ is odd, we compute $a_2$ by finding $3 a_1 + 1 = 3 \cdot 25 +
  1 = 76$.  Since $76$ is even, the next term is half that, meaning
  $a_3 = 38$.  If we keep this up, we find that our sequence begins
  \begin{align*}
    &25,\, 76,\, 38,\, 19,\, 58,\, 29,\, 88,\, 44,\, 22,\, 11,\, 34,\, 17,\, 52,\, 26, \\
    &13,\, 40,\, 20,\, 10,\, 5,\, 16,\, 8,\, 4,\, 2, \, 1, \, \ldots
  \end{align*}
  and then it repeats ``4, 2, 1, 4, 2, 1, \ldots'' just like before.
  Does this always happen?  Is it true that no matter which positive
  integer you start with, if you apply the half-if-even, $3x+1$-if-odd
  rule, you end up getting stuck in the ``4, 2, 1, \ldots'' loop?
  That this is true is the \dfn{Collatz conjecture}; it has been
  verified for all starting values below $5 \times 2^{60}$.  Nobody
  has found a value which doesn't return to one, but for all anybody
  knows there \textit{might} well be a very large initial value which
  doesn't return to one; nobody knows either way.  It is an unsolved
  problem in mathematics!

\end{exercise}
  \end{exercise}
  \end{exercise}
  \end{exercise}
  \end{exercise}
\end{document}
