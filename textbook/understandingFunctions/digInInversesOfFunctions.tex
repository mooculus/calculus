\documentclass{ximera}

\newcommand{\RR}{\mathbb R}
\renewcommand{\d}{\,d}
\newcommand{\dd}[2][]{\frac{d #1}{d #2}}
\renewcommand{\l}{\ell}
\newcommand{\ddx}{\frac{d}{dx}}
\newcommand{\dfn}{\textbf}
\newcommand{\eval}[1]{\bigg[ #1 \bigg]}


\title{Inverses of functions}

\begin{document}
\begin{abstract}
  Here we ``undo'' functions.
\end{abstract}
\maketitle

If a function maps every input to exactly one output, an inverse of
that function maps every ``output'' to exactly one ``input.''
Symbolically, if $f$ is the function, the inverse function is written
(somewhat confusingly) as $f^{-1}$.

\begin{warning}
  Keep a watchful eye:
  \begin{align*}
    f^{-1}(x) &= \text{the inverse function of $f(x)$.}\\
    f(x)^{-1} &= \text{$\frac{1}{f(x)}$.}
  \end{align*}
\end{warning}
\begin{question}
  Which of the following represents the inverse of $\sin(\theta)$ on the
  interval $[-\pi/2,\pi/2]$
  \begin{multipleChoice}
    \choice[correct]{$\sin^{-1}(\theta)$}
    \choice{$\sin(\theta)^{-1}$}
  \end{multipleChoice}
\end{question}
Inverse function are defined by the following characteristics
\[
f(f^{-1}(x)) = x\qquad\text{and}\qquad f^{-1}(f(x)) = x.
\]
These two simple equations are somewhat more subtle than they
initially appear.  While this might sound somewhat esoteric, let's see
if we can ground this in some real-life contexts.

\begin{example}
Suppose that you are filling a swimming pool using a garden
hose---though because it rained last night, the pool starts with some
water in it. The volume of water in gallons after $t$ hours of filling
the pool is modeled by:
\[
v(t) = 700t + 200 
\]
for $0\le t \le V$ where $V$ is the maximum volume of the pool. What
does the inverse of this function tell you? What is the inverse of
this function?
\begin{explanation}
While $v(t)$ tells you how many gallons of water are in the pool after
a period of time $t$, the inverse of $v$ tells you how much time must
be spent to obtain a given volume. To compute the inverse function,
first note that
\[
v(v^{-1}(t)) = t.
\]
Now write out the left-hand side of the equation
\[
v(v^{-1}(t)) = 700\cdot v^{-1}(t) + 200 
\]
and solve for $v^{-1}(t)$
\begin{align*}
  700\cdot v^{-1}(t) + 200 &= t,\\
  700\cdot v^{-1}(t)  &= t-200\\
  v^{-1}(t) &= \frac{t-200}{700}.
\end{align*}
This is a function that maps volumes to times. 
\end{explanation}
\end{example}

Let's look at how tables of values for functions and their inverses
compare.  For this example, we have:
%% insert picture 7
These tables reinforce the fact that our inputs become our outputs and vice versa.  We see that when the input for the function is $0$ the output is $200$ and when the input for the inverse is $200$, the output is $0$.  This isn't a coincidence.  This leads us to the face that the plot of the inverse is the plot of the function reflected over the line $y=x$ or $f(x)=x$.  See figure.
%% Insert picture 8

Now let's consider a different example.

\begin{example}\label{E:example-ball-bridge}
Suppose you are standing on a bridge that is 60 meters above
sea-level. You toss a ball up into the air with an initial velocity of
30 meters per second.  If $t$ is the time (in seconds) after we toss
the ball, then the height at time $t$ is approximately $h(t) = -5 t^2
+30t+60$. What does the inverse of this function tell you? What is the inverse
of this function?
\begin{explanation}
While $h(t)$ tells you how the height the ball is above sea-level at an
instant of time, the inverse of $h(t)$ tells you what time it is when
the ball is at a given height. There is only one problem: There is no
function that is the inverse of $h(t)$. Consider Figure~\ref{plot:fxn
  ball}, we can see that for some heights---namely 60 meters, there
are two times. This plot fails the horizontal line test to determine if the function is one-to-one.  Since the plot is not one-to-one, we cannot find an inverse function for $h(t)$.

While there is no inverse function for $h(t)$, we can find one if we
\textit{restrict the domain} of $h(t)$.  Looking at the plot of $h(t)$, we see that it can pass the horizontal line test if we restrict the domain to half of the parabola.  The point at which we need to split the parabola is at the vertex. See figure.
%% Insert original figure 7 but with bolded point at vertex

To find the vertex, we need to write the equation in standard parabola form by completing the square.

\begin{align*}
h &= -5t^2+30t+60\\
h &=-5\left((t^2-6t+9)-12-9\right)\\
h &=-5\left((t-3)^2-21\right)\\
h &=-5(t-3)^2+105
\end{align*}

Therefore the vertex is $(3,105)$

Now we can find the inverse on the interval $[3,\infty)$.

Note: We have added and subtracted $9$ at the same time because that's
equivalent to adding $0$.  The reason we have chosen 9 is by the
completing the square process.  If we have a quadratic in the form
$t^2-bt+c$, we need to find $(\frac {b}{2})^2$.  In this case,
$(\frac{-6}{2})^2 =9$.

From the standard form of the parabola: $h = -5(t-3)^2+105$ we can find the inverse.
\begin{align*}
h-105 &= -5(t-3)^2\\
\frac {h-105}{-5} &= (t-3)^2\\
\pm \sqrt{\frac {h-105}{-5}} &=t-3\\
3 \pm \sqrt{\frac {h-105}{-5}} &=t\\
3 \pm \sqrt{21-.2h} &=t 
\end{align*}

An alternate method to finding the inverse would be: 
Write
\begin{align*}
h &=  -5 t^2 +30t+60\\
0 &= -5 t^2 +30t+(60 - h)
\end{align*}
and solve for $t$ using the quadratic formula
\begin{align*}
t &= \frac{-30\pm \sqrt{30^2 -4(-5)(60-h)}}{2(-5)}\\
&= \frac{-30\pm \sqrt{30^2 +20(60-h)}}{-10}\\
&=3\mp \sqrt{3^2+ .2(60-h)}\\
&=3\mp \sqrt{9+ .2(60-h)}\\
&=3\mp \sqrt{21-.2h}
\end{align*}

But now we have two equations:
$t=3+ \sqrt{21-.2h}$
$t=3- \sqrt{21-.2h}$

We need to determine which equation should be used when $t>3$.
 Since $h(t)$ has its maximum value
  of $105$ when $t=3$, the largest $h$ could be is $105$. This means
  that $21-.2h \ge 0$ and so $\sqrt{21-.2h}$ is a real number. We know
  something else too, $t>3$. This means that the ``$\mp$'' that we see
  above must be a ``$+$.''  So the inverse of $h(t)$ on the interval
  $[3,\infty)$ is $t(h) = 3+ \sqrt{21-.2h}$. A similar argument will
    show that the inverse of $h(t)$ on the interval $(-\infty, 3]$ is
  $t(h) = 3- \sqrt{21-.2h}$.
\end{explanation}
\end{example}


%%Insert picture 9

\begin{image}
\begin{tikzpicture}
	\begin{axis}[
            clip=false, domain=0:7.58, axis lines =middle, xlabel=$t$,
            ylabel=$h$, every axis y label/.style={at=(current
              axis.above origin),anchor=south}, every axis x
            label/.style={at=(current axis.right of
              origin),anchor=west}, ] \addplot [very thick, penColor,
            smooth] {-5*x^2 +30*x+60};
        \end{axis}
\end{tikzpicture}
%% \caption{A plot of $h(t)=-5t^2+30t+60$. Here we can see that for each
%%   input (a value on the $t$-axis), there is exactly one output (a
%%   value on the $h$-axis). However, for each value on the $h$ axis,
%%   sometimes there are two values on the $t$-axis. Hence there is no
%%   function that is the inverse of $h(t)$.}
%% \label{plot:fxn ball}
\end{image}


\begin{image}
\begin{tikzpicture}
	\begin{axis}[
            clip=false, domain=0:7.58, axis lines =middle, xlabel=$t$,
            ylabel=$h$, every axis y label/.style={at=(current
              axis.above origin),anchor=south}, every axis x
            label/.style={at=(current axis.right of
              origin),anchor=west}, ] 
          \addplot [very thick, penColor, smooth] {-5*x^2 +30*x+60};
          \addplot [very thick, penColor2] {80};
          \addplot [very thick, penColor4] plot coordinates {(5,0) (5,110)};
        \end{axis}
\end{tikzpicture}
%% \caption{A plot of $h(t)=-5t^2+30t+60$. While this plot passes the
%%   vertical line test, and hence represents $h$ as a function of $t$,
%%   it does not pass the horizontal line test, so the function is not one-to-one.}
%% \label{plot:fxn vert/horiz}
\end{image}



We see two different cases with our examples above. To clearly
describe the difference we need a definition.

\begin{definition}\index{one-to-one} 
A function is \dfn{one-to-one} if for every value in the range,
there is exactly one value in the domain.
\end{definition}

You may recall that a plot gives $y$ as a function of $x$ if every
vertical line crosses the plot at most once, this is commonly known as
the vertical line test. A function is one-to-one if every horizontal
line crosses the plot at most once, which is commonly known as the
horizontal line test, see Figure~\ref{plot:fxn vert/horiz}.  We can
only find an inverse to a function when it is one-to-one, otherwise we
must restrict the domain as we did in
Example~\ref{E:example-ball-bridge}.


Let's look at several examples.



\begin{example}
Consider the function
\[
f(x) = x^3.
\]
Does $f(x)$ have an inverse? If so what is it? If not, attempt to
restrict the domain of $f(x)$ and find an inverse on the restricted
domain.
\end{example}


\begin{explanation}
In this case $f(x)$ is one-to-one and $f^{-1}(x) = \sqrt[3]{x}$. See Figure~\ref{plot:fxn and inverse x^3}.
\end{explanation}


\begin{image}
\begin{tikzpicture}
	\begin{axis}[
            domain=-2:2,
            xmin=-2, xmax=2,
            ymin=-2, ymax=2,
            axis lines =middle, xlabel=$x$, ylabel=$y$,
            every axis y label/.style={at=(current axis.above origin),anchor=south},
            every axis x label/.style={at=(current axis.right of origin),anchor=west},
          ]
	  \addplot [very thick, penColor, smooth] {x^3};
          \addplot [very thick, penColor2, smooth, samples=100,domain=.01:2] {x^(1/3)};
          \addplot [very thick, penColor2, smooth, samples=100,domain=-2:-.01] {-abs(x)^(1/3)};
          \addplot [very thick, penColor2] plot coordinates {(.01,.215) (-.01,-.215)};
          \addplot [dashed, textColor] {x};
          \node at (axis cs:-1.2,-.42) [penColor,anchor=west] {$f(x)$};
          \node at (axis cs:1.2,.9) [penColor2, anchor=west] {$f^{-1}(x)$};
        \end{axis}
\end{tikzpicture}
%% \caption{A plot of $f(x)=x^3$ and $f^{-1}(x) = \sqrt[3]{x}$. Note
%%   $f^{-1}(x)$ is the image of $f(x)$ after being flipped over the line
%%   $y=x$.}
%% \label{plot:fxn and inverse x^3}
\end{image}


\begin{example}
Consider the function
\[
f(x) = x^2.
\]
Does $f(x)$ have an inverse? If so what is it? If not, attempt to
restrict the domain of $f(x)$ and find an inverse on the restricted
domain.
\end{example}


\begin{explanation}
In this case $f(x)$ is not one-to-one. However, it is one-to-one on
the interval $[0,\infty)$. Hence we can find an inverse of $f(x)=x^2$
  on this interval, and it is our familiar function $\sqrt{x}$.  See
  Figure~\ref{plot:fxn and inverse x^2}.
\end{explanation}

\begin{image}
\begin{tikzpicture}
	\begin{axis}[
            domain=-2:2,
            xmin=-2, xmax=2,
            ymin=-2, ymax=2,
            axis lines =middle, xlabel=$x$, ylabel=$y$,
            every axis y label/.style={at=(current axis.above origin),anchor=south},
            every axis x label/.style={at=(current axis.right of origin),anchor=west},
          ]
	  \addplot [very thick, penColor, smooth] {x^2};
          \addplot [very thick, penColor2, smooth, samples=100,domain=0:2] {sqrt(x)};
          \addplot [dashed, textColor] {x};
          \node at (axis cs:-1.2,.55) [penColor,anchor=west] {$f(x)$};
          \node at (axis cs:1.4,1) [penColor2, anchor=west] {$f^{-1}(x)$};
        \end{axis}
\end{tikzpicture}
%% \caption{A plot of $f(x)=x^2$ and $f^{-1}(x) = \sqrt{x}$. While
%%   $f(x)=x^2$ is not one-to-one on $\RR$, it is one-to-one on
%%   $[0,\infty)$.}
%% \label{plot:fxn and inverse x^2}
\end{image}
\end{document}
