\documentclass[]{ximera}
%handout:  for handout version with no solutions or instructor notes
%handout,instructornotes:  for instructor version with just problems and notes, no solutions
%noinstructornotes:  shows only problem and solutions

%% handout
%% space
%% newpage
%% numbers
%% nooutcomes

%I added the commands here so that I would't have to keep looking them up
%\newcommand{\RR}{\mathbb R}
%\renewcommand{\d}{\,d}
%\newcommand{\dd}[2][]{\frac{d #1}{d #2}}
%\renewcommand{\l}{\ell}
%\newcommand{\ddx}{\frac{d}{dx}}
%\everymath{\displaystyle}
%\newcommand{\dfn}{\textbf}
%\newcommand{\eval}[1]{\bigg[ #1 \bigg]}

%\begin{image}
%\includegraphics[trim= 170 420 250 180]{Figure1.pdf}
%\end{image}

%add a ``.'' below when used in a specific directory.

%\newcommand{\RR}{\mathbb R}
\renewcommand{\d}{\,d}
\newcommand{\dd}[2][]{\frac{d #1}{d #2}}
\renewcommand{\l}{\ell}
\newcommand{\ddx}{\frac{d}{dx}}
\newcommand{\dfn}{\textbf}
\newcommand{\eval}[1]{\bigg[ #1 \bigg]}

\newcommand{\RR}{\mathbb R}
\renewcommand{\d}{\,d}
\newcommand{\dd}[2][]{\frac{d #1}{d #2}}
\renewcommand{\l}{\ell}
\newcommand{\ddx}{\frac{d}{dx}}
\newcommand{\dfn}{\textbf}
\newcommand{\eval}[1]{\bigg[ #1 \bigg]}




\author{Tom Needham}

\outcome{Determine a parametric equation describing a curve.}
\outcome{Sketch a parametrically-defined curve.}
\outcome{Determine the equation of the tangent line to a parametrically-defined curve.}

\title[Collaborate:]{Parametric Equations}

\begin{document}
\begin{abstract}
\end{abstract}
\maketitle

\section{Discussion Questions}

\begin{problem}
Consider the parametric curve defined by 
\begin{align*}
x(t) &= e^{2t} + 3 \\
y(t) &= 1-e^t,
\end{align*}
where $t$ runs over all real numbers. A student claims that the plot of this parametric curve is equal to the graph of the parabola $x = (1-y)^2 + 3$. Is the student correct?
\begin{solution}
The student cannot be correct. Notice that the $y$-coordinate in the parametric function satisfies 
$$
y(t) = 1- e^t < 1
$$
for all $t$. On the other hand, the graph of the given equation in Cartesian coordinates contains points with $y$-values larger than $1$; e.g., the point $(4,2)$ lies on the graph. The plot of the parametric curve is only a subset of the graph of the Cartesian equation.
\end{solution}
\end{problem}

\begin{problem}
Sketch a plot of the parametric function
\begin{align*}
x(t) &= 2 \sin (t) +2\\
y(t) &= 2 \cos (t),
\end{align*}
where $0 \leq t \leq 2\pi$. Indicate the orientation of the parametric curve on your graph.

\begin{solution}
The plot of the function is a circle of radius $2$, centered at $(2,0)$, oriented clockwise.
\end{solution}
\end{problem}


\section{Group Work}

\begin{problem}
For each example, find an equation in the variables $x$ and $y$ whose graph is the same as the graph of the parametrically-defined curve (this procedure is frequently referred to as ``eliminating the parameter").

\begin{itemize}
\item[I.] $x(t) = 2t+3$, $y(t)=2t^2-4$, $-\infty < t < \infty$  
\item[II.] $x(t) = 3 \cos (2t)$, $y(t) = 2 \sin(2t)$, $0 \leq t \leq \frac{\pi}{2}$
\end{itemize}
\begin{solution}
I. We have 
$$
t = \frac{1}{2}(x-3),
$$
so 
$$
y = 2 \cdot \left(\frac{1}{2}(x-3)\right)^2 - 4 = \frac{1}{2}(x-3)^2 - 4.
$$

II. We have
$$
\frac{x}{3} = \cos(2t)
$$
and 
$$
\frac{y}{2} = \sin(2t).
$$
By the Pythagorean identity, it follows that
$$
(x/3)^2 + (y/2)^2 = 1.
$$
\end{solution}
\end{problem}

\begin{problem}
\begin{itemize}
\item[I.] Consider the parametric function given by $x(t)=t^3$, $y(t)=t^2$, $-\infty <  t < \infty$. Find the equation of the tangent line to the plot of the parametric curve at the point corresponding to $t=1$ by ``eliminating the parameter". 

\item[II.] Consider the parametric function given by $x(t)=t^3+t$, $y(t)=t^4-3t^2$, $-\infty <  t < \infty$. A student is asked to find the equation of the tangent line to the plot of the parametric curve at the point corresponding to $t=1$. The student attempts to do so by elimating the parameter, but is unable find a Cartesian representation of the curve. They conclude that it is impossible to find the equation of the tangent line. Is the student correct? If not, find the equation of the tangent line.

\item[III.] Using any method you like, find the equation of the tangent line to the plot of the parametric function $x(t)=t^2$, $y(t)=1+2t$, $-2 \leq t \leq 2$ at the point $(x,y)=(1,-1)$. 
\end{itemize}

\begin{solution}
I. Since $t = x^{1/3}$, we obtain $y=x^{2/3}$ by eliminating the parameter. It follows that
$$
\frac{\d y}{\d x} = \frac{2}{3} x^{-1/3}.
$$
At parameter value $t=1$, we have $x = y = 1$. The slope of the tangent line at this point is $\frac{2}{3}$ and the equation of the tangent line is 
$$
y - 1 = \frac{2}{3}(x-1).
$$

II. The method of eliminating the parameter is not the only way to find the slope of the tangent line, as we can instead apply the chain rule. We have
\begin{align*}
\frac{\d y}{\d t} &= 4t^3 - 6t, \\
\frac{\d x}{\d t} &= 3 t^2 + 1,
\end{align*}
and it follows that 
$$
\frac{\d y}{\d x} = \frac{\d y/ \d t}{\d x/ \d t} = \frac{4t^3 - 6t}{3t^2 + 1}.
$$
Evaluating at $t = 1$, we see that the slope of the tangent line is 
$$
\frac{4 - 6}{3 + 1} = \frac{-2}{4} = - \frac{1}{2}.
$$
The coordinates of the point are $(x(1),y(1)) = (2,-2)$ and the equation of the tangent line is therefore
$$
y+2 = -\frac{1}{2} (x - 2).
$$

III. We first determine the parameter value associated to the point $(1,-1)$ on the graph. Since $x(t) = t^2 = 1$, the possibilities are narrowed down to $t = \pm 1$. Since $y(t) = 1+2t$, it must be that $t=-1$. We compute the slope of the tangent line by evaluating 
$$
\frac{\d y/ \d t}{\d x / \d t} = \frac{2}{2t} = \frac{1}{t}
$$
at $t = -1$. The slope of the tangent line is therefore $-1$ and we conclude that the equation of the tangent line is
$$
y + 1 = -(x-1).
$$
\end{solution}
\end{problem}

\begin{problem}
Find the Cartesian coordinates of all points on the parametric curve
\begin{align*}
x(t) &= t \ln (t) - 1,\\
y(t) &= \frac{t^3}{3}-2t + 4,
\end{align*}
where $0< t \leq 10$, where the tangent line to the curve is either horizontal or vertical.

\begin{solution}
We first calculate the derivatives
\begin{align*}
x'(t) &= \ln(t) \\
y'(t) = t^2 - 2.
\end{align*}
Vertical tangent lines for this function correspond to parameter values $t$ where $x'(t) = 0$ and $y'(t) \neq 0$. The only parameter value satisfying these conditions is $t=1$. The Cartesian coordinates of the point corresponding to this parameter value are 
$$
(x(1),y(1)) = (-1,7/3).
$$

Horizontal tangent lines correspond to parameter values where $y'(t) = 0$ and $x'(t) \neq 0$. The only parameter value satisfying this condition is $t = \sqrt{2}$---$t = -\sqrt{2}$ also satisfies $y'(t) = 0$, but this parameter value is not in the domain of the function. The corresponding point has Cartesian coordinates
$$
(\sqrt{2} \ln(\sqrt{2}) - 1, \sqrt{2}^3/3 - 2 \sqrt{2} + 4).
$$
\end{solution}
\end{problem}
\end{document}
