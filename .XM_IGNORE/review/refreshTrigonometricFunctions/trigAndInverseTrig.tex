\documentclass{ximera}

%\newcommand{\RR}{\mathbb R}
\renewcommand{\d}{\,d}
\newcommand{\dd}[2][]{\frac{d #1}{d #2}}
\renewcommand{\l}{\ell}
\newcommand{\ddx}{\frac{d}{dx}}
\newcommand{\dfn}{\textbf}
\newcommand{\eval}[1]{\bigg[ #1 \bigg]}

\newcommand{\RR}{\mathbb R}
\renewcommand{\d}{\,d}
\newcommand{\dd}[2][]{\frac{d #1}{d #2}}
\renewcommand{\l}{\ell}
\newcommand{\ddx}{\frac{d}{dx}}
\newcommand{\dfn}{\textbf}
\newcommand{\eval}[1]{\bigg[ #1 \bigg]}


\author{Jim Talamo}
\license{Creative Commons 3.0 By-NC}
\outcome{}
\begin{document}

\begin{exercise}
This exercise reviews some important points about the range of inverse trigonometric functions.

Find the following values.

\begin{itemize}
\item $\tan\left(\arcsin\left(-\frac{3}{5}\right)\right) =\answer{-\frac{3}{4} }$.
\item $\cos\left(\arctan\left(-\frac{2}{4}\right)\right) =\answer{\frac{2}{\sqrt{5} }}$.
\end{itemize}

\begin{hint}
The range of $\arcsin(x)$  is $\left[\answer{-\frac{\pi}{2}},\answer{\frac{\pi}{2}}\right]$; we can think of $\arcsin(x)$ as the angle in $\left[\answer{-\frac{\pi}{2}},\answer{\frac{\pi}{2}}\right]$ whose sine is $x$.

To compute $\tan\left(\arcsin\left(-\frac{3}{5}\right)\right) =\answer{-\frac{3}{4} }$, the triangle that we can draw should be in Quadrant IV.  We can use the Pythagorean Theorem to compute $x=\answer{4}$.

\begin{image}
  \begin{tikzpicture}
	\begin{axis}[
            xmin=-1,xmax=5,ymin=-5,ymax=1,
            clip=false,
            axis lines=center,
            ticks=none,
            unit vector ratio*=1 1 1,
            xlabel=$x$, ylabel=$y$,
            %ytick={-2,-1,...,7},
	    %xtick={-2,-1,...,10},
	    %grid = major,
            every axis y label/.style={at=(current axis.above origin),anchor=south},
            every axis x label/.style={at=(current axis.right of origin),anchor=west},
          ]
          \addplot[mark=*,penColor] coordinates {(0,0)};
          \addplot[mark=*,penColor] coordinates {(4,-3)};
          
          \addplot[very thick,penColor,] plot coordinates {(0,0)(4,0)(4,-3)(0,0)};
          \node[right] at (axis cs:4, -3) [penColor] {\Large$(4,-3)$};
          \node[above] at (axis cs:2.2, .1) [penColor2] {\Large $x=4$};
          \node[right] at (axis cs:4, -1.5) [penColor] {\Large$-3$};
          \node[below] at (axis cs:2, -1.8) [penColor] {\Large$5$};
%          \node[right] at (axis cs:.7, -.3) [penColor] {\Large $\theta$};
        \end{axis}
\end{tikzpicture}
\end{image}
Now, we can find $\tan\left(\arcsin\left(-\frac{3}{5}\right)\right)$ from the picture.

To compute $\cos\left(\arctan\left(-\frac{2}{4}\right)\right)$, note that the range of $\arctan(x)$  is $\left[\answer{-\frac{\pi}{2}},\answer{\frac{\pi}{2}}\right]$; we can think of $\arctan(x)$ as the angle in $\left[\answer{-\frac{\pi}{2}},\answer{\frac{\pi}{2}}\right]$ whose tangent is $x$.  Now, follow a similar procedure as before.
\end{hint}
\end{exercise}

\begin{exercise}
This exercise reviews some important points about the relationship between the trigonometric and inverse trigonometric functions.
\begin{itemize}
\item $\arccos\left(\cos\left(\frac{5\pi}{6}\right)\right) =\answer{\frac{5\pi}{6} }$.
\item $\arctan\left(\tan\left(\frac{5\pi}{6}\right)\right) =\answer{-\frac{\pi}{6} }$.
\end{itemize}

\begin{hint}
The range of $\arccos(x)$  is $\left[\answer{0},\answer{\pi}\right]$; we can think of $\arccos(x)$ as the angle in $\left[\answer{0},\answer{\pi}\right]$ whose cosine is $x$.

The range of $\arctan(x)$  is $\left[\answer{-\frac{\pi}{2}},\answer{\frac{\pi}{2}}\right]$; we can think of $\tan(x)$ as the angle in $\left[\answer{-\frac{\pi}{2}},\answer{\frac{\pi}{2}}\right]$ whose tangent is $x$.

Is it possible that the inverse tangent of anything to be $\frac{5\pi}{6}$?

\end{hint}

True or False: 

\begin{itemize}
\item For any $x$ in $[0,2\pi]$, $\arcsin\big(\sin(x)\big) = x$.  \wordChoice{\choice{True}\choice[correct]{False}}
\item For any $x$ in $[0,2\pi]$, $\arccos\big(\cos(x)\big) = x$.  \wordChoice{\choice{True}\choice[correct]{False}}
\item For any $x$ in $[0,2\pi]$, $\arctan\big(\tan(x)\big) = x$.  \wordChoice{\choice{True}\choice[correct]{False}}
 \end{itemize}
 
\end{exercise}
\end{document}
