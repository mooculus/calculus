\documentclass{ximera}

\newcommand{\RR}{\mathbb R}
\renewcommand{\d}{\,d}
\newcommand{\dd}[2][]{\frac{d #1}{d #2}}
\renewcommand{\l}{\ell}
\newcommand{\ddx}{\frac{d}{dx}}
\newcommand{\dfn}{\textbf}
\newcommand{\eval}[1]{\bigg[ #1 \bigg]}


\outcome{Determine how the graph of a function looks without using a calculator.}

% Synthesis of everything we've learned

% We've learned so much stuff!
% Can't we put this all together somehow?

% Let's make a concept map of everything we've learned so far.

% Holes, point discontinuities, review from limits at the beginning


\title[Break-Ground:]{Wanted: graphing procedure}

\begin{document}
\begin{abstract}
Two young mathematicians discuss how to sketch the graphs of functions.
\end{abstract}
\maketitle

Check out this dialogue between two calculus students (based on a true
story):

\begin{dialogue}
\item[Devyn] Riley, I've been thinking about the derivative. 
\item[Riley] It's all about change. It's some ``change-detector'' tool
  for math.
\item[Devyn] I know!  What's crazy is that you can use it as a tool
  for sniffing out dirt on functions.
\item[Riley] First $f'$ tells us increasing or decreasing.
\item[Devyn] Then $f''$ tells us concavity.
\item[Riley] From just that we know all local maxes and mins.
\item[Devyn] And if we use limits, we can find any asymptotes!
\item[Riley] You know, I'd like to make up a procedure based on all
  these facts, that would tell me what the graph of any function would look like.
\item[Devyn] Me too! Let's get to work!
\end{dialogue}

\begin{problem}
  On some interval, we know that $f'(x)$ is positive and $f''(x)$ is positive.
  Which of the following is the best option for the shape of the graph on that
  interval?
  \begin{multipleChoice}%%BADBAD picture in the choices ok?
    \choice[correct]{\begin{tikzpicture}[framed,scale=.5,baseline=3ex]
	\begin{axis}[
            clip=false,
            domain=0:1,
            ymax=1,
            height=4.5cm,
            ymin=0,
            axis lines=none,
          ]
          \addplot [very thick, penColor, smooth] {x^2};
        \end{axis}
        \end{tikzpicture}}
	\choice{\begin{tikzpicture}[framed,scale=.5,baseline=3ex]
	\begin{axis}[
            clip=false,
            height=4.5cm,
            domain=0:1,
            ymax=1,
            ymin=0,
            axis lines=none,
          ]
          \addplot [very thick, penColor, smooth] {-(x-1)^2+1};
        \end{axis}
\end{tikzpicture}}
	\choice{\begin{tikzpicture}[framed,scale=.5,baseline=3ex]
	\begin{axis}[
            clip=false,
            height=4.5cm,
            domain=0:1,
            ymax=1,
            ymin=0,
            axis lines=none,
          ]
          \addplot [very thick, penColor, smooth] {(x-1)^2};
        \end{axis}
\end{tikzpicture}}
	\choice{\begin{tikzpicture}[framed,scale=.5,baseline=3ex]
	\begin{axis}[
            clip=false,
            height=4.5cm,
            domain=0:1,
            ymax=1,
            ymin=0,
            axis lines=none,
          ]
          \addplot [very thick, penColor, smooth] {-x^2+1};
        \end{axis}
\end{tikzpicture}}
  \end{multipleChoice}
\end{problem}

\begin{problem}
  On some interval, we know that $f'(x)$ is negative and $f''(x)$ is positive.
  Which of the following is the best option for the shape of the graph on that
  interval?
  \begin{multipleChoice}%%BADBAD picture in the choices ok?
  	\choice{\begin{tikzpicture}[framed,scale=.5,baseline=3ex]
	\begin{axis}[
            clip=false,
            domain=0:1,
            ymax=1,
            height=4.5cm,
            ymin=0,
            axis lines=none,
          ]
          \addplot [very thick, penColor, smooth] {x^2};
        \end{axis}
        \end{tikzpicture}}
	\choice{\begin{tikzpicture}[framed,scale=.5,baseline=3ex]
	\begin{axis}[
            clip=false,
            height=4.5cm,
            domain=0:1,
            ymax=1,
            ymin=0,
            axis lines=none,
          ]
          \addplot [very thick, penColor, smooth] {-(x-1)^2+1};
        \end{axis}
\end{tikzpicture}}
	\choice[correct]{\begin{tikzpicture}[framed,scale=.5,baseline=3ex]
	\begin{axis}[
            clip=false,
            height=4.5cm,
            domain=0:1,
            ymax=1,
            ymin=0,
            axis lines=none,
          ]
          \addplot [very thick, penColor, smooth] {(x-1)^2};
        \end{axis}
\end{tikzpicture}}
	\choice{\begin{tikzpicture}[framed,scale=.5,baseline=3ex]
	\begin{axis}[
            clip=false,
            height=4.5cm,
            domain=0:1,
            ymax=1,
            ymin=0,
            axis lines=none,
          ]
          \addplot [very thick, penColor, smooth] {-x^2+1};
        \end{axis}
\end{tikzpicture}}
  \end{multipleChoice}
\end{problem}


\begin{problem}%% BADBAD This is a good problem. How do we do it?
  Below is a list of features of a graph of a function.
  \begin{enumerate}
  \item Find local extrema.
  \item Find critical points.
  \item Identify inflection points and concavity.
  \item Determine an interval that shows all relevant behavior of the function. 
  \item Find any vertical asymptotes.
  \item Compute $f''(x)$.
  \item If possible, find the $x$-intercepts.
  \item Find the $y$-intercept.
  \item Compute $f'(x)$.
  \item Find intervals of increasing/decreasing behavior.
  \item Find horizontal asymptotes.
  \item Find candidates for inflection points.
 \end{enumerate}
  In what order should we take these steps? For example, one must compute
   $f'(x)$ before computing $f''(x)$. Also, one must compute $f'(x)$ before 
   finding the critical points. There is more than one correct answer.
  \begin{freeResponse}
  Here is one possible answer to this question.  Compare it with yours!
    \begin{enumerate}
  \item Find the $y$-intercept.
  \item Find any vertical asymptotes.
  \item Compute $f'(x)$.
  \item Compute $f''(x)$.
  \item Find critical points.
  \item Find intervals of increasing/decreasing behavior.
  \item Find candidates for inflection points.
  \item Identify inflection points and concavity.
  \item Find local extrema.
  \item Find horizontal asymptotes.
  \item If possible, find the $x$-intercepts.
  \item Determine an interval that shows all relevant behavior of the function. 
  \end{enumerate}
  \end{freeResponse}
\end{problem}

%%
%% %% \begin{xarmaBoost}
%%   Write down at least \textbf{five} questions for this lecture. After
%%   you have your questions, label them as ``Level 1,'' ``Level 2,'' or
%%   ``Level 3'' where:
%% \begin{description}
%% \item[Level 1] Means you know the answer, or know exactly how to do
%%   this problem.
%% \item[Level 2] Means you think you know how to do the problem.
%% \item[Level 3] Means you have no idea how to do the problem.
%% \end{description}
%% \begin{freeResponse}
%% \end{freeResponse}
%% \end{xarmaBoost}

%%
\end{document}
