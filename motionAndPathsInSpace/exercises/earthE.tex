\documentclass{ximera}

\newcommand{\RR}{\mathbb R}
\renewcommand{\d}{\,d}
\newcommand{\dd}[2][]{\frac{d #1}{d #2}}
\renewcommand{\l}{\ell}
\newcommand{\ddx}{\frac{d}{dx}}
\newcommand{\dfn}{\textbf}
\newcommand{\eval}[1]{\bigg[ #1 \bigg]}


\author{Bart Snapp and Jim Talamo}

\outcome{Reparametrize a curve.}
\outcome{Parameterize a curve in terms of arc length.}

\begin{document}
\begin{exercise}
  The Earth travels in an orbit around the Sun that can be
  approximated by a circle. The distance from the Earth to the Sun is
  (on average) $1$
  \link[au]{https://en.wikipedia.org/wiki/Astronomical_unit}. We are
  going to make some models of the Earth's orbit in the
  $(x,y)$-plane. For all of our models, we will make the following
  assumptions:
  \begin{itemize}
  \item The Sun will be at the origin.
  \item At the starting time, $t=0$, the Earth will be at the point
    $(1,0)$ in the $(x,y)$-plane.
  \item The Earth will travel in a counterclockwise direction around
    the Sun.
  \end{itemize}
  \begin{exercise}
    Give a parametrization of the Earth's orbit that will model the
    Earth's position in terms of $t$, where the units are years.
    \[
    \vec{y}(t) = \vector{\answer{\cos(2\pi t)},\answer{\sin(2\pi t)}} 
    \]
  \end{exercise}
  \begin{exercise}
    Give a parametrization of the Earth's orbit that will model the
    Earth's position in terms of $t$, where the units are months.
    \[
    \vec{m}(t) = \vector{\answer{\cos(2\pi t/12)},\answer{\sin(2\pi t/12)}} 
    \]
  \end{exercise}
  
  \begin{exercise}
    Give a parametrization of the Earth's orbit that will model the
    Earth's position in terms of $t$, where the units are days.
    \[
    \vec{d}(t) = \vector{\answer{\cos(2\pi t/365)},\answer{\sin(2\pi t/365)}} 
    \]
  \end{exercise}
  
  \begin{exercise}
    Given that the Earth's orbit has a circumference of $940$ million
    miles, give a parametrization of the Earth's orbit that will model
    the Earth's position in terms of $S$, the distance traveled in
    millions of miles.
    \[
    \vec{e}(S) = \vector{\answer{\cos(2\pi S/940)},\answer{\sin(2\pi S/940)}} 
    \]
    \begin{exercise}
 %I've changed the content of the original exercise.
 
      True or False: This is an arc length parameterization.
    \begin{multipleChoice}
      \choice{True}
      \choice[correct]{False}
    \end{multipleChoice}
    \begin{exercise}
    It may be tempting to think that since we are using ``distance'' in our parameterization that this means we have parameterized by arclength, but we can actually check whether we have an arclength parameterization by checking if
    
    \begin{selectAll}
    \choice[correct]{$\int_0^S \left|\vec{e}'(t)\right| \d t = S$.}
    \choice[correct]{$ \left|\vec{e}'(S)\right| =1$ for all $S$.}
    \end{selectAll}
        
      Note that $\vec{e}'(S) = \vector{\answer{(-\pi/470) \sin(2\pi S/940)},\answer{(\pi/470) \cos(2\pi S/940)}}$, so $\left|\vec{e}'(S)\right| = \answer{\pi/470}$.
      
While this is enough to determine that we have not used arclength as a parameter, we also can note that 

\[
\int_0^S \left|\vec{e}'(t)\right| \d t = \int_0^S \frac{\pi}{470}  \d t = \answer{\frac{\pi}{470} S} \neq S.
\]      

Let's now find a parameterization $\vec{p}(s)$ of the orbit that uses arclength as a parameter.

Above, we found that $\left|\vec{e}'(S)\right| = \frac{\pi}{470}$, so $s = \int_0^S \left|\vec{e}'(t)\right| \d t = \answer{\frac{\pi}{470} s}$ or, solving for $s$, we find

\[
s = \answer{\frac{470}{\pi}S}.
\]

Thus, a parameterization that uses arclength as a parameter is
\[
\vec{p}(s) = \vector{\answer{\cos(s) },\answer{\sin(s)}}. 
\]

\begin{feedback}[correct]
 Note that at the \emph{start} of the problem, we chose the length unit $\unit{au}$ to describe the orbit of the Earth; that is, we assume that the Earth's initial position is at $(1,0)$.  Since the unit $\unit{au}$ is the distance the Earth is from the Sun, and we assume that the Earth travels around the Sun in a circular orbit, our units on the $x$ and $y$ axis must be in $\unit{au}$.  
 
The parameter $S$ in this problem was millions of miles, and we found that $s = \frac{470}{\pi}S$.  Since $1 \unit{au} = 150\unit{million~miles}$, note that:
 
\[ \pi/470\cdot 150 \unit{million~miles}     \approx 1\unit{au} , \]

so the $\pi/470$ is the approximate conversion factor from millions of miles to $\unit{au}$.

An important point this brings up in arclength applications in STEM fields is the following:

\begin{quote}
``Parameterization by arclength'' only has meaning if you first choose your units!
\end{quote}

\end{feedback}
 
                   \end{exercise}
        \end{exercise}
      \end{exercise}
        \end{exercise}
\end{document}
