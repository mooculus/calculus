\documentclass{ximera}

\newcommand{\RR}{\mathbb R}
\renewcommand{\d}{\,d}
\newcommand{\dd}[2][]{\frac{d #1}{d #2}}
\renewcommand{\l}{\ell}
\newcommand{\ddx}{\frac{d}{dx}}
\newcommand{\dfn}{\textbf}
\newcommand{\eval}[1]{\bigg[ #1 \bigg]}


\outcome{}

\title[Dig-In:]{Relating Velocity and Position}
\begin{document}

A central theme of this course has been that we can often gain a better understanding of a function by looking at its derivative, and then working backwards.  This has been our approach to max/min problems, curve sketching, linear approximation, etc.  So antiderivatives have really been important to us all along.

We have a graphical interpretation of the derivative as the slope of a tangent line at a point.  We have not yet found a graphical interpretation of the antiderivative.
\begin{question}
  \begin{question}
   Let $f$ be the constant function $4$. Let $F$ be any antiderivative of $f$.  Then $F(5)-F(2) = \answer{12}$.
   \begin{hint}
    This is just a fancy way of saying that $f$ is a line of slope $4$, and we are looking for the rise corresponding to a run of $5-2= 3$
   \end{hint}
   \begin{hint}
    $F(5) - F(2) = 4(5-2) = 12$
   \end{hint}
  \end{question}
  
  \begin{question}
    The area of the region bounded by the grpah of $f$, the horizontal axis, and the vertical lines $x=2$ and $x=5$ i \answer{12}.
    
    \begin{hint}
      %BADBAD PICTURE
    \end{hint}
    \begin{hint}
      This is a rectangle with height $4$ and width $3$, so the area is $12$
    \end{hint}
  \end{question}
\end{question}

The fact that these two answers are the same is the germ of one of the most ``fundamental'' ideas in all of calculus.

\begin{idea}[Two models of multiplication]
  There are two basic models of multiplication:  A "rate times time" perpective and an area perspective.  For instance, we could interpret $3*4$ as an answer to the question "If I am going $3 \textrm{mph}$ for $4$ hours, how far have I traveled?" or as the answer to the question "What is the area of a rectangle with length $3$ and width $4$?"
\end{idea}

We can apply this basic idea to the problem of approximating antiderivatives.

\begin{example}
  Let $f(x) = x^2$, and let $F$ be an antiderivative of $f$.  We want to approximate $F(3) - F(1)$.  We can 
\end{example}
\end{document}
