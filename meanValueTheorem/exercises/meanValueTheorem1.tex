\documentclass{ximera}

\newcommand{\RR}{\mathbb R}
\renewcommand{\d}{\,d}
\newcommand{\dd}[2][]{\frac{d #1}{d #2}}
\renewcommand{\l}{\ell}
\newcommand{\ddx}{\frac{d}{dx}}
\newcommand{\dfn}{\textbf}
\newcommand{\eval}[1]{\bigg[ #1 \bigg]}


\outcome{Understand the statement of the Extreme Value Theorem.}
\outcome{Understand the statement of the Mean Value Theorem.}
%\outcome{Sketch pictures to illustrate why the Mean Value Theorem is true.}
\outcome{Determine whether Rolle's Theorem or the Mean Value Theorem can be applied.}
\outcome{Find the values guaranteed by Rolle's Theorem or the Mean Value Theorem.}
%\outcome{Use the Mean Value Theorem to solve word problems.}
%\outcome{Compare and contrast the Intermediate Value Theorem, the Mean Value Theorem, and Rolle's Theorem.}
%\outcome{Identify calculus ideas which are consequences of the Mean Value Theorem.}
%\outcome{Use the Mean Value Theorem to bound the error in linear approximation.}

\author{Nela Lakos \and Kyle Parsons}

\begin{document}
\begin{exercise}

Consider the functions $f_A$, $f_B$, $f_C$, and $f_D$ graphed below.

\resizebox{0.45\textwidth}{!}{
  \begin{tikzpicture}
    \begin{axis}[
        xmin=-.3,xmax=5.3,ymin=-.3,ymax=5.3,
        clip=true,
        unit vector ratio*=1 1 1,
        axis lines=center,
        grid = major,
        ytick={-20,-19,...,20},
    	xtick={-20,-19,...,20},
        xlabel=$x$, ylabel=$y$,
        y tick label style={anchor=west},
        every axis y label/.style={at=(current axis.above origin),anchor=south},
        every axis x label/.style={at=(current axis.right of origin),anchor=west},
      ]
      \addplot[very thick,penColor,domain=1:5] plot{(x-3)^2};
      
      \addplot[penColor,only marks,mark=*] coordinates{(1,4) (5,4)};
            
      \node at (axis cs:1.3,.5) {$y=f_A(x)$};
      \end{axis}`
  \end{tikzpicture}}
\hfill
\resizebox{0.45\textwidth}{!}{
  \begin{tikzpicture}
    \begin{axis}[
        xmin=-.3,xmax=5.3,ymin=-.3,ymax=5.3,
        clip=true,
        unit vector ratio*=1 1 1,
        axis lines=center,
        grid = major,
        ytick={-20,-19,...,20},
    	xtick={-20,-19,...,20},
        xlabel=$x$, ylabel=$y$,
        y tick label style={anchor=west},
        every axis y label/.style={at=(current axis.above origin),anchor=south},
        every axis x label/.style={at=(current axis.right of origin),anchor=west},
      ]
      \addplot[very thick,penColor,domain=1:3] plot{sqrt(2*(3-x))};
      \addplot[very thick,penColor,domain=3:5] plot{sqrt(2*(x-3))};
      
      \addplot[penColor,only marks,mark=*] coordinates{(1,2) (5,2)};
      
      \node at (axis cs:1.3,.5) {$y=f_B(x)$};
      \end{axis}`
  \end{tikzpicture}}

\resizebox{0.45\textwidth}{!}{
  \begin{tikzpicture}
    \begin{axis}[
        xmin=-.3,xmax=5.3,ymin=-.3,ymax=5.3,
        clip=true,
        unit vector ratio*=1 1 1,
        axis lines=center,
        grid = major,
        ytick={-20,-19,...,20},
    	xtick={-20,-19,...,20},
        xlabel=$x$, ylabel=$y$,
        y tick label style={anchor=west},
        every axis y label/.style={at=(current axis.above origin),anchor=south},
        every axis x label/.style={at=(current axis.right of origin),anchor=west},
      ]
      \addplot[very thick,penColor,domain=1:5] plot{5-x};

      \addplot[penColor,only marks,mark=*] coordinates{(1,5) (5,1)};
      \addplot[penColor,fill=white,only marks,mark=*] coordinates{(1,4) (5,0)};
      
      \node at (axis cs:1.3,.5) {$y=f_C(x)$};
      \end{axis}`
  \end{tikzpicture}}
\hfill
\resizebox{0.45\textwidth}{!}{
  \begin{tikzpicture}
    \begin{axis}[
        xmin=-.3,xmax=5.3,ymin=-.3,ymax=5.3,
        clip=true,
        unit vector ratio*=1 1 1,
        axis lines=center,
        grid = major,
        ytick={-20,-19,...,20},
    	xtick={-20,-19,...,20},
        xlabel=$x$, ylabel=$y$,
        y tick label style={anchor=west},
        every axis y label/.style={at=(current axis.above origin),anchor=south},
        every axis x label/.style={at=(current axis.right of origin),anchor=west},
      ]
      \addplot[very thick,penColor,domain=1:5] plot{5-x};

      \addplot[penColor,only marks,mark=*] coordinates{(1,5) (5,0)};
      \addplot[penColor,fill=white,only marks,mark=*] coordinates{(1,4)};
      
      \node at (axis cs:1.3,.5) {$y=f_D(x)$};
      \end{axis}`
  \end{tikzpicture}}

Select all functions that are continuous on $[1,5]$.
\begin{selectAll}
\choice[correct]{$f_A(x)$}
\choice[correct]{$f_B(x)$}
\choice{$f_C(x)$}
\choice{$f_D(x)$}
\end{selectAll}

Select all functions that are differentiable on $(1,5)$.
\begin{selectAll}
\choice[correct]{$f_A(x)$}
\choice{$f_B(x)$}
\choice[correct]{$f_C(x)$}
\choice[correct]{$f_D(x)$}
\end{selectAll}

Answer the following true-false questions.

The Mean Value Theorem \textbf{does} apply to $f_B(x)$ on $[1,5]$, and $c=3$ is the point guaranteed to exist.
\begin{multipleChoice}
\choice{True}
\choice[correct]{False}
\end{multipleChoice}

The Mean Value Theorem \textbf{does not} apply to $f_A(x)$ on $[1,5]$. 
\begin{multipleChoice}
\choice{True}
\choice[correct]{False}
\end{multipleChoice}

The Mean Value Theorem \textbf{does} apply to $f_A(x)$ on $[1,5]$, and $c=3$ is the point guaranteed to exist.
\begin{multipleChoice}
\choice[correct]{True}
\choice{False}
\end{multipleChoice}

$f_C(x)$ attains an absolute minimum on $[1,5]$.
\begin{multipleChoice}
\choice{True}
\choice[correct]{False}
\end{multipleChoice}

$f_D(x)$ attains an absolute minimum on $[1,5]$ at a critical point.
\begin{multipleChoice}
\choice{True}
\choice[correct]{False}
\end{multipleChoice}

\end{exercise}
\end{document}