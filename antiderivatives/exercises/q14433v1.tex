\documentclass{ximera}
\newcommand{\RR}{\mathbb R}
\renewcommand{\d}{\,d}
\newcommand{\dd}[2][]{\frac{d #1}{d #2}}
\renewcommand{\l}{\ell}
\newcommand{\ddx}{\frac{d}{dx}}
\newcommand{\dfn}{\textbf}
\newcommand{\eval}[1]{\bigg[ #1 \bigg]}

\author{Emma Smith Zbarsky}
\license{Creative Commons Attribution 3.0 Unported}
\acknowledgement{https://quadbase.org/questions/q14433v1}
\begin{document}

\begin{exercise}

Find the antiderivative: \[\int 3\sin(3\theta)\; d\theta.\]


\begin{hint}
This is an antiderivative question, which means it is asking what the
most general possible function $f(\theta)$ is that satisfies
$f'(\theta) = 3\sin(3\theta)$.
\end{hint}


\begin{hint}
We know that the derivative of $\cos(3\theta)$ is $-3\sin(3\theta)$, so
the derivative of $-\cos(3\theta)$ is $3\sin(3\theta)$.

To work this out from first principles, let us begin by working with
simply $\int \sin(u)\; du$. As is easy to establish, \begin{align*}
\frac{d}{du}\left(\sin(u)\right) &= \cos(u) \\
\frac{d}{du}\left(\cos(u)\right) &= -\sin(u) \\
\frac{d}{du}\left(-\sin(u)\right) &= -\cos(u) \\
\frac{d}{du}\left(-\cos(u)\right) &= \sin(u) 
\end{align*} Reading left to right (or top to bottom), this table
gives us derivatives of sines and cosines. Reading right to left or
bottom to top gives us antiderivatives of sines and cosines. Therefore,
\[\int \sin(u)\; du = -\cos(u)+C.\] Now, in our case, $u = 3\theta$ and
$\frac{du}{d\theta} = 3$ or, by abuse of notation, $du = 3d\theta$.
Plugging this in, we see that
\[\int \sin(u)\; du = \int \sin(3\theta) 3\; d\theta = -\cos(3\theta)+C.\]
\end{hint}


\begin{multipleChoice}
\choice{$9\cos(3\theta)+C$}
\choice[correct]{$-\cos(3\theta)+C$}
\choice{$-9\cos(3\theta)+C$}
\choice{$\sin(3\theta)+C$}
\choice{$\cos(3\theta)+C$}
\end{multipleChoice}

\end{exercise}
\end{document}
