\documentclass{ximera}

\newcommand{\RR}{\mathbb R}
\renewcommand{\d}{\,d}
\newcommand{\dd}[2][]{\frac{d #1}{d #2}}
\renewcommand{\l}{\ell}
\newcommand{\ddx}{\frac{d}{dx}}
\newcommand{\dfn}{\textbf}
\newcommand{\eval}[1]{\bigg[ #1 \bigg]}


\outcome{}

\title[Dig-In:]{Relating velocity and position}
\begin{abstract}
\end{abstract}
\maketitle

\begin{document}
A central theme of this course has been that we can often gain a
better understanding of a function by looking at its derivative, and
then working backwards.  This has been our approach to max/min
problems, curve sketching, linear approximation, etc.  So
antiderivatives have really been important to us all along.

We have a graphical interpretation of the derivative as the slope of a
tangent line at a point.  We have not yet found a graphical
interpretation of the antiderivative.
\begin{question}
  \begin{question}
   Let $f$ be the constant function $4$. Let $F$ be any antiderivative
   of $f$.  Then $F(5)-F(2) = \answer{12}$.
   \begin{hint}
    This is just a fancy way of saying that $f$ is a line of slope
    $4$, and we are looking for the rise corresponding to a run of
    $5-2= 3$
   \end{hint}
   \begin{hint}
    $F(5) - F(2) = 4(5-2) = 12$
   \end{hint}
  \end{question}
  
  \begin{question}
    The area of the region bounded by the grpah of $f$, the horizontal
    axis, and the vertical lines $x=2$ and $x=5$ i \answer{12}.
    
    \begin{hint}
      %BADBAD PICTURE
    \end{hint}
    \begin{hint}
      This is a rectangle with height $4$ and width $3$, so the area is $12$
    \end{hint}
  \end{question}
\end{question}

The fact that these two answers are the same is the germ of one of the
most ``fundamental'' ideas in all of calculus.

\begin{idea}[Two models of multiplication]
  There are two basic models of multiplication: A "rate times time"
  perpective and an area perspective.  For instance, we could
  interpret $3*4$ as an answer to the question "If I am going $3
  \textrm{mph}$ for $4$ hours, how far have I traveled?" or as the
  answer to the question "What is the area of a rectangle with length
  $3$ and width $4$?"
\end{idea}

We can apply this basic idea to the problem of approximating
antiderivatives.

  Let $f$ be a given function, and let $F$ be an antiderivative of $f$.  We want to approximate $F(3) - F(1)$.  Let's get an approximation using $4$ subintervals.
  
  \begin{align*}F(3) - F(1) &= (F(3) - F(2.5))+(F(2.5) - F(2))+(F(2) - F(1.5))+(F(1.5) - F(1))\\
  &\approx F'(3)\frac{1}{2}+F'(2.5)\frac{1}{2}+F'(2)\frac{1}{2}+F'(1.5)\frac{1}{2}\\
  &=f(3)\frac{1}{2}+f(2.5)\frac{1}{2}+f(2)\frac{1}{2}+f(1.5)\frac{1}{2}
  \end{align*}
  
  We can visualize this by the following piecewise linear function
  
  %BADBAD GRAPH  graph of $f$ on $[1,3]$ with  piecewise linear approximation to $F$ below it.  One line segment is highlighted, with its corresponding rectangle also highlighted.
  
  On the one hand, we can interpret each $F'(x)\Delta x$ as an approximation in the change in the height of $F$ over the subinterval.  On the other hand, since $F' =f$, we can think of $F'(x)\Delta x = f(x)\Delta x$ as the area of a rectangle based on the subinterval, and with hieght $f(x)$.  So the total change in $F$ over $[1,3]$ can be reinterpreted as the sum of the areas of these rectangles.
  
  %BADBAD GRAPH  A lot more partitions
  
  If we make our subintervals smaller, the repeated linear approximation should get more precise, and the sum of the areas of the rectangles gets closer to the area under the curve.
  
  
\end{document}
