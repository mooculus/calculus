\documentclass{ximera}

\newcommand{\RR}{\mathbb R}
\renewcommand{\d}{\,d}
\newcommand{\dd}[2][]{\frac{d #1}{d #2}}
\renewcommand{\l}{\ell}
\newcommand{\ddx}{\frac{d}{dx}}
\newcommand{\dfn}{\textbf}
\newcommand{\eval}[1]{\bigg[ #1 \bigg]}


\outcome{}

\title[Dig-In:]{Sigma Notation}

\begin{document}
\begin{abstract}
  Sigma notation is a convenient way to express a sum of many terms.
\end{abstract}


Sigma notation may seem scary at first, but it really isn't that bad.

\begin{definition}
  Let $f$ be a function, and $m \leq n$ integers.  Then we can write
  the sum
  \[
  f(m)+f(m+1)+f(m+2)+\dots+f(n-1)+f(n) = \sum_{k=m}^{n} f(k)
  \] 
  We read this as ``The sum of $f$ of $k$ from $k$ equals $m$ to
  $k$ equals $n$.''
\end{definition}

This is pretty abstract. Let's see if we can sort this out.


\begin{question}
What are the terms of this sum?
  \[
  \sum_{k=2}^{5} \sin(k)
  \]
  \begin{prompt}
    \[
    \sin(\answer{2}) + \sin(3) + \answer{\sin(4)} + \sin(5)
    \]
  \end{prompt}
\end{question}

\begin{example}
$\sum_{k=3}^{k=4} \frac{1}{1+k} = \frac{1}{4}+\frac{1}{5}$
\end{example}

\begin{question}
	$\sum_{k=1}^{4} k  = \answer{10}$
	\begin{hint}
		$\sum_{k=1}^{4} k  = 1+2+3+4=10$
	\end{hint}
\end{question}

The variable $k$ in $\sum_{k=m}^{k=n}$ is called the ``index of summation'', or just ``the index''.  It can be any variable we like, but the letters $i,j,k$ are used traditionally.

\begin{question}
$\sum_{i=2}^{i=3} i^2  = \answer{13}$
\begin{hint}
	$\sum_{i=2}^{i=3} i^2  = 2^2+3^2 = 4+9=13$
\end{hint}
\end{question}

\begin{question}
	$1+\frac{1}{2} + \frac{1}{3} + \frac{1}{4} = \sum_{j=1}^{j= \answer{4}} \answer{1/j}$
	\begin{hint}
		There are $4$ terms, so since we start counting at $j=1$, we must go up to $j=4$.
	\end{hint}
	\begin{hint}
		$1+\frac{1}{2} + \frac{1}{3} + \frac{1}{4} = \sum_{j=1}^{j= 4} \frac{1}{j}$
	\end{hint}
\end{question}

\begin{question}
	$\sum_{i=5}^{i=5} i^3 = \answer{125}$
	\begin{hint}
		This is kind of funny, but in this case we just have one term, namely $5^3 = 125$
	\end{hint}
\end{question}

\begin{question}

	$\sum_{i=2}^{i=6} k = \answer{20}$

	 \begin{hint}
	 	$\sum_{i=2}^{i=6} k = 2+3+4+5+6 = 20$
	 \end{hint}
	
	$\sum_{j=0}^{j=4} (j+2) = \answer{20}$
	 \begin{hint}
	 	$\sum_{i=2}^{i=6} k = 2+3+4+5+6 = 20$ again!
	 \end{hint}
	
	\begin{feedback}
Did you notice how these two expressions had all the same terms?  Both are just shorthands for the sum $2+3+4+5+6$.  This is called ``reindexing'' a sum.
	\end{feedback}
\end{question}

\begin{question}
	Reindexing the sum $\sum_{j=4}^{j=7} \sin(j-2)$ to start at $k=1$, we have $\sum_{j=4}^{j=7} \sin(j-2) = \sum_{k=1}^{k=\answer{4}} \answer{\sin(k+1)}$
		\begin{hint}
			$\sum_{j=4}^{j=7} \sin(j-2) = \sin(2)+\sin(3) + \sin(4)+\sin(5) = \sum_{k=1}^{k=4} \sin(k+1)$
		\end{hint}
\end{question}

\begin{question}
	The sum $\sin(4+\frac{3}{n}) + \sin(4+\frac{6}{n})+\sin(4+\frac{9}{n})+...$ has $n$ terms. In sigma notation, this sum can be expressed as $\sum_{k=1}^{k=n} \answer{ \sin(4+\frac{3k}{n})}$
		\begin{hint}
			The $k^{\textrm{th}}$ term is of the form $4+\frac{3k}{n}$, so the sum is $\sum_{k=1}^{k=n} \sin(4+\frac{3k}{n})$
		\end{hint}
\end{question}

\begin{question}
	Fix a number $n$.  Then $\sum_{k=1}^{k=n} 1 = \answer{n}$
	 \begin{hint}
	 	By definition, $\sum_{k=1}^{k=n} 1$ is the sum of $n$ ones, which is just $n$
	 \end{hint}
\end{question}

\begin{question}
	If $\sum_{k=1}^{k=n} f(k) = n^2$, then $f(j) = \answer{2j-1}$
		\begin{hint}
			To find $f(j)$, we could think of this as $\sum_{k=1}^{k=j} f(j) - \sum_{k=1}^{k=j-1} f(j)$
		\end{hint}
		\begin{hint}
			So $f(j) = j^2 - (j-1)^2 = j^2- (j^2-2j+1) = 2j-1$
		\end{hint}
	\begin{feedback}
This is kind of cool.  It says that the sum of the first $n$ odd number is $n^2$.  Test it and see!  Can you find a geometric interpretation of this?  If you are interested by this, talk to your TA!
	\end{feedback}
\end{question}

\begin{question}
	Which of the following equations could possibly make any sense at all?  Mark all that apply.
	
	\begin{multipleChoice}
	\choice{$\sum_{j=1}^{j=n} f(j) = j^3$}
	\choice[correct]{$\sum_{j=1}^{j=n} f(n) = nf(n)$}
	\choice[correct]{$\sum_{j=1}^{j=n} f(j) = n^3$}
	\end{multipleChoice}
	
	\begin{hint}
		\begin{itemize}
		\item $\sum_{j=1}^{j=n} f(j) = j^3$ cannot make any sense, since one one side $j$ is telling us the index of a term we are summing, and on the other side it is a fixed number.  These two meanings of $j$ cannot coexist.
		\item $\sum_{j=1}^{j=n} f(n) = nf(n)$ not only makes sense, it is universally true!  It is okay that $n$ appears in all three parts of the expression, since it is just a fixed number
		\item $\sum_{j=1}^{j=n} f(j) = n^3$ is also fine.  As a bonus challenge, can you find the function $f$ which makes this true?
		\end{itemize}
	\end{hint}
\end{question}


\end{document}
