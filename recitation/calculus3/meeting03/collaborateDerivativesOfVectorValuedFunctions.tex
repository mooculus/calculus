\documentclass[handout,hints,noauthor,nooutcomes]{ximera}

\newcommand{\RR}{\mathbb R}
\renewcommand{\d}{\,d}
\newcommand{\dd}[2][]{\frac{d #1}{d #2}}
\renewcommand{\l}{\ell}
\newcommand{\ddx}{\frac{d}{dx}}
\newcommand{\dfn}{\textbf}
\newcommand{\eval}[1]{\bigg[ #1 \bigg]}


\author{Bart Snapp}

\title[Collaborate:]{Derivatives of vector-valued functions}

\begin{document}
\begin{abstract}
  We think about the derivative of vector-valued functions.
\end{abstract}
\maketitle

\textbf{Work in groups of 3--4, writing your answers on a separate
  sheet of paper.}


Previously in calculus course you learned the following metaphor for
the derivatives:
\begin{quote}
  Given a function $f:\R\to\R$, the derivative of $f$ is the slope of
  the tangent line at any point on the graph $y = f(x)$.
\end{quote}

This is really a great metaphor for functions that map from $\R$ to
$\R$. However, now we are studying vector-valued functions. A new
metaphor is needed:
\begin{quote}
  Given a vector-valued function $\vec{f}:\R^n\to\R$, the derivative
  of $\vec{f}$ is a tangent vector at any point on the graph of
  $\vec{f}$.
\end{quote}
Let's see if we can figure what this is saying.

\section{Lines}

Suppose you have a line given by the vector valued function $\vecl$.

\begin{image}
  \begin{tikzpicture}
    \begin{axis}%
      [
	xmin=-4,xmax=5,
        ymin=-2,ymax=3,
        xlabel=$x$,ylabel=$y$,
        axis lines=center,
        every axis y label/.style={at=(current axis.above origin),anchor=south},
        every axis x label/.style={at=(current axis.right of origin),anchor=west},
        clip=false,
	grid =major,
        width=9cm,
        height=5cm,
        xtick={-4,-3,...,5},
        ytick={-2,-1,...,3},
      ]
      \addplot[penColor,ultra thick,domain=-4:5] {
        -x/3+1
      };
        \addplot[color=penColor,fill=penColor,only marks,mark=*] coordinates{(3,0)};  %% closed hole
        \addplot[color=penColor,fill=penColor,only marks,mark=*] coordinates{(-3,2)};  %% closed hole
        \node[penColor,above] at (axis cs: 3,0) {$\vecl(0)$};
        \node[penColor,above right] at (axis cs: -3,2) {$\vecl(2)$};
      \end{axis}
    \end{tikzpicture}
\end{image}


\begin{problem}
  Pencil-in some tangent vectors for $\vecl$ above. 
\end{problem}

\begin{problem}
  Someone has plotted $\vecl'$ below:
  \begin{image}
  \begin{tikzpicture}
    \begin{axis}%
      [
	xmin=-4,xmax=5,
        ymin=-2,ymax=3,
        xlabel=$x$,ylabel=$y$,
        axis lines=center,
        every axis y label/.style={at=(current axis.above origin),anchor=south},
        every axis x label/.style={at=(current axis.right of origin),anchor=west},
        clip=false,
	grid =major,
        width=9cm,
        height=5cm,
        xtick={-4,-3,...,5},
        ytick={-2,-1,...,3},
      ]
      \addplot[color=penColor,fill=penColor,only marks,mark=*] coordinates{(-3,1)};  %% closed hole
      \node[penColor,above] at (axis cs: -3,1) {$\vecl'$};
    \end{axis}
    \end{tikzpicture}
  \end{image}
  Make sense of this plot. Explain what is going on to someone else.
\end{problem}


\section{Circles}

The a circle of radius $2$ centered at $(3,1$ is given by
\[
\vec{c}(t) = \vector{3 + 2 \cos(t), 1+2\sin(t)}
\]
and here is a plot:

\begin{image}
  \begin{tikzpicture}
    \begin{axis}%
      [
	xmin=-4,xmax=6,
        ymin=-2,ymax=4,
        xlabel=$x$,ylabel=$y$,
        axis lines=center,
        every axis y label/.style={at=(current axis.above origin),anchor=south},
        every axis x label/.style={at=(current axis.right of origin),anchor=west},
        clip=false,
	grid =major,
        width=10cm,
        height=6.7cm,
        xtick={-4,-3,...,6},
        ytick={-2,-1,...,4},
      ]
      \addplot[penColor,smooth,ultra thick,domain=0:360] (
       {3+2*cos(x)},{1+2*sin(x)}
      );
    \end{axis}
    \end{tikzpicture}
\end{image}

\begin{problem}
  Pencil-in some tangent vectors for $\vec{c}$ above. 
\end{problem}



\begin{problem}
  Someone has plotted $\vec{c}'$ below:
  \begin{image}
    \begin{tikzpicture}
      \begin{axis}%
        [
	xmin=-4,xmax=6,
        ymin=-3,ymax=3,
        xlabel=$x$,ylabel=$y$,
        axis lines=center,
        every axis y label/.style={at=(current axis.above origin),anchor=south},
        every axis x label/.style={at=(current axis.right of origin),anchor=west},
        clip=false,
	grid =major,
        width=10cm,
        height=6.7cm,
        xtick={-4,-3,...,6},
        ytick={-3,-2,...,3},
      ]
      \addplot[penColor,smooth,ultra thick,domain=0:360] (
       {-2*sin(x)},{2*cos(x)}
      );
    \end{axis}
    \end{tikzpicture}
  \end{image}
  Make sense of this plot. Explain what is going on to someone else.
\end{problem}

\section{Projectile motion}

Vector-valued functions are excellent for modeling projectile
motion. The function below models the path of a calculus book being
thrown from an initial height of $1\unit{m}$ at an initial velocity of
$5\unit{m}/{s}$ at a $45^\circ$ angle:
\[
\vec{p}(t)=\vector{\frac{5t}{\sqrt{2}}, 1+ \frac{5t}{\sqrt{2}}-5t^2}
\]
For your viewing pleasure here is a plot:
  \begin{image}
    \begin{tikzpicture}
      \begin{axis}%
        [
	xmin=-.5,xmax=4,
        ymin=-.5,ymax=2,
        xlabel=$x$,ylabel=$y$,
        axis lines=center,
        every axis y label/.style={at=(current axis.above origin),anchor=south},
        every axis x label/.style={at=(current axis.right of origin),anchor=west},
        clip=false,
	grid =major,
        width=10cm,
        height=6.7cm,
        xtick={0,1,...,4},
        ytick={0,1,...,2},
      ]
      \addplot[penColor,smooth,ultra thick,domain=0:1] (
       {5*x/sqrt(2)},{1+ 5*x/sqrt(2)-5*x^2}
      );
      \end{axis}
    \end{tikzpicture}
  \end{image}

\begin{problem}
  Pencil-in some tangent vectors for $\vec{p}$ above. 
\end{problem}



\begin{problem}
  Someone has plotted $\vec{p}'$ below:
  \begin{image}
    \begin{tikzpicture}
      \begin{axis}%
        [
	xmin=-.5,xmax=4,
        ymin=-.5,ymax=2,
        xlabel=$x$,ylabel=$y$,
        axis lines=center,
        every axis y label/.style={at=(current axis.above origin),anchor=south},
        every axis x label/.style={at=(current axis.right of origin),anchor=west},
        clip=false,
	grid =major,
        width=10cm,
        height=6.7cm,
        xtick={0,1,...,4},
        ytick={0,1,...,2},
      ]
      \addplot[penColor,smooth,ultra thick,domain=.1:.4] (
              {5/sqrt(2)},{5/sqrt(2)-10*x}
      );
      \end{axis}
    \end{tikzpicture}
  \end{image}
  Make sense of this plot (note, this is not a complete plot). Explain
  what is going on to someone else.
\end{problem}

\begin{problem}
  If we were to plot $\vec{p}''$ what would it look like?
\end{problem}

\begin{problem}
  Rather than plotting $\vec{p}''$, what should you do?
\end{problem}


\section{The moral of the story}


The moral of the story is this: When studying, functions from $\R$ to
$\R$, it makes a lot of sense to plot their derivatives. When dealing
with vector-valued functions, plotting their derivatives might not be
the best idea. Instead you should be plotting tangent vectors.


\end{document}
