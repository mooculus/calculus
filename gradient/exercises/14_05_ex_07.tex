\documentclass{ximera}

\newcommand{\RR}{\mathbb R}
\renewcommand{\d}{\,d}
\newcommand{\dd}[2][]{\frac{d #1}{d #2}}
\renewcommand{\l}{\ell}
\newcommand{\ddx}{\frac{d}{dx}}
\newcommand{\dfn}{\textbf}
\newcommand{\eval}[1]{\bigg[ #1 \bigg]}


\author{David Guichard \and Neal Koblitz \and H. Jerome Keisler \and Albert Scheller \and Barry Balof \and Mike Wills \and Matthew Carr \and Bart Snapp}
\license{CC-By-SA-NC}

\acknowledgement{https://www.whitman.edu/mathematics/multivariable/}

\begin{document}
\begin{exercise}
A plane perpendicular to the $(x,y)$-plane contains the point
$(2,1,8)$ on the paraboloid $z=x^2+4y^2$. The cross-section of the
paraboloid created by this plane has a slope $0$ at this point. Find
an equation of the plane.

\begin{prompt}
\[
x\cdot \answer{1} + y \cdot \answer{2} + z \cdot \answer{0} = 4
\]
\end{prompt}

\begin{hint}
  Let $\vec{n}=\vector{a,b,c}$ be the normal vector for the plane in
  question.
\end{hint}

\begin{hint}
  We know that $c=0$ because the plane is perpendicular to the
  $(x,y)$-plane.
\end{hint}

\begin{hint}
  The paraboloid $z=x^2+4y^2$ can be thought of as the level surface
  \[
  0 = x^2+4y^2-z
  \]
  of some new function of several variables:
  \[
  G(x,y,z) = x^2+4y^2-z
  \]
\end{hint}

\begin{hint}
  The gradient vector is normal to level surfaces.
\end{hint}


\begin{hint}
  $\grad G = \vector{2x,8y,-1}$
\end{hint}

\begin{hint}
  $\grad G(2,1,8) = \vector{4,8,-1}$
\end{hint}

\begin{hint}
  Since the cross-section of the paraboloid created by this plane has
  a slope $0$ at $(2,1,8)$, the tangent vector of the cross-section is
  parallel to $\vec{n}\cross\grad G(2,1,8)$.
\end{hint}

\begin{hint}
  $\vec{n}\cross\grad(2,1,8) = \vector{-b,a,8a-4b}$
\end{hint}

\begin{hint}
  Moreover, since the slope is $0$, we see
  \[
  8a-4b = 0.
  \]
\end{hint}

\begin{hint}
  So $2a=b$.
\end{hint}

\begin{hint}
  This means that $\vec{n}$ is parallel to $\vector{a,2a,0}$.
\end{hint}

\begin{hint}
  Set $\vec{n}= \vector{1,2,0}$. Now since we know a point on the
  plane, we can find the plane.
\end{hint}


\end{exercise}
\end{document}
