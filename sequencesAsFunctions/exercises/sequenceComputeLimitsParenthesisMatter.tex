\documentclass{ximera}

\newcommand{\RR}{\mathbb R}
\renewcommand{\d}{\,d}
\newcommand{\dd}[2][]{\frac{d #1}{d #2}}
\renewcommand{\l}{\ell}
\newcommand{\ddx}{\frac{d}{dx}}
\newcommand{\dfn}{\textbf}
\newcommand{\eval}[1]{\bigg[ #1 \bigg]}


\author{Jim Talamo}
\license{Creative Commons 3.0 By-bC}


\outcome{}


\begin{document}
\begin{exercise}
 We study two sequences that are closely related. 
 
Consider$\{a_n \}_{n=1}$ where $a_n = \ln\left(n^{1/n}\right)$.  Then:
\[
\lim_{n \to \infty} a_n = \answer{0}
\]

\begin{hint}
Using the properties of logarithms, we can write:

\[
a_n= \left(\answer{\frac{1}{n}}\right)ln(n)
\]
\end{hint}

Now, Consider$\{b_n \}_{n=1}$ where $b_n = (\ln(n))^{1/n}$.  Then:
\[
\lim_{n \to \infty} a_n = \answer{1}
\]

\begin{hint}
To convert the exponential indeterminate form into one we can handle, we set $L = \lim_{n \to \infty} b_n$ first take the natural logarithm of each side:


\[
\ln L= \ln \left(\lim_{n \to \infty} (\ln(n))^{1/n}\right) = \lim_{n \to \infty}  \ln \left((\ln(n))^{1/n}\right)
\]
\begin{question}

Using the properties of logarithms:

\[
\ln L= \lim_{n \to \infty} \answer{\frac{1}{n}} \ln(\ln(n))
\]
\begin{question}

Manipulate this into a form for which we can use L'Hopital's rule:

\[
\ln L = \lim_{n \to \infty} \frac{\answer{\ln(\ln(n))}}{\answer{n}}
\]

\begin{question}
Evaluating this limit gives:

\[
\ln L = \answer{0}
\]

Hence, $L= \lim_{n \to \infty} b_n = \answer{1}$.

\end{question}
\end{question}
\end{question}
\end{hint}


\end{exercise}
\end{document}
