\documentclass{ximera}

\newcommand{\RR}{\mathbb R}
\renewcommand{\d}{\,d}
\newcommand{\dd}[2][]{\frac{d #1}{d #2}}
\renewcommand{\l}{\ell}
\newcommand{\ddx}{\frac{d}{dx}}
\newcommand{\dfn}{\textbf}
\newcommand{\eval}[1]{\bigg[ #1 \bigg]}


\title[Refresh:]{Basic material}

\begin{document}
\begin{abstract}
  We review basic material for this course. 
\end{abstract}
\maketitle

\begin{problem}
  Compute
  \[
  9-16\cdot 0 + 8\div 2 = \answer{13}
  \]
\end{problem}





\begin{problem}
  Find the missing quantities that complete the square:
  \[
   x^2 + 20x +\answer{100} = (\answer{x+10})^2 
   \]
\end{problem}


\begin{problem}
  Find the missing quantities that complete the square:
  \[
  x^2-12x + \answer{36} = (x-6)^2
  \]
\end{problem}


\begin{problem}
  Use a change of variables to compute the following indefinite
  integral.
  \[
  \int \frac{4x^2}{\sqrt{5 - 8x^3}} \d x
  \]
  What is the best choice of $u$ for the change of variables?
  \[
  u = \answer{5 - 8x^3}
  \]
  Find $\d u$.
  \[
  \d u = \answer{-24x^2} \d x
  \]
  Rewrite the given integral using this change of variables:
  
  \[
  \int \frac{4x^2}{\sqrt{5 - 8x^3}} \d x = \int \answer{\frac{-1}{6} u^{-1/2}} \d u
  \]
  Find the indefinite integral:
  \[
  \int \frac{4x^2}{\sqrt{5 - 8x^3}} \d x = \answer{\frac{-1}{3}(5-8x^3)^{1/2} + C}
  \]
  (Use $C$ as the arbitrary constant.)
\end{problem}


\begin{problem}
  Find the indefinite integral.
  \[
    \int \frac{x}{\sqrt{x - 1}} \d x = \answer{\frac{2}{3}(x-1)^{3/2} + 2(x - 1)^{1/2} + C}
  \]  
  (Use $C$ as the arbitrary constant.
\end{problem}


\begin{problem}
  Use a change of variable to evaluate the following definite integral.
  \[
    \int_{-\pi/2}^{-\pi/4} -3 \cot(x) \csc^2(x) \d x = \answer{3/2} \text{(Type an exact answer.)}
  \]
\end{problem}

\begin{problem}
  Find a function $f$ such that $\int f(x) \d x = \cos(x^2) + C$.
  \[
  f(x) = \answer{-2 x \sin(x^2)}
  \]
\end{problem}

\begin{problem}
  A student claims that $\int \frac{1}{1-x^2} \d x = \ln(1 - x^2) + C$.
  Determine if the student is correct.
  If the student is incorrect, choose the best explanation that follows.
  \begin{multipleChoice}
    \choice{The student is correct.}
    \choice[correct]{The student is incorrect.}
  \end{multipleChoice}
  \begin{problem}
    If the student's claim had been correct, then
    \[
    \dd{x} \ln(1 - x^2) = \answer{\frac{1}{1-x^2}}.
    \]
    However,
    \[
    \dd{x} \ln(1 - x^2) = \answer{\frac{-2x}{1 - x^2}}
    \]
  \end{problem}
\end{problem}


\begin{problem}
  A student is asked to compute $\int x \sqrt{1 - x^2} \d x$ on a midterm and provides the following solution:
  \[
    \int x \sqrt{1 - x^2} \d x= \int x(1 -x) \d x = \int (x - x^2 )\d x = \frac{1}{2}x^2 - \frac{1}{3}x^3 + C.
  \]
  Determine if the student is correct.
  If the student is not correct, choose the most appropriate response below.
  \begin{multipleChoice}
    \choice{The student is correct.}
    \choice[correct]{The student is incorrect.}
  \end{multipleChoice}
  \begin{problem}
    The mistake was in asserting that $\sqrt{1 - x^2} \ne 1 - x$.  To
    compute this antiderivative, the student should perform a
    $u$-substitution with $u = \answer{1 - x^2}$.  Upon making this
    substitution, the correct antiderivative of $y = x \sqrt{1 - x^2}$
    is $\answer{(-1/3)(1 - x^2)^{3/2}} + C$.
  \end{problem}
\end{problem}



\begin{problem}
  A student is asked to compute $\int 3x^2 \cos(x) \d x$ on a midterm and provides the following solution:
  \[
    \int 3x^2 \cos(x) \d x = x^3 \sin(x) + C.
  \]
  Determine if the student is correct.
  If the student is not correct, choose the most appropriate response below.
  \begin{multipleChoice}
    \choice{The student is correct}
    \choice{The student is incorrect; $\int 3x^2 \cos(x) \d x = -x^3 \sin(x) + C$.}
    \choice[correct]{The student is incorrect.}
  \end{multipleChoice}
  \begin{problem}
    The mistake was in attempting to split the integral over the
    product.  If the student were correct, then using the definition
    of the antiderivative, $\dd{x}(x^3\sin(x) + C) = 3x^2 \cos(x)$.
    However,
    \[
    \dd{x}(x^3 \sin(x) + C) = \answer{3x^2 \sin(x) + x^3 \cos(x)}
    \]
  \end{problem}
  \begin{hint}
    To verify an antiderivative, differentiate and compare.
  \end{hint}
\end{problem}



\begin{problem} Calculate:
\[
\int x^2 +1 \d x = \answer{x^3/3 + x+ C}
\]
\begin{hint}
  Use $C$ for the constant. 
\end{hint}
\end{problem}

\begin{problem} Calculate:
\[
\int x^3 +6 x + 1 \d x = \answer{x^4/4 + 3x^2 + x+ C}
\]
\begin{hint}
  Use $C$ for the constant. 
\end{hint}
\end{problem}


\begin{problem} Calculate:
\[
\int 3x^4 + \frac{x}{2} \d x = \answer{3x^5/5 + x^2/4 + C}
\]
\begin{hint}
  Use $C$ for the constant. 
\end{hint}
\end{problem}



\begin{problem} Calculate:
\[
\int \pi + \sin(x) \d x = \answer{\pi x - \cos(x)+ C}
\]
\begin{hint}
  Use $C$ for the constant. 
\end{hint}
\end{problem}


\begin{problem} Calculate:
\[
\int 1+ 2x + e^x \d x = \answer{x + x^2 + e^x+C}
\]
\begin{hint}
  Use $C$ for the constant. 
\end{hint}
\end{problem}

\begin{problem} Calculate:
\[
\int \frac{5}{x^5} + \frac{4}{x^4} \d x = \answer{\frac{-5}{4 x^4}-\frac{4}{3 x^3}+C}
\]
\begin{hint}
  Use $C$ for the constant. 
\end{hint}
\end{problem}


\begin{problem} Calculate:
\[
\int \frac{7}{x^6} + \frac{x^6}{7} \d x = \answer{\frac{x^7}{49}-\frac{7}{5 x^5}+C}
\]
\begin{hint}
  Use $C$ for the constant. 
\end{hint}
\end{problem}


\begin{problem} Calculate:
\[
\int \cos(x) - 2x^2 \d x = \answer{\sin(x) - 2x^3/3+C}
\]
\begin{hint}
  Use $C$ for the constant. 
\end{hint}
\end{problem}

\begin{problem} Calculate:
\[
\int e^x - 1 - x \d x = \answer{e^x-x-x^2/2+C}
\]
\begin{hint}
  Use $C$ for the constant. 
\end{hint}
\end{problem}



\begin{problem} Calculate:
\[
\int 16 \d u = \answer{16u+C}
\]
\begin{hint}
  Use $C$ for the constant. 
\end{hint}
\end{problem}


\begin{problem} Calculate:
\[
\int  \d \theta = \answer{\theta+C}
\]
\begin{hint}
  Use $C$ for the constant. 
\end{hint}
\end{problem}

\begin{problem} Calculate:
\[
\int \pi \d \theta = \answer{\pi\theta +C}
\]
\begin{hint}
  Use $C$ for the constant. 
\end{hint}
\end{problem}


\begin{problem} Calculate:
\[
\int e \d t = \answer{e t + C}
\]
\begin{hint}
  Use $C$ for the constant. 
\end{hint}
\end{problem}



\begin{problem} Calculate:
\[
\int -1 \d u = \answer{-u+C}
\]
\begin{hint}
  Use $C$ for the constant. 
\end{hint}
\end{problem}


\begin{problem} Calculate:
\[
\int 7 \d z = \answer{7z+C}
\]
\begin{hint}
  Use $C$ for the constant. 
\end{hint}
\end{problem}


\begin{problem} Calculate:
\[
\int 0 \d x = \answer{0+C}
\]
\begin{hint}
  Use $C$ for the constant. 
\end{hint}
\end{problem}

\begin{problem} Calculate:
\[
\int -3 \d \theta = \answer{-3\theta+C}
\]
\begin{hint}
  Use $C$ for the constant. 
\end{hint}
\end{problem}

\begin{problem} Calculate:
\[
\int  \d t = \answer{t+C}
\]
\begin{hint}
  Use $C$ for the constant. 
\end{hint}
\end{problem}



\begin{problem} Calculate:
\[
\int_0^{1/2} \frac{1}{\sqrt{1-x^2}} \d x =\answer{\frac{\pi }{6}}
\]
\end{problem}


\begin{problem} Calculate:
\[
\int_0^1 \frac{1}{x^2 + 1} \d x = \answer{\frac{\pi }{4}}
\]
\end{problem}



\begin{problem} Calculate:
\[
\int_0^1 \frac{1}{x + 1} \d x = \answer{\ln(2)}
\]
\end{problem}


\begin{problem} Calculate:
\[
\int_0^{1/2} \frac{4}{\sqrt{1 - x^2}} \d x = \answer{\frac{2\pi }{3}}
\]
\end{problem}


\begin{problem} Calculate:
\[
\int_0^{1/2} \frac{-1}{\sqrt{1 - x^2}} \d x = \answer{\frac{-\pi }{6}}
\]
\end{problem}


\begin{problem} Calculate:
\[
\int_2^3 \sec^2 \theta  \d \theta = \answer{\sin(1) \sec(2) \sec(3)}
\]
\end{problem}



\begin{problem} Calculate:
\[
\int_2^3 \sec(\theta)\cdot \tan(\theta)  \d \theta = \answer{\sec(3)-\sec(2)}
\]
\end{problem}



\begin{problem} Calculate:
\[
\int_2^3 3\sec^2\theta  \d \theta = \answer{3 \sin (1) \sec (2) \sec (3)}
\]
\end{problem}


\begin{problem} Calculate:
\[
\int_0^1 e^{-x/2} \d x = \answer{2-\frac{2}{\sqrt{e}}}
\]
\end{problem}



\begin{problem} Calculate:
\[
\int_0^1 2e^{4x} \d x = \answer{\frac{1}{2} (e^4-1)}
\]
\end{problem}




\begin{problem} Calculate:
\[
\int_0^3 \cos(3x) \d x = \answer{\frac{\sin (9)}{3}}
\]
\end{problem}


\begin{problem} Calculate:
\[
\int_0^{1/2} \cos(\pi x) \d x = \answer{\frac{1}{\pi }}
\]
\end{problem}




\begin{problem} Calculate:
\[
\int_1^2 3\sin(3x) \d x = \answer{\cos (3)-\cos (6)}
\]
\end{problem}



\begin{problem} Calculate:
\[
\int_1^4 \pi \sin(- x) \d x = \answer{\pi  (\cos (4)-\cos (1))}
\]
\end{problem}



\begin{problem} Calculate:
\[
\int_3^5 \frac{\sin(\pi x)}{\pi} \d x = \answer{0}
\]
\end{problem}



\begin{problem} Calculate:
\[
\int_1^2 \cos(7 x) \d x= \answer{\frac{1}{7} (\sin (14)-\sin (7))}
\]
\end{problem}



\begin{problem} Calculate:
\[
\int_3^4 \frac{\cos(2 x)}{3} \d x=\answer{\frac{1}{6} (\sin (8)-\sin (6))}
\]
\end{problem}


\begin{problem} Calculate:
\[
\int_0^{1/2} \pi \cos(x/\pi) \d x = \answer{\pi ^2 \sin (\frac{1}{2 \pi })}
\]
\end{problem}


\begin{problem} Calculate:
\[
\int_0^1 \frac{\sin(x/\pi)}{\pi} \d x = \answer{1-\cos \left(\frac{1}{\pi }\right)}
\]
\end{problem}



\begin{problem} Calculate:
\[
\int_0^2 \pi\sin(x/\pi) \d x =\answer{2 \pi ^2 \sin ^2\left(\frac{1}{\pi }\right)}
\]
\end{problem}



\begin{problem} Calculate:
\[
\int_0^1 \frac{3}{(2x+1)} \d x = \answer{\frac{3 \ln (3)}{2}}
\]
\end{problem}


\begin{problem} Calculate:
\[
\int_0^1 \frac{1}{5-3x} \d x \answer{\frac{\ln(5/2)}{3}}
\]
\end{problem}

\begin{problem} Calculate:
\[
\int_1^2 \frac{x^3}{1 + x^4} \d x = \answer{\frac{1}{4} \ln(\frac{17}{2})}
\]
\end{problem}



\begin{problem} Calculate:
\[
\int_0^1 \frac{2+ e^{2x}}{4x + e^{2x}} \d x = \answer{\frac{1}{2} \ln(4+e^2)}
\]
\end{problem}



\begin{problem} Calculate:
\[
\int_0^1 \frac{1}{1 + x} \d x =\answer{\ln(2)}
\]
\end{problem}


\begin{problem} Calculate:
\[
\int_3^6 \frac{x^2+4x^3}{x^3+3x^4} \d x =\answer{\frac{1}{3} \ln(\frac{76}{5})}
\]
\end{problem}



\begin{problem} Calculate:
\[
\int_2^5 \frac{1}{-3-3x} \d x =\answer{\frac{-\ln(2)}{3}}
\]
\end{problem}





\end{document}
