\documentclass{ximera}

\newcommand{\RR}{\mathbb R}
\renewcommand{\d}{\,d}
\newcommand{\dd}[2][]{\frac{d #1}{d #2}}
\renewcommand{\l}{\ell}
\newcommand{\ddx}{\frac{d}{dx}}
\newcommand{\dfn}{\textbf}
\newcommand{\eval}[1]{\bigg[ #1 \bigg]}


%\outcome{Describe the goals of optimization problems generally.}
\outcome{Find all local maximums and minimums using the First and Second Derivative tests.}
%\outcome{Identify when we can find an absolute maximum or minimum on an open interval.}
%\outcome{Contrast optimization on open and closed intervals.}
\outcome{Describe the objective function and constraints in a given optimization problem.}
\outcome{Solve optimization problems by finding the appropriate extreme values.}

\author{Nela Lakos \and Kyle Parsons}

\begin{document}
\begin{exercise}

Consider the graph of the parabola $y=2x-x^2$ and the line $y=x$ on the interval $[0,1]$.  A rectangle $PQRS$, with sides parallel to the axes, is constructed so that the two left corners lie on the $y$-axis, the upper right corner lies on the parabola $y=2x-x^2$, and the lower right corner lies on the line $y=x$.

\begin{image}
 \begin{tikzpicture}
    \begin{axis}[
        xmin=-0.1,xmax=1.1,ymin=-0.1,ymax=1.1,
        clip=true,
        unit vector ratio*=1 1 1,
        axis lines=center,
        grid = both,
        minor tick num=3,
        ytick={-21,-20,...,20},
        xtick={-21,-20,...,20},
        xlabel=$x$, ylabel=$y$,
        y tick label style={anchor=west},
        every axis y label/.style={at=(current axis.above origin),anchor=south},
        every axis x label/.style={at=(current axis.right of origin),anchor=west},
      ]
      \addplot[very thick,penColor,domain=-0.1:1.1] plot{2*x-x^2};
      \addplot[very thick,penColor,domain=-0.1:1.1] plot{x};
      
	  \pgfmathsetmacro{\x}{0.43}     
      
	  \coordinate (P) at (axis cs:0,\x);
	  \coordinate (Q) at (axis cs:\x,\x);
	  \coordinate (R) at (axis cs:\x,{2*\x-\x^2});
	  \coordinate (S) at (axis cs:0,{2*\x-\x^2});
	  \coordinate (x) at (axis cs:\x,0);
	  
	  \draw[thick,dashed] (Q) -- (x);	  
	  
	  \filldraw[thick,penColor2,fill opacity=0.3] (P) -- (Q) -- (R) -- (S) -- cycle;
	  
	  \node at (P) [penColor2,yshift=-0.5pt] {\textbullet};
	  \node at (Q) [penColor2,yshift=-0.5pt] {\textbullet};      
	  \node at (R) [penColor2,yshift=-0.5pt] {\textbullet};      
	  \node at (S) [penColor2,yshift=-0.5pt] {\textbullet};      
	  \node at (x) [yshift=-0.5pt] {\textbullet};      
      
	  \node at (P) [left] {$P$};      
	  \node at (Q) [below right] {$Q$};
	  \node at (R) [above left] {$R$};
	  \node at (S) [left] {$S$};      
	  \node at (x) [below] {$x$};
      \end{axis}`
  \end{tikzpicture} 
\end{image}

In terms of $x$, the points $Q$ and $R$ are
\[
Q = \left(x,\answer{x}\right)\text{ and } R = \left(x,\answer{2x-x^2}\right).
\]

The height of the rectangle $PQRS$, in terms of $x$, is
\[
h(x) = \answer{x-x^2}.
\]

The area of the rectangle $PQRS$, in terms of $x$, is
\[
A(x) = \answer{x^2-x^3},
\]
and the domain of this function is
\[
\left[\answer{0},\answer{1}\right].
\]

The value of $x$ that maximizes the area of $PQRS$ is 
\[
x = \answer{\frac{2}{3}}.
\]

\end{exercise}
\end{document}