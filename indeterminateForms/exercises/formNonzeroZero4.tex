\documentclass{ximera}

\newcommand{\RR}{\mathbb R}
\renewcommand{\d}{\,d}
\newcommand{\dd}[2][]{\frac{d #1}{d #2}}
\renewcommand{\l}{\ell}
\newcommand{\ddx}{\frac{d}{dx}}
\newcommand{\dfn}{\textbf}
\newcommand{\eval}[1]{\bigg[ #1 \bigg]}


\outcome{Understand what is meant by the form of a limit.}
\outcome{Calculate limits of the form zero over zero.}
\outcome{Identify determinate and indeterminate forms.}
\outcome{Distinguish between determinate and indeterminate forms.}

\author{Nela Lakos}

\begin{document}




\begin{exercise}
 Evaluate the limit. Possible answers include a number, $+\infty$, $-\infty$ and $DNE$.
\[
\lim_{x\to3^{+}}\frac{x^{2}-4x+1}{3-x}=\answer{+\infty}
\]
 Justify your answer  by choosing all correct statements.
 \begin{selectAll} 
\choice{The numerator is negative and the denominator is positive and approaching zero.}
\choice{The numerator is positive and the denominator is positive and approaching zero.}
\choice{The numerator is positive and the denominator is negative and approaching zero.}
\choice[correct]{The numerator is negative and the denominator is negative and approaching zero.}
\end{selectAll}
\noindent\rule[0.5ex]{\linewidth}{0.2pt}

\begin{exercise}
 Evaluate the limit. Possible answers include a number, $+\infty$, $-\infty$ and $DNE$.
\[
\lim_{x\to3^{-}}\frac{x^{2}-4x+1}{3-x}=\answer{-\infty}
\]
 Justify your answer  by choosing all correct statements.
 \begin{selectAll} 
\choice[correct]{The numerator is negative and the denominator is positive and approaching zero.}
\choice{The numerator is positive and the denominator is positive and approaching zero.}
\choice{The numerator is positive and the denominator is negative and approaching zero.}
\choice{The numerator is negative and the denominator is negative and approaching zero.}
\end{selectAll}
\noindent\rule[0.5ex]{\linewidth}{0.2pt}
\begin{exercise}
 Evaluate the limit. Possible answers include a number, $+\infty$, $-\infty$ and $DNE$.
\[
\lim_{x\to3}\frac{x^{2}-4x+1}{3-x}=\answer{DNE}
\]
  Justify your answer  by choosing the correct statement.
  \begin{selectAll} 
\choice[correct]{The limit from the left is  not equal to the limit from the right.}
\choice{The limit from the left is  equal to the limit from the right.}
\end{selectAll}
\noindent\rule[0.5ex]{\linewidth}{0.2pt}
\end{exercise}
\end{exercise}
\end{exercise}
\end{document}