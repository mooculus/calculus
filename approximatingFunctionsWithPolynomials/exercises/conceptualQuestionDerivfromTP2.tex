\documentclass{ximera}

\newcommand{\RR}{\mathbb R}
\renewcommand{\d}{\,d}
\newcommand{\dd}[2][]{\frac{d #1}{d #2}}
\renewcommand{\l}{\ell}
\newcommand{\ddx}{\frac{d}{dx}}
\newcommand{\dfn}{\textbf}
\newcommand{\eval}[1]{\bigg[ #1 \bigg]}


\author{Jim Talamo}
\license{Creative Commons 3.0 By-bC}


\outcome{Understand the relationship between a Taylor Polynomials and the function it represents}


\begin{document}
\begin{exercise}
Suppose that $f(x)$ is an infinitely differentiable function at $x=17$ and that the \emph{fifth} degree Taylor polynomial of $f(x)$ centered at $x=-4$ is:

\[
p_5(x) = 1+3(x+4)-2(x+4)^2+4(x+4)^3+8(x+4)^5
\]

\begin{exercise}
Is there enough information to determine $f''(-4)$?

\begin{multipleChoice}
\choice[correct]{Yes}
\choice{No}
\end{multipleChoice}
In fact, $f''(-4) = \answer{-4}$.

\begin{hint}
Since the coefficients of the Taylor Polynomial and the values of the derivatives of the function are related via the formula:

\[
a_k = \frac{f^{(k)}(c)}{k!}
\]
We have that $c=\answer{-4}$, $a_3 = \answer{7}$ and that $f'''(0)$ can be found using this formula with $k=
\answer{2}$.   
\end{hint}

\end{exercise}

%%%%%%%%%%%%%%%%%%%%%%%%

\begin{exercise}
Is there enough information to determine $f^{(4)}(-4)$?

\begin{multipleChoice}
\choice[correct]{Yes}
\choice{No}
\end{multipleChoice}
In fact, $f''(-4) = \answer{0}$.

\begin{hint}
Since the coefficients of the Taylor Polynomial and the values of the derivatives of the function are related via the formula:

\[
a_k = \frac{f^{(k)}(c)}{k!}
\]
We have that $c=\answer{-4}$, $a_4 = \answer{0}$ and that $f^{(4)}(0)$ can be found using this formula with $k=
\answer{4}$.   
\end{hint}

\end{exercise}

%%%%%%%%%%%%%%%%%%%%%%%%
\begin{exercise}
Is there enough information to determine $f^{(6)}(-4)$?

\begin{multipleChoice}
\choice{Yes}
\choice[correct]{No}
\end{multipleChoice}


\begin{hint}
Since the coefficients of the Taylor Polynomial and the values of the derivatives of the function are related via the formula:

\[
a_k = \frac{f^{(k)}(c)}{k!}
\]
We have that $c=\answer{-4}$ and would need information about $a_6$, which would be the coefficient of $(x+4)^6$.  We do not have this information because we only have a \emph{fifth} degree Taylor Polynomial!
\end{hint}

\end{exercise}


%%%%%%%%%%%%%%%%%%%%%%%%
Which of the following could be computed at $x=17$ by using the Taylor Polynomial?

\begin{selectAll}
\choice[correct]{$f(-4)$}
\choice[correct]{$f'(-4)$}
\choice[correct]{$f''(-4)$}
\choice[correct]{$f'''(-4)$}
\choice[correct]{$f^{(4)}(-4)$}
\choice[correct]{$f^{(5)}(-4)$}
\choice{$f^{(6)}(-4)$}
\end{selectAll}

%%%%%%%%%%%%%%%%%%%%%%%%

\begin{exercise}
Is there enough information to determine $f''(-4.1)$?

\begin{multipleChoice}
\choice{Yes}
\choice[correct]{No}
\end{multipleChoice}

\begin{exercise}
Since the coefficients of the Taylor Polynomial and the values of the derivatives of the function are related via the formula:

\[
a_k = \frac{f^{(k)}(c)}{k!}
\]
We have that $c=\answer{-4}$ so we only can find an exact value of $f''(-4)$ by using the Taylor Polynomial!  

(It's certainly true that we could \emph{approximate} $f''(8)$ by using this Taylor Polynomial, though without more information, it's impossible to determine how good of an approximation this would be)

\end{exercise}
\end{exercise}


\end{exercise}
\end{document}
