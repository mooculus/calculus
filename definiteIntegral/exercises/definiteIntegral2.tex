\documentclass{ximera}

\newcommand{\RR}{\mathbb R}
\renewcommand{\d}{\,d}
\newcommand{\dd}[2][]{\frac{d #1}{d #2}}
\renewcommand{\l}{\ell}
\newcommand{\ddx}{\frac{d}{dx}}
\newcommand{\dfn}{\textbf}
\newcommand{\eval}[1]{\bigg[ #1 \bigg]}



\outcome{Use integral notation for both antiderivatives and definite integrals.}
%\outcome{Compute definite integrals using geometry.}
\outcome{Compute definite integrals using the properties of integrals.}
%\outcome{Justify the properties of definite integrals using algebra or geometry.}
%\outcome{Understand how Riemann sums are used to find exact area.}
%\outcome{Define net area.}
%\outcome{Approximate net area.}
\outcome{Split the area under a curve into several pieces to aid with calculations.}
%\outcome{Use symmetry to calculate definite integrals.}
%\outcome{Explain geometrically why symmetry of a function simplifies calculation of some definite integrals.}

\author{Nela Lakos \and Kyle Parsons}

\begin{document}
\begin{exercise}

Consider fixed numbers $a$ and $b$ with $a<b$.

 Given that $\int_a^b f(x) \d x = -7$ and $\int_a^b g(x) \d x = 5$, compute the following integrals or write \verb|NEI| if there is not enough information to answer the question.
\begin{hint}
Use Properties of  Definite Integrals!
\end{hint}
\begin{align*}
\int_b^a f(x) \d x &= \answer{7}\\
\int_a^b (3g(x) - 4f(x)) \d x &= \answer{43}\\
\int_b^a f(x)g(x) \d x &= \answer{NEI}\\
\int_a^a f(x)g(x) \d x &= \answer{0}\\
\int_a^b \left|f(x)\right| \d x &= \answer{NEI}\\
\int_a^{\frac{a+b}{2}} f(x) \d x + \int_{\frac{a+b}{2}}^b f(x) \d x &= \answer{-7}\\
\int_a^{\frac{a+b}{2}} f(x) \d x + \int_{\frac{a+b}{2}}^a f(x) \d x &= \answer{0}\\
\end{align*}

\end{exercise}
\end{document}