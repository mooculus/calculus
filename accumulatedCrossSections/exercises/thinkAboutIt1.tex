\documentclass{ximera}

\newcommand{\RR}{\mathbb R}
\renewcommand{\d}{\,d}
\newcommand{\dd}[2][]{\frac{d #1}{d #2}}
\renewcommand{\l}{\ell}
\newcommand{\ddx}{\frac{d}{dx}}
\newcommand{\dfn}{\textbf}
\newcommand{\eval}[1]{\bigg[ #1 \bigg]}


\author{Bart Snapp}
%Example from the original text
\license{Creative Commons 3.0 By-NC}


\outcome{Set up an integral or sum of integrals that gives the volume of a solid with known cross sections}

\begin{document}
\begin{exercise}

The following exercise applies the thought process behind building volumes from slabs with known cross section areas.

Consider a pyramid with a square base that is $20$ meters long at side of its base and whose height is $20$ meters:

\begin{image}
  \begin{tikzpicture}
    \begin{axis}[
          xmin =-4,xmax=7,ymax=4,ymin=-4,
          axis lines=center, xlabel=$x$, ylabel=$y$,
          every axis y label/.style={at=(current axis.above origin),anchor=south},
          every axis x label/.style={at=(current axis.right of origin),anchor=west},
          width=5in,
          axis on top,
          xtick={0,6}, xticklabels={$0$, $20$},
          ytick={0,3},yticklabels={$0$,$20$},
            clip=false,
      ]
      \addplot [draw=penColor, fill = fill1, very thick] plot coordinates {(-3,-3) (3,-3) (6,0) (1.5, 3) (-3,-3)};
      \addplot [draw=penColor, very thick] plot coordinates {(-3,-3) (0,0)};
      \addplot [draw=penColor, very thick] plot coordinates {(6,0) (0,0)};
      \addplot [draw=penColor, very thick] plot coordinates {(1.5,3) (3,-3)};
      \addplot [draw=penColor, very thick] plot coordinates {(1.5,3) (0,0)};

      \addplot [->] plot coordinates {(0,0) (-4,-4)};
      \node[anchor=north east] at (axis cs:-4,-4) {$z$};       
    \end{axis}
  \end{tikzpicture}
\end{image}

To solve this problem, we must ``sum up'' infinitely many
``infinitesimal'' slabs which are parallel to the base of the pyramid
to obtain the volume.
\begin{image}
  \begin{tikzpicture}
    \begin{axis}[
          xmin =-4,xmax=7,ymax=4,ymin=-4,
          axis lines=center, xlabel=$x$, ylabel=$y$,
          every axis y label/.style={at=(current axis.above origin),anchor=south},
          every axis x label/.style={at=(current axis.right of origin),anchor=west},
          axis on top,
          width=5in,
          xtick={0,6}, xticklabels={$0$, $20$},
          ytick={0,3},yticklabels={$0$,$20$},
            clip=false,
      ]
      \addplot [draw=penColor, thick] plot coordinates {(-3,-3) (3,-3) (6,0) (1.5, 3) (-3,-3)};
            
      \addplot [draw=penColor, thick] plot coordinates {(-3,-3) (0,0)};
      \addplot [draw=penColor, thick] plot coordinates {(6,0) (0,0)};
      \addplot [draw=penColor, thick] plot coordinates {(1.5,3) (3,-3)};
      \addplot [draw=penColor, thick] plot coordinates {(1.5,3) (0,0)};

      %% slab
      \addplot [draw=penColor, fill=fill1,very thick] plot coordinates {(3,2) (1,2) (0,1) (2, 1) (3,2)};
      \addplot [draw=penColor, fill=fill1,very thick] plot coordinates {(0,.8) (0,1) (2,1) (2, .8) (0,.8)};
      \addplot [draw=penColor, fill=fill1,very thick] plot coordinates {(2,1) (2, .8) (3,1.8) (3,2) (2,1)};

      %\addplot [draw=penColor, fill=fill1,very thick] plot coordinates {(3,1.8) (1,1.8) (0,.8) (2, .8) (3,1.8)};
      %\addplot [draw=penColor, fill=fill1,very thick] plot coordinates {(3,2) (1,2) (0,1) (2, 1) (3,2)};

      \addplot [draw=penColor, thick] plot coordinates {(1.5,3) (3,-3)};

      \draw[decoration={brace,mirror,raise=.1cm},decorate,thin] (axis cs:0,.8)--(axis cs:2,.8);
      \draw[decoration={brace,raise=.1cm},decorate,thin] (axis cs:3.1,2.05)--(axis cs:3.1,1.75);
      
      \addplot [->] plot coordinates {(0,0) (-4,-4)};
      \node[anchor=north east] at (axis cs:-4,-4) {$z$};

      \node at (axis cs:3.6,1.9) {$\Delta y$};
      \node at (axis cs:1,.4) {$L(y)$};       
    \end{axis}
  \end{tikzpicture}
\end{image}
The ``height'' of each slab will be $\Delta y$, and we'll let $L(y)$ be the width:
\[
L(y) = \answer[given]{20-y}
\]
\begin{hint}
  Since $L(y)$ is a linear function of $y$, and $L(0) = 20$, and
  $L(20) = 0$ we see $L(y) = 20-y$ by the point-slope formula.
\end{hint}
For each slab, the volume at height $y$ is
\begin{align*}
  \d V &= \mathrm{length} \cdot \mathrm{width}\cdot \mathrm{height}\\
  &= L(y)\cdot L(y)\cdot  \Delta y
\end{align*}
so the total approximate volume is given by:

\begin{align*}
 \mathrm{Volume} &= \sum L(y)^2 \Delta y \\
  \mathrm{Volume} &= \int_0^{20} L(y)^2 \d y \\
  &= \int_0^{20} (\answer[given]{20-y})^2 \d y
\end{align*}

To obtain the \emph{exact} volume, we simultaneously must shrink the heights and add up the contribution from each slab.  This simultaneous limiting process is done by the definite integral:

\[
V = \int_{0}^{20} L(y)^2 \d y
\]

Using the expression for $L(y)$ earlier gives:

\[
\int_0^{20} (\answer{20-y})^2 \d y
\]

Evaluating the integral shows that the volume is $\answer{8000/3}$ cubic meters.

\begin{hint}

Making the substitution $g = 20-y $, we have $\d g = - \d y$, $g$ going from $20$ to $0$, and 
	\begin{align*}
	\int_0^{20} (20-y)^2 \d y &= -\int_{\answer[given]{20}}^{\answer[given]{0}} g^2 \d g\\
		&= \int_0^{20} g^2 \d g\\
		&= \eval{\answer[given]{g^3/3}}_0^{20}\\
		&= \frac{20^3}{3}.
	\end{align*}

\end{hint}

As you may know from geometry, the volume of a pyramid is
$(1/3)(\text{height})(\text{area of base})=(1/3)(20)(400)$, which
agrees with our answer.


\end{exercise}
\end{document}