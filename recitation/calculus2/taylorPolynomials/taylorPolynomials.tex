\documentclass[noauthor]{ximera}
%handout:  for handout version with no solutions or instructor notes
%handout,instructornotes:  for instructor version with just problems and notes, no solutions
%noinstructornotes:  shows only problem and solutions

%% handout
%% space
%% newpage
%% numbers
%% nooutcomes

%I added the commands here so that I would't have to keep looking them up
%\newcommand{\RR}{\mathbb R}
%\renewcommand{\d}{\,d}
%\newcommand{\dd}[2][]{\frac{d #1}{d #2}}
%\renewcommand{\l}{\ell}
%\newcommand{\ddx}{\frac{d}{dx}}
%\everymath{\displaystyle}
%\newcommand{\dfn}{\textbf}
%\newcommand{\eval}[1]{\bigg[ #1 \bigg]}

%\begin{image}
%\includegraphics[trim= 170 420 250 180]{Figure1.pdf}
%\end{image}

%add a ``.'' below when used in a specific directory.

\newcommand{\RR}{\mathbb R}
\renewcommand{\d}{\,d}
\newcommand{\dd}[2][]{\frac{d #1}{d #2}}
\renewcommand{\l}{\ell}
\newcommand{\ddx}{\frac{d}{dx}}
\newcommand{\dfn}{\textbf}
\newcommand{\eval}[1]{\bigg[ #1 \bigg]}


\author{Jim Talamo}

\outcome{Find Taylor Polynomials.}
\outcome{Understand the relationship between the derivatives of a function and the coefficients of its Taylor Polynomial.}

\title{Taylor Polynomials}

\begin{document}
\begin{abstract}
\end{abstract}
\maketitle

\vspace{-0.9in}

\section{Discussion Questions}

\begin{problem} 
Suppose that $f(x)$ is a function for which $f(-2)=3$, $f'(-2) =-1$, $f''(-2)=3$, and $f'''(-2)=6$.  What is the second degree Taylor polynomial of $f(x)$ centered at $x=-2$?

\begin{freeResponse}
The second degree Taylor polynomial is 
$$
p_2(x) = f(-2) + f'(-2)(x+2) + \frac{f''(-2)}{2!}(x+2)^2 = 3 - (x+2) + \frac{3}{2}(x+2)^2.
$$
\end{freeResponse}
\end{problem}


\begin{problem} 
Suppose that $f(x)$ is an infinitely differentiable function for which $f''(0)=8$.  Could $p_3(x) = 1-4x+4x^2-x^3$ be the third degree Taylor polynomial of $f(x)$ centered at $x=0$?

\begin{freeResponse}
The third degree Taylor polynomial should satisfy $p_3''(0) = f''(0)$. Let us check:
$$
p_3''(x) = \frac{d}{dx} \left(-4 + 8x - 3x^2\right) = 8 - 6x,
$$
whence $p_3''(0)=8 = f''(0)$. It is possible that $p_3$ is the third degree Taylor polynomial of $f$. Indeed, $f(x)=p_3(x)$ provides an example.
\end{freeResponse}
\end{problem}


\begin{problem} 
The function $p_3(x) = 8x+4x^2-12x^3$ is the third degree Taylor polynomial for $f(x)$ centered at $x=0$.  

\begin{itemize}
\item[I.] Give the second degree Taylor polynomial for $f(x)$ centered at $x=0$.
\item[II.] A student claims that there is insufficient information given to find $f'''(0)$.  Is this student correct?  If not, what is $f'''(0)$?
\item[III.] A student claims that since $p_3'(1)=-20$, then $f'(1)=-20$.  Explain whether the student is correct. 
\item[IV.] A student claims that since $p_3^{(4)}(0) =0$, it must be true that $f^{(4)}(0) =0$.  Explain whether the student is correct.  
\end{itemize}

\begin{freeResponse}
I. The second degree Taylor polynomial is 
$$
p_2(x) = 8x+4x^2.
$$

II. By definition,
$$
p_3(x) = f(0) + f'(0)x + \frac{f''(0)}{2!}x^2 + \frac{f'''(0)}{3!} x^3,
$$
so the student is not correct. We have 
$$
\frac{f'''(0)}{3!} = -12,
$$
so that $f'''(0) = -12 \cdot 3! = -72$.

III. The values of $p_3^{(k)}(x)$, $k=0,1,2,3$, are only guaranteed to agree with those of $f^{(k)}(x)$ at $x=0$, so the student is not correct.

IV. The values of $p^{(k)}(0)$ are only guaranteed to agree with $f^{(k)}(0)$ for $k=0,1,2,3$, so the student is not correct.
\end{freeResponse}
\end{problem}


\section{Group Work}
%%%%%%%%%%%%%%%%%%%%%%%%%%%%%%%%


\begin{problem} 
Consider the function $f(x)=\sqrt{3-2x}$.

\begin{itemize}
\item Find the first, second, and third degree Taylor polynomial for $f(x)=\sqrt{3-2x}$ centered at $x=1$.
\item Use the Taylor polynomials you found above to approximate $\sqrt{2}$.  Given that $\sqrt{2} = 1.4142$ to 4 decimal places, which polynomial provides the best approximation?
\end{itemize}
\end{problem}

\begin{freeResponse}
I. We first calculate the necessary derivatives and evaluate them
\begin{center}
\begin{tabular}{|c|c|c|}
\hline
$k$ & $f^{(k)}(x)$ & $f^{(k)}(1)$ \\
\hline
$0$ & $(3-2x)^{1/2}$ & $1$ \\
\hline
$1$ & $-(3-2x)^{-1/2}$ & $-1$ \\
\hline
$2$ & $-(3-2x)^{-3/2}$ & $-1$ \\
\hline
$3$ & $-3(3-2x)^{-5/2}$ & $-3$ \\
\hline
\end{tabular}
\end{center}
The Taylor polynomials are therefore given by
\begin{align*}
p_1(x) &= 1 - (x-1) \\
p_2(x) &= 1 - (x-1) - \frac{1}{2}(x-1)^2  \\
p_3(x) &= 1 - (x-1) - \frac{1}{2}(x-1)^2 - \frac{1}{2} (x-1)^3.
\end{align*}

II. We wish to approximate $\sqrt{2} = f(1/2)$. Using our Taylor polynomials, we get the approximations
\begin{align*}
p_1(1/2) &= 1 - (1/2-1) = 3/2 = 1.5 \\
p_2(1/2) &= 1 - (1/2-1) - \frac{1}{2}(1/2-1)^2 = 1.5 - 1/2 \cdot 1/4 = 1.5 - 1/8 = 1.375. \\
p_3(1/2) &= 1 - (1/2-1) - \frac{1}{2}(1/2-1)^2 - \frac{1}{2} (1/2-1)^3  = 1.375 + 1/16 = 1.4375.
\end{align*}
The degree 3 approximation is the most accurate.
\end{freeResponse}
\end{document}
