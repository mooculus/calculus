\documentclass{ximera}

\newcommand{\RR}{\mathbb R}
\renewcommand{\d}{\,d}
\newcommand{\dd}[2][]{\frac{d #1}{d #2}}
\renewcommand{\l}{\ell}
\newcommand{\ddx}{\frac{d}{dx}}
\newcommand{\dfn}{\textbf}
\newcommand{\eval}[1]{\bigg[ #1 \bigg]}


\author{Jason Miller}
\license{Creative Commons 3.0 By-NC}


\outcome{Recognize the patterns that appear in trigonometric integrals
  and use appropriate substitutions to compute them.}


\begin{document}
\begin{exercise}
Using the substitution $u= \csc(x)$, determine the integral:

\[
\int \cot^{3}(x) \csc^{2}(x) \d x=\answer{- \frac{1}{4}\csc^{4}(x) + \frac{1}{2}\csc^{2}(x) +C }                  
\]         
(Use $C$ for the constant of integration and recall the identity $\cot^2(x) = \csc^2(x)-1$)

Using the substitution $u= \cot(x)$, determine the integral:

\[
\int \cot^{3}(x) \csc^{2}(x) \d x=\answer{-\frac{1}{4} \cot^4(x) +C }                  
\]         
(Use $C$ for the constant of integration)

\begin{multipleChoice}
\choice{These results are inconsistent with each other}
\choice[correct]{While these results look different, we can use trigonometric identities to show they differ by a constant}
\end{multipleChoice}

\begin{exercise}
Indeed, since both techniques work, we know that the antiderivatives must differ by a constant.  We find,

\begin{align*}
-\frac{1}{4} \cot^4(x) & = -\frac{1}{4} \left(\csc^2(x)-1\right)^2 \\
&= -\frac{1}{4}\left( \answer{\csc^4(x) - 2 \csc^2(x) +1}\right)  \textrm{ (expand the above expression) } \\
&= \answer{-\frac{1}{4}} \csc^4(x) + \answer{\frac{1}{2}} \csc^2(x) +\answer{-\frac{1}{4}}  \textrm{ (expand the above expression) } \\
\end{align*}
which differs from the answer to the first part by the constant $\answer{-\frac{1}{4}}$.

\end{exercise}
\end{exercise}

\end{document}
