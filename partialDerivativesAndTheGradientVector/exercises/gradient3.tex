\documentclass{ximera}

\newcommand{\RR}{\mathbb R}
\renewcommand{\d}{\,d}
\newcommand{\dd}[2][]{\frac{d #1}{d #2}}
\renewcommand{\l}{\ell}
\newcommand{\ddx}{\frac{d}{dx}}
\newcommand{\dfn}{\textbf}
\newcommand{\eval}[1]{\bigg[ #1 \bigg]}



\author{Jim Talamo}

\outcome{Calculate the gradient.}
\outcome{Practice mechanics for finding critical points.}
\outcome{Solve a system of equations.}

\begin{document}

\begin{exercise}
Suppose that $F(x,y) =x^2-xy+y^2+4y$.  Then,
\[
\grad{f}(x,y) = \vector{\answer{2x-y},\answer{2y-x+4}}.
\]

\begin{exercise}
How many distinct points $(x,y)$ are there for which $\grad{F}(x,y) =
\vec{0}$?

\begin{multipleChoice}
\choice{none}
\choice[correct]{one}
\choice{two}
\choice{more than two, but finitely many}
\choice{infinitely many}
\end{multipleChoice}

\begin{hint}
If $\grad{F}(x,y) = \vec{0}$, we must have that the $x$ and $y$ components are zero simultaneously.  We thus must have that 

\begin{align*}
2x-y&=0 \\
2y-x+4 &=0 
\end{align*}

How many options are there?
\end{hint}

\begin{exercise}
The point $(x,y)$ for which $\grad{F}(x,y) = \vec{0}$ is $(x,y) = \left(\answer{-\frac{4}{3}},\answer{-\frac{8}{3}} \right)$

\begin{hint}
We must solve the system of equations
\begin{align}
2x-y&=0 \\
x-2y &=4 
\end{align}
\end{hint}

This can be done many ways.  Multiplying the top equation by $2$ and
subtracting the second equation from it gives $\answer{3} \cdot x =
\answer{-4}$, from which we find $x = \answer{-\frac{4}{3}}$.  We can
now use either equation to find $y$.

\begin{remark}
As it turns out, the points for which $\grad{F}(x,y) = \vec{0}$ will
be very important when we try to find relative extrema of a function
of two variables.  This exercise gives a little practice with
analyzing the gradient to that end.
\end{remark}
\end{exercise}
\end{exercise}
\end{exercise}

\end{document}
