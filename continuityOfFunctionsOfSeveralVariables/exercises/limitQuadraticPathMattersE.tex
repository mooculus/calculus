\documentclass{ximera}

\author{Jim Talamo}

\newcommand{\RR}{\mathbb R}
\renewcommand{\d}{\,d}
\newcommand{\dd}[2][]{\frac{d #1}{d #2}}
\renewcommand{\l}{\ell}
\newcommand{\ddx}{\frac{d}{dx}}
\newcommand{\dfn}{\textbf}
\newcommand{\eval}[1]{\bigg[ #1 \bigg]}


\outcome{Evaluate limits of functions of several variables.}

\begin{document}
\begin{exercise}
This exercise will establish an important class of function that can be used to show that a function may tend to the same value along all paths of a certain type to a point, but for which the limit does not exist at that point.

Consider the function $F(x,y) = \frac{2x^2y}{3x^4+4y^2}$.

Let's try to analyze the function along all straight line paths through $(0,0)$.  Note that the most general form of a straight line path to $(0,0)$ is found from the point-slope form of a line below.

\[
y-y_0=m(x-x_0)
\] 
Since the line passes through $(0,0)$, we find that any straight line path through $(0,0)$ must be of the form $y=mx$.

Now, let's see what $F(x,y)$ approaches as $(x,y) \to (0,0)$ along $y=mx$.

\begin{align*}
  F(x,y) = F(x,mx) &= \frac{2x^2\left(\answer{mx}\right)}{3x^4+4\left(\answer{mx}\right)^2} \\
  &= \frac{\answer{2m} \cdot x^3 }{3x^4 + \answer{4m^2}\cdot x^2} \\
  &= \frac{2mx}{3x^2+4m^2}
\end{align*}
where $x \neq 0$.
 
 For any finite $m$, note that as $x \to 0$, $F(x,y) \to \answer{0}$ as $(x,y) \to (0,0)$ along $y=mx$.  If we approach $(0,0)$ along the $y$-axis, we also find $F(x,y) \to \answer{0}$ either by taking $m \to \infty$ or by evaluating $F(x,0)$ for $y \neq 0$.
 
 Hence, along \emph{any} straight line path, $F(x,y) \to \answer{0}$ as $(x,y) \to (0,0)$.
 
 Is this enough to determine that $\Lim{x,y}{0,0} = 0$? \wordChoice{\choice{Yes}\choice[correct]{No}}
 
 \begin{exercise}
We really must establish that $F(x,y)$ approaches the same value along \emph{any} path, and there are other paths along which $(x,y)$ can approach $(0,0)$.

Let's consider the path $y=x^2$.  

First, can $(x,y) \to (0,0)$ along this path?  \wordChoice{\choice[correct]{Yes}\choice{No}}

Now, note that along this path

\[
F(x,y) = F(x,x^2) = \frac{2x^2\left(\answer{x^2}\right)}{3x^4+4\left(\answer{x^2}\right)^2} = \frac{2x^4}{\answer{7} \cdot x^4} = \frac{2}{7} , x \neq 0.
\]

 Hence, $F(x,y) \to \answer{\frac{2}{7}}$ as $(x,y) \to (0,0)$ along the path $y=x^2$.

Can we determine now whether $\Lim{x,y}{0,0}$ exists?
 
\begin{multipleChoice}
\choice{Yes; $\Lim{x,y}{0,0} F(x,y)$ exists.}
\choice[correct]{Yes; $\Lim{x,y}{0,0} F(x,y)$ does not exist.}
\choice{No}
\end{multipleChoice}

\begin{remark}
This is an example of a function whose domain is all of $\R^2$ except the origin, which allows us to consider any path that approaches $(0,0)$ in the domain.

This example is also not hard to modify to show that even if a function tends to the same value as $(x,y) \to (0,0)$ along all quadratic paths $y=a_2x^2+a_1x$, this is still not sufficient to show that  $\Lim{x,y}{0,0}$ exists.  An example of such a function is 

\[
F_3(x,y) = \frac{x^3y}{x^6+y^3}.
\]

This will tend to $0$ as $(x,y) \to (0,0)$ along $y=a_2x^2+a_1x$ for any choice of $a_1$ and $a_2$, but will approach $1/2$ as $(x,y) \to (0,0)$ along $y=x^3$.

It's not hard to continue playing this game; the function

\[
F_4(x,y) = \frac{x^4y}{x^8+y^4}
\]

will tend to $0$ as $(x,y) \to (0,0)$ along $y=a_3x^3+a_2x^2+a_1x$ for any choice of $a_1$, $a_2$, and $a_1$, but will approach $1/2$ as $(x,y) \to (0,0)$ along $y=x^4$, and so on.

This same example can be modified to construct an example of a function that tends to the same value along any path for which $x^2+y^2 \to 0$ at any rate slower than a prescribed one, but for which the limit does not exist.

The lesson?

\begin{quote}
\textbf{When we say that the function must tend to the same value along \emph{every} path, we mean it!}
\end{quote}
\end{remark}

 \end{exercise}
 \end{exercise}
\end{document}
