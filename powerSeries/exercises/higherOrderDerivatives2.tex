\documentclass{ximera}

\newcommand{\RR}{\mathbb R}
\renewcommand{\d}{\,d}
\newcommand{\dd}[2][]{\frac{d #1}{d #2}}
\renewcommand{\l}{\ell}
\newcommand{\ddx}{\frac{d}{dx}}
\newcommand{\dfn}{\textbf}
\newcommand{\eval}[1]{\bigg[ #1 \bigg]}


\author{Jim Talamo}
\license{Creative Commons 3.0 By-bC}


\outcome{}


\begin{document}
\begin{exercise}
Suppose that $f(x) = \sum_{k=0}^{\infty} \frac{80}{k!}(x+1)^{2k}$.

\begin{exercise}
Find $f^{(80)}(-1)$.

\[
f^{(80)}(-1) = \answer{\frac{80 \cdot 80!}{40!}}
\]
\end{exercise}

\begin{hint}
A good way to proceed is to use the relationship between the coefficients of the power series and the derivatives of the function it represents.

\[
\textrm{If } f(x) = \sum_{k=0}^{\infty} a_k(x-c)^k, \textrm{ then: } a_n = \frac{f^{(n)}(c)}{n!}
\]

Here, $c=\answer{-1}$.  In order to find $f^{(80)}(-1)$, we should use $n=\answer{80}$.

The coefficient in question is thus $a_{\answer{80}}$.  

\begin{question}
By definition $a_{80}$ is:

\begin{multipleChoice}
\choice{Always the coefficient obtained by plugging in $k=80$.}
\choice[correct]{The coefficient in front of $(x-c)^{80}$.}
\end{multipleChoice}

In this case, we find $(x+1)^{2k}=(x+1)^{80}$ when $k=\answer{40}$, so:

\[
a_{80} =  \frac{80}{k!} \bigg|_{k=\answer{40}} = \frac{\answer{80}}{\answer{40!}}
\]

\begin{question}
We now use the formula $a_n = \frac{f^{(n)}(c)}{n!}$, with $n=80$ (as found earlier) to find:

\begin{align*}
a_{80} &= \frac{f^{(80)}(-1)}{80!} \\
\answer{\frac{80}{40!}} &= \frac{f^{(80)}(-1)}{\left(\answer{80}\right)!}
\end{align*}

\begin{question}
Thus, $f^{(80)}(-1) = \answer{\frac{80}{40!}} \cdot \left(\answer{80}\right)! $

\end{question}
\end{question}
\end{question}

\end{hint}

\end{exercise}
\end{document}
