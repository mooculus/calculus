\documentclass{ximera}

\newcommand{\RR}{\mathbb R}
\renewcommand{\d}{\,d}
\newcommand{\dd}[2][]{\frac{d #1}{d #2}}
\renewcommand{\l}{\ell}
\newcommand{\ddx}{\frac{d}{dx}}
\newcommand{\dfn}{\textbf}
\newcommand{\eval}[1]{\bigg[ #1 \bigg]}


\title[Dig-In:]{Applications of integrals}

\begin{document}
\begin{abstract}
Accumulated area is computed by undoing the derivative. 
\end{abstract}
\maketitle


\section{A tale of three integrals}

\outcome{Define the average value of a function.}
\outcome{Find the average value of a function.}
\outcome{State the Mean Value Theorem for integrals.}
\outcome{Use the Mean Value Theorem for integrals.}


The Fundamental Theorem of Calculus says that an accumulation function
of $f(x)$ is an antiderivative of $f(x)$.  Because of the close
relationship between an integral and an antiderivative, the integral
sign is also used to mean ``antiderivative.'' You can tell which is
intended by whether the limits of integration are included. Hence
\[
  \int_a^b f(x)\d x
\] 
is a definite integral, because it has a definite value---the signed
area between $f(x)$ and the $x$-axis.  On the other hand, we use
\[
  \int f(x)\d x
\]
to denote the antiderivative of $f(x)$, also called an
\textit{indefinite integral}\index{indefinite integral}.
This is evaluated as
\[
  \int f(x)\d x = F(x)+C.
\]
Where $F'(x) = f(x)$ and the constant $C$ indicates that there are
really an infinite number of antiderivatives. We do not need to add
this $C$ to compute definite integrals, but in other circumstances we
will need to remember that the $C$ is there, so it is best to get into
the habit of writing the $C$.


\section{Average value and mean value theorem for integrals}


\section{Velocity position displacement}


\end{document}
