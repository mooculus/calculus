\documentclass{ximera}

\newcommand{\RR}{\mathbb R}
\renewcommand{\d}{\,d}
\newcommand{\dd}[2][]{\frac{d #1}{d #2}}
\renewcommand{\l}{\ell}
\newcommand{\ddx}{\frac{d}{dx}}
\newcommand{\dfn}{\textbf}
\newcommand{\eval}[1]{\bigg[ #1 \bigg]}



\author{Jim Talamo}

\outcome{Calculate the gradient.}
\outcome{Foreshadow the fact that the gradient is orthogonal to level curves.}
\outcome{Solve a system of equations.}
\newcommand{\Pp}[2]{\frac{\partial #1}{\partial #2}}

\begin{document}

\begin{exercise}
The following exercise foreshadows an important fact that we will study again later.

Suppose that $F(x,y) =5e^{2x-y-2}$.  

The equation of the level curve associated to $F(x,y)=5$ is $\answer{2x-y} = 2$.  Letting $x(t)=t$, we find that a parametric description of the level curve to be 

\[
\vec{c}(t) = \vector{t, \answer{2t-2}}.
\]

A vector parallel to the level curve is $\vec{v} = \vector{\answer{1},\answer{2}}$.

\begin{hint}
The level curve should be a line; by setting $F(x,y)=5$, we find that 

\begin{align*}
5e^{2x-y-2} &=5 \\
e^{2x-y-2} &= \answer{1} \\
2x-y-2 &= \answer{0} 
\end{align*}

In the last line, we take the natural logarithm of both sides.  

Now, to find a vector parallel to a line, we can start by giving a parametric description of it.  Then, we can bring it into the form

\[
\vecl(t) = \vec{p} + t\vec{v}
\]
where $\vec{v}$ and $\vec{P}$ are constant vectors.  As before, the vector $\vec{v}$ will be parallel to the line.
\end{hint}

\begin{exercise}
By a routine computation, we find that $\grad{F}(x,y) = \vector{\answer{10e^{2x-y-2}}, \answer{-5e^{2x-y-2}} }$.

An interesting observation can be made though.  At any point on the level curve associated to $z=5$, we found $2x-y=2$, so at any point along the level curve, $\grad{F}(x,y) = \vector{\answer{10}, \answer{-5} }$.

Note that $\grad{F} \dotp \vec{v} = \answer{0}$ at any point along the level curve.

\begin{hint}
If $2x-y=2$, then $e^{2x-y-2} = e^{2-2} =1$.
\end{hint}

\begin{feedback}[correct]
The result of this exercise is not an accident; as we will see, there is a relationship between the gradient and level curves, but we need more mathematical technology to establish the result more generally!
\end{feedback}

\end{exercise}
\end{exercise}

\end{document}
