\documnetclass{ximera}

\newcommand{\RR}{\mathbb R}
\renewcommand{\d}{\,d}
\newcommand{\dd}[2][]{\frac{d #1}{d #2}}
\renewcommand{\l}{\ell}
\newcommand{\ddx}{\frac{d}{dx}}
\newcommand{\dfn}{\textbf}
\newcommand{\eval}[1]{\bigg[ #1 \bigg]}


\title[Dig-In:]{The Derivative of $\textit{e}^\textit{x}$}

\begin{document}
\begin{abstract}
An abstract  
\end{abstract}
\maketitle

We don't know anything about derivatives that allows us to compute the
derivatives of exponential functions without getting our hands
dirty. Let's do a little work with the definition of the derivative:
\begin{align*}
\ddx a^x &=\lim_{h\to 0} \frac{a^{x+h}-a^x}{h} \\
&=\lim_{h\to 0} \frac{a^xa^{h}-a^x}{h} \\
&=\lim_{h\to 0} a^x\frac{a^{h}-1}{h} \\
&=a^x\lim_{h\to 0} \frac{a^{h}-1}{h} \\
&=a^x \cdot \underbrace{\text{(constant)}}_{\lim_{h\to 0} \frac{a^{h}-1}{h}}
\end{align*}
There are two interesting things to note here: We are left with a
limit that involves $h$ but not $x$, which means that whatever $
\lim_{h\to 0} (a^h-1)/h$ is, we know that it is a number, that is, a
constant. This means that $a^x$ has a remarkable property: Its
derivative is a constant times itself. Unfortunately it is beyond the
scope of this text to compute the limit
\[
\lim_{h\to 0} \frac{a^h-1}{h}.
\]
However, we can look at some examples. Consider $(2^h-1)/h$ and $(3^h-1)/h$:
\begin{fullwidth}
\[
\begin{tchart}{ll}
 h &     (2^h-1)/h\\ \hline
 -1 & .5  \\
-0.1 &  \approx0.6700 \\
-0.01 & \approx0.6910 \\
-0.001 & \approx0.6929 \\
-0.0001 & \approx0.6931 \\
-0.00001 & \approx0.6932 \\
\end{tchart}
\qquad
\begin{tchart}{ll}
 h &     (2^h-1)/h\\ \hline
 1 & 1  \\
 0.1 &  \approx0.7177 \\
 0.01 & \approx0.6956 \\
 0.001 & \approx0.6934 \\
 0.0001 & \approx0.6932 \\
 0.00001 & \approx0.6932 \\
\end{tchart}
\qquad\qquad
\begin{tchart}{ll}
 h &     (3^h-1)/h\\ \hline
-1 & \approx 0.6667  \\
-0.1 &  \approx1.0404  \\
-0.01 & \approx1.0926 \\
-0.001 & \approx1.0980 \\
-0.0001 & \approx1.0986 \\
-0.00001 & \approx1.0986 \\
\end{tchart}
\qquad
\begin{tchart}{ll}
 h &     (3^h-1)/h\\ \hline
 1 & 2  \\
 0.1 &  \approx1.1612 \\
 0.01 & \approx1.1047 \\
 0.001 & \approx1.0992 \\
 0.0001 & \approx1.0987 \\
 0.00001 & \approx1.0986 \\
\end{tchart}
\]
\end{fullwidth}

While these tables don't prove a pattern, it turns out that
\[
\lim_{h\to 0}\frac{2^h-1}{h} \approx .7 \qquad\text{and}\qquad \lim_{h\to 0} \frac{3^h-1}{h} \approx 1.1.
\]
Moreover, if you do more examples you will find that the limit varies
directly with the value of $a$: bigger $a$, bigger limit; smaller $a$,
smaller limit. As we can already see, some of these limits will be
less than 1 and some larger than 1. Somewhere between $a=2$ and $a=3$
the limit will be exactly 1. This happens when 
\[
a = e = 2.718281828459045\dots.
\]
This brings us to our next definition.
\begin{definition}\index{Euler's number}
Euler's number is defined to be the number $e$ such that
\[
\lim_{h\to 0} \frac{e^h-1}{h} = 1.
\]
\end{definition}
Now we see that the function $e^x$ has a truly remarkable property:

\begin{mainTheorem}[The Derivative of $\textit{e}^\textit{x}$]\index{ex@$e^x$}\index{derivative!of ex@of $e^x$}
\[
\ddx e^x = e^x.
\]
\end{mainTheorem}
\begin{proof}  
From the limit definition of the derivative, write
\begin{align*}
\ddx e^x&=\lim_{h\to 0} \frac{e^{x+h}-e^x}{h} \\
&=\lim_{h\to 0} \frac{e^xe^{h}-e^x}{h} \\
&=\lim_{h\to 0} e^x\frac{e^{h}-1}{h} \\
&=e^x\lim_{h\to 0} \frac{e^{h}-1}{h} \\
&=e^x.
\end{align*}
\end{proof}



Hence $e^x$ is its own derivative. In other words, the slope of the
plot of $e^x$ is the same as its height, or the same as its second
coordinate: The function $ f(x)=e^x$ goes through the point $ (a,e^a)$
and has slope $e^a$ there, no matter what $a$ is. 



\begin{example}
Compute:
\[
\ddx\left(8\sqrt{x} + 7e^x \right)
\]
\end{example}

\begin{solution}
Write:
\begin{align*}
\ddx\left(8\sqrt{x} + 7e^x \right) &= 8\ddx x^{1/2} + 7\ddx e^x\\
&= 4x^{-1/2} + 7 e^x.
\end{align*}
\end{solution}


\end{document}
