\documentclass{ximera}
\newcommand{\RR}{\mathbb R}
\renewcommand{\d}{\,d}
\newcommand{\dd}[2][]{\frac{d #1}{d #2}}
\renewcommand{\l}{\ell}
\newcommand{\ddx}{\frac{d}{dx}}
\newcommand{\dfn}{\textbf}
\newcommand{\eval}[1]{\bigg[ #1 \bigg]}

\author{Jim Talamo}
\title[Refresh:]{Limits}

\begin{document}
\begin{abstract}
Recall results concerning infinite limits of rational functions.
\end{abstract}
\maketitle


Being able to determine infinite limits of rational functions quickly is an important skill.  When evaluating these limits, all that is necessary is to consider the highest degree term in the numerator and the denominator.  As a reminder as to why (and how) this works, consider the following examples.

\begin{example}
Compute $\lim_{x \to \infty} \frac{2x^2-5x+7}{3x^2+6x-5}$.

\begin{explanation}
In the numerator, the highest degree term is $2x^2$, so we can factor out an $x^2$ and obtain

\[
2x^2-5x+7 = x^2\left(2-\frac{5}{x} + \frac{7}{x^2}\right)
\]

In the denominator, the highest degree term is $3x^2$, so we can factor out an $x^2$ and obtain

\[
3x^2+6x-5 = x^2\left(3+\frac{6}{x} - \frac{5}{x^2}\right)
\]

We can now rewrite the limit.

\begin{align*}
\lim_{x \to \infty} \frac{2x^2-5x+7}{3x^2+6x-5} &= \lim_{x \to \infty} \frac{x^2\left(2-\frac{5}{x} + \frac{7}{x^2}\right)}{x^2\left(3+\frac{6}{x} - \frac{5}{x^2}\right)} \\
&= \lim_{x \to \infty} \frac{\cancel{x^2}}{\cancel{x^2}} \cdot \left[ \frac{2-\frac{5}{x} + \frac{7}{x^2}}{3+\frac{6}{x} - \frac{5}{x^2}} \right] \\
\end{align*}
Now, notice that as $x \to \infty$, 

\[
 \lim_{x \to \infty}   \frac{2-\frac{5}{x} + \frac{7}{x^2}}{3+\frac{6}{x} - \frac{5}{x^2}} = \lim_{x \to \infty}   \frac{2-\cancel{\frac{5}{x}} + \cancel{\frac{7}{x^2}}}{3+\cancel{\frac{6}{x}} - \cancel{\frac{5}{x^2}}} = \frac{2}{3}\\
\]

Note that the limit symbol is kept until we actually evaluate the limit in the final step.  
\end{explanation}
\end{example}

Here's another example.

\begin{example}
Compute $\lim_{x \to \infty} \frac{3x^3-4x+8}{2x^4-2x^2+x}$.

\begin{explanation}
In the numerator, the highest degree term is $3x^3$, so we can factor out an $x^3$ and obtain

\[
3x^3-4x+8 = x^3\left(3-\frac{4}{x^2} + \frac{8}{x^3}\right)
\]

In the denominator, the highest degree term is $2x^4$, so we can factor out an $x^4$ and obtain

\[
2x^4-2x^2+x = x^4\left(2-\frac{2}{x^2} + \frac{1}{x^3}\right)
\]

We can now rewrite the limit.

\begin{align*}
\lim_{x \to \infty}  \frac{3x^3-4x+8}{2x^4-2x^2+x} &= \lim_{x \to \infty} \frac{ x^3\left(3-\frac{4}{x^2} + \frac{8}{x^3}\right)}{x^4\left(2-\frac{2}{x^2} + \frac{1}{x^3}\right)} \\
&= \lim_{x \to \infty} \frac{x^3}{x^4} \cdot \left[ \frac{3-\frac{4}{x^2} + \frac{8}{x^3}}{2-\frac{2}{x^2} + \frac{1}{x^3}} \right] \\
&= \lim_{x \to \infty} \frac{1}{x} \cdot \left[ \frac{3-\frac{4}{x^2} + \frac{8}{x^3}}{2-\frac{2}{x^2} + \frac{1}{x^3}} \right] \\
& = 0 \cdot \frac{3}{2}
\end{align*}

Note again that the limit symbol is kept until we actually evaluate the limit in the final step.
\end{explanation}
\end{example}

Now, try your hand at the following.  If a limit is infinite, write ``$\infty$'' or ``$-\infty$''; note that Ximera will \emph{not} accept the syntax ``$+\infty$''. indicate this.  You should be able to evaluate the limits quickly as well as show the algebraic steps above if asked.

\begin{problem}

\[
\lim_{x \to \infty} \frac{x^5+2x^4+1}{2x^4-4x+2} = \answer{\infty}
\]

\[
\lim_{x \to \infty} \frac{4x^2+x^{3/2}}{3x^2-2x+1} = \answer{\frac{4}{3}}
\]


\end{problem}
\end{document}
