\documentclass{ximera}

\newcommand{\RR}{\mathbb R}
\renewcommand{\d}{\,d}
\newcommand{\dd}[2][]{\frac{d #1}{d #2}}
\renewcommand{\l}{\ell}
\newcommand{\ddx}{\frac{d}{dx}}
\newcommand{\dfn}{\textbf}
\newcommand{\eval}[1]{\bigg[ #1 \bigg]}


\author{Jim Talamo}
\license{Creative Commons 3.0 By-NC}


\outcome{Think about Partial Fraction Decomposition}


\begin{document}
\begin{exercise}
Consider the indefinite integral: 

\[
\int \frac{2x-4}{x^3-4x} \d x.
\]

If we want to integrate this, partial fraction decomposition would be a good technique to try.  However, we could first note that we could factor and simplify.  Indeed:

\[
\frac{2x-4}{x^3-4x} = \frac{2(x-2)}{x(x+2)(x-2)} = \answer{\frac{2}{x(x+2)}}
\]  

The partial fraction decomposition for this expression is:

\[
\frac{2}{x(x+2)} = \answer{\frac{A}{x}+\frac{B}{x+2}}
\]
(Use $A$ and $B$ for the constants and list the terms in the order written in the factored form)

\begin{exercise}
Solving for these constants, we find: $A = \answer{1}$ and $B=\answer{-1}$ and we conclude:

\[
\frac{2}{x(x+2)} = \answer{\frac{1}{x}-\frac{1}{x+2}}
\]


\begin{exercise}
Suppose we did not notice that there was simplifying that could be done first.  In this case, we would write:

\[
\frac{2x-4}{x^3-4x} = \frac{2x-4}{x(x+2)(x-2)} 
\]  
Without simplifying, what is the correct general partial fraction decomposition of the above expression?

\[
\frac{2x-4}{x^3-4x} = \answer{\frac{A}{x}+\frac{B}{x+2} +\frac{C}{x-2}}
\]  
(Use $A$, $B$, and $C$ for the constants and list the terms in the order written in the factored form)

\begin{exercise}

Multiplying both sides by $x^3-4x$ gives:

\[
2x-4 = \answer{(x+2)(x-2)}A+\answer{x(x-2)}B +\answer{x(x+2)}C
\]  

\begin{exercise}
Setting $x=0$ gives $A=\answer{1}$.

Setting $x=2$ gives $C=\answer{0}$.

Setting $x=-2$ gives $B=\answer{-1}$.

So, the partial fraction decomposition for this expression is:

\[
\frac{2x-4}{x(x+2)(x-2)}  = \answer{\frac{1}{x}+\frac{-1}{x+2}}
\]


\begin{multipleChoice}
\choice[correct]{The results are the same}
\choice{The results are not the same; we broke math!}
\end{multipleChoice}

\begin{exercise}
Using the above results, we find:

 \[
\int \frac{2x-4}{x^3-4x} \d x = \answer{\ln|x|-\ln|x-2|+C}
\]
(Use $C$ for the constant of integration)


\end{exercise}
\end{exercise}
\end{exercise}
\end{exercise}
\end{exercise}
\end{exercise}
\end{document}
