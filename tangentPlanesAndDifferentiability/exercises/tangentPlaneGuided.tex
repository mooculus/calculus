\documentclass{ximera}

\newcommand{\RR}{\mathbb R}
\renewcommand{\d}{\,d}
\newcommand{\dd}[2][]{\frac{d #1}{d #2}}
\renewcommand{\l}{\ell}
\newcommand{\ddx}{\frac{d}{dx}}
\newcommand{\dfn}{\textbf}
\newcommand{\eval}[1]{\bigg[ #1 \bigg]}

\usepackage{currfile}
\makeatletter
\ifxake
% The code below the \else is executed in a sagecell on the Ximera
% server, so \makerandom doesn't have to do anything when run under
% xake.
\newcommand{\makerandom}{}
\else
\newcommand {\ST@wsf }[1]{\immediate \write \ST@sf {##1}}
\newcommand{\makerandom}{%
  \ST@wsf{jobname="\currfilebase"}%
  \ST@wsf{import hashlib}%
  \ST@wsf{set_random_seed(int(hashlib.sha256(jobname.encode('utf-8')).hexdigest(), 16))}%
}
\fi
\makeatother


\author{Jim Talamo}

\outcome{Find a tangent plane to the graph of a function.}

\begin{document}

\begin{exercise}
Before we compute the tangent plane, let's take a step back and draw a few parallels for tangent lines and planes and how we find them.

\begin{itemize}
\item When we talk about lines in the $xy$-plane, we need a point and a slope.  For tangent lines, we evaluate the derivative $\dd[y]{x}$ at the point of tangency to find the slope.
\item When we talk about lines in the $xyz$-plane, ``slope'' is not well-defined; we need a vector parallel to the line and a point on the line.  For tangent lines to a curve with parameterization $\vec{r}(t)$, we evaluate the derivative $\vec{r}'(t)$ at the appropriate $t$-value to find the vector parallel to the line. 
\item When we talk about planes in the $xyz$-plane, ``slope'' also does not make sense; we need a normal vector and a point.  When we want to find the tangent plane to a surface $z=f(x,y)$, we use the gradient to construct a normal vector.
\end{itemize}

In fact, the tangent plane at $(a,b,f(a,b))$ has normal vector 

\[
\vec{n}(a,b) =\vector{f_x(a,b), f_y(a,b), -1}.
\]

Now, given that $f(x,y) = 4x^2y+e^{2x-4y+6}$, give the equation of the tangent plane to the surface $z=f(x,y)$ where $(x,y)=(1,2)$.

\begin{itemize}
\item The point on the plane can be found by evaluating $f(x,y)$ at $(1,2)$.  Doing so, we find $f(1,2) = \answer{9}$.
\item To construct a normal vector, note that

\begin{itemize}
\item $\pp[f]{x}(x,y) = \answer{8xy+2e^{2x-4y+6}}$ so $\pp[f]{x}(1,2) = \answer{18}$.
\item $\pp[f]{y}(x,y) = \answer{4x^2-4^{2x-4y+6}}$ so $\pp[f]{y}(1,2) = \answer{0}$.
\end{itemize}

Thus, a normal vector to the tangent plane is $\vec{n} = \vector{\answer{18},\answer{0},-1}$.

\end{itemize}

\begin{exercise}
We now compute the tangent plane.  Recall that the equation of the plane with normal vector $\vec{n}=\vector{a,b,c}$ that passes through $P_0 = \big(x_0,y_0,z_0\big)$ is

\[
a(x-x_0)+b(y-y_0)+c(z-z_0) = 0.
\]

Using 
\begin{align*}
18\cdot \left(x-\answer{1}\right)+\answer{0} \cdot \left(y-\answer{2}\right)+ \answer{-1} \cdot \left(z-\answer{9}\right) &= 0 \\
\end{align*}

   
\end{exercise} 
\end{exercise}
\end{document}
