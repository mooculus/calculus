\documentclass{ximera}

\newcommand{\RR}{\mathbb R}
\renewcommand{\d}{\,d}
\newcommand{\dd}[2][]{\frac{d #1}{d #2}}
\renewcommand{\l}{\ell}
\newcommand{\ddx}{\frac{d}{dx}}
\newcommand{\dfn}{\textbf}
\newcommand{\eval}[1]{\bigg[ #1 \bigg]}


	\outcome{Recognize a composition of functions.}
	\outcome{Take derivatives of compositions of functions using the chain rule.}
	\outcome{Take derivatives that require the use of multiple derivative rules.}
	\outcome{Use the chain rule to calculate derivatives from a table of values.}
	\outcome{Understand rate of change when quantities are dependent upon each other.}
	\outcome{Use order of operations in situations requiring multiple derivative rules.}
    \outcome{Justify the chain rule via the composition of linear approximations.}
    \outcome{Apply chain rule to relate quantities expressed with different units.}

\title[Dig-In:]{The chain rule}

\begin{document}
\begin{abstract}
Here we compute derivatives of compositions of functions
\end{abstract}
\maketitle


So far we have seen how to compute the derivative of a function built
up from other functions by addition, subtraction, multiplication and
division. There is another very important way that we combine
functions: composition. The \textit{chain rule} allows us to deal with
this case.


Consider
\[
h(x) = (1+2x)^5.
\] 

While there are several different ways to differentiate this function,
if we let $f(x) = x^5$ and $g(x) = 1+2x$, then we can express $h(x) =
f(g(x))$. The question is, can we compute the derivative of a
composition of functions using the derivatives of the constituents
$f(x)$ and $g(x)$? To do so, we need the \textit{chain rule}.



\begin{theorem}[Chain Rule]\index{chain rule}\index{derivative rules!chain}
If $f(x)$ and $g(x)$ are differentiable, then
\[
\ddx f(g(x)) = f'(g(x))g'(x).
\]
\begin{explanation}
Here we give a somewhat unrigoroius explanation, but it will serve our
purposes.

\begin{image}
\begin{tikzpicture}
	\begin{axis}[
            axis lines=none,
            clip=false,
          ]          
          \addplot [->,textColor] plot coordinates {(0,0) (-2,-4)}; %% x axis
          \addplot [->,textColor] plot coordinates {(0,0) (0,6)}; %% y axis
          \addplot [->,textColor] plot coordinates {(0,0) (6,0)}; %% g(x) axis

          \addplot [dashed, textColor] plot coordinates {(-.7,-1.4) (1.4,-1.4)};
          \addplot [dashed, textColor] plot coordinates {(1.4,-1.4) (2.1,0)};
          \addplot [dashed, textColor] plot coordinates {(2.1,0) (2.1,4.1)};
          
          \addplot [dashed, textColor] plot coordinates {(2.6,-2.6) (3.5,0)};
          \addplot [dashed, textColor] plot coordinates {(3.5,0) (3.5,4.1)};

          \addplot [dashed, very thick, textColor] plot coordinates {(1.4,-1.4) (.8,-2.6)};
          \addplot [dashed, very thick, textColor] plot coordinates {(2.1,4.1) (3.5,4.1)};

          \addplot [very thick, penColor5] plot coordinates {(.8,-2.6) (2.6,-2.6)};
          \addplot [very thick, penColor4] plot coordinates {(3.5,4.1) (3.5,5.5)};

          \addplot [very thick,penColor,domain=(0:4)] {2+x};
          \addplot [very thick,penColor2,domain=(0:4)] {-x};

          \node at (axis cs:3.5,4.8) [anchor=west,penColor4] {$f'(g(a)){\color{penColor5}g'(a)h}$};
          \node at (axis cs:1.7,-2.6) [anchor=north,penColor5] {$g'(a)h$};
          
          \addplot[color=penColor2,fill=penColor2,only marks,mark=*] coordinates{(1.4,-1.4)};  %% closed hole          
          \addplot[color=penColor,fill=penColor,only marks,mark=*] coordinates{(2.1,4.1)};  %% closed hole          

          \node at (axis cs:1,-2.1) [anchor=south,yslant=0,xslant=0,rotate=53] {$\overbrace{\hspace{.36in}}^{h}$};
          \node at (axis cs:7,0) [anchor=east] {$g(x)$};
          \node at (axis cs:0,6.7) [anchor=north] {$y$};
          \node at (axis cs:-2.15,-4) [anchor=north] {$x$};
          \node at (axis cs:-.7,-1.4) [anchor=east] {$a$};
        \end{axis}
\end{tikzpicture}
%% \caption{A geometric interpretation of the chain rule. Increasing $a$
%%   by a ``small amount'' $h$, increases $f(g(a))$ by $f'(g(a))g'(a)h$. Hence, 
%% \[
%% \frac{\Delta y}{\Delta x}\approx \frac{f'(g(a))g'(a)h}{h} =
%% f'(g(a))g'(a).
%% \]} 
\end{image}

\end{explanation}
\end{theorem}



It will take a bit of practice to make the use of the chain rule come
naturally---it is more complicated than the earlier differentiation
rules we have seen. Let's return to our motivating example.

\begin{example}
Compute:
\[
\ddx (1+2x)^5
\]

\begin{explanation}
Set $f(x) = x^5$ and $g(x) = 1+2x$, now
\[
f'(x) = \answer[given]{5x^4} \qquad\text{and}\qquad g'(x) = \answer[given]{2}.
\]
Hence
\begin{align*}
\ddx (1+2x)^5 &= \ddx f(g(x))\\ 
&=f'(g(x))g'(x) \\
&= 5(\answer[given]{1+2x})^4\cdot \answer[given]{2}\\
&= 10(1+2x)^4.
\end{align*}
\end{explanation}
\end{example}


Let's see a more complicated chain of compositions.

\begin{example}
Compute:
\[
\ddx \sqrt{1+\sqrt{x}}
\]

\begin{explanation}
Set 
$f(x)=\sqrt{x}$ and $g(x)=1+x$. Hence,
\[
\sqrt{1+\sqrt{x}}=f(g(\answer[given]{f}(x))) \qquad\text{and}\qquad\ddx f(g(f(x))) = f'(g(f(x)))g'(f(x))f'(x).
\]
Since 
\[
f'(x) = \answer[given]{\frac{1}{2\sqrt{x}}} \qquad\text{and}\qquad g'(x) = \answer[given]{1}
\]
We have that
\[
\ddx \sqrt{1+\sqrt{x}} = \frac{1}{2\sqrt{1+\sqrt{x}}}\cdot 1\cdot  \answer[given]{\frac{1}{2\sqrt{x}}}.
\]
\end{explanation}
\end{example}

Using the chain rule, the power rule, and the product rule it is
possible to avoid using the quotient rule entirely.

\begin{example}
Compute:
\[
\ddx \frac{x^3}{x^2+1}
\]

\begin{explanation}
Rewriting this as 
\[
\ddx x^3(x^2+1)^{-1}, 
\]
set $f(x) = \answer[given]{x^{-1}}$ and $g(x) = \answer[given]{x^2+1}$ so that $f(g(x)) = (x^2 + 1)^{-1}$. Now
\[
x^3(x^2+1)^{-1} = x^3 f(g(x)) \qquad\text{and}\qquad \ddx x^3 f(g(x)) = \answer[given]{3x^2} \cdot f(g(x))+ \answer[given]{x^3} \cdot f'(g(x))g'(x).
\]
Since $f'(x) = \answer[given]{\frac{-1}{x^2}}$ and $g'(x) = \answer[given]{2x}$, write
\[
\ddx \frac{x^3}{x^2+1} = \frac{3x^2}{x^2+1}-\frac{2x^4}{(x^2+1)^2}.
\]
\end{explanation}
\end{example}



\begin{warning}
When working with derivatives of trigonometric functions, we suggest
you use \textbf{radians} for angle measure. For example, while
\[
\sin\left((90^\circ\right)^2) = \sin\left(\left(\frac{\pi}{2}\right)^2\right),
\]
one must be careful with derivatives as
\[
\eval{\ddx \sin\left(x^2\right)}_{x=90^\circ} \ne \underbrace{2\cdot 90\cdot \cos(90^2)}_{\text{incorrect}}
\]
Alternatively, one could think of $x^\circ$ as meaning
$\frac{x\cdot\pi}{180}$, as then $90^\circ = \frac{90\cdot\pi}{180} =
\frac{\pi}{2}$. In this case
\[
2\cdot 90^\circ\cdot \cos((90^\circ)^2) = 2\cdot \frac{\pi}{2}\cdot\cos\left(\left(\frac{\pi}{2}\right)^2\right).
\]
\end{warning}



\end{document}
