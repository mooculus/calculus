\documentclass{ximera}

\newcommand{\RR}{\mathbb R}
\renewcommand{\d}{\,d}
\newcommand{\dd}[2][]{\frac{d #1}{d #2}}
\renewcommand{\l}{\ell}
\newcommand{\ddx}{\frac{d}{dx}}
\newcommand{\dfn}{\textbf}
\newcommand{\eval}[1]{\bigg[ #1 \bigg]}


\title[Dig-In:]{The inverse function theorem}

\begin{document}
\begin{abstract}
\end{abstract}
\maketitle

There is one catch to the explanations given above. To write
$\ddx(e^y)=e^y\dd[y]{x}$ we need to know that the function $y$
\textit{has} a derivative. All we have shown is that \textit{if} it
has a derivative then that derivative must be $1/x$. The
\textit{Inverse Function Theorem} guarantees this.

\begin{theorem}[Inverse Function Theorem]\index{Inverse Function Theorem}\label{theorem:IFT}
If $f(x)$ is a differentiable function, and $f'(x)$ is continuous, and
$f'(a) \neq 0$, then
\begin{enumerate}
\item $f^{-1}(y)$ is defined for $y$ near $f(a)$,
\item $f^{-1}(y)$ is differentiable near $f(a)$, 
\item $\dd{y} f^{-1}(y)$ is continuous near $f(a)$, and
\item last, but not least:
  \[
  \dd{y} f^{-1}(y)  = \frac{1}{f'(f^{-1}(y))}.
  \]
\end{enumerate}
\end{theorem}

\end{document}
