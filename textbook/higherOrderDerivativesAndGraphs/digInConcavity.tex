\documentclass{ximera}

\newcommand{\RR}{\mathbb R}
\renewcommand{\d}{\,d}
\newcommand{\dd}[2][]{\frac{d #1}{d #2}}
\renewcommand{\l}{\ell}
\newcommand{\ddx}{\frac{d}{dx}}
\newcommand{\dfn}{\textbf}
\newcommand{\eval}[1]{\bigg[ #1 \bigg]}


\outcome{Use the first derivative to determine whether a function is increasing or decreasing.}
\outcome{Identify the relationships between the function and its first and second derivatives.}
\outcome{Sketch a graph of the second derivative, given the original function.}
\outcome{Sketch a graph of the original function, given the graph of its first and second derivatives.}
\outcome{State the relationship between concavity and the second derivative.}

\title[Dig-In:]{Concavity}

\begin{document}
\begin{abstract}
  Here we examine what the second derivative tells us about the
  geometry of functions.
\end{abstract}
\maketitle



We know that the sign of the derivative tells us whether a function is
increasing or decreasing at some point. Likewise, the sign of the
second derivative $f''(x)$ tells us whether $f'(x)$ is increasing or
decreasing at $x$. We summarize the consequences of this seemingly
simple idea in the table below:

\index{concave up/down}

\begin{center}
\begin{tabular}{c|c|c|} 
  & $f'(x)<0$ & $f'(x) > 0$ \\ \hline
  %& & \\[-1.5ex]
  $f''(x)> 0$ &
  \begin{tikzpicture}
	\begin{axis}[
            clip=false,
            width=2in,
            domain=0:1,
            ymax=1,
            ymin=0,
            axis lines=none,
          ]
          \addplot [very thick, penColor, smooth] {(x-1)^2};
          \node at (axis cs:.7,.4) [textColor] {\footnotesize Concave Up};
          \node at (axis cs:.5,-.4) [text width=2in] {\footnotesize
            Here $f'(x)<0$ and $f''(x)>0$. This means that $f(x)$ slopes down and
            is getting \textit{less steep}. In this case the curve is
            \dfn{concave up}.};
        \end{axis}
\end{tikzpicture}
&
\begin{tikzpicture}
	\begin{axis}[
            clip=false,
            domain=0:1,
            ymax=1,
            width=2in,
            ymin=0,
            axis lines=none,
          ]
          \addplot [very thick, penColor, smooth] {x^2};
          \node at (axis cs:.3,.4) [textColor] {\footnotesize Concave Up};
          \node at (axis cs:.5,-.4) [text width=2in] {\footnotesize Here $f'(x)>0$ and $f''(x)>0$. This means that $f(x)$ slopes up and is
getting \textit{steeper}. In this case the curve is \textbf{concave up}.};
        \end{axis}
\end{tikzpicture}
\\[-2ex]
& & 
\\\hline 
& & \\[-1.5ex]
$f''(x)<0$ &
\begin{tikzpicture}
	\begin{axis}[
            clip=false,
            width=2in,
            domain=0:1,
            ymax=1,
            ymin=0,
            axis lines=none,
          ]
          \addplot [very thick, penColor, smooth] {-x^2+1};
          \node at (axis cs:.4,.4) [textColor] {\footnotesize Concave Down};
          \node at (axis cs:.5,-.4) [text width=2in] {\footnotesize Here $f'(x)<0$ and $f''(x)<0$. This means
that $f(x)$ slopes down and is getting \textit{steeper}. In this case the curve is \textbf{concave down}.};
        \end{axis}
\end{tikzpicture}
&
\begin{tikzpicture}
  \begin{axis}[
      clip=false,
      width=2in,
      domain=0:1,
      ymax=1,
      ymin=0,
      axis lines=none,
    ]
    \addplot [very thick, penColor, smooth] {-(x-1)^2+1};
    \node at (axis cs:.6,.4) [textColor] {\footnotesize Concave Down};
    \node at (axis cs:.5,-.4) [text width=2in] {\footnotesize Here $f'(x)>0$ and $f''(x)<0$. This means
that $f(x)$ slopes up and is getting less \textit{steep}. In this case the curve is \textbf{concave down}.};
  \end{axis}
\end{tikzpicture}
\\[-2ex]
& & 
\\\hline 
\end{tabular}
\end{center}

If we are trying to understand the shape of the graph of a function,
knowing where it is concave up and concave down helps us to get a more
accurate picture. It is worth summarizing what we have seen already in
to a single theorem.

\begin{theorem}[Test for Concavity]\index{concavity test}
Suppose that $f''(x)$ exists on an interval.
\begin{enumerate}
\item $f''(x)>0$ on that interval whenever $y=f(x)$ is concave up on that interval.
\item $f''(x)<0$ on that interval whenever $y=f(x)$ is concave down on that interval.
\end{enumerate}
\end{theorem}


\begin{example}
  Let $f$ be a continuous function and suppose that:
  \begin{itemize}
  \item $f'(x) > 0$ for $-1< x<1$.
  \item $f'(x) < 0$ for $-2< x<-1$ and $1<x<2$.
  \item $f''(x) > 0$ for $-2<x<0$ and $1<x< 2$.
  \item $f''(x) < 0$ for $0<x< 1$.  
  \end{itemize}
  Sketch a possible graph of $f$.
  \begin{explanation}
    Start by marking where the derivative changes sign and indicate
    intervals where $f$ is increasing and intervals $f$ is
    decreasing. The function $f$ has a negative derivative from $-2$
    to $x=\answer[given]{-1}$. This means that $f$ is
    \wordChoice{\choice{increasing}\choice[correct]{decreasing}} on
    this interval. The function $f$ has a positive derivative from
    $x=\answer[given]{-1}$ to $x=\answer[given]{1}$. This means that
    $f$ is
    \wordChoice{\choice[correct]{increasing}\choice{decreasing}} on
    this interval. Finally, The function $f$ has a negative derivative
    from $x=\answer[given]{1}$ to $2$. This means that $f$ is
    \wordChoice{\choice{increasing}\choice[correct]{decreasing}} on
    this interval.
  \begin{image}
    \begin{tikzpicture}
    \begin{axis}[
        xmin=-2,xmax=2,ymin=-2,ymax=2,
        axis lines=center,
        width=6in,
        height=3in,
        every axis y label/.style={at=(current axis.above origin),anchor=south},
        every axis x label/.style={at=(current axis.right of origin),anchor=west},
      ]
      \addplot [dashed, penColor2] plot coordinates {(-1,-2) (-1,2)}; %% Critical points
      \addplot [dashed, penColor2] plot coordinates {(1,-2) (1,2)}; %% Critical points

      \addplot [->, line width=10, penColor!10!background] plot coordinates {(-1+.2,-2+.2) (1-.2,2-.2)};
      \addplot [->, line width=10, penColor!10!background] plot coordinates {(-2+.2,2-.2) (-1-.2,-2+.2)};
      \addplot [->, line width=10, penColor!10!background] plot coordinates {(1+.2,2-.2) (2-.2,-2+.2)}; 
      
      %\addplot [very thick,penColor,smooth, domain=(-2:2)] {x^3+x^2-2*x)};
    \end{axis}
  \end{tikzpicture}
  \end{image}
  Now we should sketch the concavity: \wordChoice{\choice[correct]{concave up}\choice{concave down}} when the second
  derivative is positive, \wordChoice{\choice{concave up}\choice[correct]{concave down}} when the second derivative is
  negative.
    \begin{image}
    \begin{tikzpicture}
    \begin{axis}[
        xmin=-2,xmax=2,ymin=-2,ymax=2,
        axis lines=center,
        width=6in,
        height=3in,
        every axis y label/.style={at=(current axis.above origin),anchor=south},
        every axis x label/.style={at=(current axis.right of origin),anchor=west},
      ]
      \addplot [dashed, penColor2] plot coordinates {(-1,-2) (-1,2)}; %% Critical points
      \addplot [dashed, penColor2] plot coordinates {(1,-2) (1,2)}; %% Critical points

      \addplot [->, line width=10, penColor!10!background] plot coordinates {(-1+.2,-2+.2) (1-.2,2-.2)};
      \addplot [->, line width=10, penColor!10!background] plot coordinates {(-2+.2,2-.2) (-1-.2,-2+.2)};
      \addplot [->, line width=10, penColor!10!background] plot coordinates {(1+.2,2-.2) (2-.2,-2+.2)};

      \addplot [penColor3,very thick,domain=180:270] ({-1.1+.7*cos(x)}, {-.1+.7*sin(x)});
      \addplot [penColor3,very thick,domain=270:360] ({-.9+.7*cos(x)}, {-.1+.7*sin(x)});
      \addplot [penColor3,very thick,domain=90:180] ({.9+.7*cos(x)}, {.1+.7*sin(x)});
      \addplot [penColor3,very thick,domain=180:270] ({1.9+.7*cos(x)}, {.9+.7*sin(x)});
      
      %\addplot [very thick,penColor,smooth, domain=(-2:2)] {x^3+x^2-2*x)};
    \end{axis}
  \end{tikzpicture}
    \end{image}
    Finally, we can sketch our curve:
        \begin{image}
    \begin{tikzpicture}
    \begin{axis}[
        xmin=-2,xmax=2,ymin=-2,ymax=2,
        axis lines=center,
        width=6in,
        height=3in,
        every axis y label/.style={at=(current axis.above origin),anchor=south},
        every axis x label/.style={at=(current axis.right of origin),anchor=west},
      ]
      \addplot [dashed, penColor2] plot coordinates {(-1,-2) (-1,2)}; %% Critical points
      \addplot [dashed, penColor2] plot coordinates {(1,-2) (1,2)}; %% Critical points

      \addplot [->, line width=10, penColor!10!background] plot coordinates {(-1+.2,-2+.2) (1-.2,2-.2)};
      \addplot [->, line width=10, penColor!10!background] plot coordinates {(-2+.2,2-.2) (-1-.2,-2+.2)};
      \addplot [->, line width=10, penColor!10!background] plot coordinates {(1+.2,2-.2) (2-.2,-2+.2)};

      \addplot [penColor3,very thick,domain=180:270] ({-1.1+.7*cos(x)}, {-.1+.7*sin(x)});
      \addplot [penColor3,very thick,domain=270:360] ({-.9+.7*cos(x)}, {-.1+.7*sin(x)});
      \addplot [penColor3,very thick,domain=90:180] ({.9+.7*cos(x)}, {.1+.7*sin(x)});
      \addplot [penColor3,very thick,domain=180:270] ({1.9+.7*cos(x)}, {.9+.7*sin(x)});
      \addplot [penColor,ultra thick,domain=-2:1,smooth] {(-x^3+3*x)*.5};
      \addplot [penColor,ultra thick,domain=1:2,smooth] {(-(x-3)^3+3*(x-3))*.5};
      
      %\addplot [very thick,penColor,smooth, domain=(-2:2)] {x^3+x^2-2*x)};
    \end{axis}
  \end{tikzpicture}
  \end{image}
  \end{explanation}
\end{example}

\end{document}
