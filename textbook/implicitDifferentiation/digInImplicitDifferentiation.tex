\documentclass{ximera}

\newcommand{\RR}{\mathbb R}
\renewcommand{\d}{\,d}
\newcommand{\dd}[2][]{\frac{d #1}{d #2}}
\renewcommand{\l}{\ell}
\newcommand{\ddx}{\frac{d}{dx}}
\newcommand{\dfn}{\textbf}
\newcommand{\eval}[1]{\bigg[ #1 \bigg]}


\title[Dig-In:]{Implicit differentiation}

\begin{document}
\begin{abstract}
In this section we differentiate equations without expressing them in
terms of a single variable.
\end{abstract}
\maketitle

The functions we've been dealing with so far have been
\textit{explicit functions}\index{explicit function}, meaning that the
dependent variable is written in terms of the independent
variable. For example:
\[
y=3x^2-2x+1,\qquad y=e^{3x}, \qquad y = \frac{x-2}{x^2-3x+2}.
\]
However, there are another type of functions, called \textit{implicit
  functions}. In this case, the dependent variable is not stated
explicitly in terms of the independent variable. For example:
\[
x^2+y^2 = 4,\qquad x^3+y^3 = 9xy, \qquad x^4+3x^2 = x^{2/3}+y^{2/3} = 1.
\]
Your inclination might be simply to solve each of these for $y$ and go
merrily on your way. However this can be difficult and it may require
two \textit{branches}, for example to explicitly plot $x^2+y^2 = 4$,
one needs both $y= \sqrt{4-x^2}$ and $y=-\sqrt{4-x^2}$. Moreover, it
may not even be possible to solve for $y$. To deal with such
situations, we use \index{implicit differentiation}\textit{implicit
  differentiation}. We'll start with a basic example.

\begin{example}
Consider the curve defined by:
\[
x^2 + y^2 = 1
\]
\begin{enumerate}
\item Compute $\dd[y]{x}$.
\item Find the slope of the tangent line at $\left(\frac{\sqrt{2}}{2},\frac{\sqrt{2}}{2}\right)$.
\end{enumerate}
\end{example}


Let's see an illustrative example:


\begin{example}
Consider the curve defined by:
\[
x^3+y^3 = 9xy
\]
\begin{image}
\begin{tikzpicture}
	\begin{axis}[
            xmin=-6,xmax=6,ymin=-6,ymax=6,
            axis lines=center,
            xlabel=$x$, ylabel=$y$,
            every axis y label/.style={at=(current axis.above origin),anchor=south},
            every axis x label/.style={at=(current axis.right of origin),anchor=west},
          ]        
          \addplot [very thick, penColor, smooth, samples=100, domain=(-.99:0)] ({9*x/(1+x^3)},{9*x^2/(1+x^3)});
          \addplot [very thick, penColor, smooth, samples=100, domain=(-.99:0)] ({9*x^2/(1+x^3)},{9*x/(1+x^3)});
          \addplot [very thick, penColor, smooth, samples=100, domain=(0:1)] ({9*x/(1+x^3)},{9*x^2/(1+x^3)});
          \addplot [very thick, penColor, smooth, samples=100, domain=(0:1)] ({9*x^2/(1+x^3)},{9*x/(1+x^3)});
        \end{axis}
\end{tikzpicture}
%% \caption{A plot of $x^3+y^3 = 9xy$. While this is not a function of
%%   $y$ in terms of $x$, the equation still defines a relation between
%%   $x$ and $y$.}
\end{image}
\begin{enumerate}
\item Compute $\dd[y]{x}$.
\item Find the slope of the tangent line at $(4,2)$.
\end{enumerate}



\begin{explanation}
Starting with 
\[
x^3+y^3 = 9xy,
\]
we apply the differential operator $\ddx$ to both sides of the
equation to obtain
\[
\ddx \left(x^3+y^3\right) = \ddx 9xy.
\]
Applying the sum rule we see
\[
\ddx x^3+\ddx y^3 = \ddx 9xy.
\]
Let's examine each of these terms in turn. To start
\[
\ddx x^3 = 3x^2.
\]
On the other hand $\ddx y^3$ is somewhat different. Here you imagine that $y = y(x)$, and hence by the chain rule
\begin{align*}
\ddx y^3 &= \ddx (y(x))^3 \\ 
&= 3(y(x))^2 \cdot y'(x) \\
&= 3y^2\dd[y]{x}.
\end{align*}
Considering the final term $\ddx 9xy$, we again imagine that $y=y(x)$. Hence 
\begin{align*}
\ddx 9xy &= 9\ddx x\cdot y(x) \\
&= 9 \left(x\cdot y'(x) + y(x)\right)\\
&= 9x \dd[y]{x} + 9y.
\end{align*}
Putting this all together we are left with the equation
\[
3x^2 + 3y^2\dd[y]{x} =9x \dd[y]{x} + 9y.
\]
At this point, we solve for $\dd[y]{x}$. Write
\begin{align*}
3x^2 + 3y^2\dd[y]{x} &= 9x \dd[y]{x} + 9y\\
3y^2\dd[y]{x} -  9x \dd[y]{x} &= 9y - 3x^2\\
\dd[y]{x}\left(3y^2-9x\right)&= 9y - 3x^2\\
\dd[y]{x} &=\frac{9y - 3x^2}{3y^2-9x} = \frac{3y - x^2}{y^2-3x}.
\end{align*}

For the second part of the problem, we simply plug $x=4$ and $y=2$
into the formula above, hence the slope of the tangent line at $(4,2)$
is $\frac{5}{4}$. We've included a plot for your viewing pleasure:
\begin{image}
\begin{tikzpicture}
	\begin{axis}[
            xmin=-6,xmax=6,ymin=-6,ymax=6,
            axis lines=center,
            xlabel=$x$, ylabel=$y$,
            every axis y label/.style={at=(current axis.above origin),anchor=south},
            every axis x label/.style={at=(current axis.right of origin),anchor=west},
          ]        
          \addplot [very thick, penColor, smooth, samples=100, domain=(-.99:0)] ({9*x/(1+x^3)},{9*x^2/(1+x^3)});
          \addplot [very thick, penColor, smooth, samples=100, domain=(-.99:0)] ({9*x^2/(1+x^3)},{9*x/(1+x^3)});
          \addplot [very thick, penColor, smooth, samples=100, domain=(0:1)] ({9*x/(1+x^3)},{9*x^2/(1+x^3)});
          \addplot [very thick, penColor, smooth, samples=100, domain=(0:1)] ({9*x^2/(1+x^3)},{9*x/(1+x^3)});

          \addplot [very thick, penColor2, domain=(-6:6)] {(5/4)*(x-4)+2};

          \addplot[color=penColor,fill=penColor,only marks,mark=*] coordinates{(4,2)};  %% closed hole          
        \end{axis}
\end{tikzpicture}
%% \caption{A plot of $x^3+y^3 = 9xy$ along with the tangent line at
%%   $(4,2)$.}
%% \label{plot:x^3+y^3=9xy}
\end{image}
\end{explanation}

\end{example}


You might think that the step in which we solve for $\dd[y]{x}$ could
sometimes be difficult---after all, we're using implicit
differentiation here instead of the more difficult task of solving the
equation $x^3+y^3=9xy$ for $y$, so maybe there are functions where
after taking the derivative we obtain something where it is hard to
solve for $\dd[y]{x}$. In fact, \textit{this never happens}. All
occurrences $\dd[y]{x}$ arise from applying the chain rule, and whenever
the chain rule is used it deposits a single $\dd[y]{x}$ multiplied by some
other expression. Hence our expression is linear in $\dd[y]{x}$, it will
always be possible to group the terms containing $\dd[y]{x}$ together and
factor out the $\dd[y]{x}$, just as in the previous example. 

\end{document}
