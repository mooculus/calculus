\documentclass{ximera}
\newcommand{\RR}{\mathbb R}
\renewcommand{\d}{\,d}
\newcommand{\dd}[2][]{\frac{d #1}{d #2}}
\renewcommand{\l}{\ell}
\newcommand{\ddx}{\frac{d}{dx}}
\newcommand{\dfn}{\textbf}
\newcommand{\eval}[1]{\bigg[ #1 \bigg]}

\author{ Nela Lakos}
\license{Creative Commons 3.0 By-NC}
\outcome{ Define a critical point.}
\outcome{ Find critical points.}
\outcome{ Define absolute maximum and absolute minimum.}
\outcome{ Find the absolute max or min of a continuous function on a closed interval.}
\outcome{ Define local maximum and local minimum.}
\outcome{ Compare and contrast local and absolute maxima and minima.}
\outcome{ Identify situations in which an absolute maximum or minimum is guaranteed.}
\outcome{ Classify critical points.}
\outcome{ State the First Derivative Test.}
\outcome{ Apply the First Derivative Test.}
\begin{document}

\begin{exercise}

Consider the function $f(x) =x^{1/3}-x$. \\
Find the x-coordinates of the global maximum and global minimum of $f$ on the interval $ [-1,1]$ .
\begin{hint}
First, compute the derivative of $f$.
$$
f'(x) = \answer{\frac{1}{3}}x^{-\frac{2}{3}}-\answer{1}=\frac{1}{\answer{3}x^{\frac{2}{3}}}-\answer{1}
$$
The function $f$ has three critical points in the set $(-1,1)$. Note, $f'$ is undefined at one of them.
 If we call these points $a$, $b$, and $c$ and order them such
that $a < b<c $, then
$$
a = \answer{-\frac{1}{3\sqrt{3}}}
$$

$$
b=\answer{0}
$$
$$
c=\answer{\frac{1}{3\sqrt{3}}}
$$
\end {hint}

\begin{hint}

Next, we evaluate $f$ at the end points and at the critical points.
$$
f(-1) = \answer{0}
$$
$$
f(1)=\answer{0}
$$
$$
f(a)=\frac{\answer{-2}}{3\sqrt{3}}
$$
$$
f(b)=\answer{0}
$$
$$
f(c)=\frac{\answer{2}}{3\sqrt{3}}
$$
By comparing these values, you can answer the question.
\end {hint}

The function $f$ attains its global maximum at	
\begin{prompt}
$x=\answer{\frac{1}{3\sqrt{3}}}$
\end{prompt}
The function $f$ attains its global minimum at	
\begin{prompt}
$x=\answer{\frac{-1}{3\sqrt{3}}}$.
\end{prompt}
\end{exercise}
\end{document}
