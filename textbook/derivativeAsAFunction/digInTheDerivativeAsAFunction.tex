\documentclass{ximera}

\newcommand{\RR}{\mathbb R}
\renewcommand{\d}{\,d}
\newcommand{\dd}[2][]{\frac{d #1}{d #2}}
\renewcommand{\l}{\ell}
\newcommand{\ddx}{\frac{d}{dx}}
\newcommand{\dfn}{\textbf}
\newcommand{\eval}[1]{\bigg[ #1 \bigg]}


\outcome{}

\title[Dig-in:]{The derivative as a function}

\begin{document}
\begin{abstract}
\end{abstract}
\maketitle

We know that to find the derivative of a function at a point $x=a$ we write
\[
f'(a) = \lim_{h\to 0}\frac{f(a+h)-f(a)}{h}.
\]
However, if we replace the given number $a$ with a varible $x$, we now
have
\[
f'(x) = \lim_{h\to 0}\frac{f(x+h)-f(x)}{h}.
\]
This tells us the instantaneous rate of change at any given point
$x$. Given a function $f$ from the real numbers to the real numbers,
the derivative $f'$ is also a function from the real numbers to the
real numbers. The upshot is that we may study a function by examining
the graph of its derivative.

graphs of derivatvies

(postive -> Increaseing)



\begin{defintion}
  $f'(3)$ means take the derivative of $f$, and evaluate at $x=3$.
\end{defintion}
\end{document}

