\documentclass{ximera}

\newcommand{\RR}{\mathbb R}
\renewcommand{\d}{\,d}
\newcommand{\dd}[2][]{\frac{d #1}{d #2}}
\renewcommand{\l}{\ell}
\newcommand{\ddx}{\frac{d}{dx}}
\newcommand{\dfn}{\textbf}
\newcommand{\eval}[1]{\bigg[ #1 \bigg]}


\outcome{Compute the derivative of a vector-valued function.}

\author{Jim Talamo}

\begin{document}
\begin{exercise}

Consider the curve $C = \left\{(x,y) \in \R^2 : y=2x^2-7\right\}$ .

Find a Cartesian representation of the tangent line at $x=2$.  Express you final answer in the form $y=mx+b$.

\[
y= \answer{8x-15}
\]

\begin{exercise}
Now, require that $x(t) =t$.  

A parametric equation that traces out $C$ is given by

\[
\vec{r}(t) = \vector{\answer{t},\answer{2t^2-7}}.
\]

A parameterization of the tangent line to the curve where $x=2$ is

\[
\vecl(t)=\vector{\answer{t+2},\answer{8t+1}}.
\]

\begin{hint}
The curve passes through $x=2$ when $t=\answer{2}$.  

\begin{itemize}
\item A vector $\vec{v}$ parallel to the tangent line at $x=2$ is found by evaluating \wordChoice{\choice{$\vec{r}(t)$}\choice[correct]{$\vec{r}'(t)$}} when $t=\answer{2}$.
\item A point $P_0$ on the tangent line is found by evaluating \wordChoice{\choice[correct]{$\vec{r}(t)$}\choice{$\vec{r}'(t)$}} when $t=\answer{2}$.
\end{itemize}

A parametric description of the tangent line can be found from 

\[\vecl(t) = \vec{v}t+\vec{P}_0\]

\end{hint}

\begin{feedback}[correct]
Do the answers to Part I and Part II describe the same line?  Intuitively, they should since the tangent line at the point on the curve where $x=2$ should not depend on how we choose to describe the curve.

To check this, we can determine if the set of parametric equations $x(t)=t+2$ and $y(t)=8t+1$ satisfy the equation $y=8x-15$ from the first part.

\[
8[x(t)]-15 = 8 [t+2] -15 = 8t+16-15 = 8t+1=y(t).
\] 

Thus, the two equations describe the same line.
\end{feedback}
\end{exercise}
\end{exercise}
\end{document}
