\documentclass{ximera}
\newcommand{\RR}{\mathbb R}
\renewcommand{\d}{\,d}
\newcommand{\dd}[2][]{\frac{d #1}{d #2}}
\renewcommand{\l}{\ell}
\newcommand{\ddx}{\frac{d}{dx}}
\newcommand{\dfn}{\textbf}
\newcommand{\eval}[1]{\bigg[ #1 \bigg]}

\author{Steven Gubkin\and Nela Lakos \and Bobby Ramsey}
\license{Creative Commons 3.0 By-NC}

\outcome{Define linear approximation as an application of the tangent to a curve.}
\outcome{Find the linear approximation to a function at a point and use it to approximate the function value.}
\outcome{Identify when a linear approximation can be used.}


\begin{document}
\begin{exercise}
Approximate $\sin(\pi+0.3)$ by hand using the idea of linear
approximation (of a conveniently chosen function, and base point).
\begin{hint}
Let $f(x)=\sin{x}$ and $a=\pi$.
\end{hint}
\begin{hint}
Evaluate.
 $f(a)=\answer{0}$,
 $f'(x)=\cos{x}$,
  $f'(a)=\answer{-1}$.
\end{hint}
\begin{hint}
Find an expression for $L(x)$, the linear approximation of $f$ at $a$.
\end{hint}
\begin{hint}
 $L(x)=f(\pi)+f'(\pi)\cdot(\answer{x-\pi})$
\end{hint}
\begin{hint}
 $L(\pi+0.3)=-(\answer{0.3})$
\end{hint}
\begin{hint}
Approximate. 
 $f(\pi+0.3)\approx L(\pi+0.3)$
\end{hint}
\begin{prompt}
	$$\sin(\pi+0.3) \approx \answer{-0.3}$$
\end{prompt}

This is an
\begin{multipleChoice}
   \choice[correct]{underestimate}
   \choice{overestimate}
\end{multipleChoice}
because the graph of $f$ is 
\begin{multipleChoice}
   \choice[correct]{Concave Up}
   \choice{Concave Down}
\end{multipleChoice}
on the interval $\left( \answer{\pi}, \answer{\pi + 0.3} \right)$.

\end{exercise}
\end{document}
