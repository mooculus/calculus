\documentclass{ximera}

\newcommand{\RR}{\mathbb R}
\renewcommand{\d}{\,d}
\newcommand{\dd}[2][]{\frac{d #1}{d #2}}
\renewcommand{\l}{\ell}
\newcommand{\ddx}{\frac{d}{dx}}
\newcommand{\dfn}{\textbf}
\newcommand{\eval}[1]{\bigg[ #1 \bigg]}


\author{Jim Talamo}
\license{Creative Commons 3.0 By-bC}


\outcome{}


\begin{document}
\begin{exercise}
Consider the parametric equations 
\begin{align*}
x(t) &= t^2-1\\
y(t) &= t^3-t
\end{align*}

For $-\infty < t < \infty$.

First note that the point $(0,0)$ belongs to the curve. For what value of $t$ does the curve pass through the origin?

Solving $0=t^2-1$ and $0=t^3-t$ we obtain $t=1$ and $t=-1$. If we think of the parameter $t$ as time, this means that as our 
curve is being traced out in time, it actually goes through the origin twice. Thus our curve intersects itself at the origin. 

Below is a portion of the curve traced out for certain values of the parameter $t$.

\begin{image}  
  \begin{tikzpicture}  
    \begin{axis}[  
        xmin=-1.4,  
        xmax=1.4,  
        ymin=-1.2,  
        ymax=1.2,  
        axis lines=center,  
        xlabel=$x$,  
        ylabel=$y$,  
        every axis y label/.style={at=(current axis.above origin),anchor=south},  
        every axis x label/.style={at=(current axis.right of origin),anchor=west},  
      ]  
      \addplot [ultra thick, penColor, smooth, domain=(-1.3:1.3)] ({x^2-1},{x^3-x});
    \end{axis}  
  \end{tikzpicture}  
\end{image} 



Because of this, it doesn't make sense to ask what the tangent line is at the origin because there is no well defined tangent line at $(0,0)$ if we simply regard our curve as just a set of points.That is, if we zoom in on the origin, our curve does not look approximately like a line. However if we use the parametric description of the curve then we can make sense of a tangent line at two different points in time. 

Let's calculate the equation of the tangent line to the curve when $t=1$. 

We calculate $\dd[y]{x}=\frac{y'(t)}{x'(t)}=\answer{ \frac{3t^2-1}{2t}}$. The slope when $t=1$ is $\answer{1}$. Thus the tangent line to the curve when $t=1$ is
$y-\answer{0}=\answer{1}(x-\answer{0})$. 

We see the tangent line to the curve when $t=1$ below in orange

\begin{image}  
  \begin{tikzpicture}  
    \begin{axis}[  
        xmin=-1.4,  
        xmax=1.4,  
        ymin=-1.2,  
        ymax=1.2,  
        axis lines=center,  
        xlabel=$x$,  
        ylabel=$y$,  
        every axis y label/.style={at=(current axis.above origin),anchor=south},  
        every axis x label/.style={at=(current axis.right of origin),anchor=west},  
      ]  
      \addplot [ultra thick, penColor, smooth, domain=(-1.3:1.3)] ({x^2-1},{x^3-x});
      \addplot [ultra thick, penColor5, smooth,domain=-1:1] {x};  
    \end{axis}  
  \end{tikzpicture}  
\end{image} 

We can also find the tangent line to the curve when $t=-1$ using the same method as above. The tangent line is $y-\answer{0}=\answer{-1}(x-\answer{0})$. This line is shown below in purple. 

\begin{image}  
  \begin{tikzpicture}  
    \begin{axis}[  
        xmin=-1.4,  
        xmax=1.4,  
        ymin=-1.2,  
        ymax=1.2,  
        axis lines=center,  
        xlabel=$x$,  
        ylabel=$y$,  
        every axis y label/.style={at=(current axis.above origin),anchor=south},  
        every axis x label/.style={at=(current axis.right of origin),anchor=west},  
      ]  
      \addplot [ultra thick, penColor, smooth, domain=(-1.3:1.3)] ({x^2-1},{x^3-x});
      \addplot [ultra thick, penColor3, smooth,domain=-1:1] {-x};  
    \end{axis}  
  \end{tikzpicture}  
\end{image} 

This shows one advantage to using a parametric description of a curve. Even thoug the curve as a static set of points has no well defined tangent line at the origin, we can still make sense of the tangent line if we use a parametric description of the curve that presents the curve as being traced out in time. 

\begin{exercise}
Find all points where the curve has a vertical tangent line. 

We have already found that $\dd[y]{x}=\frac{3t^2-1}{2t}$. This gives the slope of the tangent line at the point associated to $t$. A vertical tangent line is obtained when the denominator vanishes and the numerator does not. This essentially corresponds to line with infinite slope. 

This means that we have a vertical tangent line when $t=\answer{0}$. This corresponds to the point $(\answer{-1}, \answer{0})$.  The equation of this tangent line is $x=\answer{-1}$. 

\begin{image}  
  \begin{tikzpicture}  
    \begin{axis}[  
        xmin=-1.4,  
        xmax=1.4,  
        ymin=-1.2,  
        ymax=1.2,  
        axis lines=center,  
        xlabel=$x$,  
        ylabel=$y$,  
        every axis y label/.style={at=(current axis.above origin),anchor=south},  
        every axis x label/.style={at=(current axis.right of origin),anchor=west},  
      ]  
      \addplot [ultra thick, penColor, smooth, domain=(-1.3:1.3)] ({x^2-1},{x^3-x});
      \addplot +[ultra thick, mark=none] coordinates {(-1, -1) (-1, 1)};  
    \end{axis}  
  \end{tikzpicture}  
\end{image} 

\begin{exercise}

Find all points where the curve has a horizontal tangent line. 

We use $\dd[y]{x}=\frac{3t^2-1}{2t}$. A horizontal tangent line has slope $0$ so we need the numerator to be zero while the denominator is nonzero. Thus we see that 
we have a horizontal tangent line when $t=\answer{\frac{1}{\sqrt{3}}}$ and $t=\answer{-\frac{1}{\sqrt{3}}}$. 

The positive $t$ value corresponds to the point $\left( \answer{ \left( \frac{1}{\sqrt{3}} \right)^2-1 } ,  \answer{ \left( \frac{1}{\sqrt{3}} \right)^3 -   \left( \frac{1}{\sqrt{3}} \right)   } \right)$. The equation of the tangent line to this point is $y=\answer{ \left( \frac{1}{\sqrt{3}} \right)^3 -   \left( \frac{1}{\sqrt{3}} \right) }$. This line is show in green below. 


The negative $t$ value corresponds to the point $\left( \answer{ \left( \frac{-1}{\sqrt{3}} \right)^2-1 } ,  \answer{ \left( \frac{-1}{\sqrt{3}} \right)^3 -   \left( \frac{-1}{\sqrt{3}} \right)   } \right)$. 
The equation of the tangent line to this point is $y=\answer{ \left( \frac{-1}{\sqrt{3}} \right)^3 -   \left( \frac{-1}{\sqrt{3}} \right) }$. This line is shown in orange below. 






\begin{image}  
  \begin{tikzpicture}  
    \begin{axis}[  
        xmin=-1.4,  
        xmax=1.4,  
        ymin=-1.2,  
        ymax=1.2,  
        axis lines=center,  
        xlabel=$x$,  
        ylabel=$y$,  
        every axis y label/.style={at=(current axis.above origin),anchor=south},  
        every axis x label/.style={at=(current axis.right of origin),anchor=west},  
      ]  
      \addplot [ultra thick, penColor, smooth, domain=(-1.3:1.3)] ({x^2-1},{x^3-x});
      \addplot [ultra thick, penColor4, smooth] { -.385};  
      \addplot [ultra thick, penColor5, smooth] {.385};
    \end{axis}  
  \end{tikzpicture}  
\end{image} 











\end{exercise}
\end{exercise}
\end{exercise}
\end{document}
