\documentclass{ximera}

\newcommand{\RR}{\mathbb R}
\renewcommand{\d}{\,d}
\newcommand{\dd}[2][]{\frac{d #1}{d #2}}
\renewcommand{\l}{\ell}
\newcommand{\ddx}{\frac{d}{dx}}
\newcommand{\dfn}{\textbf}
\newcommand{\eval}[1]{\bigg[ #1 \bigg]}


\begin{document}

\outcome{Classify critical points.}
\outcome{Apply the First Derivative Test.}
\outcome{Apply the Second Derivative Test.}
\outcome{Find inflection points.}
\outcome{Identify situations in which an absolute maximum or minimum is guaranteed.}

\begin{exercise}
Let $f(x)=x\ln x$. The domain of $f$ is
$(\answer{0},\answer{\infty})$.
\begin{exercise}
$f$ is \wordChoice{\choice{not everywhere continuous on its domain}\choice{nowhere continuous on its domain}\choice{continuous on $[0,\infty)$}\choice[correct]{everywhere continuous on its domain}} because \wordChoice{\choice{the limit $\lim_{x\to 0^+}f(x)$ does not exist} \choice{the natural logarithm is not continuous}\choice{the limit $\lim_{x\to 0^+}f(x)$ exists and is equal to $0$}\choice{$f(x)$ is the product of two functions which are continuous on its domain}}.
\begin{exercise}
Now we shall compute the limit $\lim_{x\to 0^{+}}f(x)$.

First, write $f(x)=x\ln x$ as $x\ln x=\frac{\ln x}{1/x}$. Then the limit $\lim_{x\to 0^+}\frac{\ln x}{1/x}$ is of \wordChoice{\choice{a determinant form}\choice[correct]{an indeterminate form}}. 
\begin{exercise}
In particular, $\lim_{x\to 0^+}\frac{\ln x}{1/x}$ is of the form \wordChoice{\choice[correct]{$\frac{0}{0}$}\choice{nonzero over zero}}.
\begin{exercise}
Next, we shall classify the extrema of $f$ and determine whether it has any global extrema.

The function $f$ has $\answer{1}$ critical point(s). 
\begin{exercise}
The single critical point of $f$ is located at
\[
x=\answer{1/e}
\]
Because the value of $f''(\frac{1}{e})$ is \wordChoice{\choice[correct]{positive}\choice{negative}\choice{zero}}, the local extremum $f$ has at $x=1/e$ is \wordChoice{\choice{a local maximum}\choice[correct]{a local minimum}\choice{neither a local maximum nor a local minimum}\choice{possibly a local maximum, a local minimum or neither}}.
\begin{exercise}
Since $f$ is decreasing on the interval $(\answer{0},\answer{1/e})$ and is increasing on the interval $(\answer{1/e},\answer{\infty})$, $f(x)$ has \wordChoice{\choice{no global minimum}\choice[correct]{a global minimum}} at $x=\answer{1/e}$. The global minimum of $f$ is $\answer{-1/e}$.
\begin{exercise}
On the other hand, the function $\ln x$ is \wordChoice{\choice[correct]{increasing on $(0,\infty)$}\choice{decreasing on $(0,\infty)$}\choice{increasing on $(0,e)$ and decreasing on $(e,\infty)$}\choice{decreasing on $(0,e)$ and increasing on $(e,\infty)$}} and $\ln e=\answer{1}$. 
\begin{exercise}
Thus, for all $x>e$, $f(x)\ge x\ln e$ and $x\ln e=x$. Therefore on $(e,\infty)$, $f(x)$ is positive and  \wordChoice{\choice{bounded}\choice[correct]{unbounded}}. Therefore $f(x)$ \wordChoice{\choice{has a global maximum}\choice{has no global maximum, but does have a local maximum}\choice[correct]{has no global maximum}\choice{has no global maximum, but may have a local maximum}}.
\begin{exercise}
Finally, we shall identify the inflection points of $f$ and the intervals on which $f$ is concave up or concave down.

The function $f$ has $\answer{0}$ inflection point(s).
\begin{exercise}
Since
\[
\frac{d^2}{dx^2}f(x)=\answer{1/x}
\]
$f''(x)$ is \wordChoice{\choice[correct]{positive}\choice{negative}} on $(0,\infty)$. 
\begin{exercise}
Since $f''(x)>0$ for all $x>0$, $f$ is \wordChoice{\choice[correct]{concave up}\choice{concave down}} on $(\answer{0},\answer{\infty})$.
\end{exercise}
\end{exercise}
\end{exercise}
\end{exercise}
\end{exercise}
\end{exercise}
\end{exercise}
\end{exercise}
\end{exercise}
\end{exercise}
\end{exercise}
\end{exercise}
\end{document}
