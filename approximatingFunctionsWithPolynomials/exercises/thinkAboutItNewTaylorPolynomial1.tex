\documentclass{ximera}

\newcommand{\RR}{\mathbb R}
\renewcommand{\d}{\,d}
\newcommand{\dd}[2][]{\frac{d #1}{d #2}}
\renewcommand{\l}{\ell}
\newcommand{\ddx}{\frac{d}{dx}}
\newcommand{\dfn}{\textbf}
\newcommand{\eval}[1]{\bigg[ #1 \bigg]}


\author{Jim Talamo}
\license{Creative Commons 3.0 By-bC}


\outcome{Gain practice understanding the Taylor Polynomial coefficient formula}

%%%%This exercise is linked to the composition rule exercise <INSERT HERE>%%%%%%%
\begin{document}
\begin{exercise}
The second degree Taylor Polynomial for $f(x)$ centered at $x=0$ is:

\[
p_2(x) =3+x-2x^2
\]
Then, the second degree Taylor polynomial, $P_2(x)$ for $g(x) = (x-3)f(x)$ centered at $x=0$ is:

\[
P_2(x) = \answer{-9 +7x^2}
\]

\begin{hint}
From the Taylor polynomial for $f(x)$, we find that:

\[
f(0) = \answer{3} \qquad \qquad f'(0) = \answer{1} \qquad \qquad f''(0) = \answer{-4}
\]

since the coefficients of the Taylor Polynomial and the values of the derivatives of the function are related via the formula:

\[
a_k = \frac{f^{(k)}(c)}{k!}
\]

\begin{question}
Now, the second degree Taylor polynomial for $g(x)$ will be of the form:

\[
P_2(x) = b_0+b_1x+b_2x^2
\]
where $b_k = \frac{g^{(k)}(0)}{k!}$.

So, let's start by computing the various derivatives of $g(x)$ and evaluating them at $x=0$:

\begin{question}
\begin{itemize}
\item For $g(0)$:

$g(x) = (x-3) \cdot f(x)$ so $g(0) = \answer{-3} \cdot f(0)$.  thus, $g(0) = \answer{-9}$.
\end{itemize}
\end{question}

\begin{question}
\begin{itemize}
\item For $g'(0)$:

$g'(x) = \answer{1}\cdot f(x)+(\answer{x-3}) \cdot f'(x)$ so $g'(0) = f(0)+\answer{-3} \cdot f'(0)$.  Thus, $g'(0) = \answer{0}$.
\end{itemize}
\end{question}

\begin{question}
\begin{itemize}
\item For $g''(0)$:

$g''(x) =  \answer{2} f'(x)+(\answer{x-3}) \cdot f''(x)$ so $g''(0) = \answer{2} \cdot f'(0)+ \answer{-3} \cdot f''(0)$.  Thus, $g''(0) = \answer{14}$.
\end{itemize}
\end{question}

\begin{question}
Since $b_k = \frac{g^{(k)}(0)}{k!}$, we find:

\[
b_0 = \answer{-9} \qquad \qquad b_1 = \answer{0} \qquad \qquad b_2 = \answer{7}
\]

Thus, the second degree Taylor polynomial for $g(x)$ is:

\[
P_2(x) = \answer{-9+7x^2}
\]
\end{question}
\end{question}
\end{hint}

\end{exercise}
\end{document}
