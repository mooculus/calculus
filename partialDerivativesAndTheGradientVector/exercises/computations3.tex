\documentclass{ximera}

\newcommand{\RR}{\mathbb R}
\renewcommand{\d}{\,d}
\newcommand{\dd}[2][]{\frac{d #1}{d #2}}
\renewcommand{\l}{\ell}
\newcommand{\ddx}{\frac{d}{dx}}
\newcommand{\dfn}{\textbf}
\newcommand{\eval}[1]{\bigg[ #1 \bigg]}



\author{Jim Talamo}

\outcome{Compute the partial derivative of an expression.}
\outcome{Determine whether product, quotient, or chain rule is necessary.}
\newcommand{\Pp}[2]{\frac{\partial #1}{\partial #2}}

\begin{document}

This exercise gives mechanical practice for products, quotients and compositions and explores when product, quotient, or chain rule is necessary.

%%%%%%%%%%%%%%%%%%%%%

\begin{exercise}
If $f(x,y) = 3y^2 \tan(x+2y)$, then 

\begin{itemize}
\item $\Pp{f}{x} = \answer{3y^2\sec^2(x+2y)}$
\item $\Pp{f}{y} = \answer{6y \tan(x+2y) + 6y^2 \sec^2(x+2y)}$
\end{itemize}

\begin{hint}
For $f_x(x,y)$, note that we do have a product, but the product rule is not necessary since $3y^2$ does not depend on $x$.  Thus,

\[
\Pp{f}{x} = \pp{x} \left[ 3y^2 \tan(x+2y) \right] = 3y^2 \cdot \pp{x} \left[ \tan(x+2y) \right] = 3y^2  \cdot \sec^2(x+2y) .
\]

On the other hand, for $f_x(x,y)$, the product rule is necessary since $3y^2$ does depend on $y$.  Thus,

\[
\Pp{f}{y} = \pp{y} \left[ 3y^2 \tan(x+2y) \right] =   \pp{y}  \left[ 3y^2 \right]  \cdot \tan(x+2y)+ 3y^2 \cdot \pp{y} \left[ \tan(x+2y) \right] 
\]
\end{hint}
\end{exercise}

%%%%%%%%%%%%%%%%%%%%%

\begin{exercise}
If $f(x,y) = \frac{(2x+8y^4)^3}{4\sec(x)}$, then 

\begin{itemize}
\item $f_x(x,y) = \answer{\frac{24(2x+8y^4)^2 \sec(x) - 4\sec(x)\tan(x) (2x+8y^4)^3}{16 \sec^2(x)}  }$
\item $f_y(x,y) = \answer{\frac{6y^3 \big(2x+8y^4\big)^2}{\sec(x)}}$
\end{itemize}


\end{exercise}

%%%%%%%%%%%%%%%%%%%%%

\begin{exercise}
If $f(x,y) = \frac{\sqrt{4x^2y+1}}{y^3}$, then 

\begin{itemize}
\item $\Pp{f}{x} = \answer{\frac{4x}{y^2} (4x^2y+1)^{-1/2} }$
\item $\Pp{f}{y} = \answer{\frac{2x^2y^3 (4x^2y+1)^{-1/2} - 3y^2\sqrt{4x^2y+1}}{y^6} }$
\end{itemize}


\end{exercise}

\end{document}
