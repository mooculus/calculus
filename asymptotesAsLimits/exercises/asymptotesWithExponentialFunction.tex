\documentclass{ximera}

\newcommand{\RR}{\mathbb R}
\renewcommand{\d}{\,d}
\newcommand{\dd}[2][]{\frac{d #1}{d #2}}
\renewcommand{\l}{\ell}
\newcommand{\ddx}{\frac{d}{dx}}
\newcommand{\dfn}{\textbf}
\newcommand{\eval}[1]{\bigg[ #1 \bigg]}


\outcome{Discuss what it is a limit at $\infty$.}
\outcome{Understand the relationship between limits and horizontal asymptotes.}
\outcome{Evaluate the limit as $x$ approaches $\infty$.}
\outcome{Recognize when a limit is indicating there is a horizontal asymptote.}

\author{Nela Lakos}

\begin{document}
\begin{exercise}

Let $f$ be defined by
\[
f(x) = \frac{e^x}{1+2e^x}
\]


Determine if the following limits exist and if they do, compute them analytically using the limit laws and techniques for computing limits.  Possible answers are any number, $\infty$, $-\infty$, and $DNE$.

\begin{exercise}
\[
\lim_{x\to-\infty}f(x) = \answer{0}
\]
\begin{exercise}
\[
\lim_{x\to+\infty}f(x) = \answer{\frac{1}{2}}
\]
\begin{hint}
NOTE: 

$\lim_{x\to+\infty}\frac{e^x}{1+2e^x} =\lim_{x\to+\infty }\frac{1}{\frac{1}{e^x}+2}$

\end{hint}
\end{exercise}
\end{exercise}



\begin{exercise}
By what we calculated above, the horizontal asymptotes (written in the same order as the limits above) of $f$ are 
\[
y=\answer{0}
\]
and
\[
y=\answer{\frac{1}{2}}
\]
\end{exercise}

\begin{exercise}
The function $f$ has
\begin{multipleChoice}
\choice[correct]{NO vertical asymptotes.}
\choice{has a vertical asymptote at $x=-\ln{2}$.}
\end{multipleChoice}
\begin{hint}
Note: The exponential function, $e^x$ is always positive. Therefore, the denominator of $f$ is never equal to $0$. 
\end{hint}
\end{exercise}
\begin{exercise}
The function $f$ 
\begin{multipleChoice}
\choice[correct]{is continuous}
\choice{is not continuous}
\end{multipleChoice}
 on the  entire interval $(-\infty,+\infty)$?

\begin{hint}
Note: The exponential function, $e^x$, is always positive. Therefore, the denominator of $f$ is never equal to $0$. 
\end{hint}
\end{exercise}
\end{exercise}
\end{document}