\documentclass{ximera}
\newcommand{\RR}{\mathbb R}
\renewcommand{\d}{\,d}
\newcommand{\dd}[2][]{\frac{d #1}{d #2}}
\renewcommand{\l}{\ell}
\newcommand{\ddx}{\frac{d}{dx}}
\newcommand{\dfn}{\textbf}
\newcommand{\eval}[1]{\bigg[ #1 \bigg]}


\author{Jim Talamo}
\license{Creative Commons 3.0 By-NC}
\outcome{}
\begin{document}

\begin{exercise}

The following exercise reviews an important convention involving square roots and shows a few situations in which this arises.
 
\begin{fact}
By convention, the square root is taken to be positive.
\end{fact}
 
Question: What is $\sqrt{9}$?
 \begin{multipleChoice}
\choice{$-3$}
 \choice[correct]{$3$}
 \choice{$-3$ and $3$}
 \end{multipleChoice}
 
When we are solving equations in which square roots arise and we want a negative result, we have to include both the positive and negative square roots.
 
Question: What is the solution to $x^2=9$?
  \begin{multipleChoice}
\choice{$x=-3$}
 \choice{$x=3$}
 \choice[correct]{$x=-3$ and $x=3$}
 \end{multipleChoice}
 
\begin{exercise}
In many applications, we will have curves that are described as functions of a particular variable, and we will need to solve for the other variable, as in the following example.

Suppose we are given the picture below.


 \begin{image}
            \begin{tikzpicture}
            	\begin{axis}[
            		domain=-2.6:1.8, ymax=3.5,xmax=1.8, ymin=-2.4, xmin=-2.6,
            		axis lines =center, xlabel=$x$, ylabel=$y$,
            		every axis y label/.style={at=(current axis.above origin),anchor=south},
            		every axis x label/.style={at=(current axis.right of origin),anchor=west},
            		axis on top,
            		]
                      
            	\addplot [draw=penColor2,very thick,smooth] {3-x^2};
            	\addplot [draw=penColor,very thick,smooth] {x+1};
	                            
            	\addplot [name path=A,domain=-2:1,draw=none] {3-x^2)};   
            	\addplot [name path=B,domain=-2:1,draw=none] {x+1};
            	\addplot [fillp] fill between[of=A and B];
                      
                      
            	\node at (axis cs:-1.5,2.5) [penColor2] {$y=3-x^2$};
            	\node at (axis cs:.7,1) [penColor] {$y=x+1$};
	    
	      \end{axis}
            \end{tikzpicture}
\end{image}

Since the parabola \wordChoice{\choice[correct]{passes}\choice{does not pass}} the vertical line test, it \wordChoice{\choice[correct]{can}\choice{cannot}} be described by a single function of $x$.

However, since the parabola \wordChoice{\choice{passes}\choice[correct]{does not pass}} the horizontal line test, it \wordChoice{\choice{can}\choice[correct]{cannot}} be described by a single function of $y$.  In fact, if $a < 3$, any horizontal line $y=a$ will intersect the parabola twice, so we expect that $\answer{2}$ functions of $y$ will be needed to describe the parabola.

Algebraically, we can see where this arises by solving for $x$.

\begin{align*}
y &= 3-x^2 \\
x^2 &=\answer{3-y} \\
x &= \pm \sqrt{3-y}
\end{align*}

Note that we have to take the positive and negative square root. 

\begin{itemize}
\item  For the function $x=-\sqrt{3-y}$, we will only have the \wordChoice{\choice[correct]{left}\choice{right}} half of the parabola, since by definition, the $x$-values obtained from $x=-\sqrt{3-y}$ MUST be negative.
\item  For the function $x=+\sqrt{3-y}$, we will only have the \wordChoice{\choice{left}\choice[correct]{right}} half of the parabola, since by definition, the $x$-values obtained from $x=+\sqrt{3-y}$ MUST be positive.
\end{itemize}

We update the picture now and describe all curves using functions of $y$.

 \begin{image}
            \begin{tikzpicture}
            	\begin{axis}[
            		domain=-2.6:1.8, ymax=3.5,xmax=1.8, ymin=-2.4, xmin=-2.6,
            		axis lines =center, xlabel=$x$, ylabel=$y$,
            		every axis y label/.style={at=(current axis.above origin),anchor=south},
            		every axis x label/.style={at=(current axis.right of origin),anchor=west},
            		axis on top,
            		]
		
            	\addplot [draw=penColor4,very thick,smooth,domain=-2.6:0] {3-x^2};                      
            	\addplot [draw=penColor2,very thick,smooth,domain=0:1.6] {3-x^2};
            	\addplot [draw=penColor,very thick,smooth,domain=-2.6:1.3] {x+1};
	                            
            	\addplot [name path=A,domain=-2:1,draw=none] {3-x^2)};   
            	\addplot [name path=B,domain=-2:1,draw=none] {x+1};
            	\addplot [fillp] fill between[of=A and B];
                      
             	\node at (axis cs:-1.7,2.5) [penColor4] {$x=-\sqrt{3-y}$};                     
            	\node at (axis cs:1,3.2) [penColor2] {$x=+\sqrt{3-y}$};
            	\node at (axis cs:.7,1) [penColor] {$x=y-1$};
	    
	      \end{axis}
            \end{tikzpicture}
            \end{image}

\end{exercise}
\end{exercise}
\end{document}