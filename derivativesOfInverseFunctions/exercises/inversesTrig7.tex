\documentclass{ximera}

\newcommand{\RR}{\mathbb R}
\renewcommand{\d}{\,d}
\newcommand{\dd}[2][]{\frac{d #1}{d #2}}
\renewcommand{\l}{\ell}
\newcommand{\ddx}{\frac{d}{dx}}
\newcommand{\dfn}{\textbf}
\newcommand{\eval}[1]{\bigg[ #1 \bigg]}


\outcome{Recall the meaning and properties of inverse trigonometric functions.}

\author{Nela Lakos \and Kyle Parsons \and Bobby Ramsey}

\begin{document}
\begin{exercise}

Use a right triangle to find the exact value of $\cos(\tan^{-1}(5))$.
\[
\cos(\tan^{-1}(5)) = \answer{\frac{1}{\sqrt{26}}}.
\]

In the same way, for $x>0$ use a right triangle to simplify $\cos(\tan^{-1}(x))$.
\[
\cos(\tan^{-1}(x)) = \answer{\frac{1}{\sqrt{1+x^2}}}.
\]

Find the derivative using Chain Rule.
\[
\ddx\left[\vphantom{\ddx}\sin(\tan^{-1}(x))\right] = \answer{\cos(\arctan(x))\frac{1}{1+x^2}}
\]


For $x>0$ simplify $\sin(\tan^{-1}(x))$.
\[
\sin(\tan^{-1}(x)) = \answer{\frac{x}{\sqrt{1+x^2}}}.
\]

Use this expression to take the derivative of $\sin(\tan^{-1}(x))$
\[
\ddx\left[\vphantom{\ddx}\sin(\tan^{-1}(x))\right] = \ddx\left[\vphantom{\ddx}\answer{\frac{x}{\sqrt{1+x^2}}}\right] = \answer{\frac{1}{(1+x^2)^{\frac{3}{2}}}}.
\]

Using the expression found above for $\cos(\tan^{-1}(x))$, these two derivatives are the same.






\end{exercise}
\end{document}