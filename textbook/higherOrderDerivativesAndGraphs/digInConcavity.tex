\documentclass{ximera}

\newcommand{\RR}{\mathbb R}
\renewcommand{\d}{\,d}
\newcommand{\dd}[2][]{\frac{d #1}{d #2}}
\renewcommand{\l}{\ell}
\newcommand{\ddx}{\frac{d}{dx}}
\newcommand{\dfn}{\textbf}
\newcommand{\eval}[1]{\bigg[ #1 \bigg]}


\title[Dig-In:]{Concavity}

\begin{document}
\begin{abstract}
\end{abstract}
\maketitle



We know that the sign of the derivative tells us whether a function is
increasing or decreasing. Likewise, the sign of the second derivative
$f''(x)$ tells us whether $f'(x)$ is increasing or decreasing. We summarize this in the table below:

\index{concave up/down}

\[
{\setlength{\arrayrulewidth}{5pt}
\begin{array}{c|c|c|} %% gives thick lines
 & f'(x)<0 & f'(x) > 0 \\ \hline & & \\[-1.5ex]
f''(x)> 0 & 
\begin{minipage}{2in}
\[
\begin{tikzpicture}
	\begin{axis}[
            clip=false,
            height=4.5cm,
            domain=0:1,
            ymax=1,
            ymin=0,
            axis lines=none,
          ]
          \addplot [very thick, penColor, smooth] {(x-1)^2};
          \node at (axis cs:.7,.4) [textColor] {\footnotesize Concave Up};
        \end{axis}
\end{tikzpicture}
\]
\begin{minipage}{2in}\footnotesize
Here $f'(x)<0$ and $f''(x)>0$. This means that $f(x)$ slopes down and
is getting \textit{less steep}. In this case the curve is
\textbf{concave up}.
\end{minipage}
\end{minipage}
&
\begin{minipage}{2in}
\[
\begin{tikzpicture}
	\begin{axis}[
            clip=false,
            domain=0:1,
            ymax=1,
            height=4.5cm,
            ymin=0,
            axis lines=none,
          ]
          \addplot [very thick, penColor, smooth] {x^2};
          \node at (axis cs:.3,.4) [textColor] {\footnotesize Concave Up};
        \end{axis}
\end{tikzpicture}
\]
\begin{minipage}{2in}\footnotesize
Here $f'(x)>0$ and $f''(x)>0$. This means that $f(x)$ slopes up and is
getting \textit{steeper}. In this case the curve is \textbf{concave
  up}.
\end{minipage}
\end{minipage}
\\[-2ex]
& & 
\\\hline 
& & \\[-1.5ex]
f''(x)<0 &
\begin{minipage}{2in}
\[
\begin{tikzpicture}
	\begin{axis}[
            clip=false,
            height=4.5cm,
            domain=0:1,
            ymax=1,
            ymin=0,
            axis lines=none,
          ]
          \addplot [very thick, penColor, smooth] {-x^2+1};
          \node at (axis cs:.4,.4) [textColor] {\footnotesize Concave Down};
        \end{axis}
\end{tikzpicture}
\]
\begin{minipage}{2in}\footnotesize
Here $f'(x)<0$ and $f''(x)<0$. This means
that $f(x)$ slopes down and is getting \textit{steeper}. In this case the curve is \textbf{concave down}.
\end{minipage}
\end{minipage}
&
\begin{minipage}{2in}
\[
  \begin{tikzpicture}
	\begin{axis}[
            clip=false,
            height=4.5cm,
            domain=0:1,
            ymax=1,
            ymin=0,
            axis lines=none,
          ]
          \addplot [very thick, penColor, smooth] {-(x-1)^2+1};
          \node at (axis cs:.6,.4) [textColor] {\footnotesize Concave Down};
        \end{axis}
\end{tikzpicture}
\]
\begin{minipage}{2in}\footnotesize
Here $f'(x)>0$ and $f''(x)<0$. This means
that $f(x)$ slopes up and is getting less \textit{steep}. In this case the curve is \textbf{concave down}.
\end{minipage}
\end{minipage}
\\[-2ex]
& & 
\\\hline 
\end{array}}
\]


If we are trying to understand the shape of the graph of a function,
knowing where it is concave up and concave down helps us to get a more
accurate picture. It is worth summarizing what we have seen already in
to a single theorem.

\begin{theorem}[Test for Concavity]\index{concavity test}
Suppose that $f''(x)$ exists on an interval.
\begin{enumerate}
\item $f''(x)>0$ on that interval whenever $y=f(x)$ is concave up on that interval.
\item $f''(x)<0$ on that interval whenever $y=f(x)$ is concave down on that interval.
\end{enumerate}
\end{theorem}

\end{document}
