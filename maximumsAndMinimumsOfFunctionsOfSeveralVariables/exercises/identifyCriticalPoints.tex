\documentclass{ximera}

\newcommand{\RR}{\mathbb R}
\renewcommand{\d}{\,d}
\newcommand{\dd}[2][]{\frac{d #1}{d #2}}
\renewcommand{\l}{\ell}
\newcommand{\ddx}{\frac{d}{dx}}
\newcommand{\dfn}{\textbf}
\newcommand{\eval}[1]{\bigg[ #1 \bigg]}


\author{Jim Talamo}
\license{Creative Commons 3.0 By-NC}


\outcome{Find the critical points of a function of two variables.}
\outcome{Use the second derivative test to identify local extrema.}
\outcome{Find local extrema of functions of two variables.}

\begin{document}
\begin{exercise}
The following exercise explores the nature of finding critical points.  You may assume that each function below is differentiable for all $(x,y) \in \R^2$.

Suppose that $F(x,y)$ is a function, and it is known that $\grad{F}(x,y) = \vector{x^2+2x,y^2-4y}$.  Then $F(x,y)$ has $\answer{4}$ distinct critical point(s).

\begin{exercise}
The $x$-values for the critical points are $x=\answer{-2}$ and $x=\answer{0}$ (type the smaller $x$-value first) and the $y$-values for the critical points are $y=\answer{0}$ and $y= \answer{4}$ (type the smaller $y$-value first).

\begin{exercise}
Select all of the following that are critical points.

\begin{selectAll}
\choice[correct]{$(-2,0)$}
\choice[correct]{$(-2,4)$}
\choice[correct]{$(0,0)$}
\choice[correct]{$(0,4)$}
\end{selectAll}
\end{exercise}
\end{exercise}

%%%%%%%%%%%%%%%%%%%%%%%%%%%%%%%%%%%%%%%%%%%%%%%%%%%%%%%%

Suppose that $G(x,y)$ is a function, and it is known that $\grad{G}(x,y) = \vector{x+2y,y^2+2x}$.  Then $G(x,y)$ has $\answer{2}$ distinct critical point(s).

\begin{exercise}
The $x$-values for the critical points are $x=\answer{-8}$ and $x=\answer{0}$ (type the smaller $x$-value first) and the $y$-values for the critical points are $y=\answer{0}$ and $y= \answer{4}$ (type the smaller $y$-value first).

\begin{exercise}
Select all of the following that are critical points.

\begin{selectAll}
\choice{$(-8,0)$}
\choice[correct]{$(-8,4)$}
\choice[correct]{$(0,0)$}
\choice{$(0,4)$}
\end{selectAll}
\end{exercise}
\end{exercise}

%%%%%%%%%%%%%%%%%%%%%%%%%%%%%%%%%%%%%%%%%%%%%%%%%%%%%%%%
 
 Suppose that $H(x,y)$ is a function, and it is known that $\grad{H}(x,y) = \vector{x+2y,2x-y-1}$.  Then $H(x,y)$ has $\answer{1}$ distinct critical point(s).

\begin{exercise}
The critical point is $\left(\answer{\frac{2}{5}}, \answer{-\frac{1}{5}} \right)$.

\end{exercise}

%%%%%%%%%%%%%%%%%%%%%%%%%%%%%%%%%%%%%%%%%%%%%%%%%%%%%%%%

After working through each of these parts, can you understand what makes the number of critical points these functions have different from each other? 

\end{exercise}
\end{document}
