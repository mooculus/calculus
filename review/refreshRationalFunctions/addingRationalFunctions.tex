\documentclass{ximera}

\newcommand{\RR}{\mathbb R}
\renewcommand{\d}{\,d}
\newcommand{\dd}[2][]{\frac{d #1}{d #2}}
\renewcommand{\l}{\ell}
\newcommand{\ddx}{\frac{d}{dx}}
\newcommand{\dfn}{\textbf}
\newcommand{\eval}[1]{\bigg[ #1 \bigg]}



\author{Jim Talamo}
\license{Creative Commons 3.0 By-NC}
\outcome{}
%\newcommand\Mydiv[2]{%
%$\strut#1$\kern.25em\smash{\raise.3ex\hbox{$\big)$}}$\mkern-8mu
%\overline{\enspace\strut#2}$}

\begin{document}

\begin{exercise}


The technique of partial fractions requires that we are able to
factor the denominator of rational expressions.  We then have to
look for simpler rational functions whose sum or difference could
give the original function.  For�� instance, it would be hard to��
evaluate
\[
\int \frac{1}{x^2+x}\d x.
\]
��However, it is true�� that
\begin{align*}
\frac{1}{x^2+x} &= \frac{1}{x(x+1)}\\
&=\frac{1}{x}-\frac{1}{x+1}.
\end{align*}
and it is easy to integrate the right-hand side of this expression.

We will study how to obtain the right-hand side of this�� expression,
but a conceptual understanding of this procedure that�� ``undoes'' the
procedure of finding a common denominator is greatly aided by
remembering how to add rational functions.

Here's an example.

\begin{example}
Express $\frac{2}{x^2} + \frac{3}{x} + \frac{1}{x+1}$ as a single rational function.

\begin{explanation}
We start by finding the least common denominator.  Note that since $x^2$ is divisible by $x$, the least common denominator is $x^2(x+1)$.  

We now have to multiply and divide each term by the factor necessary to obtain this expression in the denominator.

\[
\frac{2}{x^2} + \frac{3}{x} + \frac{1}{x+1} = \frac{2}{x^2} \cdot \frac{x+1}{x+1}+ \frac{3}{x} \cdot \frac{x(x+1)}{x(x+1)} + \frac{1}{x+1}\cdot \frac{x^2}{x^2}
\]

Now, distribute to the numerators.

\[
\frac{2}{x^2} + \frac{3}{x} + \frac{1}{x+1} = \frac{2(x+1)}{x^2(x+1)} + \frac{3x(x+1)}{x^2(x+1)}  + \frac{x^2}{x^2(x+1)}
\]

Since the fractions all have the same denominator, we can combine them.

\[
\frac{2}{x^2} + \frac{3}{x} + \frac{1}{x+1} = \frac{2(x+1)+3x(x+1)+x^2}{x^2(x+1)}
\]

Simplify the numerator to finish.
\begin{align*}
\frac{2}{x^2} + \frac{3}{x} + \frac{1}{x+1} &= \frac{2x+2+3x^2+3x+x^2}{x^2(x+1)} \\
&= \frac{4x^2+5x+2}{x^2(x+1)}
\end{align*}

\end{explanation}

We can use this result to compute $\int \frac{4x^2+5x+2}{x^2(x+1)} \d x$.  In fact,

\[
\int \frac{4x^2+5x+2}{x^2(x+1)} \d x = \answer{-\frac{2}{x}+3\ln(|x|)+\ln(|x+1|)}+C
\]
(don't forget the absolute value signs around the logarithms)

\begin{hint}
Rather than trying to integrate $\frac{4x^2+5x+2}{x^2(x+1)}$, instead find $\int \frac{2}{x^2} + \frac{3}{x} + \frac{1}{x+1} \d x$ since

\[
\frac{4x^2+5x+2}{x^2(x+1)}=\frac{2}{x^2} + \frac{3}{x} + \frac{1}{x+1}.
\]

\end{hint}

\end{example}

In most of the examples we study, we will be given the single rational function and we will have to figure out how to cook up the sum of fractions whose antiderivatives we can compute, but to gain algebraic intuition for adding fractions, express the sums of rational functions below as a single rational function.

\begin{exercise}
Write the following expression as a single rational function.
\[
\frac{7}{x-2}+\frac{5}{x} = \frac{\answer{-10+12x}}{\answer{-2x+x^2}}
\]
\end{exercise}

\begin{exercise}
Write the following expression as a single rational function.
\[
\frac{x-7}{x^2-4} + \frac{1}{x-3} - \frac{2}{x-5} = \frac{\answer{-14x^2+75x-109}}{\answer{(-5 + x) (-3 + x) (-4 + x^2)}}
\]
\end{exercise}

\begin{exercise}
Write the following expression as a single rational function.
\[
\frac{2}{x^2+7x+11} + \frac{5}{x^2-9} = \frac{\answer{37 + 35 x + 7 x^2}}{\answer{(-9 + x^2) (11 + 7 x + x^2)}}
\]
\end{exercise}

\end{exercise}
\end{document}
