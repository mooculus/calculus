\documentclass{ximera}

\newcommand{\RR}{\mathbb R}
\renewcommand{\d}{\,d}
\newcommand{\dd}[2][]{\frac{d #1}{d #2}}
\renewcommand{\l}{\ell}
\newcommand{\ddx}{\frac{d}{dx}}
\newcommand{\dfn}{\textbf}
\newcommand{\eval}[1]{\bigg[ #1 \bigg]}


\author{Jim Talamo}

\license{Creative Commons 3.0 By-SA}
\outcome{Given formulas for a parametric curve, identify the corresponding plot.}

\begin{document}

\begin{exercise}
Suppose that a bug walks along the parabola $y=x^2+3$ and does so in a manner in which $\frac{dx}{dt} = 3$.  If the bug starts walking at $x=1$ at time $t=0$, and we want $x(t)$ to measure the bug's $x$-coordinate at time $t$, then we have

\[
x(t) = \answer{3t+1}.
\]

\begin{hint}
Since $\frac{dx}{dt} = 3$, we can integrate to obtain 

\[x(t) = \answer{3t+C}\]

(use $C$ as the constant of integration).

Now, since $x(0) = 1$, $C=\answer{1}$.
\end{hint}

\begin{exercise}
Suppose that we now want to use a vector-valued function to give the position of the bug at time $t$.  Then, $y(t) = \answer{9t^2+6t+4}$ and the position vector is 

\[
\vec{p}(t) = \vector{x(t),y(t)} = \vector{\answer{3t+1},\answer{9t^2+6t+4}}, t \geq 0.
\]

\begin{hint}
Since $y=x^2+3$ and we have that $x(t) = 3t+1$, we can find $y(t)$ by substituting the expression for $x(t)$ into the equation for the parabola.  
\end{hint}
\begin{exercise}
Try to imagine the bug's motion subject to these conditions (or use the instructions in the parametric equations homework to design a Desmos sheet).  As $t>0$, we should have that $\frac{dy}{dt}$ is \wordChoice{\choice[correct]{increasing}\choice{decreasing}\choice{constant}}.

Now, find the time $t$ when $\frac{dy}{dt} = 60$.  How far is the bug away from the origin at this time?

$\frac{dy}{dt} = 60$ when $t=\answer{3}$, and the bug is $\answer{\sqrt{10709} }$ units from the origin at this time.

\begin{hint}
We can compute $\frac{dy}{dt}$ from $y(t) = 9t^2+6t+4$ to find $\frac{dy}{dt} = \answer{18t+6}$.  Setting this equal to $60$, we find that $t=\answer{3}$.

\begin{feedback}[correct]
To find the distance from the origin the bug is at this time, note that we can find the position of the bug from evaluating $\vec{p}(t)$ at $t=3$, and the distance from the origin will be exactly $|\vec{p}(3)|$.
\end{feedback}
\end{hint}

\end{exercise}
\end{exercise}
\end{exercise}

\end{document}
