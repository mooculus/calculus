\documentclass{ximera}

\newcommand{\RR}{\mathbb R}
\renewcommand{\d}{\,d}
\newcommand{\dd}[2][]{\frac{d #1}{d #2}}
\renewcommand{\l}{\ell}
\newcommand{\ddx}{\frac{d}{dx}}
\newcommand{\dfn}{\textbf}
\newcommand{\eval}[1]{\bigg[ #1 \bigg]}


\outcome{Understand what information the derivative gives concerning when a function is increasing or decreasing.}
\outcome{Understand what information the second derivative gives concerning concavity of a function.}
\outcome{Interpret limits as giving information about functions.}
\outcome{Determine how the graph of a function looks based on an analytic description of the function.}

\author{Nela Lakos \and Kyle Parsons}

\begin{document}
\begin{exercise}

Consider a function $f$ satisfying the following conditions.
\begin{itemize}
\item $f(0)=0$ and $f'(0)=0$.
\item $\lim_{x\to-5}f(x) = -\infty$.
\item $f'(x)<0$ on $(-\infty,-5)$ and $(0,\infty)$.
\item $f'(x)>0$ on $(-5,0)$.
\item $f''(x)<0$ on $(-15,-5)$ and $(-5,15)$.
\item $f''(x)>0$ on $(-\infty,-15)$ and $(15,\infty)$.
\end{itemize}

$f$ is increasing and concave down on the interval $\left(\answer{-5},\answer{0}\right)$

$f$ is increasing and concave up on NO interval.

$f$ is decreasing and concave down on the intervals (from left to right) $\left(\answer{-15},\answer{-5}\right)$ and $\left(\answer{0},\answer{15}\right)$.

$f$ is decreasing and concave up on the intervals (from left to right) $\left(\answer{-\infty},\answer{-15}\right)$ and $\left(\answer{15},\answer{\infty}\right)$

$f$ has $\answer{0}$ local minima.

$f$ has a local maximum at $x=\answer{0}$.

$f$ has inflection points (from left to right) at $x=\answer{-15}$ and $x=\answer{15}$.

\resizebox{0.45\textwidth}{!}{
  \begin{tikzpicture}
    \begin{axis}[
        xmin=-20.3,xmax=20.3,ymin=-20.3,ymax=20.3,
        clip=true,
        unit vector ratio*=1 1 1,
        axis lines=center,
        grid = major,
        ytick={-20,-15,...,20},
    	xtick={-20,-15,...,20},
        xlabel=$x$, ylabel=$y$,
        y tick label style={anchor=west},
        every axis y label/.style={at=(current axis.above origin),anchor=south},
        every axis x label/.style={at=(current axis.right of origin),anchor=west},
      ]
      \addplot[very thick,penColor,domain=-20.3:-15] plot{0.1*(x+15)^2-0.3*(x+15)+3*ln(10)};
      \addplot[very thick,penColor,domain=-15:-5] plot{3*ln(-5-x)};
      \addplot[very thick,penColor,domain=-5:0] plot{x/(x+5)};
      \addplot[very thick,penColor,domain=0:15] plot{-x^2/20};
      \addplot[very thick,penColor,domain=15:20.3] plot{(x-15)^2/10-1.5*(x-15)-45/4};

      \node at (axis cs:12,12) {\huge$A$};
      \end{axis}`
  \end{tikzpicture}}
\hfill
\resizebox{0.45\textwidth}{!}{
  \begin{tikzpicture}
    \begin{axis}[
        xmin=-20.3,xmax=20.3,ymin=-20.3,ymax=20.3,
        clip=true,
        unit vector ratio*=1 1 1,
        axis lines=center,
        grid = major,
        ytick={-20,-15,...,20},
    	xtick={-20,-15,...,20},
        xlabel=$x$, ylabel=$y$,
        y tick label style={anchor=west},
        every axis y label/.style={at=(current axis.above origin),anchor=south},
        every axis x label/.style={at=(current axis.right of origin),anchor=west},
      ]
      %\addplot[very thick,penColor,domain=-20.3:-15] plot{0.1*(x+15)^2-0.3*(x+15)+3*ln(10)};
      \addplot[very thick,penColor,domain=-20.3:-5,samples=20] plot{3*ln(-5-x)};
      \addplot[very thick,penColor,domain=-5:0] plot{x/(x+5)};
      \addplot[very thick,penColor,domain=0:15] plot{-x^2/20};
      \addplot[very thick,penColor,domain=15:20.3] plot{(x-15)^2/10-1.5*(x-15)-45/4};

      \node at (axis cs:12,12) {\huge$B$};
      \end{axis}`
  \end{tikzpicture}}

\resizebox{0.45\textwidth}{!}{
  \begin{tikzpicture}
    \begin{axis}[
        xmin=-20.3,xmax=20.3,ymin=-20.3,ymax=20.3,
        clip=true,
        unit vector ratio*=1 1 1,
        axis lines=center,
        grid = major,
        ytick={-20,-15,...,20},
    	xtick={-20,-15,...,20},
        xlabel=$x$, ylabel=$y$,
        y tick label style={anchor=west},
        every axis y label/.style={at=(current axis.above origin),anchor=south},
        every axis x label/.style={at=(current axis.right of origin),anchor=west},
      ]
      \addplot[very thick,penColor,domain=-20.3:-15] plot{0.1*(x+15)^2-0.3*(x+15)+3*ln(10)};
      \addplot[very thick,penColor,domain=-15:-5] plot{3*ln(-5-x)};
      \addplot[very thick,penColor,domain=-5:0] plot{-x/(x+5)};
      \addplot[very thick,penColor,domain=0:15] plot{-x^2/20};
      \addplot[very thick,penColor,domain=15:20.3] plot{(x-15)^2/10-1.5*(x-15)-45/4};

      \node at (axis cs:12,12) {\huge$C$};
      \end{axis}`
  \end{tikzpicture}}
\hfill
\resizebox{0.45\textwidth}{!}{
  \begin{tikzpicture}
    \begin{axis}[
        xmin=-20.3,xmax=20.3,ymin=-20.3,ymax=20.3,
        clip=true,
        unit vector ratio*=1 1 1,
        axis lines=center,
        grid = major,
        ytick={-20,-15,...,20},
    	xtick={-20,-15,...,20},
        xlabel=$x$, ylabel=$y$,
        y tick label style={anchor=west},
        every axis y label/.style={at=(current axis.above origin),anchor=south},
        every axis x label/.style={at=(current axis.right of origin),anchor=west},
      ]
      \addplot[very thick,penColor,domain=-20.3:-15] plot{0.1*(x+15)^2-0.3*(x+15)+3*ln(10)};
      \addplot[very thick,penColor,domain=-15:-5] plot{3*ln(-5-x)};
      \addplot[very thick,penColor,domain=-5:0] plot{x/(x+5)};
      \addplot[very thick,penColor,domain=0:20.3] plot{x^3/400};
      %\addplot[very thick,penColor,domain=0:15] plot{-x^2/20};
      %\addplot[very thick,penColor,domain=15:20.3] plot{(x-15)^2/10-1.5*(x-15)-45/4};

      \node at (axis cs:12,12) {\huge$D$};
      \end{axis}`
  \end{tikzpicture}}
  
Which diagram shows the graph of a function that has all of the above properties?
\begin{multipleChoice}
\choice[correct]{$A$}
\choice{$B$}
\choice{$C$}
\choice{$D$}
\end{multipleChoice}

\end{exercise}
\end{document}