\documentclass{ximera}

\newcommand{\RR}{\mathbb R}
\renewcommand{\d}{\,d}
\newcommand{\dd}[2][]{\frac{d #1}{d #2}}
\renewcommand{\l}{\ell}
\newcommand{\ddx}{\frac{d}{dx}}
\newcommand{\dfn}{\textbf}
\newcommand{\eval}[1]{\bigg[ #1 \bigg]}


\outcome{Understand the connection between curves and parameterizations.}

\author{Jim Talamo}

\begin{document}
\begin{exercise}
 The curve $\mathcal{C}$ traced out by  $\vec{p}(t)$ is shown below.  The points associated to $\vec{p}(t)$ for various $t$ values are shown on the curve.

\begin{image}
    \begin{tikzpicture}
      \begin{axis}[
          xmin=-1.4, xmax=1.4, ymin =-1.4, ymax = 1.1,
          axis lines=center,  
          xlabel=$x$,  
          ylabel=$y$,  
          every axis y label/.style={at=(current axis.above origin),anchor=south},  
          every axis x label/.style={at=(current axis.right of origin),anchor=west}
        ]
        \addplot [penColor,ultra thick,domain=0:360,smooth,samples=100] ({-((x/180)^(.5))*cos(x)+((x/180)^(.5))*sin(x)},{((x/180)^(.5))*sin(x)});
        \node[above right,penColor] at (axis cs: 0,-.3) {$\vec{p}(0)$};
%        \node[above right,penColor] at (axis cs: -1.3,0) {$\vec{p}(4)$};
        \node[above right,penColor] at (axis cs: {-((110/180)^(.5))*cos(110)+((110/180)^(.5))*sin(110)},{((110/180)^(.5))*sin(110)}) {$\vec{p}(2)$};
        \node[above left,penColor] at (axis cs: {-((45/180)^(.5))*cos(45)-.1+((45/180)^(.5))*sin(45)-.1},{((45/180)^(.5))*sin(45)-.1}) {$\vec{p}(1)$};
        \node[above right,penColor] at (axis cs: {-((215/180)^(.5))*cos(215)+.1+((215/180)^(.5))*sin(215)},{((215/180)^(.5))*sin(215)-.2}) {$\vec{p}(3)$};
        
        \addplot[color=penColor,fill=penColor,only marks,mark=*] coordinates{(-{((45/180)^(.5))*cos(45)+((45/180)^(.5))*sin(45)},{((45/180)^(.5))*sin(45)})};
        \addplot[color=penColor,fill=penColor,only marks,mark=*] coordinates{(-{((110/180)^(.5))*cos(110)+((110/180)^(.5))*sin(110)},{((110/180)^(.5))*sin(110)})};
        \addplot[color=penColor,fill=penColor,only marks,mark=*] coordinates{(-{((207/180)^(.5))*cos(207)+((207/180)^(.5))*sin(207)},{((207/180)^(.5))*sin(207)})};
%        \addplot[color=penColor,fill=penColor,only marks,mark=*] coordinates{(-{((270/180)^(.5))*cos(270)+((270/180)^(.5))*sin(270)},{((270/180)^(.5))*sin(270)})};
        \addplot[color=penColor,fill=penColor,only marks,mark=*] coordinates{(0,0)};
        
      \end{axis}
    \end{tikzpicture}
\end{image}

Provide the most accurate response to the following questions.
 
 Which of the following vectors is parallel to $\vec{p}(1)?$
 \begin{multipleChoice}
 \choice{$\vector{1,0}$}
 \choice[correct]{$\vector{0,1}$}
 \choice{$\vector{1,1}$}
 \choice{$\vector{1,-1}$}
 \choice{$\vector{-1,1}$}
 \choice{more than one of these.}
 \end{multipleChoice}
 
 Which of the following vectors is parallel to $\vec{p}'(1)?$
 \begin{multipleChoice}
 \choice{$\vector{1,0}$}
 \choice{$\vector{0,1}$}
 \choice[correct]{$\vector{1,1}$}
 \choice{$\vector{1,-1}$}
 \choice{$\vector{-1,1}$}
 \choice{more than one of these.}
 \end{multipleChoice}
 
 Which of the following vectors is orthogonal to $\vec{p}'(2)?$
 \begin{multipleChoice}
 \choice{$\vector{1,0}$}
 \choice[correct]{$\vector{0,1}$}
 \choice{$\vector{1,1}$}
 \choice{$\vector{1,-1}$}
 \choice{$\vector{-1,1}$}
 \choice{more than one of these.}
 \end{multipleChoice}
 
Which of the following vectors is parallel to $\vector{y(3),-x(3)}?$
 \begin{multipleChoice}
 \choice{$\vector{1,0}$}
 \choice{$\vector{0,1}$}
 \choice[correct]{$\vector{1,1}$}
 \choice{$\vector{1,-1}$}
 \choice{$\vector{-1,1}$}
 \choice{more than one of these.}
 \end{multipleChoice}
 
 \begin{feedback}[correct]
 Make sure you understand the logic behind each response.
\begin{itemize}
\item Note that $\vec{p}(1)$ will extend from the origin to the point associated to $\vec{p}(1)$, so any vector with an $x$-component of $0$ and a nonzero $y$-component will be parallel to $\vec{p}(1)$.
\item Note that $\vec{p}'(1)$ will be a tangent vector to the curve at the point associated to $\vec{p}(1)$.  From the image, the $x$ and $y$ components of such a vector should be approximately equal.  
\item Note that $\vec{p}'(2)$ will be a tangent vector to the curve at the point associated to $\vec{p}(2)$.  From the image, the $x$-component of such a vector can be nonzero, but the $y$-component should be $0$.  Thus, a vector \emph{orthogonal} to this will have nonzero $y$-component, but its $x$-component should be $0$.
\item Note that $\vec{p}(3)$ will extend from the origin to the point associated to $\vec{p}(3)$.  From the image, it looks like $x(3) \approx -y(3)$ so any vector parallel to $\vector{y(3),-x(3)}$ will have approximately the same $x$ and $y$ components.
\end{itemize}
 \end{feedback}
 \end{exercise}
\end{document}
