\documentclass{ximera}

\newcommand{\RR}{\mathbb R}
\renewcommand{\d}{\,d}
\newcommand{\dd}[2][]{\frac{d #1}{d #2}}
\renewcommand{\l}{\ell}
\newcommand{\ddx}{\frac{d}{dx}}
\newcommand{\dfn}{\textbf}
\newcommand{\eval}[1]{\bigg[ #1 \bigg]}


\title{The definite integral}

\begin{document}

\begin{abstract}
%Stuff can go here later if we want!
\end{abstract}

\maketitle

\begin{sectionOutcomes}

After completing this section, students should be able to do the following.

\begin{itemize}
\item Use integral notation for both antiderivatives and definite integrals.
%\item Compute definite integrals using limits of Riemann Sums.
\item Compute definite integrals using geometry.
\item Compute definite integrals using the properties of integrals.
\item Justify the properties of definite integrals using algebra or geometry.
\item Understand how Riemann sums are used to find exact area.
%\item Use limits of Riemann sums to find the exact area under a curve.
\item Define net area.
\item Approximate net area.
%\item Understand how Riemann sums are used to find exact area.
\item Split the area under a curve into several pieces to aid with calculations.
\item Use symmetry to calculate definite integrals.
\item Explain geometrically why symmetry of a function simplifies calculation of some definite integrals.
  %\item Generalize the process of taking a Riemann sum so that the intervals need not be equally spaced.
\end{itemize}

\end{sectionOutcomes}

\end{document}
