\documentclass{ximera}

%\usepackage{todonotes}

\newcommand{\todo}{}

\usepackage{esint} % for \oiint
\ifxake%%https://math.meta.stackexchange.com/questions/9973/how-do-you-render-a-closed-surface-double-integral
\renewcommand{\oiint}{{\large\bigcirc}\kern-1.56em\iint}
\fi

\def\xmNotExpandableAsAccordion{true}

\graphicspath{
  {./}
  {ximeraTutorial/}
  {basicPhilosophy/}
  {functionsOfSeveralVariables/}
  {normalVectors/}
  {lagrangeMultipliers/}
  {vectorFields/}
  {greensTheorem/}
  {shapeOfThingsToCome/}
  {dotProducts/}
  {../productAndQuotientRules/exercises/}
  {../normalVectors/exercisesParametricPlots/}
  {../continuityOfFunctionsOfSeveralVariables/exercises/}
  {../partialDerivatives/exercises/}
  {../chainRuleForFunctionsOfSeveralVariables/exercises/}
  {../commonCoordinates/exercisesCylindricalCoordinates/}
  {../commonCoordinates/exercisesSphericalCoordinates/}
  {../greensTheorem/exercisesCurlAndLineIntegrals/}
  {../greensTheorem/exercisesDivergenceAndLineIntegrals/}
  {../shapeOfThingsToCome/exercisesDivergenceTheorem/}
  {../greensTheorem/}
  {../shapeOfThingsToCome/}
}

\newcommand{\mooculus}{\textsf{\textbf{MOOC}\textnormal{\textsf{ULUS}}}}

\usepackage{tkz-euclide}\usepackage{tikz}
\usepackage{tikz-cd}
\usetikzlibrary{arrows}
\tikzset{>=stealth,commutative diagrams/.cd,
  arrow style=tikz,diagrams={>=stealth}} %% cool arrow head
\tikzset{shorten <>/.style={ shorten >=#1, shorten <=#1 } } %% allows shorter vectors

\usetikzlibrary{backgrounds} %% for boxes around graphs
\usetikzlibrary{shapes,positioning}  %% Clouds and stars
\usetikzlibrary{matrix} %% for matrix
\usepgfplotslibrary{polar} %% for polar plots
\usepgfplotslibrary{fillbetween} %% to shade area between curves in TikZ
%\usetkzobj{all}
%\usepackage[makeroom]{cancel} %% for strike outs
%\usepackage{mathtools} %% for pretty underbrace % Breaks Ximera
%\usepackage{multicol}
\usepackage{pgffor} %% required for integral for loops



%% http://tex.stackexchange.com/questions/66490/drawing-a-tikz-arc-specifying-the-center
%% Draws beach ball
\tikzset{pics/carc/.style args={#1:#2:#3}{code={\draw[pic actions] (#1:#3) arc(#1:#2:#3);}}}



\usepackage{array}
\setlength{\extrarowheight}{+.1cm}   
\newdimen\digitwidth
\settowidth\digitwidth{9}
\def\divrule#1#2{
\noalign{\moveright#1\digitwidth
\vbox{\hrule width#2\digitwidth}}}





\newcommand{\RR}{\mathbb R}
\newcommand{\R}{\mathbb R}
\newcommand{\N}{\mathbb N}
\newcommand{\Z}{\mathbb Z}

\newcommand{\sagemath}{\textsf{SageMath}}


%\renewcommand{\d}{\,d\!}
\renewcommand{\d}{\mathop{}\!d}
\newcommand{\dd}[2][]{\frac{\d #1}{\d #2}}
\newcommand{\pp}[2][]{\frac{\partial #1}{\partial #2}}
\renewcommand{\l}{\ell}
\newcommand{\ddx}{\frac{d}{\d x}}

\newcommand{\zeroOverZero}{\ensuremath{\boldsymbol{\tfrac{0}{0}}}}
\newcommand{\inftyOverInfty}{\ensuremath{\boldsymbol{\tfrac{\infty}{\infty}}}}
\newcommand{\zeroOverInfty}{\ensuremath{\boldsymbol{\tfrac{0}{\infty}}}}
\newcommand{\zeroTimesInfty}{\ensuremath{\small\boldsymbol{0\cdot \infty}}}
\newcommand{\inftyMinusInfty}{\ensuremath{\small\boldsymbol{\infty - \infty}}}
\newcommand{\oneToInfty}{\ensuremath{\boldsymbol{1^\infty}}}
\newcommand{\zeroToZero}{\ensuremath{\boldsymbol{0^0}}}
\newcommand{\inftyToZero}{\ensuremath{\boldsymbol{\infty^0}}}



\newcommand{\numOverZero}{\ensuremath{\boldsymbol{\tfrac{\#}{0}}}}
\newcommand{\dfn}{\textbf}
%\newcommand{\unit}{\,\mathrm}
\newcommand{\unit}{\mathop{}\!\mathrm}
\newcommand{\eval}[1]{\bigg[ #1 \bigg]}
\newcommand{\seq}[1]{\left( #1 \right)}
\renewcommand{\epsilon}{\varepsilon}
\renewcommand{\phi}{\varphi}


\renewcommand{\iff}{\Leftrightarrow}

\DeclareMathOperator{\arccot}{arccot}
\DeclareMathOperator{\arcsec}{arcsec}
\DeclareMathOperator{\arccsc}{arccsc}
\DeclareMathOperator{\si}{Si}
\DeclareMathOperator{\scal}{scal}
\DeclareMathOperator{\sign}{sign}


%% \newcommand{\tightoverset}[2]{% for arrow vec
%%   \mathop{#2}\limits^{\vbox to -.5ex{\kern-0.75ex\hbox{$#1$}\vss}}}
\newcommand{\arrowvec}[1]{{\overset{\rightharpoonup}{#1}}}
%\renewcommand{\vec}[1]{\arrowvec{\mathbf{#1}}}
\renewcommand{\vec}[1]{{\overset{\boldsymbol{\rightharpoonup}}{\mathbf{#1}}}}
\DeclareMathOperator{\proj}{\vec{proj}}
\newcommand{\veci}{{\boldsymbol{\hat{\imath}}}}
\newcommand{\vecj}{{\boldsymbol{\hat{\jmath}}}}
\newcommand{\veck}{{\boldsymbol{\hat{k}}}}
\newcommand{\vecl}{\vec{\boldsymbol{\l}}}
\newcommand{\uvec}[1]{\mathbf{\hat{#1}}}
\newcommand{\utan}{\mathbf{\hat{t}}}
\newcommand{\unormal}{\mathbf{\hat{n}}}
\newcommand{\ubinormal}{\mathbf{\hat{b}}}

\newcommand{\dotp}{\bullet}
\newcommand{\cross}{\boldsymbol\times}
\newcommand{\grad}{\boldsymbol\nabla}
\newcommand{\divergence}{\grad\dotp}
\newcommand{\curl}{\grad\cross}
%\DeclareMathOperator{\divergence}{divergence}
%\DeclareMathOperator{\curl}[1]{\grad\cross #1}
\newcommand{\lto}{\mathop{\longrightarrow\,}\limits}

\renewcommand{\bar}{\overline}

\colorlet{textColor}{black} 
\colorlet{background}{white}
\colorlet{penColor}{blue!50!black} % Color of a curve in a plot
\colorlet{penColor2}{red!50!black}% Color of a curve in a plot
\colorlet{penColor3}{red!50!blue} % Color of a curve in a plot
\colorlet{penColor4}{green!50!black} % Color of a curve in a plot
\colorlet{penColor5}{orange!80!black} % Color of a curve in a plot
\colorlet{penColor6}{yellow!70!black} % Color of a curve in a plot
\colorlet{fill1}{penColor!20} % Color of fill in a plot
\colorlet{fill2}{penColor2!20} % Color of fill in a plot
\colorlet{fillp}{fill1} % Color of positive area
\colorlet{filln}{penColor2!20} % Color of negative area
\colorlet{fill3}{penColor3!20} % Fill
\colorlet{fill4}{penColor4!20} % Fill
\colorlet{fill5}{penColor5!20} % Fill
\colorlet{gridColor}{gray!50} % Color of grid in a plot

\newcommand{\surfaceColor}{violet}
\newcommand{\surfaceColorTwo}{redyellow}
\newcommand{\sliceColor}{greenyellow}




\pgfmathdeclarefunction{gauss}{2}{% gives gaussian
  \pgfmathparse{1/(#2*sqrt(2*pi))*exp(-((x-#1)^2)/(2*#2^2))}%
}


%%%%%%%%%%%%%
%% Vectors
%%%%%%%%%%%%%

%% Simple horiz vectors
\renewcommand{\vector}[1]{\left\langle #1\right\rangle}


%% %% Complex Horiz Vectors with angle brackets
%% \makeatletter
%% \renewcommand{\vector}[2][ , ]{\left\langle%
%%   \def\nextitem{\def\nextitem{#1}}%
%%   \@for \el:=#2\do{\nextitem\el}\right\rangle%
%% }
%% \makeatother

%% %% Vertical Vectors
%% \def\vector#1{\begin{bmatrix}\vecListA#1,,\end{bmatrix}}
%% \def\vecListA#1,{\if,#1,\else #1\cr \expandafter \vecListA \fi}

%%%%%%%%%%%%%
%% End of vectors
%%%%%%%%%%%%%

%\newcommand{\fullwidth}{}
%\newcommand{\normalwidth}{}



%% makes a snazzy t-chart for evaluating functions
%\newenvironment{tchart}{\rowcolors{2}{}{background!90!textColor}\array}{\endarray}

%%This is to help with formatting on future title pages.
\newenvironment{sectionOutcomes}{}{} 



%% Flowchart stuff
%\tikzstyle{startstop} = [rectangle, rounded corners, minimum width=3cm, minimum height=1cm,text centered, draw=black]
%\tikzstyle{question} = [rectangle, minimum width=3cm, minimum height=1cm, text centered, draw=black]
%\tikzstyle{decision} = [trapezium, trapezium left angle=70, trapezium right angle=110, minimum width=3cm, minimum height=1cm, text centered, draw=black]
%\tikzstyle{question} = [rectangle, rounded corners, minimum width=3cm, minimum height=1cm,text centered, draw=black]
%\tikzstyle{process} = [rectangle, minimum width=3cm, minimum height=1cm, text centered, draw=black]
%\tikzstyle{decision} = [trapezium, trapezium left angle=70, trapezium right angle=110, minimum width=3cm, minimum height=1cm, text centered, draw=black]


\author{Jim Talamo}
\license{Creative Commons 3.0 By-bC}


\outcome{Understand why Taylor Polynomials are centered at $x-c$}


\begin{document}
\begin{exercise}
Recall that if a function $f(x)$ has at least $n$ derivatives at a point $x=c$, then the $n$-th order Taylor Polynomial is found by requiring that the function and its first $n$ derivatives are respectively equal to the value of the Taylor Polynomial and its first $n$ derivatives at $x=c$:

\begin{align*}
f(c) &= p_n(c) \\
f'(c) &= p_n'(c) \\
f''(c) &= p_n''(c) \\ 
& \vdots \\
f^{(n)}(c) &= p_n^{(n)}(c)
\end{align*}
or, more succinctly:

\[ f^{(k)}(c) = p_n^{(k)}(c) \textrm{ for all } k=0,1,2, \ldots , n \]

where $f^{(k)}(x)$ is notation for $\frac{d^k}{dx^k}\left[f(x)\right]$.

This lead to the requirement that:
\begin{multipleChoice}
\choice{$a_k = f^{(k)}(c) $}
\choice[correct]{$a_k = \frac{f^{(k)}(c)}{k!}$}
\choice{$f^{(k)}(c) = \frac{a_k}{k!}$}
\end{multipleChoice}

We thus take as the definition for the $n$-th order Taylor Polynomial, $p_n(x)$ centered at $x=c$ for $f(x)$:

\[
p_n(x) = a_0 +a_1(x-c)+a_2(x-c)^2+\ldots+a_n(x-c)^n
\]
where $a_k = \frac{f^{(k)}(c)}{k!}$ for $k=0,1,2, \ldots , n$.

One question that may arise is why we choose to write the polynomial in powers of $x-c$ rather than just $x$.  To get a feel for why the choice of using powers of $x-c$ is preferred, work through the exercise below.

Consider the function $f(x) = \sqrt{2x-1}$. We will find the second degree Taylor Polynomial centered at $x=1$.

\begin{exercise}
\begin{exercise}
Look for a polynomial of the form
\[
p_2(x) = a_0 +a_1(x-1)+a_2(x-1)^2
\]
The coefficients will be found by using the requirements that:

\begin{align*}
f(1) &= p_2(1) \\
f'(1) &= p_2'(1) \\
f''(1) &= p_2''(1) 
\end{align*} 

Complete the tables below:

\begin{tabular}{|c||c|c|}
\hline
$k$ \quad & \quad \quad $f^{(k)}(x)$  \quad \quad & \quad \quad $f^{(k)}(1)$ \quad \quad \\
\hline 
$0$ \quad & \quad \quad $(2x-1)^{1/2}$  \quad \quad & \quad \quad $\answer{1}$ \quad \quad  \\
\hline
$1$ \quad & \quad \quad $\answer{(2x-1)^{-1/2}}$ \quad \quad & \quad \quad $\answer{1}$ \quad \quad \\
\hline
$2$ \quad & \quad \quad $\answer{-(2x-1)^{-3/2}}$ \quad \quad & \quad \quad $\answer{-1}$ \quad \quad \\
\hline 
\end{tabular}

%%%%%TaylorPoly%%%%%%
\begin{tabular}{|c||c|c|}
\hline
$k$ \quad & \quad \quad $p_2^{(k)}(x)$  \quad \quad & \quad \quad $p_2^{(k)}(1)$ \quad \quad \\
\hline 
$0$ \quad & \quad \quad $a_0+a_1(x-1)+a_2(x-1)^2$  \quad \quad & \quad \quad $a_0$ \quad \quad  \\
\hline
$1$ \quad & \quad \quad $a_1+\answer{2}a_2(x-1)$ \quad \quad & \quad \quad $a_1+\answer{0}$ \quad \quad \\
\hline
$2$ \quad & \quad \quad $\answer{2}a_2$ \quad \quad & \quad \quad $2a_2$ \quad \quad \\
\hline
\end{tabular}

\begin{exercise}
Now, we can use the requirements to find the unknown coefficients:

\begin{itemize}
\item The requirement $f(1) = p_2(1)$ gives $a_0 = \answer{1}$. 
\item The requirement $f'(1) = p_2'(1)$ gives $a_1 = \answer{1}$. 
\item The requirement $f''(1) = p_2''(1)$ gives $a_2 = \answer{-\frac{1}{2}}$. 
\end{itemize}

Note that each condition easily allows us to isolate a singe unknown constant!  Hence, the second degree Taylor polynomial for $f(x) =\sqrt{2x-1}$ is:

\begin{align*}
p_2(x) &= \answer{1}+\answer{1}(x-1)+\answer{-\frac{1}{2}}(x-1)^2\\
\end{align*}
\end{exercise}
\end{exercise}

%%%%%%inefficeint way%%%%%%
\begin{exercise}
Instead, we can look for a polynomial of the form:
\[
P_2(x) = b_0 +b_1x+b_2x^2
\]
Note that we now have a polynomial in powers of $x$ instead of $x-1$.  The coefficients will still be found by using the requirements that:

\begin{align*}
f(1) &= P_2(1) \\
f'(1) &= P_2'(1) \\
f''(1) &= P_2''(1) 
\end{align*} 

Complete the tables below:

\begin{tabular}{|c||c|c|}
\hline
$k$ \quad & \quad \quad $f^{(k)}(x)$  \quad \quad & \quad \quad $f^{(k)}(1)$ \quad \quad \\
\hline 
$0$ \quad & \quad \quad $(2x-1)^{1/2}$  \quad \quad & \quad \quad $\answer{1}$ \quad \quad  \\
\hline
$1$ \quad & \quad \quad $\answer{(2x-1)^{-1/2}}$ \quad \quad & \quad \quad $\answer{1}$ \quad \quad \\
\hline
$2$ \quad & \quad \quad $\answer{-(2x-1)^{-3/2}}$ \quad \quad & \quad \quad $\answer{-1}$ \quad \quad \\
\hline 
\end{tabular}

%%%%%TaylorPoly%%%%%%
\begin{tabular}{|c||c|c|}
\hline
$k$ \quad & \quad \quad $P_2^{(k)}(x)$  \quad \quad & \quad \quad $P_2^{(k)}(1)$ \quad \quad \\
\hline 
$0$ \quad & \quad \quad $b_0+b_1x+b_2x^2$  \quad \quad & \quad \quad $\answer{1}b_0+\answer{1}b_1+\answer{1}b_2$ \quad \quad  \\
\hline
$1$ \quad & \quad \quad $b_1+\answer{2}b_2x$ \quad \quad & \quad \quad  $\answer{0}b_0+\answer{1}b_1+\answer{2}b_2$ \quad \quad \\
\hline
$2$ \quad & \quad \quad $\answer{2}b_2$ \quad \quad & \quad \quad  $\answer{0}b_0+\answer{0}b_1+\answer{2}b_2$ \quad \quad \\
\hline
\end{tabular}

\begin{exercise}
Now, we can use the requirements to find the unknown coefficients:

\begin{itemize}
\item The requirement $f(1) = P_2(1)$ gives $\answer{1}=b_0+b_1+b_2$. 
\item The requirement $f'(1) = P_2'(1)$ gives $\answer{1} = b_1 +\answer{2}b_2$. 
\item The requirement $f''(1) = P_2''(1)$ gives $\answer{-1} = \answer{2}b_2$. 
\end{itemize}

We now have a system of equations to solve!  It is not difficult to do this since the last condition gives: $b_2 = \answer{-\frac{1}{2}}$.

Now, substitute $b_2$ into the second to last condition to find $b_1 = \answer{2}$. 

Finally, substitute both $b_1$ and $b_2$ into the first condition to find $b_0 = \answer{-\frac{1}{2}}$.

Hence, this second degree Taylor polynomial for $f(x) =\sqrt{2x-1}$ is:

\begin{align*}
P_2(x) &= \answer{-\frac{1}{2}}+\answer{2}x+\answer{-\frac{1}{2}}x^2\\
\end{align*}


Note that the coefficients for both polynomials are different.  However, did we actually produce two different polynomials?  We can explore this by expand the first Taylor polynomial $p_3(x)$:

\begin{align*}
p_2(x) &=1+(x-1)-\frac{1}{2}(x-1)^2\\
&=1+(x-1)-\frac{1}{2} \left(\answer{x^2-2x+1}\right)\\
&= \answer{-\frac{1}{2}}+\answer{2}x+\answer{-\frac{1}{2}}x^2
\end{align*}

Are the polynomials $p_2(x)$ and $P_2(x)$ the same?

\begin{multipleChoice}
\choice[correct]{Yes}
\choice{No}
\end{multipleChoice}

\begin{feedback}[correct]
\begin{remark}
In general, these polynomials will always be the same.  We thus make the choice to expand in powers of $x-c$ rather than $x$ because expanding in powers of $x-c$ allows for a computationally efficient way to find the coefficients.  

Furthermore, we will shortly be looking for an ``infinite degree'' Taylor polynomial, and we will not be able to solve the infinite system of equations that arises if we require the polynomial to be written in powers of $x$ rather than $x-c$.
\end{remark}

\end{feedback}
\end{exercise}

\end{exercise}

\end{exercise}
\end{exercise}
\end{document}
