\documentclass{ximera}
\newcommand{\RR}{\mathbb R}
\renewcommand{\d}{\,d}
\newcommand{\dd}[2][]{\frac{d #1}{d #2}}
\renewcommand{\l}{\ell}
\newcommand{\ddx}{\frac{d}{dx}}
\newcommand{\dfn}{\textbf}
\newcommand{\eval}[1]{\bigg[ #1 \bigg]}

\author{Steven Gubkin\and nela Lakos}
\license{Creative Commons 3.0 By-NC}
\begin{document}
\begin{exercise}

Let $f$ be a function.  Consider the function $A(t) = \frac{1}{t} \int_0^t f(x)\d x $, which gives the average of $f$ on the interval $[0,t]$, for $t>0$.  

Suppose that $A$ has a local maximum at $t=b$, then $A'(b) = \answer{0}$ (answer with a number).  Now, the formula for $A'(t)$ is
\begin{align*}
A'(t) &= \answer{\frac{f(t)}{t}} - \answer{\frac{1}{t^2}}\int_0^tf(x)d x\\
&= \frac{1}{t}\left(\answer{f(t)}-A(t)\right)
\end{align*}
Now since $A'(b) = \answer{0}$ (answer with a number), we can conclude that $A(b) = \answer{f(b)}$ (answer with an expression in terms of $f$).



Suppose $f(x)  = x(1-x)$.  For what value of $t$ is $A(t)$ maximized?
\begin{hint}
You have to find a critical point of $A$.
This means that we have to solve the equation

\[
A'(x)=0.
\]
That is equivalent to solving 
\[
f(x)=A(x).
\]
\end{hint}
\begin{hint}
let's solve the equation
\[
f(x)=A(x).
\]
\[
 x(1-x)=\frac{1}{x} \int_0^x f(t)\d t
\]
\[
 x-x^2=\frac{1}{x} \int_0^x (t-t^2)\d t
\]
\[
 x-x^2= \frac{x}{2}-\frac{x^2}{3}
\]
Divide by $x$.
\[
 1-x= \frac{1}{2}-\frac{x}{3}
\]
Solve for $x$.
How do we know that the function $A$ has a maximum at this point?
\end{hint}
\[
t = \answer{\frac{3}{4}}
\]

\end{exercise}
\end{document}