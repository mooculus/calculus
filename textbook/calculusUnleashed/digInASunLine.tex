\documentclass{ximera}

\newcommand{\RR}{\mathbb R}
\renewcommand{\d}{\,d}
\newcommand{\dd}[2][]{\frac{d #1}{d #2}}
\renewcommand{\l}{\ell}
\newcommand{\ddx}{\frac{d}{dx}}
\newcommand{\dfn}{\textbf}
\newcommand{\eval}[1]{\bigg[ #1 \bigg]}


%%
%% From https://en.wikipedia.org/wiki/Wallis_product
%%
\title[Dig-In:]{A Sun-line}

\begin{document}
\begin{abstract}
  Here we ``play'' with mathematics.
\end{abstract}
\maketitle

Imagine that the world was flat and that everyday, the Sun (infinitely
far away) acutally rises up at $6\unit{am}$ in the East, and sets at
$6\unit{pm}$ in the west.

[[PICTURE]]

In the diagram above, we see a rod of height $h$ casting a shadow. We
can see that the slope of the line connecting the point representing
the Sun to the tip of the rod is
\[
\frac{R\sin\left(\frac{n\pi}{12}\right)-h}{R\cos\left(\frac{n\pi}{12}\right)}
\]
If we take the limit as $R$ goes to infinity, then we find,
\begin{align*}
  \lim_{R\to \infty} \frac{R\sin\left(\frac{n\pi}{12}\right)-h}{R\cos\left(\frac{n\pi}{12}\right)} &=\lim_{R\to \infty}\frac{R\sin\left(\frac{n\pi}{12}\right)-h}{R\cos\left(\frac{n\pi}{12}\right)}\cdot\frac{1/R}{1/R}\\
  &= \lim_{R\to \infty}\frac{\sin\left(\frac{n\pi}{12}\right)-h/R}{\cos\left(\frac{n\pi}{12}\right)}\\
  &= \frac{\sin\left(\frac{n\pi}{12}\right)}{\cos\left(\frac{n\pi}{12}\right)}\\
  &= \tan\left(\frac{n\pi}{12}\right)
\end{align*}

\section{What does the graph look like?}

We'll use all of our curve sketching techniques to try to understand
this function on the interval $[-2\pi,2\pi]$, as gesture of
friendship, we'll tell you that $\si(x)$ is continuous on
$[-2\pi,2\pi]$.  The first thing we should do is compute the first
derivative of $\si(x)$.

\begin{example}
  Compute:
  \[
  \ddx \si(x)
  \]
  \begin{explanation}
    By the First Fundamental Theorem of Calculus,
    \begin{align*}
    \ddx \si(x) &= \ddx \int_0^x \frac{\sin(t)}{t}\d t\\
    &=\frac{\sin(x)}{x}.
    \end{align*}
  \end{explanation}
\end{example}

Now we'll compute the second derivative of $\si(x)$.

\begin{example}
  Compute:
  \[
  \dd[^2]{x^2}\si(x)
  \]
  \begin{explanation}
    By our previous work and the quotient rule we see
    \begin{align*}
      \dd[^2]{x^2} \si(x) &= \ddx \frac{\sin(x)}{x}\\
      &=\frac{x\cos(x) -\sin(x)}{x^2}.
    \end{align*}
  \end{explanation}
\end{example}

Now we should find the $y$-intercept.

\begin{example}
  Compute $\si(0)$.
  \begin{explanation}
    Here $\si(0)=0$, as $\si$ is an accumulation
    function, and at $x=0$, no area has been accumulated.
  \end{explanation}
\end{example}

%% Vertical asymptotes

%% end behavior

Now we'll look for critical points, where the derivative is zero or
undefined.

\begin{example}
  Find the critical points of $\si(x)$.
  \begin{explanation}
    The critical points are where $\si'(x)= 0$ or it does not exist.
    Since
    \[
    \si'(x) = \frac{\sin(x)}{x}.
    \]
    We see that this derivative does not exist at $x=0$, and for $x\ne
    0$, $\si'(x)=0$ precisely when $\sin(x)$ is zero. Since $\sin(x)$
    is zero at $x = -2\pi, -\pi, \pi, 2\pi$, we see that the critical
    points are where
    \[
    x= -2\pi, -\pi,0, \pi, 2\pi.
    \]
  \end{explanation}
\end{example}

We'll identify which of these are maximums and minimums.

\begin{example}
  Find the local extrema of $\si(x)$ on the interval $[-2\pi,2\pi]$.
  \begin{explanation}
    The critical points are at
    \[
    x= -2\pi, -\pi,0, \pi, 2\pi.
    \]
    We will use the first derivative test to identifiy which of these
    are local extrema.
    
  \end{explanation}
\end{example}

Inflection points are harder. Let's try our hand.

\begin{example}
  Find the inflection points of $\si(x)$.
  \begin{explanation}
    We start by looking at the second derivative of $\si(x)$,
    \[
    \si''(x) = \frac{x\cos(x) -\sin(x)}{x^2}.
    \]
    The first candidate for an infection point is $x=0$, since the
    second derivative does not exist. To find other inflection points
    on $[-2\pi,2\pi]$, we need to find when
    \[
    x\cos(x) -\sin(x) =0.
    \]
    This is zero when $x=0$, anywhere else?
  \end{explanation}
\end{example}



\end{document}
