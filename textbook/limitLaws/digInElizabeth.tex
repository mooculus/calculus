\documentclass{ximera}

\outcome{Learn things.}

\newcommand{\RR}{\mathbb R}
\renewcommand{\d}{\,d}
\newcommand{\dd}[2][]{\frac{d #1}{d #2}}
\renewcommand{\l}{\ell}
\newcommand{\ddx}{\frac{d}{dx}}
\newcommand{\dfn}{\textbf}
\newcommand{\eval}[1]{\bigg[ #1 \bigg]}

\title{TITLE}
\begin{document}
\begin{abstract}

ABSTRACT

\end{abstract}
\maketitle

The following limits hold.
\begin{enumerate}
\item \parbox{80pt}{Constants:} $\displaystyle \lim_{x\to c} b = b$
\item	\parbox{80pt}{Identity }						$\displaystyle \lim_{x\to c} x = c$
\item	\parbox{80pt}{Sums/Differences:} $\displaystyle \lim_{x\to c}(f(x)\pm g(x)) = L\pm K$
\item	\parbox{80pt}{Scalar Multiples:}	$\displaystyle \lim_{x\to c} b\cdot f(x) = bL$
\item	\parbox{80pt}{Products:}	$\displaystyle \lim_{x\to c} f(x)\cdot g(x) = LK$
\item	\parbox{80pt}{Quotients:} $\displaystyle \lim_{x\to c} f(x)/g(x) = L/K$, ($K\neq 0)$
\item	\parbox{80pt}{Powers:} 	$\displaystyle \lim_{x\to c} f(x)^n = L^n$
\item	\parbox{80pt}{Roots:}		\parbox[t]{185pt}{$\displaystyle \lim_{x\to c} \sqrt[n]{f(x)} = \sqrt[n]{L}$}% \qquad \small (if $n$ is even then $L$ must be greater than 0; when $n$ is odd, it is true for all $L$.)}
\item	\parbox{80pt}{Compositions:} \parbox[t]{200pt}{Adjust our previously given limit situation to: $$\lim_{x\to c}f(x) = L \text{\ and\ } \lim_{x\to L} g(x) = K.$$ Then $ \lim_{x\to c}g(f(x)) = K$.}
\end{enumerate}

\end{document}