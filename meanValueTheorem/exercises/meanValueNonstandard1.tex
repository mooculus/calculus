\documentclass{ximera}
\newcommand{\RR}{\mathbb R}
\renewcommand{\d}{\,d}
\newcommand{\dd}[2][]{\frac{d #1}{d #2}}
\renewcommand{\l}{\ell}
\newcommand{\ddx}{\frac{d}{dx}}
\newcommand{\dfn}{\textbf}
\newcommand{\eval}[1]{\bigg[ #1 \bigg]}

\author{Steven Gubkin\and Nela Lakos}
\license{Creative Commons 3.0 By-NC}

\outcome{Find the values guaranteed by Rolle's Theorem or the Mean Value Theorem.}
\outcome{Understand the statement of the Mean Value Theorem.}
\outcome{Determine whether Rolle's Theorem or the Mean Value Theorem can be applied.}
\begin{document}
\begin{exercise}

Let $f(x) = Ax^2+Bx+C$.  Find the constant $c$, known to exist by the Mean Value Theorem, for which $f'(c) = \frac{f(b)-f(a)}{b-a}$, where $a$ and $b$ are any two distinct real numbers.  The result is a very interesting property of quadratics.


\begin{hint}
First, substitute: 

\[
f'(c) = \frac{f(b)-f(a)}{b-a}=\frac{Ab^2+Bb+C-(Aa^2+Ba+C)}{b-a}
\]
Then, simplify.
\end{hint}


\begin{hint}
\[
f'(c) = \frac{f(b)-f(a)}{b-a}=\frac{A(b^2-a^2)+B(b-a)}{b-a}=\frac{A(b-a)(b+a)+B(b-a)}{b-a}=A(b+a)+B
\]
\end{hint}

\begin{hint}
Then, we compute $f'(x)$.

\[
f'(x) =2A\answer{x}+B 
\]
\end{hint}

\begin{hint}
Then, we solve the equation below for $c$

\[
2A(\answer{c})+B =A\left(\answer{b+a}\right)+B
\]
\end{hint}
\begin{prompt}
	$$c = \frac{b+a}{\answer{2}}$$
\end{prompt}

\end{exercise}
\end{document}
