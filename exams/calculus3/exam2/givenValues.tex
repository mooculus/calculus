\documentclass{ximera}

\newcommand{\RR}{\mathbb R}
\renewcommand{\d}{\,d}
\newcommand{\dd}[2][]{\frac{d #1}{d #2}}
\renewcommand{\l}{\ell}
\newcommand{\ddx}{\frac{d}{dx}}
\newcommand{\dfn}{\textbf}
\newcommand{\eval}[1]{\bigg[ #1 \bigg]}


\author{Bart Snapp \and Jim Fowler}

\license{Creative Commons 4.0 By-SA}


\begin{document}

%For problems $1$--$4$, c
Consider the differentiable function
$F:\R^2\to\R$ where
\begin{align*}
  F(2,1) &= 1        &  F^{(1,0)}(2,1) &= 3 & F^{(2,0)}(2,1) &= -2\\
  F^{(1,1)}(2,1) &=3 &  F^{(0,1)}(2,1) &= 4 & F^{(0,2)}(2,1) &= -8  
\end{align*}

% outcome: write down taylor polynomial
\begin{problem}
  Write down the second degree Taylor polynomial for $F$ centered at $(2,1)$.
  \begin{prompt}
    \[
    P_2(x,y) = \answer{1+3(x-2) + 4(y-1)-(x-2)^2-4(y-1)^2+3(x-2)(y-1)}
    \]
  \end{prompt}
\vspace{1in}
\end{problem}



% outcome: Address misconception about derivatives actually needing to vanish
\begin{problem}
  The value $F(2,1)$ is:
  \begin{multipleChoice}
    \pdfOnly{\begin{multicols}{2}}
  \choice{a local min}
  \choice{a local max}
  \choice{a saddle}
  \choice[correct]{none of these}
    \pdfOnly{\end{multicols}}
  \end{multipleChoice}
\end{problem}

% outcome: second derivative test
\begin{problem}
  Let $G(x,y) = F(x,y) - 3x - 4y$. The value $G(2,1)$ is:
  \begin{multipleChoice}
    \pdfOnly{\begin{multicols}{2}}
\choice{a local min}
\choice[correct]{a local max}
\choice{a saddle}
\choice{none of these}
\pdfOnly{\end{multicols}}
\end{multipleChoice}
\end{problem}


%% \begin{problem}
%%   Give a vector-valued formula for a line passing through the point
%%   $(2, 1)$ when $t=0$ such that when $F$ is constrained to your
%%   line, $F$ has an extrema at the point $(2, 1)$.
%%   \begin{prompt}
%%   Express the ``vector'' defining your line with a unit-vector with
%%   positive $x$ component.
%%   \[
%%   \vecl(t) = \vector{\answer{2},\answer{1}}+t \vector{\answer{4/5},\answer{-2/5}}
%%   \]
%%   \end{prompt}
%% \end{problem}

\begin{problem}
  Let $h(t) = F(t^4+1,t^3)$.  What is $h'(1)$?
  \begin{prompt}
    \[
    h'(1) = \answer{24}
    \]
  \end{prompt}
  \vfill
\end{problem}

\hrule

\begin{problem}
  \begin{selectAll}
    \choice{The function $F(x,y) = x^6+y^6$ locally looks like an elliptic
      paraboloid.}
    \choice{If $D(\vec{c})>0$, then $|\grad F(\vec{c})|>0$.}%% FIRST DERIV VS SECOND
    \choice{If $F^{(1,1)}(\vec{c})=0$, then $F$ has a maximum or minimum at $\vec{c}$.}
    \choice{If $F$ has a saddle-point at $(2,5)$, then there are unit vectors $\uvec{u}$ and $\uvec{v}$ such that $D_{\uvec{u}}(F(2,5))>0$ and $D_{\uvec{v}}(F(2,5))<0$.} %% GRAD IS ZERO
    \choice{If $F$ has a saddle-point at $(1,-1)$ and $G$ has a saddle-point at $(1,-1)$, then $F+G$ has a saddle-point at $(1,-1)$.}
    \choice[correct]{None of these.}
  \end{selectAll}    
\end{problem}





\end{document}
