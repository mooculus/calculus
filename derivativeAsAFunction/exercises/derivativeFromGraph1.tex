\documentclass{ximera}

\newcommand{\RR}{\mathbb R}
\renewcommand{\d}{\,d}
\newcommand{\dd}[2][]{\frac{d #1}{d #2}}
\renewcommand{\l}{\ell}
\newcommand{\ddx}{\frac{d}{dx}}
\newcommand{\dfn}{\textbf}
\newcommand{\eval}[1]{\bigg[ #1 \bigg]}


\outcome{Relate the graph of the function to the graph of its dervative.}
\outcome{Determine whether a piecewise function is differentiable.}

\author{Nela Lakos \and Kyle Parsons}

\begin{document}
\begin{exercise}

The graph of a function $f$ is given in the figure below.


\begin{image}
  \begin{tikzpicture}
    \begin{axis}[
        xmin=-6.3,xmax=6.3,ymin=-6.3,ymax=4.3,
        clip=true,
        unit vector ratio*=1 1 1,
        axis lines=center,
        grid = major,
        ytick={-6,-5,...,4},
    xtick={-6,-5,...,6},
        xlabel=$x$, ylabel=$y$,
        every axis y label/.style={at=(current axis.above origin),anchor=south},
        every axis x label/.style={at=(current axis.right of origin),anchor=west},
      ]
      \addplot[very thick,penColor,domain=-6.3:.9,samples=50] plot{x^2/(x-1)};
      \addplot[very thick,penColor,domain=1:6.3,samples=50] plot{1/x};
      \addplot[penColor,only marks,mark=*] coordinates{(1,1)};
      
      \draw[thick,red,dashed] (axis cs:1,4.3) -- (axis cs:1,-6.3);
      
      \addplot[black,only marks,mark=*] coordinates{(-3,-9/4) (0,0) (.7,.49/(.7-1)};
      \node at (axis cs:-3,-9/4) [black, above left] {$A$};
      \node at (axis cs:0,0) [black, above left] {$B$};
      \node at (axis cs:.7,-.49/.3) [black, right] {$C$};
      
      \node at (axis cs:3.5,1.5) [black] {$y = f(x)$};
      \end{axis}`
  \end{tikzpicture}
\end{image}

Let $m_A$ be the slope of the tangent line to the curve $y = f(x)$ at the point $A$, $m_B$ at the point $B$, and $m_C$ at the point $C$.  Select the correct statement.

\begin{multipleChoice}
\choice{$m_A = m_B = m_C$}
\choice{$m_A < m_B < m_C$}
\choice[correct]{$m_C < m_B < m_A$}
\choice{$m_C < m_A < m_B$}
\choice{$m_A < m_C < m_B$}
\choice{None of the above}
\end{multipleChoice}

\begin{exercise}

With $f$ as before, find the following values or write \verb|ND| if the value is undefined.

\begin{align*}
f'(0) &= \answer{0}\\
f'(1) &= \answer{ND}
\end{align*}

\end{exercise}
\end{exercise}
\end{document}