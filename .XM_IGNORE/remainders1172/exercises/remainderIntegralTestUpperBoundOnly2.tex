\documentclass{ximera}

\newcommand{\RR}{\mathbb R}
\renewcommand{\d}{\,d}
\newcommand{\dd}[2][]{\frac{d #1}{d #2}}
\renewcommand{\l}{\ell}
\newcommand{\ddx}{\frac{d}{dx}}
\newcommand{\dfn}{\textbf}
\newcommand{\eval}[1]{\bigg[ #1 \bigg]}


\author{Jim Talamo}
\license{Creative Commons 3.0 By-NC}

%This exercise parallels remainderIntegralTest1 but only discusses the upper bound
\outcome{Understand the relationship between sequences, sequences of partial sums, and remainders.}

\begin{document}

\begin{exercise}

Consider the series $\sum_{k=1}^{\infty} 8ke^{-2k}$.  Suppose we want to approximate $\sum_{k=1}^{\infty} 8ke^{-2k}$ by using $\sum_{k=1}^{4} 8ke^{-2k}$.  

According to the remainder results for the integral test, find the \emph{maximum} possible error made by using $\sum_{k=1}^{4} 8ke^{-2k}$ to approximate $\sum_{k=1}^{\infty} 8ke^{-2k}$.  Report your answer to four decimal places.

\[
\textrm{maximum possible error} \leq \answer[tolerance=.0001]{.0060}
\]

Thus, $\sum_{k=1}^{\infty} 8ke^{-2k} \approx  \answer[tolerance=.0001]{1.44595}$ (report your answer to four decimal places).

\begin{hint}
We can determine an upper bound for the error from the inequality $ r_n \leq \int_{n}^{\infty} 8xe^{-2x} \d x$.  Using integration by parts , we find

\[\int 8x e^{-2x} \d x = \answer{-(4x+2)e^{2x}}+C.\]

Since we want to use $\sum_{k=1}^{4} 8ke^{-2k}$ for our approximation, we should choose $n = \answer{4}$, and we find, to four decimal places, that for this choice of $n$,

\[  
\int_n^{\infty} 8x e^{-2x} \d x = \answer[tolerance=.0001]{.0060}.
\]

\end{hint}


%%%%%%%%%%%%%%%%%%%%%%%%%%%%%%%%%%%%%%%%%%%%%%%%%%%%%%%%%%%%%%%%

\begin{exercise}
If we use the integral test estimate what is the smallest value of $N$ that should be chosen so $\sum_{k=1}^{N} 8ke^{-2k}$ is within $.0001$ of the value of $\sum_{k=1}^{\infty} 8ke^{-2k}$?

\[
N=\answer{7}
\]

\begin{hint}
We want $N$ so $\int_N^{\infty} 8xe^{-2x} \d x \leq .0001$.  

\[
\int_N^{\infty} 8xe^{-2x} \d x = \lim_{b \to \infty} \eval{\answer{-(4x+2)e^{-2x}}}_N^{b} = \lim_{b \to \infty} \eval{-(4b+2)e^{-2b}+(4N+2)e^{-2N}} = \answer{(4N+2)e^{-2N}}.
\]

(Notice that $\lim_{b \to \infty} \eval{-(4b+2)e^{-2b}} = \lim_{b \to \infty} \eval{-\frac{4b+2}{e^{2b}}} = \answer{0}$).

Thus, we set $\answer{(4N+2)e^{-2N}} \leq .0001$ and we can find that the smallest integer $N$ by using either a guess and check method, or a computer algebra system (or graphing calculator) to find that $N = \answer{7}$.

\end{hint}
\end{exercise}

\end{exercise}
\end{document}


