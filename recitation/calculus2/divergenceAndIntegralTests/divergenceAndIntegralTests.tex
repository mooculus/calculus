\documentclass[handout]{ximera}
%handout:  for handout version with no solutions or instructor notes
%handout,instructornotes:  for instructor version with just problems and notes, no solutions
%noinstructornotes:  shows only problem and solutions

%% handout
%% space
%% newpage
%% numbers
%% nooutcomes

%I added the commands here so that I would't have to keep looking them up
%\newcommand{\RR}{\mathbb R}
%\renewcommand{\d}{\,d}
%\newcommand{\dd}[2][]{\frac{d #1}{d #2}}
%\renewcommand{\l}{\ell}
%\newcommand{\ddx}{\frac{d}{dx}}
%\everymath{\displaystyle}
%\newcommand{\dfn}{\textbf}
%\newcommand{\eval}[1]{\bigg[ #1 \bigg]}

%\begin{image}
%\includegraphics[trim= 170 420 250 180]{Figure1.pdf}
%\end{image}

%add a ``.'' below when used in a specific directory.

\newcommand{\RR}{\mathbb R}
\renewcommand{\d}{\,d}
\newcommand{\dd}[2][]{\frac{d #1}{d #2}}
\renewcommand{\l}{\ell}
\newcommand{\ddx}{\frac{d}{dx}}
\newcommand{\dfn}{\textbf}
\newcommand{\eval}[1]{\bigg[ #1 \bigg]}




\author{Tom Needham}

\outcome{Use the divergence test to conclude that a series diverges.}
\outcome{Use the integral test to determine whether a series converges or diverges.}

\title[]{The Divergence Test and The Integral Test}

\begin{document}
\begin{abstract}
\end{abstract}
\maketitle

\vspace{-0.9in}

\section{Discussion Questions}

\begin{problem}
Determine whether the following statements are true or false, and explain your answer.

I. If the sequence $\{a_n\}$ satisfies $\lim_{n\rightarrow \infty} a_n = 0$, then the series $\sum_{n=1}^\infty a_n$ converges.

II. If the sequence $\{a_n\}$ satisfies $\lim_{n \rightarrow \infty} a_n \neq 0$, then the series $\sum_{n=1}^\infty a_n$ diverges.

III. If the series $\sum_{n=1}^\infty a_n$ converges, then $\lim_{n\rightarrow \infty} a_n = 0$. 

IV. If the series $\sum_{n=1}^\infty a_n$ diverges, then $\lim_{n\rightarrow \infty} a_n \neq 0$. 

\end{problem}

\begin{freeResponse}
solution
\end{freeResponse}

\begin{problem}
Suppose that the series $\sum_{k=1}^\infty a_k$ has sequence of partial sums $\{s_n\}$ given by the explicit formula
$$
s_n = \frac{2n+1}{4n+3}.
$$
Student A claims that the series $\sum_{k=1}^\infty a_k$ converges because $\lim_{n \rightarrow \infty} s_n = \frac{1}{2}$. Student B claims that $\lim_n\rightarrow s_n = \frac{1}{2} \neq 0$, so the divergence test implies that the series $\sum_{k=1}^\infty a_k$ diverges. Which (if either) student is correct?
\end{problem}

\section{Group Work}

\begin{problem}
Determine whether the series
$$
\sum_{k=1}^\infty \frac{k^3 - \ln(k+10) + k^5}{(2k+100)^5}
$$
converges or diverges.
\end{problem}

\begin{freeResponse}
solution.
\end{freeResponse}


\begin{problem}
Determine whether the series
$$
\sum_{k=2}^\infty \frac{1}{k \ln (k)}
$$
converges or diverges.
\end{problem}

\begin{problem}
For each part, determine which of the following properties each of the given sequences $\{a_k\}$ and $\{s_n\}$ has: monotone increasing, monotone decreasing, bounded above, bounded below. 

I. Let $\{a_k\}_{k=1}^\infty$ be the sequence defined by $a_k = \frac{1}{k^3}$ and let $\{s_n\}_{n=1}^\infty$ denote its sequence of partial sums. 

II. Let $\{a_k\}_{k=1}^\infty$ be the sequence defined by $a_k = \frac{2^k}{2^k +1}$ and let $\{s_n\}_{n=1}^\infty$ denote its sequence of partial sums.
\end{problem}

\end{document}
