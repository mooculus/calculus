\documentclass{ximera}

\newcommand{\RR}{\mathbb R}
\renewcommand{\d}{\,d}
\newcommand{\dd}[2][]{\frac{d #1}{d #2}}
\renewcommand{\l}{\ell}
\newcommand{\ddx}{\frac{d}{dx}}
\newcommand{\dfn}{\textbf}
\newcommand{\eval}[1]{\bigg[ #1 \bigg]}


\begin{document}
\outcome{Recognize and distinguish between secant and tangent lines.}
\outcome{Understand the definition of the derivative at a point.}
An oil tank is to be drained for cleaning. There are $V(t)$ gallons of oil left in the tank $t$ minutes after the draining began, where $V(t)=45(60-t)^2$.
\begin{exercise}
Find the \textbf{average rate} at which oil drains during the first $15$ minutes. 
\[
\answer{-4725}\unit{gal/m}
\]
\begin{exercise}
Find the \textbf{average rate} at which oil drains during the time interval $[10,15]$. 
\[
\answer{-4275}\unit{gal/m}
\]
\begin{exercise}
Find the \textbf{rate} at which oil is flowing out the tank $15$ minutes after the draining began.
\[
\answer{-4050}\unit{gal/m}
\]
\begin{exercise}
Find the \textbf{average rate}, $AR(\Delta t)$, at which oil drains during the time interval $[15+\Delta t,15]$ if $-1<\Delta t<0$. 
\[
AR(\Delta t)=\answer{-45(90+\Delta t)}\unit{gal/m}
\]
Now find the \textbf{average rate}, $AR(\Delta t)$, at which oil drains during the time interval $[15,15+\Delta t]$ if $0<\Delta t<1$. 
\[
AR(\Delta t)=\answer{45(-90+\Delta t)}\unit{gal/m}
\]
\begin{exercise}
Using your result, compute the limit
\[
\lim_{\Delta t\to 0}AR(\Delta t)=\answer{-4050}\unit{gal/m}
\]
\begin{exercise}
What does this limit $\lim_{\Delta t\to 0}AR(\Delta t)$ represent?
\begin{multipleChoice}
\choice{The average rate of change at $\Delta t=0$.}
\choice{The quantity $V(15)$.}
\choice[correct]{The derivative of $V(t)$ at $t=15$.}
\choice{The derivative of $V(t)$ at $t=0$.}
\end{multipleChoice}
\begin{exercise}
Suppose the oil tank has been completely drained and is now being refilled. Let $Vol(t)$ denote the gallons of oil in the tank $t$ minutes after the refilling process began. Suppose we are told that the function $Vol(t)$ is a \textbf{differentiable function} and that the tank is so large that it will take a while to fill. 

Find an expression for the \textbf{average rate}, $\bar{r}(t)$, at which oil flows into the tank in the first $t$ minutes. Express $\bar{r}$ \textbf{symbolically} in terms of the function $Vol(t)$, the variable $t$ and the quantity $Vol(0)$.
\[
\bar{r}(t)=\answer{\frac{Vol(t)-Vol(0)}{t}}\unit{gal/m}
\]
%% \begin{hint}
%% While the parameters of the problem have changed, the crux of the exercise has not. Refer to your thinking in the first to come up with a solution. 
%% \end{hint}
\begin{exercise}
Now let $\bar{r}_{\Delta}(t_0,t_f)$ be the function denoting the \textbf{average rate} at which oil drains in the time interval $[t_0,t_f]$ for $t_0<t_f$. Express $\bar{r}_{\Delta}$ \textbf{symbolically} in terms of the function $Vol$ and the variables $t_0$ and $t_f$.
\[
\bar{r}_{\Delta}(t_0,t_f)=\answer{\frac{Vol(t_f)-Vol(t_0)}{t_f-t_0}}
\]
\begin{exercise}
Find the \textbf{rate} at which oil is flowing out the tank $15$ minutes after the draining began. (This is not a trick question! Your answer should be particularly simple.)
\[
\answer{V'(15)}\unit{gal/m}
\]
\end{exercise}
\end{exercise}
\end{exercise}
\end{exercise}
\end{exercise}
\end{exercise}
\end{exercise}
\end{exercise}
\end{exercise}
\end{document}
