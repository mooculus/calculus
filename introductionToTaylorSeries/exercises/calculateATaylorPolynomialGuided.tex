\documentclass{ximera}

\newcommand{\RR}{\mathbb R}
\renewcommand{\d}{\,d}
\newcommand{\dd}[2][]{\frac{d #1}{d #2}}
\renewcommand{\l}{\ell}
\newcommand{\ddx}{\frac{d}{dx}}
\newcommand{\dfn}{\textbf}
\newcommand{\eval}[1]{\bigg[ #1 \bigg]}


\author{Jim Talamo}
\license{Creative Commons 3.0 By-bC}


\outcome{}


\begin{document}

\begin{exercise}
For the function $f(x) = e^{2x} + 4\cos(x)$, find the sum of the first 4 nonzero terms in the Taylor series centered at $x=0$.

Rather than taking derivatives to find the coefficients, we can use the rules for sums, products and compositions. The relevant series to know here are:

\begin{selectAll}
\choice{$\sin(x)$}
\choice[correct]{$\cos(x)$}
\choice[correct]{$e^x$}
\choice{$\frac{1}{1-x}$}
\end{selectAll}

Also, since we are asked to find the sum of the first four nonzero terms in the Taylor series for $f(x)=e^{2x} + 4\cos(x)$ centered at $x=0$, we should:

\begin{multipleChoice}
\choice{Substitute $k=0,1,2,3,$ and $4$ into the summand above and write out the sum of those terms}
\choice[correct]{Write out the first four powers of $x$ in the series above that have a nonzero coefficient}
\choice{Write out terms in the series above until we exhibit the $x^4$ term}
\end{multipleChoice}

\begin{exercise}
%%%%%%%%%%%%%%%%%%%%%
\begin{exercise}
For the term involving the exponential, it may not be clear how many terms we will need to exhibit, so let's write out five terms (we can always discard extra terms or add more later if necessary):

\[
e^x= \answer{1}+\answer{1}x+\answer{\frac{1}{2}}x^2+\answer{\frac{1}{6}}x^3+\answer{\frac{1}{24}}x^4 + \cdots
\]

\begin{exercise}
Now use the rule for composition to find the series for $e^{2x}$ by substituting $2x$ in for $x$ in the series above:

\[
e^{2x}= 1+\left(\answer{2x} \right) +\frac{1}{2} \left(\answer{2x} \right)^2 + \frac{1}{6}\left(\answer{2x} \right)^3+\frac{1}{24}\left(\answer{2x} \right)^4+ \cdots
\]

Simplifying this gives:

\[
e^{2x} = \answer{1+2x+2x^2+\frac{4}{3}x^3+\frac{2}{3}x^4}+ \cdots
\]

\end{exercise}
\end{exercise}
%%%%%%%%%%%%%%%%%%%%%
\begin{exercise}
For the term involving the cosine, it may not be clear how many terms we will need to exhibit, so let's write out five terms (we can always discard extra terms or add more later if necessary):

\[
\cos(x)= \answer{1}+\answer{-\frac{1}{2}}x^2+\answer{\frac{1}{4!}}x^4+\answer{-\frac{1}{6!}}x^6+\answer{\frac{1}{8!}}x^8 + \cdots
\]

\begin{exercise}
Now use the rule for composition to find the series for $4 \cos(x)$ by multiplying the series above by $4$:

\[
4 \cos(x) = \answer{4-2x^2+\frac{1}{6}x^4- \frac{4}{6!} x^6 +\frac{4}{8!}x^8}+ \cdots
\]

\end{exercise}
\end{exercise}
%%%%%%%%%%%%%%%%%%%%%
Now, note that we have exhibited several terms in the relevant series.  To find the sum of the first four nonzero terms in the sum, note that if we add the the expressions:

\[e^{2x} = 1+x+2x^2+\frac{4}{3}x^3+\frac{2}{3}x^4+ \cdots\]

and

\[ 4 \cos(x) = 4-2x^2+\frac{1}{6}x^4- \frac{4}{6!} x^6 +\frac{4}{8!}x^8+ \cdots \]

the series for $e^{2x}+4 \cos(x)$ will be valid:

\begin{multipleChoice}
\choice[correct]{up to the $x^4$ term}
\choice{up to the $x^8$ term}  
\end{multipleChoice}

Note that when we add these, the result will only be valid up to the lower/lowest common power of $x$!  Thus, we are only guaranteed that the expression has the correct coefficients up to the $x^4$ term here.  If this does not produce at least $4$ nonzero powers of $x$, we must use more terms.  Note however:

\[ 
\begin{array}{rlrrrrr}
e^{2x} &= 1&+x&+2x^2&+\frac{4}{3}x^3&+\frac{2}{3}x^4&+ \cdots \\ [3ex]
4 \cos(x) &= 4& &-2x^2& &+\frac{1}{6}x^4& + \cdots \\[2ex]
\hline \\ [1ex]
e^{2x}+4\cos(2x) & = \answer{5}&+\answer{1}x&+\answer{0}x^2&+\answer{\frac{4}{3}}x^3&+\answer{\frac{5}{6}}x^4& +\cdots
\end{array}
\]
(add the coefficients of the like powers together)

Are there four powers of $x$ whose coefficients are nonzero?

\begin{multipleChoice}
\choice{No}
\choice[correct]{Yes}
\end{multipleChoice}

Thus, the sum of the first four nonzero terms in the Taylor series centered at $x=0$ for  $f(x) = e^{2x} + 4\cos(x)$ is:

\[
f(x) = e^{2x} + 4\cos(x) = \answer{5+x+\frac{4}{3}x^3+\frac{5}{6}x^4} + \cdots 
\]

\begin{exercise}
Using the coefficients of the above, determine what $f''(0)$ and $f'''(0)$ are.

\[
f''(0) = \answer{0} \qquad \qquad \qquad f'''(0) = \answer{8}
\]

\begin{hint}
Remember that the coefficients of the series and the derivatives of the function evaluated at the center are related via the formula:

\[
a_k = \frac{f^{(k)}(c)}{k!}
\]
where $a_k$ is the coefficient of $(x-c)^k$.
\end{hint}

\end{exercise}
\end{exercise}
%%%%%%%%%%%%%%%%%%%%
\end{exercise}

\end{document}
