\documentclass{ximera}

\newcommand{\RR}{\mathbb R}
\renewcommand{\d}{\,d}
\newcommand{\dd}[2][]{\frac{d #1}{d #2}}
\renewcommand{\l}{\ell}
\newcommand{\ddx}{\frac{d}{dx}}
\newcommand{\dfn}{\textbf}
\newcommand{\eval}[1]{\bigg[ #1 \bigg]}


\begin{document}
\begin{exercise}

\outcome{Discuss what it means for a limit to equal $\infty$.}
\outcome{Understand the relationship between limits and vertical asymptotes.}
\outcome{Evaluate the limit as $x$ approaches a point where there is a vertical asymptote.}
\outcome{Recognize when a limit is indicating there is a vertical asymptote.}

Let 
\[
f(x)=\begin{cases}
\frac{x^2-x-12}{x+3} & \text{if $x<4$ and $x\ne -3$}\\
5 & \text{if $x=-3$}\\
\frac{x}{x-4} & \text{if $x>4$}.
\end{cases}
\]
Determine if the following limits exist. If they exist, compute them
analytically using the limit laws and techniques for computing
limits. If a limit does not exist, writen `DNE' and explain why.  Do
not use a table of values, a graph, or L'H\^opital's rule to justify
your answer.
\begin{exercise}
  \[
  \lim_{x\to -3}f(x)=\answer{-7}
  \]
\end{exercise}
\begin{exercise}
  \[
  \lim_{x\to 4^{-}}f(x)=\answer{0}
  \]
\end{exercise}
\begin{exercise}
  \[
  \lim_{x\to 4^+}f(x)=\answer{\infty}
  \]
  \begin{exercise}
    When $x$ is greater than $4$,
    \[
    f(x) = \answer{\frac{x}{x-4}}.
    \]
    As $x$ approaches $4$ from the right, $x-4$ is \wordChoice{\choice[correct]{positive}\choice{negative}\choice{zero}} and approaching zero.
  \end{exercise}
\end{exercise}
\begin{exercise}
  \[
  \lim_{x\to 4}f(x)=\answer{DNE}
  \]   
\end{exercise}

List the vertical asymptotes of $f$ from least to greatest: $x=\answer{4}$


\end{exercise}
\end{document}

