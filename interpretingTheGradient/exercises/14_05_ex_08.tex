\documentclass{ximera}

\newcommand{\RR}{\mathbb R}
\renewcommand{\d}{\,d}
\newcommand{\dd}[2][]{\frac{d #1}{d #2}}
\renewcommand{\l}{\ell}
\newcommand{\ddx}{\frac{d}{dx}}
\newcommand{\dfn}{\textbf}
\newcommand{\eval}[1]{\bigg[ #1 \bigg]}


\author{David Guichard \and Neal Koblitz \and H. Jerome Keisler \and Albert Scheller \and Barry Balof \and Mike Wills \and Matthew Carr}
\license{CC-By-SA-NC}

\acknowledgement{https://www.whitman.edu/mathematics/multivariable/}

\begin{document}
\begin{exercise}
A plane perpendicular to the $(x,y)$-plane contains the point
$(3,2,2)$ on the paraboloid $36z=4x^2+9y^2$. The line tangent to the
cross-section of the paraboloid created by this plane is parallel to
the $(x,y)$-plane at this point. Find an equation of the plane.
\begin{prompt}
\[
x\cdot \answer{2} + y \cdot \answer{3} + z \cdot \answer{0} = 12
\]
\end{prompt}

\begin{hint}
  Let $\vec{n}=\vector{a,b,c}$ be the normal vector for the plane in
  question.
\end{hint}

\begin{hint}
  We know that $c=0$ because the plane is perpendicular to the
  $(x,y)$-plane.
\end{hint}

\begin{hint}
  The paraboloid $36z=4x^2+9y^2$ can be thought of as the level surface
  \[
  0 =4x^2+9y^2-36z
  \]
  of some new function of several variables:
  \[
  G(x,y,z) = 4x^2+9y^2-36z
  \]
\end{hint}

\begin{hint}
  The gradient vector is normal to level surfaces.
\end{hint}


\begin{hint}
  $\grad G = \vector{8x,18y,-36}$
\end{hint}

\begin{hint}
  $\grad G(3,2,2) = \vector{24,36,-36}$
\end{hint}

\begin{hint}
  Since line tangent to the cross-section of the paraboloid created by
  this plane is parallel to the $(x,y)$-plane at $(3,2,2)$, the
  tangent vector of the cross-section is parallel to
  $\vec{n}\cross\grad G(3,2,2)$.
\end{hint}

\begin{hint}
  $\vec{n}\cross\grad(3,2,2) = \vector{-36b,36a,36a-24b}$
\end{hint}

\begin{hint}
  Moreover, since this vector is parallel to the $(x,y)$-plane,
  \[
  36a-24b = 0.
  \]
\end{hint}

\begin{hint}
  So $3a=2b$.
\end{hint}

\begin{hint}
  This means that $\vec{n}$ is parallel to $\vector{a,3a/2,0}$, which
  is also parallel to $\vector{2a,3a,0}$.
\end{hint}

\begin{hint}
  Set $\vec{n}= \vector{2,3,0}$. Now since we know a point on the
  plane, we can find the plane.
\end{hint}

\end{exercise}
\end{document}
