\documentclass{ximera}

\newcommand{\RR}{\mathbb R}
\renewcommand{\d}{\,d}
\newcommand{\dd}[2][]{\frac{d #1}{d #2}}
\renewcommand{\l}{\ell}
\newcommand{\ddx}{\frac{d}{dx}}
\newcommand{\dfn}{\textbf}
\newcommand{\eval}[1]{\bigg[ #1 \bigg]}


\title{Linear approximation}

\begin{document}

\begin{abstract}
%Stuff can go here later if we want!
\end{abstract}

\maketitle

\begin{sectionOutcomes}

After completing this section, students should be able to do the following.

\begin{itemize}
	\item Define linear approximation as an application of the tangent to a curve.
	\item Find the linear approximation to a function at a point and use it to approximate the function value.
	\item Identify when a linear approximation can be used.
	\item Label a graph with the appropriate quantities used in linear approximation.
	\item Find the error of a linear approximation.
	\item Use the second derivative to discuss whether the linear approximation over or underestimates the actual function value.
	\item Contrast the notation and meaning of $dy$ versus $\Delta y$.
	\item Understand that the error shrinks faster than the displacement in the input.
        \item Justify the chain rule via the composition of linear approximations.
\end{itemize}

\end{sectionOutcomes}

\end{document}
