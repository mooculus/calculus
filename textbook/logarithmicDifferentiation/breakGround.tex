\documentclass{ximera}

\newcommand{\RR}{\mathbb R}
\renewcommand{\d}{\,d}
\newcommand{\dd}[2][]{\frac{d #1}{d #2}}
\renewcommand{\l}{\ell}
\newcommand{\ddx}{\frac{d}{dx}}
\newcommand{\dfn}{\textbf}
\newcommand{\eval}[1]{\bigg[ #1 \bigg]}


\outcome{}

\title[Break-Ground:]{Multiplication to addition}

\begin{document}
\begin{abstract}
Two young mathematicians think about derivatives and logarithms.
\end{abstract}
\maketitle

\begin{dialogue}
\item[Devyn] Riley, why is the product rule so much harder than the sum rule?
\item[Riley] Ever since 2nd grade, I've known that multiplication is
  \textit{harder} than addition.
\item[Devyn] I know! I was reading somewhere that a slide-rule somehow
  turns ``multiplication into addition.''
\item[Riley] Wow! I wonder how that works?
\item[Devyn] I \textit{think} it has something to do with logs?
\item[Riley] What? How does this work?
\end{dialogue}

Devyn is right, logarithms are used (and were invented) to convert
difficult multiplication problems into simpler addition problems.

\begin{problem}
  Let $f(x) = \sin(x) \cdot \cos(x) \cdot e^x$. Compute
  \[
  \ddx f(x)\begin{prompt} = \answer{\cos(x)\cos(x)e^x - \sin(x)\sin(x)e^x + \sin(x)\cos(x) e^x}\end{prompt}
  \]
\end{problem}

Now, let's see what happens if we do the same problem but we take the
natural log of both sides first:
\begin{align*}
  f(x) &= \sin(x)\cdot\cos(x)\cdot e^x\\
  \ln(f(x)) &= \ln(\sin(x)\cdot\cos(x)\cdot e^x)\\
  \ln(f(x)) &=\ln(\sin(x)) + \ln(\cos(x)) + \ln(e^x)
\end{align*}

Now we'll take the derivative of both sides of the equation.
By the chain rule
\[
\ddx \ln(f(x))= \frac{f'(x)}{f(x)}
\]


\begin{problem}
  Compute (using the chain rule)
  \[
  \ddx \ln(\sin(x))  \begin{prompt}=\answer{\frac{\cos(x)}{\sin(x)}}\end{prompt}
  \]
\end{problem}

\begin{problem}
  Compute (using the chain rule)
  \[
  \ddx \ln(\cos(x))  \begin{prompt}=\answer{\frac{-\sin(x)}{\cos(x)}}\end{prompt}
  \]
\end{problem}

\begin{problem}
  Compute (using the chain rule)
  \[
  \ddx \ln(e^x)  \begin{prompt}=\answer{1}\end{prompt}
  \]
\end{problem}

So we have
\begin{align*}
  \frac{f'(x)}{f(x)} &= \frac{\cos(x)}{\sin(x)} - \frac{\sin(x)}{\cos(x)} + 1\\
  f'(x) &= f(x) \left(\frac{\cos(x)}{\sin(x)} - \frac{\sin(x)}{\cos(x)} + 1\right)\\
  &= \sin(x)\cos(x)e^x\left(\frac{\cos(x)}{\sin(x)} - \frac{\sin(x)}{\cos(x)} + 1\right)
\end{align*}



%% \begin{xarmaBoost}
%%   Write down at least \textbf{five} questions for this lecture. After
%%   you have your questions, label them as ``Level 1,'' ``Level 2,'' or
%%   ``Level 3'' where:
%% \begin{description}
%% \item[Level 1] Means you know the answer, or know exactly how to do
%%   this problem.
%% \item[Level 2] Means you think you know how to do the problem.
%% \item[Level 3] Means you have no idea how to do the problem.
%% \end{description}
%% \begin{freeResponse}
%% \end{freeResponse}
%% \end{xarmaBoost}



\end{document}
