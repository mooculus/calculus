\documentclass{ximera}

\newcommand{\RR}{\mathbb R}
\renewcommand{\d}{\,d}
\newcommand{\dd}[2][]{\frac{d #1}{d #2}}
\renewcommand{\l}{\ell}
\newcommand{\ddx}{\frac{d}{dx}}
\newcommand{\dfn}{\textbf}
\newcommand{\eval}[1]{\bigg[ #1 \bigg]}


\title{Operations and compositions of functions}
\begin{document}
\begin{abstract}
  We discuss operations of functions, and compositions of functions.
\end{abstract}
\maketitle


Functions can be combined using our standard operations: addition, subtraction, multiplication and division.

\begin{example}
Suppose we have: 
\begin{align}
  f(x) &={{x}^{2}}+4 \\ 
  g(x) &=\frac{1}{x} \\ 
  h(x) &=3+\sqrt{x}  
\end{align}

Let's first note the domain of each function.  The domain of $f$ is
$(-\infty ,\infty )$, domain of $g$ is $(-\infty ,0)\cup (0,\infty )$,
domain of $h$ is $[0,\infty)$
\begin{enumerate}
\item Find $f+g$
  
  Solution: $f+g={{x}^{2}}+4+\frac{1}{x}$  The domain is $(-\infty ,0)\cup (0,\infty )$
\item Find$g-h$ 

  Solution: $g-h=\frac{1}{x}-(3+\sqrt{x})=\frac{1}{x}-3-\sqrt{x}$ The domain is $(0,\infty )$
\item Find$f\cdot h$
  
Solution: $f\cdot g=\left( {{x}^{2}}+4 \right)\cdot \left( 3+\sqrt{x} \right)=3{{x}^{2}}+4\sqrt{x}+{{x}^{3/2}}+12$ The domain is$[0,\infty )$
\item Find$\frac{g}{h}$ 

  Solution:
  \begin{align*}
    \frac{g}{h} &=\frac{\left( \frac{1}{x} \right)}{3+\sqrt{x}}\\
    &=\frac{1}{x}\div \left( 3+\sqrt{x} \right)\\
    &=\frac{1}{x}\cdot \frac{1}{\left( 3+\sqrt{x} \right)}\\
    &=\frac{1}{x\left( 3+\sqrt{x} \right)}\\
    &=\frac{1}{3x+{{x}^{3/2}}}
  \end{align*}
  The domain is $(0,\infty )$
\end{enumerate}
\end{example}

We can generalize to say:
\begin{definition}
If $f$ is a function with domain $A$ and $g$ is a function with domain $B$, then we can define operations on these functions as:
\begin{align*}
(f+g)(x) &=f(x)+g(x) Domain A \cap B\\
(f-g)(x) &=f(x)-g(x)  Domain A \cap B\\
(fg)(x) &=f(x)g(x)  Domain A \cap B\\
\frac{f}{g}(x) = \frac{f(x)}{g(x)} Domain A \cap B except where g(x)=0
\end{align*}
\end{definition}

We also have what we call composition of functions.  Composition of
functions can be thought of as putting one function inside another.
We denote this is $f(g(x))=(f\circ g)(x)$

\begin{example}
 Suppose we have:
\begin{align}
  f(x)&={{x}^{2}}+5x+4 \\ 
  g(x)&=\frac{1}{x} \\ 
  h(x)&=3+\sqrt{x}
\end{align}

Let’s first note the domain of each function.  The domain of $f$
is $(-\infty ,\infty )$, domain of g is $(-\infty ,0)\cup (0,\infty )$,
domain of $h$ is $[0,\infty )$
 
a)	$f\circ g$ 
Solution: $f\circ g=f(g(x))$ We can think of this as everywhere we see an “x” in f, we’ll replace it with the function g. So we have: $f(g(x))={{\left( \frac{1}{x} \right)}^{2}}+5\left( \frac{1}{x} \right)+4$.  $(-\infty ,0)\cup (0,\infty )$
b)	$f\circ f$ 
Solution: $f\circ f={{\left( {{x}^{2}}+5x+4 \right)}^{2}}+5\left( {{x}^{2}}+5x+4 \right)+4$.  The domain is $(-\infty ,\infty )$
c)	$h\circ g$  
Solution: $h\circ g=h(g(x))=3+\sqrt{\frac{1}{x}}$ The domain is $(0,\infty )$
d)	$g\circ h$ 
Solution:$g\circ h=g(h(x))=\frac{1}{\sqrt{x}}$ The domain is $(0,\infty )$
\end{example}
\end{document}
