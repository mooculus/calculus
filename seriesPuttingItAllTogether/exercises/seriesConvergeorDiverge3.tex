\documentclass{ximera}

\newcommand{\RR}{\mathbb R}
\renewcommand{\d}{\,d}
\newcommand{\dd}[2][]{\frac{d #1}{d #2}}
\renewcommand{\l}{\ell}
\newcommand{\ddx}{\frac{d}{dx}}
\newcommand{\dfn}{\textbf}
\newcommand{\eval}[1]{\bigg[ #1 \bigg]}


\author{Jim Talamo}
\license{Creative Commons 3.0 By-bC}


\outcome{}


\begin{document}
\begin{exercise}

Select all of the following series that converge.

\begin{selectAll}
\choice[correct]{$\sum_{k=1}^{\infty}ke^{-k^2}$}
\choice[correct]{$\sum_{n=1}^{\infty} \frac{3\sqrt{n^5}}{n^4}$}
\choice{$\sum_{n=1}^{\infty}\frac{\sqrt[3]{n^6-n}}{10+3n^2}$}
\choice[correct]{$\sum_{n=1}^{\infty} \frac{ne^n}{n!}$}
\choice[correct]{$\sum_{n=1}^{\infty}(1+5^n)3^{2-2n}$}
\end{selectAll}

\begin{hint}
We say that a convergence test is \emph{applicable} if the assumptions of the test are met.  We say that a test is \emph{conclusive} if it can be used to determine whether a series converges or diverges.

%%%%%%%%%%%  FOR FIRST SERIES %%%%%%%%%%%
\begin{question}
For the series $\sum_{k=1}^{\infty}ke^{-k^2}$, is the series geometric or a $p$-series?

\begin{multipleChoice}
\choice{The series is a geometric series.}
\choice{The series is a $p$-series.}
\choice[correct]{The series is neither a geometric series nor a $p$-series.}
\end{multipleChoice}

 Which of the following tests are \emph{applicable}?

\begin{selectAll}
\choice[correct]{divergence test}
\choice[correct]{integral test}
\choice[correct]{comparison test}
\choice[correct]{limit comparison test}
\choice[correct]{ratio test}
\choice[correct]{root test}
\choice{alternating series test}
\end{selectAll}


Note that the improper integral $\int_{1}^{\infty} xe^{-x^2} \d x$ corresponding to our series would be easy to evaluate since the derivative of $-x^2$ is $-2x$. This suggests we try the integral test. 

The improper integral 
\[
\int_{1}^{\infty} xe^{-x^2} \d x
\]

\begin{multipleChoice}
\choice[correct]{converges}
\choice{diverges}
\end{multipleChoice}

Therefore we can conclude that the series 

\[
 \sum_{k=1}^{\infty}ke^{-k^2}
\]

\begin{multipleChoice}
\choice{diverges by the integral test}
\choice[correct]{converges by the integral test} 
\end{multipleChoice}




\end{question}

%%%%%%%%%%%  FOR SECOND SERIES %%%%%%%%%%%


\begin{question}
For the series $\sum_{n=1}^{\infty} \frac{3\sqrt{n^5}}{n^4}$, is the series geometric or a $p$-series?

\begin{multipleChoice}
\choice{The series is a geometric series.}
\choice[correct]{The series is a $p$-series.}
\choice{The series is neither a geometric series nor a $p$-series.}
\end{multipleChoice}


This means that 
\[
 \sum_{n=1}^{\infty} \frac{3\sqrt{n^5}}{n^4}
\]

\begin{multipleChoice}
\choice[correct]{converges}
\choice{diverges}
\end{multipleChoice}

since $p=\answer{ \frac{3}{2}}$. 


\end{question}

%%%%%%%%%%%  FOR THIRD SERIES %%%%%%%%%%%
\begin{question}
For the series $\sum_{n=1}^{\infty}\frac{\sqrt[3]{n^6-n}}{10+3n^2}$, note that:

\begin{multipleChoice}
\choice{The series is a geometric series.}
\choice{The series is a $p$-series.}
\choice{The series is telescoping.}
\choice[correct]{The series is none of the above}
\end{multipleChoice}

For very large $n$, we see that the numerator behaves like $\answer{n^2}$ and the dominant term in the denominator 
is $\answer{3n^2}$. 

Thus $\lim_{n \to \infty} \frac{\sqrt[3]{n^6-n}}{10+3n^2}=\answer{ \frac{1}{3}}$. 



Thus our series

\[
\sum_{n=1}^{\infty}\frac{\sqrt[3]{n^6-n}}{10+3n^2}
\]

\begin{multipleChoice}
\choice{converges}
\choice[correct]{diverges}
\end{multipleChoice}


\end{question}

%%%%%%%%%%%  FOR FOURTH SERIES %%%%%%%%%%%
\begin{question}
For the series  $\sum_{n=1}^{\infty} \frac{ne^n}{n!}$, is the series geometric or a $p$-series?

\begin{multipleChoice}
\choice{The series is a geometric series.}
\choice{The series is a $p$-series.}
\choice[correct]{The series is neither a geometric series nor a $p$-series.}
\end{multipleChoice}


 Which of the following tests are \emph{applicable}?

\begin{selectAll}
\choice[correct]{divergence test}
\choice[correct]{comparison test}
\choice[correct]{limit comparison test}
\choice[correct]{ratio test}
\choice[correct]{root test}
\choice{alternating series test}
\end{selectAll}


The presence of the factorial indicates that the ratio test would likely be a good test to use. 

Using the ratio test tells us that 
\[
\sum_{n=1}^{\infty} \frac{ne^n}{n!}
\]

\begin{multipleChoice}
\choice[correct]{converges}
\choice{diverges}
\end{multipleChoice}

\end{question}

%%%%%%%%%%%  FOR FIFTH SERIES %%%%%%%%%%%
\begin{question}

For the series $\sum_{n=1}^{\infty}(1+5^n)3^{2-2n}$, is the series geometric or a $p$-series?

\begin{multipleChoice}
\choice{The series is a geometric series.}
\choice{The series is a $p$-series.}
\choice[correct]{The series is neither a geometric series nor a $p$-series.}
\end{multipleChoice}

Let's rewrite our series into a more convenient form. 

\begin{align*}
\sum^{\infty}_{n=1} (1+5^n)3^{2-2n}&=\sum^{\infty}_{n=1}\frac{(1+5^n)9}{3^{2n}} \\
&=9\sum^{\infty}_{n=1} \left( \frac{1}{3^{2n}} + \frac{5^n}{3^{2n}} \right)
\end{align*}

Recall that if $\sum^{\infty}_{n=1} a_n$ and $\sum^{\infty}_{n=1} b_n$ both converge then $\sum^{\infty}_{n=1} \left( a_n+b_n \right)$ also converges. Thus we could investigate the convergence of each of the pieces

The series
\[
\sum^{\infty}_{n=1} \frac{1}{3^{2n}}
\]

\wordChoice{\choice[correct]{converges}\choice{diverges}} since it is a geometric series with $r=\answer{ \frac{1}{9}}$. 

The series $\sum^{\infty}_{n=1} \frac{5^n}{3^{2n}}$  \wordChoice{\choice[correct]{converges}\choice{diverges}} 
since it is geometric with $r=\answer{\frac{5}{9}}$. 

Thus our original series
\[
\sum_{n=1}^{\infty}(1+5^n)3^{2-2n}
\]

\begin{multipleChoice}
\choice[correct]{converges}
\choice{diverges}
\end{multipleChoice}

\end{question}
\end{hint}







\begin{quote}
You should be able to give a detailed solution for each of your choices.  Such a solution should include:

\begin{itemize}
\item What test you chose.
\item Why you are allowed to use that test.
\item The computations required for the conclusions of the test.
\item An explanation why the series either converges or diverges by the test you chose.
\end{itemize}

\end{quote}


\end{exercise}
\end{document}
