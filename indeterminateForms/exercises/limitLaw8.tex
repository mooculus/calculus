\documentclass{ximera}

\newcommand{\RR}{\mathbb R}
\renewcommand{\d}{\,d}
\newcommand{\dd}[2][]{\frac{d #1}{d #2}}
\renewcommand{\l}{\ell}
\newcommand{\ddx}{\frac{d}{dx}}
\newcommand{\dfn}{\textbf}
\newcommand{\eval}[1]{\bigg[ #1 \bigg]}


\outcome{Understand what is meant by the form of a limit.}
\outcome{Calculate limits of the form zero over zero.}
\outcome{Identify determinate and indeterminate forms.}
\outcome{Distinguish between determinate and indeterminate forms.}

\author{Nela Lakos \and Kyle Parsons}

\begin{document}
\begin{exercise}

Let $g(x) = 2\left|x-1\right|$, and $h(x) = -(x-1)^2$.\\

For the following limits determine whether the form of the limit is determinate or indeterminate, determine the form of the limit and compute the value of the limit.  Possible answers include a number, $+\infty$, $-\infty$ and $DNE$.

\[
\lim_{x\to1}\frac{h(x)}{g(x)} = \answer{0}
\]


Choose all correct statements.
\begin{selectAll} 
\choice{The limit is of determinate form.}
\choice[correct]{The limit is of indeterminate form.}
\choice[correct]{The limit is of the form $\dfrac{0}{0}$.}
\choice{The limit is of the form $\dfrac{\#}{0}$.}
\end{selectAll}

\noindent\rule[0.5ex]{\linewidth}{0.2pt}
\[
\lim_{x\to1^{+}}\frac{g(x)}{h(x)} = \answer{-\infty}
\]


Choose all correct statements.
\begin{selectAll} 
\choice{The limit is of determinate form.}
\choice[correct]{The limit is of indeterminate form.}
\choice[correct]{The limit is of the form $\dfrac{0}{0}$.}
\choice{The limit is of the form $\dfrac{\#}{0}$.}
\end{selectAll}

\noindent\rule[0.5ex]{\linewidth}{0.2pt}

\[
\lim_{x\to1^{-}}\frac{g(x)}{h(x)} = \answer{-\infty}
\]


Choose all correct statements.
\begin{selectAll} 
\choice{The limit is of determinate form.}
\choice[correct]{The limit is of indeterminate form.}
\choice[correct]{The limit is of the form $\dfrac{0}{0}$.}
\choice{The limit is of the form $\dfrac{\#}{0}$.}
\end{selectAll}

\noindent\rule[0.5ex]{\linewidth}{0.2pt}
\[
\lim_{x\to1}\frac{g(x)}{h(x)} = \answer{-\infty}
\]


Choose all correct statements.
\begin{selectAll} 
\choice{The limit is of determinate form.}
\choice[correct]{The limit is of indeterminate form.}
\choice[correct]{The limit is of the form $\dfrac{0}{0}$.}
\choice{The limit is of the form $\dfrac{\#}{0}$.}
\end{selectAll}

\noindent\rule[0.5ex]{\linewidth}{0.2pt}


\[
\lim_{x\to1^-}\frac{g(x)-g(1)}{x-1} = \answer{-2}
\]



Choose all correct statements.
\begin{selectAll} 
\choice{The limit is of determinate form.}
\choice[correct]{The limit is of indeterminate form.}
\choice[correct]{The limit is of the form $\dfrac{0}{0}$.}
\choice{The limit is of the form $\dfrac{\#}{0}$.}
\end{selectAll}
\noindent\rule[0.5ex]{\linewidth}{0.2pt}

\[
\lim_{x\to1}\frac{g(x)-g(1)}{x-1} = \answer{2}
\]


Choose all correct statements.
\begin{selectAll} 
\choice{The limit is of determinate form.}
\choice[correct]{The limit is of indeterminate form.}
\choice[correct]{The limit is of the form $\dfrac{0}{0}$.}
\choice{The limit is of the form $\dfrac{\#}{0}$.}
\end{selectAll}
\noindent\rule[0.5ex]{\linewidth}{0.2pt}

\[
\lim_{x\to1}\frac{g(x)-g(1)}{x-1} = \answer{DNE}
\]



Choose all correct statements.
\begin{selectAll} 
\choice{The limit is of determinate form.}
\choice[correct]{The limit is of indeterminate form.}
\choice[correct]{The limit is of the form $\dfrac{0}{0}$.}
\choice{The limit is of the form $\dfrac{\#}{0}$.}
\end{selectAll}


\end{exercise}
\end{document}