\documentclass{ximera}

\outcome{Learn things.}

\newcommand{\RR}{\mathbb R}
\renewcommand{\d}{\,d}
\newcommand{\dd}[2][]{\frac{d #1}{d #2}}
\renewcommand{\l}{\ell}
\newcommand{\ddx}{\frac{d}{dx}}
\newcommand{\dfn}{\textbf}
\newcommand{\eval}[1]{\bigg[ #1 \bigg]}

\title{Limit Laws and Continuity}
\begin{document}
\begin{abstract}

ABSTRACT

\end{abstract}
\maketitle

Limit Laws

We learned previously that just estimating limits by plugging in nearby values does not always give the correct answer.  We need some theorems that tell us how to quickly find limits, at least in the most common cases.  In this section, we present a handful of tools to compute many limits without explicitly working with the definition of limit. Each of these could be proved directly as we did in the previous section. 

\begin{theorem}[Limit Laws]\index{limit laws}\label{theorem:limit-laws}
Suppose that $\lim_{x\to a}f(x)=L$, $\lim_{x\to a}g(x)=M$, $k$
is some constant, and $n$ is a positive integer.
\begin{itemize}
\item[\textbf{Limit of a Constant}] $\lim_{x\to a} k = k$.
\item[\textbf{Limit of x}] $\lim_{x\to a}x =a$.
\item[\textbf{Constant Multiple Law}] $\lim_{x\to a} kf(x) = k\lim_{x\to a}f(x)=kL$.
\item[\textbf{Sum/Difference Law}] $\lim_{x\to a} (f(x) \pm g(x)) = \lim_{x\to a}f(x) \pm \lim_{x\to a}g(x)=L \pm M$.  
\item[\textbf{Product Law}] $\lim_{x\to a} (f(x)g(x)) = \lim_{x\to a}f(x)\cdot\lim_{x\to a}g(x)=LM$. 
\item[\textbf{Quotient Law}] $\lim_{x\to a} \frac{f(x)}{g(x)} =
  \frac{\lim_{x\to a}f(x)}{\lim_{x\to a}g(x)}=\frac{L}{M}$, if $M\ne0$.
\item[\textbf{Power Law}] $\lim_{x\to a} f(x)^n = \left(\lim_{x\to a}f(x)\right)^n=L^n$.
\item[\textbf{Root Law}] $\lim_{x\to a} \sqrt[n]{f(x)} = \sqrt[n]{\lim_{x\to
    a}f(x)}=\sqrt[n]{L}$ provided if $n$ is even, then $f(x)\ge 0$
  near $a$.
\item[\textbf{Composition Law}] If $\lim_{x\to a}g(x)=M$ and
  $\lim_{x\to M}f(x) = f(M)$, then $\lim_{x\to a} f(g(x)) = f(M)$.
\end{itemize}
\label{thm:limit laws}
\end{theorem}


\begin{example}
Find  $\lim_{x\to1}(x-2)$
\end{example}
\begin{solution}
$\lim_{x\to1}(x-2)$ = $\lim_{x\to1}(x)-\lim_{x\to1}(2)$ \\
What limit law was used in this step? \answer[given]{Sum/Difference Law}\\
 $=\lim_{x\to1}(x)-2$ \\
What limit law was used in this step? \answer[given]{Limit of a Constant}\\
$=1-2$ \\
What limit law was used in this step? \answer[given]{Limit of x}\\
$=-1$
\end{solution}
%How do I make text answers?

This first example might seem trivial, but remember how we tricked ourselves in the previous section.  The limit laws allow us to know that as long as we follow them, we are actually getting the right answer for the limit.

\begin{example}
Compute $\lim_{x\to 1}{x^2-3x+5\over x-2}$. 
\end{example}
\begin{solution}
Using limit laws, 
\begin{align*}
\lim_{x\to 1}{x^2-3x+5\over x-2}&=
\dfrac{\lim_{x\to 1}(x^2-3x+5)}{\lim_{x\to1}(x-2)}  \text{\ \ (assuming $\lim_{x\to1}(x-2) \neq 0$)} \\
&=\frac{\lim_{x\to 1}(x^2)-\lim_{x\to1}(3x)+\lim_{x\to1}(5)}{\lim_{x\to1}(x)-\lim_{x\to1}(2)} \\
&=\dfrac{\left(\lim_{x\to 1}x\right)^2-3(\lim_{x\to1}x)+5}{\lim_{x\to1}(x)-2} \\
&=\dfrac{1^2-3\cdot1+5}{1-2} \\
&=\dfrac{1-3+5}{-1} = -3.
\end{align*}
\textit{On your own}: Go through each step in this example an determine which limit laws were used.  It is important to make sure you are only using real limit laws and not making up your own rules!
\end{solution}
%I think the way that the x -> a part is not underneath the limit in this solution is confusing/distracting to students.  Can this be fixed?
%Can we use color in the book.  For example,  \text{\ \ (assuming $\lim_{x\to1}(x-2) \neq 0$)}  should be a different color.





Continuity

Often, the $\displaystyle \lim_{x\to c}f(x)$ is equal to $f(c)$.  In those situations, finding a limit is the same things as plugging the limiting value into the function.  We call this property continuity.

\begin{definition}
  A function $f$ is \dfn{continuous at a point} $a$ if $\lim_{x\to a}
  f(x) = f(a)$.
\end{definition}
\index{continuous}

We can often use continuity to determine limit values.  For example, the limit laws "Limit of a Constant" and "Limit of $x$" can be restated as the functions $f(x)=k$ and  $f(x)=x$ are continuous at every real number $x=a$.

When checking for whether a function $f(x)$ is continuous at a point $a$, we must check three things:
\begin{enumerate}
\item $f(a)$ is defined.  That is, $a$ is in the domain of $f(x)$.
\item $\displaystyle \lim_{x\to a}f(x)$ exists.
\item $\lim_{x\to a} f(x) = f(a)$.
\end{enumerate}

The first two of these statements are implied by the third statement.

\begin{example}
Find the discontinuities (the values for $x$ where a function is not
continuous) for the function described below:
\begin{image}
\begin{tikzpicture}
	\begin{axis}[
            domain=0:10,
            ymax=5,
            ymin=0,
            samples=100,
            axis lines =middle, xlabel=$x$, ylabel=$y$,
            every axis y label/.style={at=(current axis.above origin),anchor=south},
            every axis x label/.style={at=(current axis.right of origin),anchor=west}
          ]
	  \addplot [very thick, penColor, smooth, domain=(4:10)] {3 + sin(deg(x*2))/(x-1)};
          \addplot [very thick, penColor, smooth, domain=(0:4)] {1};
          \addplot[color=penColor,fill=background,only marks,mark=*] coordinates{(4,3.30)};  %% open hole
          \addplot[color=penColor,fill=background,only marks,mark=*] coordinates{(6,2.893)};  %% open hole
          \addplot[color=penColor,fill=penColor,only marks,mark=*] coordinates{(4,1)};  %% closed hole
          \addplot[color=penColor,fill=penColor,only marks,mark=*] coordinates{(6,2)};  %% closed hole
        \end{axis}
\end{tikzpicture}
%% \caption{A plot of a function with discontinuities at $x=4$ and $x=6$.}
%% \label{plot:discontinuous-function}
\end{image}



From Figure~\ref{plot:discontinuous-function} we see that $\lim_{x\to 4} f(x)$ does not exist as
\[
\lim_{x\to 4-}f(x) = 1\qquad\text{and}\qquad \lim_{x\to 4+}f(x) \approx 3.5
\]
Hence $\lim_{x\to 4} f(x) \ne f(4)$, and so $f(x)$ is not
continuous at $x=4$.

We also see that $\lim_{x\to 6} f(x) \approx 3$ while $f(6) =
2$. Hence $\lim_{x\to 6} f(x) \ne f(6)$, and so $f(x)$ is not
continuous at $x=6$.
\end{example}

Building from the definition of \textit{continuous at a point}, we can
now define what it means for a function to be \textit{continuous} on
an interval.

\begin{definition}
  A function $f$ is \dfn{continuous on the interval $I$} if $\lim_{x\to a}
  f(x) = f(a)$ for all $a$ in $I$.
\end{definition}

We can now restate the limit laws "Limit of a Constant" and "Limit of $x$" as the functions $f(x)=k$ and  $f(x)=x$ are continuous for all real numbers.

Graphically, a function is continuous on an interval $I$ if you can draw the function on that interval without any breaks in the graph.  This is often referred to as being able to draw the graph "without picking up your pencil."  

Many of the functions we are most familiar with are continuous on their natural domains, that is, wherever they are defined.

\begin{theorem}[Continuity of Famous Functions]\index{continuity of famous functions}\label{theorem:continuity}
The following types of functions are continuous wherever they are defined:
\begin{itemize}
\item Polynomial Functions
\item Rational Functions
\item Roots
\item Exponential Functions
\item Logarithmic Functions 
\item Trigonometric Functions
\item Inverse Trigonometric Functions
\end{itemize}
\end{theorem}

Now, the limit laws "Limit of a Constant" and "Limit of $x$" as the functions $f(x)=k$ and  $f(x)=x$ are just special cases of continuity of Polynomials.

We can combine this theorem with the Limit Laws to find the limits of most functions.
\begin{example}
$\lim_{x \to 0} \sqrt{e^{x}+\cos(x)}$
\end{example}
\begin{solution}
\begin {align}
\lim_{x \to 0} \sqrt{e^{x}+\cos(x)} &= 
\sqrt{\lim_{x \to 0} (e^{x}+\cos(x))} \text{\ \ (by continuity of Roots)}\\
&= \sqrt{\lim_{x \to 0} (e^{x})+\lim_{x \to 0} (\cos(x))} \text{\ \ (by Sum/Difference Law)}\\
&= \sqrt{e^{\lim_{x \to 0} (x)}+\lim_{x \to 0} (\cos(x))} \text{\ \ (by continuity of Exponentials)}\\
&= \sqrt{e^{\lim_{x \to 0} (x)}+ \cos(\lim_{x \to 0}(x)))} \text{\ \ (by continuity of Trig)}\\
&= \sqrt{e^{0}+ \cos(0)}=\sqrt{1+1}=\sqrt{2} \text{\ \ (by continuity of Polynomials)}
\end{align}
\end{solution}
\end{document}