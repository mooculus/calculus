\documentclass{ximera}

\newcommand{\RR}{\mathbb R}
\renewcommand{\d}{\,d}
\newcommand{\dd}[2][]{\frac{d #1}{d #2}}
\renewcommand{\l}{\ell}
\newcommand{\ddx}{\frac{d}{dx}}
\newcommand{\dfn}{\textbf}
\newcommand{\eval}[1]{\bigg[ #1 \bigg]}


\title[Dig-In:]{Mirrored surfaces}

\begin{document}
\begin{abstract}
  We think about how to reflect light of off surfaces. 
\end{abstract}
\maketitle

Suppose you have a function $F:\R^2\to\R$ via
\[
F(x,y) = x^2 + y^2
\]
In this case, the surface is called a \dfn{paraboloid}.  and we would
like to understand how light would reflect off such an object, if we
were to build it as a physical object.  Suppose $\vec{\ell}(t)$ is an
arbitrary (parametrized!) straight line, given by
\[
\vec{\ell}(t) = \vec{\ell}_0 + t \vec{v}
\]
and suppose the image of $\vec{\ell}(t)$ intersects the image of
$\vec{F}$ at a point $(a,b,c)$.

\begin{sageCell}
# The purple surface
f(x,y) = x^2 + y^2

# The red light beam
t = var('t')
ell0 = vector([0.85,0.5,1.5])
v = vector([-0.2,-0.1,-1])

plot3d( f, (x,-1,1), (y,-1,1) ) + parametric_plot3d( ell0 + t * v, (t,-0.5,2), color="red", thickness=0.1)
\end{sageCell}


When we zoom in close enough to the point $(a,b,c)$ where the light
ray hits the surface, we imagine that the mirrored surface is a plane.

\begin{question}
  What is the tangent plane to $F$ at the point $(a,b,c)$?
  \begin{prompt}
    \[
    z = \answer{2a(x-a)+2b(y-b)}
    \]
  \end{prompt}
\end{question}
When the light ray hits the surface, it is heading in the direction
\begin{multipleChoice}
  \choice{$\pp[x]{\vec{F}}$}
  \choice{$\pp[y]{\vec{F}}$}
  \choice[correct]{$\vec{v}$}
\end{multipleChoice}
This means that the reflected light will be heading in the direction
%% \begin{multipleChoice}
%%   \choice{$\vec{v} + 2 \proj_{\vec{v}}(\vec{n})}
%%   \choice{$\vec{v} - 2 \proj_{\vec{v}}(\vec{n})}
%%   \choice{$\vec{v} + 2 \proj_{\pp{\vec{F}}{x} \cross \pp{\vec{F}}{y}}(\vec{v})$}
%%   \choice[correct]{$\vec{v} - 2 \proj_{\pp{x}{\vec{F}} \cross \pp{y}{\vec{F}}}( \vec{v})$}
%% \end{multipleChoice}
Such formulas justify the time we spent thinking deeply about $\proj$.  And now we can plot all this together.
\begin{sageCell}
# The purple surface
a = var('a'); b = var('b')
f = vector([a,b,a^2 + b^2])

# The red light beam
t = var('t') ; ell0 = vector([0.85,0.5,1.5]) ; v = vector([0.1,-0.2,-1])

# The intersection of the surface and the light beam
solutions = solve( [f[i] == (ell0 + t*v)[i] for i in range(3)], [a,b,t], solution_dict=True)
solution = [s for s in solutions if t.subs(s) > 0][0]
xyz = f.subs(solution)

# The reflected vector
proj = lambda v, w: (v.dot_product(w))/(w.dot_product(w)) * w
n = derivative(f,a).cross_product( derivative(f,b) )
n = n.subs(solution)
rv = v - 2 * proj(v,n)

# Plot the surface, the incoming beam, and the outgoing beam
parametric_plot3d( f, (a,-1,1), (b,-1,1) ) + parametric_plot3d( ell0 + t * v, (t,-0.5,t.subs(solution)), color="red", thickness=0.1) + parametric_plot3d( xyz + t * rv, (t,0,2), color="red", thickness=0.1)
\end{sageCell}



\begin{example}[Vertical light]
  By playing around with the vector $\vec{v}$, i.e., the direction of
  the incoming light, we can discover some significant, qualitative
  features of reflections in paraboloids.

  Let $\vec{F}(x,y) = (x,y,x^2 + y^2)$, so that
  \begin{align*}
    \pp{\vec{F}}{x} &= \vector{ 1, 0, \answer{2x}} \\
    \pp{\vec{F}}{y} &= \vector{0, 1, \answer{2y}} \\
    \vec{n} &= \pp{\vec{F}}{x} \cross \pp{\vec{F}}{y} &= \vector{-2x, -2y, \answer{1}}. \\
  \end{align*}

  We want to focus in on the case of ``vertical light'' which arrives directly from above.  One way to 
  restrict to this situation would be to assume that both $\vec{v} \dotp \langle 1, 0, 0 \rangle = 0$ and
  $\vec{v} \dotp \vector{ 0, 1, 0 } = 0$, but to simplify the situation further, simply suppose
  that $\vec{v} = \vector{ 0, 0, -1 }$.  Then
  \begin{align*}
    \vec{v} - 2 \proj_{\vec{n}} (\vec{v}) 
    &= \vec{v} - 2 \frac{\vec{v} \dotp \vec{n}}{\vec{n} \dotp \vec{n}} \vec{n} \\
    &= \vec{v} - 2 \frac{\answer{-1}}{\vec{n} \dotp \vec{n}} \vec{n} \\
  \end{align*}
  This reflected vector simplies further to
  \[
    \frac{-4}{1 + 4x^2 + 4y^2} \vector{\answer{x}, \answer{y}, \answer{x^2 + y^2 - 1/4}}.
  \]
  The significance of this is that, starting at the point
  $(x,y,x^2 + y^2)$ and moving in the direction of this vector, we pass through the point 
  \[
    \left( 0, 0, \answer[given]{1/4} \right),
  \]
  meaning that \textit{every} vertical beam of light reflects off the
  paraboloid and is focused on common point!  This fact is what makes
  solar ovens and satellite dishes possible.
  
\end{example}

\end{document}
