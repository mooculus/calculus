\documentclass{ximera}

\newcommand{\RR}{\mathbb R}
\renewcommand{\d}{\,d}
\newcommand{\dd}[2][]{\frac{d #1}{d #2}}
\renewcommand{\l}{\ell}
\newcommand{\ddx}{\frac{d}{dx}}
\newcommand{\dfn}{\textbf}
\newcommand{\eval}[1]{\bigg[ #1 \bigg]}


\outcome{Understand the relationship between position, velocity and acceleration.}


\title[Break-Ground:]{What could it represent?}

\begin{document}
\begin{abstract}
Two young mathematicians discuss whether integrals are defined properly.
\end{abstract}
\maketitle


Check out this dialogue between two calculus students (based on a true
story):

\begin{dialogue}
\item[Devyn] Riley, I like integrals.
\item[Riley] I feel fancy when I make an integral sign.
\item[Devyn] I know! An integral computes the signed area between a curve
  $y=f(x)$ and the $x$-axis. But why \textit{signed} area? Maybe we should
  just compute plain old area.
\item[Riley] Makes sense to me!
\item[Deyvn] Unless\dots maybe there are other applications where
  ``signed'' area makes more sense.
\end{dialogue}

One really great way to think about integrals is that they
``accumulate rates.''

\begin{problem}
  Write down as many examples of ``rates'' and ``accumulated rates'' as you can. For example:
  \begin{quote}
    $5$ miles per hour is a rate, and $5$ miles is then an accumulated rate. 
  \end{quote}
  \begin{freeResponse}
  \end{freeResponse}
\end{problem}

%% \begin{xarmaBoost}
%%   Write down at least \textbf{five} questions for this lecture. After
%%   you have your questions, label them as ``Level 1,'' ``Level 2,'' or
%%   ``Level 3'' where:
%% \begin{description}
%% \item[Level 1] Means you know the answer, or know exactly how to do
%%   this problem.
%% \item[Level 2] Means you think you know how to do the problem.
%% \item[Level 3] Means you have no idea how to do the problem.
%% \end{description}
%% \begin{freeResponse}
%% \end{freeResponse}
%% \end{xarmaBoost}



\end{document}
