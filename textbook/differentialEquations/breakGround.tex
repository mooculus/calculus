\documentclass{ximera}

\newcommand{\RR}{\mathbb R}
\renewcommand{\d}{\,d}
\newcommand{\dd}[2][]{\frac{d #1}{d #2}}
\renewcommand{\l}{\ell}
\newcommand{\ddx}{\frac{d}{dx}}
\newcommand{\dfn}{\textbf}
\newcommand{\eval}[1]{\bigg[ #1 \bigg]}


\outcome{Define a differential equation.}

\title[Break-Ground:]{Modeling the spread of infectious diseases}

\begin{document}
\begin{abstract}
  A dialogue where two young mathematicians discuss differential equations.
\end{abstract}
\maketitle

% Misconception: difference between equation versus function

% Find solutions to functional equations

% I'm thinking of a function whose derivative is three times that function.

% It must be zero!

% Well that wasn't what I was thinking of.

% e^(3x)

% Include more testimonials from people
Check out this dialogue between two calculus students (based on a true
story):

\begin{dialogue}
\item[Devyn] Riley, check out this book, it has some cool applications
  of calculus.
\begin{quote}
  The following differential equations can be used to model the spread
  of an infectious disease:
  \[
  \mathrm{infect}'(t) = 0.000004 \cdot \mathrm{suscept}(t) \cdot \mathrm{infect}(t) - 0.047\cdot \mathrm{infect}(t),\\
  \]
  Here $\mathrm{infect}(t)$ is the number of people infected by the
  disease on day $t$, and $\mathrm{suscept}(t)$ is the number of
  who are susceptible to the disease on day $t$.
\end{quote}
\item[Riley] Woah. That's like a formula that not only has derivatives
  in it, but is itself a formula for a derivative. Wow. Much calculus. 
\item[Devyn] I wonder how you solve equations like this.
\item[Riley] I wonder if we can sometimes just use facts about the
  derivative to give us an approximation.
\end{dialogue}

\begin{problem}
  Suppose at day $13$ ($t=13$) we know that $\mathrm{suscept}(13) =
  5028$ and $\mathrm{infect}(13) = 4012$. What is $\mathrm{infect}'(13)$?
  \begin{prompt}
    \[
    \mathrm{infect}'(13) = \answer{-107.875}
    \]
  \end{prompt}
\end{problem}

\begin{problem}
  Do your best to explain why the formula above is reasonable.
  \begin{hint}
    Don't worry, just do your best.
  \end{hint}
  \begin{freeResponse}
    Answers will vary.
  \end{freeResponse}
\end{problem}

%% \begin{enumerate}
%% \item Explain why the two formulas above are reasonable, though you
%%   don't need to worry about the precise values of the coefficients.
%% \item Discuss the outlook on day $13$ if you know
%%   $\mathrm{suscept}(13) = 5028$ and $\mathrm{infect}(13) = 4012$. In
%%   particular, you should talk about both $\mathrm{infect}'(x)$ and
%%   $\mathrm{suscept}'(x)$.
%% \item Compute $\mathrm{infect}''(x)$ on day 13. What does this say
%%   about the outlook?
%% \end{enumerate}




%% \begin{xarmaBoost}
%%   Write down at least \textbf{five} questions for this lecture. After
%%   you have your questions, label them as ``Level 1,'' ``Level 2,'' or
%%   ``Level 3'' where:
%% \begin{description}
%% \item[Level 1] Means you know the answer, or know exactly how to do
%%   this problem.
%% \item[Level 2] Means you think you know how to do the problem.
%% \item[Level 3] Means you have no idea how to do the problem.
%% \end{description}
%% \begin{freeResponse}
%% \end{freeResponse}
%% \end{xarmaBoost}


\end{document}
