\documentclass{ximera}

\newcommand{\RR}{\mathbb R}
\renewcommand{\d}{\,d}
\newcommand{\dd}[2][]{\frac{d #1}{d #2}}
\renewcommand{\l}{\ell}
\newcommand{\ddx}{\frac{d}{dx}}
\newcommand{\dfn}{\textbf}
\newcommand{\eval}[1]{\bigg[ #1 \bigg]}


\author{Bart Snapp}

\title[Dig-In:]{Grad, Curl, Div}

\begin{document}
\begin{abstract}
  We explore the relationship between the gradient, the curl, and the
  divergence of a vector field.
\end{abstract}
\maketitle

At this point in our study, we have many fundamental theorems. Let's
try to use them together, and see what we can find.

\section{A two-dimensional dream}

So far we have two fundamental theorems of calculus for functions of
two variables.
\begin{itemize}
\item The fundamental theorem of line integrals:
  \[
  \int_C \grad F \dotp\d\vec{p} = F(\vec{b}) -F(\vec{a})
  \]
\item Green's theorem:
  \[
  \iint_R \curl \vec{F} \d A = \oint_{\partial R} \vec{F} \dotp \d \vec{p}
  \]
\end{itemize}
Wouldn't it be cool if we could \textit{combine} these theorems to
make something like:
\[
\iint_R \curl \grad F \d A = \oint_{\partial R} \grad F \dotp \d \vec{p} = F(\vec{b}) - F(\vec{a})
\]
A double integral, equaling a single integral, equaling a difference!
It's an amazing idea, and while it is true, we shouldn't get too
excited. In fact each expression above is equal to zero.

Upshot: $\curl\grad F = 0$ for all functions $\vec{F}:\R^2\to\R^2$
with continuous second derivatives.


\section{A three-dimensional dream}

Working in a similar way to how we worked above, let us recall the fundamental theorems of calculus for functions of three variables.

\begin{itemize}
\item The fundamental theorem of line integrals:
  \[
  \int_C \grad F \dotp\d\vec{p} = F(\vec{b}) - F(\vec{a}) 
  \]
\item Stokes' theorem:
  \[
  \iint_R \curl\vec{F} \d S = \oint_{\partial R} \vec{F} \dotp \d \vec{p}
  \]
\item The divergence theorem:
  \[
  \iiint_R \divergence\vec{F} \d V = \oiint_{\partial R} \vec{F}\dotp\uvec{n} \d S
  \]
\end{itemize}
Putting Stokes' theorem and the divergence theorem together, we find the beautiful expression:
\[
\iiint_R \divergence\left(\curl\vec{F}\right)\d V = \oiint_{\partial R} \left(\curl \vec{F}\right)\dotp \uvec{n} \d S = \oint_{\partial\partial R} \vec{F}\dotp \d \vec{p}
\]
Again, what a fantastic idea, a triple integral, equaling a double
integral, equaling a single integral!


\[
\oint_{\partial\partial R} \vec{F}\dotp \d \vec{p} = 0
\]


\[
\iiint_R \divergence\left(\curl\vec{F}\right)\d V = \oiint_{\partial R} \left(\curl \vec{F}\right)\dotp \uvec{n} \d S = \oint_{\partial\partial R} \vec{F}\dotp \d \vec{p}
\]
This tells us that the circulation along \textbf{any} closed surface is zero!

\[
\divergence\left(\curl\vec{F}\right) =  0
\]

%% \begin{image}
%%   \begin{tikzcd}[ampersand replacement=\&, row sep=.1em]
%%     C^{\infty}(\R^2,\R) \arrow[r,"\grad"] \& C^{\infty}(\R^2,\R^2) \arrow[r,"\curl"] \& C^{\infty}(\R^2,\R)\\
%%     F \arrow[r,mapsto] \& \grad F  \arrow[r,mapsto] \& 0
%%   \end{tikzcd}
%% \end{image}

%% \begin{image}
%%   \begin{tikzcd}[ampersand replacement=\&, row sep=.1em]
%%     C^{\infty}(\R^3,\R) \arrow[r,"\grad"] \& C^{\infty}(\R^3,\R^3) \arrow[r,"\curl"] \& C^{\infty}(\R^3,\R^3) \arrow[r,"\divergence"] \& C^{\infty}(\R^3,\R)\\
%%     F \arrow[r,mapsto] \& \grad F  \arrow[r,mapsto] \& \vec{0}   \&\\
%%                        \& \vec{F}  \arrow[r,mapsto] \& \curl\vec{F} \arrow[r,mapsto]   \& 0
%%   \end{tikzcd}
%% \end{image}
  

\end{document}
