\documentclass{ximera}

\newcommand{\RR}{\mathbb R}
\renewcommand{\d}{\,d}
\newcommand{\dd}[2][]{\frac{d #1}{d #2}}
\renewcommand{\l}{\ell}
\newcommand{\ddx}{\frac{d}{dx}}
\newcommand{\dfn}{\textbf}
\newcommand{\eval}[1]{\bigg[ #1 \bigg]}


%\outcome{Given a velocity function, calculate displacement and distance traveled.}
%\outcome{Given a velocity function, find the position function.}
%\outcome{Given an acceleration function, find the velocity function.}
%\outcome{Understand the difference between displacement and distance traveled.}
%\outcome{Understand the relationship between position, velocity and acceleration.}

\author{Nela Lakos \and Kyle Parsons}

\begin{document}
\begin{exercise}

The graph of $f$ on the interval $[0,5]$ is given below.

\begin{image}
  \begin{tikzpicture}
    \begin{axis}[
        xmin=-0.3,xmax=5.3,ymin=-0.3,ymax=2.3,
        clip=true,
        unit vector ratio*=1 1 1,
        axis lines=center,
        grid = major,
        ytick={-1,0,...,36},
        xtick={0,1,...,10},
        xlabel=$x$, ylabel=$y$,
        every axis y label/.style={at=(current axis.above origin),anchor=south},
        every axis x label/.style={at=(current axis.right of origin),anchor=west},
      ]
      \addplot[ultra thick,penColor,domain=0:1,samples=2] {x};
      \addplot[ultra thick,penColor,domain=1:2,samples=2] {1};
      \addplot[ultra thick,penColor,domain=2:3,samples=2] {x-1};
      \addplot[ultra thick,penColor,domain=3:4,samples=2] {2};
      \addplot[ultra thick,penColor,domain=4:5,samples=2] {6-x};    
        
      \node at (axis cs:1,1.5) {$y=f(x)$};
      \end{axis}`
  \end{tikzpicture}
\end{image}

Using geometry, we can evaluate $\int_0^5 f(x) \d x$ to find it equals  $\answer{6.5}$.

Again, using geometry, we can evaluate $\int_1^4 f(x) \d x$ to find it equals  $\answer{4.5}$.

Finally, we can use geometry to find that  $\int_0^2 4f(x) \d x - \int_1^3 5f(x) \d x = \answer{-6.5}$.

The average value of $f$ on $[0,5]$ is 
\[
\bar{f} = \answer{1.3}.
\]

The points $c$ in $[0,5]$ where $f(c) = \bar{f}$ are 
\[
c_1=\answer{2.3}
\]
 and
  \[
  c_2=\answer{4.7},
  \]
   (answer from left to right).
   
\begin{hint}

Since $1<\bar{f}<2$, the point $c$ will be found only at the intervals where $1<f(x)<2$.

By inspecting the graph of $f$, we can see that the only such intervals  are $(2,3)$ and $(4,5)$.


From the graph we can conclude that

 $f(x)=x-1$, for  $2<x<3$; and that
 
 
 $f(x)=6-x$, for $4<x<5$.
 

Now we can solve the equation
\[
\bar{f} =f(c).
\]
\end{hint}
\end{exercise}
\end{document}