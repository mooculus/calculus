\documentclass{ximera}

\author{Bart Snapp}

\newcommand{\RR}{\mathbb R}
\renewcommand{\d}{\,d}
\newcommand{\dd}[2][]{\frac{d #1}{d #2}}
\renewcommand{\l}{\ell}
\newcommand{\ddx}{\frac{d}{dx}}
\newcommand{\dfn}{\textbf}
\newcommand{\eval}[1]{\bigg[ #1 \bigg]}


\outcome{Use Green's Theorem as a planimeter.}

\title[Dig-In:]{Green's theorem as a planimeter}

\begin{document}
\begin{abstract}
  A planimeter computes the area of a region by tracing the boundary.
\end{abstract}
\maketitle

Green's theorem may seem rather abstract, but as we will see, it is a
fantastic tool for computing the areas of arbitrary bounded
regions. In particular, Green's Theorem is a theoretical
\link[planimeter]{http://en.wikipedia.org/wiki/Planimeter}. A
\dfn{planimeter} is a ``device'' used for measuring the area of a
region. Ideally, one would ``trace'' the border of a region, and the
planimeter would tell you the area of the region.


\section{How is Green's theorem a planimeter?}

Recall Green's Theorem:
\begin{theorem}[Green's Theorem]\index{Green's Theorem}
  If the components of $\vec{F}:\R^2\to\R^2$ have continuous partial
  derivatives and $C$ is a boundary of a closed region $R$ and
  $\vec{p}(t)$ parameterizes $C$ in a counterclockwise direction with
  the interior on the left, then
  \[
  \iint_R \curl\vec{F}\d A = \oint_C \vec{F}\dotp\d\vec{p} 
  \]
\end{theorem}

Given a vector field $\vec{F}:\R^2\to\R^2$, if $\curl\vec{F} = 1$,
then the left-hand side of the conclusion of Green's Theorem gives the
area of the region $R$:
\[
\iint_R \curl\vec{F}\d A = \iint_R \d A
\]

So now the question becomes, which vector fields have $\curl\vec{F} =
1$?

Here are three basic candidates:
\begin{itemize}
\item $\vec{F}(x,y) = \vector{0,x}$
\item $\vec{F}(x,y) = \vector{-y,0}$
\item $\vec{F}(x,y) = \vector{-y/2,x/2}$
\end{itemize}
\begin{question}
  The key idea that connects the three vector fields above is:
  \begin{selectAll}
    \choice[correct]{Their curl is $1$.}
    \choice{They are conservative fields.}
    \choice{They are gradient fields.}
    \choice[correct]{When used in combination with Green's Theorem, they help compute area.}
  \end{selectAll}
\end{question}

Once we have a vector field whose curl is $1$, we may then apply
Green's Theorem to use a line integral to compute the area.

\begin{warning}
  You must parameterize $C$ with $\vec{p}(t)$ for $a\le t\le b$ such that:
  \begin{itemize}
    \item $C$ is drawn in a counterclockwise direction.
    \item $C$ is drawn exactly once.
    \item The interior of $R$ is to the left of the direction of
      $\vec{p}'(t)$.
  \end{itemize}
\end{warning}




\section{Computing areas with Green's Theorem}

Now let's do some examples.

\begin{example}
  Compute the area of the trapezoid below using Green's Theorem.
  \begin{image}
    \begin{tikzpicture}
      \begin{axis}%
        [
	  ymin=-.5,ymax=2.5,
	  xmin=-.5,xmax=4.5,
          axis lines =middle, xlabel=$x$, ylabel=$y$,
          every axis y label/.style={at=(current axis.above origin),anchor=south},
          every axis x label/.style={at=(current axis.right of origin),anchor=west},
          grid=both,
          grid style={dashed, gridColor},
          % xtick={-2,...,4},
          % ytick={-3,...,3},
	]
        \addplot[penColor,ultra thick] coordinates{
            (0,0) (1,2) (3,2) (4,0) (0,0) (1,2)
        };
        
      \end{axis}
    \end{tikzpicture}
  \end{image}
  \begin{explanation}
    In this case, set $\vec{F}(x,y) = \vector{0,x}$. Since $\curl\vec{F} =
    1$, Green's Theorem says:
    \[
    \iint_R \d A  = \oint_C \vector{0,x}\dotp\d\vec{p}
    \]
    We need to parameterize our paths in a counterclockwise
    direction. We'll break it into four line segments each parameterized
    as $t$ runs from $0$ to $1$:
    \begin{image}
      \begin{tikzpicture}
        \begin{axis}%
          [
	  ymin=-.5,ymax=2.5,
	  xmin=-.5,xmax=4.5,
          axis lines =middle, xlabel=$x$, ylabel=$y$,
          every axis y label/.style={at=(current axis.above origin),anchor=south},
          every axis x label/.style={at=(current axis.right of origin),anchor=west},
          grid=both,
          grid style={dashed, gridColor},
         % xtick={-2,...,4},
         % ytick={-3,...,3},
	]
        \addplot[penColor,ultra thick] coordinates{
            (0,0) (1,2) 
        };
        \addplot[penColor,->,ultra thick] coordinates{
            (1,2) (.5,1) 
        };
        
        \addplot[penColor2,ultra thick] coordinates{
            (1,2) (3,2) 
        };
        \addplot[penColor2,ultra thick,->] coordinates{
            (3,2) (2,2) 
        };
        
        \addplot[penColor4,ultra thick] coordinates{
            (3,2) (4,0) 
        };
        \addplot[penColor4,ultra thick,->] coordinates{
            (4,0) (3.5,1) 
        };
        
        \addplot[penColor5,ultra thick] coordinates{
            (4,0) (0,0) 
        };
        \addplot[penColor5,ultra thick,->] coordinates{
            (0,0) (2,0) 
        };
        
        \node[above,penColor5] at (axis cs: 2,0) {$\vecl_1$};
        \node[penColor4] at (axis cs: 3.2,1) {$\vecl_2$};
        \node[below,penColor2] at (axis cs: 2,2) {$\vecl_3$};
        \node[penColor] at (axis cs: .8,1) {$\vecl_4$};

        \end{axis}
      \end{tikzpicture}    
    \end{image}
    where:
    \begin{align*}
      \vecl_1(t) &= \vector{4t,\answer[given]{0}}\\
      \vecl_2(t) &= \vector{\answer[given]{4-t},2t}\\
      \vecl_3(t) &= \vector{3-2t,\answer[given]{2}}\\
      \vecl_4(t) &= \vector{\answer[given]{1-t},2-2t}
    \end{align*}
    and each draws the line as $t$ runs from $0$ to $1$.  Write:
    \begin{align*}
      \oint_C \vec{F}\dotp\d\vec{p} = \int_0^1 &\vec{F}(\vecl_1(t))\dotp \vecl_1'(t) \d t
    + \int_0^1 \vec{F}(\vecl_2(t))\dotp \vecl_2'(t) \d t\\
    &+ \int_0^1 \vec{F}(\vecl_3(t))\dotp \vecl_3'(t) \d t
    + \int_0^1 \vec{F}(\vecl_4(t))\dotp \vecl_4'(t) \d t
    \end{align*}
    For each of the integrands above, say $\vecl(t) =
    \vector{x(t),y(t)}$, we will write
    \[
    \vec{F}(\vecl(t))\dotp \vecl'(t) = \vector{0,x(t)} \dotp \vector{x'(t),y'(t)}
    \]
    and combine them into a single integral. Write with me
    \begin{align*}
      \oint_C \vec{F}\dotp\d\vec{p} &= \int_0^1 \left(\answer[given]{2(4-t) -2(1-t)} \right)\d t\\
      &= \eval{\answer[given]{6t}}_0^1\\
      &=\answer[given]{6}. 
    \end{align*}
    So the this is the area of the trapezoid.
  \end{explanation}
\end{example}

\begin{example}
  Compute the area of the ellipse
  \[
  \frac{x^2}{a^2} + \frac{y^2}{b^2} = 1
  \]
  using Green's Theorem.
  \begin{explanation}
    To start, we'll set $\vec{F}(x,y) = \vector{-y/2,x/2}$. Since
    $\curl\vec{F} = 1$, Green's Theorem says:
    \[
    \iint_R \d A  = \oint_C \vector{-y/2,x/2}\dotp\d\vec{p}
    \]
    We can parameterize the boundary of the ellipse with
    \begin{align*}
      x(t) &= a \cos(t)\\
      y(t) &= b \sin(t)
    \end{align*}
    for $0\le t<2\pi$. Write with me
    \begin{align*}
      \iint_R \d A &= \oint_C \vector{-y/2,x/2}\dotp\vector{x'(t),y'(t)}\d t\\
      &= \int_0^{2\pi} \vector{\answer[given]{-b\sin(t)/2},\answer[given]{a\cos(t)/2}}\dotp\vector{\answer[given]{-a\sin(t)},\answer[given]{b\cos(t)}}\d t\\
      &= \int_0^{2\pi} (1/2)\left(ab\sin^2(t) + ab\cos^2(t)\right) \d t\\
      &= \int_0^{2\pi} (1/2)ab \d t\\
      &= \answer[given]{\pi ab}. 
    \end{align*}
    Done. Green's Theorem for the win!
  \end{explanation}
\end{example}

Finally, what do you do if you have a very strangely shaped curve? You approximate it with a polygonal curve. Check out next example.

\begin{example}
  Compute the area of the polygonal region below using Green's Theorem.
  \begin{image}
    \begin{tikzpicture}
      \begin{axis}%
        [
	  ymin=-3.2,ymax=3.2,
	  xmin=-3.2,xmax=4.2,
          axis lines =middle, xlabel=$x$, ylabel=$y$,
          every axis y label/.style={at=(current axis.above origin),anchor=south},
          every axis x label/.style={at=(current axis.right of origin),anchor=west},
          grid=both,
          grid style={dashed, gridColor},
          xtick={-3,...,4},
          ytick={-3,...,3},
	]
        \addplot[penColor,ultra thick] coordinates{
          (1,-1) (3,1) (0,1) (-1,3) (-2,-1) (1,-3)
          (1,-1) (3,1)
        };
        
      \end{axis}
    \end{tikzpicture}
  \end{image}
  \begin{explanation}
    In this case, set $\vec{F}(x,y) = \vector{0,x}$. Now we'll break this curve
    into six line segments that draw this curve in a counterclockwise
    fashion as a parameter $t$ runs from $0$ to $1$.
    \begin{image}
    \begin{tikzpicture}
      \begin{axis}%
        [
	  ymin=-3.2,ymax=3.2,
	  xmin=-3.2,xmax=4.2,
          axis lines =middle, xlabel=$x$, ylabel=$y$,
          every axis y label/.style={at=(current axis.above origin),anchor=south},
          every axis x label/.style={at=(current axis.right of origin),anchor=west},
          grid=both,
          grid style={dashed, gridColor},
          xtick={-3,...,4},
          ytick={-3,...,3},
	]
        \addplot[penColor,ultra thick] coordinates{
          (1,-1) (3,1) 
        };
        \addplot[penColor,ultra thick,->] coordinates{
          (1,-1) (2,0) 
        };
        
        \addplot[penColor2,ultra thick] coordinates{
          (3,1) (0,1) 
        };
        \addplot[penColor2,ultra thick,->] coordinates{
          (3,1) (1.5,1) 
        };
        
        \addplot[penColor3,ultra thick] coordinates{
          (0,1) (-1,3) 
        };
        \addplot[penColor3,ultra thick,->] coordinates{
          (0,1) (-.5,2) 
        };
        
        \addplot[penColor4,ultra thick] coordinates{
          (-1,3) (-2,-1) 
        };
        \addplot[penColor4,ultra thick,->] coordinates{
          (-1,3) (-1.5,1) 
        };
        
        \addplot[penColor5,ultra thick] coordinates{
          (-2,-1) (1,-3)
        };
        \addplot[penColor5,ultra thick,->] coordinates{
          (-2,-1) (-.5,-2)
        };
        
        \addplot[yellow!70!black,ultra thick] coordinates{
          (1,-3) (1,-1)
        };
        \addplot[yellow!70!black,ultra thick,->] coordinates{
          (1,-3) (1,-2)
        };
        \node[above left,penColor] at (axis cs: 2,0) {$\vecl_1$};
        \node[penColor2,above] at (axis cs: 1.5,1) {$\vecl_2$};
        \node[below left,penColor3] at (axis cs: -.5,2) {$\vecl_3$};
        \node[left,penColor4] at (axis cs: -1.5,1) {$\vecl_4$};
        \node[below,penColor5] at (axis cs: -.5,-2) {$\vecl_5$};
        \node[right,penColor6] at (axis cs: 1,-2) {$\vecl_6$};       
        
      \end{axis}
    \end{tikzpicture}
    \end{image}
    where:
    \begin{align*}
      \vecl_1(t) &= \vector{1+2t,\answer[given]{-1+2t}}\\
      \vecl_2(t) &= \vector{\answer[given]{3-3t},1}\\
      \vecl_3(t) &= \vector{-t,\answer[given]{1+2t}}\\
      \vecl_4(t) &= \vector{-1-t,\answer[given]{3-4t}}\\
      \vecl_5(t) &= \vector{\answer[given]{-2+3t},-1-2t}\\
      \vecl_6(t) &= \vector{1,\answer[given]{-3+2t}}
    \end{align*}
    and each draws the line as $t$ runs from $0$ to $1$.  Write:
    \begin{align*}
      \oint_C \vec{F}\dotp\d\vec{p} =
      &\int_0^1 \vec{F}(\vecl_1(t))\dotp \vecl_1'(t) \d t\\
      &+ \int_0^1 \vec{F}(\vecl_2(t))\dotp \vecl_2'(t) \d t\\
      &+ \int_0^1 \vec{F}(\vecl_3(t))\dotp \vecl_3'(t) \d t\\
      &+ \int_0^1 \vec{F}(\vecl_4(t))\dotp \vecl_4'(t) \d t\\
      &+ \int_0^1 \vec{F}(\vecl_5(t))\dotp \vecl_5'(t) \d t\\
      &+ \int_0^1 \vec{F}(\vecl_6(t))\dotp \vecl_6'(t) \d t
    \end{align*}
    For each of the integrands above, say $\vecl(t) =
    \vector{x(t),y(t)}$, we will write
    \[
    \vec{F}(\vecl(t))\dotp \vecl'(t) = \vector{0,x(t)} \dotp \vector{x'(t),y'(t)}
    \]
    and combine them into a single integral. Write with me
    \begin{align*}
      \oint_C \vec{F}\dotp\d\vec{p} &= \int_0^1 \left(\answer[given]{12} \right)\d t\\
      &= \eval{\answer[given]{12t}}_0^1\\
      &=\answer[given]{12}. 
    \end{align*}
    So the this is the area of the region.
  \end{explanation}
\end{example}

Green's Theorem gives a fairly easy method for computing any the area
of any polygonal region. Any region with a ``smooth'' border can be approximated by a polygonal region. The upshot? Green's Theorem is a powerful tool for computing area.

For some interesting extra reading check out:
\begin{itemize}
\item \link[\textit{The Surveyor's Area Formula}, B.\ Braden, College Math Journal, September
  1986]{http://www.jstor.org/stable/2686282}
\end{itemize}

\end{document}
