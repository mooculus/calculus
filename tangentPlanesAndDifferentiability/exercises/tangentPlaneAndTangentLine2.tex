\documentclass{ximera}

\newcommand{\RR}{\mathbb R}
\renewcommand{\d}{\,d}
\newcommand{\dd}[2][]{\frac{d #1}{d #2}}
\renewcommand{\l}{\ell}
\newcommand{\ddx}{\frac{d}{dx}}
\newcommand{\dfn}{\textbf}
\newcommand{\eval}[1]{\bigg[ #1 \bigg]}



\author{Jim Talamo}

\outcome{Find a tangent plane to the graph of a function.}

\begin{document}

\begin{exercise}
 
Find the equation of the tangent plane at the point on the surface $z=e^{x^2+3y-4}$ where $x=-1$ and $y=1$.

\[
\answer{-2} \cdot \left(x-\answer{ -1}\right) + \answer{ 3 } \cdot \left(y-\answer{ 1 }\right) + \big(-1\big) \cdot \left(z-\answer{1}\right) =0
\]
Writing this in the form $ax+by+cz=d$ gives

\[
\answer{-2} \cdot x + \answer{3} \cdot y -z = \answer{4}.
\]

\begin{exercise}
The point $(-1,1)$ lies on the circle $x^2+y^2 = \answer{2}$, so we find the curve on the surface associated to this curve in the domain.  

Let's use sines and cosines to parameterize the circle.

\begin{align*}
x(t) &= \answer{\sqrt{2}} \cos(t) \\
y(t) &= \answer{\sqrt{2}} \sin(t) \\
\end{align*}

A parameterization of the desired circle in the $xy$- plane is

\[
\vec{r}(t) = \vector{\answer{\sqrt{2}\cos(t)},\answer{\sqrt{2} \sin(t)}},
\]

and a parameterization of the associated curve on the surface is 

\[
\vec{R}(t) = \vector{\answer{\sqrt{2}\cos(t)},\answer{\sqrt{2} \sin(t)}, \answer{ e^{2\cos^2(t)+3\sqrt{2}\sin(t)-4} }}.
\]

\begin{feedback}[correct]
Note that if we set $x=t$, we will to use both $\vec{r}_1(t) = \vector{t,\sqrt{2-t^2}}, -\sqrt{2} \leq t \leq \sqrt{2}$ and $\vec{r}_2(t) = \vector{t,-\sqrt{2-t^2}}, -\sqrt{2} \leq t \leq \sqrt{2}$ to trace out the entire circle in the $xy$-plane.

\end{feedback}
\begin{exercise}
Find a parametric description of the tangent line to the curve in the previous part at the point $(-1,1,1)$.

\[
\vec{l}(t) =  \vector{\answer{-t-1 },\answer{ -t+1 },\answer{ -t+1}}
\]

\begin{hint}
Suppose that $t=t_0$ is the $t$-value for which $\vec{r}(t_0) = \vector{-1,1}$.  We do not need to exhibit $t_0$ explicitly; we will only need to use the facts

\begin{align*}
x(t_0) &= \sqrt{2} \cos(t_0) & \textrm{ so }&  & \cos(t_0) &=-\frac{1}{\sqrt{2}}. \\
y(t_0) &= \sqrt{2} \sin(t_0) & \textrm{ so } &  & \sin(t_0) &=\frac{1}{\sqrt{2}}. \\
\end{align*}

\begin{hint}
A vector parallel to the tangent line is $\vec{R}'(t_0)$.  

\begin{align*}
\vec{R}'(t) &= \vector{\answer{-\sqrt{2}\sin(t)},\answer{\sqrt{2}\cos(t) },\answer{ (-4\cos(t)\sin(t) +3\sqrt{2}\cos(t))\cdot  e^{2\cos^2(t)+3\sqrt{2}\sin(t)-4} }} \\
\end{align*}

To find $\vec{R}'(t_0)$, note $\cos(t_0)=-\frac{1}{\sqrt{2}}$ and $\sin(t_0) = \frac{1}{\sqrt{2}}$, so

\[
\vec{R}(t_0) = \vector{\answer{-1 },\answer{ -1 },\answer{ -1}} \\
\]
Use $\vec{l}(t) =\vec{v}t+\vec{P}_0$ where $P_0 = (-1,1,1)$.
\end{hint}
\end{hint}

To determine whether the tangent line $\vec{l}(t)$ lies on the tangent plane, we must check whether

\[
-2\big[x(t)\big]+3\big[y(t)\big]-\big[z(t)\big] = 4
\]

for all $t$.  

Does the tangent line lie on the tangent plane?  \wordChoice{\choice[correct]{Yes}\choice{No}}

\begin{feedback}[correct]
Note that 

\begin{align*}
-2\big[x(t)\big]+3\big[y(t)\big]-\big[z(t)\big] &=-2\big[-t-1\big]+3\big[-t+1\big]-\big[-t+1\big] \\
&= 2t+2-3t+3+t-1 \\
&= 4
\end{align*}
\end{feedback}


\end{exercise}
\end{exercise}
\end{exercise}
\end{document}
