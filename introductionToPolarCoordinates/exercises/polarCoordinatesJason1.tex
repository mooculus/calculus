\documentclass{ximera}

\newcommand{\RR}{\mathbb R}
\renewcommand{\d}{\,d}
\newcommand{\dd}[2][]{\frac{d #1}{d #2}}
\renewcommand{\l}{\ell}
\newcommand{\ddx}{\frac{d}{dx}}
\newcommand{\dfn}{\textbf}
\newcommand{\eval}[1]{\bigg[ #1 \bigg]}


\author{Jason Miller}
\license{Creative Commons 3.0 By-bC}


\outcome{}

\begin{document}
\begin{exercise}

%The points below are given in polar coordinates. Express them in Cartesian coordinates. 

Express $\left( 6, \frac{3\pi}{2}\right)$ in Cartesian coordinates: $\left( \answer{0}, \answer{-6 }\right)$

Express $\left(3, \frac{7 \pi}{4}\right)$ in Cartesian coordinates: $\left( \answer{\frac{3\sqrt{2}}{2}}, \answer{\frac{-3\sqrt{2}}{2}} \right)$

Express $\left( -5, \frac{4\pi}{3} \right)$ in Cartesian coordinates: $\left( \answer{ \frac{5}{2}}, \answer{\frac{5\sqrt{3}}{2}} \right)$

Express $\left( -10, \frac{-5\pi}{3} \right)$ in Cartesian coordinates: $\left( \answer{ -5 },  \answer{ \frac{-10 \sqrt{3}}{2}    }     \right)$

\begin{exercise}

The points given below are in Cartesian coordinates. Express them in polar coordinates. Use angle values that lie in $[0, 2\pi)$ and use positive values for the radius to get a unique representation. 


Express $\left( 3, 4 \right)$ in polar coordinates: $\left( \answer{ 5}, \answer{  \arctan\left( \frac{4}{3}\right) } \right) $. 

Express $\left(-7, 3 \right)$ in polar coordinates: $\left( \answer{ \sqrt{58} } , \answer{  \arctan\left( \frac{-3}{7} \right) + \pi    } \right)$. 

Express $\left(  3  ,  -8   \right)$ in polar coordinates: $\left (  \answer{ \sqrt{73}  }, \answer{\arctan\left( \frac{-8}{3}\right) + 2\pi } \right)$



\end{exercise}
\end{exercise}
\end{document}
