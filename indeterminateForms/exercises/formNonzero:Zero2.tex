\documentclass{ximera}

\newcommand{\RR}{\mathbb R}
\renewcommand{\d}{\,d}
\newcommand{\dd}[2][]{\frac{d #1}{d #2}}
\renewcommand{\l}{\ell}
\newcommand{\ddx}{\frac{d}{dx}}
\newcommand{\dfn}{\textbf}
\newcommand{\eval}[1]{\bigg[ #1 \bigg]}


\outcome{Understand what is meant by the form of a limit.}
\outcome{Calculate limits of the form zero over zero.}
\outcome{Identify determinate and indeterminate forms.}
\outcome{Distinguish between determinate and indeterminate forms.}

\author{Nela Lakos}

\begin{document}


\begin{exercise}
Let $g(x) = \frac{2}{x+3}-\frac{1}{x+2}$, and $h(x) = x-4$.\\\\

\begin{enumerate}
\item Evaluate the limit.
\[
\lim_{x\to4}g(x) = \lim_{x\to4}\dfrac{\answer{x+1}}{(x+2)(x+3)}=\answer{\frac{5}{42}}
\]

\item Choose all correct statements regarding  the form of the limit. 
\[
\lim_{x\to4}\dfrac{ \frac{2}{x+3}-\frac{1}{x+2}}{x-4}
\]
Choose all correct statements.
\begin{selectAll} 
\choice[correct]{The limit is of determinate form.}
\choice{The limit is of indeterminate form.}
\choice{The limit is of the form $\dfrac{0}{0}$.}
\choice[correct]{The limit is of the form $\dfrac{\#}{0}$.}
\end{selectAll}
\noindent\rule[0.5ex]{\linewidth}{0.2pt}
\end{enumerate}
\begin{exercise}
 Evaluate the limit. Possible answers include a number, $+\infty$, $-\infty$ and $DNE$.
\[
\lim_{x\to4+}\dfrac{ \frac{2}{x+3}-\frac{1}{x+2}}{x-4}=\answer{+\infty}
\]
 Justify your answer  by choosing all correct statements.
 \begin{selectAll} 
\choice{The numerator is negative and the denominator is positive and approaching zero.}
\choice[correct]{The numerator is positive and the denominator is positive and approaching zero.}
\choice{The numerator is positive and the denominator is negative and approaching zero.}
\choice{The numerator is negative and the denominator is negative and approaching zero.}
\end{selectAll}
\noindent\rule[0.5ex]{\linewidth}{0.2pt}
\begin{exercise}
 Evaluate the limit. Possible answers include a number, $+\infty$, $-\infty$ and $DNE$.

\[
\lim_{x\to4-}\dfrac{ \frac{2}{x+3}-\frac{1}{x+2}}{x-4}=\answer{-\infty}
\]
 Justify your answer  by choosing all correct statements.
  \begin{selectAll} 
\choice{The numerator is negative and the denominator is positive and approaching zero.}
\choice{The numerator is positive and the denominator is positive and approaching zero.}
\choice[correct]{The numerator is positive and the denominator is negative and approaching zero.}
\choice{The numerator is negative and the denominator is negative and approaching zero.}
\end{selectAll}
\noindent\rule[0.5ex]{\linewidth}{0.2pt}
\begin{exercise}
  Evaluate the limit. Possible answers include a number, $+\infty$, $-\infty$ and $DNE$.
\[
\lim_{x\to4}\dfrac{ \frac{2}{x+3}-\frac{1}{x+2}}{x-4}=\answer{DNE}
\]
 Justify your answer  by choosing all correct statements.
  \begin{selectAll} 
\choice[correct]{The limit from the left is  not equal to the limit from the right.}
\choice{The limit from the left is  equal to the limit from the right.}
\end{selectAll}
\noindent\rule[0.5ex]{\linewidth}{0.2pt}
\end{exercise}\end{exercise}
\end{exercise}
\end{exercise}
\end{document}