\documentclass{ximera}

\newcommand{\RR}{\mathbb R}
\renewcommand{\d}{\,d}
\newcommand{\dd}[2][]{\frac{d #1}{d #2}}
\renewcommand{\l}{\ell}
\newcommand{\ddx}{\frac{d}{dx}}
\newcommand{\dfn}{\textbf}
\newcommand{\eval}[1]{\bigg[ #1 \bigg]}


\outcome{}

\title[Break-Ground:]{A secret of the definite integral}

\begin{document}
\begin{abstract}
Two young mathematicians discuss what calculus is all about.
\end{abstract}
\maketitle


Check out this dialogue between two calculus students (based on a true
story):

\begin{dialogue}
\item[Devyn] Ah. So now we have a connection between derivatives and
  integrals.
\item[Riley] Right, the derivative of the accumulation function is the
  ``inside'' function.
\item[Devyn] So how do we use this to compute area?
\end{dialogue}

Sometimes it helps to think about the most basic examples. Consider
\[
\int_2^5 4 \d t
\]
We know (by geometry) that this computes the area of a $3\times 4$
rectangle which equals $12$. On the other hand, if we consider the
accumulation function
\[
F(x) = \int_2^x 4 \d t,
\]
we see that
\[
F(5) = \int_2^5 4 \d t.
\]
\begin{problem}
  What is $F(2)$?
  \begin{prompt}
    \[
    F(2) = \answer{0}
    \]
  \end{prompt}
\end{problem}

\begin{problem}
  On the other hand, the First Fundamental Theorem of Calculus says that if
  \[
  F(x) = \int_2^x 4 \d t,
  \]
  then $F'(x) = 4$. Armed with this knowledge, and the fact that $F(2)
  = 0$, what must $F(x)$ be?
  \begin{prompt}
    \[
    F(x) = \answer{4x-8}
    \]
  \end{prompt}
\end{problem}





%%% \begin{xarmaBoost}
%%   Write down at least \textbf{five} questions for this lecture. After
%%   you have your questions, label them as ``Level 1,'' ``Level 2,'' or
%%   ``Level 3'' where:
%% \begin{description}
%% \item[Level 1] Means you know the answer, or know exactly how to do
%%   this problem.
%% \item[Level 2] Means you think you know how to do the problem.
%% \item[Level 3] Means you have no idea how to do the problem.
%% \end{description}
%% \begin{freeResponse}
%% \end{freeResponse}
%% \end{xarmaBoost}


\end{document}
