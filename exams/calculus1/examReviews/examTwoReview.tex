\documentclass{ximera}
\newcommand{\RR}{\mathbb R}
\renewcommand{\d}{\,d}
\newcommand{\dd}[2][]{\frac{d #1}{d #2}}
\renewcommand{\l}{\ell}
\newcommand{\ddx}{\frac{d}{dx}}
\newcommand{\dfn}{\textbf}
\newcommand{\eval}[1]{\bigg[ #1 \bigg]}

\author{Tom Dinitz and Nela Lakos}
\license{Creative Commons 3.0 By-NC}
\title{Exam Two Review}

\begin{document}
\begin{abstract}
Review questions for exam 2.
\end{abstract}
\maketitle
%Exercise 1

\begin{exercise}
(a) Suppose $s(t)$ is the position of an object moving along a line at time $t\geq 0$. 

What is the average velocity between times $t=a$ and $t=b$?
\begin{prompt}
\begin{multipleChoice}
\choice{$s(b)-s(a)$}
\choice[correct]{$\frac{s(b)-s(a)}{b-a}$}
\choice{$\frac{b-a}{s(b)-s(a)}$}
\choice{$b-a$}
\end{multipleChoice}
\end{prompt}

(b)The table gives the position $s(t)$ of an object moving along a line at time $t$, over a three second interval.

\[
\begin{array}{|c|c|c|c|c|c|c|c|}
\hline
t& 0& 0.5& 1& 1.5& 2& 2.5& 3\\ \hline
s(t)& 0 & 22 & 32 & 48 & 54 & 64 & 74 \\ \hline
\end{array}
\]

The average velocity over the interval $[1,3]$ is $v_{av}=\begin{prompt}\answer{21}\end{prompt}$

(c) Make a conjecture about the value of the instantenous velcoty at $t=1$: $\begin{prompt}\answer{26}\end{prompt}$
\end{exercise}

%Exercise 2
\begin{exercise}
(a) Fill in the blanks: $\lim_{\answer[format=string]{h}\to\answer{0}}\frac{f(\answer{x+h})-f(\answer{x}}{\answer{h}}$

(b) Let $f(x)=\frac{1}{x-4}$. Using the (limit) defintion of the derivative in (a), compute $f'(x)=\answer{-(x-4)^{-2}}$
\end{exercise}

%Exercise 3
\begin{exercise}
The graph of a function $f$ is given below

INSERT GRAPH

(1) (a) Select the figure which has a secant line drawn through the points $(0,f(0))$ and $(4,f(4))$.

INSERT GRAPHS

(b) Select the figure which has the line tangent to the curve at $x=1$

INSERT GRAPHS

(2) Mutliple Choice:

(i) \begin{multipleChoice}
\choice{$f(1)=f(3)$}
\choice{$f(1)>f(3)$}
\choice[correct]{$f(1)<f(3)$} 
\end{multipleChoice}

(ii) \begin{multipleChoice}
\choice{$f'(1)=f'(3)$}
\choice[correct]{$f'(1)>f'(3)$} 
\choice{$f'(1)<f'(3)$} 
\end{multipleChoice}

(iii) Select the best approximation of $f'(3)$
\begin{multipleChoice}
\choice{$f'(3)\approx 2$}
\choice{$f'(3)\approx -2$}
\choice{$f'(3)\approx 1$}
\choice{$f'(3)\approx -1$}
\choice[correct]{$f'(3)\approx \frac{1}{2}$}
\choice{$f'(3)\approx -\frac{1}{2}$}  
\end{multipleChoice}
\end{exercise}

%Exercise 4
\begin{exercise}
Let $f$ be a function such that $f(4)=3$ and $f'(4)=-6$.

(a) Find the limit or say it does not exist (DNE).

$\lim_{x\to 4} \frac{f(x)-f(4)}{x-4}=\answer{-6}$

(b) An equation of the tangent line to the curve $y=f(x)$ at the point where $x=4$ is given by $y=\answer{-6x+27}$
\end{exercise}

%Exercise 5
\begin{exercise}
(i) Given $e^{xy}=e+x-y$, use implicit differentiation to find $\frac{dy}{dx}=\answer{\frac{1-ye^{xy}}{1+xe^{xy}}}$

(ii) Given y, find $\frac{dy}{dx}$:

(a) $y=\left(5\sin(x)\right)\arctan\left(\frac{x}{x+1}\right)$. $\frac{dy}{dx}=\answer{5\sin(x)\frac{(x+1)^{-1}-x(x+1)^{-2}}{1+\frac{x^2}{(x+1)^2}}+5\cos(x)\arctan(\frac{x}{x+1})}$

(b) $y=\frac{1}{\sqrt{\tan(x^2)+5)}}$. $\frac{dy}{dx}=\answer{\frac{-x\sec^2(x^2)}{(\tan(x^2)+5)^{1.5}}}$

(c) $y=xe^{ax}+\arcsin(ax)$. $\frac{dy}{dx}=\answer{e^{ax}+axe^{ax}+\frac{a}{\sqrt{1-a^2x^2}}}$

(d) $y(x)=\left(\sin(x)\right)^{ax}$. $\frac{dy}{dx}=\answer{\sin(x)^{ax}(ax\cot(x)+a\log(\sin(x)))}$

(ii) Find the following values or state the value is undefined (DNE):

(a) $\sin\left(\frac{17\pi}{6}\right)=\answer{\frac{1}{2}}$

(b) $\arcsin\left(\frac{17\pi}{6}\right)=\answer[format=string]{DNE}$

(c) $\arcsin\left(\frac{-\sqrt{3}}{2}\right)=\answer{-\frac{\pi}{3}}$

(d) $e^{-2\ln(3)}=\answer{3^{-2}}$

(e) $\ln\left(-e^2\right)=\answer[format=string]{DNE}$

(f) $\ln\left(e^{2t}\right)=\answer{2t}$

(g) $10^{3\log_10(4)}=\answer{4^3}$

(h) $\ln(1)=\answer{0}$

(i) $\arctan\left(-\sqrt{3}\right)=\answer{\frac{-\pi}{3}}$

(j) At $x=\frac{\pi}{3}$, $\ddx\left(\cot(x)\right)=\answer{-\frac{4}{3}}$

(k) At $x=-\sqrt{3}$, $\ddx\left(\arccot(x)\right)=\answer{-\frac{1}{4}}$
\end{exercise}

\begin{exercise}
A table of values for $f(x)$ and $f'(x)$, along with a graph of a function $g(x)$ is shown below.
\[
\begin{array}{c|c|c}
x & f(x)&f'(x)\\ \hline
1 & 2 & 4\\ \hline
2 & 3 & 5\\ \hline
3 & 4 & 1\\ \hline
\end{array}
\]
\begin{image}
\begin{tikzpicture}
    \begin{axis}[
            domain=0:6,
            ymax=5,
        ymin=-1,
            samples=100,
            axis lines =middle, xlabel=$x$, ylabel=$y$,
        ytick={1,2,3,4},
            every axis y label/.style={at=(current axis.above origin),anchor=south},
            every axis x label/.style={at=(current axis.right of origin),anchor=west}
          ]
          \addplot [very thick, penColor, smooth, domain=(0:1)] {3*x};
          \addplot [very thick, penColor, smooth, domain=(1:4)] {4-x};
      \addplot [very thick, penColor, smooth, domain=(4:6)] {(x-4)^2};
        \end{axis}
\end{tikzpicture}
\end{image}

(i) At $x=2$, $\ddx\left[f(x)g(x)\right]$=\begin{multipleChoice}
\choice{$\frac{13}{3}$}
\choice{$\frac{4}{3}$}
\choice{12}
\choice{4}
\choice[correct]{DNE}
\choice{None of the above}
\end{multipleChoice}

(ii) At $x=3$, $\ddx\left[f(g(x))\right]$=\begin{multipleChoice}
\choice{-1}
\choice[correct]{-4}
\choice{$\frac{1}{4}$}
\choice{4}
\choice{DNE}
\choice{None of the above}
\end{multipleChoice}

(iii) At $x=2$, $\ddx\left[g(f(x))\right]$=\begin{multipleChoice}
\choice{-1}
\choice[correct]{-5}
\choice{$\frac{1}{5}$}
\choice{0}
\choice{DNE}
\choice{None of the above}
\end{multipleChoice}

(iv) $f^{-1}(3)=$ \begin{multipleChoice}
\choice{4}
\choice{1}
\choice{$\frac{1}{3}$}
\choice[correct]{2}
\choice{DNE}
\choice{None of the above}
\end{multipleChoice}

(v) At $x=3$, $\ddx\left[f^{-1}(x)\right]$=\begin{multipleChoice}
\choice{$\frac{1}{4}$}
\choice[correct]{$\frac{1}{5}$}
\choice{2}
\choice{4}
\choice{DNE}
\choice{None of the above}
\end{multipleChoice}
\end{exercise}

%Exercise 7
\begin{exercise}
The function $f$ is defined by $f(x)=\frac{x}{\sqrt{x^2-9}}$

(a) The domain of $f$ is $(\answer{-\infty},\answer{-3})\cup (\answer{3},\answer{\infty})$
(b) The function $f$ is \begin{multipleChoice}
\choice[correct]{odd}
\choice{even}
\choice{neither}
\end{multipleChoice}

(c) Select all horizonal asymptotes
\begin{selectAll}
\choice{y=0}
\choice[correct]{y=1}
\choice{x=1}
\choice[correct]{y=-1}
\choice{There are no horizontal asymptotes}
\end{selectAll}

(d) Select all vertical asymptotes
\begin{selectAll}
\choice[correct]{x=-3}
\choice{y=1}
\choice[correct]{x=3}
\choice{y=3}
\choice{There are no vertical asymptotes}
\end{selectAll}  

(e) $f'(x)=\answer{\frac{-9}{(x^2-9)^{1.5}}}$

(f) $f''(x)=\answer{\frac{27x}{(x^2-9)^{2.5}}}$

(g) Fill in the blanks with the appropraite \underline{underlined} choices.

\begin{tabular}{|c|c|c|}\hline
 & f is \underline{increasing} & f is \underline{concave UP} \\
On this interval & f is \underline{decreasing} & f is \underline{concave DOWN} \\
 & f is \underline{Not Defined} & f is \underline{Not Defined} \\ \hline
$(-\infty,-3)$ & $\answer[format=string]{decreasing}$& $\answer[format=string]{concave DOWN}$\\ \hline
$(-3,3)$ & $\answer[format=string]{Not Defined}$& $\answer[format=string]{Not Defined}$\\ \hline
$(3,\infty)$ & $\answer[format=string]{decreasing}$& $\answer[format=string]{concave UP}$\\\hline 
\end{tabular}

(h) Select the correct graph for $f$:

INSERT GRAPHS

(i) Is the function $f$ one-to-one:
\begin{multipleChoice}
\choice[correct]{Yes}
\choice{No}
\end{multipleChoice}
\end{exercise}

%Exercise 8
\begin{exercise}
The graph of $f'$ (the derivative of $f$) on the interval $[-6,7]$ is shown in the figure.

INSERT GRAPH

Use the given graph of $f'$ to answer the following questions about f:

(a) On what interval(s) is $f$ decreasing: $[\answer{-6},\answer{-2}]$ and $[\answer{6},\answer{7}]$

(b) The critical points of $f$ are $x=\answer{-2},\answer{4},\answer{6}$

(c) Which critical points correspond to local maxima? $x=\answer{6}$

(d) Which critical poitns correspond to neither local maxima nor local minima? $x=\answer{4}$

(e) On what intervals is $f$ concave up? $[\answer{-6},\answer{0}]$ and $[\answer{4},\answer{5}]$
(f) At what points does $f$ have an inflection point? $x=\answer{0},\answer{4},\answer{6}$
\end{exercise}

%Exercise 9
\begin{exercise}
Given that $f(1)=2$, $f'(1)=3$, and $f''(1)=-1$, find the following values or state `cannot be determined':

(a) $\left(\ddx\frac{(x+5)f'(x)}{f(x)}\right)_{x=1}=\answer{\frac{-9}{4}}$

(b) $\lim_{x\to 1}f(x)=\answer{2}$

(c) $\left(\ddx f^{-1}(x)\right)_{x=2}=\answer{\frac{1}{3}}$

(d) $\lim_{h\to 0}\frac{f(1+h)-f(1)}{h}=\answer{3}$

\end{exercise}

%Exercise 10
\begin{exercise}
Given that $h(x)=\ln(\sec(x)+\tan(x))$, $f'(x) = \answer{\sec(x)}$
\end{exercise}

%Exercise 11

\begin{exercise}
The position, $s(t)$, of an object moving along a horizontal line is given by $s(t)=3\sin\left(\frac{\pi}{4}t\right)$, $t\geq0$, where $s$ is measured in feet and $t$ in seconds.

(a) The position of the particle at time $t=2$ is $\answer{3}$

(b) The \underline{average velocity}, $v_{av}$ of the object over the interval $[0,t]$ is $\answer{\frac{3\sin(\frac{\pi}{4}t)}{t}}$

(c) Using the expression found in part (b), evaluate $\lim_{t\to0^+} v_{av} = \answer{\frac{3\pi}{4}}$

(d) Find the velocity of the particle: $v(t)=\answer{\frac{3\pi}{4}\cos(\frac{\pi}{4}t)}$

(e) Find the acceleartion of the particle: $a(t)=\answer{-\frac{9\pi^2}{16}\sin(\frac{\pi}{4}t)}$

\end{exercise}

%Exercise 12
\begin{exercise}
The figure below shows the graphs of $f$, $f'$, and another function $g$.

INSERT GRAPH

Which curve is which: $f=\answer[format=string]{c}$; $f'=\answer[format=string]{a}$; $g=\answer[format=string]{b}$
\end{exercise}
%Exercise 13

\begin{exercise}
The (entire) graph of a function $f$ is shown in the figure below.

INSERT GRAPH

Give answers for the following, or write `DNE'.

(a) Find all points where $f$ attains its absolute maximum: $x=\answer{2}$

(b) Find all points where $f$ attains its absolute minimum: $x=\answer{5}$

(c) Find all points where $f$ has a local minimum: $x=\answer{1},\answer{5}$

(d) Find all points where $f$ has a local maximum: $x=\answer{0},\answer{2}$

(e) Find all critical points: $x=\answer{1},\answer{2}$
\end{exercise}

%Exercise 14
\begin{exercise}
Assume that a function $f$ is continuous on its domain, $(-1,6)$. The graph of $f'$, the derivative of $f$, is shown in the figure below.

INSERT GRAPH

(a) Find the x-coordinates of all critical points of $f$ (or write NONE): $x=\answer{0},\answer{3},\answer{5}$

(b) Find the x-coordinates of all local maxima of $f$ (or write NONE): $x=\answer{5}$

(c) Find the x-coordinates of all local minima of $f$ (or write NONE): $x=\answer{3}$

(d) Find the interval(s) on which $f$ is increasing: $[\answer{3},\answer{5}]$

(e) Find the interval(s) on which $f$ is concave down: $[\answer{0},\answer{1}]$

(f) Find the x-coordinates of all inflection points (or write NONE): $x=\answer{0},\answer{1}$
\end{exercise}

%Exercise 15
\begin{exercise}
A curve is given by the equation $\tan(3y+x)=x^2-1$

(i) (a) use implicit differentiation to find the derivative: $\frac{dy}{dx}=\answer{\frac{2x}{3\sec^2(3y+x)}-\frac{1}{3}}$

(b) Check (algebraically), that the point $(1,-\frac{1}{3})$ lies on the curve.

(ii) Part of the curve $tan(3y+x)=x^2-1$ that contains the point $(1,-\frac{1}{3})$ is shown in the figure below:

INSERT GRAPH

(a) Find an explicit expression for $y$, $y=y(x)$, represnted by the graph above: $y(x)=\answer{\frac{1}{3}(\arctan(x^2-1)-x)}$

(b) Using the explicit expression in part (a), find $\frac{dy}{dx}=\answer{\frac{1}{3}(\frac{2x}{(x^2-1)^2+1}-1)}$

(c) At $x=1$, $\frac{dy}{dx}=\frac{1}{3}$

(d) Select the graph with the line tangent to the curve at the point $(1,-\frac{1}{3})$

INSERT GRAPHS

(e) Write an equation for this tangent line: $y=\answer{\frac{1}{3}(x-1)-\frac{1}{3}}$
\end{exercise}

%Exercise 16
\begin{exercise}
Select the graph of $f$ given that it satisfies all of the following conditions:

(a) Domain of $f=(-\infty,-2)\cup (-2,\infty)$,

(b) f is continuous on its domain and differentiable all all points in the domain except at $x=8$

(c) $f(2)=0, f(8)=10$,

(d) $\lim_{x\to -2} f(x)=-8$, $\lim_{x\to-\infty}f(x)=5$, $\lim_{x\to\infty}f(x)=5$,

(e) $f'(x)<0$ on $(-\infty,-2)$, and on $(8,\infty)$,

(f) $f'(x)>0$ on $(-2,8)$,

(g) $f''(x)<0$ on $(-\infty,-2)$ and on $(-2,2)$,

(h) $f''(x)>0$ on $(2,8)$ and on $(8,\infty)$

INSERT GRAPHS
\end{exercise}

%Exercise 17
\begin{exercise}
A function $f'$ (the derivative of $f$) is given by $f'(x)=(x-5)^3$. Answer the following, or put DNE

(a) List all interval(s) on which $f$ is increasing: $[\answer{5},\infty)$

(b) Find all points where $f$ has a loal maximum: $x=\answer[format=string]{DNE}$

(c) Find all points where $f$ has a local minimum: $x=\answer{5}$

(d) $f''(x)=\answer{3(x-5)^2}$

(e) List all interval(s) on which $f$ is concave up: $(-\infty,\infty)$

(f) List all inflection points of $f$: $x=\answer[format=string]{DNE}$

\end{exercise}

%Exercise 18
\begin{exercise}
The function $s(t)=\frac{12}{t+1}$ gives the position of an object moving along a line at time $t\geq 0$. 

(a) Is the function $s$ one-to-one? 
\begin{multipleChoice}
\choice[correct]{Yes}
\choice{No}
\end{multipleChoice}

(b) Find the inverse of $s$: $s^{-1}(t)=\answer{\frac{12}{t}-1}$

(c) Find $s^{-1}(5)=\answer{\frac{7}{5}}$

(d) Interpret your answer to part (c):
\begin{multipleChoice}
\choice{$s^{-1}(5)$ gives the speed of the object at time 5}
\choice{$s^{-1}(5)$ gives the position of the object at time 5}
\choice[correct]{$s^{-1}(5)$ gives the time when the object was at position 5}
\choice{$s^{-1}(5)$ gives the time when the object had speed 5}
\end{multipleChoice}

(e) The average velocity between times $t=0$ and $t=2$ is $v_{av}=\answer{-4}$

(f) Find an expression for the instantaneous velocity at time $t>0$: $v(t)=\answer{-12(t+1)^-2}$

(g) Find an expressin for the instantaneous acceleration at time $t>0$: $a(t)=\answer{24(t+1)^{-3}}$

(h) Is the velocity increasing or decreasing for $t>0$?
\begin{multipleChoice}
\choice[correct]{Increasing}
\choice{Decreasing}
\end{multipleChoice}

(i) Is the speed increasing or decreasing for $t>0$?
\begin{multipleChoice}
\choice{Increasing}
\choice[correct]{Decreasing}
\end{multipleChoice}
\end{exercise}

%Exercise 19
\begin{exercise}
A ladder $10$ ft long rests against a vertical wall. If the bottom of the ladder slides away from the wal at a speed of $2$ ft/s, how fast is the angle between the top of the ladder and the wall changing when the angle is $\frac{\pi}{3}$ radians?  $\answer{\frac{2}{5}}$
\end{exercise}

%Exercise 20
\begin{exercise}
The position function for an object at any time $t$ is given by $s(t)=-40t^2+160t+480$.

(a) Find the velocity $v(t)$ at any time $t$: $v(t)=\answer{-80t+160}$

(b) Find the acceleration $a(t)$ at any time $t$: $a(t)=-80$

(c) At what time is the object furthest from the oriign in the positive direction? $t=\answer{2}$

(d) What are the velocity and acceleartion at that time? $v=\answer{0}$, $a=\answer{-80}$

(e) At what (positive) time is the object at the origin? $t=\answer{6}$

(f) What are the velocity and acceleartion at that time? $v=\answer{-320}$, $a=\answer{-80}$

(g) Find the time interval(s) when the velocity is decreasing: $(\answer{-\infty},\answer{\infty})$
\end{exercise}

%Exercise 20
\begin{exercise}
The (entire) graph of a function $f$ is given in the figure below:

INSERT GRAPH

(a) At $x=3$, $\ddx\left[f^{-1}(x)\right]=\answer{\frac{-1}{2}}$

(b) List all critical points of $f$: $\answer{-8},\answer{3}$

(c) List the x-coordinates of all local minima of $f$, or say `none': $x=\answer{3}$

(d) List the x-coordinates of all local maxima of $f$, or say `none': $x=\answer{-8}$

(e) List the x-coordinates of all absolute maxima of $f$, or say `none': $x=\answer[format=string]{none}$

(f) List the x-coordinates of all local maxima of $f$, or say `none': $x=\answer[format=string]{none}$

(g) Select the correct graph of $f'$

INSERT GRAPHS
\end{exercise}

%Exercise 21
\begin{exercise}
(i) Let $f(x)=x^4-8x^2+10$. 

(a) Find the critical points of $f$: $x=\answer{-2},\answer{0},\answer{2}$

(b) Find the absolute maximum of $f$ on the interval $[-3,1]$: $\answer{19}$

(c) Find the absolute minimum of $f$ on the interval $[-3,1]$: $\answer{-6}$

(ii) Let $f(x)=x\ln(x)$

(a) Find the domain of the function $f$: $(\answer{0},\answer{\infty})$

(b) Find the critical points of $f$: $\answer{\frac{1}{e}}$

(c) Find the interval(s) on which $f$ is increasing: $(\answer{e},\answer{\infty})$

(d) Find the interval(s) on which $f$ is decreasing: $(\answer(0),\answer{e^{-1}})$

(e) Find the absolute minimum value of $f$ on its domain or say that it does not exist (DNE): $\answer{-\frac{1}{e}}$

(f) Find the absolute maximum value of $f$ on its domain or say that it does not exist (DNE): $\answer[format=string]{DNE}$

(g) Find all inflection points of $f$, or, if non exist, say DNE: $x=\answer[format=string]{DNE}$

(i) Find all interval(s) on which f is concave up: $(\answer{0},\answer{\infty})$
(iii) Let $f(x)=x\sqrt{6-x^2}$

(a) Find the domain of the function $f$: $[\answer{-\sqrt{6}},\answer{\sqrt{6}}]$
(b) State the interval(s) of continuity of $f$: $[\answer{-\sqrt{6}},\answer{\sqrt{6}}]$

(c) Find the critical points of $f$: $\answer{-\sqrt{6}},\answer{-\sqrt{3}},\answer{\sqrt{3}}, \answer{\sqrt{6}}$

(d) Find the absolute maximum of $f$ on its domain: $\answer{3}$

(e) Find the absolute minimum of $f$ on its domain: $\answer{-3}$
\end{exercise}

%Exercise 22
\begin{exercise}
A water tank is to be drained for cleaning. There are $V$ liters of water lef tin the tank $t$ minutes after the draining began, where $V=42(60-t)^2$

(a) Find the average rate at which water drains during the first $10$ minutes: $\answer{-4620}$

(b) Find the rate at which the volume of water is changing $10$ minutes after the draining began. $\answer{-4200}$

(c) $\frac{d}{dt}\left(\frac{\frac{dV}{dt}(t)}{V(t)}\right)=\answer{-2(60-t)^{-2}}$

(d) What are the units of the answer to part (c)?

(e) Find the rate of the rate at which the volume of water is chaning 10 minutes after draining begins: $\answer{-84}$

(f) Is the rate at which the volume of the water is changing increase or decreasing (during the draining)? 
\begin{multipleChoice}
\choice[correct]{Increasing}
\choice{Decreasing}
\end{multipleChoice}

(g) Assume that the tank has the shape of a rectangular box $7$m long, $6$m wide, and $5$m high. What is the rate of change of the water depth when the water depth is $3$m? (HINT: 1 liter = .001 $m^3$) $\answer{\frac{-\sqrt{30}}{50}}$

\end{exercise}

%Exercise 23
\begin{exercise}
We are inflating a spherical balloon at a rate of 3 $cm^3/sec$. 

(a) At what rate is the radius increasing when the radius is $6$ cm? $\answer{\frac{1}{48\pi}}$

(b) At what rate is the surface area increasing at that moment? $\answer{1}$
\end{exercise}

%Exercise 24
\begin{exercise}
Two cars leave an intersection. One heads west at $30$ mi/hr. The other leaves 30 minutes later, heading north at $40$ mi/hr. How fast is the distance between them changing $1$ hr after the first car left the intersection? $\answer{\frac{31.25}{\sqrt{.8125}}}$


\end{exercise}

\end{document}
