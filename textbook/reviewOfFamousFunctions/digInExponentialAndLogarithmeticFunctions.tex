\documentclass{ximera}

\newcommand{\RR}{\mathbb R}
\renewcommand{\d}{\,d}
\newcommand{\dd}[2][]{\frac{d #1}{d #2}}
\renewcommand{\l}{\ell}
\newcommand{\ddx}{\frac{d}{dx}}
\newcommand{\dfn}{\textbf}
\newcommand{\eval}[1]{\bigg[ #1 \bigg]}


\title[Dig-In:]{Exponential and logarithmetic functions}


\begin{document}
\begin{abstract}
  Exponential and logarithmetic functions illuminated.
\end{abstract}
\maketitle

Exponential functions may seem somewhat esoteric at first, but they
model may pheonomema in the real-world.




\section{What are exponential and logarithmetic functions?}


\begin{definition}
  An \dfn{exponential function} is a function of the form
  \[
  f(x) = b^x
  \]
  where the $b\ne 1$ is a positive real number. The domain of an
  exponetial function is $(-\infty,\infty)$.
\end{definition}


\begin{definition}
  An \dfn{logarithmetic function} is a function defined as follows
  \[
  \log_b(x) = y \qquad\text{exactly when}\qquad b^y = x
  \]
  where the $b\ne 1$ is a positive real number. The domain of an
  logorithmetic function is $(0,\infty)$.
\end{definition}

In either definition above $b$ is the \dfn{base}.




\section{What can the graphs look like?}

\subsection{Graphs of exponential functions}

\begin{example}
  Here we see the the graphs of four exponential functions.
  \begin{image}
    \begin{tikzpicture}
      \begin{axis}[
          domain=-2:2,
          xmin=-2, xmax=2,
          ymin=-.5, ymax=4,
          axis lines =middle, xlabel=$x$, ylabel=$y$,
          every axis y label/.style={at=(current axis.above origin),anchor=south},
          every axis x label/.style={at=(current axis.right of origin),anchor=west},
        ]
	\addplot [very thick, penColor, smooth] {e^x};
        \addplot [very thick, penColor2, smooth] {2^x)};
        \addplot [very thick, penColor3, smooth] {(1/2)^x)};
        \addplot [very thick, penColor4, smooth] {(1/3)^x)};
        
        
        
        
        \node at (axis cs:-1.5, 2 ) [penColor3,anchor=west] {$A$};
        \node at (axis cs:-1, 3.5 ) [penColor4,anchor=west] {$B$};
        \node at (axis cs:0.8, 3.5 ) [penColor,anchor=west] {$C$};
        \node at (axis cs:1.2, 2 ) [penColor2,anchor=west] {$D$};
        
      \end{axis}
    \end{tikzpicture}
  \end{image}
  match curves $A$, $B$, $C$, and $D$ with the functions
  \[
  e^x, \qquad \left(\frac{1}{2}\right)^{x}, \qquad  \left(\frac{1}{3}\right)^{x}, \qquad 2^{x}.
  \]
  \begin{explanation}
    First note that since $e> 0$, $e^x$ increases as $x$
    increases. Hence $e^x$ is either $C$ or $D$. On the other hand,
    $2^x$ could also be either $C$ or $D$.  Comparing these functions
    at $x=1$, $e^1 > 2^1$, we see that $e^x$ corresponds to
    $\answer[given]{C}$ and that $2^x$ must correspond to
    $\answer[given]{D}$.

    On the other hand, $\left(\frac{1}{2}\right)^{x}$ decreases as $x$
    increases. Hence $\left(\frac{1}{2}\right)^{x}$ is either $A$ or
    $B$. On the other hand, $\left(\frac{1}{3}\right)^{x}$ could also
    be either $A$ or $B$. Comparing these functions at $x=1$,
    $\left(\frac{1}{2}\right)^{1} >\left(\frac{1}{3}\right)^{1}$,
    $\left(\frac{1}{2}\right)^{x}$ cooresponds to $\answer[given]{A}$.
    and that $\left(\frac{1}{3}\right)^{x}$ must correspond to
    $\answer[given]{B}$.
  \end{explanation}
\end{example}


\begin{example}
  Here we see the the graphs of four logarithmetic functions.
  \begin{image}
    \begin{tikzpicture}
      \begin{axis}[
          domain=0.05:4,
          xmin=-.5, xmax=4,
          ymin=-2, ymax=2,
          axis lines =middle, xlabel=$x$, ylabel=$y$,
          every axis y label/.style={at=(current axis.above origin),anchor=south},
          every axis x label/.style={at=(current axis.right of origin),anchor=west},
        ]
	\addplot [very thick, penColor, smooth] {ln(x)}; % C
        \addplot [very thick, penColor2, smooth] {ln(x)/ln(2)}; % D
        \addplot [very thick, penColor3, smooth, samples=100] {ln(x)/ln(1/2))}; % A
        \addplot [very thick, penColor4, smooth, samples=100] {ln(x)/ln(1/3))}; %B
        
        
        \node at (axis cs:.5, 1.3 ) [penColor3,anchor=west] {$A$};
        \node at (axis cs:.2, .5 ) [penColor4,anchor=west] {$B$};
        \node at (axis cs:0.2, -.5 ) [penColor,anchor=west] {$C$};
        \node at (axis cs:.5, -1.3 ) [penColor2,anchor=west] {$D$};
        
      \end{axis}
    \end{tikzpicture}
  \end{image}
  match curves $A$, $B$, $C$, and $D$ with the functions
  \[
  \ln(x),\qquad \log_{1/2}(x), \qquad \log_{1/3}(x),\qquad \log_2(x).
  \]
  \begin{explanation}
    First remember what $\ln(x)=y$ means:
    \[
    \ln(x) = y \qquad\text{exactly when}\qquad b^y = x.
    \]
    So now examine each of these functions along the horizontal line
    $y=1$
    \begin{image}
      \begin{tikzpicture}
        \begin{axis}[
            domain=0.05:4,
            xmin=-.5, xmax=4,
            ymin=-2, ymax=2,
            axis lines =middle, xlabel=$x$, ylabel=$y$,
            every axis y label/.style={at=(current axis.above origin),anchor=south},
            every axis x label/.style={at=(current axis.right of origin),anchor=west},
          ]
	  \addplot [very thick, penColor, smooth] {ln(x)}; % C
          \addplot [very thick, penColor2, smooth] {ln(x)/ln(2)}; % D
          \addplot [very thick, penColor3, smooth, samples=100] {ln(x)/ln(1/2))}; % A
          \addplot [very thick, penColor4, smooth, samples=100] {ln(x)/ln(1/3))}; %B
          \addplot [dashed] {1};
        
          
          \node at (axis cs:.5, 1.3 ) [penColor3,anchor=west] {$A$};
          \node at (axis cs:.2, .5 ) [penColor4,anchor=west] {$B$};
          \node at (axis cs:0.2, -.5 ) [penColor,anchor=west] {$C$};
          \node at (axis cs:.5, -1.3 ) [penColor2,anchor=west] {$D$};
          
        \end{axis}
      \end{tikzpicture}
    \end{image}
    , $e^x$ increases as $x$
    increases. Hence $e^x$ is either $B$ or $D$. Comparing these
    functions at $x=1$, $e^1> \left(\frac{1}{3}\right)^1$, $e^x$
    corresponds to $\answer[given]{C}$.

    On the other hand, $\left(\frac{1}{2}\right)^{-x}$ decreases as
    $x$ increases. Hence $\left(\frac{1}{2}\right)^{-x}$ is either $A$
    or $B$. Comparing these functions at $x=-1$,
    $\left(\frac{1}{2}\right)^{-(-1)}< 2^{-(-1)}$,
    $\left(\frac{1}{2}\right)^{-x}$ cooresponds to
    $\answer[given]{D}$.

    At this point we see that $\left(\frac{1}{3}\right)^{x}$
    corresponds to $\answer[given]{B}$ and that $2^{-x}$ corresponds
    to $\answer[given]{A}$.
  \end{explanation}
\end{example}



\subsection{Graphs of logarithmetic functions}


\begin{example}
\end{example}

\section{Rules of exponential functions and logarithms}
Rules of logs and exp

For each positive real number $a$, we now have (sort of) defined a
function $f(x) = a^x$, the exponential function with base $a$, which
is defined for all real numbers $x$.  For positive integers, it agrees
with the repeated multiplication definition, and for non-positive
integers and rationals it is defined in a way that it continues to
enjoy the nice algebraic properties ($a^{b+c} = a^b \cdot a^c$ and
$(a^b)^c = a^{bc}$) we expect of it.  For irrational numbers we do not
yet have a solid definition, but we will learn some in the next
chapters.  Basically the values at the irrationals are determined by
approximating the irrationals with rationals.


\begin{question}
What exponent 
  \begin{hint}
		$(2^4) \cdot (2^3) = (2 \cdot 2\cdot 2 \cdot 2) \cdot  (2 \cdot 2\cdot 2) = 2^7 $
	\end{hint}

	$2^4 \cdot 2^3 = 2^{\answer{7}}$
\end{question}


\begin{question}
	Which of the following are equal to $5^4$?
	\begin{selectAll}
		\choice[correct]{$5 \cdot 5^3$}
		\choice{$4 \cdot 4 \cdot 4 \cdot 4 \cdot 4$}
		\choice{$20$}
		\choice[correct]{$5^2 \cdot 5^2$}
		\choice[correct]{$(5^2)^2$}
	\end{selectAll}
\end{question}

\begin{question}
	Which of the following are equal to $(7^3)^2$?
	\begin{selectAll}
		\choice{$7^5$}
		\choice[correct]{$(7^3) \cdot (7^3)$}
		\choice[correct]{$7^6$}
		\choice{$7\cdot 7 \cdot 7 \cdot 7 \cdot 7$}
	\end{selectAll}
\end{question}

\end{document}
