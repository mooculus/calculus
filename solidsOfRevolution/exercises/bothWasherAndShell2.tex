\documentclass{ximera}

\newcommand{\RR}{\mathbb R}
\renewcommand{\d}{\,d}
\newcommand{\dd}[2][]{\frac{d #1}{d #2}}
\renewcommand{\l}{\ell}
\newcommand{\ddx}{\frac{d}{dx}}
\newcommand{\dfn}{\textbf}
\newcommand{\eval}[1]{\bigg[ #1 \bigg]}


\author{ Jason Miller}
\license{Creative Commons 3.0 By-NC}


\outcome{Find the bounded area between two curves}
\outcome{Find volume of solid described using cross sections}

\begin{document}
\begin{exercise}

The region $R$ is bounded by the curves $y=-x^{2}+2x$ and $y=0$. 


 \begin{image}
            \begin{tikzpicture}
            	\begin{axis}[
            		domain=-10:10, ymax=1.8,xmax=2.9, ymin=-0.8, xmin=-.4,
            		axis lines =center, xlabel=$x$, ylabel=$y$,
            		every axis y label/.style={at=(current axis.above origin),anchor=south},
            		every axis x label/.style={at=(current axis.right of origin),anchor=west},
            		axis on top,
            		]
                      
            	\addplot [draw=penColor,very thick,smooth,samples=100] {-x^2+2*x};
            	\addplot [draw=penColor2,very thick,smooth] {0};

                       
            	\addplot [name path=A,domain=0:2,draw=none] {-x^2+2*x};   
            	\addplot [name path=B,domain=0:2,draw=none] {0};
            	\addplot [fillp] fill between[of=A and B];
		
            	\node at (axis cs:1.8,1.1) [penColor] {$y=-x^2+2x$};
		
            	\end{axis}
            \end{tikzpicture}
            \end{image}


A solid has the region $R$ as its base.  Cross sections through the base taken perpendicular to the $x$-axis are squares.  Set up an integral that gives the volume of the solid.


  \begin{multipleChoice}
    \choice[correct]{we should integrate with respect to $x$.}
    \choice{we should integrate with respect to $y$.}
  \end{multipleChoice}

How many integrals will we need to express the volume of the solid? $\answer{1}$. 


\begin{exercise} 
The integral that gives the volume of is: 
\[
V = \int_{\answer{0}}^{\answer{2}} \answer{\left(-x^{2}+2x \right)^2} \d x
\] 


\end{exercise}
\end{exercise}

A solid is now formed by revolving $R$ about the $y$-axis.

To find the volume of the solid using an integral or sum of integrals with respect to $y$, which method should be used?

  \begin{multipleChoice}
    \choice{Shell Method}
    \choice[correct]{Washer Method}
  \end{multipleChoice}

How many integrals with respect to $y$ will we need to express the volume of the solid? $\answer{2}$


\begin{exercise}

A sum of integrals with respect to $y$ that expresses the volume of the solid of revolution is: 
\[
\int_{\answer{0}}^{\answer{1}} \answer{\pi \left( 1+\sqrt{1-y} \right)^{2} - \pi \left(1 -\sqrt{1-y} \right)^{2}} \d y
\]
\begin{hint}
Rewrite $y=-x^{2}+2x$ as $x^{2}-2x+y=0$, which is a quadratic polynomial in $x$. Then use the quadratic formula to find $x$ as a function of $y$. 
\end{hint}

\end{exercise}

To use the Shell Method to set up an integral or sum of integrals that would give the volume of the solid: 

  \begin{multipleChoice}
    \choice[correct]{we should integrate with respect to $x$.}
    \choice{we should integrate with respect to $y$.}
  \end{multipleChoice}

How many integrals will we need to express the volume of the solid using the Shell Method: $\answer{1}$. 


\begin{exercise} 
The integral that gives the volume is: 
\[
V=\int_{\answer{0}}^{\answer{2}} \answer{ 2\pi x \left(-x^{2}+2x \right)} \d x
\] 


\end{exercise}






\end{document}
