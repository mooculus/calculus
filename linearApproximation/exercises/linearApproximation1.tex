\documentclass{ximera}
\newcommand{\RR}{\mathbb R}
\renewcommand{\d}{\,d}
\newcommand{\dd}[2][]{\frac{d #1}{d #2}}
\renewcommand{\l}{\ell}
\newcommand{\ddx}{\frac{d}{dx}}
\newcommand{\dfn}{\textbf}
\newcommand{\eval}[1]{\bigg[ #1 \bigg]}

\author{Steven Gubkin\and Nela Lakos}
\license{Creative Commons 3.0 By-NC}

\outcome{Define linear approximation as an application of the tangent to a curve.}
\outcome{Find the linear approximation to a function at a point and use it to approximate the function value.}
\outcome{Identify when a linear approximation can be used.}
\begin{document}
\begin{exercise}

Approximate $\sqrt{9.5}$ by hand using the idea of linear approximation (of a conveniently chosen function, and base point).  Using the second derivative, explain whether you think this is an underestimate or an overestimate.
\begin {hint}

The number that you have to approximate should be expressed as

 $\sqrt{9.5}=f(x)$.
Obviously, $f(x)=\sqrt{x}$, and $x=9.5$.
\end {hint}
\begin {hint}

You will now approximate  $\sqrt{9.5}=f(x)$
 with $L (x)$, where $L$ is the linear approximation of $f$ at $a$.
In other words, $\sqrt{9.5}=f(x)\approx L(x)$.

 You have to choose the "base" number $a$  wisely, so that $x=9.5$ is near $a$, and so that you can easily and exactly compute the values $f(a)$ and $f'(a)$.
 These two values are needed when we compute the expression for $L(x)$. 
\end {hint}
\begin {hint}
The obvious choice for $a$ is $a=9$. Any other number for which it is easy to compute $\sqrt{a}$ is farther away from $x=9.5$.
Now we have to compute $f(a)=f(9)$ and $f'(a)=f'(9)$.
\begin{prompt}
	$$f(9)=  \answer{3}$$
	\end{prompt}
	\begin{prompt}
	$$f'(x)= \frac{1}{ \answer{2\sqrt{x}}}$$
	\end{prompt}
	\begin{prompt}
	$$f'(9)= \frac{1}{ \answer{6}}$$
	\end{prompt}
\end {hint}
\begin {hint}

Now, we have to find an expression for $L(x)=f(9)+f'(9)(x-9)$.
\begin{prompt}
	$$L(x)= \answer{3}+\answer{\frac{1}{6}}(x-9)$$
	\end{prompt}
 
\end {hint}
\begin {hint}

Now, we have to evaluate $L(x)=L(9.5)$.
\begin{prompt}
	$$L(9.5)= \answer{3}+\answer{\frac{1}{6}}(9.5-9)$$
	\end{prompt}
 
\end {hint}
\begin{prompt}
	$$\sqrt{9.5} \approx \answer{3+\frac{1}{12}}$$
Is this an underestimate or overestimate?
\begin{hint}
 \begin{image}
%\begin{marginfigure}
\begin{tikzpicture}
	\begin{axis}[
            xmin=0,xmax=25,ymin=0,ymax=6,
            axis lines=center,
            ticks=none,
            %width=3in,
            %height=2in,
            unit vector ratio*=1 1 1,
            xlabel=$x$, ylabel=$y$,
            every axis y label/.style={at=(current axis.above origin),anchor=south},
            every axis x label/.style={at=(current axis.right of origin),anchor=west},
          ]        
          \addplot [ thick, penColor, smooth, domain=(0:25)] {sqrt(x)};
          \addplot [thick, penColor2,smooth, domain=(0:25)] {3+(1/6)*(x-9)};
          \node at (axis cs:18,3.5) [penColor] {$y=f(x)$};
          \node at (axis cs:18.5,5.4) [penColor2] {$y=L(x)$}; 
          \node at (axis cs:9.1,4.5) [penColor2] {$(9,f(9))$}; 
            \addplot[color=penColor3,fill=penColor3,only marks,mark=*] coordinates{(9,3)};  %% closed hole         
        \end{axis}
\end{tikzpicture}
%\caption{A linear approximation of $f(x) = \sin(x)$ at $x=0$.}
%\label{figure:la sin}
%\end{marginfigure}
\end{image}
\end{hint}
\begin {hint}
Recall: $f''(x)=-\frac{1}{4\cdot x^{\answer{\frac{3}{2}}}}$. 

Since $f''(x)<0$ on $\left(9,\answer{9.5}\right)$,  the graph of $f$ is concave down. Therefore, the tangent line lies above the graph of $f$, except at $x=9$, where $f(9)=L(9)$.


This means that $L(x)\ge f(x)$. So, $L(9.5)\ge f(9.5)$.
\end{hint}
This is an
\begin{multipleChoice}
   \choice{underestimate}
   \choice[correct]{overestimate}
\end{multipleChoice}
\end{prompt}

\end{exercise}
\end{document}
