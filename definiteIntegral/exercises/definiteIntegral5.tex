\documentclass{ximera}

\newcommand{\RR}{\mathbb R}
\renewcommand{\d}{\,d}
\newcommand{\dd}[2][]{\frac{d #1}{d #2}}
\renewcommand{\l}{\ell}
\newcommand{\ddx}{\frac{d}{dx}}
\newcommand{\dfn}{\textbf}
\newcommand{\eval}[1]{\bigg[ #1 \bigg]}



%\outcome{Use integral notation for both antiderivatives and definite integrals.}
\outcome{Compute definite integrals using geometry.}
%\outcome{Compute definite integrals using the properties of integrals.}
%\outcome{Justify the properties of definite integrals using algebra or geometry.}
\outcome{Understand how Riemann sums are used to find exact area.}
%\outcome{Define net area.}
%\outcome{Approximate net area.}
%\outcome{Split the area under a curve into several pieces to aid with calculations.}
\outcome{Use symmetry to calculate definite integrals.}
\outcome{Explain geometrically why symmetry of a function simplifies calculation of some definite integrals.}

\author{Nela Lakos \and Kyle Parsons}

\begin{document}
\begin{exercise}

Evaluate the following limits of Riemann sums on the indicated intervals.

\begin{align*}
\text{On } [-2,20],\,\lim_{n\to\infty}\sum_{k=1}^n x^* \Delta x &= \answer{198}\\
\text{On } [-2,2],\,\lim_{n\to\infty}\sum_{k=1}^n 6(x^*)^3 \Delta x &= \answer{0}\\
\text{On } [-7,7],\,\lim_{n\to\infty}\sum_{k=1}^n 6\sin(x^*) \Delta x &= \answer{0}\\
\text{On } [-7,7],\,\lim_{n\to\infty}\sum_{k=1}^n 6(x^*)^2\sin(x^*) \Delta x &= \answer{0}\\
\text{On } [4,7],\,\lim_{n\to\infty}\sum_{k=1}^n \sqrt{9-(x^*-4)^2} \Delta x &= \answer{\frac{9\pi}{4}}
\end{align*}

\end{exercise}
\end{document}