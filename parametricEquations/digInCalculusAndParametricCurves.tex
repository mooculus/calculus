\documentclass{ximera}

\outcome{Find the derivative of a parametric curve.}
\outcome{Integrate a parametric curve that does not cross itself.}

\newcommand{\RR}{\mathbb R}
\renewcommand{\d}{\,d}
\newcommand{\dd}[2][]{\frac{d #1}{d #2}}
\renewcommand{\l}{\ell}
\newcommand{\ddx}{\frac{d}{dx}}
\newcommand{\dfn}{\textbf}
\newcommand{\eval}[1]{\bigg[ #1 \bigg]}


\title[Dig-In:]{Calculus and parametric curves}

\begin{document}
\begin{abstract}
  We discuss derivatives and integrals of parametric curves.  
\end{abstract}
\maketitle

\section{Derivatives}

Let's get right to it. Given a parametric function $(x(t),y(t))$ and
recalling that
\begin{align*}
  \d x &= x'(t) \d t\\
  \d y &= y'(t) \d t,
\end{align*}
we can see how to compute the derivative of $y$ with respect to $x$
using differentials:
\[
\frac{\d y}{\d x} = \frac{y'(t) \d t}{x'(t) \d t} = \frac{y'(t)}{x'(t)}
\]
provided that $x'(t) \ne 0$.

We use this to define the tangent line.

\begin{definition}
Let a curve $C$ be parametrized by $x(t)$ and $y(t)$, where $x$ and
$y$ are differentiable functions on some interval $I$ containing
$t=t_0$. The \dfn{tangent line} to $C$ at $t=t_0$ is the line through
\[
\big(x(t_0),y(t_0)\big)\text{ with slope }m=\frac{y'(t_0)}{x'(t_0)},
\]
provided $x'(t_0)\neq 0$.

The \dfn{normal line} to $C$ at $t=t_0$ is the line through 
\[
\big(x(t_0),y(t_0)\big)\text{ with slope }m=\frac{-x'(t_0)}{y'(t_0)},
\]
provided $y'(t_0)\neq 0$.  The normal line is \textbf{perpendicular}
to the tangent line.
\end{definition}

The definition leaves two special cases to consider. When the tangent
line is horizontal, the normal line is undefined by the above
definition as $y'(t_0)=0$. Likewise, when the normal line is
horizontal, the tangent line is undefined. It seems reasonable that
these lines be defined (one can draw a line tangent to the ``right
side'' of a circle, for instance), so we add the following to the
above definition.

\begin{itemize}
\item If the tangent line at $t=t_0$ has a slope of $0$, the normal
  line to $C$ at $t=t_0$ is the vertical line $x=x(t_0)$.
\item If the normal line at $t=t_0$ has a slope of $0$, the tangent
  line to $C$ at $t=t_0$ is the line $x=x(t_0)$.
\end{itemize}
Time for some examples. 

\begin{example}
  Let $x=5t^2-6t+4$ and $y=t^2+6t-1$, and let $C$ be the curve defined
  by these equations. Find the equations of the tangent and normal lines
  to $C$ at $t=3$.
  \begin{explanation}
    We start by computing
    \[
    x'(t) = \answer[given]{10t-6}
    \]
    and
    \[
    y'(t) =\answer[given]{2t+6}.
    \]
    Thus
    \[
    \dd[y]{x} = \answer[given]{\frac{2t+6}{10t-6}}.
    \]
    Make note of something that might seem unusual: $\dd[y]{x}$ is a
    function of $t$, not $x$. Just as points on the curve are found in
    terms of $t$, so are the slopes of the tangent lines.
		
    The point on $C$ at $t=3$ is $(31,26)$. The slope of the tangent
    line is $m=1/2$ and the slope of the normal line is $m=-2$. Thus,
    \begin{itemize}
    \item the equation of the tangent line is
      $y=\answer[given]{\frac{(x-31)}{2}+26}$, and
    \item the equation of the normal line is
      $y=\answer[given]{-2(x-31)+26}$.
    \end{itemize}
  \end{explanation}
\end{example}


\begin{example}
  Find where the circle, defined by $x=\cos t$ and $y=\sin t$ on
  $[0,2\pi]$, has vertical and horizontal tangent lines.
  \begin{explanation}
    We compute the derivative 
    \[
    \dd[y]{x} = \frac{y'(t)}{x'(t)} = \answer[given]{-\frac{\cos t}{\sin t}}.
    \]
    The derivative is $0$ when $\cos t= \answer[given]{0}$. This
    happens when $t=\pi/2$, and $t= 3\pi/2$. These are the points
    $(0,1)$ and $(0,-1)$ on the circle.

    The normal line is horizontal (and hence, the tangent line is
    vertical) when $\sin t=\answer[given]{0}$. This happens when $t=
    0$, $t=pi$, $t=2\pi$, corresponding to the points $(-1,0)$ and
    $(0,1)$ on the circle. These results should make intuitive sense.
  \end{explanation}
\end{example}





\section{Integrals}


Assuming that the curve given by a parametric formula $(x(t),y(t))$
represents $y$ as a function of $x$, that is traced out exactly once,
we can integrate our parametric formula without too much additional
trouble.  Again, recall that
\[
\d x = x'(t) \d t
\]
So we may write
\[
\int_a^b y \d x = \int_\alpha^\beta y(t) \cdot x'(t) \d t
\]
where $x(\alpha) = a$ and $x(\beta) = b$.

We'll be talking about this in more detail soon, so a simple example
should suffice.

\begin{example}
  Let $x(t) = \cos(t)$ and $y(t)= \sin(t)$ as $t$ runs from $0$ to
  $\pi$:
  \begin{image}
    \begin{tikzpicture}
      \begin{axis}[
          xmin=-1.2,xmax=1.2,ymin=-.2,ymax=1.2,
          axis lines=center,
          xlabel=$x$, ylabel=$y$,
          unit vector ratio*=1 1 1,
          every axis y label/.style={at=(current axis.above origin),anchor=south},
          every axis x label/.style={at=(current axis.right of origin),anchor=west},
        ]        
        \addplot [very thick, penColor, smooth, domain=(0:180)] ({cos(x)},{sin(x)});
      \end{axis}
    \end{tikzpicture}
  \end{image}
  Compute the area between this region and the $x$-axis on the
  interval $[-1,1]$.
  \begin{explanation}
    We would like to compute:
    \[
    \int_{-1}^1 y \d x
    \]
    The parametric equations allow us to make a substitution: 
    \begin{align*}
      y(t) &=\answer[given]{\sin(t)}\\
      \d x &=\answer[given]{-\sin(t)}\d t
    \end{align*}
    now then integral becomes
    \[
    \int_{-1}^{1} y \d x= \int_\alpha^\beta (\sin(t))(-\sin(t))\d t
    \]
    Now we need to find $\alpha$ and $\beta$. Note that using
    \[
    x(t)=\cos(t)
    \]
    When $x=-1$, $t=\answer[given]{\pi}$. Likewise, when $x=1$, $t=\answer[given]{0}$. Hence our
    integral becomes
    \begin{align*}
      \int_{-1}^{1} y \d x&= \int_\pi^0 (\sin(t))(-\sin(t))\d t\\
      &= \int_{\answer[given]{0}}^{\answer[given]{\pi}}\sin^2(t)\d t\\
      &= \int_0^\pi\frac{1-\cos(2t)}{2}\d t\\
      &= \int_0^\pi\frac{1}{2}-\frac{\cos(2t)}{2}\d t\\
      &= \eval{\frac{t}{2}-\frac{\sin(2t)}{4}}_0^\pi\\
      &= \answer[given]{\frac{\pi}{2}}
    \end{align*}
  \end{explanation}
\end{example}



\end{document}






