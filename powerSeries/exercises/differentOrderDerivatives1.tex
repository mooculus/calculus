\documentclass{ximera}

\newcommand{\RR}{\mathbb R}
\renewcommand{\d}{\,d}
\newcommand{\dd}[2][]{\frac{d #1}{d #2}}
\renewcommand{\l}{\ell}
\newcommand{\ddx}{\frac{d}{dx}}
\newcommand{\dfn}{\textbf}
\newcommand{\eval}[1]{\bigg[ #1 \bigg]}


\author{Jim Talamo}
\license{Creative Commons 3.0 By-bC}


\outcome{}


\begin{document}
\begin{exercise}
Suppose that $f(x) = \sum_{k=1}^{\infty} \frac{40}{k}x^{2k-2}$.  This exercise asks you to find several values of the function and its derivatives at the center of the series.

\begin{exercise}
Find $f(0)$.

\[
f(0) = \answer{40}
\]

\begin{hint}
Write out a few terms of $f(x)$ and evaluate the expression at $x=0$.
\end{hint}
\end{exercise}

%%%%%%%%%%%%%%%%%%%%

\begin{exercise}
Find $f''(0)$.

\[
f''(0) = \answer{40}
\]

\begin{hint}
Is it easier to compute the series represented by $f''(x)$ explicitly or use the relationship between the coefficients of the power series and the derivatives of the function $f(x)$?
\end{hint}
\end{exercise}

%%%%%%%%%%%%%%%%%%%%

\begin{exercise}
Find $f^{(54)}(0)$.

\[
f^{(54)}(0) = \answer{\frac{10}{7} \cdot 54!}
\]

\begin{hint}
Is it easier to compute the series represented by $f^{(54)}(x)$ explicitly or use the relationship between the coefficients of the power series and the derivatives of the function $f(x)$?

If you want to use the formula:

\[
a_n = \frac{f^{(n)}(c)}{n!}
\]
then $c=\answer{0}$ and $n=\answer{28}$. 
\end{hint}
\end{exercise}

What can you take away from this exercise?
\end{exercise}

\end{document}
