\documentclass{ximera}

\newcommand{\RR}{\mathbb R}
\renewcommand{\d}{\,d}
\newcommand{\dd}[2][]{\frac{d #1}{d #2}}
\renewcommand{\l}{\ell}
\newcommand{\ddx}{\frac{d}{dx}}
\newcommand{\dfn}{\textbf}
\newcommand{\eval}[1]{\bigg[ #1 \bigg]}


\author{Bart Snapp}

\begin{document}
Consider the following contour plot for a differentiable function $F :
\R^2 \to \R$.  The $x$-axis is horizontal and the $y$-axis is
vertical.  Assume nothing too complicated happens between the contour
lines.
\begin{image}[4in]
  \includegraphics{contourPlot.png}
\end{image}

\begin{problem}
List all \textbf{labeled points} where $F$ locally looks like a
saddle point.
\begin{prompt}
  \begin{selectAll}
    \choice[correct]{P}
    \choice{Q}
    \choice{R}
    \choice{S}
    \choice{T}
    \choice[correct]{U}
    \choice{V}
    \choice{W}
  \end{selectAll}
\end{prompt}
\vfill
\end{problem}

\begin{problem}
  At point $\mathsf{S}=(0,0)$, suppose you know that : $|\grad F|=2$,
  $F^{(2,0)} = -2$, $F^{(0,2)} = -1$, and $F^{(1,1)}= 3$. Write down
  the second degree Taylor polynomial for $F$ at $\mathsf{S}$.
  \begin{prompt}
    \[
    P_2(x,y) = \answer{7+2x -x^2 -y^2/2+3xy}
    \]
  \end{prompt}
  \vfill
\end{problem}


\begin{problem}
  At point $\mathsf{S}=(0,0)$, suppose you know that : $|\grad F|=2$.
  Let $\vec{p}(t)= \vector{2t^2-2,t^3-1}$. Compute:
  $\eval{\dd{t}F(\vec{p}(t))}_{t=1}$
  \begin{prompt}
    \[
    \eval{\dd{t}F(\vec{p}(t))}_{t=1} = \answer{8}
    \]
  \end{prompt}
  \vfill
\end{problem}




\begin{problem}
  At the point $\mathsf{T}$ suppose you know that: The point
  $\mathsf{T}$ is a local maximum for $F$, $\frac{\partial^2
    F}{\partial x^2} = -3$, $\frac{\partial^2 F}{\partial y^2} =
  -2$. What is the maximum value for $\left|\frac{\partial^2
    F}{\partial x\partial y}\right|$ at $\mathsf{T}$?
  \begin{prompt}
    \[
    \left|\frac{\partial^2 F}{\partial x\partial y}\right|\le \answer{\sqrt{6}}
    \]
  \end{prompt}
  \vfill
\end{problem}

\begin{problem}
  Let $G:\R^2\to\R$ be a differentiable function such that when $G$ is
  constrained to the curve $F(x,y)=9$, $G$ is maximized at
  $\mathsf{W}$. Assuming $|\grad{G}|=1$ at $\mathsf{W}$, give all
  possible vectors for $\grad{G}$ at point $\mathsf{W}$. \textbf{Give
    a brief explanation}.
  \begin{prompt}
    Writing you answer in the order of increasing $y$-components:
    \[
    \vector{\answer{0},\answer{-1}} \quad \vector{\answer{0},\answer{1}}
    \]
    \begin{feedback}[correct]
      By the method of Lagrange multipliers, $\grad G$ must be
      parallel to $\grad F$, which is perpendicular to the level curve
      at $\mathsf{W}$.
    \end{feedback}
  \end{prompt}
  \vfill
\end{problem}




\end{document}
