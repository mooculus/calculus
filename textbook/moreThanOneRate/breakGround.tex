\documentclass{ximera}

\newcommand{\RR}{\mathbb R}
\renewcommand{\d}{\,d}
\newcommand{\dd}[2][]{\frac{d #1}{d #2}}
\renewcommand{\l}{\ell}
\newcommand{\ddx}{\frac{d}{dx}}
\newcommand{\dfn}{\textbf}
\newcommand{\eval}[1]{\bigg[ #1 \bigg]}


\outcome{Solve related rates word problems.}
\outcome{Calculate derivatives of expressions with multiple variables implicitly.}


\title[Break-Ground:]{A changing circle}

\begin{document}
\begin{abstract}
Here we see a dialogue where two young mathematicians discuss a circle
that is changing.
\end{abstract}
\maketitle

Check out this dialogue between two calculus students (based on a true
story):

\begin{dialogue}
  \item[Devyn] Riley, I've been thinking about calculus.   
  \item[Riley] YOLO.
  \item[Devyn] Consider a circle of some radius $r$.
  \item[Riley] Ha! What else would we ever call the ``radius?''
  \item[Devyn] Exactly. Now the formula for the perimeter of
    a circle is?
  \item[Riley] $P=2\cdot\pi \cdot r$ baby.
  \item[Devyn] And its area?
  \item[Riley] You know it's $A=\pi\cdot r^2$.
  \item[Devyn] Right, but here's what's bugging me: If I know
    $r'$, what is $P'$? What's $A'$?
  \item[Riley] Oooh. Ouch. Hmmm. I wanna say it's
    \[
    P' = 2\cdot\pi \cdot r' \qquad\text{and}\qquad  A' = \pi (r')^2 
    \]
    but I'm not sure that is right.
  \item[Devyn] Yeah\dots me too. But I'm not sure that's
    right either. Are we forgetting something?
\end{dialogue}
  
\begin{problem}
  Do you think our young mathematicians above are correct?
  \begin{multipleChoice}
    \choice{Yes. $P' = 2\cdot\pi \cdot r'$ and $A' = \pi (r')^2$.}
    \choice[correct]{No. While  $P' = 2\cdot\pi \cdot r'$,  $A' \ne \pi (r')^2$.}
    \choice{No. While  $A' = \pi (r')^2$, $P' \ne 2\cdot\pi \cdot r'$.}
    \choice{No. $P' \ne 2\cdot\pi \cdot r'$ and  $A' \ne \pi (r')^2$.}
    \choice{There is no way to tell.}
  \end{multipleChoice}
\end{problem}

\begin{problem}
  Set $r(t)=3\cdot t$. What is $r'(t)$ when $r=15$?
  \begin{prompt}
    $r'(t)=\answer{3}$
  \end{prompt}
\end{problem}

\begin{problem}
  Set $r=3\cdot t$. Now $P(t) = 2\cdot \pi\cdot 3\cdot t$. What is
  $P'(t)$ when $r=15$?
  \begin{prompt}
    $P'(t)=\answer{2*pi*3}$
  \end{prompt}
\end{problem}

\begin{problem}
  Describe what $P'$ means in this context.
  \begin{freeResponse}
    It describes how fast the perimeter is changing at a particular
    instant in time.
  \end{freeResponse}
\end{problem}

\begin{problem}
  Set $r=3\cdot t$. Now $A(t) = \pi\cdot (3\cdot t)^2$. What is
  $A'(t)$ when $r=15$?
  \begin{prompt}
    $A'(t)=\answer{2*pi*15*3}$
  \end{prompt}
\end{problem}

\begin{problem}
  Describe what $A'$ means in this context.  Does it make sense that
  $A'$ is positive?
  \begin{freeResponse}
    It describes how fast the area is changing.  It does make sense that
    $A'$ is positive, since as the radius gets larger, the area should
    get larger, too.
  \end{freeResponse}
\end{problem}

\begin{problem}
  What, if anything, did our two young mathematicians forget about above?
  \begin{freeResponse}
 They forgot the chain rule.
  \end{freeResponse}
\end{problem}

\needed{../leveledQuestions.tex}

\end{document}
