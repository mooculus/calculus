\documentclass{ximera}

\newcommand{\RR}{\mathbb R}
\renewcommand{\d}{\,d}
\newcommand{\dd}[2][]{\frac{d #1}{d #2}}
\renewcommand{\l}{\ell}
\newcommand{\ddx}{\frac{d}{dx}}
\newcommand{\dfn}{\textbf}
\newcommand{\eval}[1]{\bigg[ #1 \bigg]}


\outcome{Consider values of a function at inputs approaching a given point.}
\outcome{Understand the concept of a limit.}
\outcome{Calculate limits from a graph (or state that the limit does not exist).}

\author{Nela Lakos \and Kyle Parsons}

\begin{document}
\begin{exercise}

The entire graph of a function $f$ is given below.

\begin{image}
  \begin{tikzpicture}
    \begin{axis}[
        xmin=-1.3,xmax=6.3,ymin=-4.3,ymax=8.3,
        clip=false,
        unit vector ratio*=1 1 1,
        axis lines=center,
        grid = major,
        xtick={-1,0,...,6},
    ytick={-4,-3,...,8},
        xlabel=$x$, ylabel=$y$,
        every axis y label/.style={at=(current axis.above origin),anchor=south},
        every axis x label/.style={at=(current axis.right of origin),anchor=west},
      ]
      \addplot[very thick,penColor,domain=0:2,samples=50] {-x^2+6};
      \addplot[very thick,penColor] plot coordinates {(2,0) (5,-3)};
       
      \addplot[color=penColor,fill=white,only marks,mark=*] coordinates{(2,0)};  %% open hole
      \addplot[color=penColor,fill=white,only marks,mark=*] coordinates{(2,2)};  %% open hole
       
      \addplot[color=penColor,fill=penColor,only marks,mark=*] coordinates{(0,6)};  %% closed hole
      \addplot[color=penColor,fill=penColor,only marks,mark=*] coordinates{(2,7)};  %% closed hole
      \addplot[color=penColor,fill=penColor,only marks,mark=*] coordinates{(5,-3)};  %% closed hole
       
      \node[penColor] at (axis cs:4, 4.5) [penColor] {$y=f(x)$};
      \end{axis}`
  \end{tikzpicture}
\end{image}

The slope of line  at the point $\left(4,-2\right)$ is $\answer{-1}$.\\

Find the following limits, if they exist.  If a limit does not exist, explain why.

\[
\lim_{x\to2^+}f(x) = \answer{0}
\]
\begin{multipleChoice}
\choice[correct]{The limit does exist.}
\choice{The limit does not exist because $f$ is not defined the same to the left and right of 2.}
\choice{The limit does not exist because $f(2)$ is very different from the values of $f$ near 2.}
\choice{The limit does not exist because $\lim_{x\to2^-}f(x) \neq \lim_{x\to2^+}f(x)$.}
\end{multipleChoice}

\noindent\rule[0.5ex]{\linewidth}{.2pt}

\[
\lim_{x\to2^-}f(x) = \answer{2}
\]
\begin{multipleChoice}
\choice[correct]{The limit does exist.}
\choice{The limit does not exist because $f$ is not defined the same to the left and right of 2.}
\choice{The limit does not exist because $f(2)$ is very different from the values of $f$ near 2.}
\choice{The limit does not exist because $\lim_{x\to2^-}f(x) \neq \lim_{x\to2^+}f(x)$.}
\end{multipleChoice}

\noindent\rule[0.5ex]{\linewidth}{.2pt}

\[
\lim_{x\to2}f(x) = \answer{DNE}
\]
\begin{multipleChoice}
\choice{The limit does exist.}
\choice{The limit does not exist because $f$ is not defined the same to the left and right of 2.}
\choice{The limit does not exist because $f(2)$ is very different from the values of $f$ near 2.}
\choice[correct]{The limit does not exist because $\lim_{x\to2^-}f(x) \neq \lim_{x\to2^+}f(x)$.}
\end{multipleChoice}

\noindent\rule[0.5ex]{\linewidth}{.2pt}

\[
\lim_{x\to4}f(x) = \answer{-2}
\]
\begin{multipleChoice}
\choice[correct]{The limit does exist.}
\choice{The limit does not exist because $f$ is flat near 4.}
\choice{The limit does not exist because $f(4)$ is very different from the values of $f$ near 4.}
\choice{The limit does not exist because $\lim_{x\to4^-}f(x) \neq \lim_{x\to4^+}f(x)$.}
\end{multipleChoice}

\end{exercise}
\end{document}