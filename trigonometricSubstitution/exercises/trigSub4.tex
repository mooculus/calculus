\documentclass{ximera}

\newcommand{\RR}{\mathbb R}
\renewcommand{\d}{\,d}
\newcommand{\dd}[2][]{\frac{d #1}{d #2}}
\renewcommand{\l}{\ell}
\newcommand{\ddx}{\frac{d}{dx}}
\newcommand{\dfn}{\textbf}
\newcommand{\eval}[1]{\bigg[ #1 \bigg]}


\author{Jason Miller}
\license{Creative Commons 3.0 By-NC}


\outcome{}


\begin{document}
\begin{exercise}
Determine the integral using trig substitution. 
\[
\int \frac{1}{\sqrt{1-x^{2}}} \d x
\]

We should use the trig substitution $x=\answer{ \sin(\theta)}$. 

Hence $\d x=\answer{ \cos(\theta) } \d \theta$.

Expressing our integral in terms of $\theta$ gives:

\begin{exercise}

\[
\int \frac{1}{\sqrt{1-x^{2}}} \d x=\int \answer{ 1  }   \d \theta 
\]


\begin{exercise}
The antiderivative in terms of $\theta$ is 
\[
\int \d \theta = \answer{ \theta + C}
\]
(use $C$ for the constant of integration)

\begin{exercise}
Switching back to the variable $x$, our original integral is:

\[
\int \frac{1}{\sqrt{1-x^{2}}} \d x= \answer{ \arcsin(x) + C }
\]
(Use $C$ for the constant of integration)

\begin{exercise}
Note that since:
\[
\ddx \arcsin(x)=\frac{1}{\sqrt{1-x^{2}}}
\]
we could have obtained this result by simply writing down the corresponding antidifferentiation formula.  Still, it does not hurt to emphasize no matter how we choose solve a problem (provided that we do it correctly), we will obtain the same result!

\end{exercise}
\end{exercise}
\end{exercise}
\end{exercise}
\end{exercise}
\end{document}
