\documentclass{ximera}
\newcommand{\RR}{\mathbb R}
\renewcommand{\d}{\,d}
\newcommand{\dd}[2][]{\frac{d #1}{d #2}}
\renewcommand{\l}{\ell}
\newcommand{\ddx}{\frac{d}{dx}}
\newcommand{\dfn}{\textbf}
\newcommand{\eval}[1]{\bigg[ #1 \bigg]}

\author{Steven Gubkin}
\license{Creative Commons 3.0 By-NC}

\outcome{Use the product and quotient rule to calculate derivatives from a table of values.}
\outcome{Use the product rule to calculate derivatives.}
\outcome{Use the quotient rule to calculate derivatives.}
\outcome{Combine derivative rules to take derivatives of more complicated functions.}

\begin{document}

\begin{exercise}

Let $Q(x) = \frac{A(x)}{B(x)}$

If you know that $A(-4) > 0$, $A'(-4) > 0$, $B(-4) > 0$, and $B'(-4) > 0$, what can you say about the sign of $Q'(-4)$?

\begin{multipleChoice}
\choice{$Q'(-4)>0$}
\choice{$Q'(-4)<0$}
\choice{$Q'(-4) = 0$}
\choice[correct]{We cannot determine the sign of $Q'(-4)$}
\end{multipleChoice}

Try to make sense of this by thinking about small increases/decreases in the quantities $A$ and $B$, in addition to symbolic calculation.

\end{exercise}

\end{document}
