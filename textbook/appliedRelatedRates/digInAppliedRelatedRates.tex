\documentclass{ximera}

\newcommand{\RR}{\mathbb R}
\renewcommand{\d}{\,d}
\newcommand{\dd}[2][]{\frac{d #1}{d #2}}
\renewcommand{\l}{\ell}
\newcommand{\ddx}{\frac{d}{dx}}
\newcommand{\dfn}{\textbf}
\newcommand{\eval}[1]{\bigg[ #1 \bigg]}


\title[Dig-In:]{Applied related rates}

\begin{document}
\begin{abstract}
\end{abstract}
\maketitle

WRITE SOMETHING


\begin{itemize}
\item[\textbf{Draw a picture.}] If possible, draw a schematic picture with all the relevant information. 
\item[\textbf{Find an equation.}] We want an equation that relates all relevant functions. 
\item[\textbf{Differentiate the equation.}] Here we will often use
  implicit differentiation.
\item[\textbf{Evaluate the equation at the desired values.} ] The known values
  should let you solve for the relevant rate.
\end{itemize}

Let's see a concrete example. 

\begin{example}
\label{exam:receding airplane}
A plane is flying directly away from you at $500$ mph at an altitude of
$3$ miles.  How fast is the plane's distance from you increasing at the
moment when the plane is flying over a point on the ground $4$ miles
from you?


\begin{explanation}
We'll use our general strategy to solve this problem. To start,
\textbf{draw a picture}.
\begin{image}
\begin{tikzpicture}
\draw[penColor2, dashed, very thick] (0,0) -- (5,4);
%\draw[penColor, dashed, very thick] (0,0) -- (0,4);
\draw[penColor, dashed, very thick] (5,0) -- (5,4);
\draw[penColor, dashed, very thick] (0,0) -- (5,0);
\draw[->,penColor, very thick] (1,4) -- (6,4);
\draw [penColor, fill] (5,4) circle [radius=.07];
\node [left,penColor] at (0,0) {\scalebox{3} \Ladiesroom};
\node [right,penColor] at (6,4) {\scalebox{3}{\ding{40}}};
\node [right,penColor] at (5,2) {$3$ miles};
\node [above,penColor] at (3,4) {$p'(t) = 500$ mph};
\node [above,penColor] at (5,4) {$p(t)$};
\node [below,penColor] at (2.5,0) {$4$ miles};
\node [left,penColor2] at (2.4,2) {$s(t)$ miles};
\end{tikzpicture}
\end{image}
Next we need to \textbf{find an equation}. By the Pythagorean Theorem
we know that
\[
p^2+3^2=s^2.
\] 
Now we \textbf{differentiate the equation}. Write
\[
2p(t)p'(t)  = 2s(t) s'(t).
\] 
Now we'll \textbf{evaluate the equation at the desired values}.  We
are interested in the time at which $p(t)=4$ and $p'(t) =
500$. Additionally, at this time we know that $4^2+9=s^2$, so
$s(t)=5$.  Putting together all the information we get
\[
2(4)(500)=2(5)s'(t),
\]
thus $s'(t)=400$ mph.
\end{explanation}
\end{example}





\begin{example} Water is poured into a conical container at the rate of 10
cm${}^3$/sec.  The cone points directly down, and it has a height of
30 cm and a base radius of 10 cm.  How fast is the water level rising
when the water is 4 cm deep?

\begin{explanation}
To start, \textbf{draw a picture}.
\begin{image}
\begin{tikzpicture}
\draw[penColor,very thick] (0,4) ellipse (4 and 1);
\draw[very thick,penColor!20!background] (2,2) arc (0:180:2 and .5);% top half of ellipse
\draw[very thick,penColor] (-2,2) arc (180:360:2 and .5);% bottom half of ellipse
\draw[penColor, very thick] (3.97,3.85) -- (0,0);
\draw[penColor, very thick] (-3.97,3.85) -- (0,0);
\draw[penColor, very thick] (0,4) -- (4,4);
\draw[penColor!50!background, very thick] (0,2) -- (2,2);
\draw[->,line width=0.4cm, penColor!20!background] (0,6) -- (0,4.25);
\draw[dashed, penColor2, very thick] (2.1,0) -- (2.1,2);
\draw[dashed, penColor, very thick] (-4.1,0) -- (-4.1,4);
\node[right, penColor] at (.4,5.6) {$\dd[V]{t} = 10$ cm$^3$/sec};
\node[below, penColor] at (2,4) {$10$ cm};
\node[above, penColor] at (1,2) {$r$ cm};
\node[right, penColor2] at (2.1,1) {$h(t) = 4$ cm};
\node[left, penColor] at (-4.1,2) {$30$ cm};
\end{tikzpicture}
\end{image}
Note, no attempt was made to draw this picture to scale, rather we
want all of the relevant information to be available to the
mathematician.

Now we need to \textbf{find an equation}. The formula for the volume of a cone tells us that 
\[
V = \frac{\pi}{3} r^2 h.
\]

Now we must \textbf{differentiate the equation}. We should use implicit differentiation, and treat each of the variables as functions of $t$. Write
\begin{equation}\label{equation:cone/water}
\dd[V]{t} = \frac{\pi}{3}\left(2rh \dd[r]{t} + r^2 \dd[h]{t}\right).
\end{equation}

At this point we \textbf{evaluate the equation at the desired values}.
At first something seems to be wrong, we do not know $\dd[r]{t}$.
However, the dimensions of the cone of water must have the same
proportions as those of the container.  That is, because of similar
triangles, 
\[
\frac{r}{h}=\frac{10}{30} \qquad\text{so}\qquad r={h/3}.
\]  
In particular, we see that when $h = 4$, $r=4/3$ and 
\[
\dd[r]{t} = \frac{1}{3}\cdot \dd[h]{t}.
\]
Now we can \textbf{evaluate the equation at the desired
  values}. Starting with Equation~\ref{equation:cone/water}, we plug
in $\dd[V]{t} = 10$, $r = 4/3$, $\dd[r]{t} = \frac{1}{3}\cdot \dd[h]{t}$
and $h=4$. Write
\begin{align*}
10 &= \frac{\pi}{3}\left(2\cdot \frac{4}{3}\cdot 4 \cdot\frac{1}{3}\cdot\dd[h]{t} + \left(\frac{4}{3}\right)^2 \dd[h]{t}\right)\\
10 &= \frac{\pi}{3}\left(\frac{32}{9}\dd[h]{t} + \frac{16}{9} \dd[h]{t}\right)\\
10 &= \frac{16\pi}{9}\dd[h]{t}\\
\frac{90}{16\pi} &= \dd[h]{t}.
\end{align*}
Thus, $\dd[h]{t}=\frac{90}{16\pi}$ cm/sec.
\end{explanation}
\end{example}


\begin{example}
A swing consists of a board at the end of a $10$ ft long rope.  Think
of the board as a point $P$ at the end of the rope, and let $Q$ be the
point of attachment at the other end.  Suppose that the swing is
directly below $Q$ at time $t=0$, and is being pushed by someone who
walks at 6 ft/sec from left to right.  What is the angular speed of
the rope in deg/sec after 1 sec?

\begin{explanation}
To start, \textbf{draw a picture}.
\begin{image}
\begin{tikzpicture}[scale=1.3]
\draw[penColor!50!background, very thick] (0,3) -- (0,-1);
\draw[penColor, very thick] (0,3) -- (2.12,-.12);
\draw [penColor!50!background, very thick] (-2.12,-.12) arc [radius=3, start angle=225, end angle= 315];
\draw [penColor2, very thick] (0,2.3) arc [radius=.7, start angle=270, end angle= 305];
\draw[->, penColor, very thick] (0,-.12) -- (2,-.12);

\node[penColor] at (2.12,-.12) {\scalebox{3} \Ladiesroom};
\node[penColor,right] at (2.3,-.12) {$P$};
\node[penColor,right,above] at (0,3) {$Q$};
\node[penColor,right] at (1.06,1.5) {$10$ ft};
\node[penColor,above] at (1,-.12) {$\dd[x]{t} = 6$ ft/sec};
\node[penColor2,right,above] at (.3,2) {$\theta$};
\end{tikzpicture}
\end{image}
Now we must \textbf{find an equation}. From the right triangle in our
picture, we see
\[
\sin(\theta)=x/10.
\]
We can now \textbf{differentiate the equation}. Taking derivatives we obtain 
\[
\cos(\theta)\cdot \theta'(t)=0.1 x'(t).
\]
Now we can \textbf{evaluate the equation at the desired values}.  When
$t=1$ sec, the person was pushed by someone who walks $6$
ft/sec. Hence we have a $6-8-10$ right triangle, with $x'(t) = 6$, and
$\cos\theta=8/10$. Thus
\[
(8/10) \theta'(t) =6/10,
\]
and so  $\theta'(t)=6/8=3/4$ rad/sec, or approximately $43$ deg/sec.
\end{explanation} 
\end{example}



We have seen that sometimes there are apparently more than two
variables that change with time, but as long as you know the rates of
change of all but one of them you can find the rate of change of the
remaining one.  As in the case when there are just two variables, take
the derivative of both sides of the equation relating all of the
variables, and then substitute all of the known values and solve for
the unknown rate.



\begin{example}
A road running north to south crosses a road going east to west at the
point $P$.  Cyclist $A$ is riding north along the first road, and cyclist $B$ is
riding east along the second road.  At a particular time, cyclist $A$ is $3$
kilometers to the north of $P$ and traveling at $20$ km/hr, while cyclist
$B$ is $4$ kilometers to the east of $P$ and traveling at $15$ km/hr.
How fast is the distance between the two cyclists changing?


\begin{explanation}
We start the same way we always do, we \textbf{draw a picture}.
\begin{image}
\begin{tikzpicture}
\draw[->,penColor!50!background, very thick] (-1,0) -- (4,0);
\draw[->,penColor!50!background, very thick] (0,-1) -- (0,4);
\draw[->,penColor, very thick] (0,3) -- (0,4);
\draw[->,penColor, very thick] (3,0) -- (4,0);
\draw [penColor, fill] (0,0) circle [radius=.07];
\draw [penColor, fill] (3,0) circle [radius=.07];
\draw [penColor, fill] (0,3) circle [radius=.07];
\draw[dashed,penColor2, very thick] (3,0) -- (0,3);

\node[penColor,rotate=90,right] at (.5,3) {\scalebox{-2} \Bicycle};
\node[penColor,right] at (0,.2) {$P$};
\node[penColor,left] at (-.3,3) {$a'(t) = 20$ km/hr};
\node[penColor,left] at (0,1.5) {$3$ km};
\node[penColor,below] at (1.5,0) {$4$ km};
\node[penColor,below] at (4,0) {$b'(t)= 15$ km/hr};
\node[penColor2,above] at (1.6,1.6) {$c(t)$};
\node[penColor,right,above] at (3.5,0) {\scalebox{-2}[2] \Bicycle};
\end{tikzpicture}
\end{image}
Here $a(t)$ is the distance of cyclist $A$ north of $P$ at time $t$,
and $b(t)$ the distance of cyclist $B$ east of $P$ at time $t$, and
$c(t)$ is the distance from cyclist $A$ to cyclist $B$ at time $t$.

We must \textbf{find an equation}.  By the Pythagorean Theorem,
\[
c(t)^2=a(t)^2+b(t)^2.
\] 
Now we can \textbf{differentiate the equation}. Taking derivatives we get 
\[
2c(t)c'(t)=2a(t)a'(t)+2b(t)b'(t).
\]
Now we can  \textbf{evaluate the equation at the desired values}.
We know that $a(t) = 3$, $a'(t) = 20$, $b(t) = 4$ and $b'(t) = 15$. Hence 
by the Pythagorean Theorem, $c(t) = 5$. So 
\[
2\cdot 5 \cdot c'(t) = 2 \cdot 3\cdot 20 + 2 \cdot 4 \cdot 15
\]
solving for $c'(t)$ we find $c'(t) = 24$ km/hr.
\end{explanation}
\end{example}
\end{document}
