\documentclass{ximera}
\newcommand{\RR}{\mathbb R}
\renewcommand{\d}{\,d}
\newcommand{\dd}[2][]{\frac{d #1}{d #2}}
\renewcommand{\l}{\ell}
\newcommand{\ddx}{\frac{d}{dx}}
\newcommand{\dfn}{\textbf}
\newcommand{\eval}[1]{\bigg[ #1 \bigg]}

\author{Bart Snapp}
\outcome{Compute Taylor polynomials of functions of several variables.}
\outcome{View the construction of a Taylor polynomial as an iterative process.}
\begin{document}
\begin{exercise}
  Compute the degree $2$ Taylor polynomial for
  \[
  F(x,y) = e^x \cdot e^y
  \]
  centered at $(1/2,-1/2)$. Start by making a table of partial
  derivatives along with their value when evaluated at $(1/2,-1/2)$:
  \[
  \begin{array}{lcl}
    F(x,y) = e^x\cdot e^y & \Rightarrow &F(0,0) = \answer{1}\\
    F^{(1,0)}(x,y) = \answer{e^x e^y} & \Rightarrow & F^{(1,0)}(0,0) = \answer{1}\\
    F^{(0,1)}(x,y) = \answer{e^x e^y} &\Rightarrow  & F^{(0,1)}(0,0) = \answer{1}\\
    F^{(2,0)}(x,y) = \answer{e^x e^y} &\Rightarrow & F^{(2,0)}(0,0) = \answer{1}\\
    F^{(0,2)}(x,y) = \answer{e^x e^y} &\Rightarrow & F^{(0,2)}(0,0) = \answer{1}\\
    F^{(1,1)}(x,y) = \answer{e^x e^y} &\Rightarrow & F^{(1,1)}(0,0) = \answer{1}
    \end{array}
    \]
    The degree $2$ Taylor polynomial is:
    \[
    P_2(x,y) =\answer{1+x+y+(x+y)^2/2}
    \]
\end{exercise}
\end{document}
