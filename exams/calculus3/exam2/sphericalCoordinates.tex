\documentclass{ximera}

\newcommand{\RR}{\mathbb R}
\renewcommand{\d}{\,d}
\newcommand{\dd}[2][]{\frac{d #1}{d #2}}
\renewcommand{\l}{\ell}
\newcommand{\ddx}{\frac{d}{dx}}
\newcommand{\dfn}{\textbf}
\newcommand{\eval}[1]{\bigg[ #1 \bigg]}


\author{Jim Fowler}
\license{Creative Commons 4.0 By-SA}

\begin{document}

\begin{exercise}
  
  Let $D$ be the solid unit sphere.  Show that
  \[
    \iiint_D x^2 \, dV = \answer{\frac{4}{15} \pi}.
  \]
  and that
  \[
    \iiint_D x^3 \, dV = \answer{0}
  \]
  and that
  \[
    \iiint_D x^4 \, dV = \answer{\frac{4}{35} \pi}.
  \]

  \begin{hint}
By symmetry, $\iiint_D x^3 \, dV = 0$.
\end{hint}

\begin{hint}
We use spherical coordinates.  By symmetry, we can simplify things a bit by replacing $x$ with $z$.  In what follows, $n = 2$ or $n = 4$.
 \begin{align*}
   \iiint_D z^n \, dV
 &= \int_{\theta=0}^{2\pi} \int_{\varphi=0}^{\pi} \int_{r = 0}^1 (r \cos \varphi)^n \, r^2 \sin \varphi \, dr \, d\varphi \, d\theta \\
 &= \int_{\theta=0}^{2\pi} \int_{\varphi=0}^{\pi} \int_{r = 0}^1 r^{n + 2} \cos^n \varphi \sin \varphi \, dr \, d\varphi \, d\theta \\
 &= \left( \int_{\theta=0}^{2\pi} d\theta \right) \left(\int_{\varphi=0}^{\pi} \cos^n \varphi \sin \varphi \, d\varphi \right) \left( \int_{r = 0}^1 r^{n+2} \, dr \right) \\
 &= 2 \pi \left(\int_{\varphi=-1}^{1} u^n \, du \right) \frac{1}{n+3} \\
 &= 2 \pi \frac{1 - (-1)^{n+1}}{n+1} \cdot \frac{1}{n+3} \\
 &= \frac{4 \pi}{(n+1)(n+3)}.
 \end{align*}
\end{hint}

\begin{hint}
And therefore $\iiint_D x^2 \, dV = \frac{4}{15} \, \pi$.
\end{hint}
\begin{hint}
  And moreover $\iiint_D x^4 \, dV = \frac{4}{35} \, \pi$.
\end{hint}

\end{exercise}
\end{document}
