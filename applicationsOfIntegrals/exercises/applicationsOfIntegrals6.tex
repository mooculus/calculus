\documentclass{ximera}

\newcommand{\RR}{\mathbb R}
\renewcommand{\d}{\,d}
\newcommand{\dd}[2][]{\frac{d #1}{d #2}}
\renewcommand{\l}{\ell}
\newcommand{\ddx}{\frac{d}{dx}}
\newcommand{\dfn}{\textbf}
\newcommand{\eval}[1]{\bigg[ #1 \bigg]}


%\outcome{Given a velocity function, calculate displacement and distance traveled.}
%\outcome{Given a velocity function, find the position function.}
%\outcome{Given an acceleration function, find the velocity function.}
%\outcome{Understand the difference between displacement and distance traveled.}
%\outcome{Understand the relationship between position, velocity and acceleration.}

\author{Nela Lakos \and Kyle Parsons}

\begin{document}
\begin{exercise}

Consider a particle moving along a straight line.  The figure below gives the velocity function of the particle.  Assume that at $t=0$ the particle is at position 0.

\begin{image}
  \begin{tikzpicture}
    \begin{axis}[
        xmin=-0.3,xmax=8.3,ymin=-1.3,ymax=1.3,
        clip=true,
        unit vector ratio*=1 1 1,
        axis lines=center,
        grid = major,
        ytick={-1,0,...,36},
        xtick={0,1,...,10},
        xlabel=$t$, ylabel=$v$,
        every axis y label/.style={at=(current axis.above origin),anchor=south},
        every axis x label/.style={at=(current axis.right of origin),anchor=west},
      ]
      \addplot[ultra thick,penColor,domain=0:2,samples=2] {x/2};    
      \addplot[ultra thick,penColor,domain=2:6,samples=2] {2-x/2};
      \addplot[ultra thick,penColor,domain=6:8,samples=2] {-1};
        
      \node at (axis cs:6.5,0.5) {$v=v(t)$};
      \end{axis}`
  \end{tikzpicture}
\end{image}

The displacement of the particle between $t=0$ and $t=8$ is $\answer{-1}$.

The average velocity of the particle on the interval $[0,8]$ is $\answer{-\frac{1}{8}}$.

The total \emph{distance} traveled between $t=0$ and $t=8$ is $\answer{5}$.

The position of the particle at $t=3$ is $\answer{\frac{7}{4}}$.

The position of the particle at $t=5$ is $\answer{\frac{7}{4}}$.

The position function for the particle on the interval $[6,8]$ is
\[
s(t) = \answer{1-t}.
\]

The acceleration function for the particle on the interval $[6,8]$ is
\[
a(t) = \answer{0}.
\]

\end{exercise}
\end{document}