\documentclass{ximera}
\newcommand{\RR}{\mathbb R}
\renewcommand{\d}{\,d}
\newcommand{\dd}[2][]{\frac{d #1}{d #2}}
\renewcommand{\l}{\ell}
\newcommand{\ddx}{\frac{d}{dx}}
\newcommand{\dfn}{\textbf}
\newcommand{\eval}[1]{\bigg[ #1 \bigg]}


\title[Dig-In:]{A menagerie of strange limits}
\begin{document}
\begin{abstract}
  Train your intuition against some strange functions
\end{abstract}
\maketitle

In this section, we test the boundaries of the definition of a limit by looking at some ``strange'' functions.  A nice side effect will be to expand your mind on what a function is.

\begin{question}
	Consider the function $f(x) = \sin(\frac{\pi}{x})$
	
		Fill out the following table of values for $f$:
			\begin{tableanswer}
 x & f(x) \\ \hline
 1 &  \answer{0} \\
 \frac{1}{2}&  \answer{0} \\
\frac{1}{10} &  \answer{0)}\\
 \frac{1}{100} &   \answer{0} \\
 0 & \text{undefined}\\
 -\frac{1}{200} &  \answer{0}
 -\frac{1}{50}&   \answer{0}\\
 -\frac{1}{3}&   \answer{0}\\
\end{tableanswer}

If you only had access to the table above, you would probably think that $\lim_{x \to 0} f(x) = $\answer{0}.

However, consider the following table:

			\begin{tableanswer}
 x & f(x) \\ \hline
 1 &  \answer{1} \\
 \frac{1}{2}&  \answer{1} \\
\frac{1}{10} &  \answer{1)}\\
 \frac{1}{100} &   \answer{1} \\
 0 & \text{undefined}\\
 -\frac{1}{200} &  \answer{1}
 -\frac{1}{50}&   \answer{1}\\
 -\frac{1}{3}&   \answer{1}\\
\end{tableanswer}

This table would imply that $\lim_{x \to 0} f(x) = $\answer{1}.

Thus  $\lim_{x \to 0} f(x)$ does not exist.

You may be able to better understand this example by looking at the graph of $f$:

%graph
\end{question}

Moral of the story:  to inspect $\lim_{x \to a} f(x)$ it is \textbf{not sufficient} to look at a table of values of $f$. 

Here is another example which demonstrates the same moral:

\begin{question}
	Define the function $g(x)$ by $g(x) = 1$ if $x$ is rational and $g(x) = 0$ if $x$ is irrational.  So $g(\frac{3}{7}) =$ \answer{1} and $g(\sqrt(2)) =$ \answer{0}.
	This function does not have a limit \textbf{anywhere} since any real number $a$ has both rational and irrational numbers  as close to $a$ as you like.
	
	It is very difficult to graph the function $g$, but a sketch which gives some idea of its behavior is provided below:
	
	%graph
\end{question}

\begin{question}
We can modify the last example a bit to produce another strange function:

Define $h(x)$ by $h(x) = 2$ if $x$ is rational and $h(x) = x$ if $x$ is irrational.  First make a ``sketch'' of the graph of $h$ similar to the graph of $g$ above.

Looking at this graph, it seems like $\lim_{x \to a} $ exists for one and only one value of $a$.  This $a$ is $a = $\answer{2}  
\end{question}


\end{document}