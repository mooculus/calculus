\documentclass{ximera}

\newcommand{\RR}{\mathbb R}
\renewcommand{\d}{\,d}
\newcommand{\dd}[2][]{\frac{d #1}{d #2}}
\renewcommand{\l}{\ell}
\newcommand{\ddx}{\frac{d}{dx}}
\newcommand{\dfn}{\textbf}
\newcommand{\eval}[1]{\bigg[ #1 \bigg]}


\outcome{Learn something.}


\title[Break-Ground:]{Derivatives of products are tricky}

\begin{document}
\begin{abstract}
Here we see a dialogue where young mathematicians talk about
derivatives of products and products of derivatives.
\end{abstract}
\maketitle

Check out this dialogue between two calculus students (based on a true
story):

\begin{dialogue}
\item[Devyn] Hey Riley, remember the sum rule for derivatives?
\item[Riley] You know I do.
\item[Devyn] What do you think that the ``product rule'' will be?
\item[Riley] Let's give this a spin:
  \[
  \dd{x} \left(f(x)\cdot g(x)\right) = f'(x) \cdot g'(x)?
  \]
\item[Devyn] Hmmm, let's give this theory an acid test. Set
  \[
  f(x) = x^2+1\qquad\text{and}\qquad g(x) = x^3-3x
  \]
  Now
  \begin{align*}
    f'(x)g'(x) &= (2x)(3x^2-3)\\
    &= 6x^3-6x.
  \end{align*}
\item[Riley] On the other hand,
  \begin{align*}
    f(x)g(x) &= (x^2+1)(x^3-3x)\\
    &=x^5-3x^3+x^3-3x\\
    &=x^5-2x^3-3x.
  \end{align*} 
\item[Devyn] And so, 
  \[
  \ddx \left(f(x) \cdot g(x)\right) = 5x^4-6x^2-3.
  \]
\item[Riley] Wow. Hmmm. It looks like our guess was incorrect.
\item[Devyn] I've got a feeling that the so-called ``product rule''
  might be a bit tricky.
\end{dialogue}

\begin{problem}
  Above, our intrepid young mathematicians guess that the ``product rule'' might be:
  \[
  \dd{x} \left(f(x)\cdot g(x)\right) = f'(x) \cdot g'(x)?
  \]
  Does this \textbf{ever} hold true?
  \begin{freeResponse}
  \end{freeResponse}
\end{problem}

\begin{problem}
  repeated sum rule
\end{problem}

%% \begin{xarmaBoost}
%%   Write down at least \textbf{five} questions for this lecture. After
%%   you have your questions, label them as ``Level 1,'' ``Level 2,'' or
%%   ``Level 3'' where:
%% \begin{description}
%% \item[Level 1] Means you know the answer, or know exactly how to do
%%   this problem.
%% \item[Level 2] Means you think you know how to do the problem.
%% \item[Level 3] Means you have no idea how to do the problem.
%% \end{description}
%% \begin{freeResponse}
%% \end{freeResponse}
%% \end{xarmaBoost}



\end{document}
