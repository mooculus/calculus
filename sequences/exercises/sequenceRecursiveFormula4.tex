\documentclass{ximera}

\newcommand{\RR}{\mathbb R}
\renewcommand{\d}{\,d}
\newcommand{\dd}[2][]{\frac{d #1}{d #2}}
\renewcommand{\l}{\ell}
\newcommand{\ddx}{\frac{d}{dx}}
\newcommand{\dfn}{\textbf}
\newcommand{\eval}[1]{\bigg[ #1 \bigg]}


\author{Jim Talamo}
\license{Creative Commons 3.0 By-bC}


\outcome{}


\begin{document}
\begin{exercise}

  Consider the sequence $a_{n}$ defined recursively by the
  rule
  \[
  a_n = {a_{n-1}} {a_{n-2}} + 3 \, {a_{n-1}} - {a_{n-2}}
  \]
  and the facts that $a_0 = -3$ and $a_1 = 5$.  What is $a_4$?
  
  
      \[
      a_4 = \answer[given]{111}.
      \]
  
%    \begin{hint}
%      We have been told the first two terms of the sequence, namely $a_0 = -3$ and $a_1 = 5$.
%    \end{hint}
%    \begin{hint}
%      We also have a rule $a_n = {a_{n-1}} {a_{n-2}} + 3 \, {a_{n-1}} - {a_{n-2}}$ which lets us compute the third term $a_2$ using these first two terms $a_0$ and $a_1$.
%    \end{hint}
%    \begin{hint}
%      To compute $a_2$, we set $n = 2$ in the recursive rule $a_n = {a_{n-1}} {a_{n-2}} + 3 \, {a_{n-1}} - {a_{n-2}}$.  This gives us $a_2 = {a_{2-1}} {a_{2-2}} + 3 \, {a_{2-1}} - {a_{2-2}}$.
%    \end{hint}
%    \begin{hint}
%      Plugging $a_{2-1} = a_{1} = 5$ and $a_{2-2} = a_{0} = -3$ into the rule, we learn $a_2 = 3$.
%    \end{hint}
%    \begin{hint}
%      To compute $a_3$, we set $n = 3$ in the recursive rule $a_n = {a_{n-1}} {a_{n-2}} + 3 \, {a_{n-1}} - {a_{n-2}}$, giving $a_3 = {a_{3-1}} {a_{3-2}} + 3 \, {a_{3-1}} - {a_{3-2}}$.
%    \end{hint}
%    \begin{hint}
%      To evaluate that, we will have to know $a_{3-1} = a_{2}$, but we just found out that $a_{2} = 3$.
%    \end{hint}
%    \begin{hint}
%      Plugging $a_{3-1} = 3$ and $a_{3-2} = a_{1} = 5$ into the rule, we find $a_3 = 19$.
%    \end{hint}
%    \begin{hint}
%      To compute $a_4$, we set $n = 4$ in the recursive rule $a_n = {a_{n-1}} {a_{n-2}} + 3 \, {a_{n-1}} - {a_{n-2}}$, giving $a_4 = {a_{4-1}} {a_{4-2}} + 3 \, {a_{4-1}} - {a_{4-2}}$.
%    \end{hint}
%    \begin{hint}
%      To evaluate that, we will have to know $a_{4-1} = a_{3}$, but we just found out that $a_{3} = 19$.
%    \end{hint}
%    \begin{hint}
%      Plugging $a_{4-1} = 19$ and $a_{4-2} = a_{2} = 3$ into the rule, we learn $a_4 = 111$.
%    \end{hint}
%    \begin{hint}
%      So we conclude $a_4 = 111$.
%    \end{hint}



\end{exercise}
\end{document}
