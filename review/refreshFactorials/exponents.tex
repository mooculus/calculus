\documentclass{ximera}

\newcommand{\RR}{\mathbb R}
\renewcommand{\d}{\,d}
\newcommand{\dd}[2][]{\frac{d #1}{d #2}}
\renewcommand{\l}{\ell}
\newcommand{\ddx}{\frac{d}{dx}}
\newcommand{\dfn}{\textbf}
\newcommand{\eval}[1]{\bigg[ #1 \bigg]}


\title[Refresh:]{Exponentials}
\author{Jim Talamo}
\begin{document}
\begin{abstract}
  Remember our facts about exponentials.
\end{abstract}
\maketitle


\begin{exercise}
Ratios of exponential terms simplify nicely. In order to simplify expressions with exponentials, it is helpful to remember the laws of exponents.

\underline{Laws of Exponents}
\begin{itemize}
\item $a^{n+m} = a^n \cdot a^m$
\item $a^{n-m} = \frac{a^n}{a^m}$
\item $\left(a^n\right)^m = a^{n \cdot m}$
\item $(a \cdot b)^n = a^n \cdot b^n$
\item $\left(\frac{a}{b}\right)^n = \frac{a^n}{b^n}$
\end{itemize}

Let's see these in action.

\begin{example}
Simplify $\frac{9^n\cdot 2^{n+3}}{2^{n+1} \cdot 3^{2n}}$.

\begin{explanation}
We use the laws of exponents above.

\[
\frac{9^n\cdot 2^{n+3}}{2^{n+1} \cdot 3^{2n}} = \frac{9^n \cdot 2^n \cdot 2^3}{2^n \cdot 2^1 \cdot \left(3^2\right)^n}= \frac{\cancel{9^n} \cdot \cancel{2^n} \cdot 2^3}{\cancel{2^n} \cdot 2^1 \cdot \cancel{9^n} } =4
\]
\end{explanation}
\end{example}

\begin{example}
Suppose that $a_k = 5\left(\frac{1}{2}\right)^{2k}$.  Simplify $\frac{a_{k+1}}{a_k}$.

\begin{explanation}
We write out terms.  Notice that 

\[
a_{k+1} = 5\left(\frac{1}{2}\right)^{2(k+1)} = 5\left(\frac{1}{2}\right)^{2k+2}.
\]

Thus, we can write out the desired ration and simplify.

\[
\frac{a_{k+1}}{a_k} = \frac{5\left(\frac{1}{2}\right)^{2k+2}}{5\left(\frac{1}{2}\right)^{2k}}= \frac{\cancel{5\left(\frac{1}{2}\right)^{2k}}\cdot \left(\frac{1}{2}\right)^2}{\cancel{5\left(\frac{1}{2}\right)^{2k}}} = \frac{1}{4}
\]
\end{explanation}
\end{example}

Now, try some examples.

\begin{problem}
Suppose $a_k=2^k$
  and $b_k=3^{2k}$. Simplify the following.
  \[
  \frac{a_{k+1}}{a_k} = \answer{2}
  \]
  \[
  \frac{b_{k+1}}{b_k}=\answer{9}
  \]
\end{problem}


\begin{problem}
  Suppose $a_k$ is a sequence whose $k$th term is given by:
  \[
  a_k=4^{k^2}
  \]
  Note: The $k^2$ is in the exponent. Simplify
  \[
  \frac{a_{k+1}}{a_k} = \answer{4^{2k+1}}
  \]
\end{problem}
%
%\section{Factorials}
%
%\begin{problem}
%  Given an integer $n$, the notation ``$n!$'' is defined as follows:
%  \[
%  n! = n(n-1)(n-2)\dots (3)(2)(1)
%  \]
%  For instance, $3!=3\cdot2\cdot 1 = 6$. Compute:
%  \begin{align*}
%    4! &= \answer{24}\\
%    6! &= \answer{720}\\
%    \frac{6!}{4!} &= \answer{30}
%  \end{align*}
%  \begin{problem}
%    How did you compute $\frac{6!}{4!}$ in the last problem? You could
%    certainly find $6!$ and $4!$ separately, then divide, but there is
%    a nicer way to do this:
%    \begin{align*}
%      \frac{6!}{4!} &= \frac{6\cdot5\cdot4\cdot3\cdot2\cdot1}{4\cdot3\cdot2\cdot1}\\
%      &=\frac{6\cdot5\cdot\not{4}\cdot\not{3}\cdot\not{2}\cdot\not{1}}{\not{4}\cdot\not{3}\cdot\not{2}\cdot\not{1}}\\
%      &= 6\cdot 5.
%    \end{align*}
%    Ratios of factorials are always easiest to compute by canceling
%    like terms! Compute the following, simplify your final answers.
%    \begin{align*}
%      \frac{5!}{4!} &= \answer{5}\\
%      \frac{6!}{3!\cdot 3!} &= \answer{20}\\
%      \frac{k!}{(k+2)!}  &=\answer{\frac{1}{(k+2)(k+1)}}
%    \end{align*}
%  \end{problem}
%\end{problem}
%
%\begin{problem}
%  Often, factorials arise in sequences, and it is important to
%  understand how to write out various terms. Suppose $a_k$ is a
%  sequence whose $k$th term is given by:
%  \[
%  a_k = (2k+1)!
%  \]
%  \begin{enumerate}
%  \item $a_2 = \answer{120}$
%  \item Find an expression for $a_{k+1}$. Express your answer in the form $(ck+d)!$.
%    \[
%    a_{k+1} = \left(\answer{2} \cdot k + \answer{3}\right)!
%    \]
%  \item Calculate and simplify:
%    \[
%    \frac{a_{k+1}}{a_k} = \answer{(2k+3)(2k+2)}
%    \]
%  \end{enumerate}
%\end{problem}
%
%\begin{problem}
%  Let $a_k$ be a sequence whose $k$th term is given by:
%  \[
%  a_k = \frac{(2k)!}{(k!)^2}
%  \]
%  Calculate and simplify:
%  \begin{align*}
%    \frac{a_{k+1}}{a_k} &= \answer{\frac{4k+2}{k+1}}\\
%    \lim_{k\to\infty} \frac{a_{k+1}}{a_k} &= \answer{4}
%  \end{align*}
%\end{problem}

\end{exercise}

\end{document}
