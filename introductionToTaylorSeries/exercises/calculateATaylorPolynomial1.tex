\documentclass{ximera}

\newcommand{\RR}{\mathbb R}
\renewcommand{\d}{\,d}
\newcommand{\dd}[2][]{\frac{d #1}{d #2}}
\renewcommand{\l}{\ell}
\newcommand{\ddx}{\frac{d}{dx}}
\newcommand{\dfn}{\textbf}
\newcommand{\eval}[1]{\bigg[ #1 \bigg]}


\author{Jim Talamo}
\license{Creative Commons 3.0 By-bC}


\outcome{}


\begin{document}
\begin{exercise}
For the function $f(x) = \frac{3x}{1+2x^2}$, find the seventh degree Taylor polynomial centered at $x=0$.

The seventh degree Taylor polynomial is $p_7(x) = \answer{ 3x-6x^3+12x^5-24x^7 }$.

\begin{hint}
We can construct this polynomial by using the Taylor Series centered at $x=0$ for a known function and the rules for sums, products, and compositions.

Which function below has a Taylor Series that would be helpful to start?
\begin{multipleChoice}
\choice{$\sin(x)$}
\choice{$\cos(x)$}
\choice{$e^x$}
\choice[correct]{$\frac{1}{1-x}$}
\end{multipleChoice}

\begin{question}
Write out several terms in the series for $\frac{1}{1-x}$:

\[
\frac{1}{1-x} = \answer{1} + \answer{1}x+\answer{1}x^2+\answer{1}x^3+\answer{1}x^4 + \ldots
\]
(We may not have exhibited enough terms, but we can always exhibit more if the above is not sufficient.)

\begin{question}
Use this as well as the rules for products and compositions:

First, note that:

\[
\frac{1}{1+2x^2} = \frac{1}{1-\answer{-2x^2}}
\]

We can thus use the rules for composition to find the first several terms in the series for this function:

\begin{align*}
\frac{1}{1-x} &= 1+x+x^2+x^3 + \ldots \\
 \frac{1}{1-\answer{-2x^2}} &= 1+(\answer{-2x^2})+(\answer{-2x^2})^2+(\answer{-2x^2})^3 +(\answer{-2x^2})^4 + \ldots \\
&= \answer{1-2x^2+4x^4-8x^6+16x^8} + \ldots \textrm{ (simplify your answer from the line above)} \\
\end{align*}

\begin{question}
Now, note that:

\[
\frac{3x}{1+2x^2} = \answer{3x} \cdot \frac{1}{1-(-2x^2)}
\]

We can thus use the rules for products to find the first several terms in the series for this function:
\begin{align*}
3x \cdot \frac{1}{1+2x^2} &= 3x \cdot (  \answer{1-2x^2+4x^4-8x^6+16x^8} +\ldots)    \\
&= \answer{3x-6x^3+12x^5-24x^7+48x^9} + \ldots \\
\end{align*}
(simplify your answer from the line above)

From this, you can extract the seventh degree Taylor polynomial.

\end{question}
\end{question}
\end{question}

\end{hint}

\end{exercise}
\end{document}
