\documentclass[]{ximera}
%handout:  for handout version with no solutions or instructor notes
%handout,instructornotes:  for instructor version with just problems and notes, no solutions
%noinstructornotes:  shows only problem and solutions

%% handout
%% space
%% newpage
%% numbers
%% nooutcomes

%I added the commands here so that I would't have to keep looking them up
%\newcommand{\RR}{\mathbb R}
%\renewcommand{\d}{\,d}
%\newcommand{\dd}[2][]{\frac{d #1}{d #2}}
%\renewcommand{\l}{\ell}
%\newcommand{\ddx}{\frac{d}{dx}}
%\everymath{\displaystyle}
%\newcommand{\dfn}{\textbf}
%\newcommand{\eval}[1]{\bigg[ #1 \bigg]}

%\begin{image}
%\includegraphics[trim= 170 420 250 180]{Figure1.pdf}
%\end{image}

%add a ``.'' below when used in a specific directory.

\newcommand{\RR}{\mathbb R}
\renewcommand{\d}{\,d}
\newcommand{\dd}[2][]{\frac{d #1}{d #2}}
\renewcommand{\l}{\ell}
\newcommand{\ddx}{\frac{d}{dx}}
\newcommand{\dfn}{\textbf}
\newcommand{\eval}[1]{\bigg[ #1 \bigg]}




\author{Jim Talamo}

\outcome{Use definite integrals to mass of a wire of variable density.}
\outcome{Use definite integrals to compute work done by a variable force.}
\outcome{Use definite integrals to solve lifting problems.}
\outcome{Use definite integrals to model an unfamiliar physical situation.}

\title[]{Physical Applications of Integration}

\begin{document}
\begin{abstract}
\end{abstract}
\maketitle

\vspace{-0.9in}

\section{Discussion Questions}

\begin{problem} The force $F$ (in Newtons) required to stretch a spring $x$ meters from equilibrium obeys Hooke's Law, which states that $F(x) = k x$, where $k$ (in $N$/$m$) is a constant.

Suppose that $k=4 N/m$ and the spring is stretched $5 m$ from equilibrium.  A student claims that since the force required to stretch the spring this distance is $F(5) = 4(5) = 20 N$, so the work required to move it these $5 m$ can be found as follows.
\[
W = F \cdot d = 20 \cdot 5 = 100 J
\]

Determine if the student is correct or incorrect.  If the student is incorrect, provide a correct solution.

\end{problem}

\begin{freeResponse}

\end{freeResponse}

\begin{problem}
Newton's Law of Gravitation states that the gravitational force between two particles is directly proportional to the product of their masses and inversely proportional to the square of the distance between their centers:

\[
F(r) = \frac{GMm}{r^2},
\]
where $r$ is the distance between the centers of the particles, $M$ and $m$ are the masses of the particles, and $G=6.674 \times 10^{-11} \unit{N \cdot m^2 /kg^2}$ is the constant of proportionality.

Suppose that a rocket ship is launched from the surface of the earth into space.  Write down an integral that calculates the work necessary to launch the rocket ship from the surface of the earth to just beyond the Earth's atmosphere.

You may assume that the Earth can be treated as a particle, the radius of the Earth is $6371000 \unit{m}$,  the mass of the Earth is $5.972 \times 10^{24} \unit{kg}$, the mass of the rocket ship is $505,846 \unit{kg}$, and the distance between the surface of the earth and the edge of the atmosphere${^1}$\footnote{$^1$This is often referred to as the \emph{K{\'a}rm{\'a}n line} and commonly represents the boundary between the Earth's atmosphere and outer space.} is $100,000 \unit{m}$.
\end{problem}


\begin{freeResponse}

\end{freeResponse}
%%%%%%%%%%%%%%%%%%%%%%%%%%%%%%%%

\section{Group Work}
\begin{problem}

Suppose that the density profile of a wire that extends from $x=0$ to $x=10$ is given by $\rho(x) =2x$.

\begin{itemize}
\item[I.] Does the half of the rod from $x=0$ to $x=5$  have less, equal, or more mass than the half of the rod from $x=5$ to $x=10$?
\item[II.] Suppose that a value for $a$ so the portion of the rod from $x=0$ to $x=a$ has the same mass as the portion of the rod from $x=a$ to $x=10$.  
\item[A.] Without performing any calculations, do you expect $a<5$, $a=5$, or $a>5$?
\item[B.] Find $a$.  Does the result agree with your answer to Part A?
\end{itemize}

\end{problem}




\begin{freeResponse} 

\end{freeResponse}


\begin{problem}
The force required to displace a particle $x$ units is given by

\[
F(x) = \left\{ \begin{array}{ll} 8, &0\leq x \leq 2 \\  2+3x,  &2  < x \leq 8  \end{array} \right.
\]
Find the work required to move that particle from $x=0$ to $x=6$.
\end{problem}

\begin{freeResponse}

\end{freeResponse}

%%%%%%%%%%%%%%%%%%%%%%%%%%%%%%%%%

\begin{problem}

An inverted conical tank has base with diameter $6\unit{m}$ and height
$9\unit{m}$. Suppose the tank is filled to a height of $6\unit{m}$
with acetone ($\rho=785 \unit{kg}/\unit{m}^3$). Set up, but do not evaluate, an integral that would give the work required
to pump the acetone out of the tank (use $g=9.8\unit{m}/\unit{s}^2$).

\resizebox {7cm} {!} { \begin{tikzpicture}
\begin{axis}[
domain=-3:3,
xmin=-3.5, xmax=4.5,
xtick={-3,-2,-1,1,2,3},
ymin=-1, ymax=11,
axis lines =center,
xlabel=$x$, ylabel=$y$, every axis y label/.style={at=(current axis.above origin),anchor=south},
every axis x label/.style={at=(current axis.right of origin),anchor=west},
axis on top,
]

\draw[penColor,very thick,smooth,fill=fill4] (axis cs: 0,9) ellipse (300 and 10);
\draw[penColor,very thick,smooth,fill=blue,opacity=0.25] (axis cs: 0,6) ellipse (200 and 6.67);
\draw[penColor,very thick,smooth] (axis cs:2,6) arc (360:180:200 and 6.67);
\draw[penColor,very thick,dashed] (axis cs:2,6) arc (0:180:200 and 6.67);
\addplot [penColor,very thick,smooth,domain=0:3]	{3*x};
\addplot [penColor,very thick,smooth,domain=-3:0]	{-3*x};

\addplot [name path=A,domain=0:3,draw=none] {3*x};   
\draw [name path=B,draw=none] (axis cs: 3,9) arc (0:180:300 and 10);
\addplot [fill4,opacity=0.5] fill between[of=A and B];

\addplot [name path=C,domain=-3:0.01,draw=none] {-3*x};   
\draw [name path=D,draw=none] (axis cs: -3,9) arc (180:360:300 and 10);
\addplot [fill4,opacity=0.5] fill between[of=C and D];

\addplot [name path=E,domain=-3:0.01,draw=none] {-3*x};   
\draw [name path=F,draw=none]  (axis cs:2,6) arc (0:180:200 and 6.67);
\addplot [blue,opacity=0.25] fill between[of=E and F];

\draw[decoration={brace,raise=.1cm},decorate,thin] (axis cs:2,6) -- (axis cs:2,0);
\node[anchor=west] at (axis cs:2.2,3) {$6\unit{m}$};
\draw[decoration={brace,raise=.1cm},decorate,thin] (axis cs:3,9) -- (axis cs:3,0);
\node[anchor=west] at (axis cs:3.2,4.5) {$9\unit{m}$};
\addplot[thick] plot coordinates {(0,6) (2,6)};
         
\end{axis}
\end{tikzpicture}}


%\[
%W= \int_{y=\answer{0}}^{y=\answer{6}} \answer{7693\pi\left(\frac{y}{3}\right)^2 (9-y)}\d y = \answer[tolerance=10]{276948\pi} \unit{J}
%\]
%
%\begin{hint}
%Set the height $y=0$ at the base of the tank.  We want to use the formula:
%
%\[ 
%W = \int_{y=0}^{y=b} \rho g A(y) d(y) \d y
%\]
%
%Since $b$ is the height to which the tank is filled, $b=\answer{6}$.
%
%Since $h$ is the height to which the water must be moved, $h=\answer{9}$.
%
%You want to move a slice at height $y$ to a height of $9$. Letting
%$d(y)$ represent the distance that a slice at height $y$ travels to
%get to a height of $9$, we set $d(y) = \answer{9-y}$.
%
%The cross-sectional area of this tank is:
%
%\begin{multipleChoice}
%\choice{is constant.}
%\choice[correct]{varies depending on the height $y$ of the slice}.
%\end{multipleChoice}
%
%\begin{question}
%We can find the radius of the slice in terms of $y$ by using similar triangles:
%
%\begin{image}
%\begin{tikzpicture}
%\begin{axis}[
%domain=-3:3,
%xmin=-3.5, xmax=4.5,
%xtick={-3,-2,-1,1,2,3},
%ymin=-1, ymax=11,
%axis lines =center,
%xlabel=$x$, ylabel=$y$, every axis y label/.style={at=(current axis.above origin),anchor=south},
%every axis x label/.style={at=(current axis.right of origin),anchor=west},
%axis on top,
%]
%
%%top ellipse
%\draw[penColor,very thick,smooth,fill=fill4] (axis cs: 0,9) ellipse (300 and 10);
%\draw[penColor,very thick,smooth,fill=blue,opacity=0.25] (axis cs: 0,6) ellipse (200 and 6.67);
%
%%middle
%\draw[penColor,very thick,smooth] (axis cs:2,6) arc (360:180:200 and 6.67);
%\draw[penColor,very thick,dashed] (axis cs:2,6) arc (0:180:200 and 6.67);
%
%%slice
%\draw[penColor2,very thick,smooth] (axis cs:1,3) arc (360:180:98 and 4);
%\draw[penColor2,very thick,dashed] (axis cs:1,3) arc (0:180:98 and 4);
%
%\addplot [penColor,very thick,smooth,domain=0:3]	{3*x};
%\addplot [penColor,very thick,smooth,domain=-3:0]	{-3*x};
%
%\addplot [name path=A,domain=0:3,draw=none] {3*x};   
%\draw [name path=B,draw=none] (axis cs: 3,9) arc (0:180:300 and 10);
%\addplot [fill4,opacity=0.5] fill between[of=A and B];
%
%\addplot [name path=C,domain=-3:0.01,draw=none] {-3*x};   
%\draw [name path=D,draw=none] (axis cs: -3,9) arc (180:360:300 and 10);
%\addplot [fill4,opacity=0.5] fill between[of=C and D];
%
%\addplot [name path=E,domain=-3:0.01,draw=none] {-3*x};   
%\draw [name path=F,draw=none]  (axis cs:2,6) arc (0:180:200 and 6.67);
%\addplot [blue,opacity=0.25] fill between[of=E and F];
%
%\draw[decoration={brace,raise=.1cm},decorate,thin,penColor2] (axis cs:1,3) -- (axis cs:1,0);
%\node[anchor=west,penColor2] at (axis cs:1.2,1.5) {$y$};
%\node[anchor=west,penColor2] at (axis cs:.3,3.8) {$x$};
%\node[anchor=west] at (axis cs:1,9.5) {$3$m};
%\draw[decoration={brace,raise=.1cm},decorate,thin] (axis cs:3,9) -- (axis cs:3,0);
%\node[anchor=west] at (axis cs:3.2,4.5) {$9$m};
%
%%horizontal lines
%\addplot[thick,penColor2] plot coordinates {(0,3) (1,3)};
%\addplot[thick] plot coordinates {(0,9) (3,9)};
%
%         
%\end{axis}
%\end{tikzpicture}
%\end{image}
%
%From similar triangles, we find $\frac{x}{y} = \answer{\frac{3}{9}}$, so $x= \answer{\frac{1}{3}y}$.
%
%\begin{question}
%The area is $A = \pi r^2$.  We see that $r$ is a horizontal distance, which we must express in terms of $y$.  Hence, $r=  \answer{\frac{1}{3} y}$ and $A=\answer{\pi \left(\frac{y}{3}\right)^2}$.
%
%Now, substitute all of these relevant quantities into the integral.
%\end{question}
%\end{question}
%
%\end{hint}
%
%\end{exercise}

\end{problem}

\begin{freeResponse}

\end{freeResponse}


\end{document}
