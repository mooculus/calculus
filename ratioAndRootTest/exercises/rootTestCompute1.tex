\documentclass{ximera}

\newcommand{\RR}{\mathbb R}
\renewcommand{\d}{\,d}
\newcommand{\dd}[2][]{\frac{d #1}{d #2}}
\renewcommand{\l}{\ell}
\newcommand{\ddx}{\frac{d}{dx}}
\newcommand{\dfn}{\textbf}
\newcommand{\eval}[1]{\bigg[ #1 \bigg]}


\author{Jim Talamo}
\license{Creative Commons 3.0 By-bC}


\outcome{}


\begin{document}
\begin{exercise}
Consider the series $\sum_{k=1}^{\infty} \left(\frac{2}{k}\right)^k$.

\begin{multipleChoice}
\choice[correct]{The Root Test applies to this series.}
\choice{The Root Test does not apply to this series.}
\end{multipleChoice}

Why is the Root Test preferable to the Ratio Test here?
\begin{multipleChoice}
\choice[correct]{The Root Test will allow for a more efficient solution since the summand involves the $k$-th power of an expression.}
\choice{The assumptions for the Root Test are not met.}
\end{multipleChoice}

To use the Root Test, note that the sequence $\{a_n\}_{n=1}$ is given by the rule:

\[
a_n = \answer{\left(\frac{2}{n}\right)^n}
\]

Thus, to use the Root Test, we must compute the limit:

\[
L = \lim_{n \to \infty} \sqrt[n]{\answer{\left(\frac{2}{n}\right)^n}} 
\]

Evaluating this limit, we find:

\[
L = \answer{0}
\]

\begin{hint}
Simplifying: $ \sqrt[n]{\left(\frac{2}{n}\right)^n} = \answer{ \frac{2}{n}}$
\end{hint}

Hence, the Root Test:
\begin{multipleChoice}
\choice[correct]{guarantees that the series converges.}
\choice{guarantees that the series converges to $0$.}
\choice{guarantees that the series diverges.}
\choice{is inconclusive.}
\end{multipleChoice}

\end{exercise}
\end{document}
