\documentclass{ximera}

\newcommand{\RR}{\mathbb R}
\renewcommand{\d}{\,d}
\newcommand{\dd}[2][]{\frac{d #1}{d #2}}
\renewcommand{\l}{\ell}
\newcommand{\ddx}{\frac{d}{dx}}
\newcommand{\dfn}{\textbf}
\newcommand{\eval}[1]{\bigg[ #1 \bigg]}

%\newcommand{\RR}{\mathbb R}
\renewcommand{\d}{\,d}
\newcommand{\dd}[2][]{\frac{d #1}{d #2}}
\renewcommand{\l}{\ell}
\newcommand{\ddx}{\frac{d}{dx}}
\newcommand{\dfn}{\textbf}
\newcommand{\eval}[1]{\bigg[ #1 \bigg]}


\author{}
\license{Creative Commons 3.0 By-NC}
\outcome{}
\begin{document}

\begin{exercise}

   A student is asked to compute $\int x \sqrt{1 - x^2} \d x$ on a midterm and provides the following solution:
   \[
     \int x \sqrt{1 - x^2} \d x = \int x(1 - x) \d x = \int (x - x^2) \d x = \frac{1}{2}x^2 - \frac{1}{3} x^3 + C.
   \]
   
   \begin{problem}
   Determine if the student is correct and choose the most appropriate response below.
   \begin{multipleChoice}
     \choice{The student is correct.}
     \choice{The student is incorrect: $\int x(1 - x) \d x = \frac{1}{2}x^2\left( x - \frac{1}{2}x^2 \right) + C$.}
     \choice[correct]{The student is incorrect: $\sqrt{1 - x^2} \ne 1 - x$.}
   \end{multipleChoice}
   \end{problem}
   
   \begin{problem}
     Compute the antiderivative $\int x \sqrt{1 - x^2} \d x$ by performing a
     $u$-substitution with $u = \answer{1 - x^2}$.  Upon making this
     substitution, the correct antiderivative of $x \sqrt{1 - x^2}$ is
     $\answer{\frac{-1}{3} (1 - x^2)^{3/2}} + C$.
   \end{problem}

\end{exercise}
\end{document}