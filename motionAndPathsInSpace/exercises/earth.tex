\documentclass{ximera}

\newcommand{\RR}{\mathbb R}
\renewcommand{\d}{\,d}
\newcommand{\dd}[2][]{\frac{d #1}{d #2}}
\renewcommand{\l}{\ell}
\newcommand{\ddx}{\frac{d}{dx}}
\newcommand{\dfn}{\textbf}
\newcommand{\eval}[1]{\bigg[ #1 \bigg]}


\author{Bart Snapp}

\outcome{Reparametrize a curve.}
\outcome{Parameterize a curve in terms of arc length.}

\begin{document}
\begin{exercise}
  The Earth travels in an orbit around the Sun that can be
  approximated by a circle. The distance from the Earth to the Sun is
  (on average) $1$
  \link[au]{https://en.wikipedia.org/wiki/Astronomical_unit}. We are
  going to make some models of the Earth's orbit in the
  $(x,y)$-plane. For all of our models, we will make the following
  assumptions:
  \begin{itemize}
  \item The Sun will be at the origin.
  \item At the starting time, $t=0$, the Earth will be at the point
    $(1,0)$ in the $(x,y)$-plane.
  \item The Earth will travel in a counterclockwise direction around
    the Sun.
  \end{itemize}
  \begin{exercise}
    Give a parametrization of the Earth's orbit that will model the
    Earth's position in terms of $t$, where the units are years.
    \[
    \vec{y}(t) = \vector{\answer{\cos(2\pi t)},\answer{\sin(2\pi t)}} 
    \]
  \end{exercise}
  \begin{exercise}
    Give a parametrization of the Earth's orbit that will model the
    Earth's position in terms of $t$, where the units are months.
    \[
    \vec{m}(t) = \vector{\answer{\cos(2\pi t/12)},\answer{\sin(2\pi t/12)}} 
    \]
  \end{exercise}
  \begin{exercise}
    Give a parametrization of the Earth's orbit that will model the
    Earth's position in terms of $t$, where the units are days.
    \[
    \vec{d}(t) = \vector{\answer{\cos(2\pi t/365)},\answer{\sin(2\pi t/365)}} 
    \]
  \end{exercise}
  \begin{exercise}
    Given that the Earth's orbit has a circumference of $940$ million
    miles, give a parametrization of the Earth's orbit that will model
    the Earth's position in terms of $s$, the distance traveled in
    millions of miles.
    \[
    \vec{e}(s) = \vector{\answer{\cos(2\pi s/940)},\answer{\sin(2\pi s/940)}} 
    \]
    \begin{exercise}
 %The exercise below is incorrect, and while I have left it untouched here, I've modified it in 1172, where it goes under the name eartthE.  First, there is a difference between "distance as a parameter" and "arclength as a parameter" (aside from how would one define the former without the latter).  Also, as stated, the exercise below contradicts the definition of the arclength parameter as in the text since  \int_0^s |\vec{p}'(t)| \d t \neq s, which is equivalent (under the requirement that s(0)=0) to |\vec{p}'(t)|=1 for all t.
 
 %The content/lesson that units matter is a very good one and should require students to think about the crux of this exercise (how do units affect the essence of what we mean by arclength) in more detail.   For non-math STEM majors, choosing units at the start of the problem is essential (here, it's been chosen to work in au), and only after choosing units does ``arclength parameterization'' have any meaning.
 
      True or False: This is an arc length parameterization.
    \begin{multipleChoice}
      \choice[correct]{True}
      \choice{False}
    \end{multipleChoice}
    \begin{exercise}
      Compute:
      \[
      \vec{e}'(s)
      \begin{prompt}
        = \vector{\answer{(-\pi/470) \sin(2\pi s/940)},\answer{(\pi/470) \cos(2\pi s/940)}}
      \end{prompt}
      \]
      \begin{exercise}
        Compute:
        \[
        |\vec{e}'(s)| = \answer{\pi/470}
        \]
        \begin{feedback}
          Note this will not be $1$!
        \end{feedback}
        \begin{exercise}
          Why is the magnitude of the derivative not $1$?
          \begin{multipleChoice}
            \choice{We made a mistake.}
            \choice[correct]{The units are the key.}
          \end{multipleChoice}
          \begin{feedback}
            Note, $1\unit{au} =
            150\unit{million~miles}$. Multiplying $\pi/470\cdot 150
            \approx 1$.
          \end{feedback}
        \end{exercise}
      \end{exercise}
    \end{exercise}
    \end{exercise}
  \end{exercise}
\end{exercise}
\end{document}
