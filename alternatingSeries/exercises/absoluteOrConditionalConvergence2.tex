\documentclass{ximera}

\newcommand{\RR}{\mathbb R}
\renewcommand{\d}{\,d}
\newcommand{\dd}[2][]{\frac{d #1}{d #2}}
\renewcommand{\l}{\ell}
\newcommand{\ddx}{\frac{d}{dx}}
\newcommand{\dfn}{\textbf}
\newcommand{\eval}[1]{\bigg[ #1 \bigg]}


\author{Jim Talamo}
\license{Creative Commons 3.0 By-bC}


\outcome{}


\begin{document}
\begin{exercise}

Determine whether the series:

\[
\sum_{k=1}^{\infty} \frac{(-1)^k \cdot(3k+1)}{2k^2+3k+1}
\]

converges absolutely, converges conditionally, or diverges by completing the exercise that follows.

We begin by checking for:

\begin{multipleChoice}
\choice[correct]{absolute convergence.  If the series converges, we still need to determine if $\sum_{k=1}^{\infty} |a_k|$ converges in order to classify whether the convergence is absolute or conditional.}
\choice{conditional convergence.  The Alternating Series Test is easy to apply here.}
\end{multipleChoice}

Thus, we will check whether the series:

\[
\sum_{k=1}^{\infty} \left| \frac{(-1)^k \cdot(3k+1)}{2k^2+3k+1} \right| = \sum_{k=1}^{\infty} \frac{3k+1}{2k^2+3k+1}
\]

converges.  Now that the summand is positive, we can use more tests!  Since the summand is a rational expression in $k$, the test to use is the:
\begin{multipleChoice}
\choice{Ratio Test}
\choice{Root Test}
\choice{Comparison Test}
\choice[correct]{Limit Comparison Test}
\end{multipleChoice}

\begin{exercise}
Which of the following would be a good series to use as a point of comparison?

\begin{multipleChoice}
\choice[correct]{$\sum_{k=1}^{\infty} \frac{1}{k}$}
\choice{$\sum_{k=1}^{\infty} \frac{1}{k^2}$}
\choice{$\sum_{k=1}^{\infty} \frac{1}{k^3}$}
\choice{$\sum_{k=1}^{\infty} \left(\frac{1}{2}\right)^k$}
\end{multipleChoice}

So, let $b_n = \frac{1}{n}$.  Then, $\lim_{n \to \infty} \frac{a_n}{b_n} = \answer{\frac{3}{2}}$.  So, the Limit Comparison Test guarantees that both series either converge or both series diverge.

The chosen series is:
\begin{multipleChoice}
\choice{is a geometric series with $|r|<1$.  It converges.}
\choice{is a geometric series with $|r|\geq1$.  It diverges.}
\choice{is a $p$-series with $p>1$.  It converges.}
\choice[correct]{is a $p$-series with $p \leq 1$.  It diverges.}
\end{multipleChoice}

Hence:

\begin{multipleChoice}
\choice{$\sum_{k=1}^{\infty} \frac{1}{k^2+5k+7}$ converges.}
\choice[correct]{$\sum_{k=1}^{\infty} \frac{1}{k^2+5k+7}$ diverges.}
\end{multipleChoice}

So, the original series $\sum_{k=1}^{\infty} \frac{\sin(k)}{k^2}$:
\begin{multipleChoice}
\choice[correct]{does not converge absolutely.}
\choice{diverges.}
\end{multipleChoice}

\begin{exercise}
Now, let's apply the Alternating Series Test.  Indeed the series is alternating since we can write it in the form $\sum_{k=1}^{\infty} (-1)^k a_k$, where $a_k = \answer{\frac{3k+1}{2k^2+3k+1}}$. Also $a_n$ is decreasing for large $n$ and $\lim_{n \to \infty} a_n = \answer{0}$, so:

\begin{multipleChoice}
\choice[correct]{the series converges by the Alternating Series Test.}
\choice{the series diverges by the Alternating Series Test.}
\choice{the Alternating Series Test is inconclusive.}
\end{multipleChoice}

Hence, the series $\sum_{k=1}^{\infty} \frac{(-1)^k \cdot(3k+1)}{2k^2+3k+1}$:

\begin{multipleChoice}
\choice{converges absolutely}
\choice{converges conditionally}
\choice{diverges}
\end{multipleChoice}

\begin{hint}
Remember that we say a series $\sum_{k=1}^{\infty} a_k$:

\begin{itemize}
\item \emph{converges absolutely} if $\sum_{k=1}^{\infty} |a_k|$ converges.
\item  \emph{converges conditionally} if $\sum_{k=1}^{\infty} a_k$ converges but $\sum_{k=1}^{\infty} |a_k|$ diverges.
\end{itemize}

\end{hint}

\end{exercise}

\end{exercise}
\end{exercise}
\end{document}
