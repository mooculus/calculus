\documentclass{ximera}

\newcommand{\RR}{\mathbb R}
\renewcommand{\d}{\,d}
\newcommand{\dd}[2][]{\frac{d #1}{d #2}}
\renewcommand{\l}{\ell}
\newcommand{\ddx}{\frac{d}{dx}}
\newcommand{\dfn}{\textbf}
\newcommand{\eval}[1]{\bigg[ #1 \bigg]}


\author{Jim Talamo}
\license{Creative Commons 3.0 By-bC}


\outcome{}


\begin{document}
\begin{exercise}
Consider the series $\sum_{k=1}^{\infty} \frac{k^2 3^k}{k!}$.

\begin{multipleChoice}
\choice[correct]{The Ratio Test applies to this series.}
\choice{The Ratio Test does not apply to this series.}
\end{multipleChoice}

Why is the Ratio Test preferable to the Root Test here?
\begin{multipleChoice}
\choice[correct]{The Root Test would require that we evaluate $\lim_{n \to \infty} \sqrt[n]{n!}$}
\choice{The assumptions for the Root Test are not met.}
\end{multipleChoice}

To use the Ratio Test, note that the sequence $\{a_n\}_{n=1}$ is given by the rule:

\[
a_n = \answer{\frac{n^2 3^n}{n!}}
\]

Thus, to use the Ratio Test, we must compute the limit:

\[
L = \lim_{n \to \infty} \answer{\frac{(n+1)^2 3^{n+1}}{(n+1)!}\frac{n!}{n^2 3^n}}
\]

Evaluating this limit, we find:

\[
L = \answer{0}
\]
Hence, the Ratio Test:
\begin{multipleChoice}
\choice[correct]{guarantees that the series converges.}
\choice{guarantees that the series converges to $0$.}
\choice{guarantees that the series diverges.}
\choice{is inconclusive.}
\end{multipleChoice}

\end{exercise}
\end{document}
