\documentclass{ximera}

\newcommand{\RR}{\mathbb R}
\renewcommand{\d}{\,d}
\newcommand{\dd}[2][]{\frac{d #1}{d #2}}
\renewcommand{\l}{\ell}
\newcommand{\ddx}{\frac{d}{dx}}
\newcommand{\dfn}{\textbf}
\newcommand{\eval}[1]{\bigg[ #1 \bigg]}


\outcome{Recall how to find limits for forms that are not indeterminate.}
\outcome{Define an indeterminate form.}
\outcome{Determine if a form is indeterminate.}
\outcome{Convert indeterminate forms to the form zero over zero or infinity over infinity.}
\outcome{Define l’Hôpital’s Rule and identify when it can be used.}
\outcome{Use l’Hôpital’s Rule to find limits.}

\author{Nela Lakos \and Kyle Parsons}

\begin{document}
\begin{exercise}

Consider
\[
\lim_{x\to0^+} (\tan(x))^{\sin(x)}.
\]

Select the form of the limit.
\begin{multipleChoice}
\choice[correct]{$\zeroToZero$}
\choice{$\oneToInfty$}
\choice{$\inftyToZero$}
\end{multipleChoice}

Now
\[
\lim_{x\to0^+} (\tan(x))^{\sin(x)} = e^{\lim_{x\to0^+}\answer{\sin(x) \ln(\tan(x))}}.
\]

The value of the limit in the exponential above is $\answer{0}$, so
\[
\lim_{x\to0^+} (\tan(x))^{\sin(x)} = e^{\answer{0}} = \answer{1}.
\]

\end{exercise}
\end{document}