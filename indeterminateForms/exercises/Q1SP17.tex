\documentclass{ximera}

\newcommand{\RR}{\mathbb R}
\renewcommand{\d}{\,d}
\newcommand{\dd}[2][]{\frac{d #1}{d #2}}
\renewcommand{\l}{\ell}
\newcommand{\ddx}{\frac{d}{dx}}
\newcommand{\dfn}{\textbf}
\newcommand{\eval}[1]{\bigg[ #1 \bigg]}


\outcome{Understand what is meant by the form of a limit.}
\outcome{Determining the form of a limit.}
\outcome{Calculate limits of the form zero over zero.}

\begin{document}

\begin{exercise}
Let $g(x)=|x-1|$ and let $h(x)=\sqrt{x}-1$. 

\begin{exercise}
State the form of the limit.
\[
\lim_{x\to 1^{+}}\frac{h(x)}{g(x)}
\]
The limit  is of the form \wordChoice{\choice{nonzero over zero} \choice[correct]{zero over zero}}.
\begin{exercise}
Multiply the numerator and denominator of that fraction by the conjugate of $h(x)$. Then for all $x>1$,  
\[
\frac{h(x)}{g(x)}=\frac{1}{\answer{\sqrt{x}+1}}.
\]
\begin{exercise}
Use this ``simplified'' formula for $\dfrac{h(x)}{g(x)}$ to evaluate the limit:
\[
\lim_{x\to 1^+}\frac{h(x)}{g(x)}=\answer{\frac{1}{2}}.
\]
\end{exercise}
\end{exercise}
\end{exercise}
\end{exercise}
\end{document}
