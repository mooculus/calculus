\documentclass{ximera}

\newcommand{\RR}{\mathbb R}
\renewcommand{\d}{\,d}
\newcommand{\dd}[2][]{\frac{d #1}{d #2}}
\renewcommand{\l}{\ell}
\newcommand{\ddx}{\frac{d}{dx}}
\newcommand{\dfn}{\textbf}
\newcommand{\eval}[1]{\bigg[ #1 \bigg]}


\author{Jim Fowler}
\license{Creative Commons 4.0 By-SA}

\begin{document}

\begin{exercise}
  
Suppose  $F : \R^2 \to \R$  is a function with 
$\grad F \dotp \vector{ 1, 0 } > 0$  and
$\grad F \dotp \vector{ 0, 1 } < 0$.

Suppose  $G : \R^2 \to \R$  is a function with  
$\grad G \dotp \vector{ 1, 0 } > 0$  and
$\grad G \dotp \vector{ 0, 1 } > 0$.


Suppose $H : \R^2 \to \R$  is a function with      
$\grad H \dotp \vector{ 1, 0 } < 0$ and
$\grad H \dotp \vector{ 0, 1 } > 0$.

Then
$\displaystyle\frac{\partial}{\partial s} F(G(s,t),H(s,t))$ is
\begin{multipleChoice}
  \choice{negative}
  \choice{zero}
  \choice[correct]{positive}
  \choice{unknowable from the given data.}      
\end{multipleChoice}

\begin{hint}
By the chain rule,
\begin{align*}
  \frac{\partial}{\partial x} F(G(x,y),H(x,y))
  &=\grad F \dotp\vector{\pp[G]{x},\pp[H]{x}}\\
&= \frac{\partial F}{\partial x} \frac{\partial G}{\partial x} + \frac{\partial F}{\partial y} \frac{\partial H}{\partial x} \\
&= \mbox{positive} \cdot \mbox{positive} + \mbox{negative} \cdot \mbox{negative} \\
&= \mbox{positive} + \mbox{positive} \\
&= \mbox{positive}.
\end{align*}
\end{hint}
\begin{hint}
We conclude $\frac{\partial}{\partial x} F(G(x,y),H(x,y)) > 0$.
\end{hint}

\end{exercise}
\end{document}
