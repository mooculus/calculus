\documentclass{ximera}
\newcommand{\RR}{\mathbb R}
\renewcommand{\d}{\,d}
\newcommand{\dd}[2][]{\frac{d #1}{d #2}}
\renewcommand{\l}{\ell}
\newcommand{\ddx}{\frac{d}{dx}}
\newcommand{\dfn}{\textbf}
\newcommand{\eval}[1]{\bigg[ #1 \bigg]}

\author{Steven Gubkin}
\license{Creative Commons 3.0 By-NC}
\outcome{ Define a critical point.}
\outcome{ Find critical points.}
\outcome{ Define absolute maximum and absolute minimum.}
\outcome{ Find the absolute max or min of a continuous function on a closed interval.}
\outcome{ Define local maximum and local minimum.}
\outcome{ Compare and contrast local and absolute maxima and minima.}
\outcome{ Identify situations in which an absolute maximum or minimum is guaranteed.}
\outcome{ Classify critical points.}
\outcome{ State the First Derivative Test.}
\outcome{ Apply the First Derivative Test.}
\begin{document}

\begin{exercise}

The function $f(x) = x^3-6x+1$ has two critical points.  If we call these critical point $a$ and $b$, and order them such that $a < b$, then

$$
a = \answer{-\sqrt{2}}
$$

$$
b=\answer{\sqrt{2}}
$$

On $(-\infty,a)$, $f$ is \wordChoice{\choice[correct]{increasing} \choice{decreasing}}

On $(a,b)$, $f$ is \wordChoice{\choice{increasing} \choice[correct]{decreasing}}

On $(b,\infty)$, $f$ is \wordChoice{\choice[correct]{increasing} \choice{decreasing}}

\end{exercise}
\end{document}
