\documentclass{ximera}

%\newcommand{\RR}{\mathbb R}
\renewcommand{\d}{\,d}
\newcommand{\dd}[2][]{\frac{d #1}{d #2}}
\renewcommand{\l}{\ell}
\newcommand{\ddx}{\frac{d}{dx}}
\newcommand{\dfn}{\textbf}
\newcommand{\eval}[1]{\bigg[ #1 \bigg]}

\newcommand{\RR}{\mathbb R}
\renewcommand{\d}{\,d}
\newcommand{\dd}[2][]{\frac{d #1}{d #2}}
\renewcommand{\l}{\ell}
\newcommand{\ddx}{\frac{d}{dx}}
\newcommand{\dfn}{\textbf}
\newcommand{\eval}[1]{\bigg[ #1 \bigg]}


\author{}
\license{Creative Commons 3.0 By-NC}
\outcome{}
\begin{document}

\begin{exercise} Calculate $y'$ for each of the following functions.

\begin{itemize}
\item If $y=\sqrt{2x+1}$, then $y' = \answer{\frac{1}{\sqrt{2x+1}}}$.
\item If $y=e^x+x^e$, then $y' = \answer{e^x+ex^{e-1}}$.
\item If $y=x\sin(\pi x)$, then $y' = \answer{\sin(\pi x) +\pi x \cos(\pi x)  }$.
\item If $y=\ln(x^4+5)$, then $y' = \answer{\frac{4x^3}{x^4+5}}$.
\item If $y=\left(\frac{\cos(2x)}{x}\right)^3$, then $y' = \answer{3\left(\frac{\cos(2x)}{x}\right)^2 \cdot \frac{-2x\sin(2x)-\cos(2x)}{x^2}}$.
\end{itemize}

\end{exercise}

\begin{exercise}
Find the equation of the tangent line to the graph of $y=3e^{2x}+1$ at the point $x=0$.  Express your final answer in the form $y=mx+b$.

The tangent line is $y=\answer{6x+4}$.

\begin{hint}
The slope of the tangent line is given by $y'(0)$.  Since $y=3e^{2x}+1$,

\[
y' = \answer{6e^{2x}}
\]

and thus $y'(0) = \answer{6}$.

A point on the line is found using the function.  Since $y(0) = \answer{4}$, a point on the line is $(x,y)= \left(\answer{0},\answer{4}\right)$.

Now, use the point-slope form for a line

\[
y-y_0=m(x-x_0)
\]
and solve for $y$.
\end{hint}

\end{exercise}
\end{document}