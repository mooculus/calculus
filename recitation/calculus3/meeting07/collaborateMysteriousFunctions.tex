\documentclass[handout,noauthor,nooutcomes]{ximera}

\newcommand{\RR}{\mathbb R}
\renewcommand{\d}{\,d}
\newcommand{\dd}[2][]{\frac{d #1}{d #2}}
\renewcommand{\l}{\ell}
\newcommand{\ddx}{\frac{d}{dx}}
\newcommand{\dfn}{\textbf}
\newcommand{\eval}[1]{\bigg[ #1 \bigg]}


\author{Jim Fowler \and Bart Snapp}

\title{Mysterious functions}

\begin{document}
\begin{abstract}
  We describe  mysterious functions
\end{abstract}
\maketitle

\textbf{Work in groups of 3--4, writing your answers on a separate
  sheet of paper.}

\begin{problem}
Consider the differentiable function $F:\R^2\to\R$ where:
\begin{align*}
  F(2,3) &= 1        &  F^{(1,0)}(2,3) &= 0 & F^{(2,0)}(2,3) &= -3\\
  F^{(1,1)}(2,3) &=0 &  F^{(0,1)}(2,3) &= 0 & F^{(0,2)}(2,3) &= -4  
\end{align*}
\begin{enumerate}
\item Write down the second degree Taylor polynomial for $F$ centered at $(2,3)$.
\item Is the point $(2,3,F(2,3))$ a local min, a local max, a saddle,
  or none of these?
\end{enumerate}
\end{problem}


\begin{problem}
Consider the differentiable function $G:\R^2\to\R$ where:
\begin{align*}
  G(-2,1) &= -4       &  G^{(1,0)}(-2,1) &= 0 & G^{(2,0)}(-2,1) &= 2\\
  G^{(1,1)}(-2,1) &=3 &  G^{(0,1)}(-2,1) &= 0 & G^{(0,2)}(-2,1) &= 5  
\end{align*}
\begin{enumerate}
\item Write down the second degree Taylor polynomial for $G$ centered at $(-2,1)$.
\item Is the point $(-2,1,G(-2,1))$ a local min, a local max, a saddle,
  or none of these?
\end{enumerate}
\end{problem}


\begin{problem}
Consider the differentiable function $H:\R^2\to\R$ where:
\begin{align*}
  H(2,2) &= 3       &  H^{(1,0)}(2,2) &= 0 & H^{(2,0)}(2,2) &= 3\\
  H^{(1,1)}(2,2) &=7 &  H^{(0,1)}(2,2) &= 0 & H^{(0,2)}(2,2) &= 2  
\end{align*}
\begin{enumerate}
\item Write down the second degree Taylor polynomial for $H$ centered at $(2,2)$.
\item Is the point $(2,2,H(2,2))$ a local min, a local max, a saddle,
  or none of these?
\end{enumerate}
\end{problem}

\begin{problem} 
Consider the differentiable function $I:\R^2\to\R$ where:
\begin{align*}
  I(1,-1) &= -4       &  I^{(1,0)}(1,-1) &= 1 & I^{(2,0)}(1,-1) &= 2\\
  I^{(1,1)}(1,-1) &=-6 &  I^{(0,1)}(1,-1) &= 2 & I^{(0,2)}(1,-1) &= 2  
\end{align*}
\begin{enumerate}
\item Write down the second degree Taylor polynomial for $I$ centered at $(1,-1)$.
  \item Let $\vecl(t)$ parameterize a line from $(0,0)$ to $(2,-2)$ as
  $t$ runs from $0$ to $1$. Compute $\eval{\dd{t} F
  (\vecl(t))}_{t=1/2}$.
\item Is the point $(1,-1,I(1,-1))$ a local min, a local max, a saddle,
  or none of these?
\end{enumerate}
\end{problem}


\begin{problem} 
Consider the differentiable function $J:\R^2\to\R$ where:
\begin{align*}
  J(-3,1) &= -4       &  J^{(1,0)}(-3,1) &= -2 & J^{(2,0)}(-3,1) &= -4\\
  J^{(1,1)}(-3,1) &=3 &  J^{(0,1)}(-3,1) &= -1 & J^{(0,2)}(-3,1) &= -3  
\end{align*}
\begin{enumerate}
\item Write down the second degree Taylor polynomial for $J$ centered at $(-3,1)$.
\item Let $\vecl(t)$ parameterize a line from $(3,-1)$ to $(-6,2)$ as
  $t$ runs from $0$ to $1$. Compute $\eval{\dd{t} F(\vecl(t))}_{t=2/3}$.
\item Is the point $(-3,1,J(-3,1))$ a local min, a local max, a saddle,
  or none of these?
\end{enumerate}
\end{problem}








\end{document}
