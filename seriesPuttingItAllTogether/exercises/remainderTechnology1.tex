\documentclass{ximera}

\newcommand{\RR}{\mathbb R}
\renewcommand{\d}{\,d}
\newcommand{\dd}[2][]{\frac{d #1}{d #2}}
\renewcommand{\l}{\ell}
\newcommand{\ddx}{\frac{d}{dx}}
\newcommand{\dfn}{\textbf}
\newcommand{\eval}[1]{\bigg[ #1 \bigg]}


\author{Jim Talamo}
\license{Creative Commons 3.0 By-NC}


\outcome{Understand the relationship between the sequence of remainders and the convergence of the series.}

\begin{document}

\begin{exercise}

Calculate the value of $\sum_{k=1}^{\infty} \frac{3}{k^{5/2}}$ to within $.0001$ of its exact value.  Use technology to perform all necessary computations.

%N=62

To within $.0001$, we find that $\sum_{k=1}^{\infty}  \frac{1}{k^{4/3}} \approx \answer[tolerance=.00006]{4.02446}$
\begin{hint}
This integral isn't too bad to compute.  However, the point of this exercise is to use technology, so take advantage of it!  By using technology to solve $\int_N^{N+1} \frac{3}{x^{5/2}} \d x =.0001$, we find that $N = \answer{62}$ (remember to round up).  The error estimates can now be used to find a range of possible values for the series.
\end{hint}

\end{exercise}
\end{document}
