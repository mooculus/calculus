\documentclass{ximera}

\newcommand{\RR}{\mathbb R}
\renewcommand{\d}{\,d}
\newcommand{\dd}[2][]{\frac{d #1}{d #2}}
\renewcommand{\l}{\ell}
\newcommand{\ddx}{\frac{d}{dx}}
\newcommand{\dfn}{\textbf}
\newcommand{\eval}[1]{\bigg[ #1 \bigg]}


\author{Jim Talamo}
\license{Creative Commons 3.0 By-NC}


\outcome{Review integration techniques}


\begin{document}
\begin{exercise}
Consider the initial value problem below:

\[
\dd[y]{x} = \frac{6y}{x^2-2x-8}  , \qquad y(0)=b \textrm{ where } b>0.
\]

$<$ IMAGE TO BE INCLUDED ON AN UPDATE - PLEASE LOOK AT THE SAMPLE PROBLEMS FROM CLASS $>$
%The direction field for the differential equation is:
%
%\begin{image}
%{\def\length{sqrt(1+((6*y)/(x^2-2*x-8))^2)}
%\begin{tikzpicture}
%  \begin{axis}[
%      xmin=-2.3, xmax=1.5,ymin=-1,ymax=9.8,domain=-2.3:9.8,view={0}{90},
%      axis lines =center, xlabel=$x$, ylabel=$y$,
%      every axis y label/.style={at=(current axis.above origin),anchor=south},
%      every axis x label/.style={at=(current axis.right of origin),anchor=west},
%      axis on top,
%    ] 
%    \addplot3 [penColor, quiver={u={1/\length}, v={(6*y)/(x^2-2*x-8)/(\length)},scale arrows=.3},samples=30] {0};
%]  \end{axis}
%\end{tikzpicture}}
%\end{image}




The solution of the initial value problem is $y(x) = \answer{ \frac{b}{2} \cdot \frac{x-4}{x+2} }$.

\begin{hint}
The differentials $\d y$ and $\d x$ are related by:

\[
\left(\answer{\frac{1}{y}} \right) \d y = \left(\frac{6}{\answer{x^2-2x-8} } \right) \d x
\]

\begin{question}
We can now integrate both sides.

To integrate the righthand side:

\begin{multipleChoice}
\choice{Use the integration formula involving an inverse tangent.}
\choice{Use a $u$-substitution}
\choice{Use integration by parts}
\choice[correct]{Use partial fraction decomposition}
\end{multipleChoice}

To perform the partial fraction decomposition:

\[
\frac{6}{x^2-2x-8} = \frac{6}{\left(x-\answer{4}\right)\left(x+\answer{2}\right)} = \frac{A}{x-\answer{4}}+\frac{B}{x+\answer{2}}
\]

We should multiply both sides by $\answer{(x-4)(x+2)}$.  After doing so, we obtain:

\[
6=A\left(\answer{x+2}\right)+B\left(\answer{x-4}\right)
\]

From this, we can find: $A=\answer{1}$ and $B=\answer{-1}$, so:

\[
\frac{6}{x^2-2x-8} = \frac{\answer{1}}{x-4}+\frac{\answer{-1}}{x+2}
\]


\begin{question}
Since the differentials $\d y$ and $\d x$ are related by:

\[
\frac{1}{y}  \d y = \left(\frac{1}{x-4}-\frac{1}{x+2} \right) \d x
\]

we can integrate both sides to obtain:

\[
\answer{\ln|y|} = \answer{\ln|x-4| - \ln|x+2|} +C\\
\]

Using the rules of logarithms, $ \ln|x-4| - \ln|x+2| =  \ln\left| \answer{\frac{x-4}{x+2}} \right|$, so we can solve the above for $y$ to find:

\[
y= C \cdot \answer{\frac{x-4}{x+2}} \\
\]
  
\begin{question}  
Using the initial condition $y(0)=b$, we find $C=\answer{-\frac{b}{2}}$.

(type your answer in terms of $b$)

We thus have $y = \answer{-\frac{b}{2}} \cdot \frac{x-4}{x+2}$.


\end{question}
\end{question}
\end{question}
 \end{hint}
 
\begin{exercise}
Note that we specified $b>0$ at the start of the problem.  Does $\lim_{b \to -2^+} y(x)$ depend on the choice of $b$?
\begin{multipleChoice}
\choice{Yes}
\choice[correct]{No}
\end{multipleChoice}
In fact, we find:

\[
\lim_{b \to -2^+} y(x) = \answer{\infty}
\]
(Use $\infty$ or $-\infty$ where appropriate, or write ``DNE'' when the limit does not exist otherwise)
\end{exercise}
 
\end{exercise}
\end{document}
