\documentclass{ximera}

\newcommand{\RR}{\mathbb R}
\renewcommand{\d}{\,d}
\newcommand{\dd}[2][]{\frac{d #1}{d #2}}
\renewcommand{\l}{\ell}
\newcommand{\ddx}{\frac{d}{dx}}
\newcommand{\dfn}{\textbf}
\newcommand{\eval}[1]{\bigg[ #1 \bigg]}


\author{Jim Talamo}
\license{Creative Commons 3.0 By-NC}


\outcome{Apply the alternating series remainder results.}

\begin{document}

\begin{exercise}
Use the alternating series results to find the smallest $N$ so $\sum_{k=1}^N \frac{(-1)^k}{k^2}$ is guaranteed to approximate $\sum_{k=1}^\infty \frac{(-1)^k}{k^2}$ to within $.001$ of its true value.  

\[
N = \answer{ 31 } 
\]

\begin{hint}
We need to find $N$ so $a_{N+1} = \answer{\frac{1}{(N+1)^2}} \leq .001$.  Using technology, we find that we have equality when $N=\answer{30.6}$ (report you answer to one decimal place), so we choose $N = \answer{31}$.
\end{hint}

\begin{exercise}
Compute  $\sum_{k=1}^N \frac{(-1)^k}{k^2}$ for this value of $N$.  Report your answer to four decimal places.

\[
\sum_{k=1}^N \frac{(-1)^k}{k^2} = \answer[tolerance=.0002]{-.8230}. 
\]

\begin{hint}
Substitute $N=31$ and use technology to perform the addition!
\end{hint}
\begin{feedback}
This means that to within $.001$, we have that $\sum_{k=1}^\infty \frac{(-1)^k}{k^2} \approx -.8230 $
\end{feedback}
\end{exercise}
\end{exercise}
\end{document}
