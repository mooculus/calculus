\documentclass{ximera}

\newcommand{\RR}{\mathbb R}
\renewcommand{\d}{\,d}
\newcommand{\dd}[2][]{\frac{d #1}{d #2}}
\renewcommand{\l}{\ell}
\newcommand{\ddx}{\frac{d}{dx}}
\newcommand{\dfn}{\textbf}
\newcommand{\eval}[1]{\bigg[ #1 \bigg]}


\title[Dig-In:]{Continuity}
\begin{document}
\begin{abstract}
Using continuity to find limits
\end{abstract}
\maketitle

%This is the first dig in in this section

We learned previously that just estimating limits by plugging in nearby values
does not always give the correct answer.  We need some theorems that
tell us how to quickly find limits, at least in some special cases.  

The simplest limits to find are when the limit is the same things as plugging the value into the function. That is, when
\[
\lim_{x\to c}f(x) = f(c).
\]
We call this property \textit{continuity}.

\begin{definition}\index{continuous}
  A function $f$ is \dfn{continuous at a point} $a$ if
  \[
  \lim_{x\to a}f(x) = f(a).
  \]
\end{definition}

For many of the functions we are most familiar with, we can use continuity to determine limit values.  

It is very important to note that saying a function $f(x)$ is continuous at a point
$a$ is really making three statements:
\begin{enumerate}
\item $f(a)$ is defined.  That is, $a$ is in the domain of $f(x)$.
\item $\lim_{x\to a}f(x)$ exists.
\item $\lim_{x\to a} f(x) = f(a)$.
\end{enumerate}

The first two of these statements are implied by the third statement.

\begin{example}
Find the discontinuities (the values for $x$ where a function is not
continuous) for the function described below:
\begin{image}
\begin{tikzpicture}
	\begin{axis}[
            domain=0:10,
            ymax=5,
            ymin=0,
            samples=100,
            axis lines =middle, xlabel=$x$, ylabel=$y$,
            every axis y label/.style={at=(current axis.above origin),anchor=south},
            every axis x label/.style={at=(current axis.right of origin),anchor=west}
          ]
	  \addplot [very thick, penColor, smooth, domain=(4:10)] {3 + sin(deg(x*2))/(x-1)};
          \addplot [very thick, penColor, smooth, domain=(0:4)] {1};
          \addplot[color=penColor,fill=background,only marks,mark=*] coordinates{(4,3.30)};  %% open hole
          \addplot[color=penColor,fill=background,only marks,mark=*] coordinates{(6,2.893)};  %% open hole
          \addplot[color=penColor,fill=penColor,only marks,mark=*] coordinates{(4,1)};  %% closed hole
          \addplot[color=penColor,fill=penColor,only marks,mark=*] coordinates{(6,2)};  %% closed hole
        \end{axis}
\end{tikzpicture}
%% \caption{A plot of a function with discontinuities at $x=4$ and $x=6$.}
%% \label{plot:discontinuous-function}
\end{image}

List the $x$-values of the points of discontinuity.  $\answer[given]{4,6}$ %%BADBAD Need list of answers
\end{example}

\begin {explanation}
From Figure~\ref{plot:discontinuous-function} we see that $\lim_{x\to
  4} f(x)$ does not exist because
\[
\lim_{x\to 4-}f(x) = 1\qquad\text{and}\qquad \lim_{x\to 4+}f(x) \approx 3.5
\]
Hence $\lim_{x\to 4} f(x) \ne f(4)$, and so $f(x)$ is not
continuous at $x=4$.

We also see that $\lim_{x\to 6} f(x) \approx 3$ while $f(6) =
2$. Hence $\lim_{x\to 6} f(x) \ne f(6)$, and so $f(x)$ is not
continuous at $x=6$.
\end{explanation}

Building from the definition of \textit{continuous at a point}, we can
now define what it means for a function to be \textit{continuous} on
an interval.

\begin{definition}
  A function $f$ is \dfn{continuous on the interval $I$} if $\lim_{x\to a}
  f(x) = f(a)$ for all $a$ in $I$.
\end{definition}

Graphically, a function is continuous on an interval $I$ if you can
draw the function on that interval without any breaks in the graph.
This is often referred to as being able to draw the graph ``without
picking up your pencil.''

Many of the functions we are most familiar with are continuous.

\begin{theorem}[Continuity of Famous Functions]\index{continuity of famous functions}\label{theorem:continuity}
The following functions are continuous on the given intervals for $k$ a real number and $b$ a positive real number:
\begin{itemize}
\item $f(x)=k$, $-\infty <x <\infty$
\item $f(x)=x$,  $-\infty <x <\infty$
%We can get polynomials and rationals with the limit laws
\item $f(x)=x^k$, $-\infty<x<\infty$ (Note: When $k$ involves taking an even root, $x$ must be positive.)
\item $f(x)=b^x$,  $-\infty <x <\infty$
\item $f(x)=log_b(x)$, $0<x<\infty$ 
\item $f(x)=\sin(x)$, $-\infty <x <\infty$
\item $f(x)=\cos(x)$, $-\infty <x <\infty$
\end{itemize}
\end{theorem}

This theorem can be summarized by saying that the functions listed above are continuous wherever they are defined, that is, on their natural domain.

\begin{example}
$\lim_{x\to 3} x^\pi = \answer[given]{3^\pi}$
  \begin{explanation}
   The function $f(x)=x^\pi$ is of the form $x^k$ for a real number $k$.  Therefore, $f(x)=x^\pi$ is continuous for all real values of $x$.  In particular, $f(x)$ is continuous at $x=3$.  Since $x^\pi$ is continuous at $3$, we know that $\lim_{x\to 3} f(x) = f(3)$.  That is, $\lim_{x\to 3} x^\pi = 3^\pi$
  \end{explanation}  
\end{example}


Because the limit of a continuous function is the same as the function value, we can pass limits inside continuous functions.  

\begin{theorem}[Continuity of Composite Functions]\index{continuity of famous functions}\label{theorem:continuity}
Let $\lim_{x\to a} g(x) = L$ and let $f(x)$ be continuous at the point $g(a)$.  Then, $\lim_{x\to a} f(g(x)) = f(\lim_{x\to a} g(x))$.  In particular, if $g(x)$ is continuous at $a$, then $f(g(x))$ is continuous at $a$.
\end{theorem}

\begin{example}
$\lim_{x \to 0} \sqrt{\cos(x)} = \answer[given]{1}$
\begin{explanation}
\begin{align*}
  \lim_{x \to 0} \sqrt{\cos(x)} &= 
  \sqrt{\lim_{x \to 0} (\cos(x))} \text{\ \ (by continuity of $x^k$, assuming $\lim_{x \to 0} \cos(x) >0$})\\
  &= \sqrt{\cos(0)} = \sqrt{1} = 1 \text{\ \ (by continuity of $\cos(x)$)}\\
\end{align*}
\end{explanation}
\end{example}
\end{document}
