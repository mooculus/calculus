\documentclass{ximera}

\newcommand{\RR}{\mathbb R}
\renewcommand{\d}{\,d}
\newcommand{\dd}[2][]{\frac{d #1}{d #2}}
\renewcommand{\l}{\ell}
\newcommand{\ddx}{\frac{d}{dx}}
\newcommand{\dfn}{\textbf}
\newcommand{\eval}[1]{\bigg[ #1 \bigg]}


\outcome{}

\title[Dig-In:]{The definite integral}

\begin{document}
\begin{abstract}
  Definite integrals compute signed area.
\end{abstract}
\maketitle


Definite integrals, often simply called integrals, compute signed area. 

\begin{definition}\index{integral}\index{definite integral}
The \dfn{definite integral}
\[
\int_a^b f(x) \d x
\]
computes the signed area between $y=f(x)$ and the $x$-axis on the
interval $[a,b]$.
\begin{itemize}
  \item If the region is above the $x$-axis, then the area has
    positive sign.
  \item If the region is below the $x$-axis, then the area has
    negative sign.
\end{itemize}
Note, when working with signed area, ``positive'' and ``negative''
area cancel each other out.
\end{definition}

\begin{question}
Consider the following graph of $y=f(x)$:
\begin{image}
  \begin{tikzpicture}
    \begin{axis}[
        width=6in,
        height=3in,
        xmin=-.5, xmax=5.5,ymin=-1.2,ymax=1.2,domain=0:6,
        axis lines =center, xlabel=$x$, ylabel=$y$,
        every axis y label/.style={at=(current axis.above origin),anchor=south},
        every axis x label/.style={at=(current axis.right of origin),anchor=west},
        axis on top,
    ] 
      \addplot [draw=none, fill=fillp,domain=0:1] {x} \closedcycle;
      \addplot [draw=none, fill=fillp,domain=1:5] {1.5-x/2} \closedcycle;
      
      \addplot [penColor,very thick,domain=0:1] {x};
      \addplot [penColor,very thick,domain=1:5] {1.5-x/2};
  \end{axis}
  \end{tikzpicture}
\end{image}
Compute:
\begin{enumerate}
\item $\int_0^3 f(x) \d x \begin{prompt}= \answer{1.5}\end{prompt}$
\item $\int_3^5 f(x) \d x \begin{prompt}= \answer{-1}\end{prompt}$
\item $\int_0^5 f(x) \d x \begin{prompt}= \answer{0.5}\end{prompt}$
\item $\int_0^3 5\cdot f(x) \d x \begin{prompt}= \answer{7.5}\end{prompt}$
\item $\int_1^1 5\cdot f(x) \d x \begin{prompt}= \answer{0}\end{prompt}$
\end{enumerate}
\begin{hint}
  Use the formula for the area of a triangle.
\end{hint}
\begin{hint}
  Remember, we are dealing with ``signed'' area here:
  \begin{image}
  \begin{tikzpicture}
    \begin{axis}[
        width=6in,
        height=3in,
        xmin=-.5, xmax=5.5,ymin=-1.2,ymax=1.2,domain=0:6,
        axis lines =center, xlabel=$x$, ylabel=$y$,
        every axis y label/.style={at=(current axis.above origin),anchor=south},
        every axis x label/.style={at=(current axis.right of origin),anchor=west},
        axis on top,
    ] 
      \addplot [draw=none, fill=fillp,domain=0:1] {x} \closedcycle;
      \addplot [draw=none, fill=fillp,domain=1:3] {1.5-x/2} \closedcycle;
      \addplot [draw=none, fill=filln,domain=3:5] {1.5-x/2} \closedcycle;
      
      \addplot [penColor,very thick,domain=0:1] {x};
      \addplot [penColor,very thick,domain=1:5] {1.5-x/2};
      \node at (axis cs:1.1,.4) [textColor] {\scalebox{2}{$\boldsymbol+$}};
      \node at (axis cs:4.5,-.4) [textColor] {\scalebox{2}{$\boldsymbol-$}};
  \end{axis}
  \end{tikzpicture}
\end{image}
\end{hint}
\end{question}


Our previous question hopefully gives us enough insight that this next
theorem is unsurprising.

\begin{theorem}[Properties of the definite integral]
Let $f$ and $g$ be defined on a closed interval $[a,b]$ that contains the
value $c$, and let $k$ be a constant. The following
hold:
\begin{enumerate}
\item $\int_a^a f(x)\d x = 0$
\item $\int_a^c f(x)\d x + \int_c^b f(x)\d x = \int_a^b f(x)\d x$
\item $\int_a^bf(x)\d x = -\int_b^a f(x)\d x$
\item $\int_a^b f(x)\pm g(x)\d x = \int_a^bf(x)\d x \pm \int_a^bg(x)\d x$
\item $\int_a^bk\cdot f(x)\d x = k\cdot\int_a^bf(x)\d x$
\end{enumerate}
\begin{explanation}
  We will address each property in turn:
\begin{enumerate}
\item Here, there is no ``area under the curve'' when the region has
  no width; hence this definite integral is $0$.
\item This states that total area is the sum of the areas of
  subregions. Here a picture is worth a thousand words:
  \begin{image}
    \begin{tikzpicture}[
        declare function = {f(\x) = -sin(deg(\x)) + 3;} ]
      \begin{axis}[
          domain=-.2:7, xmin =-.2,xmax=7,ymax=5,ymin=-.2,
          width=6in,
          height=3in,
          xtick={1,3.5,6}, 
          xticklabels={$a$,$c$,$b$},
          ytick style={draw=none},
          yticklabels={},
          axis lines=center, xlabel=$x$, ylabel=$y$,
          every axis y label/.style={at=(current axis.above origin),anchor=south},
          every axis x label/.style={at=(current axis.right of origin),anchor=west},
          axis on top,
          ]
        \addplot [draw=none,fill=fill4,domain=1:3.5, smooth] {f(x)} \closedcycle;
        \addplot [draw=none,fill=fill5,domain=3.5:6, smooth] {f(x)} \closedcycle;
        \addplot [very thick,penColor, smooth] {f(x)};
        \addplot [dashed] plot coordinates {(3.5,0) (3.5,{f(3.5)})};
        
        \node at (axis cs:2.25,1) {\large$\int_a^c f(x)\d x$};
        \node at (axis cs:4.75,1) {\large$\int_c^b f(x)\d x$};
      \end{axis}
    \end{tikzpicture}
  \end{image}		
  It is important to note that this still holds true even if
  $a<b<c$. We discuss this in the next point.
  
\item For now, this property can be viewed a merely a convention to
  make other properties work well. However, later we will see how this
  property has a justification all its own.

\item This states that when one scales a function by, for instance, $7$,
  the area of the enclosed region also is scaled by a factor of
  $7$.
\item This states that the integral of the sum is the sum of the
  integrals.
\end{enumerate}
\end{explanation}
\end{theorem}

Due to the geometric nature of integration, geometric properties of
functions can help us compute integrals.

\begin{definition}
  A function $f$ is an \dfn{odd} function if
  \[
  f(-x) = -f(x),
  \]
  and a function $g$ is an \dfn{even} function if
  \[
  g(-x) = g(x).
  \]
\end{definition}

The names \textit{odd} and \textit{even} come from the fact that these
properties are shared by functions of the form $x^n$ where $n$ is
either odd or even. For example, if $f(x) = x^3$, then
\[
f(-7) = -f(7),
\]
and if $g(x) = x^4$ then
\[
g(-7) = g(7).
\]
Geometrically, even functions have \textit{horizontal
  symmetry}. Cosine is an even function:
\begin{image}
 \begin{tikzpicture}
	\begin{axis}[
            xmin=-6.75,xmax=6.75,ymin=-1.5,ymax=1.5,
            axis lines=center,
            xtick={-6.28, -4.71, -3.14, -1.57, 0, 1.57, 3.142, 4.71, 6.28},
            xticklabels={$-2\pi$,$-3\pi/2$,$-\pi$, $-\pi/2$, $0$, $\pi/2$, $\pi$, $3\pi/2$, $2\pi$},
            ytick={-1,1},
            %ticks=none,
            width=6in,
            height=3in,
            unit vector ratio*=1 1 1,
            xlabel=$\theta$, ylabel=$x$,
            every axis y label/.style={at=(current axis.above origin),anchor=south},
            every axis x label/.style={at=(current axis.right of origin),anchor=west},
          ]        
          \addplot [very thick, penColor2, samples=100,smooth, domain=(-6.75:6.75)] {cos(deg(x))};
          \node at (axis cs:-1.57,.75) [penColor2] {$\cos(\theta)$};
        \end{axis}
\end{tikzpicture}
\end{image}
On the other hand, odd functions have $180^\circ$ rotational symmetry
around the origin. Sine is an odd function:
\begin{image}
\begin{tikzpicture}
	\begin{axis}[
            xmin=-6.75,xmax=6.75,ymin=-1.5,ymax=1.5,
            axis lines=center,
            xtick={-6.28, -4.71, -3.14, -1.57, 0, 1.57, 3.142, 4.71, 6.28},
            xticklabels={$-2\pi$,$-3\pi/2$,$-\pi$, $-\pi/2$, $0$, $\pi/2$, $\pi$, $3\pi/2$, $2\pi$},
            ytick={-1,1},
            %ticks=none,
            width=6in,
            height=3in,
            unit vector ratio*=1 1 1,
            xlabel=$\theta$, ylabel=$x$,
            every axis y label/.style={at=(current axis.above origin),anchor=south},
            every axis x label/.style={at=(current axis.right of origin),anchor=west},
          ]        
          \addplot [very thick, penColor, samples=100,smooth, domain=(-6.75:6.75)] {sin(deg(x))};
          
          \node at (axis cs:3.14,.75) [penColor] {$\sin(\theta)$};
        \end{axis}
\end{tikzpicture}
\end{image}
\begin{question}
  Let $f(x)$ be an odd function defined for all real numbers.
  \[
  \int_{-4}^4 f(x) \d x \begin{prompt}=\answer{0}\end{prompt}
  \]
  \begin{feedback}
    First note
    \begin{align*}
      \int_{-4}^4 f(x) \d x &= \int_{-4}^0 f(x) \d x + \int_0^4 f(x) \d x \\
      &= -\int_0^4 f(x) \d x + \int_0^4 f(x) \d x\\
      &= 0.
    \end{align*}
  \end{feedback}
\end{question}
\end{document}
