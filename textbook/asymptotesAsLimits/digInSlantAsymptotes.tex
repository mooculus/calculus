\documentclass{ximera}

\newcommand{\RR}{\mathbb R}
\renewcommand{\d}{\,d}
\newcommand{\dd}[2][]{\frac{d #1}{d #2}}
\renewcommand{\l}{\ell}
\newcommand{\ddx}{\frac{d}{dx}}
\newcommand{\dfn}{\textbf}
\newcommand{\eval}[1]{\bigg[ #1 \bigg]}


\outcome{Define a slant asymptote.}
\outcome{Approximate a slant asymptote from the graph of a function.}
\outcome{Find slant asymptotes algebraically and graphically.}

\title[Dig-In:]{Slant asymptotes}

\begin{document}
\begin{abstract}
We explore functions that ``shoot to infinity'' at certain points in
their domain.
\end{abstract}
\maketitle

If we think of an asymptote as a ``line that a function looks like
when the input is large,'' then there are three types of asymptotes:
\begin{enumerate}
\item Vertical Asymptotes.
\item Horizontal Aymptotes.
\item Slant Asymptotes.
\end{enumerate}

Recall, a function $f$ has a vertical asymptote at $x=a$ if

\[
\lim_{x\to a} f(x) = \pm\infty, \qquad \lim_{x\to a+} f(x) = \pm\infty, \qquad\text{or}\qquad \lim_{x\to a-} f(x) = \pm\infty,
\]
In this case, the asymptote is the vertical line
\[
x = a.
\]
On the other hand, a function $f$ has a horizontal asymptote if
\[
\lim_{x\to \infty} f(x) = L \qquad\text{or}\qquad \lim_{x\to -\infty}
f(x) = L,
\]
and in this case, the asymptote is the line
\[
\l(x) = L.
\]
On the other hand, a \textit{slant asymptote} is a somewhat different
beast.

\begin{definition}\index{slant asymptote}\index{asymptote!slant}
  If there is a nonhorizontal line $\l(x) = m\cdot x+b$ such that
  \[
  \lim_{x\to \infty}\left(f(x) - \l(x)\right) = 0,
  \]
  then $\l$ is a \dfn{slant asymptote} for $f$.
\end{definition}
\begin{question}
  Consider the plot of the following function. 
  \begin{image}
    \begin{tikzpicture}
      \begin{axis}[
          samples=100,
          axis lines =middle, xlabel=$x$, ylabel=$y$,
          every axis y label/.style={at=(current axis.above origin),anchor=south},
          every axis x label/.style={at=(current axis.right of origin),anchor=west},
        ]
	\addplot [very thick, penColor, smooth,domain=0:5] {x/2-1};
        \addplot [very thick, penColor, smooth,domain=-5:0] {sqrt(-x)};
        \addplot[color=penColor,fill=background,only marks,mark=*] coordinates{(0,-1)};  %% open hole
        \addplot[color=penColor,fill=penColor,only marks,] coordinates{(0,0)};  %% closed hole
      \end{axis}
    \end{tikzpicture}
  \end{image}
  What is the slant asymptote of this function?
  \begin{prompt}
    \[
    \l(x) = \answer{x/2 -1}
    \]
  \end{prompt}
\end{question}

To analytically find slant asymptotes, one must find the required
information to determine a line:
\begin{itemize}
\item The slope.
\item The $y$-intercept.
\end{itemize}
While there are several ways to do this, we will give a method that is
fairly general.


\begin{example}
  Find the slant asymptote of
  \[
f(x) = \frac{x^2-3x+2}{x+2}.
  \]
  \begin{explanation}
    We are looking to see if there is a line $\l$ such that
    \[
    \lim_{x\to \infty}\left(f(x) - \l(x)\right) = 0.
    \]
    We will imagine that we have such a line
    \[
    \l(x) = m\cdot x + b
    \]
    and attempt to find such a line. We see
    \begin{align*}
      m &=\lim_{x\to\infty}\frac{\frac{x^2-3x+2}{x+2}}{x}\\
      &= \lim_{x\to\infty}\frac{\frac{x^2-3x+2}{x+2}}{x}\\
      &= \lim_{x\to\infty}\frac{x^2-3x+2}{x^2+2x}\\
      &= \lim_{x\to\infty}\frac{x^2-3x+2}{x^2+2x}\cdot\frac{1/x^2}{1/x^2}\\
      &= \lim_{x\to\infty}\frac{1-3/x+2/x^2}{1+2/x}\\
      &= \answer[given]{1}.
    \end{align*}
    So $m=1$. To find the $y$-intercept, write
    \begin{align*}
      b &=\lim_{x\to\infty} \left(\frac{x^2-3x+2}{x+2} - x\right)\\
      &=\lim_{x\to\infty} \frac{x^2-3x+2-x^2}{x+2}\\
      &=\lim_{x\to\infty} \frac{-3x+2}{x+2}\\
      &=\lim_{x\to\infty} \frac{-3x+2}{x+2}\cdot\frac{1/x}{1/x}\\
      &=\lim_{x\to\infty} \frac{-3+2/x}{1+2/x}\\
      &= \answer[given]{-3}.
    \end{align*}
    Hence our slant asymptote is $\l(x) = x-3$.
  \end{explanation}
\end{example}



\end{document}
