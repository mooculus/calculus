\documentclass{ximera}

\newcommand{\RR}{\mathbb R}
\renewcommand{\d}{\,d}
\newcommand{\dd}[2][]{\frac{d #1}{d #2}}
\renewcommand{\l}{\ell}
\newcommand{\ddx}{\frac{d}{dx}}
\newcommand{\dfn}{\textbf}
\newcommand{\eval}[1]{\bigg[ #1 \bigg]}


\author{Jim Fowler}

\outcome{Find a point along a curve where the position is orthogonal to the velocity.}

\begin{document}

\begin{exercise}

  Let $\vec{p}(t) = \vector{t+1, t^2 + 2t + 1, t+1}$.

  Find $t$ so that $\vec{p}(t)$ is orthogonal to its derivative $\vec{p}'(t)$.

  \begin{prompt}
    \[ t = \answer{-1}. \]
  \end{prompt}

  \begin{hint}
    Calculate $\vec{p}'(t) = \vector{1, 2t + 2, 1}$.
  \end{hint}

  \begin{hint}
    Consequently $\vec{p}(t) \dotp \vec{p}'(t) = \vector{t+1, t^2 + 2t + 1, t+1} \dotp \vector{1, 2t + 2, 1}$.
  \end{hint}

  \begin{hint}
    Therefore $\vec{p}(t) \dotp \vec{p}'(t) = (t+1) +  (t^2 + 2t + 1)(2t+2) +  (t+1)$.
  \end{hint}

  \begin{hint}
    Simplifying yields that $\vec{p}(t) \dotp \vec{p}'(t) = 2 \, t^{3} + 6 \, t^{2} + 8 \, t + 4$ which factors as $2 \, {\left(t^{2} + 2 \, t + 2\right)} {\left(t + 1\right)}$.
  \end{hint}

  \begin{hint}
    Note that $t^{2} + 2 \, t + 2$ has no real roots.
  \end{hint}      

  \begin{hint}
    So the only way that $\vec{p}(t) \dotp \vec{p}'(t)$ vanishes is when $t+1 = 0$, meaning $t = -1$.
  \end{hint}
  
\end{exercise}
\end{document}


%%% Local Variables:
%%% mode: latex
%%% TeX-master: t
%%% End:
