\documentclass{ximera}

\newcommand{\RR}{\mathbb R}
\renewcommand{\d}{\,d}
\newcommand{\dd}[2][]{\frac{d #1}{d #2}}
\renewcommand{\l}{\ell}
\newcommand{\ddx}{\frac{d}{dx}}
\newcommand{\dfn}{\textbf}
\newcommand{\eval}[1]{\bigg[ #1 \bigg]}



\author{Jim Talamo}

\outcome{Compute the partial derivative of an expression.}
\newcommand{\Pp}[2]{\frac{\partial #1}{\partial #2}}

\begin{document}

This exercise gives mechanical practice calculating partial derivatives for polynomials.
\begin{exercise}


If $f(x,y) = \ln(2x+3y^2)$, then $\Pp{f}{x} = \answer{\frac{2}{2x+3y^2}}$.

\begin{hint}
To compute the partial derivative with respect to $x$, we will treat all other variables as constants and differentiate expressions that explicitly depend on $x$ the same way we would before.

\[
\Pp{f}{x} = \pp{x}\left[ 4x+3y^2 \right] =\pp{x}\left[ 4x \right] +\pp{x}\left[ 3y^2 \right] = 4 + 0
\]
\end{hint}

\end{exercise}

%%%%%%%%%%%%%%%%%%%%%

\begin{exercise}
If $f(x,y) = 12xy^5$, then $\Pp{f}{y} = \answer{60xy^4}$.

\begin{hint}
To compute the partial derivative with respect to $y$, we will treat all other variables as constants and differentiate expressions that explicitly depend on $x$ the same way we would before.

\[
\Pp{f}{x} = \pp{x}\left[ 12xy^5 \right] =12x \cdot \pp{x}\left[ y^5 \right] = 12x \cdot 5y^4
\]
\end{hint}

\end{exercise}

%%%%%%%%%%%%%%%%%%%%%

\begin{exercise}
If $f(x,y) = x^3+4x^2y^7+5y^6+7$, then $\Pp{f}{x} = \answer{3x^2+8xy^7}$ and $\Pp{f}{y} = \answer{28x^2y^6+30y^5}$.
\end{exercise}

\end{document}
