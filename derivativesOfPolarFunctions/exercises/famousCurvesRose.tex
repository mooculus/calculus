\documentclass{ximera}

\newcommand{\RR}{\mathbb R}
\renewcommand{\d}{\,d}
\newcommand{\dd}[2][]{\frac{d #1}{d #2}}
\renewcommand{\l}{\ell}
\newcommand{\ddx}{\frac{d}{dx}}
\newcommand{\dfn}{\textbf}
\newcommand{\eval}[1]{\bigg[ #1 \bigg]}


%\outcome{Find tangent lines to parametric curves}
\author{Jim Talamo}

\begin{document}
\begin{exercise}
Desmos can allow us to view several famous examples of polar curves.

One famous example is a \emph{rose curve}.  These are curves of the form $r= A \cos(n \theta)$ or $r= A \sin(n \theta)$.

\begin{exercise}
Let's study what $r= a \cos(n \theta)$ looks like and examine the effect of the parameters $A$ and $n$.  

Set $A=1$, and graph $r=\cos(n \theta)$ for $n=1,2,3,4,$ and $5$ using Desmos.  Now, fill out the following tables:

$\bullet$ How many times is the rose traced out when $\theta$ varies between $0$ and $2 \pi$?

\[
\begin{array}{r|c|c|c|c|c}
\textrm{value for } n & 1 & 2 & 3 & 4 & 5 \\
\hline
\textrm{\# times traced out:} & 2 & \answer{1} & \answer{2} & \answer{1} & \answer{2}
\end{array}
\]

\begin{exercise}
You should notice a pattern! Which of the following describes the pattern?
\begin{multipleChoice}
\choice{The curve is traversed once when $\theta$ varies from $0$ to $2 \pi$ regardless of the choice of $n$.}
\choice{The curve is traversed once when $\theta$ varies from $0$ to $2 \pi$ if $n$ is odd and twice if $n$ is even.}
\choice[correct]{The curve is traversed once when $\theta$ varies from $0$ to $2 \pi$ if $n$ is even and twice if $n$ is odd.}
\end{multipleChoice}
\end{exercise}

$\bullet$ How many petals does each rose have?

\[
\begin{array}{r|c|c|c|c|c}
\textrm{value for } n & 1 & 2 & 3 & 4 & 5 \\
\hline
\textrm{\# petals:} & 1 & \answer{4} & \answer{3} & \answer{8} & \answer{5}
\end{array}
\]

\begin{exercise}
You should notice a pattern! Which of the following describes the pattern?
\begin{multipleChoice}
\choice{The curve has $n$ petals.}
\choice[correct]{The curve has $n$ petals if $n$ is odd and $2n$ petals if $n$ is even.}
\choice{The curve has $n$ petals if $n$ is even and $2n$ petals if $n$ is odd.}
\end{multipleChoice}
\end{exercise}

$\bullet$ Now, let's examine the effect of the parameter $A$.  Set $n=3$, and look at the graph of $r=A \cos(3\theta)$ (If you type this expression, Desmos should prompt you for a slider for $A$).  We can define the length of a petal to be the maximum distance between the origin and a point on the curve.  It should seem reasonable that this is the distance shown below:

\begin{image}  
  \begin{tikzpicture}  
    \begin{axis}[  
        xmin=-2.5,  
        xmax=2.5,  
        ymin=-2.5,  
        ymax=2.5,  
        axis lines=center,  
        xlabel=$x$,  
        ylabel=$y$,  
        every axis y label/.style={at=(current axis.above origin),anchor=south},  
        every axis x label/.style={at=(current axis.right of origin),anchor=west},  
      ]  
         \addplot[data cs=polar,penColor,domain=0:360,samples=360,smooth, thick] (x,{2*cos(3*x)});
         \draw (axis cs:2,2.9) node { $r=1+\sin\theta$};
         \addplot[only marks, mark=*] coordinates {(0, 0)};
         \addplot[only marks, mark=*] coordinates {(2, 0)};
	 \addplot [draw=black,fill=gray!50,ultra thick] coordinates {(0,0)(2,0)};
	  \draw[decoration={brace,raise=.1cm},decorate,thin] (axis cs:0,.5)--(axis cs:2,.5);
	  \node at (axis cs:1,1.2) [penColor] {\small{length of a petal}};
	    \node at (axis cs:1,.9) [penColor] {\small{(maximum distance)}};
         \end{axis}  
  \end{tikzpicture}  
\end{image} 

Now, change the value of $A$.  From the picture, what does the parameter $A$ represent for $r=A\cos(3 \theta)$?

\begin{multipleChoice}
\choice{$A$ is the period of the petal.}
\choice[correct]{$A$ is the length of a petal.}
\end{multipleChoice}

Now, choose a particular value of $A$ and let $n$ vary.  What effect does this have on the length of a leaf?

\begin{multipleChoice}
\choice{The length of a petal is $An$.}
\choice[correct]{The length of any petal is $A$; the length is independent of $n$.}
\choice{Neither of these.}
\end{multipleChoice}

How can we verify this?  Since we have defined this length as the maximum distance between the origin and a point on the rose, we can write down an expression for the distance between the origin and a point on the rose, then maximize it!

\begin{image}  
  \begin{tikzpicture}  
    \begin{axis}[  
        xmin=-2.5,  
        xmax=2.5,  
        ymin=-2.5,  
        ymax=2.5,  
        axis lines=center,  
        xlabel=$x$,  
        ylabel=$y$,  
        every axis y label/.style={at=(current axis.above origin),anchor=south},  
        every axis x label/.style={at=(current axis.right of origin),anchor=west},  
      ]  
         \addplot[data cs=polar,penColor,domain=0:360,samples=360,smooth, thick] (x,{2*cos(3*x)});
         \draw (axis cs:2,2.9) node { $r=1+\sin\theta$};
         \addplot[only marks, mark=*] coordinates {(0, 0)};
         \addplot[only marks, mark=*] coordinates {(1.381,0.365)};
	 \addplot [draw=black,fill=gray!50,ultra thick] coordinates {(0,0)(1.381,0.365)};
	 	 \node at (axis cs:.45,0.5) [penColor] {$d(x,y)$};
		 \node at (axis cs:1.381,0.6) [penColor] {$(x,y)$};
         \end{axis}  
  \end{tikzpicture}  
\end{image} 
\end{exercise}

Using the distance formula:

\[
d(x,y) = \sqrt{\answer{x^2+y^2}}
\]
(type an answer in terms of $x$ and $y$)

In order to maximize this, we need to express $d$ in terms of a single variable.  Which is the most natural choice?

\begin{multipleChoice}
\choice{$x$}
\choice{$y$}
\choice{$r$}
\choice[correct]{$\theta$}
\end{multipleChoice}
Since we have a parametric description of the curve in terms of $\theta$, this is the natural choice! In fact:

\begin{align*}
x(\theta) &= \answer{A\cos(3\theta)\cos(\theta)} \\
y(\theta) &= \answer{A\cos(3\theta)\sin(\theta)}
\end{align*}

Thus:

\[
d(\theta) = \sqrt{[x(\theta)]^2+[y(\theta)]^2} = \sqrt{\left[\answer{A\cos(3\theta)\cos(\theta)}\right]^2+\left[ \answer{A\cos(3\theta)\sin(\theta)} \right]^2}
\]

\begin{exercise}
We can simplify the expression under the square root:
\begin{align*}
\left[A\cos(3\theta)\cos(\theta)\right]^2+\left[A\cos(3\theta)\sin(\theta) \right]^2 &= A^2\cos^2(3\theta) \cos^2(\theta) + A^2\cos^2(3\theta) \sin^2(\theta)\\
&=  A^2\cos^2(3\theta) \left( \answer{\cos^2(\theta)+\sin^2(\theta)} \right) \\
&= A^2\cos^2(3\theta)
\end{align*}

Now, by symmetry, we can consider only the upper half of the petal that is bisected by the positive $x$-axis, where $\answer{0} \leq \theta \leq \answer{\frac{\pi}{6}}$.

Since $d(\theta) =\sqrt{A^2\cos^2(3\theta)} = A\cos(3\theta)$ (because $\cos(3\theta)\geq 0$ for $0 \leq \theta \leq \frac{\pi}{6}$, no absolute value is necessary, all we have the optimization problem:

\begin{quote}
Maximize $d(\theta) = A\cos(3\theta)$ on the closed interval $\left[0,\frac{\pi}{6}\right]$.
\end{quote}

Since the maximum of $A\cos(3\theta)$ is $\answer{A}$, and this occurs when $\theta = \answer{0}$, the length of the petal is $\answer{A}$.

\end{exercise}
%%%%%%%%%%%%%%%%%%%%%%%%%%%%%
\begin{exercise}
Let's study what $r= a \sin(n \theta)$ looks like and examine the effect of the parameters $a$ and $n$.  

Set $a=1$, and graph $r=\sin(n \theta)$ for $n=1,2,3,4,$ and $5$ using Desmos.  Now, fill out the following tables:

$\bullet$ How many times is the rose traced out when $\theta$ varies between $0$ and $2 \pi$?

\[
\begin{array}{r|c|c|c|c|c}
\textrm{value for } n & 1 & 2 & 3 & 4 & 5 \\
\hline
\textrm{\# times traced out:} & 2 & \answer{1} & \answer{2} & \answer{1} & \answer{2}
\end{array}
\]

\begin{exercise}
You should notice a pattern! Which of the following describes the pattern?
\begin{multipleChoice}
\choice{The curve is traversed once when $\theta$ varies from $0$ to $2 \pi$ regardless of the choice of $n$.}
\choice{The curve is traversed once when $\theta$ varies from $0$ to $2 \pi$ if $n$ is odd and twice if $n$ is even.}
\choice[correct]{The curve is traversed once when $\theta$ varies from $0$ to $2 \pi$ if $n$ is even and twice if $n$ is odd.}
\end{multipleChoice}
\end{exercise}

$\bullet$ How many petals does each rose have?

\[
\begin{array}{r|c|c|c|c|c}
\textrm{value for } n & 1 & 2 & 3 & 4 & 5 \\
\hline
\textrm{\# petals:} & 1 & \answer{4} & \answer{3} & \answer{8} & \answer{5}
\end{array}
\]

\begin{exercise}
You should notice a pattern! Which of the following describes the pattern?
\begin{multipleChoice}
\choice{The curve has $n$ petals.}
\choice[correct]{The curve has $n$ petals if $n$ is odd and $2n$ petals if $n$ is even.}
\choice{The curve has $n$ petals if $n$ is even and $2n$ petals if $n$ is odd.}
\end{multipleChoice}
\end{exercise}

We could also ask what role $A$ has in the length of a leaf here.  The computation is a bit lengthier, but similar in spirit to the one before.  You are invited (and encouraged!) to explore it on your own.

\end{exercise}
\end{exercise}
\end{document}