\documentclass{ximera}

\newcommand{\RR}{\mathbb R}
\renewcommand{\d}{\,d}
\newcommand{\dd}[2][]{\frac{d #1}{d #2}}
\renewcommand{\l}{\ell}
\newcommand{\ddx}{\frac{d}{dx}}
\newcommand{\dfn}{\textbf}
\newcommand{\eval}[1]{\bigg[ #1 \bigg]}


\author{Bart Snapp and Jim Talamo}

\outcome{Determine whether functions of several variables are continuous.}

\title[Dig-In:]{Continuity}

\begin{document}
\begin{abstract}
We investigate what continuity means for functions of several variables.
\end{abstract}
\maketitle

Now that we have defined limits, we can define continuity.

\begin{definition}
  Let $F:\R^n\to \R$ and $\pt{a}$ be an interior point of the domain of $F$. We say $F$ is \dfn{continuous} at $\pt{x}=\pt{a}$, if
  \begin{itemize}
  \item $F(\pt{a})$ exists.
  \item $\lim_{\pt{x}\to\pt{a}} F(\pt{x})$ exists.
  \item $\lim_{\pt{x}\to\pt{a}} F(\pt{x}) = F(\pt{a}).$
  \end{itemize}
  $F$ is \dfn{continuous on an open set} $S$ if $F$ is continuous at
  all $\pt{x} \in S$.
\end{definition}

The \textit{limit laws} can be used to write corresponding \textit{continuity laws}.

\begin{theorem}[Limit Laws]
  Let $F:\R^n\to \R$ and $G:\R^n\to \R$ be continuous functions of several
  variables, and $b$ be a real number.
  \begin{align*}
    \pt{x} &= \point{x_1,x_2,\dots,x_n}\\ \pt{a} &=
    \point{a_1,a_2,\dots,a_n},
  \end{align*}
  where
  \[
  \lim_{\pt{x}\to\pt{a}}F(\pt{x}) = L \quad \text{and}\quad \lim_{\pt{x}\to\pt{a}} G(\pt{x}) = M.
  \]
\begin{description}
\item[Constant Law] $F(\pt{x}) = b$ is continuous.
\item[Identity Law] $F(\pt{x}) = x_i$ is continuous.
\item[Sum/Difference Law] $F(\pt{x})\pm G(\pt{x})$ is continuous.
\item[Scalar Multiple Law] $b\cdot F(\pt{x})$ is continuous.
\item[Product Law] $F(\pt{x})\cdot G(\pt{x})$ is continuous.
\item[Quotient Law] $\frac{F(\pt{x})}{G(\pt{x})}$ is continuous where  $G(\pt{x}) \neq 0$.
\end{description}
\end{theorem}

\begin{question}
  True or false: If $F:\R^2\to\R$ and $G:\R^2\to\R$ are continuous
  functions on an open disk $B$, then $F\pm G$ is continuous on $B$.
  \begin{prompt}
    \begin{multipleChoice}
      \choice[correct]{True}
      \choice{False}
  \end{multipleChoice}
  \end{prompt}
\end{question}

\begin{question}
  True or false: If $F:\R^2\to\R$ and $G:\R^2\to\R$ are continuous
  functions on an open disk $B$, then $F/G$ is continuous on $B$.
  \begin{prompt}
    \begin{multipleChoice}
      \choice{True}
      \choice[correct]{False}
    \end{multipleChoice}
    \begin{feedback}
      The function $F/G$ may or may not be continuous, it depends on
      whether $G(x,y)=0$. If $G(x,y)=0$, then $F/G$ not continuous at that point.
    \end{feedback}
  \end{prompt}
\end{question}


\begin{theorem}[Composition Limit Law]
  Let $f:\R\to\R$ be a continuous function on an interval $I$. Let
  $G:\R^n\to \R$ be a function whose range is contained in (or equal
  to) $I$, Then
  \[
  \lim_{\pt{x}\to\pt{a}} f( G(\pt{x})) = f(\lim_{\pt{x}\to\pt{a}}G(\pt{x}))
  \]
\end{theorem}


\begin{corollary}[Composition of Composite Functions]
  Let $G:\R^n\to \R$ be continuous on an open disk $B$, where the
  range of $G$ on $B$ is $I$, and let $f$ be a single variable
  function that is continuous on $I$. Then
  \[
  f\circ G(\pt{x}) =f(G(\pt{x})),
  \]
  is continuous on $B$.
\end{corollary}



\begin{example}
  Show that the function
  \[
  F(x,y) = \sin(x^2\cos(y))
  \]
  is continuous for all points in $\R^2$.
  \begin{explanation}
    Let
    \[
    F_1(x,y) = x^2.
    \]
    Since $y$ is not actually used in the function, and polynomials
    \wordChoice{\choice[correct]{are continuous}\choice{are not
        continuous}}, we conclude $F_1$ is continuous everywhere. A
    similar statement can be made about
    \[
    F_2(x,y) = \cos(y).
    \]
    Setting
    \[
    F_3=F_1\cdot F_2
    \]
    we obtain a continuous function from $\R^2\to \R$. Since sine \wordChoice{\choice[correct]{is
    continuous}\choice{is not continuous}} for all real values, the composition of sine with $F_3$
    is continuous. Hence, $\sin (F_3(x,y)) = \sin(x^2\cos y)$ is
    continuous everywhere.
    \begin{onlineOnly}
      We finish by presenting you with a plot of $F$:
      \begin{center}
        \geogebra{TNETssA9}{800}{600} %https://ggbm.at/TNETssA9
      \end{center}
    \end{onlineOnly}
  \end{explanation}
\end{example}


\begin{example}
  Let
  \[
  F(x,y) = \begin{cases}
    \frac{\cos(y)\sin(x)}{x} & x\neq 0 \\
    \cos(y) & x=0
  \end{cases}
  \]
  Is $F$ continuous at $(0,0)$? Is $F$ continuous everywhere?
  \begin{explanation}
    To determine if $F$ is continuous at $(0,0)$, we need to compare
    \[
    \lim_{\point{x,y}\to\point{0,0}} F(x,y)\quad\text{to}\quad F(0,0).
    \]
    Applying the definition of $F$, we see that:
    \[
    F(0,0) = \answer[given]{1}
    \]
    We now consider the limit
    \[
    \lim_{\point{x,y}\to\point{0,0}}F(x,y).
    \]
    Substituting $0$ for $x$ and $y$ in $(\cos(y)\sin(x))/x$ returns the
    indeterminate form \zeroOverZero, so we need to do more work to
    evaluate this limit.
    
    Consider two related limits:
    \begin{align*}
      \lim_{\point{x,y}\to\point{0,0}} \cos(y)\\
      \lim_{\point{x,y}\to\point{0,0}} \frac{\sin(x)}{x}.
    \end{align*}
    The first limit does not contain $x$, and since $\cos(y)$ is
    continuous,
    \begin{align*}
    \lim_{\point{x,y}\to\point{0,0}} \cos(y) &=\lim_{y\to 0} \cos(y) \\
    &=\answer[given]{1}
    \end{align*}
    The second limit does not contain $y$. But we know
    \begin{align*}
      \lim_{\point{x,y}\to\point{0,0}} \frac{\sin(x)}{x} &= \lim_{x\to 0} \frac{\sin(x)}{x} \\
      &= \answer[given]{1}.
    \end{align*}
    Finally, we know that we can combine these two limits so that 
    $\lim_{\point{x,y}\to\point{0,0}} \frac{\cos(y)\sin(x)}{x}$
    \begin{align*}
      &= \lim_{\point{x,y}\to\point{0,0}} (\cos(y))\left(\frac{\sin(x)}{x}\right) \\ 
      &=\left(\lim_{\point{x,y}\to\point{0,0}} \cos(y)\right)\left(\lim_{\point{x,y}\to\point{0,0}} \frac{\sin(x)}{x}\right) \\
            &=\answer[given]{1}\cdot \answer[given]{1}.
    \end{align*}
    We have found that $\lim_{\point{x,y}\to\point{0,0}} \frac{\cos(y)\sin(x)}{x} =
    F(0,0)$, so $F$ is continuous at $(0,0)$.

    A similar analysis shows that $F$ is continuous at all points in
    $\mathbb{R}^2$. As long as $x\neq0$, we can evaluate the limit
    directly; when $x=0$, a similar analysis shows that the limit is $\cos
    y$. Thus we can say that $F$ is continuous everywhere.
    \begin{onlineOnly}
      We finish by presenting you with a plot of $F$:
      \begin{center}
        \geogebra{VK6thpMa}{800}{600} %https://ggbm.at/VK6thpMa
      \end{center}
    \end{onlineOnly}
  \end{explanation}
\end{example}


\end{document}
