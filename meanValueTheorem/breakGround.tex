\documentclass{ximera}

\newcommand{\RR}{\mathbb R}
\renewcommand{\d}{\,d}
\newcommand{\dd}[2][]{\frac{d #1}{d #2}}
\renewcommand{\l}{\ell}
\newcommand{\ddx}{\frac{d}{dx}}
\newcommand{\dfn}{\textbf}
\newcommand{\eval}[1]{\bigg[ #1 \bigg]}


\outcome{Use the Mean Value Theorem to solve word problems.}
\outcome{Compare and contrast the Intermediate Value Theorem, the Mean Value Theorem, and Rolle's Theorem.}

% BADBAD This is not done

% Harvard calculus left the mean value theorem out.

% How is this related to the "racetrack principle"?

%% \title[Break-Ground:]{Mean value theorem}

%% \begin{document}
%% \begin{abstract}
%% A dialogue where students discuss the mean value theorem.
%% \end{abstract}
%% \maketitle

%% Check out this dialogue between two calculus students (based on a true story):

%% \begin{dialogue}
%% \item[Devyn] Riley, I've been thinking about angry values.
%% \item[Riley] What do you mean?
%% \item[Devyn] Oh, yeah, yeah, I meant mean values.
%% \end{dialogue}

% Use racetrack principle 

% rigorously translating infinitesimal data into data on points

% Rolle's theorem

% I drove to Urbana-Champaign in 5 hours; it's 300 miles away, so I was going 60 mph.

% But were you STOPPED at the beginning and STOPPED at the end?  So you must have been going faster than 60mph at some point.

% the mean policeman theorem:  https://vimeo.com/101691769
% My reaction: My position function is nondifferentiable at every point, so I never had a well-defined speed.

% Some example where the mean value theorem FAILS.  Kinks are better than discontinuity.

% Why the plus C?

% Quartic polynomial with only two real roots


\title[Break-Ground:]{Let's run to class}

\begin{document}
\begin{abstract}
Two young mathematicians race to math class.
\end{abstract}
\maketitle

Check out this dialogue between two calculus students (based on a true
story):

%% Brad suggest maybe make one constant rate

\begin{dialogue}
\item[Devyn] Riley, I want to go to math class. Now. Let's race!
\item[Riley] Yes. I love math class. Let's do race! On, your
  mark\dots. Ready. Steady. Go!
\item[Devyn] You may think you're fast, but I'm catching up!
\item[Riley] Noooooo!
\item[Devyn] Now I'm winning! I've never won a foot race in my life!
\item[Riley] Never\dots give\dots up!
\item[Devyn] Whew! We both made it to math class at exactly the same
  time!
\item[Riley] Wow. We should run to every class. Hey I have a question,
  was there a time during our race that we were running at exactly the
  same speed?
\end{dialogue}

\begin{problem}
  Which of the following describes the race above?
  \begin{multipleChoice}
    \choice{Devyn was leading until the end, when the race finished in a tie.}
    \choice{Riley was leading until the end, when the race finished in a tie.}
    \choice{Devyn was leading, then Riley was leading until the end, when the race finished in a tie.}
    \choice[correct]{Riley was leading, then Devyn was leading until the end, when the race finished in a tie.}
    \choice{None of the above.}
  \end{multipleChoice}
\end{problem}

\begin{problem}
  What can you say about Devyn's and Riley's average velocities?
  \begin{multipleChoice}
    \choice{Devyn has the larger average velocity.}
    \choice{Riley has the larger average velocity.}
    \choice[correct]{Their average velocities are equal.}
    \choice{None of the above.}
  \begin{feedback}
    Since Devyn and Riley start and stop at the same time and place,
    their average velocities are equal.
  \end{feedback}
  \end{multipleChoice}
\end{problem}

\begin{problem}
  Record your guess to Riley's question: is there a moment during the race where Devyn and Riley were running at exactly the same speed?
  \begin{freeResponse}
  	Enter your guess of ``yes'' or ``no'', then come back after the Dig-In to see if your guess was correct!
  \end{freeResponse}
\end{problem}

%%% \begin{xarmaBoost}
%%   Write down at least \textbf{five} questions for this lecture. After
%%   you have your questions, label them as ``Level 1,'' ``Level 2,'' or
%%   ``Level 3'' where:
%% \begin{description}
%% \item[Level 1] Means you know the answer, or know exactly how to do
%%   this problem.
%% \item[Level 2] Means you think you know how to do the problem.
%% \item[Level 3] Means you have no idea how to do the problem.
%% \end{description}
%% \begin{freeResponse}
%% \end{freeResponse}
%% \end{xarmaBoost}



\end{document}
