\documentclass{ximera}

\newcommand{\RR}{\mathbb R}
\renewcommand{\d}{\,d}
\newcommand{\dd}[2][]{\frac{d #1}{d #2}}
\renewcommand{\l}{\ell}
\newcommand{\ddx}{\frac{d}{dx}}
\newcommand{\dfn}{\textbf}
\newcommand{\eval}[1]{\bigg[ #1 \bigg]}


\title{Practice}

\begin{document}
\begin{abstract}
  Try these problems.
\end{abstract}
\maketitle


\begin{exercise}
\outcome{Define a vertical asymptote.}
\outcome{Calculate limits of the form number over zero.}
\outcome{Calculate limits of the form zero over zero.}
\tag{limit}
Consider 
\[a(x) = \frac{x^2+x-6}{x^2-x-12}.
\]
Find all vertical asymptotes.
\begin{prompt}
\begin{multipleChoice}
\choice[correct]{There are vertical asymptotes}
\choice{There are no vertical asymptotes}
\end{multipleChoice}
\begin{exercise} 
\[
x=\answer{4}
\]
\end{exercise}
\end{prompt}
\end{exercise}

\begin{exercise}
\outcome{Define a vertical asymptote.}
\outcome{Calculate limits of the form number over zero.}
\outcome{Calculate limits of the form zero over zero.}
\tag{limit}
Consider 
\[s(x) = \frac{x^2-2 x-15}{x^2+3 x+2}.
\]
Find all vertical asymptotes.
\begin{prompt}
\begin{multipleChoice}
\choice[correct]{There are vertical asymptotes}
\choice{There are no vertical asymptotes}
\end{multipleChoice}
\begin{exercise}Write your answers from least to greatest:
\[
x=\answer{-2}\qquad\text{and}\qquad x=\answer{-1}
\]
\end{exercise}
\end{prompt}
\end{exercise}


\begin{exercise}

\tag{derivative}

\outcome{Identify where a function is and is not continuous.}

Give intervals on which each of the following functions are
continuous. Write combinations of intervals going from left to right
on the number line.

\begin{enumerate}
\item $\frac{1}{e^x+1}$ is continuous on $(\answer{-\infty},\answer{\infty})$.
\item $\frac{1}{x^2-1}$ is continuous on $(\answer{-\infty},\answer{-1})$ and $(\answer{-1},\answer{1})$ and $(\answer{1},\answer{\infty})$.
\item $\sqrt{5-x}$ is continuous on $(\answer{-\infty},\answer{5}]$.
\item $\sqrt{5-x^2}$ is continuous on $[\answer{-5},\answer{5}]$.
\end{enumerate}

\end{exercise}




\begin{exercise}

\outcome{Calculate limits of piecewise functions.}
\outcome{Identify where a function is and is not continuous.}

\tag{piecewise}
\tag{continuity}

Let
\[
f(x) =
\begin{cases}
  x^2-1, &x < 3 \\
  x+5,  &x\geq 3.
\end{cases}
\]
Is $f$ continuous everywhere? 

\begin{multipleChoice}
\choice[correct]{Yes}
\choice{No}
\end{multipleChoice}

\end{exercise}







\begin{exercise}

\outcome{Calculate limits using the limit laws.}
\outcome{Calculate limits of piecewise functions.}
\outcome{Evaluate the limit as x approaches a point where there is a vertical asymptote.}
%\outcome{Calculate the limit as x approaches $\pm\infty$ of common functions algebraically.}

\tag{limits}
Let
\[
g(x)=\begin{cases}
-2e^{x} & \text{if $x<0$}\\
\frac{x+6}{x-3} & \text{if $x\ge0$}.
\end{cases}
\]

Find
\begin{enumerate}
\item		$\lim_{x\to 0^{-}} g(x)\begin{prompt} = \answer{-2}\end{prompt}$
\item		$\lim_{x\to 0^+} g(x)\begin{prompt} = \answer{-2}\end{prompt}$
\item		$\lim_{x\to 3^{-}} g(x)\begin{prompt} = \answer{-\infty}\end{prompt}$
\item		$\lim_{x\to 3^{+}} g(x)\begin{prompt} = \answer{\infty}\end{prompt}$
\item		$\lim_{x\to -\infty} g(x)\begin{prompt} = \answer{0}\end{prompt}$
\item		$\lim_{x\to +\infty} g(x)\begin{prompt} = \answer{1}\end{prompt}$
\end{enumerate}
\end{exercise}

\begin{exercise}
\outcome{State the Intermediate Value Theorem including hypotheses.}
\tag{continuity}
\tag{intermediate value theorem}
The Intermediate Value Theorem states:
If $f$ is a continuous function for all $x$ in the closed interval
$[a,b]$ and $r$ is between
\wordChoice{\choice{$a$}\choice[correct]{$f(a)$}} and
\wordChoice{\choice{$b$}\choice[correct]{$f(b)$}}, then there is a
number \wordChoice{\choice[correct]{$u$}\choice{$f(u)$}} in
  \wordChoice{\choice{$[f(a),f(b)]$}\choice[correct]{$[a, b]$}} such that
  \wordChoice{\choice[correct]{$f(u) = r$}\choice{$f(r) = u$}}.
\begin{hint}
Consider the following graph:
\begin{image}
\begin{tikzpicture}
\begin{axis}[
            domain=0:6, ymin=0, ymax=2.2,xmax=6,
            axis lines =left, xlabel=$x$, ylabel=$y$,
            every axis y label/.style={at=(current axis.above origin),anchor=south},
            every axis x label/.style={at=(current axis.right of origin),anchor=west},
            xtick={1,3.597,5}, ytick={.203,1,1.679},
            xticklabels={$a$,$u$,$b$}, yticklabels={$f(a)$,$r$,$f(b)$},
            axis on top,
          ]
          \addplot [draw=none, fill=fill2, domain=(0:7)] {1.679} \closedcycle;
          \addplot [draw=none, fill=background, domain=(0:7)] {.203} \closedcycle;
          \addplot [textColor,dashed] plot coordinates {(0,1.679) (6,1.679)};
          \addplot [textColor,dashed] plot coordinates {(0,.203) (6,.203)};
          \addplot [textColor,dashed] plot coordinates {(5,0) (5,1.679)};
          \addplot [textColor,dashed] plot coordinates {(1,0) (1,.203)};
          \addplot [textColor,dashed] plot coordinates {(3.587,0) (3.597,1)};
          \addplot [penColor2,domain=(0:6)] {1};
          \addplot [very thick,penColor, smooth,domain=(0:2.5)] {sin(deg((x - 4)/2)) + 1.2};
          \addplot [very thick,penColor, smooth,domain=(4:6)] {sin(deg((x - 4)/2)) + 1.2};
          \addplot [very thick,dashed,penColor!50!background, smooth,domain=(2.5:4)] {sin(deg((x - 4)/2)) + 1.2}; 
          \addplot [color=penColor!50!background,fill=penColor!50!background,only marks,mark=*] coordinates{(3.587,1)};  %% closed hole          
          \addplot [color=penColor,fill=penColor,only marks,mark=*] coordinates{(1,.203)};  %% closed hole          
          \addplot [color=penColor,fill=penColor,only marks,mark=*] coordinates{(5,1.679)};  %% closed hole          
        \end{axis}
\end{tikzpicture}
\end{image}
\end{hint}
\end{exercise}

\begin{exercise}

\outcome{Understand the connection between continuity of a function and the value of a limit.}
\outcome{Understand what it means for a function to be continuous.}

\tag{continuity}

Is the function
\[
f(x)=\left\{\begin{array}{ccc} 
\frac{x^2-64}{x^2-11 x+24},		& & x\ne8\\
5, & & x=8
\end{array}\right.
\]
continuous at $x=0$ or $x=8$?

\begin{prompt}
\begin{multipleChoice}
\choice{$f$ is continuous at both $x=0$ and $x=8$.}
\choice[correct]{$f$ is continuous at $x=0$ but not at $x=8$.}
\choice{$f$ is continuous at at $x=8$ but not at $x=0$.}
\choice{$f$ is not continuous at $x=0$ and $x=8$.}
\end{multipleChoice}
\end{prompt}

\end{exercise}

\begin{exercise}

\outcome{Explain why certain points exist using the Intermediate Value Theorem.}

\tag{continuity}
\tag{intermediate value theorem}

Let $f$ be continuous on $\left[1,5\right]$ where $f(1)=-2$ and $f(5)=-10$. Does a value $1<c<5$ exist such that $f(c)=-9$?

\begin{multipleChoice}
\choice{There does not exist a value.}
\choice[correct]{Yes, by the Intermediate Value Theorem}
\choice{Yes, by the Mean Value Theorem}
\choice{There does not necessarily exist such a value}
\end{multipleChoice}

\end{exercise}


\end{document}
