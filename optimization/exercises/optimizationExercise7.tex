\documentclass{ximera}

\newcommand{\RR}{\mathbb R}
\renewcommand{\d}{\,d}
\newcommand{\dd}[2][]{\frac{d #1}{d #2}}
\renewcommand{\l}{\ell}
\newcommand{\ddx}{\frac{d}{dx}}
\newcommand{\dfn}{\textbf}
\newcommand{\eval}[1]{\bigg[ #1 \bigg]}


%\outcome{Describe the goals of optimization problems generally.}
%\outcome{Find all local maximums and minimums using the First and Second Derivative tests.}
%\outcome{Identify when we can find an absolute maximum or minimum on an open interval.}
%\outcome{Contrast optimization on open and closed intervals.}
%\outcome{Describe the objective function and constraints in a given optimization problem.}
%\outcome{Solve optimization problems by finding the appropriate extreme values.}

\author{Nela Lakos \and Kyle Parsons}

\begin{document}
\begin{exercise}

A right circular cone is to be made with a fixed slant height of 8 ft.
\begin{image}
\begin{tikzpicture}

\coordinate (c) at (0,0);
\coordinate (t) at (0,3);
\coordinate (r) at (2,0);
\coordinate (l) at (-2,0);

\filldraw[color=white,fill=penColor,fill opacity=0.3] {(c)++(0,-0.02)} circle [x radius = 2.002,y radius=0.3];

\draw[penColor] (l) ++ (0,-0.02) arc[x radius=2.002,y radius=.3,start angle=180,end angle=366];
\draw[dashed,penColor] (l) ++ (0,-0.02) arc[x radius=2.002,y radius=.3,start angle=180,end angle=0];
\draw[penColor] {(l)++(0.005,0)} -- (t) -- node[above,sloped]{\tiny slant height} (r);

\draw[penColor,dashed] (c) -- node[above,sloped]{\tiny height} (t);

\node at (c) [penColor] {\tiny\textbullet};

\end{tikzpicture}
\end{image}

The height that maximizes the volume of the cone is
\[
h = \answer{\frac{8}{\sqrt{3}}}
\]



\end{exercise}
\end{document}
