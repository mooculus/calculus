\documentclass{ximera}

\newcommand{\RR}{\mathbb R}
\renewcommand{\d}{\,d}
\newcommand{\dd}[2][]{\frac{d #1}{d #2}}
\renewcommand{\l}{\ell}
\newcommand{\ddx}{\frac{d}{dx}}
\newcommand{\dfn}{\textbf}
\newcommand{\eval}[1]{\bigg[ #1 \bigg]}


\title[Dig-In:]{Exponential and logarithmetic functions}


\begin{document}
\begin{abstract}
  Hello
\end{abstract}
\maketitle

Exponential functions are useful in many ... 

\section{What are exponential and logarithmetic functions?}


\begin{definition}
  An \dfn{exponential function} is a function of the form
  \[
  f(x) = a^x
  \]
  where the $a\ne 1$ is a positive real number. The domain of an
  exponetial function is $(-\infty,\infty)$.
\end{definition}


\section{What can the graphs look like?}

For each positive real number $a$, we now have (sort of) defined a
function $f(x) = a^x$, the exponential function with base $a$, which
is defined for all real numbers $x$.  For positive integers, it agrees
with the repeated multiplication definition, and for non-positive
integers and rationals it is defined in a way that it continues to
enjoy the nice algebraic properties ($a^{b+c} = a^b \cdot a^c$ and
$(a^b)^c = a^{bc}$) we expect of it.  For irrational numbers we do not
yet have a solid definition, but we will learn some in the next
chapters.  Basically the values at the irrationals are determined by
approximating the irrationals with rationals.



\section{Rules of exponential functions and logarithms}
Rules of logs and exp


\end{document}
