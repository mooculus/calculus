\documentclass{ximera}

\newcommand{\RR}{\mathbb R}
\renewcommand{\d}{\,d}
\newcommand{\dd}[2][]{\frac{d #1}{d #2}}
\renewcommand{\l}{\ell}
\newcommand{\ddx}{\frac{d}{dx}}
\newcommand{\dfn}{\textbf}
\newcommand{\eval}[1]{\bigg[ #1 \bigg]}



\outcome{Use integral notation for both antiderivatives and definite integrals.}
\outcome{Compute definite integrals using geometry.}
\outcome{Compute definite integrals using the properties of integrals.}
%\outcome{Justify the properties of definite integrals using algebra or geometry.}
%\outcome{Understand how Riemann sums are used to find exact area.}
\outcome{Define net area.}
%\outcome{Approximate net area.}
\outcome{Split the area under a curve into several pieces to aid with calculations.}
%\outcome{Use symmetry to calculate definite integrals.}
%\outcome{Explain geometrically why symmetry of a function simplifies calculation of some definite integrals.}

\author{Nela Lakos \and Kyle Parsons}

\begin{document}
\begin{exercise}

The graph of each function below is constructed from a quarter circle and two line segments.  Using geometry, compute the following integrals.

\begin{image}
  \begin{tikzpicture}
    \begin{axis}[
        xmin=-0.3,xmax=6.3,ymin=-0.3,ymax=2.3,
        clip=true,
        unit vector ratio*=1 1 1,
        axis lines=center,
        grid = major,
        ytick={0,1,...,36},
        xtick={0,1,...,6},
        xlabel=$x$, ylabel=$y$,
        every axis y label/.style={at=(current axis.above origin),anchor=south},
        every axis x label/.style={at=(current axis.right of origin),anchor=west},
        compat = 1.5.1
      ]
      \fill [fill=penColor,fill opacity=0.5] (axis cs:0,0) -- (axis cs:2,0) arc[radius=200,start angle=0,end angle = 90] -- cycle;
      \fill [fill=penColor,fill opacity=0.5] (axis cs:2,0) -- (axis cs:6,0) -- (axis cs:6,2) -- (axis cs:4,2) -- cycle;
     
      \draw [ultra thick,penColor] (axis cs:2,0) arc[radius=200,start angle=0,end angle=90];
      \draw [ultra thick,penColor] (axis cs:2,0) -- (axis cs:4,2) -- (axis cs:6,2);
        
      \node at (axis cs:2.5,1.5) {$f(x)$};
      \end{axis}`
  \end{tikzpicture}
\end{image}
\begin{hint}

$ \int_0^6 f(x) \d x =\int_0^2 f(x) \d x +\int_2^4 f(x) \d x +\int_4^6 f(x) \d x = $

=Area(1/4 of a circle)+Area(triangle)+Area (square)

\end{hint}
\[
\int_0^6 f(x) \d x = \answer{\pi+6}
\]

\hrulefill

\begin{image}
  \begin{tikzpicture}
    \begin{axis}[
        xmin=-0.3,xmax=6.3,ymin=-2.3,ymax=2.3,
        clip=true,
        unit vector ratio*=1 1 1,
        axis lines=center,
        grid = major,
        ytick={-2,-1,...,36},
        xtick={0,1,...,6},
        xlabel=$x$, ylabel=$y$,
        every axis y label/.style={at=(current axis.above origin),anchor=south},
        every axis x label/.style={at=(current axis.right of origin),anchor=west},
        compat = 1.5.1
      ]
      \fill [fill=penColor,fill opacity=0.5] (axis cs:0,0) -- (axis cs:2,0) arc[radius=200,start angle=0,end angle = 90] -- cycle;
      \fill [fill=penColor,fill opacity=0.5] (axis cs:2,0) -- (axis cs:6,0) -- (axis cs:6,-2) -- (axis cs:4,-2) -- cycle;
     
      \draw [ultra thick,penColor] (axis cs:2,0) arc[radius=200,start angle=0,end angle=90];
      \draw [ultra thick,penColor] (axis cs:2,0) -- (axis cs:4,-2) -- (axis cs:6,-2);
        
      \node at (axis cs:2.5,1.5) {$f(x)$};
      \end{axis}`
  \end{tikzpicture}
\end{image}
\[
\int_0^6 f(x) \d x = \answer{\pi-6}
\]

\hrulefill

\begin{image}
  \begin{tikzpicture}
    \begin{axis}[
        xmin=-0.3,xmax=6.3,ymin=-2.3,ymax=2.3,
        clip=true,
        unit vector ratio*=1 1 1,
        axis lines=center,
        grid = major,
        ytick={-2,-1,...,36},
        xtick={0,1,...,6},
        xlabel=$x$, ylabel=$y$,
        every axis y label/.style={at=(current axis.above origin),anchor=south},
        every axis x label/.style={at=(current axis.right of origin),anchor=west},
        compat = 1.5.1
      ]
      \fill [fill=penColor,fill opacity=0.5] (axis cs:0,0) -- (axis cs:0,-2) arc[radius=200,start angle=-90,end angle=0] -- cycle;
      \fill [fill=penColor,fill opacity=0.5] (axis cs:2,0) -- (axis cs:6,0) -- (axis cs:6,-2) -- (axis cs:4,-2) -- cycle;
     
      \draw [ultra thick,penColor] (axis cs:0,-2) arc[radius=200,start angle=-90,end angle=0];
      \draw [ultra thick,penColor] (axis cs:2,0) -- (axis cs:4,-2) -- (axis cs:6,-2);
        
      \node at (axis cs:2.5,1.5) {$f(x)$};
      \end{axis}`
  \end{tikzpicture}
\end{image}
\[
\int_0^6 f(x) \d x = \answer{-\pi-6}
\]

\hrulefill

\begin{image}
  \begin{tikzpicture}
    \begin{axis}[
        xmin=-0.3,xmax=6.3,ymin=-2.3,ymax=2.3,
        clip=true,
        unit vector ratio*=1 1 1,
        axis lines=center,
        grid = major,
        ytick={-2,-1,...,36},
        xtick={0,1,...,6},
        xlabel=$x$, ylabel=$y$,
        every axis y label/.style={at=(current axis.above origin),anchor=south},
        every axis x label/.style={at=(current axis.right of origin),anchor=west},
        compat = 1.5.1
      ]
      \fill [fill=penColor,fill opacity=0.5] (axis cs:0,0) -- (axis cs:0,-2) arc[radius=200,start angle=-90,end angle=0] -- cycle;
      \fill [fill=penColor,fill opacity=0.5] (axis cs:2,0) -- (axis cs:6,0) -- (axis cs:6,2) -- (axis cs:4,2) -- cycle;
     
      \draw [ultra thick,penColor] (axis cs:0,-2) arc[radius=200,start angle=-90,end angle=0];
      \draw [ultra thick,penColor] (axis cs:2,0) -- (axis cs:4,2) -- (axis cs:6,2);
        
      \node at (axis cs:2.5,1.5) {$f(x)$};
      \end{axis}`
  \end{tikzpicture}
\end{image}
\[
\int_0^6 f(x) \d x = \answer{-\pi+6}
\]

\hrulefill

\begin{image}
  \begin{tikzpicture}
    \begin{axis}[
        xmin=-0.3,xmax=6.3,ymin=-2.3,ymax=2.3,
        clip=true,
        unit vector ratio*=1 1 1,
        axis lines=center,
        grid = major,
        ytick={-2,-1,...,36},
        xtick={0,1,...,6},
        xlabel=$x$, ylabel=$y$,
        every axis y label/.style={at=(current axis.above origin),anchor=south},
        every axis x label/.style={at=(current axis.right of origin),anchor=west},
        compat = 1.5.1
      ]
      \fill [fill=penColor,fill opacity=0.5] (axis cs:0,0) -- (axis cs:0,-2) -- (axis cs:2,-2) arc[radius=200,start angle=-90,end angle=0] -- cycle;
      \fill [fill=penColor,fill opacity=0.5] (axis cs:4,0) -- (axis cs:6,0) -- (axis cs:6,2) -- cycle;
     
      \draw [ultra thick,penColor] (axis cs:0,-2) -- (axis cs:2,-2) arc[radius=200,start angle=-90,end angle=0];
      \draw [ultra thick,penColor] (axis cs:4,0) -- (axis cs:6,2);
        
      \node at (axis cs:2.5,1.5) {$f(x)$};
      \end{axis}`
  \end{tikzpicture}
\end{image}
\[
\int_0^6 f(x) \d x = \answer{-\pi-2}
\]

\hrulefill

\begin{image}
  \begin{tikzpicture}
    \begin{axis}[
        xmin=-0.3,xmax=6.3,ymin=-2.3,ymax=2.3,
        clip=true,
        unit vector ratio*=1 1 1,
        axis lines=center,
        grid = major,
        ytick={-2,-1,...,36},
        xtick={0,1,...,6},
        xlabel=$x$, ylabel=$y$,
        every axis y label/.style={at=(current axis.above origin),anchor=south},
        every axis x label/.style={at=(current axis.right of origin),anchor=west},
        compat = 1.5.1
      ]
      \fill [fill=penColor,fill opacity=0.5] (axis cs:0,0) -- (axis cs:4,0) arc[radius=200,start angle=0,end angle=90] -- cycle;
      \fill [fill=penColor,fill opacity=0.5] (axis cs:4,0) -- (axis cs:6,0) -- (axis cs:6,0) -- (axis cs:4,0) -- cycle;
     
      \draw [ultra thick,penColor] (axis cs:0,0) -- (axis cs:2,2) arc[radius=200,start angle=90,end angle=0];
      \draw [ultra thick,penColor] (axis cs:4,0) -- (axis cs:6,0);
        
      \node at (axis cs:2.5,1.5) {$f(x)$};
      \end{axis}`
  \end{tikzpicture}
\end{image}
\[
\int_0^6 f(x) \d x = \answer{2+\pi}
\]


\end{exercise}
\end{document}