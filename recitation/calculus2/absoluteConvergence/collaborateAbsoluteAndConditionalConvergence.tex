\documentclass[noauthor]{ximera}
%handout:  for handout version with no solutions or instructor notes
%handout,instructornotes:  for instructor version with just problems and notes, no solutions
%noinstructornotes:  shows only problem and solutions

%% handout
%% space
%% newpage
%% numbers
%% nooutcomes

%I added the commands here so that I would't have to keep looking them up
%\newcommand{\RR}{\mathbb R}
%\renewcommand{\d}{\,d}
%\newcommand{\dd}[2][]{\frac{d #1}{d #2}}
%\renewcommand{\l}{\ell}
%\newcommand{\ddx}{\frac{d}{dx}}
%\everymath{\displaystyle}
%\newcommand{\dfn}{\textbf}
%\newcommand{\eval}[1]{\bigg[ #1 \bigg]}

%\begin{image}
%\includegraphics[trim= 170 420 250 180]{Figure1.pdf}
%\end{image}

%add a ``.'' below when used in a specific directory.

\newcommand{\RR}{\mathbb R}
\renewcommand{\d}{\,d}
\newcommand{\dd}[2][]{\frac{d #1}{d #2}}
\renewcommand{\l}{\ell}
\newcommand{\ddx}{\frac{d}{dx}}
\newcommand{\dfn}{\textbf}
\newcommand{\eval}[1]{\bigg[ #1 \bigg]}




\author{Tom Needham and Jim Talamo}

\outcome{Determine whether a series converges absolutely or conditionally.}

\title[Collaborate:]{Absolute and Conditional Convergence}

\begin{document}
\begin{abstract}
\end{abstract}
\maketitle

\section{Discussion Questions}

\begin{problem}
Consider the series $\sum_{k=1}^{\infty} \frac{(-1)^k}{k^p}$. 

\begin{itemize}
\item For which values of $p$ does the series converge? 
\item For which values of $p$ does the series converge absolutely? 
\item For which values of $p$ does the series converge conditionally? 
\end{itemize}

\begin{freeResponse}
First, we make two important observations.

\begin{itemize}
\item The series is alternating.
\item If $p \leq 0$, the series will diverge by the divergence test since $\lim_{k \to \infty} \frac{1}{k^p}$ will not exist.
\end{itemize}

So, we need only to consider $p>0$.  Note that for any $p>0$, $\frac{1}{k^p}$ is decreasing and $\lim_{k \to \infty} \frac{1}{k^p}$, so the alternating series test ensures that $\sum_{k=1}^{\infty} \frac{1}{k^p}$ will converge.  

Now recall the definitions of absolute and conditional convergence.

\begin{definition}
Consider the series $\sum_{k=n_0}^{\infty} a_k$. 

\begin{itemize}
\item $\sum_{k=n_0}^{\infty} a_k$ converges absolutely if $\sum_{k=n_0}^{\infty} \left|a_k\right|$ converges.
\item $\sum_{k=n_0}^{\infty} a_k$ converges conditionally if $\sum_{k=n_0}^{\infty} a_k$ converges but $\sum_{k=n_0}^{\infty} \left|a_k\right|$ diverges.
\end{itemize}
\end{definition}

Thus, to check for either absolute and conditional convergence, we need to check when $\sum_{k =1}^{\infty} \left| \frac{(-1)^k}{k^p} \right|$ converges.  Noting that $\sum_{k =1}^{\infty} \left| \frac{(-1)^k}{k^p} \right| = \sum_{k =1}^{\infty}  \frac{1}{k^p}$  is a $p$-series, we find $\sum_{k =1}^{\infty} \left| \frac{(-1)^k}{k^p} \right|$ converges when $p >1$.  Thus, the series 

\begin{itemize}
\item converges absolutely when $p>1$.
\item converges conditionally when $0<p<1$.
\end{itemize}

\end{freeResponse}
\end{problem}

\begin{problem}
A student is asked to determine if the series $\sum_{k=1}^{\infty} \frac{(-1)^k}{2k+1}$ converges and provides the following reasoning. 

\begin{quote}
We check the series for absolute convergence.   Since $\sum_{k=1}^{\infty} \left| \frac{(-1)^k}{2k+1} \right| = \sum_{k=1}^{\infty} \frac{1}{2k+1}$ and we can use the limit comparison test to show $\sum_{k=1}^{\infty} \frac{1}{2k+1}$ diverges, then the original series $\sum_{k=1}^{\infty} \frac{(-1)^k}{2k+1}$ does not converge absolutely.  Hence, it diverges.
\end{quote}
Determine if the student is correct or incorrect.  If the student is incorrect, provide a correct response.
\begin{freeResponse}
The student is incorrect. Indeed, the Alternating Series Test implies that the series converges, since the terms meet the hypotheses of the test and $\lim_{k \rightarrow \infty} 1/(2k+1)$. A likely mistake in the student's reasoning is that they were confused about the following (true) statement: If a series converges absolutely, then it must converge. The student apparently attempted to apply the converse of this statement (if a series does not converge absolutely, then it must diverge), which is not true.
\end{freeResponse}
\end{problem}

\begin{problem}
For the following, give an example of a series with the described properties or explain why no such series exists.

\begin{itemize}
\item[I.] A convergent series $\sum_{k=1}^{\infty} a_k$ for which $\sum_{k=1}^{\infty} |a_k|$ diverges.

\item[II.] A convergent series $\sum_{k=1}^{\infty} |a_k|$ for which $\sum_{k=1}^{\infty} a_k$ diverges.

\item[III.] A sequence $\{a_k\}$ of positive terms for which $\sum_{k=1}^{\infty} a_k$ converges conditionally.

\end{itemize}


\begin{freeResponse}
I. A basic example is the alternating harmonic series.

II. No such example can exist. If a series converges absolutely, then it must converge.

III. No such example can exist. A series converges conditionally if $\sum |a_k|$ diverges but $\sum a_k$ converges. Since all $a_k$ are assumed to be positive, $|a_k| = a_k$ for all $k$, so it is not possible for the series to converge conditionally.
\end{freeResponse}
\end{problem}

\section{Group Work}

\begin{problem}
Determine if the series $\sum_{k=1}^\infty \frac{(-1)^k \cdot  (k+2)}{k^3+\ln(k) +10k}$ converges absolutely, converges conditionally, or diverges.

\begin{freeResponse}
We first test for absolute convergence by checking the convergence properties of the series 
$$
\sum_{k=1}^\infty \left|\frac{(-1)^k \cdot  (k+2)}{k^3+\ln(k) +10k}\right| = \sum_{k=1}^\infty \frac{k+2}{k^3+\ln(k) +10k}.
$$
A straightforward application of the limit comparison test, comparing with the series $\sum 1/k^2$, shows that this series converges. We conclude that the series converges absolutely. 
\end{freeResponse}
\end{problem}

\begin{problem}
Let $\{s_n\}$ denote the sequence of partial sums for the series $\sum_{k=1}^\infty a_k$ and suppose it is known that $s_n = \frac{\sin(n)}{n^2}.$
\begin{itemize}
\item[I.] Determine if $\sum_{k=1}^{\infty} a_k$ converges.
\item[II.] Determine if $\sum_{k=1}^{\infty} s_k$ converges.
\end{itemize}

\begin{freeResponse}
I. The series $\sum a_k$ converges to $\lim_{n\rightarrow \infty} s_n = 0$, by the definition of series convergence.

II. First consider the series 
$$
\sum_{k=1}^{\infty} |s_k| = \sum_{k=1}^{\infty} \left|\frac{\sin(k)}{k^2} \right|.
$$
Since $|\sin(x)| \leq 1$ for all $x$, we have 
$$
\left|\frac{\sin(k)}{k^2} \right| \leq \frac{1}{k^2}
$$
for all $k$. Since the $p$-Series Test tells us that the series $\sum 1/k^2$ converges, The Direct Comparison Test therefore implies that $\sum |s_k|$ converges. It follows that $\sum s_k$ converges absolutely, and therefore converges.
\end{freeResponse}
\end{problem}

\end{document}
