\documentclass{ximera}

\newcommand{\RR}{\mathbb R}
\renewcommand{\d}{\,d}
\newcommand{\dd}[2][]{\frac{d #1}{d #2}}
\renewcommand{\l}{\ell}
\newcommand{\ddx}{\frac{d}{dx}}
\newcommand{\dfn}{\textbf}
\newcommand{\eval}[1]{\bigg[ #1 \bigg]}


\outcome{Understand what is meant by the form of a limit.}
\outcome{Calculate limits of the form zero over zero.}
\outcome{Identify determinate and indeterminate forms.}
\outcome{Distinguish between determinate and indeterminate forms.}

\author{Nela Lakos \and Kyle Parsons}

\begin{document}
\begin{exercise}

Let $g(x) = 2\left|x-1\right|$, and $h(x) = -(x-1)^2$.\\

For the following limits determine the form of the limit and compute the value.  Possible answers include $\infty$, $-\infty$ and $DNE$.

\[
\lim_{x\to1}\frac{g(x)}{h(x)} = \answer{-\infty}
\]

\begin{multipleChoice}
\choice{The limit is determinate.}
\choice[correct]{The limit is of the form $\zeroOverZero$}
\choice{The limit is of the form $\numOverZero$}
\end{multipleChoice}

\noindent\rule[0.5ex]{\linewidth}{0.2pt}

\[
\lim_{x\to1^-}\frac{g(x)-g(1)}{x-1} = \answer{-2}
\]

\begin{multipleChoice}
\choice{The limit is determinate.}
\choice[correct]{The limit is of the form $\zeroOverZero$}
\choice{The limit is of the form $\numOverZero$}
\end{multipleChoice}

\noindent\rule[0.5ex]{\linewidth}{0.2pt}

\[
\lim_{x\to1^+}\frac{g(x)-g(1)}{x-1} = \answer{2}
\]

\begin{multipleChoice}
\choice{The limit is determinate.}
\choice[correct]{The limit is of the form $\zeroOverZero$}
\choice{The limit is of the form $\numOverZero$}
\end{multipleChoice}

\noindent\rule[0.5ex]{\linewidth}{0.2pt}

\[
\lim_{x\to1}\frac{g(x)-g(1)}{x-1} = \answer{DNE}
\]

\begin{multipleChoice}
\choice{The limit is determinate.}
\choice[correct]{The limit is of the form $\zeroOverZero$}
\choice{The limit is of the form $\numOverZero$}
\end{multipleChoice}


\end{exercise}
\end{document}