\documentclass{ximera}

\newcommand{\RR}{\mathbb R}
\renewcommand{\d}{\,d}
\newcommand{\dd}[2][]{\frac{d #1}{d #2}}
\renewcommand{\l}{\ell}
\newcommand{\ddx}{\frac{d}{dx}}
\newcommand{\dfn}{\textbf}
\newcommand{\eval}[1]{\bigg[ #1 \bigg]}


\author{Jim Fowler}
\license{Creative Commons 4.0 By-SA}

\begin{document}

\begin{exercise}
  

  Suppose $f : \R^2 \to \R$ is a function with  $\displaystyle\frac{\partial f}{\partial x} > 0$ and $\displaystyle\frac{\partial f}{\partial y} < 0$.

  Suppose $g : \R^2 \to \R$  is a function with      $\displaystyle\frac{\partial g}{\partial x} < 0$ and $\displaystyle\frac{\partial g}{\partial y} > 0$,

  
Suppose  $h : \R^2 \to \R$  is a function with      $\displaystyle\frac{\partial h}{\partial x} > 0$ and  $\displaystyle\frac{\partial h}{\partial y} < 0$.

Then $\displaystyle\frac{\partial}{\partial x} f(g(x,y),h(x,y))$ is
\begin{multipleChoice}
  \choice[correct]{negative}
  \choice{zero}
  \choice{positive}
  \choice{unknowable from the given data.}      
\end{multipleChoice}

\begin{hint}
 By the chain rule,
 \begin{align*}
 \frac{\partial}{\partial x} f(g(x,y),h(x,y))
 &= \frac{\partial f}{\partial x} \frac{\partial g}{\partial x} + \frac{\partial f}{\partial y} \frac{\partial h}{\partial x} \\
 &= \mbox{positive} \cdot \mbox{negative} + \mbox{negative} \cdot \mbox{positive} \\
 &= \mbox{negative} + \mbox{negative} \\
 &= \mbox{negative},
 \end{align*}
\end{hint}
\begin{hint}
 So we conclude \fbox{$\displaystyle\frac{\partial}{\partial x} f(g(x,y),h(x,y)) < 0$}.
\end{hint}

\end{exercise}
\end{document}