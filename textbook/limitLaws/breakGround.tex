\documentclass{ximera}

\newcommand{\RR}{\mathbb R}
\renewcommand{\d}{\,d}
\newcommand{\dd}[2][]{\frac{d #1}{d #2}}
\renewcommand{\l}{\ell}
\newcommand{\ddx}{\frac{d}{dx}}
\newcommand{\dfn}{\textbf}
\newcommand{\eval}[1]{\bigg[ #1 \bigg]}


\outcome{}


\title[Break-Ground:]{Arithmetic of large and small}

\begin{document}
\begin{abstract}
Here we see a dialogue where we investigate the arithmetic of large
and small numbers.
\end{abstract}
\maketitle


Check out this dialogue between two calculus students (based on a true
story):

\begin{dialogue}
\item[Devyn] Hey Riley, remember when we did stuff with
  \textit{numbers} in math class?
\item[Riley] Oh the glory years! Now all we have are $a$, $b$, $c$, and $x$, $y$, $z$.
\item[Devyn] I know! Here's a crazy idea, let's replace those $a$,
  $b$, $c$'s, and $x$, $y$, $z$'s with ``large'' and ``small.''
\item[Riley] Wow. So you could do things like:
  \[
  \text{large}+\text{small}\qquad\text{and}\qquad \frac{\text{small}}{\text{large}}
  \]
\item[Devyn] And maybe
  \[
  \text{large}+\text{small} = \text{large}
  \]
  because the answer is still ``large.''
\item[Riley] Right! And maybe
  \[
  \frac{\text{small}}{\text{large}} = \text{small}
  \]
  because a small number divided by a large number is even smaller, and hence ``small.''
\item[Devyn] And now
  \[
  \frac{\text{large}}{\text{large}} = \dots
  \]
  hmmmm.
  \item[Riley] Yes, I'm not sure here. Because even if two numbers are
    ``large'' one may be way larger than the other.
\end{dialogue}

Big number times big number is big number.
Big number plus big number is big number
Big number dividied by big number is ??
\end{document}
