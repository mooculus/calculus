\documentclass{ximera}
\newcommand{\RR}{\mathbb R}
\renewcommand{\d}{\,d}
\newcommand{\dd}[2][]{\frac{d #1}{d #2}}
\renewcommand{\l}{\ell}
\newcommand{\ddx}{\frac{d}{dx}}
\newcommand{\dfn}{\textbf}
\newcommand{\eval}[1]{\bigg[ #1 \bigg]}

\author{Bart Snapp}
\license{Creative Commons 3.0 By-NC}
\outcome{Set up and evaluate a volume integral using the Washer Method}
\begin{document}

\begin{exercise} 
Use the Washer Method to write down an integral that gives the volume of a right circular cone with base radius $10$ and height $20$.

\[
\text{V} = 
  \int_{x=\answer{0}}^{x=\answer{20}} \answer{\frac{\pi}{4} x^2} \d x.
\]

Evaluating this integral shows that the volume is $\answer[given]{\frac{2000\pi}{3}}$ cubic units.


\begin{hint}
We can view this cone as produced by the line $y=x/2$ rotated about
the $x$-axis:
\begin{image}
\begin{tikzpicture}
  \begin{axis}[
      xmin=0, xmax=20,domain=0:20,
      clip=false, axis lines =center,
      xlabel=$x$, ylabel=$y$,
      every axis y label/.style={at=(current axis.above origin),anchor=south},
      every axis x label/.style={at=(current axis.right of origin),anchor=west},
      axis on top,
    ]
    \addplot [penColor,very thick,smooth,domain=0:20] {x/2};
    \addplot [penColor,very thick,smooth,domain=0:20] {-x/2};

    \draw[penColor,very thick,fill=fill1] (axis cs:11.5,0) ellipse (20 and 600);
    \draw[penColor,very thick,fill=fill1] (axis cs:12,0) ellipse (20 and 600);
    
    \draw[penColor,very thick] (axis cs:20,0) ellipse (20 and 1000);
    
    \draw[decoration={brace,mirror,raise=.1cm},decorate,thin] (axis cs:11,-6)--(axis cs:12.5,-6);  
    \node[anchor=north] at (axis cs:11.75,-6.4) {$\Delta x$};
  \end{axis}
\end{tikzpicture}
\end{image}

At a particular point on the $x$-axis, the radius of the resulting cone is the $y$-coordinate of the corresponding point on the line
$y=x/2$. The area of the cross-section is given by: 

\[
A(x) = \answer[given]{\pi \left(\frac{x}{2}\right)^2}
\]
The infinitesimal volume of each disc is then $A(x) \d x$, so the
total volume is the integral of these infinitesimal volumes from $x =
0$ to $x = 20$.
\[
\text{Volume} = 
  \int_0^{20} A(x) \d x=\answer[given]{\frac{2000\pi}{3}}.
\]
\end{hint}

\begin{exercise}
Recall from geometry that the volume of a right circular cone with base radius $r$ and height $h$ is given by $V =\frac{\pi}{3}r^2h$.  

Using this formula, we find that the volume of the cone is $\answer{\frac{2000 \pi}{3}}$ cubic units.

Do the previous results agree?

\begin{multipleChoice}
\choice[correct]{Yes}
\choice{No}
\end{multipleChoice}
\end{exercise}
\end{exercise}

\end{document}