\documentclass{ximera}

\newcommand{\RR}{\mathbb R}
\renewcommand{\d}{\,d}
\newcommand{\dd}[2][]{\frac{d #1}{d #2}}
\renewcommand{\l}{\ell}
\newcommand{\ddx}{\frac{d}{dx}}
\newcommand{\dfn}{\textbf}
\newcommand{\eval}[1]{\bigg[ #1 \bigg]}


\title[Dig-In:]{Limits by the definition}

\begin{document}
\begin{abstract}
  The definition of a limit gives rigor to the concepts of calculus.
\end{abstract}
\maketitle



Now we are going to get our hands dirty, and really use the definition
of a limit.
%% \marginnote{Recall, $\lim_{x\to a}f(x)=L$, if for every
%%   $\epsilon>0$ there is a $\delta > 0$ so that whenever $0< |x -a|<
%%   \delta$, we have $|f(x)- L|<\epsilon$.}

\begin{image}
\begin{tikzpicture}
	\begin{axis}[
            domain=1:3, 
            axis lines =left, xlabel=$x$, ylabel=$y$,
            every axis y label/.style={at=(current axis.above origin),anchor=south},
            every axis x label/.style={at=(current axis.right of origin),anchor=west},
            xtick={1.8,2,2.2}, ytick={3,4,5},
            xticklabels={$2-\delta$,$2$,$2+\delta$}, yticklabels={$4-\epsilon$,$4$,$4+\epsilon$},
            axis on top,
          ]          
          \addplot [color=textColor, fill=fill2, smooth, domain=(1:2.236)] {5} \closedcycle;
          \addplot [color=textColor, dashed, fill=fill1, domain=(1:2.2)] {4.84} \closedcycle;       
          \addplot [color=textColor, dashed, fill=fill2, domain=(1:1.8)] {3.24} \closedcycle;       
          \addplot [textColor, very thick, smooth, domain=(1:2)] {4};
          \addplot [color=textColor, fill=background, smooth, domain=(1:1.8)] {3} \closedcycle;
	  \addplot [draw=none, fill=background, smooth] {x^2} \closedcycle;
          \addplot [fill=fill1, draw=none, domain=1.8:2.2] {x^2} \closedcycle;
          \addplot [textColor, very thick] plot coordinates {(2,0) (2,4)};
          \addplot [textColor] plot coordinates {(1.8,0) (1.8,3.24)};
          \addplot [textColor] plot coordinates {(2.2,0) (2.2,4.84)};
	  \addplot [very thick,penColor, smooth] {x^2};
        \end{axis}
\end{tikzpicture}
%% \label{plot:x^2 lim dfn}
%% \caption{The $(\epsilon,\delta)$-criterion for $\lim_{x\to 2}
%%   x^2=4$. Here $\delta= \min\left(\dfrac{\epsilon}{5},1\right)$.}
\end{image}

\begin{example} Show that $\lim_{x\to 2} x^2=4$.

  
We want to show that for any given $\epsilon>0$, we can find a
$\delta>0$ such that
\[
|x^2 -4|<\epsilon
\]
whenever $0<|x - 2|<\delta$. Start by factoring the left-hand side of
the inequality above
\[
|x+2||x-2|<\epsilon.
\]
Since we are going to assume that $0<|x - 2|<\delta$, we will focus on
the factor $|x+2|$. Since $x$ is assumed to be close to $2$, suppose that $x\in[1,3]$. In this case
\[
|x+2| \le 3+2 = 5,
\]
and so we want
\begin{align*}
5\cdot |x-2| &< \epsilon\\
|x-2| &< \frac{\epsilon}{5}
\end{align*}
Recall, we assumed that $x\in[1,3]$, which is equivalent to
$|x-2|\le 1$. Hence we must set $\delta = \min\left(\dfrac{\epsilon}{5},1\right)$.
\end{example}


When dealing with limits of polynomials, the general strategy is
always the same. Let $p(x)$ be a polynomial. If showing
\[
\lim_{x\to a} p(x) = L,
\]
one must first factor out $|x-a|$ from $|p(x) - L|$. Next bound $x\in
[a-1,a+1]$ and estimate the largest possible value of
\[
\left|\frac{p(x) -L}{x-a}\right|
\]
for $x\in[a-1,a+1]$, call this estimation $M$. Finally, one must set
$\delta = \min\left(\frac{\epsilon}{M}, 1\right)$.

As you work with limits, you find that you need to do the same
procedures again and again. The next theorems will expedite this
process.
\begin{theorem}[Limit Product Law]\label{theorem:limit-product} 
Suppose $\lim_{x\to a} f(x)=L$ and $\lim_{x\to a}g(x)=M$. Then
\[
\lim_{x\to a} f(x)g(x) = LM.
\]
\end{theorem}

%% \marginnote[1.5in]{We will use this same trick again of ``adding $0$'' in
%%   the proof of Theorem~\ref{theorem:product-rule}.}
%% \marginnote[.5in]{This is all straightforward except perhaps for the
%%   ``$\le$''. This follows from the \textit{Triangle
%%     Inequality}\index{triangle inequality}. The \textbf{Triangle
%%     Inequality} states: If $a$ and $b$ are any real numbers then
%%   $|a+b|\le |a|+|b|$.}

\begin{proof} 
Given any $\epsilon$ we need to find a $\delta$ such that
\[
0<|x - a|< \delta
\]
implies 
\[
|f(x)g(x)-LM|< \epsilon.  
\]
Here we use an algebraic trick, add $0 = -f(x)M+f(x)M$:
\begin{align*}
|f(x)g(x)-LM| &= |f(x)g(x){\color{penColor2}-f(x)M+f(x)M}-LM| \\
&=|f(x)(g(x)-M)+(f(x)-L)M| \\
&\le |f(x)(g(x)-M)|+|(f(x)-L)M| \\
&=|f(x)||g(x)-M|+|f(x)-L||M|.
\end{align*}
Since $\lim_{x\to a}f(x) =L$, there is a value $\delta_1$ so that
$0<|x-a|<\delta_1$ implies $|f(x)-L|<|\epsilon/(2M)|$. This means that
$0<|x-a|<\delta_1$ implies $|f(x)-L||M|< \epsilon/2$. 
\[
|f(x)g(x)-LM|\le|f(x)||g(x)-M|+\underbrace{|f(x)-L||M|}_{<\dfrac{\epsilon}{2}}.
\]
If we can make $|f(x)||g(x)-M|<\epsilon/2$, then we'll be done. We can
make $|g(x)-M|$ smaller than any fixed number by making $x$ close
enough to $a$. Unfortunately, $\epsilon/(2f(x))$ is not a fixed number
since $x$ is a variable.

Here we need another trick. We can find a $\delta_2$ so that
$|x-a|<\delta_2$ implies that $|f(x)-L|<1$, meaning that $L-1 < f(x) <
L+1$. This means that $|f(x)|<N$, where $N$ is either $|L-1|$ or
$|L+1|$, depending on whether $L$ is negative or positive. The
important point is that $N$ doesn't depend on $x$. Finally, we know
that there is a $\delta_3$ so that $0<|x-a|<\delta_3$ implies
$|g(x)-M|<\epsilon/(2N)$. Now we're ready to put everything
together. Let $\delta$ be the smallest of $\delta_1$, $\delta_2$, and
$\delta_3$. Then $|x-a|<\delta$ implies that
\[
|f(x)g(x)-LM|\le\underbrace{|f(x)|}_{<N}\underbrace{|g(x)-M|}_{<\dfrac{\epsilon}{2N}}+\underbrace{|f(x)-L||M|}_{<\dfrac{\epsilon}{2}}.
\]
so
\begin{align*}
|f(x)g(x)-LM|&\le|f(x)||g(x)-M|+|f(x)-L||M| \\
&<N{\epsilon\over 2N}+\left|{\epsilon\over 2M}\right||M| \\
&={\epsilon\over 2}+{\epsilon\over 2}=\epsilon.
\end{align*}
This is just what we needed, so by the definition of a limit,
$\lim_{x\to a}f(x)g(x)=LM$.
\end{proof}



Another useful way to put functions together is
composition\index{composition of functions}. If $f(x)$ and $g(x)$ are
functions, we can form two functions by composition: $f(g(x))$ and
$g(f(x))$. For example, if $f(x)=\sqrt{x}$ and $g(x)=x^2+5$, then
$f(g(x))=\sqrt{x^2+5}$ and $g(f(x))=(\sqrt{x})^2+5=x+5$.  This brings
us to our next theorem.

%% \marginnote[.5in]{This is sometimes written as \[ 
%% \lim_{x\to a}f(g(x)) = \lim_{g(x)\to M}f(g(x)).
%% \]}
\begin{theorem}[Limit Composition Law]\label{thm:limit of composition}
Suppose that $\lim_{x\to a}g(x)=M$ and $\lim_{x\to M}f(x) = f(M)$. Then
\[
\lim_{x\to a} f(g(x)) = f(M).
\]
\end{theorem}

\begin{warning}
You may be tempted to think that the condition on $f(x)$ in
Theorem~\ref{thm:limit of composition} is unneeded, and that it will
always be the case that if $\lim_{x\to a}g(x)=M$ and 
$\lim_{x\to M}f(x) = L$ then
\[
\lim_{x\to a} f(g(x)) = L.
\]
However, consider
\[
f(x) =\begin{cases}
3 & \text{if $x=2$,}\\
4 & \text{if $x\ne 2$.}
\end{cases}
\]
and $g(x) = 2$. Now the conditions of Theorem~\ref{thm:limit of composition} are not satisfied, and
\[
\lim_{x\to 1} f(g(x)) = 3 \qquad{but}\qquad \lim_{x\to 2} f(x) = 4.
\]
\end{warning}


Many of the most familiar functions do satisfy the conditions of
Theorem~\ref{thm:limit of composition}. For example:

\begin{theorem}[Limit Root Law]
Suppose that $n$ is a positive integer. Then
$$\lim_{x\to a}\root n\of{x} = \root n\of{a},$$
provided that $a$ is positive if $n$ is even.
\label{thm:continuity of roots}
\end{theorem}

This theorem is not too difficult to prove from the definition of limit.

\end{document}
