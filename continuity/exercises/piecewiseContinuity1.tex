\documentclass{ximera}

\newcommand{\RR}{\mathbb R}
\renewcommand{\d}{\,d}
\newcommand{\dd}[2][]{\frac{d #1}{d #2}}
\renewcommand{\l}{\ell}
\newcommand{\ddx}{\frac{d}{dx}}
\newcommand{\dfn}{\textbf}
\newcommand{\eval}[1]{\bigg[ #1 \bigg]}


\outcome{Identify where a function is, and is not continuous.}
\outcome{Understand the connection between continuity of a function and the value of a limit.}
\outcome{Make a piecewise function continuous.}

\author{Nela Lakos \and Kyle Parsons \and Bobby Ramsey}

\begin{document}
\begin{exercise}

Consider

\[
g(x) = 
\begin{cases}
\frac{2x+b}{x-5} & \text{if }x<0\\ \\
\frac{x+16}{x^2-16} & \text{if }x\geq0\text{ and }x\neq4
\end{cases}
\]

We will find the constant $b$ so that $g$ is continuous at 0.  We will start with the definition of continuity: $\displaystyle \lim_{x\to0}g(x) = g(0)$.  

\begin{align*}
	\lim_{x\to0^-}g(x) &= \answer{-\frac{b}{5}}\\
	\lim_{x\to0^+} g(x)&= \answer{-1}
\end{align*}


In order for $\lim_{x\to0}g(x)$ to exist, we need $\lim_{x\to0^-}g(x) = \lim_{x\to0^+}g(x)$. For this we need $b = \answer{5}$.  

\begin{exercise}
	That verifies that $\displaystyle \lim_{x\to 0}g(x)$ exists. We now need to verify that $g(0)$ exists.
	
	\[g(0) = \answer{-1}\]



Since we have now verified that $\displaystyle \lim_{x\to0}g(x)$ exists, that $g(0)$ exists, and that they have the same value. That guarantees $g$ is continuous at 0.



\begin{exercise}

Let $b$ have the value you found above.  Then the intervals of continuity for $g$ from left to right are $\left(\answer{-\infty}\, , \,\answer{4}\right)\, , \,\left(\answer{4}\, ,\,\answer{\infty}\right)$.

\end{exercise}
\end{exercise}
\end{exercise}
\end{document}