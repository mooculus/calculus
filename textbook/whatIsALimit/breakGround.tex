\documentclass{ximera}

\newcommand{\RR}{\mathbb R}
\renewcommand{\d}{\,d}
\newcommand{\dd}[2][]{\frac{d #1}{d #2}}
\renewcommand{\l}{\ell}
\newcommand{\ddx}{\frac{d}{dx}}
\newcommand{\dfn}{\textbf}
\newcommand{\eval}[1]{\bigg[ #1 \bigg]}


\outcome{}


\title[Break-Ground:]{Stars and velocity}

\begin{document}
\begin{abstract}
Here we see a dialogue where two young mathematicians discuss stars
and instantaneous velocity.
\end{abstract}
\maketitle

Check out this dialogue between two calculus students (based on a true
story):

\begin{dialogue}
\item[Devyn] Riley, did you know I like looking at the stars at night?
\item[Riley] Stars are freaking awesome balls of nuclear fire whose
  light took thousands of years to reach us.
\item[Devyn] I know! But did you know that the best way to see a very
  dim star is to look \textbf{near} it but \textbf{not exactly at} it? It's
  because then you can use the ``rods'' in your eye, which work better
  in low light than the ``cones'' in your eyes.
\item[Riley] That's amazing! Hey! That reminds me of something\dots.
  How a GPS or a phone computes velocity!
\item[Devyn] This is gonna be good. Let's here it!
\item[Riley] To compute velocity from position look at
  \[
  \frac{\text{change in position}}{\text{change in time}}
  \]
\item[Devyn] And then we study this as the change in time gets closer
  and closer to zero.
\item[Riley] Just like with stars, we can study something by looking
  \textbf{near} a point, but \textbf{not exactly at} a point.
\item[Devyn] O.M.G.\ Life's a rich tapestry.
\item[Riley] Poet, you know it.
\end{dialogue}


\begin{problem}
  Suppose you take a road trip from Columbus Ohio to Urbana-Champaign
  Illinois. Moreover, suppose your position is modeled by
  \[
  s(t) = 36t^2 -4.8t^3 \qquad\text{(miles West of Columbus)} %% note the model is wrong
  \]
  where $t$ is measured in hours and runs from $0$ to $5$ hours. 
\end{problem}

\begin{problem}
  What is the average velocity for the entire trip?
\begin{hint}
  Remember, 
  \[
  \text{change in distance} = \text{rate}\cdot\text{change in time}.
  \]
\end{hint}
\begin{hint}
  So, 
  \[
  \frac{\Delta\text{distance}}{\Delta\text{time}} = \text{rate}.
  \]
\end{hint}
\begin{hint}
So, 
\[
\frac{\Delta\text{distance}}{\Delta\text{time}} = \frac{300}{5}.
\]
\end{hint}
\begin{prompt}
  The average velocity is \answer{60} miles per hour.
\end{prompt}
\end{problem}


\begin{problem}
  Use a calculator to estimate the instantaneous velocity at $t=2$.
  \begin{hint}
  Remember, 
  \[
  \text{change in distance} = \text{rate}\cdot\text{change in time}.
  \]
  \end{hint}
  \begin{hint}
  So, 
  \[
  \frac{\Delta\text{distance}}{\Delta\text{time}} = \text{rate}.
  \]
\end{hint}
  \begin{hint}
    Compute
    \[
    \frac{36(2+\Delta t)^2 -4.8(2+\Delta t)^3 -\left(36\cdot 2^2 -4.8\cdot 2^3\right) }{\Delta t}
    \]
    for smaller, and smaller values of $\Delta t$.
\end{hint}
  \begin{prompt}
    The instantaneous velocity, (rounded to the nearest tenth) is \answer{86.4} miles per hour.
  \end{prompt}
\end{problem}


\begin{problem}
  Considering the work above, when we want to compute instantaneous
  velocity, we need to compute
  \[
  \frac{\text{change in position}}{\text{change in time}}
  \]
  when (chose all that apply)
  \begin{multipleChoice}%% BADBAD
    \choice{The ``change in time'' is zero.}
    \choice{The ``change in time'' is near zero.}
    \choice[correct]{The ``change in time'' is gets closer and closer to zero.}
    \choice[correct]{The ``change in time'' approaches zero.}
    \choice[correct]{The ``change in time'' goes to zero.}
  \end{multipleChoice}
\end{problem}


Computing average velocities for smaller, and smaller, values of $h$
as we did above is tedious. Nevertheless, this is exactly how a GPS
determines velocity from position! On the other hand, we are human
beings and have better things to do than just compute all day
long. What would really help us out is a formula.



%% \begin{xarmaBoost}
%%   Write down at least \textbf{five} questions for this lecture. After
%%   you have your questions, label them as ``Level 1,'' ``Level 2,'' or
%%   ``Level 3'' where:
%% \begin{description}
%% \item[Level 1] Means you know the answer, or know exactly how to do
%%   this problem.
%% \item[Level 2] Means you think you know how to do the problem.
%% \item[Level 3] Means you have no idea how to do the problem.
%% \end{description}
%% \begin{freeResponse}
%% \end{freeResponse}
%% \end{xarmaBoost}



\end{document}
