\documentclass{ximera}

\newcommand{\RR}{\mathbb R}
\renewcommand{\d}{\,d}
\newcommand{\dd}[2][]{\frac{d #1}{d #2}}
\renewcommand{\l}{\ell}
\newcommand{\ddx}{\frac{d}{dx}}
\newcommand{\dfn}{\textbf}
\newcommand{\eval}[1]{\bigg[ #1 \bigg]}



%\outcome{Find tangent lines to parametric curves}
\author{Jim Talamo and Alex Beckwith}

\begin{document}
\begin{exercise}

Suppose that a curve $C$ is defined by the parametric equations:
\[
\begin{cases}
x(t) &= 2\cos(t)+\sin(t) \\
y(t) &= 4\cos(t)
\end{cases} , \textrm{ for } -\infty < t < \infty
\]
Find the derivative $\frac{dy}{dx}$ in terms of $t$.

\[
\frac{dy}{dx} = \frac{\answer{-4\sin(t)}}{\answer{-2\sin(t)+\cos(t)}}
\]

\begin{hint}
Using the chain rule allows us to write $\frac{dy}{dx} = \frac{dy/dt}{dx/dt}$.
\end{hint}

%%%Horizontal Tangent Lines%%%%%%
\begin{exercise}
How many horizontal tangent lines does the curve $C$ have?
\begin{multipleChoice}
\choice{$0$}
\choice{$1$}
\choice[correct]{$2$}
\choice{$3$}
\end{multipleChoice}

The horizontal tangent lines occur when $t=\answer{0}$ and $t=\answer{\pi}$.

\begin{exercise}
When $t=0$, the horizontal tangent line is $y=\answer{4}$.

When $t=\pi$, the horizontal tangent line is $y=\answer{-4}$.
\end{exercise}
\end{exercise}

%%%vertical Tangent Lines%%%%%%
\begin{exercise}
Next we want to find all tangent lines to $C$ with slope 2. 

The values of $t$ for which $\frac{dy}{dx} =2$ are $t=\answer{\frac{\pi}{2}}$ and $t=\answer{\frac{3\pi}{2}}$.

(type the smaller $t$-value first)

\begin{exercise}
When $t=\frac{\pi}{2}$, the Cartesian equation of the tangent line is $y=\answer{2}x+\answer{-2}$.

When $t=\frac{3\pi}{2}$, the Cartesian equation of the tangent line is $y=\answer{2}x+\answer{2}$.
\end{exercise}
\end{exercise}


\end{exercise}


\end{document}
