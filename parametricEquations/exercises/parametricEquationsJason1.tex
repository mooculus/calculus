\documentclass{ximera}

\newcommand{\RR}{\mathbb R}
\renewcommand{\d}{\,d}
\newcommand{\dd}[2][]{\frac{d #1}{d #2}}
\renewcommand{\l}{\ell}
\newcommand{\ddx}{\frac{d}{dx}}
\newcommand{\dfn}{\textbf}
\newcommand{\eval}[1]{\bigg[ #1 \bigg]}


\author{Jim Talamo}
\license{Creative Commons 3.0 By-bC}


\outcome{}


\begin{document}
\begin{exercise}
Consider the parametric equations 
\begin{align*}
x(t) &= t^2\\
y(t) &= t^3
\end{align*}

Fore $0\leq t < \infty$


Consider the point $(4,8)$. We want to determine the equation of the tangent line to our curve at this point. 

First, how can we verify if this point lies on the curve generated by our parametric equations? 

We need to determine whether there is a value of the parameter $t$
that simultaneously satisfies $t^2=4$ and $t^3=8$. 
In this case, $t=\answer{2}$. 

Now that we now that $(4,8)$ lies on our curve, let's find the equation of the tangent line at this point. 

Recall that point-slope form of the equation of a line is given by $y-y_{0}=m(x-x_{0})$. Here $m$ is the slope of the line and $(x_{0}, y_{0})$ is a point that lies on the line. 

We need to find the slope of the line through the point $(4,8)$. Since our curve is given parametrically, we use $\dd[y]{x}=\frac{y'(t)}{x'(t)}$ to determine the slope. 

Calculating, we obtain $y'(t)=\answer{3t^2}$ and $x'(t)=\answer{2t}$. Thus $\dd[y]{x}=\answer{\frac{3}{2}t}$. 

Thus the tangent line to our curve at the point $(4, 8)$ is $y-\answer{8}=\answer{3}(x-\answer{4})$. 

\begin{exercise}

We can also find a description of our curve in Cartesian coordinates. Notice that given our parametric equations that $x^3=\answer{t^6}$ and that $y^2=\answer{t^6}$ (give answers in terms of $t$).

Thus the the $x$-coordinate and $y$-coordinate of any point on our given curve satisfies $y^{\answer{2}}=x^{\answer{3}}$. 

This equation does not describe a function but we can solve for $y$ to obtain two functions $y=x^{\answer{\frac{3}{2}}}$ and $y=-x^{\answer{\frac{3}{2}}}$. Since our curve was traced out by $0\leq t < \infty$, this describes the function $y=x^{\answer{\frac{3}{2}}}$. 

Now let's calculate the derivative at the point $(4,8)$ using this description. We calculate $\dd[y]{x}=\answer{\frac{3}{2}x^{\frac{1}{2}}}$. 
Evaluating this derivative at $x=4$ gives $\answer{3}$.

This is the same answer we obtained before using a different description of our curve. 

\end{exercise}
\end{exercise}
\end{document}
