\documentclass{ximera}
\newcommand{\RR}{\mathbb R}
\renewcommand{\d}{\,d}
\newcommand{\dd}[2][]{\frac{d #1}{d #2}}
\renewcommand{\l}{\ell}
\newcommand{\ddx}{\frac{d}{dx}}
\newcommand{\dfn}{\textbf}
\newcommand{\eval}[1]{\bigg[ #1 \bigg]}

\author{Jim Fowler and Bart Snapp}

\begin{document}

Consider the following contour plot for a differentiable function $F :
\R^2 \to \R$.  The $x$-axis is horizontal and the $y$-axis is
vertical.  Assume nothing too complicated happens between the contour
lines.
\begin{image}[6in]
\includegraphics{contourPlot.png}
\end{image}

\begin{problem}
On the plot above, draw your best guess as to what the gradient vector looks like at the point $A$.
\begin{prompt}
  \begin{multipleChoice}
  \choice[correct]{I drew the vector}
\end{multipleChoice}
  \begin{feedback}
    Your vector should look like this:
    \begin{image}
      \includegraphics{contourWithGrad.png}
    \end{image}
    Gradient vectors are perpendicular to level curves and point in the initial direction of greatest increase for the function.
  \end{feedback}
\end{prompt}
\end{problem}

\begin{problem}
  Do you think $\frac{\partial F}{\partial y}$ is positive or negative at the point $B$?  Why?
\begin{prompt}
  \begin{multipleChoice}
    \choice[correct]{positive}
    \choice{negative}
  \end{multipleChoice}
  \begin{feedback}
    As increase the $y$-value from point $B$, the value of $F$ increases.
  \end{feedback}
\end{prompt}
\end{problem}

\begin{problem}
  Let $\uvec{u} = \vector{ \frac{1}{\sqrt{2}}, \frac{1}{\sqrt{2}} }$.  Which is larger: $D_{\uvec{u}} F$ or $|\grad F|$ at the point $C$?  Why?
  \begin{prompt}
    \begin{multipleChoice}
      \choice{$D_{\uvec{u}} F$}
      \choice[correct]{$|\grad F|$}
    \end{multipleChoice}
    \begin{feedback}
      At point $C$, $\uvec{u}$ points nearly along the level curve, so $D_{\uvec{u}} F$ is near zero. On the other hand, $|\grad F|$ is nonzero. 
    \end{feedback}
  \end{prompt}
\end{problem}


\begin{problem}
The point $D$ is at $(-4,3)$ and the point $E$ is at $(4,-2)$. Let
$\vec{v} = \grad F(-4,3)$ and let $\vec{w} = \grad F(4,-2)$.  Do you
think $|\vec{v}|$ or $|\vec{w}|$ is bigger?  Why?
\begin{prompt}
  \begin{multipleChoice}
    \choice{$|\vec{v}|$}
    \choice[correct]{$|\vec{w}|$}
  \end{multipleChoice}
  \begin{feedback}
    The contour lines are closer together at point $E$ than at point $D$. Hence $|\vec{w}|$ is bigger than $|\vec{v}|$.
  \end{feedback}
\end{prompt}
\end{problem}

%%%%%%%%%%%%%%%%%%%%%%%%%%%%%%%%%%%%%%%%
%%%%%%%%%%%%%%%%%%%%%%%%%%%%%%%%%%%%%%%%
%% Chain Rule
%%%%%%%%%%%%%%%%%%%%%%%%%%%%%%%%%%%%%%%%
%%%%%%%%%%%%%%%%%%%%%%%%%%%%%%%%%%%%%%%%

\begin{problem}
  Let $\vecl(t)$ parameterize a line from point $C$ to point $A$ as $t$
  runs from $0$ to $1$. Is $\eval{\dd{t} F(\vecl(t))}_{t=1}$ positive, zero, or
  negative? Why?
  \begin{prompt}
    \begin{multipleChoice}
      \choice{positive}
      \choice{zero}
      \choice[correct]{negative}
    \end{multipleChoice}
    \begin{feedback}
      Since the tangent vector of $\vecl$ points from $C$ to $A$, and
      this is going against the gradient vector at $A$, corresponding
      to $t=1$.
    \end{feedback}
  \end{prompt}
\end{problem}


\end{document}
