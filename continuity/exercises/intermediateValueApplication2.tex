\documentclass{ximera}

\newcommand{\RR}{\mathbb R}
\renewcommand{\d}{\,d}
\newcommand{\dd}[2][]{\frac{d #1}{d #2}}
\renewcommand{\l}{\ell}
\newcommand{\ddx}{\frac{d}{dx}}
\newcommand{\dfn}{\textbf}
\newcommand{\eval}[1]{\bigg[ #1 \bigg]}


\outcome{Application of Intermediate Value Theorem.}

\author{Nela Lakos }


\begin{document}
\begin{exercise}
Assume you invest   $\$1000$ in an account for $5$ years ($60$ months), with an annual interest rate of $r$, compounded monthly.
The amount of money in your account after $5$ years will be
\[
A(r)=1000[(1+r/12)^60]
\]
Use the Intermediate Value Theorem to show that there is a value of $r$ in the interval $(0,0.1)$ , that is an interest rate between $0\%$ and $10\%$,
that will allow you to accomplish your goal of saving $\$15000$ in $5$ years.

Let's solve the problem in steps.

STEP 1

The function $A$ is continuous and its domain is
\[
\text{Domain of }A=[\answer{0},\answer{0.1}]
\] 
STEP 2

\begin{exercise}

Compute the values
\[
A(0)=\answer{1000},
A(\answer{0.1})=1645.31.
\]

\end{exercise}

\begin{exercise}
 Let $L=15000$. Select all the following statements that are correct.

\begin{selectAll}
\choice{Since $A(0)< L<A(0.1)$ , the IVT implies that there is a number $r$ in $\left(0,15000\right)$ such that $A(r)=L$.}
\choice[correct]{Since $A(0)< L<A(0.1)$ , the IVT implies that there is a number $r$ in $\left(0,0.1\right)$ such that $A(r)=L$.}
\choice{Since $0< L< 0.1$ , the IVT implies that there is a number $r$ in $\left(0,0.1\right)$ such that $A(r)=L$.}
\choice[correct]{Since $1000< L< 1645.31$, the IVT implies that there is a number $r$ in $\left(0,0.1\right)$ such that $A(r)=L$.}
\end{selectAll}

\end{exercise}
When we apply the Intermediate Value Theorem to $A$ on the interval $\left[0,0.1\right]$ with $L=15000$, we are guaranteed the existence of at least one point $r$ such that.
\[
A(r)=\answer{15000}
\]
\end{exercise}

\end{document}