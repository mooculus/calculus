\documentclass{ximera}

\newcommand{\RR}{\mathbb R}
\renewcommand{\d}{\,d}
\newcommand{\dd}[2][]{\frac{d #1}{d #2}}
\renewcommand{\l}{\ell}
\newcommand{\ddx}{\frac{d}{dx}}
\newcommand{\dfn}{\textbf}
\newcommand{\eval}[1]{\bigg[ #1 \bigg]}


\outcome{}


\title[ Dig-In:]{L'H\^{o}pital's rule 1}

\begin{document}

\begin{abstract}
   Limits of quoteints can be solved using linearizations of the numerator and denominator functions.
\end{abstract}

Consider the limit \(\displaystyle\lim_{x \to 1} \frac{x^4-2x+1}{4x^4 - 15x +11}\).  To evaluate this limit using the 
techniques we already know, we would need to factor the numerator and denominator, which can be hard.

To gain some insight into this problem, lets graph the numerator and denominator on the same graph. Adjust the slider
to zoom in on the point $(1,0)$.

\todo{Make interactive graph with slider graph which does this.  Make sure it has coordinate grids.  
Maybe we can use Desmos?  Look at https://www.desmos.com/calculator/klphvr1rkx}

Since the numerator and denominator are differentiable at the point $(1,0)$, 
when we zoom in far enough they start to look like their tangent lines.  It is tempting to try to replace the functions
with their tangent line approximations when we take the limit, since that would make the algebra a lot easier.  Let's
give in to that temptation now.

\begin{question}
  \begin{hint}
      \begin{hint}
        $A'(x) = 4x^3-2$
      \end{hint}
      $A'(x)=$\answer{4x^3-2}
  \end{hint}
  \begin{hint}
    So $A'(1)=$\answer{2}
  \end{hint}
  \begin{hint}
    Using point slope form, the equation of the tangent line is $L_A(x) = 2(x-1)$
  \end{hint}
  
  The tangent line to $A(x) = x^4-2x+1$ at the point $(1,0)$ is $L_A(x) =\answer{2*(x-1)}$
\end{question}

\begin{question}
\begin{hint}
      \begin{hint}
        $A'(x) = 16x^3-15$
      \end{hint}
      $A'(x)=$\answer{16x^3-15}
  \end{hint}
  \begin{hint}
    So $A'(1)=$\answer{1}
  \end{hint}
  \begin{hint}
    Using point slope form, the equation of the tangent line is $L_A(x) = x-1$
  \end{hint}
  The tangent line to $B(x) = 4x^4-15x+11$ at the point $(1,0)$ is $ L_B(x)= \answer{x-1}$
\end{question}

\begin{question}
Since $A(x) \approx L_A(x)$  and $B(x) \approx L_B(x)$ near $x=1$, we hope that 
\begin{hint}
  $\frac{L_A(x)}{L_B(x)}  = \frac{2(x-1)}{x-1}$
\end{hint}
$\displaystyle \lim_{x \to 1} \frac{A(x)}{B(x)} = \displaystyle \lim_{x \to 1} \frac{L_A(x)}{L_B(x)}=$\answer{2}.
\end{question}

We just used differential calculus (finding tangent lines) to find a limit which would have been much harder to find
by factoring!

\end{document}
