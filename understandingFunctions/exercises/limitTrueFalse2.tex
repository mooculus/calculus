\documentclass{ximera}

\newcommand{\RR}{\mathbb R}
\renewcommand{\d}{\,d}
\newcommand{\dd}[2][]{\frac{d #1}{d #2}}
\renewcommand{\l}{\ell}
\newcommand{\ddx}{\frac{d}{dx}}
\newcommand{\dfn}{\textbf}
\newcommand{\eval}[1]{\bigg[ #1 \bigg]}


\outcome{Understand the relationship between limits and vertical asymptotes.}

\author{Nela Lakos \and Kyle Parsons}

\begin{document}
\begin{exercise}

Select True if the statement is \textbf{always} true; otherwise, select False.

Let $f$ be a one-to-one function and $f^{-1}$ its inverse.  If the point $(2,5)$ lies on the graph of $f$, then the point $(5,2)$ lies on the graph of $f^{-1}$.

\begin{multipleChoice}
\choice[correct]{True}
\choice{False}
\end{multipleChoice}

\begin{feedback}
The graph of $f$ contains all points of the form $\left(a,f(a)\right)$ whereas the graph of $f^{-1}$ contains all points of the form $\left(f(a),f^{-1}(f(a))\right)=(f(a),a)$.
\end{feedback}

\begin{exercise}

\[
\sin^{-1}(\pi) = 0
\]

\begin{multipleChoice}
\choice{True}
\choice[correct]{false}
\end{multipleChoice}

\begin{feedback}
The range of $\sin$ is $\left[-1,1\right]$ and so the domain of $\sin^{-1}$ is $\left[-1,1\right]$.  We see then than $\pi$ is not in the domain of $\sin^{-1}$ so $\sin^{-1}(\pi)$ does not exist.
\end{feedback}

\end{exercise}
\end{exercise}
\end{document}