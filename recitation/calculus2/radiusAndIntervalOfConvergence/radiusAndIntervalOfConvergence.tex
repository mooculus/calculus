\documentclass[]{ximera}
%handout:  for handout version with no solutions or instructor notes
%handout,instructornotes:  for instructor version with just problems and notes, no solutions
%noinstructornotes:  shows only problem and solutions

%% handout
%% space
%% newpage
%% numbers
%% nooutcomes

%I added the commands here so that I would't have to keep looking them up
%\newcommand{\RR}{\mathbb R}
%\renewcommand{\d}{\,d}
%\newcommand{\dd}[2][]{\frac{d #1}{d #2}}
%\renewcommand{\l}{\ell}
%\newcommand{\ddx}{\frac{d}{dx}}
%\everymath{\displaystyle}
%\newcommand{\dfn}{\textbf}
%\newcommand{\eval}[1]{\bigg[ #1 \bigg]}

%\begin{image}
%\includegraphics[trim= 170 420 250 180]{Figure1.pdf}
%\end{image}

%add a ``.'' below when used in a specific directory.

\newcommand{\RR}{\mathbb R}
\renewcommand{\d}{\,d}
\newcommand{\dd}[2][]{\frac{d #1}{d #2}}
\renewcommand{\l}{\ell}
\newcommand{\ddx}{\frac{d}{dx}}
\newcommand{\dfn}{\textbf}
\newcommand{\eval}[1]{\bigg[ #1 \bigg]}




\author{Jim Talamo}

\outcome{Find Taylor Polynomials.}
\outcome{Understand the relationship between the derivatives of a function and the coefficients of its Taylor Polynomial.}

\title[handout]{Radius and Interval of Convergence}

\begin{document}
\begin{abstract}
\end{abstract}
\maketitle

\vspace{-0.9in}

\section{Discussion Questions}

\begin{problem} 
Suppose that $f(x) = \sum_{k=0}^{\infty} a_k x^k$ and it is known that $\sum_{k=0}^{\infty} 2^k a_k $ converges but $ \sum_{k=0}^{\infty} (-1)^k 2^k a_k $ diverges.

\begin{itemize}
\item[I.] What is the radius of convergence of the power series?
%Add something that discusses why diverging at x=-2 does not mean that the ROC is less than 2
\item[II.] What is the interval of convergence of the power series?
\end{itemize}

\begin{solution}
I. Since $\sum 2^k a_k = f(2)$ converges, the radius of convergence is at least $2$. Since $\sum (-1)^k 2^k a_k = f(-2)$, the radius of convergence is at most $2$. We conclude that the radius of convergence is exactly $2$.

II. Since the radius of convergence is $2$, with convergence at the right endpoint and divergence at the left endpoint, the interval of convergence is $(-2,2]$. 
\end{solution}
\end{problem}


\begin{problem} 
Suppose that $f(x) = \sum_{k=0}^{\infty} a_k (x+3)^k$ and it is known that $\sum_{k=0}^{\infty} 5^k a_k$ converges and the series represented by $f(-10)$ diverges.

\begin{itemize}
\item[I.] Does the series represented by $f(5)$ converge, diverge, or is more information needed to determine this?
\item[II.] Does the series $\sum_{k=1}^{\infty} a_k$ converge, diverge, or is more information needed?
\item[III.] What is the smallest and largest possible radius and interval of convergence for the series represented by $f(x)$?
\end{itemize}


\begin{solution}
I. The power series is centered at $x=-3$.  Since $5$ is farther from the center than $-10$, and $f(-10)$ diverges, we conclude that the series represented by $f(5)$ must diverge.

II. The series $\sum a_k$ is equal to $f(-2)$. Since $-2$ is closer to the center of series than $2$ and  $f(2) = \sum 5^k a_k$ converges, the series represented by $f(-2)$ converges.

III. That the series $\sum 5^k a_k = f(2)$ converges implies that the radius of convergence of the series is at least $5$. Since $f(-10)$ diverges, the radius of convergence is at most $7$.
\end{solution}
\end{problem}


\section{Group Work}
%%%%%%%%%%%%%%%%%%%%%%%%%%%%%%%%


\begin{problem} 
Suppose that $f(x) = \sum_{k=1}^{\infty} \frac{k^2x^k}{3^k+1}$.  Determine whether the series represented by $f(2)$ converges or diverges.

%Note that they can sub in  x=2, or find the radius of convergence and use it to answer the question.

\begin{solution}
The series represented by $f(2)$ is 
$$
\sum_{k=1}^{\infty} \frac{k^2 2^k}{3^k+1}.
$$
First note that $\frac{k^2 2^k}{3^k+1} \leq \frac{k^2 2^k}{3^k}$ for all $k$. To apply the Ratio Test to the series with terms given by this upper bounding sequence, we compute the limit
$$
\lim_{k\rightarrow \infty} \frac{(k+1)^2 2^{k+1}}{3^{k+1}} \cdot \frac{3^k}{k^2 2^k} = \lim_{k\rightarrow \infty} \frac{2(k+1)^2}{3k^2} = \frac{2}{3}.
$$
It follows that the series represented by $f(2)$ converges.
\end{solution}
\end{problem}


\begin{problem} 
Find the radius and interval of convergence for the following power series:

\begin{tabular}{lll}
I.  $f(x) = \sum_{k=1}^{\infty} \frac{x^k}{k!}$ \qquad  \qquad II. $f(x) = \sum_{k=1}^{\infty} \frac{(-1)^kx^{2k}}{\sqrt{k}}$  \qquad  \qquad III. $f(x) = \sum_{k=1}^{\infty} k! x^k$
\end{tabular}

\begin{solution}
I. We apply the Ratio Test to the series 
$$
\sum_{k=1}^{\infty} \left|\frac{x^k}{k!}\right| = \sum_{k=1}^{\infty} \frac{|x|^k}{k!}
$$
in order to test for absolute convergence. The relevant limit is 
$$
\lim_{k \rightarrow \infty} \frac{|x|^{k+1}}{(k+1)!} \cdot \frac{k!}{|x|^k} = \lim_{k \rightarrow \infty} \frac{|x|}{k+1} = 0
$$
for all $x$. The radius of convergence is therefore $\infty$ and the interval of convergence is $(-\infty,\infty)$.

II. We once again test for absolute convergence by considering the series 
$$
\sum_{k=1}^{\infty} \left|\frac{(-1)^kx^{2k}}{\sqrt{k}}\right| = \sum_{k=1}^{\infty} \frac{x^{2k}}{\sqrt{k}}.
$$
Applying the Ratio Test, we have
$$
\lim_{k\rightarrow \infty} \frac{x^{2(k+1)}}{\sqrt{k+1}} \frac{\sqrt{k}}{x^{2k}} = \lim_{k\rightarrow \infty} x^2 \frac{\sqrt{k}}{\sqrt{k+1}} = x^2.
$$
A sufficient condition for convergence is $x^2 < 1$, or $-1 < x < 1$. The radius of convergence is therefore $1$. To find the interval of convergence, we test the endpoints $f(1)$ and $f(-1)$ separately. The series represented by $f(1)$ is 
$$
\sum_{k=1}^{\infty} \frac{(-1)^k}{\sqrt{k}},
$$
which converges by the Alternating Series Test. Similarly, the series represented by $f(-1)$ converges by the Alternating Series Test. The interval of convergence is therefore $[-1,1]$. 

III. By comparing growth rates, we see that for any $x > 0$, 
$$
\lim_{k\rightarrow \infty} k! x^k  = \infty,
$$
and for any $x < 0$, the limit does not exist. It follows that the radius of convergence for this series is zero. The ``interval" of convergence is the singleton set $\{0\}$. 
\end{solution}
\end{problem}


\end{document}
