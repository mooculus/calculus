\documentclass[]{ximera}
%handout:  for handout version with no solutions or instructor notes
%handout,instructornotes:  for instructor version with just problems and notes, no solutions
%noinstructornotes:  shows only problem and solutions

%% handout
%% space
%% newpage
%% numbers
%% nooutcomes

%I added the commands here so that I would't have to keep looking them up
%\newcommand{\RR}{\mathbb R}
%\renewcommand{\d}{\,d}
%\newcommand{\dd}[2][]{\frac{d #1}{d #2}}
%\renewcommand{\l}{\ell}
%\newcommand{\ddx}{\frac{d}{dx}}
%\everymath{\displaystyle}
%\newcommand{\dfn}{\textbf}
%\newcommand{\eval}[1]{\bigg[ #1 \bigg]}

%\begin{image}
%\includegraphics[trim= 170 420 250 180]{Figure1.pdf}
%\end{image}

%add a ``.'' below when used in a specific directory.

\newcommand{\RR}{\mathbb R}
\renewcommand{\d}{\,d}
\newcommand{\dd}[2][]{\frac{d #1}{d #2}}
\renewcommand{\l}{\ell}
\newcommand{\ddx}{\frac{d}{dx}}
\newcommand{\dfn}{\textbf}
\newcommand{\eval}[1]{\bigg[ #1 \bigg]}




\author{Tom Needham}

\outcome{Use the technique of trigonometric substitution to compute integrals.}

\title[]{Trigonometric Substitution}

\begin{document}
\begin{abstract}
\end{abstract}
\maketitle

\vspace{-0.45in}

\section{Discussion Questions}

\begin{problem}
A student claim that
$$
\int \sqrt{x^2-4} \d x = \int x-2 \d x = \frac{1}{2}x^2 - 2x + C.
$$
Is the student correct? If not, determine a likely error that the student made in the calculation.
\end{problem}

\begin{freeResponse}
The student is not correct. The student's mistake is in the claim that $\sqrt{x^2-4}=x-2$. This claim is clearly false; the two functions do not even have the same domain and they evaluate to different values for most inputs in their common domain.
\end{freeResponse}

\begin{problem}
Each of the following integrals can be evaluated using trigonometric substitution. For each example, determine which trigonometric substitution should be used.

\begin{center}
\begin{tabular}{lll}
I. $\int x^3 \sqrt{4-x^2} \d x $ \hspace{.2in} II.$\int \frac{x^3}{\sqrt{x^2+16}} \d x$ \hspace{.2in} III. $\int \frac{1}{x^2 \sqrt{x^2-2}} \d x$
\end{tabular}
\end{center}
\end{problem}

\begin{freeResponse}
Trigonometric substitution is based on the Pythagorean identities:
$$
\sin^2 \theta  = 1 - \cos^2 \theta \;\; \mbox{ and } \;\;  1 + \tan^2 \theta = \sec^2 \theta.
$$
To choose the correct substitution, one applies the relevant identity.

I. Let $x = 2 \sin \theta$, so that the first Pythagorean identity can be applied.

II. Let $x=4 \tan \theta$, so that the second Pythagorean identity can be applied.

III. Let $x=\sqrt{2}\sec \theta$, so that the second Pythagorean identity can be applied.
\end{freeResponse}

\begin{problem}
For $x>2$, the trigonometric substitution $x = 2 \sec \theta$ gives the result: $$\int \dfrac{4}{x^2\sqrt{x^2-4}} \, dx = \int  \cos \theta \, d \theta.$$
Evaluate $\displaystyle \int \dfrac{8}{x^2\sqrt{x^2-4}} \, dx$.

\vspace{2mm}
\begin{tabular}{lll}
A.  $\dfrac{\sqrt{x^2-4}}{x}+C$    \hspace{10mm} & B. $-\dfrac{\sqrt{x^2-4}}{x}+C$   \hspace{12mm} & C.  $4 \ln\bigg(x^2\sqrt{x^2-4}\bigg) +C$  \\[3ex]
D. $\dfrac{x}{\sqrt{x^2-4}}+C$   \hspace{15mm} & E.  $-\dfrac{x}{\sqrt{x^2-4}}+C$ \hspace{10mm} & F. None of these\\  [2 ex] 
\end{tabular}
\end{problem}

\begin{freeResponse}
The answer is F. We use the given information to write
\begin{align*}
\displaystyle \int \dfrac{8}{x^2\sqrt{x^2-4}} \d x &= 2 \int \dfrac{4}{x^2\sqrt{x^2-4}} \d x \\
&= 2 \int \cos \theta \d \theta \\
&= 2 \sin \theta + C \\
&= 2 \frac{\sqrt{x^2-4}}{x} + C.
\end{align*}
The last expression is found using a reference triangle for $\sec \theta = \frac{x}{2}$. 
\end{freeResponse}



\section{Group Work}

\begin{problem}
\begin{enumerate}
\item[I.] Given that $\sin \theta = \frac{2}{5}$ and that $\theta$ lies in the second quadrant, determine the value of $\cos \theta$.
\item[II.] Given that $\tan \theta = x$, for some value $x$, and that $\theta$ lies in the fourth quadrant, determine the values of $\cos \theta$ and $\csc \theta$ in terms of $x$.
\end{enumerate}
\end{problem}

\begin{freeResponse}
I. A reference triangle for $\theta$ has hypotenuse $5$ and opposite side $2$. The adjacent side therefore has length $\sqrt{5^2-2^2} = \sqrt{21}$. Since $\theta$ lies in the second quadrant, cosine must be negative, and we conclude that $\cos\theta = -\frac{\sqrt{21}}{5}$.

II. A reference triangle for $\theta$ has opposite side length $x$ and adjacent side length $1$. The hypotenuse length is therefore $\sqrt{x^2+1}$. Since $\theta$ lies in the fourth quadrant, $\cos \theta$ is positive and $\csc \theta$ is negative. Therefore
$$
\cos \theta = \frac{1}{\sqrt{x^2+1}} \;\; \mbox{ and } \;\; \csc \theta = -\frac{\sqrt{x^2+1}}{x}.
$$
\end{freeResponse}

\begin{problem}
Evaluate the following integrals using trigonometric substitution.
\begin{center}
\begin{tabular}{lll}
I. $\int_{\sqrt{2}}^2 \frac{1}{t^3 \sqrt{t^2 -1}} \d t$ \hspace{.2in} II. $\int \frac{27 x^2}{(4+9x^2)^{3/2}} \d x$ \hspace{.2in} III. $\int \frac{x^2}{\sqrt{4x-x^2}} \d x$
\end{tabular}
\end{center}
\end{problem}

\begin{freeResponse}
I. Let $t = \sec \theta$, so that $\d t = \sec \theta \tan \theta \d \theta$. Then
\begin{align*}
\int \frac{1}{t^3\sqrt{t^2-1}} \d t &= \int \frac{1}{\sec^3 \theta \tan \theta} \sec \theta \tan \theta \d \theta \\
&= \int \frac{1}{\sec^2 \theta} \d \theta \\
&= \int \cos^2 \theta \d \theta \\
&= \int \frac{1}{2} + \frac{\cos (2 \theta)}{2} \d \theta \\
&= \frac{1}{2}\theta + \frac{\sin (2\theta)}{4} + C \\
&= \frac{1}{2} \theta-+\frac{\sin \theta \cos \theta}{2} + C \\
&= \frac{1}{2} \sec^{-1}\theta + \frac{\sqrt{t^2-1}}{2t^2} + C,
\end{align*}
where we find the last term using a reference triangle for $\theta$. Since $\sec \theta = t$, a reference triangle can be formed with hypotenuse $t$ and adjacent side length $1$, so that the opposite side length is $\sqrt{t^2-1}$. It follows that $\cos \theta = \frac{1}{t}$ and $\sin \theta = \frac{\sqrt{t^2-1}}{t}$. Therefore
$$
\int_{\sqrt{2}}^2 \frac{1}{t^3 \sqrt{t^2 -1}} \d t = \eval{\frac{1}{2} \sec^{-1}\theta + \frac{\sqrt{t^2-1}}{2t^2}}_{\sqrt{2}}^2 = \frac{\pi}{6} + \frac{\sqrt{3}}{8} -  \frac{\pi}{8} - \frac{1}{4}.
$$


II. We choose the substitution $ x = \frac{2}{3} \tan\theta$, so that $\d x = \frac{2}{3} \sec^2 \theta \d \theta$. Making the substitution into the integral gives:
\begin{align*}
\int \frac{27 x^2}{(4+9x^2)^{3/2}} \, dx & =  \int \dfrac{27 \cdot  \frac{4}{9 \tan^2 \theta}}{(4+9 \cdot \frac{4}{9}\tan^2 \theta)^{3/2}} \, \left[ \dfrac{2}{3} \sec^2 \theta \d \theta \right] \\ 
&=  \int \dfrac{8 \tan^2 \theta}{ (4[1+1\tan^2 \theta])^{3/2}} \,  \cdot \sec^2 \theta \d \theta  \\
&=  \int \dfrac{8 \tan^2 \theta}{8 (\sec^2 \theta)^{3/2}} \,  \cdot \sec^2 \theta \d \theta  \\
&=  \int \dfrac{   \tan^2 \theta}{\sec^3 \theta} \,  \cdot \sec^2 \theta \d \theta  \\
&=  \int \dfrac{   \tan^2 \theta}{\sec \theta} \d \theta \\
&=  \int \dfrac{   \sec^2 \theta + 1}{\sec \theta} \,  d\theta  \\
&=  \int \left[\dfrac{   \sec^2 \theta}{\sec \theta} + \dfrac{1}{\sec \theta}\right] \,  d\theta  \\
&=  \int ( \sec \theta + \cos \theta )  \,  d\theta  \\
&=  \ln \left| \sec \theta + \tan \theta \right| - \sin \theta +C  \\
&= \ln \left| \dfrac{\sqrt{4+9x^2}}{2}+\dfrac{3x}{2} \right| - \dfrac{3x}{\sqrt{4x^2+9}} +C.
\end{align*}
To derive the last line, we use a reference triangle for $\theta$, with $\tan \theta = \frac{3x}{2}$. This triangle has opposite side length $3x$ and adjacent side length $2$, and therefore has hypotenuse length $\sqrt{9x^2 + 4}$. It follows that $\sec \theta = \frac{\sqrt{9x^2+4}}{2}$.

III. We first complete the square under the square root sign, using 
$$
4x-x^2 = -(x^2-4x) = -((x-2)^2-4) = 4-(x-2)^2.
$$
We therfore wish to choose a substitution so that $4-(x-2)^2$ transforms to $4-4\sin^2\theta$. The correct substituion is therefore $x-2 = 2 \sin \theta$, or $x= 2 \sin \theta + 2$. Then $\d x = 2 \cos \theta \d \theta$, and 
\begin{align*}
\int \frac{x^2}{\sqrt{4x-x^2}} \d x & = \int \frac{x^2}{\sqrt{4-(x-2)^2}} \d x \\
&= \int \frac{4 \sin^2 \theta + 4 \sin \theta + 4}{2 \cos \theta} 2 \cos \theta \d \theta \\
&= \int 4 \sin^2 \theta + 4 \sin \theta + 4 \d \theta \\
&= 4 \int \frac{1}{2} - \frac{\cos (2 \theta)}{2} \d \theta -4 \cos \theta + 4 \theta \\
&= 2 \theta - \sin (2 \theta) - 4 \cos \theta + 4 \theta + C \\
&= 6 \theta - 2 \sin \theta \cos \theta - 4 \cos \theta + C \\
&= 6 \arcsin \left(\frac{x-2}{2}\right) - (x-2)\frac{\sqrt{4-(x-2)^2}}{2} - 2 \sqrt{4-(x-2)^2} + C.
\end{align*}
The last line follows from a reference triangle. A reference triangle for $\theta$, with $\sin \theta = (x-2)/2$, has opposite side length $x-2$ and hypotenuse $2$. It therefore has adjacent side length $\sqrt{4-(x-2)^2}$, so that $\cos \theta = \sqrt{4-(x-2)^2}/2$.
\end{freeResponse}

\begin{problem}
Evaluate the integral
$$
\int \frac{x}{x^2-9} \d x
$$
in two ways: first via trigonometric substitution, then via $u$-substitution. Which method takes less work?
\end{problem}

\begin{freeResponse}
Using trigonometric substitution, we take $x = 3 \sec \theta$. Then $\d x = 3 \sec \theta \tan \theta \d \theta$, and 
\begin{align*}
\int \frac{x}{x^2-9} \d x &= \int \frac{3 \sec \theta}{9(\sec^2 \theta - 1)} 3 \sec \theta \tan \theta \d \theta \\
&= \int \frac{\sec \theta}{\tan^2 \theta} \sec \theta \tan \theta \d \theta \\
&= \int \frac{\sec^2 \theta}{\tan^2 \theta} \d \theta.
\end{align*}
Now let $u = \tan \theta$ so that $\d u = \sec^2 \theta \d \theta$. The integral becomes
$$
\int \frac{\sec^2 \theta}{\tan \theta} \d \theta = \int \frac{1}{u} \d u = \ln |u| + C = \ln | \tan \theta | + C.
$$
Finally, using a reference triangle we conclude that $\tan \theta = \frac{\sqrt{x^2-9}}{3}$, so the answer is 
$$
\ln \left| \frac{\sqrt{x^2-9}}{3}\right| + C.
$$

On the other hand, we can immediately make the substitution $u = x^2 - 9$, so that $\d u = 2x \d x$, and 
$$
\int \frac{x}{x^2-9} \d x = \frac{1}{2} \int \frac{1}{u} \d u = \frac{1}{2} \ln |u| + C = \frac{1}{2} \ln |x^2-9| + C.
$$
Note that (by applying log rules) this differs from the previous answer by addition of a constant, so that the two answers are equivalent. Apparently, directly using $u$-substitution takes much less work. 
\end{freeResponse}


\end{document}
