\documentclass{ximera}
\newcommand{\RR}{\mathbb R}
\renewcommand{\d}{\,d}
\newcommand{\dd}[2][]{\frac{d #1}{d #2}}
\renewcommand{\l}{\ell}
\newcommand{\ddx}{\frac{d}{dx}}
\newcommand{\dfn}{\textbf}
\newcommand{\eval}[1]{\bigg[ #1 \bigg]}


\begin{document}

\outcome{Find the linear approximation to a function at a point and use it to approximate the function value.}
\outcome{Use the Mean Value Theorem to bound the error in linear approximation.}

Let $f$ be a differentiable function on the interval $(0,6)$ with the particular values of $f$ and $f'$ given in the table below.
\[
\begin{array}{c|c|c|}
x & f(x) & f'(x)\\ \hline
1 &  0 &     4\\
2 &  1 &     3\\
3 &  5 &     2\\
4 &  6 &     5\\
\end{array}
\]

\begin{exercise}
Find $L(x)$, the linear approximation to $f$ at $x=3$. 
\[
L(x)=\answer{2x-1}
\]
\begin{exercise}
Use the linear approximation $L(x)$ to estimate the value of $f(2.8)$.
\[
f(2.8)\approx\answer{4.6}
\]
\begin{exercise}
Find the average rate of change of $f$ on the interval $[1,4]$.
\[
\text{average rate of change of $f$ on $[1,4]$}=\answer{2}
\]
\begin{exercise}
Choose the correct statement.
\begin{multipleChoice}
\choice{There exists a point $c$ in $(1,4)$ such that $f'(c)=\frac{f(4)-f(1)}{4-1}$, although the function $f$ does not satisfy the conditions on the mean value theorem on $[1,4]$.}
\choice[correct]{There exists a point $c$ in $(1,4)$ such that $f'(c)=\frac{f(4)-f(1)}{4-1}$, since the function $f$ satisfies the conditions on the mean value theorem on $[1,4]$.}
\choice{There exists a point $c$ in $(1,4)$ such that $f'(c)=\frac{f(4)-f(1)}{4-1}$, but we don't have enough information to find such a point.}
\choice{There exists no point $c$ in $(1,4)$ such that $f'(c)=\frac{f(4)-f(1)}{4-1}$.}
\end{multipleChoice}
\begin{exercise}
\begin{exercise}
Again, choose the correct statement.
\begin{multipleChoice}
\choice{For each $x$ in $[2,3]$, there exists a point $c$ in $(2,3)$ such that $f'(c)=\frac{f(x)-f(2)}{x-2}$, although the function $f$ does not satisfy the conditions on the mean value theorem on $[2,3]$.}
\choice[correct]{For each $x$ in $[2,3]$, there exists a point $c$ in $(2,3)$ such that $f'(c)=\frac{f(x)-f(2)}{x-2}$, since the function $f$ satisfies the conditions on the mean value theorem on $[2,3]$.}
\choice{There may be $x$ in $[2,3]$, for which there exists a point $c$ in $(2,3)$ such that $f'(c)=\frac{f(x)-f(2)}{x-2}$, but not necessarily every $x$ in $[2,3]$.}
\choice{For each $x$ in $[2,3]$, there does not exist a point $c$ in $(2,3)$ such that $f'(c)=\frac{f(x)-f(2)}{x-2}$.}
\end{multipleChoice}
\begin{exercise}
Suppose we know that $2\le f'(x)\le 3$ for $x$ in $(2,3)$. By the mean value theorem, for each $x$ in$ [2,3]$, there exists $c$ in $(2,3)$ such that
\[
f(x)=\answer{f'(c)(x-2)+1}
\] 
\begin{exercise}
Recall that we assumed $2\le f'(c)\le 3$ for each $c$ in $(2,3)$. Use your answer above to bound $f(x)$ for $x$ in $[2,3]$ from above and from below.
\[
\answer{2(x-2)+1}\le f(x)\le \answer{3(x-2)+1}
\]
\begin{exercise}
Use the bounds $2(x-2)+1\le f(x)\le 3(x-2)+1$ for $x$ in $[2,3]$ you computed in the previous part to approximate $f(2.8)$ from above and below.
\[
\answer{2.6}\le f(2.8)\le \answer{3.4}
\]
\begin{exercise}
Let $L(x)$ be the linear approximation to $f$ at $x=3$ you computed before. We used $L(x)$ to approximate $f(2.8)$ and found that $f(2.8)$ is approximately equal to $4.6$. Using our previous answers, we can deduce that this approximation to $f(2.8)$ is a \wordChoice{\choice[correct]{over approximation}\choice{under approximation}}.
\end{exercise}
\end{exercise}
\end{exercise}
\end{exercise}
\end{exercise}
\end{exercise}
\end{exercise}
\end{exercise}
\end{exercise}
\end{exercise}

\end{document}
