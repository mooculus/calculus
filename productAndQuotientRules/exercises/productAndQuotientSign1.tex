\documentclass{ximera}
\newcommand{\RR}{\mathbb R}
\renewcommand{\d}{\,d}
\newcommand{\dd}[2][]{\frac{d #1}{d #2}}
\renewcommand{\l}{\ell}
\newcommand{\ddx}{\frac{d}{dx}}
\newcommand{\dfn}{\textbf}
\newcommand{\eval}[1]{\bigg[ #1 \bigg]}

\author{Steven Gubkin}
\license{Creative Commons 3.0 By-NC}

\outcome{Use the first derivative to determine whether a function is increasing or decreasing.}
\outcome{Define higher order derivatives.}
\outcome{Compare differing notations for higher order derivatives.}
\outcome{Identify the relationships between the function and its first and second derivatives.}
\outcome{Sketch a graph of the second derivative, given the original function.}
\outcome{Sketch a graph of the original function, given the graph of its first and second derivatives.}
\outcome{Sketch a graph of a function satisfying certain constraints on its higher-order derivatives.}
\outcome{State the relationship between concavity and the second derivative.}
\outcome{Interpret the second derivative of a position function as acceleration.}
\outcome{Calculate higher order derivatives.}

\begin{document}

\begin{exercise}

Let $P(x) = A(x)B(x)$

If you know that $A(2) > 0$, $A'(2) > 0$, $B(2) > 0$, and $B'(2) < 0$, what can you say about the sign of $P'(2)$?

\begin{multipleChoice}
\choice{$P'(2)>0$}
\choice{$P'(2)<0$}
\choice{$P'(2) = 0$}
\choice[correct]{We cannot determine the sign of $P'(2)$}
\end{multipleChoice}

Try to make sense of this by thinking about small increases/decreases in the quantities $A$ and $B$, in addition to symbolic calculation.

\end{exercise}

\end{document}
