\documentclass{ximera}

\newcommand{\RR}{\mathbb R}
\renewcommand{\d}{\,d}
\newcommand{\dd}[2][]{\frac{d #1}{d #2}}
\renewcommand{\l}{\ell}
\newcommand{\ddx}{\frac{d}{dx}}
\newcommand{\dfn}{\textbf}
\newcommand{\eval}[1]{\bigg[ #1 \bigg]}


%\outcome{Define area.}
\outcome{Understand the relationship between area under a curve and sums of rectangles.}
\outcome{Approximate area under a curve.}
\outcome{Compute left, right, and midpoint Riemann sums with 10 or fewer rectangles.}

\author{Nela Lakos \and Kyle Parsons \and Bobby Ramsey}

\begin{document}
\begin{exercise}

Consider the function
\[
f(x) = 36-x^2
\]
on the interval $[0,6]$.  We will approximate the area of the region bounded by the graph of $f$ and the $x$-axis.  See the figure below.

\begin{image}
  \begin{tikzpicture}
    \begin{axis}[
        xmin=-0.3,xmax=6.3,ymin=-0.3,ymax=36.3,
        clip=true,
        unit vector ratio*=6 1 1,
        axis lines=center,
        grid = major,
        ytick={0,4,...,36},
        xtick={0,0.5,...,6},
        xlabel=$x$, ylabel=$y$,
        every axis y label/.style={at=(current axis.above origin),anchor=south},
        every axis x label/.style={at=(current axis.right of origin),anchor=west},
      ]      
      \pgfplotsinvokeforeach{0.5,1,...,6}{\draw[very thick,penColor2,fill,fill opacity=0.3] (axis cs:{#1-0.5},0) rectangle (axis cs:{#1},{36-#1^2});}
      
      \addplot[very thick,penColor,domain=0:6] plot{36-x^2};
      
      \node at (axis cs:4,30) {$y=f(x)$};
      \end{axis}`
  \end{tikzpicture}
\end{image}

The image depicts a \wordChoice{\choice{Left}\choice[correct]{Right}\choice{Midpoint}} Riemann sum with $n=\answer{12}$ rectangles.

\begin{align*}
	x_6^* &= \answer{3}\\
	f(x_6^*) &= \answer{27}\\
	\Delta x &= \answer{\frac{1}{2}}\\
	\sum_{k=1}^n f(x_k^*)\Delta x &= \answer{\frac{539}{4}}
\end{align*}

The area approximation given by this Riemann Sum is an \wordChoice{\choice{overestimate}\choice[correct]{underestimate}} because 
\begin{multipleChoice}
	\choice{the rectangles contain more area than in the region.}
	\choice[correct]{there is more area in the region than contained in the rectangles.}
	\choice{it is impossible to tell.}
\end{multipleChoice}

\end{exercise}

\end{document}
