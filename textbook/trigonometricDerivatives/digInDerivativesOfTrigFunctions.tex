\documentclass{ximera}

\newcommand{\RR}{\mathbb R}
\renewcommand{\d}{\,d}
\newcommand{\dd}[2][]{\frac{d #1}{d #2}}
\renewcommand{\l}{\ell}
\newcommand{\ddx}{\frac{d}{dx}}
\newcommand{\dfn}{\textbf}
\newcommand{\eval}[1]{\bigg[ #1 \bigg]}





\title[Dig-In:]{The derivatives of trigonometric functions}

\begin{document}
\begin{abstract}
  We find the derivatives of trigonometric functions.
\end{abstract}
\maketitle

Up until this point of the course we have been ignoring a large class
of functions---trigonometric functions other than $\sin(x)$. We know that
\[
\ddx \sin(x) = \cos(x).
\]
Armed with this fact we will discover the derivatives of all of the
standard trigonometric functions.


Of course, now that we know the derivative of the sine, we can compute
derivatives of more complicated functions involving the sine.

%\break

\begin{theorem}[The Derivative of cos(\textit{x})]\index{derivative!of cosine}
\[
\ddx \cos(x) = -\sin(x).
\]
\begin{proof}
Recall that
\begin{align*}
\cos(x) &= \sin\left(x+\frac{\pi}{2}\right), \\
\sin(x) &= -\cos\left(x+\frac{\pi}{2}\right).
\end{align*}
Now:
\begin{align*}
\ddx \cos(x) &= \ddx \sin\left(x+\frac{\pi}{2}\right)\\
&=\cos\left(x+\frac{\pi}{2}\right)\cdot 1 \\
&= -\sin(x).
\end{align*}
\end{proof}
\end{theorem}


Next we have:

\begin{theorem}[The Derivative of tan(\textit{x})]\index{derivative!of tangent}
\[
\ddx \tan(x) = \sec^2(x).
\]

\begin{proof}
We'll rewrite $\tan(x)$ as $\frac{\sin(x)}{\cos(x)}$ and use the quotient rule. Write
\begin{align*}
\ddx\tan(x) &= \ddx\frac{\sin(x)}{\cos(x)}\\
&=\frac{\cos^2(x) + \sin^2(x)}{\cos^2(x)}\\
&=\frac{1}{\cos^2(x)}\\
&=\sec^2(x).
\end{align*}
\end{proof}
\end{theorem}

Finally, we have

\begin{theorem}[The Derivative of sec(\textit{x})]\index{derivative!of secant}
\[
\ddx \sec(x) = \sec(x)\tan(x).
\]


\begin{proof}
We'll rewrite $\sec(x)$ as $(\cos(x))^{-1}$ and use the power rule and the chain rule. Write
\begin{align*}
\ddx \sec(x) &= \ddx(\cos (x))^{-1}\\
&=-1(\cos(x))^{-2}(-\sin(x)) \\
&= \frac{\sin(x)}{\cos^2(x)} \\
&= \sec(x)\tan(x).
\end{align*}
\end{proof}
\end{theorem}

The derivatives of the cotangent and cosecant are similar and left as
exercises. 

Putting this all together, we have:

\begin{theorem}[The Derivatives of Trigonometric Functions] \hfil
\begin{itemize}
\item $\ddx \sin(x) = \cos(x)$.
\item $\ddx \cos(x) = -\sin(x)$.
\item $\ddx \tan(x) = \sec^2(x)$.
\item $\ddx \sec(x) = \sec(x)\tan(x)$.
\item $\ddx \csc(x) = -\csc(x)\cot(x)$.
\item $\ddx \cot(x) = -\csc^2(x)$.
\end{itemize}
\end{theorem}


\begin{warning}
When working with derivatives of trigonometric functions, we suggest
you use \textbf{radians} for angle measure. For example, while
\[
\sin\left((90^\circ\right)^2) = \sin\left(\left(\frac{\pi}{2}\right)^2\right),
\]
one must be careful with derivatives as
\[
\left. \ddx \sin\left(x^2\right)\right|_{x=90^\circ} \ne \underbrace{2\cdot 90\cdot \cos(90^2)}_{\text{incorrect}}
\]
Alternatively, one could think of $x^\circ$ as meaning
$\frac{x\cdot\pi}{180}$, as then $90^\circ = \frac{90\cdot\pi}{180} =
\frac{\pi}{2}$. In this case
\[
2\cdot 90^\circ\cdot \cos((90^\circ)^2) = 2\cdot \frac{\pi}{2}\cdot\cos\left(\left(\frac{\pi}{2}\right)^2\right).
\]
\end{warning}

\end{document}
