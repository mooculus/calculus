\documentclass{ximera}

\newcommand{\RR}{\mathbb R}
\renewcommand{\d}{\,d}
\newcommand{\dd}[2][]{\frac{d #1}{d #2}}
\renewcommand{\l}{\ell}
\newcommand{\ddx}{\frac{d}{dx}}
\newcommand{\dfn}{\textbf}
\newcommand{\eval}[1]{\bigg[ #1 \bigg]}


\title[Dig-In:]{Exponential and logarithmic functions}


\begin{document}
\begin{abstract}
  Exponential and logarithmic functions illuminated.
\end{abstract}
\maketitle

Exponential and logarithmic functions may seem somewhat esoteric at
first, but they model many phenomena in the real-world.




\section{What are exponential and logarithmic functions?}


\begin{definition}
  An \dfn{exponential function} is a function of the form
  \[
  f(x) = b^x
  \]
  where  $b\neq 1$ is a positive real number. The domain of an
  exponential function is $(-\infty,\infty)$.
\end{definition}

\begin{question}
  Is $b^{-x}$ an exponential function?
  \begin{prompt}
  \begin{multipleChoice}
    \choice[correct]{yes}
    \choice{no}
  \end{multipleChoice}
  \end{prompt}
  \begin{feedback}
    Note that
    \[
    b^{-x} = \left(b^{-1}\right)^x = \left(\frac{1}{b}\right)^x.
    \]
  \end{feedback}
\end{question}



\begin{definition}
  A \dfn{logarithmic function} is a function defined as follows
  \[
  \log_b(x) = y \qquad\text{means that}\qquad b^y = x
  \]
  where  $b\ne 1$ is a positive real number. The domain of a
  logarithmic function is $(0,\infty)$.
\end{definition}

In either definition above $b$ is called the \dfn{base}.

Remember that with exponential and logarithmic functions, there is one very special
base:
\[ e = 2.7182818284590\ldots \]
This is an irrational number that you will see frequently.  The exponential with base $e$,
$f(x) = e^x$ is often called the `natural exponential' function.  For the logarithm with base $e$,
we have a special notation, $\ln(x)$ is `natural logarithm' function.

\subsection{Connections between exponential functions and logarithms}

Let $b$ be a positive real number with $b\ne 1$.
\begin{itemize}
\item $b^{\log_b(x)} = x$ for all positive $x$
\item $\log_b(b^x) = x$ for all real $x$
\end{itemize}

\begin{question}
  What exponent makes the following expression true?
  \[
  3^x = e^{\left( x \cdot \answer{\ln 3} \right)}.
  \]
\end{question}


\section{What can the graphs look like?}

\subsection{Graphs of exponential functions}

\begin{example}
  Here we see the the graphs of four exponential functions.
  \begin{image}
    \begin{tikzpicture}
      \begin{axis}[
          domain=-2:2,
          xmin=-2, xmax=2,
          ymin=-.5, ymax=4,
          axis lines =middle, xlabel=$x$, ylabel=$y$,
          every axis y label/.style={at=(current axis.above origin),anchor=south},
          every axis x label/.style={at=(current axis.right of origin),anchor=west},
        ]
	\addplot [very thick, penColor, smooth] {e^x};
        \addplot [very thick, penColor2, smooth] {2^x)};
        \addplot [very thick, penColor3, smooth] {(1/2)^x)};
        \addplot [very thick, penColor4, smooth] {(1/3)^x)};
        
        
        
        \node at (axis cs:-1.5, 2 ) [penColor3,anchor=west] {$A$};
        \node at (axis cs:-.8, 2.6 ) [penColor4,anchor=west] {$B$};
        \node at (axis cs:0.6, 2.6 ) [penColor,anchor=west] {$C$};
        \node at (axis cs:1.2, 2 ) [penColor2,anchor=west] {$D$};
        
      \end{axis}
    \end{tikzpicture}
  \end{image}
  Match the curves $A$, $B$, $C$, and $D$ with the functions
  \[
  e^x, \qquad \left(\frac{1}{2}\right)^{x}, \qquad  \left(\frac{1}{3}\right)^{x}, \qquad 2^{x}.
  \]
  \begin{explanation}
    One way to solve these problems is to compare these functions
    along the vertical line $x=1$,
    \begin{image}
      \begin{tikzpicture}
        \begin{axis}[
            domain=-2:2,
            xmin=-2, xmax=2,
            ymin=-.5, ymax=4,
            axis lines =middle, xlabel=$x$, ylabel=$y$,
            every axis y label/.style={at=(current axis.above origin),anchor=south},
            every axis x label/.style={at=(current axis.right of origin),anchor=west},
          ]
	  \addplot [very thick, penColor, smooth] {e^x}; %C
          \addplot [very thick, penColor2, smooth] {2^x)};%D
          \addplot [very thick, penColor3, smooth] {(1/2)^x)};%A
          \addplot [very thick, penColor4, smooth] {(1/3)^x)};%B
            
          \node at (axis cs:-1.5, 2 ) [penColor3,anchor=west] {$A$};
          \node at (axis cs:-.8, 2.6 ) [penColor4,anchor=west] {$B$};
          \node at (axis cs:0.6, 2.6 ) [penColor,anchor=west] {$C$};
          \node at (axis cs:1.2, 2 ) [penColor2,anchor=west] {$D$};

          \addplot [textColor, dashed] plot coordinates {(1,-.5) (1,4)};

          \addplot[color=penColor,fill=penColor,only marks,mark=*] coordinates{(1,e)}; %C
          \addplot[color=penColor2,fill=penColor2,only marks,mark=*] coordinates{(1,2)}; %D
          \addplot[color=penColor3,fill=penColor3,only marks,mark=*] coordinates{(1,1/2)}; %A
          \addplot[color=penColor4,fill=penColor4,only marks,mark=*] coordinates{(1,1/3)}; %B
        \end{axis}
      \end{tikzpicture}
    \end{image}
    Note
    \[
    \left(\frac{1}{3}\right)^1 < \left(\frac{1}{2}\right)^1  < 2^1 < e^1.
    \]
    Hence we see:
    \begin{itemize}
    \item $\left(\frac{1}{3}\right)^{x}$ corresponds to
      $\answer[given]{B}$.
    \item $\left(\frac{1}{2}\right)^{x}$ corresponds to $\answer[given]{A}$.
    \item $2^x$ corresponds to $\answer[given]{D}$.
    \item $e^x$ corresponds to $\answer[given]{C}$.
    \end{itemize}
  \end{explanation}
\end{example}



\subsection{Graphs of logarithmic functions}


\begin{example}
  Here we see the the graphs of four logarithmic functions.
  \begin{image}
    \begin{tikzpicture}
      \begin{axis}[
          domain=0.05:4,
          xmin=-.5, xmax=4,
          ymin=-2, ymax=2,
          axis lines =middle, xlabel=$x$, ylabel=$y$,
          every axis y label/.style={at=(current axis.above origin),anchor=south},
          every axis x label/.style={at=(current axis.right of origin),anchor=west},
        ]
	\addplot [very thick, penColor, smooth] {ln(x)}; % C
        \addplot [very thick, penColor2, smooth] {ln(x)/ln(2)}; % D
        \addplot [very thick, penColor3, smooth, samples=100] {ln(x)/ln(1/2))}; % A
        \addplot [very thick, penColor4, smooth, samples=100] {ln(x)/ln(1/3))}; %B
        
        
        \node at (axis cs:.5, 1.3 ) [penColor3,anchor=west] {$A$};
        \node at (axis cs:.2, .5 ) [penColor4,anchor=west] {$B$};
        \node at (axis cs:0.2, -.5 ) [penColor,anchor=west] {$C$};
        \node at (axis cs:.5, -1.3 ) [penColor2,anchor=west] {$D$};
        
      \end{axis}
    \end{tikzpicture}
  \end{image}
  Match the curves $A$, $B$, $C$, and $D$ with the functions
  \[
  \ln(x),\qquad \log_{1/2}(x), \qquad \log_{1/3}(x),\qquad \log_2(x).
  \]
  \begin{explanation}
    First remember what $\log_b(x)=y$ means:
    \[
    \log_b(x) = y \qquad\text{means that}\qquad b^y = x.
    \]
    Moreover, $\ln(x) = \log_e(x)$ where $e= 2.71828\dots$.  So now
    examine each of these functions along the horizontal line $y=1$
    \begin{image}
      \begin{tikzpicture}
        \begin{axis}[
            domain=0.05:4,
            xmin=-.5, xmax=4,
            ymin=-2, ymax=2,
            axis lines =middle, xlabel=$x$, ylabel=$y$,
            every axis y label/.style={at=(current axis.above origin),anchor=south},
            every axis x label/.style={at=(current axis.right of origin),anchor=west},
          ]
	  \addplot [very thick, penColor, smooth] {ln(x)}; % C
          \addplot [very thick, penColor2, smooth] {ln(x)/ln(2)}; % D
          \addplot [very thick, penColor3, smooth, samples=100] {ln(x)/ln(1/2))}; % A
          \addplot [very thick, penColor4, smooth, samples=100] {ln(x)/ln(1/3))}; %B
          \addplot [dashed] {1};
        
          
          \node at (axis cs:.5, 1.3 ) [penColor3,anchor=west] {$A$};
          \node at (axis cs:.2, .5 ) [penColor4,anchor=west] {$B$};
          \node at (axis cs:0.2, -.5 ) [penColor,anchor=west] {$C$};
          \node at (axis cs:.5, -1.3 ) [penColor2,anchor=west] {$D$};

          \addplot[color=penColor,fill=penColor,only marks,mark=*] coordinates{(e,1)}; %C
          \addplot[color=penColor2,fill=penColor2,only marks,mark=*] coordinates{(2,1)}; %D
          \addplot[color=penColor3,fill=penColor3,only marks,mark=*] coordinates{(1/2,1)}; %A
          \addplot[color=penColor4,fill=penColor4,only marks,mark=*] coordinates{(1/3,1)}; %B
        \end{axis}
      \end{tikzpicture}
    \end{image}
    Note again (this is from the definition of a logarithm)
    \[
    \left(\frac{1}{3}\right)^1 < \left(\frac{1}{2}\right)^1  < 2^1 < e^1.
    \]
    Hence we see:
    \begin{itemize}
    \item $\log_{1/3}(x)$ corresponds to $\answer[given]{B}$.
    \item $\log_{1/2}(x)$ corresponds to $\answer[given]{A}$.
    \item $\log_2(x)$ corresponds to $\answer[given]{D}$.
    \item $\ln(x)$ corresponds to $\answer[given]{C}$.
    \end{itemize}
  \end{explanation}
\end{example}



\section{Properties of exponential functions and logarithms}

Working with exponential and logarithmic functions is often simplified by  
applying properties of these functions.  These properties will make appearances 
throughout our work.

\subsection{Properties of exponents}
Let $b$ be a positive real number with $b\neq 1$.
\begin{itemize}
  \item $b^m\cdot b^n = b^{m+n}$
  \item $b^{-1} = \frac{1}{b}$
  \item $\left(b^m\right)^n = b^{mn}$
\end{itemize}
\begin{question}
  What exponent makes the following true?
  \[
  2^4 \cdot 2^3 = 2^{\answer{7}}
  \]
  \begin{hint}
    \[
    (2^4) \cdot (2^3) = (2 \cdot 2\cdot 2 \cdot 2) \cdot  (2 \cdot 2\cdot 2)
    \]
  \end{hint}
\end{question}

\subsection{Properties of logarithms}
Let $b$ be a positive real number with $b\neq 1$.
\begin{itemize}
\item $\log_b(m\cdot n) = \log_b(m) + \log_b(n)$
\item $\log_b(m^n) = n\cdot \log_b(m)$
\item $\log_b\left(\frac{1}{m}\right) = \log_b(m^{-1}) = -\log_b(m)$
\item $\log_a(m) = \frac{\log_b(m)}{\log_b(a)}$
\end{itemize}

\begin{question}
  What value makes the following expression true?
  \[
  \log_2\left(\frac{8}{16}\right) = 3-\answer{4}
  \]
\end{question}


\begin{question}
  What makes the following expression true?
  \[
  \log_3(x) = \frac{\ln(x)}{\answer{\ln(3)}}
  \]
\end{question}

\begin{example}
	Solve the equation: $\displaystyle 5^{2x-3} = 7$.
	\begin{explanation}
		Since we can't easily rewrite both sides as exponentials with the same base, we'll use logarithms instead.  Above we said that
		$\log_b(x) = y$ means that $b^y = x$.  That statement means that each exponential equation has an equivalent logarithmic form
		and vice-versa.  We'll convert to a logarithmic equation and solve from there.
		\begin{align*}
			5^{2x-3} &= 7\\
			\log_{\answer{5}}\left(  \answer{7} \right) &= 2x-3
		\end{align*}
		From here, we can solve for $x$ directly.
		\begin{align*}
			2x &= \log_{5}\left(7\right) + 3\\
			x &= \frac{\log_{5}\left(7\right) + 3}{2}
		\end{align*}
	\end{explanation} 
\end{example}


\begin{example}
	Solve the equation: $\displaystyle e^{2x} = e^x + 6$. 
	\begin{explanation}
		Immediately taking logarithms of both sides will not help here, as the right side has multiple terms.  We know that logarithms
		behave well with products and quotients, but not with sums. 
		Notice that $e^{2x} = \left(e^x\right)^2$. (This is a common trick that you will likely see many times.)  
		\begin{align*}
			e^{2x} &= e^x + 6\\
			\left(e^x\right)^2 &= e^x + 6\\
			\left(e^x\right)^2 - e^x - 6 &= 0
		\end{align*}
		Our equation is really a quadratic equation in $e^x$.  
		The left-hand side factors as $\left( e^x - \answer{3}\right) \left(e^x + \answer{2}\right)$, so we are dealing
		with \[ e^x - \answer{3} = 0  \qquad  \textrm{and} \qquad e^x+\answer{2} = 0.\]
		For the first factor:
		\begin{align*}
			e^x &= \answer{3}\\
			x &= \ln\left( \answer{3}\right).
		\end{align*}
		
		From the second factor: $\displaystyle e^x = \answer{-2}$.  Recall from above that the range of the exponential function is $(0, \infty)$. 
		There is no input to make the output a negative
		number, so $e^x = -2$ has no solutions. 
		
		The solution to $\displaystyle e^{2x} = e^x + 6$ is $x = \answer{\ln(3)}$.
	\end{explanation}
\end{example}

\end{document}
