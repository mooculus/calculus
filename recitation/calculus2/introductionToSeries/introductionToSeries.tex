\documentclass[]{ximera}
%handout:  for handout version with no solutions or instructor notes
%handout,instructornotes:  for instructor version with just problems and notes, no solutions
%noinstructornotes:  shows only problem and solutions

%% handout
%% space
%% newpage
%% numbers
%% nooutcomes

%I added the commands here so that I would't have to keep looking them up
%\newcommand{\RR}{\mathbb R}
%\renewcommand{\d}{\,d}
%\newcommand{\dd}[2][]{\frac{d #1}{d #2}}
%\renewcommand{\l}{\ell}
%\newcommand{\ddx}{\frac{d}{dx}}
%\everymath{\displaystyle}
%\newcommand{\dfn}{\textbf}
%\newcommand{\eval}[1]{\bigg[ #1 \bigg]}

%\begin{image}
%\includegraphics[trim= 170 420 250 180]{Figure1.pdf}
%\end{image}

%add a ``.'' below when used in a specific directory.

\newcommand{\RR}{\mathbb R}
\renewcommand{\d}{\,d}
\newcommand{\dd}[2][]{\frac{d #1}{d #2}}
\renewcommand{\l}{\ell}
\newcommand{\ddx}{\frac{d}{dx}}
\newcommand{\dfn}{\textbf}
\newcommand{\eval}[1]{\bigg[ #1 \bigg]}




\author{Tom Needham}

\outcome{Recognize and evaluate geometric series.}
\outcome{Use properties of a sequence to determine properties of its sequence of partial sums.}

\title[]{Introduction to Series}

\begin{document}
\begin{abstract}
\end{abstract}
\maketitle

\vspace{-0.9in}

\section{Discussion Questions}

\begin{problem}
Let $\{a_n\}_{n=1}^\infty$ be the sequence given by the formula
$$
a_n = \frac{n!}{3^n}.
$$
Find the first three terms in the sequence of partial sums associated to $\{a_n\}$. 
\end{problem}

\begin{freeResponse}
The first three terms in the partial sum sequence are 
\begin{align*}
s_1 &= a_1 = \frac{1}{3} \\
s_2 &= a_1 + a_2 = \frac{1}{3} + \frac{2}{9} = \frac{5}{9} \\
s_3 &= s_2 + a_3 = \frac{5}{9} + \frac{6}{27} = \frac{7}{9}.
\end{align*}
\end{freeResponse}

\begin{problem}
Determine which of the following are geometric series.
\begin{center}
\begin{tabular}{lll}
I. $\sum_{n=10}^\infty 3 \cdot 2^{1+n}$ \hspace{.2in} II. $\sum_{n=0}^\infty \left(\frac{1}{3n}\right)^n$ \hspace{.2in} III. $\sum_{k=2}^\infty 5 \cdot \frac{1}{k^2}$ \hspace{.2in} IV. $\sum_{k=2}^\infty 5 \cdot \frac{1}{2^k}$
\end{tabular}
\end{center}
\end{problem}

\begin{freeResponse}
A geometric series is a series of the form
$$
\sum_{n=n_0}^N A \cdot r^n
$$
for some integer $n_0$ and some constants $A$ and $r$. This is equivalent to the statement that the consecutive terms in the series have a common constant ratio for all $n$. 

I. This series is clearly geometric.

II. This is not a geometric series. We can see this by inspection, but to be thorough we should check the ratio of the terms. Since
$$
\frac{\left(\frac{1}{3(n+1)}\right)^{n+1}}{\left(\frac{1}{3n}\right)^n} = \left(\frac{n}{n+1}\right)^n \frac{1}{3(n+1)}
$$
is not constant (check a few terms, to be sure!), the series is not geometric.

III. This series is not geometric. Checking ratios of consecutive terms, we see that
$$
\frac{5 \frac{1}{(k+1)^2}}{ 5 \frac{1}{k^2}} = \frac{k^2}{(k+1)^2}
$$
is nonconstant.

IV. This series is geometric, since a slight algebraic manipulation puts it into exactly the standard form:
$$
\sum_{k=2}^\infty 5 \cdot \frac{1}{2^k} = \sum_{k=2}^\infty 5 \cdot \left(\frac{1}{2}\right)^k.
$$
\end{freeResponse}

\begin{problem}
Suppose that $\{a_n\}_{n=0}^\infty$ is a sequence with sequence of partial sums $\{s_n\}_{n=0}^\infty$ given by the explicit formula
$$
s_n = \left(\frac{1}{3}\right)^n.
$$
Find the values of the following series.
\begin{center}
\begin{tabular}{ll}
I. $\sum_{n=0}^\infty s_n$ \hspace{.6in} II. $\sum_{n=0}^\infty a_n$
\end{tabular}
\end{center}
\end{problem}

\begin{freeResponse}
I. Using our result on sums of geometric series, we have
$$
\sum_{n=0}^\infty s_n = \sum_{n=0}^\infty \left(\frac{1}{3}\right)^n = \frac{1}{1-1/3} = \frac{3}{2}.
$$

II. By definition,
$$
\sum_{n=0}^\infty a_n = \lim_{n \rightarrow \infty} s_n = 0.
$$
\end{freeResponse}

\begin{problem}
Let $\{a_n\}_{n=1}^\infty$ be a sequence and $\{s_n\}_{n=1}^\infty$ its sequence of partial sums. Fill in the missing values in the following table.

\begin{center}
\begin{tabular}{ |c|c|c| } 
 \hline
$n$ & $a_n$ & $s_n$ \\ 
 \hline
 \hline
  3 & 3 & 2 \\ 
  \hline
    4 & 5 &  \\ 
 \hline
   5 & 1 & 8 \\ 
 \hline
   6 &  & 10 \\ 
 \hline
\end{tabular}
\end{center}

\end{problem}

\begin{freeResponse}
We need to find $s_4$ and $a_6$. To find $s_4$, note that 
$$
s_4 = s_3 + a_4 = 2 + 5 = 7.
$$
Similarly, to find $a_6$, note that
$$
s_6 = s_5 + a_ 6,
$$
or 
$$
10 = 8 + a_6,
$$
whence $a_6 = 2$. 
\end{freeResponse}

\section{Group Work}

\begin{problem}
Let $\{a_n\}_{n=1}^\infty$ be a sequence with sequence of partial sums $\{s_n\}_{n=1}^\infty$ given explicitly by the formula
$$
s_n = \frac{n^3-5n^2}{(1+2n)^3}.
$$
Determine the values of the following series.
\begin{center}
\begin{tabular}{ll}
I. $\sum_{n=1}^\infty a_n$ \hspace{.6in} II. $\sum_{n=3}^\infty a_n$
\end{tabular}
\end{center}
\end{problem}

\begin{freeResponse}
I. By definition, 
$$
\sum_{n=1}^\infty a_n = \lim_{n\rightarrow \infty} s_n = \lim_{n\rightarrow \infty}\frac{n^3-5n^2}{(1+2n)^3} = \frac{1}{8},
$$
where we have computed the limit using the method of growth rate comparison.

II. We have
$$
\sum_{n=3}^\infty a_n = \sum_{n=1}^\infty a_n - \sum_{n=1}^3 a_n = \sum_{n=1}^\infty a_n - s_3 = \frac{1}{8} - \frac{-18}{343},
$$
where the last two terms are computed using part I and the explicit formula for $s_n$, respectively.
\end{freeResponse}

\begin{problem}
Given that 
$$
\sum_{k=2}^\infty \frac{3}{k^2 + k} = \frac{3}{2},
$$
determine the value of 
$$
\sum_{k=4}^\infty \frac{9}{k^2 + k}.
$$
\end{problem}

\begin{freeResponse}
We have
$$
\sum_{k=4}^\infty \frac{3}{k^2 + k} = \sum_{k=2}^\infty \frac{3}{k^2 + k} - \sum_{k=2}^3 \frac{3}{k^2 + k} = \frac{3}{2} - \frac{3}{6} - \frac{3}{12} = \frac{5}{6},
$$
and it follows that 
$$
\sum_{k=4}^\infty \frac{9}{k^2 + k} = 3 \sum_{k=4}^\infty \frac{3}{k^2 + k} = 3 \cdot \frac{5}{6} = \frac{5}{2}.
$$
\end{freeResponse}

\begin{problem}
Determine whether each of the following series converges or diverges. If it converges, find its value.
\begin{center}
\begin{tabular}{lll}
I. $\sum_{n=0}^\infty 2^{1-2n}$ \hspace{.2in} II. $\sum_{n=0}^\infty \frac{5 \cdot 2^{n+3}}{3^{2n}}$ \hspace{.2in} III. $\sum_{n=0}^\infty 7 \cdot \left(\frac{5}{9}\right)^{2-3n}$ \hspace{.2in} IV. $\sum_{k=-3}^\infty \left(\frac{1}{4}\right)^{k+1}$
\end{tabular}
\end{center}
\end{problem}

\begin{freeResponse}
The general strategy is to algebraically manipulate the terms in each sum to put them into the standard form for a geometric series.

I. The first series can be rewritten as 
$$
\sum_{n=0}^\infty 2^{1-2n} = \sum_{n=0}^\infty 2 \cdot 2^{-2n} = \sum_{n=0}^\infty 2 \cdot \left(\frac{1}{4}\right)^n = \frac{2}{1-1/4} = \frac{8}{3}.
$$

II. A similar approach works here, and we have
$$
\sum_{n=0}^\infty \frac{5 \cdot 2^{n+3}}{3^{2n}} = \sum_{n=0}^\infty \frac{5 \cdot 2^3 \cdot 2^n}{9^n} = \sum_{n=0}^\infty 40 \cdot \left(\frac{2}{9}\right)^n = \frac{40}{1-2/9} = \frac{360}{7}.
$$

III. Continuing with this approach, we have
$$
\sum_{n=0}^\infty 7 \cdot \left(\frac{5}{9}\right)^{2-3n} = \sum_{n=0}^\infty 7 \cdot \left(\frac{5}{9}\right)^2 \left(\frac{5}{9}\right)^{-3n} = \sum_{n=0}^\infty 7 \cdot \left(\frac{5}{9}\right)^2 \left(\frac{9^3}{5^3}\right)^{n}.
$$
In this case, the ratio $9^3/5^3$ is larger than $1$ and we conclude that this series diverges.

IV. Finally, this example requires us to manipulate the series by separating off the piece which doesn't correspond to the standard indexing. That is, 
\begin{align*}
\sum_{k=-3}^\infty \left(\frac{1}{4}\right)^{k+1} &= \sum_{k=-3}^{-1} \left(\frac{1}{4}\right)^{k+1} + \sum_{k=0}^\infty \left(\frac{1}{4}\right)^{k+1} \\
&= \sum_{k=-3}^{-1} \left(\frac{1}{4}\right)^{k+1} + \sum_{k=0}^\infty \frac{1}{4} \left(\frac{1}{4}\right)^{k} \\
&= 4^2 + 4 + 1 + \frac{1/4}{1-1/4} = \frac{64}{3}.
\end{align*}
\end{freeResponse}
\end{document}
