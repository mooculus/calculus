\documentclass{ximera}

\newcommand{\RR}{\mathbb R}
\renewcommand{\d}{\,d}
\newcommand{\dd}[2][]{\frac{d #1}{d #2}}
\renewcommand{\l}{\ell}
\newcommand{\ddx}{\frac{d}{dx}}
\newcommand{\dfn}{\textbf}
\newcommand{\eval}[1]{\bigg[ #1 \bigg]}


\outcome{State the Intermediate Value Theorem including hypotheses.}

\author{Nela Lakos \and Kyle Parsons}


\begin{document}
\begin{exercise}


The figure below shows the graph of a function $f$.

\begin{image}
  \begin{tikzpicture}
    \begin{axis}[
        xmin=0,xmax=5.3,ymin=-1.3,ymax=4.3,
        clip=false,
        unit vector ratio*=1 1 1,
        axis lines=center,
        grid = major,
        ytick={-1,-0,...,4},
    xtick={1,2,...,5},
        xlabel=$x$, ylabel=$y$,
        every axis y label/.style={at=(current axis.above origin),anchor=south},
        every axis x label/.style={at=(current axis.right of origin),anchor=west},
      ]
      \addplot[very thick, penColor, domain=0.7:2] {x+2};
      \addplot[very thick, penColor, domain=2:3] {4-2*sqrt(x-2)};
      \addplot[very thick, penColor, domain=3:4] {sqrt(x-3)+2};
      \addplot[very thick, penColor, domain=4:5.3] {-2*x+11};

      \node[penColor] at (axis cs:2, 1.2) [penColor] {$y=f(x)$};
      \end{axis}`
  \end{tikzpicture}
\end{image}





Select all the following statements that are correct.

\begin{selectAll}
\choice [correct]{When we apply the Intermediate value theorem to $f$ on the interval $\left[1,5\right]$ with $L=2$, we are guaranteed the existence of at least one point $c$ such that $f(c)=L$.}
\choice {When we apply the Intermediate value theorem to $f$ on the interval $\left[1,4\right]$ with $L=2$, we are guaranteed the existence of at least one point $c$ such that $f(c)=L$.}
\choice {When we apply the Intermediate value theorem to $f$ on the interval $\left[1,5\right]$ with $L=4$, we are guaranteed the existence of at least one point $c$ such that $f(c)=L$.}
\choice {When we apply the Intermediate value theorem to $f$ on the interval $\left[3,5\right]$ with $L=3$, we are guaranteed the existence of at least one point $c$ such that $f(c)=L$.}
\choice[correct] {When we apply the Intermediate value theorem to $f$ on the interval $\left[2,5\right]$ with $L=\pi$, we are guaranteed the existence of at least one point $c$ such that $f(c)=L$.}
\choice[correct] {When we apply the Intermediate value theorem to $f$ on the interval $\left[2,4\right]$ with $L=\pi$, we are guaranteed the existence of at least one point $c$ such that $f(c)=L$.}
\choice {When we apply the Intermediate value theorem to $f$ on the interval $\left[1,5\right]$ with $L=\pi$, we are guaranteed the existence of at least one point $c$ such that $f(c)=L$.}
\end{selectAll}



\end{exercise}
\end{document}