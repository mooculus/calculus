\documentclass{ximera}
\newcommand{\RR}{\mathbb R}
\renewcommand{\d}{\,d}
\newcommand{\dd}[2][]{\frac{d #1}{d #2}}
\renewcommand{\l}{\ell}
\newcommand{\ddx}{\frac{d}{dx}}
\newcommand{\dfn}{\textbf}
\newcommand{\eval}[1]{\bigg[ #1 \bigg]}

\author{ Nela Lakos\and Bart Snapp}
\license{Creative Commons 3.0 By-NC}
\title{Learning Outcomes for Calculus I }

\begin{document}

\begin{abstract}
Concepts, facts and skills
\end{abstract}
\maketitle
\section{Understanding functions}
\begin{itemize}
	\item State the definition of a function.
	\item Find the domain and range of a function.
	\item Distinguish between functions by considering their domains.
	\item Determine where a function is positive or negative.
	\item Plot basic functions.
        \item Perform basic operations and compositions on
          functions.
        \item Work with piecewise defined functions.
	\item Determine if a function is one-to-one.
	\item Recognize different representations of the same function.
        \item Define and work with inverse functions.
        \item Plot inverses of basic functions.
	\item Find inverse functions (algebraically and graphically).
        \item Find the largest interval containing a given point
          where the function is invertible.
	\item Determine the intervals on which a function has an inverse.

\end{itemize}

\section{Review of famous functions}


\begin{itemize}
	\item Know the graphs and properties of ``famous'' functions.
	\item Know and use the properties of exponential and logarithmic functions.
	\item Understand the relationship between exponential and logarithmic functions.
        \item Understand the definition of a rational function.
	\item Understand the properties of trigonometric functions.
	\item Evaluate expressions and solve equations involving
          trigonometric functions and inverse trigonometric functions.
\end{itemize}
\section{What is a limit?}
     \begin{itemize}
	\item Consider  values of a function  at inputs approaching a given point.
	\item Understand the concept of a limit.
        \item Use limits to understand local behavior of functions.
	\item Calculate limits from a graph (or state that the limit does not exist).
	\item Understand possible issues when estimating limits using
          nearby values.
	\item Define a one-sided limit.
	\item Explain the relationship between one-sided and two-sided limits.
	\item Distinguish between limit values and function values.
	\item Identify when a limit does not exist.
	\item Define continuity in terms of limits.
	\item Use the continuity of famous functions (on their domains) when computing limits. 
\end{itemize}
\section{Limit laws}
\begin{itemize}
\item Calculate limits using the limit laws.
 \item Calculate limits by replacing a function with a continuous
  function that has the same limit.
\item Understand the Squeeze Theorem and how it can be used to find limit values.
\item Calculate limits using the Squeeze Theorem.\end{itemize}

\end{document}
