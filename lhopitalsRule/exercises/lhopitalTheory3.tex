\documentclass{ximera}
\newcommand{\RR}{\mathbb R}
\renewcommand{\d}{\,d}
\newcommand{\dd}[2][]{\frac{d #1}{d #2}}
\renewcommand{\l}{\ell}
\newcommand{\ddx}{\frac{d}{dx}}
\newcommand{\dfn}{\textbf}
\newcommand{\eval}[1]{\bigg[ #1 \bigg]}

\author{Steven Gubkin}
\license{Creative Commons 3.0 By-NC}

\outcome{Recall how to find limits for forms that are not indeterminate.}
\outcome{Define an indeterminate form.}
\outcome{Determine if a form is indeterminate.}
\outcome{Convert indeterminate forms to the form zero over zero or infinity over infinity.}
\outcome{Define L’hopital’s Rule and identify when it can be used.}
\outcome{Use L’hopital’s Rule to find limits.}
\begin{document}
\begin{exercise}

Decide whether l'H\^{o}pital's rule immediately applies to the following limit.  If it does not, explain why not.  Find the limit by any means necessary, or state that it does not exist. 

\[
\lim_{x \to 0} \frac{\sin^2(x)\sin(\frac{1}{x})}{x\sin(\frac{1}{2x})}
\]

\begin{prompt}
	Call the numerator $f(x)$ and the denominator $g(x)$. 

	\begin{multipleChoice}
	\choice{l'H\^{o}pital's rule applies}
	\choice{l'H\^{o}pital's rule does not apply since the limit is not an indeterminate form }
	\choice[correct]{l'H\^{o}pital's rule does not apply $g'(x)$ has zeros on every interval $(a, \infty)$ }
	\choice{l'H\^{o}pital's rule does not apply because $\lim_{x \to \infty} \frac{f'(x)}{g'(x)}$ does not exist.}
\end{multipleChoice}

If the limit does not exist, write $DNE$.

\[
\lim_{x \to 0} \frac{\sin^2(x)\sin(\frac{1}{x})}{x\sin(\frac{1}{2x})} = \answer{0}
\]

\end{prompt}

\end{exercise}
\end{document}