\documentclass{ximera}

\newcommand{\RR}{\mathbb R}
\renewcommand{\d}{\,d}
\newcommand{\dd}[2][]{\frac{d #1}{d #2}}
\renewcommand{\l}{\ell}
\newcommand{\ddx}{\frac{d}{dx}}
\newcommand{\dfn}{\textbf}
\newcommand{\eval}[1]{\bigg[ #1 \bigg]}


\author{Jim Talamo and Alex Beckwith}
\license{Creative Commons 3.0 By-NC}


\outcome{Set up an integral that gives the length of a curve segment and evaluate it.}

\begin{document}
\begin{exercise}

An inverted conical tank has base with diameter $6\unit{m}$ and height
$9\unit{m}$. Suppose the tank is filled to a height of $6\unit{m}$
with acetone ($\rho=785 \unit{kg}/\unit{m}^3$). Find the work required
to pump the acetone out of the tank (use $g=9.8\unit{m}/\unit{s}^2$).
\begin{image}
\begin{tikzpicture}
\begin{axis}[
domain=-3:3,
xmin=-3.5, xmax=4.5,
xtick={-3,-2,-1,1,2,3},
ymin=-1, ymax=11,
axis lines =center,
xlabel=$x$, ylabel=$y$, every axis y label/.style={at=(current axis.above origin),anchor=south},
every axis x label/.style={at=(current axis.right of origin),anchor=west},
axis on top,
]

\draw[penColor,very thick,smooth,fill=fill4] (axis cs: 0,9) ellipse (300 and 10);
\draw[penColor,very thick,smooth,fill=blue,opacity=0.25] (axis cs: 0,6) ellipse (200 and 6.67);
\draw[penColor,very thick,smooth] (axis cs:2,6) arc (360:180:200 and 6.67);
\draw[penColor,very thick,dashed] (axis cs:2,6) arc (0:180:200 and 6.67);
\addplot [penColor,very thick,smooth,domain=0:3]	{3*x};
\addplot [penColor,very thick,smooth,domain=-3:0]	{-3*x};

\addplot [name path=A,domain=0:3,draw=none] {3*x};   
\draw [name path=B,draw=none] (axis cs: 3,9) arc (0:180:300 and 10);
\addplot [fill4,opacity=0.5] fill between[of=A and B];

\addplot [name path=C,domain=-3:0.01,draw=none] {-3*x};   
\draw [name path=D,draw=none] (axis cs: -3,9) arc (180:360:300 and 10);
\addplot [fill4,opacity=0.5] fill between[of=C and D];

\addplot [name path=E,domain=-3:0.01,draw=none] {-3*x};   
\draw [name path=F,draw=none]  (axis cs:2,6) arc (0:180:200 and 6.67);
\addplot [blue,opacity=0.25] fill between[of=E and F];

\draw[decoration={brace,raise=.1cm},decorate,thin] (axis cs:2,6) -- (axis cs:2,0);
\node[anchor=west] at (axis cs:2.2,3) {$6\unit{m}$};
\draw[decoration={brace,raise=.1cm},decorate,thin] (axis cs:3,9) -- (axis cs:3,0);
\node[anchor=west] at (axis cs:3.2,4.5) {$9\unit{m}$};
\addplot[thick] plot coordinates {(0,6) (2,6)};
         
\end{axis}
\end{tikzpicture}
\end{image}

\[
W= \int_{y=\answer{0}}^{y=\answer{6}} \answer{7693\pi\left(\frac{y}{3}\right)^2 (9-y)}\d y = \answer[tolerance=10]{276948\pi} \unit{J}
\]

\begin{hint}
Set the height $y=0$ at the base of the tank.  We want to use the formula:

\[ 
W = \int_{y=0}^{y=b} \rho g A(y) d(y) \d y
\]

Since $b$ is the height to which the tank is filled, $b=\answer{6}$.

Since $h$ is the height to which the water must be moved, $h=\answer{9}$.

You want to move a slice at height $y$ to a height of $9$. Letting
$d(y)$ represent the distance that a slice at height $y$ travels to
get to a height of $9$, we set $d(y) = \answer{9-y}$.

The cross-sectional area of this tank is:

\begin{multipleChoice}
\choice{is constant.}
\choice[correct]{varies depending on the height $y$ of the slice}.
\end{multipleChoice}

\begin{question}
We can find the radius of the slice in terms of $y$ by using similar triangles:

\begin{image}
\begin{tikzpicture}
\begin{axis}[
domain=-3:3,
xmin=-3.5, xmax=4.5,
xtick={-3,-2,-1,1,2,3},
ymin=-1, ymax=11,
axis lines =center,
xlabel=$x$, ylabel=$y$, every axis y label/.style={at=(current axis.above origin),anchor=south},
every axis x label/.style={at=(current axis.right of origin),anchor=west},
axis on top,
]

%top ellipse
\draw[penColor,very thick,smooth,fill=fill4] (axis cs: 0,9) ellipse (300 and 10);
\draw[penColor,very thick,smooth,fill=blue,opacity=0.25] (axis cs: 0,6) ellipse (200 and 6.67);

%middle
\draw[penColor,very thick,smooth] (axis cs:2,6) arc (360:180:200 and 6.67);
\draw[penColor,very thick,dashed] (axis cs:2,6) arc (0:180:200 and 6.67);

%slice
\draw[penColor2,very thick,smooth] (axis cs:1,3) arc (360:180:98 and 4);
\draw[penColor2,very thick,dashed] (axis cs:1,3) arc (0:180:98 and 4);

\addplot [penColor,very thick,smooth,domain=0:3]	{3*x};
\addplot [penColor,very thick,smooth,domain=-3:0]	{-3*x};

\addplot [name path=A,domain=0:3,draw=none] {3*x};   
\draw [name path=B,draw=none] (axis cs: 3,9) arc (0:180:300 and 10);
\addplot [fill4,opacity=0.5] fill between[of=A and B];

\addplot [name path=C,domain=-3:0.01,draw=none] {-3*x};   
\draw [name path=D,draw=none] (axis cs: -3,9) arc (180:360:300 and 10);
\addplot [fill4,opacity=0.5] fill between[of=C and D];

\addplot [name path=E,domain=-3:0.01,draw=none] {-3*x};   
\draw [name path=F,draw=none]  (axis cs:2,6) arc (0:180:200 and 6.67);
\addplot [blue,opacity=0.25] fill between[of=E and F];

\draw[decoration={brace,raise=.1cm},decorate,thin,penColor2] (axis cs:1,3) -- (axis cs:1,0);
\node[anchor=west,penColor2] at (axis cs:1.2,1.5) {$y$};
\node[anchor=west,penColor2] at (axis cs:.3,3.8) {$x$};
\node[anchor=west] at (axis cs:1,9.5) {$3$m};
\draw[decoration={brace,raise=.1cm},decorate,thin] (axis cs:3,9) -- (axis cs:3,0);
\node[anchor=west] at (axis cs:3.2,4.5) {$9$m};

%horizontal lines
\addplot[thick,penColor2] plot coordinates {(0,3) (1,3)};
\addplot[thick] plot coordinates {(0,9) (3,9)};

         
\end{axis}
\end{tikzpicture}
\end{image}

From similar triangles, we find $\frac{x}{y} = \answer{\frac{3}{9}}$, so $x= \answer{\frac{1}{3}y}$.

\begin{question}
The area is $A = \pi r^2$.  We see that $r$ is a horizontal distance, which we must express in terms of $y$.  Hence, $r=  \answer{\frac{1}{3} y}$ and $A=\answer{\pi \left(\frac{y}{3}\right)^2}$.

Now, substitute all of these relevant quantities into the integral.
\end{question}
\end{question}

\end{hint}

\end{exercise}
\end{document}
