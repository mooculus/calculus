\documentclass{ximera}

\newcommand{\RR}{\mathbb R}
\renewcommand{\d}{\,d}
\newcommand{\dd}[2][]{\frac{d #1}{d #2}}
\renewcommand{\l}{\ell}
\newcommand{\ddx}{\frac{d}{dx}}
\newcommand{\dfn}{\textbf}
\newcommand{\eval}[1]{\bigg[ #1 \bigg]}


\author{Jim Talamo}
\license{Creative Commons 3.0 By-bC}


\outcome{}


\begin{document}
\begin{exercise}

The following explores the relationship between a sequence and its sequence of partial sums.  For each question, let $\{a_n\}$ be a sequence and let $s_n = \sum_{k=1}^n a_k$.  Note that the \emph{logic} here is important!

Suppose that it is known that $\sum_{k=1}^{\infty} a_k = 4$.  

Since $\sum_{k=1}^{\infty} a_k=4$ converges, we know that $\lim_{n \to \infty} s_n = \answer{4}$.  If the limit of a \emph{sequence} is not zero, a certain test allows us to conclude whether the \emph{series} (obtained by adding all of the terms in the sequence) converges or diverges! 

\begin{exercise}
Thus:
\begin{multipleChoice}
\choice{$\sum_{k=1}^{\infty} s_k$ converges to $4$.}
\choice{$\sum_{k=1}^{\infty} s_k$ converges to $0$.}
\choice{$\sum_{k=1}^{\infty} s_k$ converges but more information is needed to determine its value.}
\choice{$\sum_{k=1}^{\infty} s_k$ could converge or diverge.}
\choice[correct]{$\sum_{k=1}^{\infty} s_k$ diverges.}
\end{multipleChoice}
\end{exercise}

\begin{exercise}
Suppose that it is known that $\sum_{k=1}^{\infty} s_k$ diverges.  Consider the following examples:

\begin{exercise}
Suppose that $s_n = 1$ for all $n$.  Then:
\begin{multipleChoice}
\choice{$\sum_{k=1}^{\infty} s_k$ converges to $1$.}
\choice[correct]{$\sum_{k=1}^{\infty} s_k$ diverges by the divergence test.}
\end{multipleChoice}
However, in this case $\lim_{n \to \infty} s_n = \answer{1}$, so:
\begin{multipleChoice}
\choice[correct]{$\sum_{k=1}^{\infty} a_k$ converges to $1$.}
\choice{$\sum_{k=1}^{\infty} a_k$ converges but more information is needed to determine its value.}
\choice{$\sum_{k=1}^{\infty} a_k$ could converge or diverge.}
\choice{$\sum_{k=1}^{\infty} a_k$ diverges.}
\end{multipleChoice} 

\begin{exercise}
Suppose that $s_n = n$ for all $n$.  Then:
\begin{multipleChoice}
\choice{$\sum_{k=1}^{\infty} s_k$ converges to $1$.}
\choice[correct]{$\sum_{k=1}^{\infty} s_k$ diverges by the divergence test.}
\end{multipleChoice}
However, in this case:

\[
\lim_{n \to \infty} s_n = \answer{\infty}
\]
(Use $\infty$ or $-\infty$ where appropriate and type ``DNE" if the limit otherwise does not exist).

Thus,
\begin{multipleChoice}
\choice{$\sum_{k=1}^{\infty} a_k$ converges to $1$.}
\choice{$\sum_{k=1}^{\infty} a_k$ converges but more information is needed to determine its value.}
\choice{$\sum_{k=1}^{\infty} a_k$ could converge or diverge.}
\choice[correct]{$\sum_{k=1}^{\infty} a_k$ diverges.}
\end{multipleChoice} 

\end{exercise}


\end{exercise}
Thus, if we are only given that $\sum_{k=1}^{\infty} s_k$ diverges:
\begin{multipleChoice}
\choice{$\sum_{k=1}^{\infty} a_k$ converges to $0$.}
\choice{$\sum_{k=1}^{\infty} a_k$ converges but more information is needed to determine its value.}
\choice[correct]{$\sum_{k=1}^{\infty} a_k$ could converge or diverge.}
\choice{$\sum_{k=1}^{\infty} a_k$ diverges.}
\end{multipleChoice}
\end{exercise}




\end{exercise}
\end{document}
