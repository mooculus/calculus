\documentclass{ximera}

\newcommand{\RR}{\mathbb R}
\renewcommand{\d}{\,d}
\newcommand{\dd}[2][]{\frac{d #1}{d #2}}
\renewcommand{\l}{\ell}
\newcommand{\ddx}{\frac{d}{dx}}
\newcommand{\dfn}{\textbf}
\newcommand{\eval}[1]{\bigg[ #1 \bigg]}


\author{Bart Snapp}

\outcome{Give the equation for a sphere or ball.}

\title[Dig-In:]{Drawing a sphere}

\begin{document}
\begin{abstract}
  Learn how to draw a sphere.
\end{abstract}
\maketitle

A key challenge in mathematics is converting formulas and equations
into ideas. We want to get to the point that when you see something
like
\[
S = \{(x,y,z):(x-1)^2+(y-2)^2+(z-3)^2=16\}
\]
you say to yourself, ``Hey, that's a sphere of radius $4$ centered at
the point $(1,2,3)$. Let's state this generally.
\begin{theorem}
  The set of points on the outside of a sphere of radius $r$
  centered at $(a,b,c)$ are given by the set:
  \[
  S = \{(x,y,z):(x-a)^2+(y-b)^2+(z-c)^2=r^2\}
  \]
  The set of points forming the filled-in ball of radius $r$ centered
  at $(a,b,c)$ are given by the set:
  \[
  B = \{(x,y,z):(x-a)^2+(y-b)^2+(z-c)^2\le r^2\}
  \]
\end{theorem}
Above we give implicit formulas. Sometimes it is better to have
parametric formulas.
\begin{theorem}
  The parametric formula for a sphere of radius $r$ centered at
  $(a,b,c)$ is give by:
  \begin{align*}
    x(\phi,\theta) &=r\cdot\cos(\theta)\sin(\phi)\\
    y(\phi,\theta) &=r\cdot\sin(\theta)\sin(\phi)\\
    z(\phi,\theta) &=r\cdot\cos(\phi)
  \end{align*}
  for $0\le \phi\le \pi$ and $0\le\theta< 2\pi$.

  If you want a parametric formula for the filled-in ball of radius
  $r$, you write:
  \begin{align*}
    x(\rho,\phi,\theta) &=\rho\cdot\cos(\theta)\sin(\phi)\\
    y(\rho,\phi,\theta) &=\rho\cdot\sin(\theta)\sin(\phi)\\
    z(\rho,\phi,\theta) &=\rho\cdot\cos(\phi)
  \end{align*}
  where  $0\le \phi\le \pi$, $0\le\theta< 2\pi$, and $0\le \rho\le r$.
\end{theorem}
  


\end{document}
