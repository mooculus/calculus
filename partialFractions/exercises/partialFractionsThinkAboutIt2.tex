\documentclass{ximera}

\newcommand{\RR}{\mathbb R}
\renewcommand{\d}{\,d}
\newcommand{\dd}[2][]{\frac{d #1}{d #2}}
\renewcommand{\l}{\ell}
\newcommand{\ddx}{\frac{d}{dx}}
\newcommand{\dfn}{\textbf}
\newcommand{\eval}[1]{\bigg[ #1 \bigg]}


\author{Jim Talamo}
\license{Creative Commons 3.0 By-NC}


\outcome{Think about Partial Fraction Decomposition}
\outcome{Explore why we take the convention of treating repeated factors the way we do}

\begin{document}
\begin{exercise}
Consider the indefinite integral: 

\[
\int \frac{3x^2-8x+1}{x^3-2x^2+x} \d x.
\]

If we want to integrate this, partial fraction decomposition would be a good technique to try.  Factoring gives:

\[
\frac{3x^2-8x+1}{x^3-2x^2+x}  =\frac{3x^2-8x+1}{x(x^2-2x+1)} =\frac{3x^2-8x+1}{x(x-1)^2} 
\]  

The partial fraction decomposition for this expression is:

\[
\frac{3x^2-8x+1}{x^3-2x^2+x}= \frac{\answer{A}}{x}+\answer{\frac{B}{x-1}+\frac{C}{(x-1)^2}}
\]
(Use $A$, $B$, and $C$ for the constants and list the terms in the order written in the factored form and write the higher powers of repeated terms first)

Solving for these constants, we find: $A = \answer{1}$, $B=\answer{2}$, and $C=\answer{-4}$ and we conclude:

\[
\frac{3x^2-8x+1}{x^3-2x^2+x}= \answer{\frac{1}{x}+\frac{2}{x-1}+\frac{-4}{(x-1)^2}}
\]

Thus, we conclude:
\[
\int \frac{3x^2-8x+1}{x^3-2x^2+x} \d x = \int \frac{1}{x}+\frac{2}{x-1}-\frac{4}{(x-1)^2} \d x = \answer{\ln|x|+2\ln|x-1|+\frac{4}{x-1} +C}
\]
(Use $C$ for the constant of integration)


\begin{exercise}
Suppose that instead of realizing that $x^2-2x+1$ is a reducible quadratic, we treated it as an irreducible quadratic and wrote:


\[
\frac{3x^2-8x+1}{x^3-2x^2+x}  =\frac{3x^2-8x+1}{x(x^2-2x+1)} =\frac{A}{x}+\frac{Bx+C}{x^2-2x+1} 
\]  

and then proceeded to find the constants.

Multiplying both sides by $x(x^2-2x+1)$ gives:

\[
3x^2-8x+1 = A(\answer{x^2-2x+1})+(Bx+C)\answer{x}
\]  

\begin{exercise}
Setting $x=0$ gives $A=\answer{1}$.

We can then substitute in $A=1$, expand both sides, and collect like powers of $x$ to obtain:

\[
3x^2-8x+1 = \left(\answer{1+B}\right)x^2+\left(\answer{C-2}\right)x+\answer{1}
\]
(You should carry out all of the steps mentioned and write the result above)

We find that $B=\answer{2}$ and $C=\answer{-6}$

\begin{exercise}
Using the above results, we find:

 \[
 \int \frac{3x^2-8x+1}{x^3-2x^2+x} \d x =\int \frac{1}{x}+\frac{2x-6}{x^2-2x+1} \d x
\]
The question remains how to compute this last integral.  It cannot be done by a simple $u$-substitution, but we can notice that the derivative of the denominator is:

\[ \ddx \left[x^2-2x+1\right] = \answer{2x-2}
\]
and we can try to find a clever way to split the fraction by writing:

\[
 \int \frac{3x^2-8x+1}{x^3-2x^2+x} \d x =\int \frac{1}{x}+\frac{2x-2-4}{x^2-2x+1} \d x=\int \frac{1}{x}+\frac{2x-2}{x^2-2x+1} -\frac{4}{x^2-2x+1} \d x
\]

Now, by noting that $x^2-2x+1 = (\answer{x-1})^2$, we may write:

\[
 \int \frac{3x^2-8x+1}{x^3-2x^2+x} \d x =\int \frac{1}{x}+\frac{2x-2}{x^2-2x+1} -\frac{4}{x^2-2x+1} \d x =\int \frac{1}{x}+\frac{2(x-1)}{(x-1)^2} -\frac{4}{(x-1)^2} \d x
\]

Simplifying this gives:

\[
 \int \frac{3x^2-8x+1}{x^3-2x^2+x} \d x = \int \frac{1}{x}+\answer{\frac{2}{(x-1)}} -\frac{4}{(x-1)^2} \d x
\]

\begin{feedback}
You should notice that this is exactly the same integral we computed earlier!  By choosing to treat $x^2-2x+1$ as an irreducible quadratic in the partial fraction decomposition, we obtained an algebraically equivalent form, but it required additional work to write the expression in a way that we can use to compute the antiderivative!

The process used to determine the partial fraction decomposition accomplishes two tasks: (1) to give an expression that, when  added together, is equivalent to the original rational function and (2) give this expression in a form that is most convenient to integrate.
\end{feedback}

\end{exercise}
\end{exercise}
\end{exercise}
\end{exercise}
\end{document}
