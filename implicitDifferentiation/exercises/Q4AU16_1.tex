\documentclass{ximera}

\newcommand{\RR}{\mathbb R}
\renewcommand{\d}{\,d}
\newcommand{\dd}[2][]{\frac{d #1}{d #2}}
\renewcommand{\l}{\ell}
\newcommand{\ddx}{\frac{d}{dx}}
\newcommand{\dfn}{\textbf}
\newcommand{\eval}[1]{\bigg[ #1 \bigg]}


\begin{document}

\begin{exercise}

\outcome{Implicitly differentiate expressions.}
\outcome{Find the equation of the tangent line for curves that are not plots of functions.}

The curve defined by 
\[
x^2+xy+y^2=7
\]
is called a ``rotated ellipse.'' This is not a misnomer, as its graph below demonstrates.
\begin{image}
  \begin{tikzpicture}
    \begin{axis}[
            domain=-4:4, xmin =-4,xmax=4,ymax=4,ymin=-4,
            samples=1000,
            width=4in,
            height=4in,
            xtick={-4,-3,...,4},
            ytick={-4,-3,...,4},
            axis lines=center, xlabel=$x$, ylabel=$y$,
            every axis y label/.style={at=(current axis.above origin),anchor=south},
            every axis x label/.style={at=(current axis.right of origin),anchor=west},
            axis on top,
      ]
      \addplot [ultra thick,penColor,smooth,domain=0:2*pi] ({-sqrt(7)*cos(deg(x))+sqrt(7/3)*sin(deg(x))},{sqrt(7)*cos(deg(x))+sqrt(7/3)*sin(deg(x))});
    \end{axis}
  \end{tikzpicture}
\end{image}
Use implicit differentiation to find the derivative $\frac{dy}{dx}$.
\[
\frac{dy}{dx}=\answer{-\frac{2x+y}{2y+x}}
\]
\begin{exercise}
Using your answer, find the function $\ell$ whose graph is the line tangent to the curve at the point $(1,2)$.
\[
\ell(x)=\answer{-\frac{4}{5}(x-1)+2}
\]
\end{exercise}
\end{exercise}
\end{document}
