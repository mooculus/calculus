\documentclass{ximera}

\begin{document}
\begin{abstract}
\end{abstract}

\section{Understanding functions}

\begin{problem}
  Suppose $f(x) = x^2$.  How does $f(f(x))$ compare to $f(x)$?
  \begin{multipleChoice}
    \choice{Whenever $x$ is close to one, $f(f(x))$ is close to zero.}
    \choice{Whenever $x$ is close to one, $f(f(x))$ is larger than $x$.}
    \choice{Whenever $x$ is close to zero, $f(f(x))$ is larger than $x$.}
    \choice[correct]{Whenever $x$ is large, $f(f(x))$ is larger than $x$.}
  \end{multipleChoice}
\end{problem}

\section{Review of famous functions}

\begin{problem}
   Which of the following statements is true?
   \begin{multipleChoice}
     \choice{$\sin^{-1}(x)$ is the inverse function of $\sin(x)$}
     \choice[correct]{$\sin\left(\sin^{-1}\left(\frac{1}{2}\right)\right)
       = \frac{1}{2}$} 
     \choice{$\sin^{-1}\left(\sin\left(\frac{5\pi}{2}\right)\right) = \frac{5\pi}{2}$}
     
     \choice{$\sin^{-1}(x) = \frac{1}{\sin(x)}$}
   \end{multipleChoice}  
\end{problem}

\begin{problem}
  The expression $\log_b(x) = y$ is equivalent to:
  \begin{multipleChoice}
    \choice{$b^x = y$}
    \choice[correct]{$b^y = x$}
    \choice{$x^b = y$}
    \choice{$x^y = b$}
    \choice{$y^b = x$}
    \choice{$y^x = b$}
  \end{multipleChoice}  
\end{problem}

\begin{problem}
  Suppose $f(x) = x \sin^2 x$.  What is true about $f(x)$?
  \begin{multipleChoice}
    \choice[correct]{When $x$ is close to zero, $f(x)$ is close to zero.}
    \choice{When $x$ is very large, $f(x)$ is very large.}
    \choice{When $x$ is very negative, $f(x)$ is very negative.}
    \choice{When $x$ is close to one, $f(x)$ is close to one.}
  \end{multipleChoice}
\end{problem}



\section{What is a limit?}

\begin{problem}
  Suppose $x$ is a positive number close to zero, and $y$ is a very large number.  What can be said about $y/x$?
  \begin{multipleChoice}
    \choice[correct]{It is very large.}
    \choice{It is close to zero.}
    \choice{It is very negative.}
    \choice{It could be very positive or very negative.}
  \end{multipleChoice}
\end{problem}

\begin{problem}
  Suppose $x$ is a number close to zero, and $y$ is a very large number.  What can be said about $y/x$?
  \begin{multipleChoice}
    \choice{It is very large.}
    \choice{It is close to zero.}
    \choice{It is very negative.}
    \choice[correct]{It could be very positive or very negative.}
  \end{multipleChoice}
\end{problem}

\begin{problem}
  Suppose $x$ and $y$ are numbers close to zero.  What can be said about $x+y$?
  \begin{multipleChoice}
    \choice[correct]{It is close to zero.}
    \choice{It is negative.}
    \choice{It is positive.}
    \choice{It is larger than $x$.}
  \end{multipleChoice}
\end{problem}

\section{Limit laws}

\section{Indeterminate forms}
\section{Using limits to detect asymptotes}

\begin{problem}
  Suppose whenever $x$ is very large, $f(x)$ is close to 1.  What is true about $f$?
  \begin{multipleChoice}
    \choice{The graph of $f$ does not cross the line $y = 1$.}
    \choice{The graph of $f$ does not cross the line $x = 1$.}
    \choice{Whenever $x$ is close to zero, $f(1/x)$ is close to 1.}
    \choice[correct]{Whenever $x$ is very large, $f(x^2)$ is close to 1.}
  \end{multipleChoice}
\end{problem}

\section{Continuity and the Intermediate Value Theorem}

\begin{problem}
  Suppose $f$ is a continuous function so that whenever $0 \leq x \leq 1$ we have $0 \leq f(x) \leq 1$.  What can be about $f$?
  \begin{multipleChoice}
    \choice[correct]{There is an $x$ so that $f(x) = x$.}
    \choice{There is an $x$ so that $f(x) > 1$.}
    \choice{There is an $x$ so that $f(x) > x$.}
    \choice{There is an $x$ so that $f(x) < 0$.}
    \choice{There is an $x$ so that $f(x) < x$.}
  \end{multipleChoice}
\end{problem}

\section{An Application of limits}
\section{Definition of the derivative}

\begin{problem}
  Let $x = \log_b 1001 - \log_b 1000$ and $y = \log_b 101 - \log_b 100$.  How do $x$ and $y$ compare?
  \begin{multipleChoice}
    \choice[correct]{$x < y$}
    \choice{$x > y$}
    \choice{$x = y$}
    \choice{It depends on the base $b$.}
  \end{multipleChoice}
\end{problem}

\section{The derivative as a function}



\section{Higher order derivatives and graphs}

\begin{problem}
Suppose $f$ and $g$ have the same second derivative at every point.  
\begin{multipleChoice}
\choice[correct]{It is possible that $f$ is increasing at $1$ and $g$ is decreasing at $1$.}
\choice{Either $f$ and $g$ are both increasing or both decreasing at $1$.}
\end{multipleChoice}
\end{problem}

\section{Rules of differentiation}

\begin{problem}
  Suppose $p(x) = x^4 + x^3 - x^2 - x + 1$ and $f(x)$ is some differentiable function.
\begin{multipleChoice}
\choice{$g'(x) = e^{(x e^x)}$ by the chain rule}
\choice[correct]{$g'(x) = e^{(x + e^x)}$ by the chain rule}
\choice{$g'(x) = e^{(e^x)}$ by the chain rule}
\end{multipleChoice}
\end{problem}

\section{The product and quotient rules}

\section{The chain rule}

\begin{problem}
Suppose $f(x) = e^x$ so that $f'(x) = f(x)$.  Let $g(x) = e^{f(x)}$.  What is $g'(x)$?
\begin{multipleChoice}
\choice{$g'(x) = e^{(x e^x)}$ by the chain rule}
\choice[correct]{$g'(x) = e^{(x + e^x)}$ by the chain rule}
\choice{$g'(x) = e^{(e^x)}$ by the chain rule}
\end{multipleChoice}
\end{problem}

\section{Mean Value Theorem}

\begin{problem}
  Suppose $f$ is an increasing function, and $x$ and $h$ are positive numbers between $0$ and $1$.  How does $f(x + h)$ compare to $f(x + h^2)$?
  \begin{multipleChoice}
    \choice[correct]{$f(x+h) > f(f+h^2)$}
    \choice{$f(x+h) < f(f+h^2)$}
    \choice{It depends on the function $f$.}
  \end{multipleChoice}
\end{problem}

\begin{problem}
  Suppose $f$ and $g$ are increasing functions.  What can be said about the function $h(x) = f(g(x))$?
  \begin{multipleChoice}
    \choice[correct]{It is an increasing function.}
    \choice{It is a decreasing function.}
    \choice{It depends on exactly which functions $f$ and $g$ are being considered.}
  \end{multipleChoice}
\end{problem}


\begin{problem}
  Suppose $f$ and $g$ are decreasing functions.  What can be said about the function $h(x) = f(g(x))$?
  \begin{multipleChoice}
    \choice[correct]{It is an increasing function.}
    \choice{It is a decreasing function.}
    \choice{It depends on exactly which functions $f$ and $g$ are being considered.}
  \end{multipleChoice}
\end{problem}


\section{Linear approximation}

\begin{problem}
  Suppose $y = mx + b$ is the equation for a tangent line at the point $(100,\log 100)$ to the graph of $f(x) = \log x$.  How does $200m + b$ compare to $\log 200$?
  \begin{multipleChoice}
    \choice[correct]{$200m + b > \log 200$}
    \choice{$200m + b > \log 200$}
    \choice{$200m + b = \log 200$}
  \end{multipleChoice}
\end{problem}

\section{Maximums and minimums}

\begin{problem}
  Suppose a smooth function $f$ has infinitely many local minima.  What is true of $f$?
  \begin{multipleChoice}
    \choice[correct]{$f$ has infinitely many local maxima.}
    \choice{$f(x)$ can be made as large as desired by choosing appropriate $x$}
    \choice{$f(x)$ can be made as small as desired by choosing appropriate $x$}
    \choice{None of these statements are true.}
  \end{multipleChoice}
\end{problem}

\begin{problem}
  Suppose a function $f$ has no global maximum value.  What is true of $f$?
  \begin{multipleChoice}
    \choice{$f$ has infinitely many local minima.}
    \choice{$f(x)$ can be made as large as desired by choosing appropriate $x$}
    \choice{$f(x)$ can be made as small as desired by choosing appropriate $x$}
    \choice[correct]{None of these statements are true.}
  \end{multipleChoice}
\end{problem}

\section{Optimization}

\section{Applied optimization}

\section{Derivatives of trigonometric functions}

\begin{problem}
  Suppose a function $f$ has the property that $f'' = -f$.  What is true of $f$?
  \begin{multipleChoice}
    \choice{$f(x) = \cos x$}
    \choice[correct]{$f^{(4)}(x) = f(x)$}
    \choice{$f(x) = \tan x$}
    \choice{$f^{(4)}(x) = -f(x)$}
  \end{multipleChoice}
\end{problem}

\section{Implicit differentiation}

\begin{problem}
  Suppose $f$ is an increasing function and $f(x) \cdot g(x)$ is a constant.  What must be true of $g$?
  \begin{multipleChoice}
    \choice{$g$ must be increasing}
    \choice[correct]{$g$ could be constant}
    \choice{$g$ must be decreasing}
  \end{multipleChoice}
\end{problem}

\begin{problem}
  Suppose $f(x)^2 + g(x)^2 = 10$ and suppose $f'(a) > 0$.  What must be true about $g'(a)$?
  \begin{multipleChoice}
    \choice{$g'(a) > 0$}
    \choice{$g'(a) = 0$}
    \choice[correct]{$g'(a) < 0$}
  \end{multipleChoice}
\end{problem}

\section{Derivatives of inverse functions}

\begin{problem}
  There is a function $f$ so that $f(0) = 0$ and when $h$ is small, $f^{-1}(h)$ is close to $10 h$.  What is a good guess for $f'(0)$?
  \begin{multipleChoice}
    \choice{$10$}
    \choice{$-10$}
    \choice[correct]{$1/10$}
    \choice{$-1/10$}
  \end{multipleChoice}
\end{problem}

\begin{problem}
  Suppose $f$ is a function and $f'(0) = 0$.  What is true about $f$?
  \begin{multipleChoice}
    \choice{$f$ can not be an invertible function}
    \choice[correct]{$f$ could be an invertible function}
  \end{multipleChoice}
\end{problem}

\section{Logarithmic differentiation}

\begin{problem}
  Consider the continuous function $f$ and its logarithmic derivative $g(x) = \frac{f'(x)}{f(x)}$.  Suppose $g(x) = 0.05$ when $0 < x < 10$ and $g(x) = -0.05$ when $10 < x < 20$.  How do $f(0)$ and $f(20)$ compare?
  \begin{multipleChoice}
    \choice[correct]{$f(20) > f(0)$}
    \choice{$f(0) < f(20)$}
    \choice{$f(0) = f(20)$}
  \end{multipleChoice}
\end{problem}

\section{Advanced graphing of functions}

\section{More than one rate}

\section{Applied related rates}

\begin{problem}
  A weight is dropped from the top of a building.  Ignoring air
  resistance, when the weight is halfway between the top of the
  building and the ground, it is travelling at $v$ m/s.  Just before
  the weight hits the ground, how fast is the weight moving?
  \begin{multipleChoice}
    \choice[correct]{More than $2v$ m/s}
    \choice{Exactly $2v$ m/s}
    \choice{Less than $2v$ m/s}
  \end{multipleChoice}
\end{problem}

\section{Differential equations}

\begin{problem}
  Suppose $f''(x) = f'(x) + f(x)$.  If $f'(1) = 0$ and $f(1) = 2$, what is true about $f$?
  \begin{multipleChoice}
    \choice[correct]{$f$ has a local minimum at the point $1$}
    \choice{$f$ has a local maximum at the point $1$}
    \choice{$f$ has an inflection point at $1$}
  \end{multipleChoice}
\end{problem}

\begin{problem}
  Suppose the slope of a tangent line to the graph of $f$ at the point $x$ is $x \cdot f(x)$ and suppose $f(0) = 1$.  What must be true of $f$?
  \begin{multipleChoice}
    \choice[correct]{By choosing $x$ large enough, $f(x)$ can be made as large as desired}
    \choice{By choosing $x$ large enough, $f(x)$ can be made as close to zero as desired}
    \choice{By choosing $x$ large enough, $f(x)$ can be made as negative as desired}
  \end{multipleChoice}
\end{problem}

\section{Antiderivatives}

\section{L'Hopital's rule}


\section{Approximating the area under a curve}

\begin{problem}
  Consider a rectangle $R$ which is $2 \times 1$ and a square $S$
  which is $\sqrt{2} \times \sqrt{2}$.  What is true about $R$ and
  $S$?
  \begin{multipleChoice}
    \choice[correct]{The square $S$ can be cut up into pieces and rearranged to form $R$.}
    \choice{Because $\sqrt{2}$ is irrational, it is not possible to cut $S$ into pieces to form $R$.}
    \choice{Because integration involves a limit, it is not possible to cut $S$ into pieces to form $R$.}
  \end{multipleChoice}
\end{problem}

\begin{problem}
  Suppose $\int_0^2 f(x) \, dx$ is positive.  What can we say about $f(1)$?
  \begin{multipleChoice}
    \choice{$f(1)$ must be positive.}
    \choice{$f(1)$ could be positive or zero.}
    \choice[correct]{$f(1)$ could be positive, negative, or zero.}
  \end{multipleChoice}
\end{problem}

\section{Net area}

\begin{problem}
  Suppose $f$ and $g$ are continuous functions and $f(x) = |g(x)|$.  What can we say about $\int_0^1 f(x) \, dx$?
  \begin{multipleChoice}
    \choice{$\int_0^1 f(x) \, dx$ must be positive.}
    \choice[correct]{$\int_0^1 f(x) \, dx$ could be positive or zero.}
    \choice{$\int_0^1 f(x) \, dx$ could be positive, negative, or zero.}
  \end{multipleChoice}
\end{problem}

\section{The definite integral}

\section{First Fundamental Theorem of Calculus}

\section{Second Fundamental Theorem of Calculus}

\section{The idea of substitution}

\section{Working with substitution}

\section{Applications of integrals}



\end{document}

%%% Local Variables:
%%% mode: latex
%%% TeX-master: t
%%% End:
