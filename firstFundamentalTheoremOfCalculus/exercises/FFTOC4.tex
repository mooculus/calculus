\documentclass{ximera}

\newcommand{\RR}{\mathbb R}
\renewcommand{\d}{\,d}
\newcommand{\dd}[2][]{\frac{d #1}{d #2}}
\renewcommand{\l}{\ell}
\newcommand{\ddx}{\frac{d}{dx}}
\newcommand{\dfn}{\textbf}
\newcommand{\eval}[1]{\bigg[ #1 \bigg]}


\outcome{Define accumulation functions.}
\outcome{Calculate and evaluate accumulation functions.}
%\outcome{State the First Fundamental Theorem of Calculus.}
\outcome{Take derivatives of accumulation functions using the First Fundamental Theorem of Calculus.}
%\outcome{Use accumulation functions to find information about the original function.}
\outcome{Understand the relationship between the function and the derivative of its accumulation function.}

\author{Nela Lakos \and Kyle Parsons}

\begin{document}
\begin{exercise}

The graph of $f$ is shown below.

\begin{image}
  \begin{tikzpicture}
    \begin{axis}[
        xmin=-0.3,xmax=6.3,ymin=-1.3,ymax=3.3,
        clip=true,
        unit vector ratio*=1 1 1,
        axis lines=center,
        grid = major,
        ytick={-1,0,...,36},
        xtick={0,1,...,6},
        xlabel=$t$, ylabel=$y$,
        every axis y label/.style={at=(current axis.above origin),anchor=south},
        every axis x label/.style={at=(current axis.right of origin),anchor=west},
      ]
      \addplot[ultra thick,penColor,domain=0:1] {-x};
      \addplot[ultra thick,penColor,domain=1:2] {-1};
      \addplot[ultra thick,penColor,domain=2:6] {x-3};    
        
      \node at (axis cs:1.5,2.5) {$y=f(t)$};
      \end{axis}`
  \end{tikzpicture}
\end{image}

Let
\[
A(x) = \int_0^x f(t) \d t \text{ for } 0\leq x\leq6 \text{ and } B(x) = \int_3^x f(t) \d t \text{ for } 3\leq x\leq6.
\]

Evaluate the following expressions.
\begin{align*}
A(3) &= \answer{-2}\\
\end{align*}
\begin{hint}
Recall: $A(3)= \int_0^3 f(t) \d t $. This definite integral gives  signed area of the region between the curve $y=f(x)$ and the interval $[0,3]$ on the $x-$axis. Use geometry!
\end{hint}

\begin{align*}
B(3) &= \answer{0}\\
\end{align*}
\begin{hint}
Recall: $B(3)= \int_3^3 f(t) \d t $.
\end{hint}
\begin{align*}
B'(1.5) &= \answer{-1}
\end{align*}
\begin{hint}
Recall: $A'(x)=f(x)$
\end{hint}
\begin{hint}
Recall: $A'(1.5)=f(1.5)$. Determine the value $f(1.5)$ from the graph.
\end{hint}
For $3\leq x\leq6$, we can express $B'(x)$ as
\[
B'(x) = \answer{x-3}.
\]
\begin{hint}
Recall: $B'(x)=f(x)$ for  $3\leq x\leq6$.
\end{hint}
\begin{hint}
Recall: $B'(x)=f(x)$ for  $3\leq x\leq6$. $f$ is a linear function on the interval $[3,6]$.
\end{hint}
For $3\leq x\leq6$, we can experss $B(x)$ as
\[
B(x) = \answer{\frac{(x-3)^2}{2}}.
\]
\begin{hint}
Recall: $B(x)= \int_3^x f(t) \d t$ for  $3\leq x\leq6$.  This definite integral gives signed area of the region between the graph of $f$ and the interval $[3,x]$ on the $x-$axis.
This region is a triangle with the base $(x-3)$ and the height $f(x)$. Compute the area of the triangle!
\end{hint}
\[
\int_0^5\left|f(t)\right| \d t = \answer{4}
\]
\begin{hint}
This integral gives the area of the region between the graph of $|f|$ and the interval $[0,5]$ on the $x-$axis. Use geometry! 
It can be expressed as $-A(3)+B(5)$.
\end{hint}
\end{exercise}
\end{document}
