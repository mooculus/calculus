\documentclass{ximera}

\newcommand{\RR}{\mathbb R}
\renewcommand{\d}{\,d}
\newcommand{\dd}[2][]{\frac{d #1}{d #2}}
\renewcommand{\l}{\ell}
\newcommand{\ddx}{\frac{d}{dx}}
\newcommand{\dfn}{\textbf}
\newcommand{\eval}[1]{\bigg[ #1 \bigg]}


\author{Jim Talamo}
\license{Creative Commons 3.0 By-bC}

\outcome{Answer conceptual questions about the radius an interval of convergence}

\begin{document}
\begin{exercise}
Suppose that $f(x) = \sum_{k=0}^{\infty} a_k (x-3)^k$ and it is known that:

\begin{itemize}
\item $\sum_{k=0}^{\infty} a_k$ converges.
\item The series represented by $f(-5)$ diverges.
\end{itemize}

There are many questions that can be asked that require a conceptual understanding of the radius and interval of convergence.  

We first use the given information to determine where the series must converge and must diverge.

%%%%%%%%%%%%%%%%%%%%%%%%%%%%%
\begin{exercise}
Let's explore what the first condition tells us.  First, note that to relate the given information to the function $f(x)$, we need to find an $x$-value for which the function equals $\sum_{k=0}^{\infty} a_k$.  Indeed, we find $f(\answer{4}) = \sum_{k=0}^{\infty} a_k$.  

\begin{exercise}
The series represented by $f(x)$ when $x=4$ converges.  Since the center of this series is at $x=\answer{3}$, $x=4$ is $\answer{1}$ unit from the center of the series.  Thus, the \emph{minimum} possible radius of convergence is $\answer{1}$.
\end{exercise}
\end{exercise}

%%%%%%%%%%%%%%%%%%%%%%%%%%%%%

\begin{exercise}
Let's explore what the second condition tells us.  The series represented by $f(x)$ when $x=-5$ diverges.  Since the center of this series is at $x=\answer{3}$ and $x=-5$ is $\answer{8}$ units from the center of the series, the \emph{maximum} possible radius of convergence is $\answer{8}$.
\end{exercise}

%%%%%%%%%%%%%%%%%%%%%%%%%%%%%
\begin{exercise}
Now, we can sketch the minimum and maximum possible interval of convergence, and use it to answer answer many questions.  We start by indicating on a number line where the series must converge, must diverge, and where more information is needed:

%%%%%%%%%BEGIN IMAGE%%%%%%%%%%%  
\begin{image}
\begin{tikzpicture}

\begin{axis}
	[
	domain=-4:10,
	axis x line = middle,
	axis line style=-,
	axis y line = none,
	xmin=-5,
	xmax=11,
	ymin=-1,
	ymax=1,
	xtick={-1.5},
	xticklabels={},
	]
%%%%number line%%%%	
	\draw[very thick,penColor] (2,0) -- (4,0);
	\draw[->,very thick,penColor2] (-1.5,0) -- (-5,0);
	\draw[->,very thick,penColor2] (7.5,0) -- (11,0);
	\draw[very thick,penColor4] (-1.5,0) -- (2,0);
	\draw[very thick,penColor4] (4,0) -- (7.5,0);

%Vertical LInes%%%
	\draw[thick,black,dashed] (-1.5,-.25) -- (-1.5,.25);
	\draw[thick,black,dashed] (2,-.25) -- (2,.25);
	\draw[thick,black,dashed] (4,-.25) -- (4,.25);
	\draw[thick,black,dashed] (7.5,-.25) -- (7.5,.25);

%%%%Center%%%%%	
	\node at (3,-.05) [below, penColor] {$3$};
	\draw[very thick,penColor] (3,-.05) -- (3,.05);
	
	
%%%%CONVERGENCE: arrows and minimum IOC
	\draw[->,thick,black] (3.1,.1) -- (4,.1);
	\draw[->,thick,black] (2.9,.1) -- (2,.1);
      	\node[black] at (3.55,.17) {\small{$1$}};
        \node[black] at (2.55,.17) {\small{$1$}};
       	%%%%%%brace and text
        \draw [penColor,thick,decoration={brace,mirror,raise=2em},decorate] 
        (axis cs:2,0) --
        (axis cs:4,0);
        \node[penColor] at (3,-.41) {converges};
	%%%Endpoints
	\node at (2,0) [scale=.5,shape=circle, draw=penColor4, fill=white] {};
	\node at (1.7,-.05) [below, black] {$2$};
	\node at (4,0) [scale=.5,shape=circle, fill=penColor, draw=penColor] {};
	\node at (4.3,-.05) [below, black] {$4$};    
	
%%%%DIVERGENCE: arrows and maximum IOC
	\draw[->,thick,black] (3.1,.05) -- (7.5,.05);
	\draw[->,thick,black] (2.9,.05) -- (-1.5,.05);
      	\node[black] at (5.35,.12) {\small{$8$}};
        \node[black] at (.75,.12) {\small{$8$}};
        %%%%%%brace and text
    	\draw [penColor2,thick,decoration={brace,mirror,raise=2em},decorate] 
        (axis cs:7.5,0) --
        (axis cs:11,0);
    	\node[penColor2] at (9.5,-.4) {diverges};
    	\draw [penColor2,thick,decoration={brace,raise=2em},decorate] 
        (axis cs:-1.5,0) --
        (axis cs:-5,0);
	\node[penColor2] at (-3.5,-.4) {diverges};
	%%%Endpoints
	\node at (-1.5,0) [scale=.5,shape=circle, draw=penColor2, fill=penColor2] {};
	\node at (-2.2,-.05) [below, black] {$-5$};
	\node at (7.5,0) [scale=.5,shape=circle, fill=white, draw=penColor4] {};
	\node at (8,-.05) [below, black] {$11$};
	%\node at (-1.5,0) [scale=.5,shape=circle, fill=penColor] {};        

%%%%no man's land

    \draw [penColor4,thick,decoration={brace,raise=2em},decorate] 
        (axis cs:-1.5,0) --
        (axis cs:2,0);

    \draw [penColor4,thick,decoration={brace,raise=2em},decorate] 
        (axis cs:4,0) --
        (axis cs:7.5,0);

    \node[penColor4] at (3,.4) {may or may not converge};
\end{axis}

\end{tikzpicture}
\end{image}

%%%%%%%%%END IMAGE%%%%%%%%%%%

Note that an open circle indicates that convergence cannot be determined without further information, while each closed circle has been colored coded with the choice of color for convergence and divergence.

Thus:

\begin{exercise}
The series represented by $f(-5)$, which is $\sum_{k=0}^{\infty} \left(\answer{-8}\right)^k a_k$:
\begin{multipleChoice}
\choice{converges}
\choice[correct]{diverges}
\choice{could converge or diverge; more information is needed.}
\end{multipleChoice}
\end{exercise}

 \begin{exercise}
The series  represented by $f(2)$, which is $\sum_{k=0}^{\infty} \left(\answer{-1}\right)^k a_k$:
\begin{multipleChoice}
\choice{converges}
\choice{diverges}
\choice[correct]{could converge or diverge; more information is needed.}
\end{multipleChoice}
\end{exercise}

\begin{exercise}
The series  represented by $f(4)$, which is $\sum_{k=0}^{\infty} \left(\answer{1}\right)^k a_k$:
\begin{multipleChoice}
\choice[correct]{converges}
\choice{diverges}
\choice{could converge or diverge; more information is needed.}
\end{multipleChoice}
\end{exercise}

\begin{exercise}
The series  represented by $f(11)$, which is $\sum_{k=0}^{\infty} \left(\answer{8}\right)^k a_k$:
\begin{multipleChoice}
\choice{converges}
\choice{diverges}
\choice[correct]{could converge or diverge; more information is needed.}
\end{multipleChoice}
\end{exercise}

We have previously seen examples of series that converge at one endpoint of the interval of convergence, but not the other; more information about the coefficients $a_k$ would be needed to determine what happens at $x=2$ and $x=11$.


%%%%%%%%%%%%%%%%%%%%%%%%%%%%%%%

Now, answer the following questions.

\begin{exercise}
Select all of the series below that \emph{MUST} converge.
\begin{selectAll}
\choice[correct]{$\sum_{k=0}^{\infty} 5a_k$}
\choice{$\sum_{k=0}^{\infty} 2^k a_k$}
\choice[correct]{$\sum_{k=0}^{\infty} \left(\frac{1}{3}\right)^k a_k$}
\choice[correct]{The series represented by $f(2.5)$}
\choice{The series represented by $f(1)$}
\end{selectAll}
(Note that for the second and third series, one method you could use
is to find the $x$-values for which $f(x)$ equals the given series,
then use the picture to determine what happens at these $x$-values)
\end{exercise}

\begin{exercise}
Select all of the series below that \emph{MUST} diverge.
\begin{selectAll}
\choice{$\sum_{k=0}^{\infty} 5^ka_k$}
\choice[correct]{$\sum_{k=0}^{\infty} 10^k a_k$}
\choice{The series represented by $f(10)$}
\choice[correct]{The series represented by $f(15)$}
\end{selectAll}
\end{exercise}

\begin{exercise}
The series represented by $f(5)$:

\begin{multipleChoice}
\choice{must converge.}
\choice{must diverge.}
\choice[correct]{could converge or diverge; more information is needed.}
\end{multipleChoice}
\end{exercise}

\end{exercise}
\end{exercise}
\end{document}
