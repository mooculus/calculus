\documentclass{ximera}

\newcommand{\RR}{\mathbb R}
\renewcommand{\d}{\,d}
\newcommand{\dd}[2][]{\frac{d #1}{d #2}}
\renewcommand{\l}{\ell}
\newcommand{\ddx}{\frac{d}{dx}}
\newcommand{\dfn}{\textbf}
\newcommand{\eval}[1]{\bigg[ #1 \bigg]}


\author{Jim Talamo}
\license{Creative Commons 3.0 By-bC}


\outcome{}


\begin{document}
\begin{exercise}
Consider the series $\sum_{k=1}^{\infty} \left(1-\frac{3}{k}\right)^k$.  Note that this series is a ``fringe case" of a geometric series; for any value of $k$, the base is slightly less than $1$. 

If we want to use the Root Test, we must evaluate:

\[
\lim_{n \to \infty} \sqrt[n]{\left(1-\frac{3}{n}\right)^n} = \answer{1}
\]

Hence, the Root Test:
\begin{multipleChoice}
\choice{guarantees that the series converges.}
\choice{guarantees that the series converges to $0$.}
\choice{guarantees that the series diverges.}
\choice[correct]{is inconclusive.}
\end{multipleChoice}

To determine what this series does, recall the result:

\[
\lim_{n \to \infty} \left(1+\frac{a}{n}\right)^n = e^a
\]

Using this, we find:

\[
\lim_{n \to \infty} \left(1-\frac{3}{n}\right)^n = \answer{e^{-3}}
\]

Hence, the series:
\begin{multipleChoice}
\choice{converges by the Root Test.}
\choice{diverges by the Root Test.}
\choice{converges by the Divergence Test.}
\choice[correct]{diverges by the Divergence Test.}
\end{multipleChoice}
\end{exercise}

\begin{exercise}
Now, consider the series $\sum_{k=1}^{\infty} \left(1-\frac{3}{k}\right)^{k^2}$.  Note that the exponent has been strengthened!
 
 \begin{exercise}
\[
\lim_{n \to \infty} \left(1-\frac{3}{n}\right)^{n^2} = \answer{0}
\]

\begin{hint}
You cannot use the given formula, but you can use the technique discussed in the limits of sequences section that allows us to compute exponential indeterminate forms!
\end{hint}
 
 Thus:
 \begin{multipleChoice}
\choice{The series diverges by the Divergence Test.}
\choice[correct]{The Divergence Test is inconclusive.}
\end{multipleChoice}

To proceed, we can think about trying to use the Root Test.
\begin{multipleChoice}
\choice[correct]{The Root Test applies to this series.}
\choice{The Root Test does not apply to this series.}
\end{multipleChoice}

Why is the Root Test preferable to the Ratio Test here?
\begin{multipleChoice}
\choice[correct]{The Root Test will allow for a more efficient solution since the summand involves the $k$-th power of an expression.}
\choice{The assumptions for the Root Test are not met.}
\end{multipleChoice}

To use the Root Test, we must compute the limit:

\[
L = \lim_{n \to \infty} \sqrt[n]{\answer{ \left(1-\frac{3}{n}\right)^{n^2}}} 
\]

Evaluating this limit, we find:

\[
L = \answer{e^{-3}}
\]

\begin{hint}
Simplifying: $ \sqrt[n]{\left(1-\frac{3}{n}\right)^{n^2}} = \left(1-\frac{3}{n}\right)^{n^2/n} = \left(1-\frac{3}{n}\right)^{\answer{n}} $.  

This is the same limit that we had before, but arises in a Root Test calculation!
\end{hint}

Hence, the Root Test:
\begin{multipleChoice}
\choice[correct]{guarantees that the series converges.}
\choice{guarantees that the series converges to $0$.}
\choice{guarantees that the series diverges.}
\choice{is inconclusive.}
\end{multipleChoice}

\end{exercise}
\end{exercise}
\end{document}
