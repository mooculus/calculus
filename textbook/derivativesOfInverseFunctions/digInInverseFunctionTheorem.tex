\documentclass{ximera}

\outcome{Understand how the derivative of an inverse function relates to the original derivative.}

\newcommand{\RR}{\mathbb R}
\renewcommand{\d}{\,d}
\newcommand{\dd}[2][]{\frac{d #1}{d #2}}
\renewcommand{\l}{\ell}
\newcommand{\ddx}{\frac{d}{dx}}
\newcommand{\dfn}{\textbf}
\newcommand{\eval}[1]{\bigg[ #1 \bigg]}


\title[Dig-In:]{The inverse function theorem}

\begin{document}
\begin{abstract}
  We see the theoretical underpinning of finding the derivative of an
  inverse function at a point.
\end{abstract}
\maketitle

There is one catch to all the explanations given above where we
computed derivatives of inverse functions. To write something like
\[
\ddx(e^y)=e^y\cdot y'
\]
we need to know that the function $y$ \textit{has} a derivative. All
we have shown is that \textit{if} it has a derivative then that
derivative must be $1/x$. The \textit{Inverse Function Theorem}
guarantees this.

\begin{theorem}[Inverse Function Theorem]\index{Inverse Function Theorem}\label{theorem:IFT}
If $f$ is a differentiable function that is one-to-one near $a$ and
$f'(a) \neq 0$, then
\begin{enumerate}
\item $f^{-1}(x)$ is \textbf{defined} for $x$ near $b=f(a)$,
\item $f^{-1}(x)$ is \textbf{differentiable} near $b=f(a)$, 
\item last, but not least:
  \[
  \eval{\ddx f^{-1}(x)}_{x=b}  = \frac{1}{f'(a)}\qquad\text{where}\qquad b = f(a).
  \]
\end{enumerate}
\end{theorem}

The inverse function theorem gives us a recipe for computing the
derivatives of inverses of functions at points.

\begin{example}
  Let $f$ be a differentiable function that has an inverse. In the
  table below we give several values for both $f$ and $f'$:
  \[
  \begin{array}{|c|c|c|}\hline
    x & f  & f' \\ \hline \hline
    2 & 0  & 2  \\ \hline
    3 & 1  & -2 \\ \hline
    4 & -3 & 0  \\ \hline
  \end{array}
  \]
  Compute
  \[
  \ddx f^{-1}(x)\;\text{at $x=1$.}
  \]
  \begin{explanation}
    From the table above we see that
    \[
    1 = f(\answer[given]{3}).
    \]
    Hence, by the inverse function theorem
    \[
    \left(f^{-1}(1)\right)' = \frac{1}{f'(\answer[given]{3})} = \answer[given]{\frac{-1}{2}}.
    \]
  \end{explanation}
\end{example}

If one example is good, two are better:

\begin{example}
  Let $f$ be a differentiable function that has an inverse. In the
  table below we give several values for both $f$ and $f'$:
  \[
  \begin{array}{|c|c|c|}\hline
    x & f  & f' \\ \hline \hline
    2 & 0  & 2  \\ \hline
    3 & 1  & -2 \\ \hline
    4 & -3 & 0  \\ \hline
  \end{array}
  \]
  Compute
  \[
  \left(f^{-1}(0)\right)'
  \]
  \begin{explanation}
    Note,
    \[
    \left(f^{-1}(0)\right)' = \ddx f^{-1}(x)\;\text{at $x=0$.}
    \]
    From the table above we see that
    \[
    0 = f(\answer[given]{2}).
    \]
    Hence, by the inverse function theorem
    \[
    \left(f^{-1}(0)\right)' = \frac{1}{f'(\answer[given]{2})} = \answer[given]{\frac{1}{2}}.
    \]
  \end{explanation}
\end{example}

Finally, let's see an example where the theorem does not apply.

\begin{example}
  Let $f$ be a differentiable function that has an inverse. In the
  table below we give several values for both $f$ and $f'$:
  \[
  \begin{array}{|c|c|c|}\hline
    x & f  & f' \\ \hline \hline
    2 & 0  & 2  \\ \hline
    3 & 1  & -2 \\ \hline
    4 & -3 & 0  \\ \hline
  \end{array}
  \]
  Compute
  \[
  \eval{\ddx f^{-1}(x)}_{x=-3}
  \]
  \begin{explanation}
    From the table above we see that
    \[
    -3 = f(\answer[given]{4}).
    \]
    Ah! But here, $f'(\answer[given]{4}) = \answer[given]{0}$, so we have no guarantee that the
    inverse exists near the point $x=-3$, but even if it did the inverse would not be differentiable there.
      \end{explanation}
\end{example}


\end{document}
