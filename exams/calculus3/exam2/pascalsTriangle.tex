\documentclass{ximera}

\newcommand{\RR}{\mathbb R}
\renewcommand{\d}{\,d}
\newcommand{\dd}[2][]{\frac{d #1}{d #2}}
\renewcommand{\l}{\ell}
\newcommand{\ddx}{\frac{d}{dx}}
\newcommand{\dfn}{\textbf}
\newcommand{\eval}[1]{\bigg[ #1 \bigg]}


\author{Jim Fowler}
\license{Creative Commons 4.0 By-SA}

\begin{document}

\begin{exercise}
  Let $D$ be the disk of radius $R$.  Compute
  $$
  \iint_D \left( x^8-28 x^6 y^2+70 x^4 y^4-28 x^2 y^6+y^8 \right) \, dA = \answer{0}.
  $$
  % this is integral of z^8  dz dzbar, which is d(z^9/9 dzbar)
You might want to draw Pascal's triangle and reflect for a moment on
these coefficients.  These are not coincidences.

\begin{hint}
  We can do this with polar coordinates.  Let $I$ be the value of the integral.
  \begin{align*}
    I &= \iint_D \left( x^8-28 x^6 y^2+70 x^4 y^4-28 x^2 y^6+y^8 \right) \, dA \\
      &= 
\int_{r =0}^R \int_{\theta = 0}^{2\pi} \left( x^8-28 x^6 y^2+70 x^4 y^4-28 x^2 y^6+y^8 \right) \, r \, d\theta \, dr \\
      &= 
        \int_{r =0}^R \int_{\theta = 0}^{2\pi} \left( r^{8} \cos^8\left(\theta\right) - 28 \, r^{8} \cos^6 \left(\theta\right) \sin^2 \left(\theta\right) + 70 \, r^{8} \cos^4 \left(\theta\right) \sin^4 \left(\theta\right) - 28 \, r^{8} \cos^2 \left(\theta\right) \sin^6 \left(\theta\right) + r^{8} \sin^8 \left(\theta\right) \right) \, r \, d\theta \, dr \\
      &= 
        \int_{r =0}^R \int_{\theta = 0}^{2\pi} r^9 \left( \cos^8\left(\theta\right) - 28 \, \cos^6 \left(\theta\right) \sin^2 \left(\theta\right) + 70 \, \cos^4 \left(\theta\right) \sin^4 \left(\theta\right) - 28 \, \cos^2 \left(\theta\right) \sin^6 \left(\theta\right) + \sin^8 \left(\theta\right) \right) \, d\theta \, dr \\
      &= 
        \frac{R^{10}}{10} \int_{\theta = 0}^{2\pi} \left( \cos^8\left(\theta\right) - 28 \, \cos^6 \left(\theta\right) \sin^2 \left(\theta\right) + 70 \, \cos^4 \left(\theta\right) \sin^4 \left(\theta\right) - 28 \, \cos^2 \left(\theta\right) \sin^6 \left(\theta\right) + \sin^8 \left(\theta\right) \right) \, d\theta.
  \end{align*}
\end{hint}

\begin{hint}
  This is equal to zero by making use of calculations like
  \begin{align*}
    \int_0^{2\pi} \cos^8 \theta \, d\theta &=   \int_0^{2\pi} \sin^8 \theta \, d\theta = \frac{35}{64} \, \pi, \\
    \int_0^{2\pi} \cos^6 \theta \, \sin^2 \theta \, d\theta &=   \int_0^{2\pi} \cos^6 \theta \, \sin^2 \theta \, d\theta = \frac{5}{64} \, \pi, \\
    \int_0^{2\pi} \cos^4 \theta \, \sin^4 \theta \, d\theta &= \frac{3}{64} \, \pi.
  \end{align*}
\end{hint}
\begin{hint}
  Then use the fact that
  $$
  \frac{35}{64} \, \pi - 28  \frac{5}{64} \, \pi + 70 \frac{3}{64} \, \pi - 28 \frac{5}{64} \, \pi + \frac{35}{64} \, \pi = 0.
  $$
\end{hint}
\begin{hint}
  A deeper question remains: \textit{how did I come up with this polynomial?}  Why are these numbers in Pascal's triangle?
\end{hint}
\end{exercise}

\end{document}
