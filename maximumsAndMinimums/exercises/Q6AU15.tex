\documentclass{ximera}

\newcommand{\RR}{\mathbb R}
\renewcommand{\d}{\,d}
\newcommand{\dd}[2][]{\frac{d #1}{d #2}}
\renewcommand{\l}{\ell}
\newcommand{\ddx}{\frac{d}{dx}}
\newcommand{\dfn}{\textbf}
\newcommand{\eval}[1]{\bigg[ #1 \bigg]}


\begin{document}
\author{Kyle Parson\and Nela Lakos}
\outcome{Classify critical points.}
\outcome{Apply the First Derivative Test.}
\outcome{Apply the Second Derivative Test.}
\outcome{Find inflection points.}


\begin{exercise}
Let $f(x)=x\ln x$. The domain of $f$ is
$(\answer{0},\answer{\infty})$.
\begin{exercise}
\begin{selectAll}
\choice{$f$ is not everywhere continuous on $(0,\infty)$}
\choice[correct]{$f$ is continuous on $[M,\infty)$ for some $M>0$}
\choice{There is some $M>0$ such that $f$ is not continuous on $[M,\infty)$.}
\choice{$f$ is nowhere continuous on $(0,\infty)$}
\choice[correct]{$f$ is continuous on $(\varepsilon,\infty)$ for every $\varepsilon>0$.}
\choice[correct]{$f$ is continuous on $(0,\infty)$}
\end{selectAll}
\begin{exercise}
The preceding statements regarding $f$ are true because
\begin{multipleChoice}
\choice{$f$ is the product of two functions whose intervals of continuity have union equal to $(0,\infty)$.}
\choice{the natural logarithm is not continuous.}
\choice{the limit $\lim_{x\to 0^+}f(x)$ exists and is equal to $0$.}
\choice[correct]{$f$ is the product of two functions which are continuous on $(0,\infty)$.}
\end{multipleChoice}
\begin{exercise}
Next, we shall classify the critical points of $f$.

First, compute 
\[
f'(x)=\answer{1+\ln(x)}
\]
\begin{exercise}
The function $f$ has $\answer{1}$ critical point(s). 
\begin{exercise}
The single critical point of $f$ is located at
\[
x=\answer{1/e}
\]
\begin{exercise}
Now compute
\[
f''(x)=\answer{\frac{1}{x}}
\]
\begin{exercise}
Because the value of $f''(\frac{1}{e})=\frac{1}{1/e}=e$ is \wordChoice{\choice[correct]{positive}\choice{negative}\choice{zero}}, the critical point the function  $f$ has at $x=1/e$ is \wordChoice{\choice{a local maximum}\choice[correct]{a local minimum}\choice{neither a local maximum nor a local minimum}\choice{possibly a local maximum, a local minimum or neither}} by the second derivative test.

\begin{exercise}
Finally, we shall identify the inflection points of $f$ and the intervals on which $f$ is concave up or concave down.

The function $f$ has $\answer{0}$ inflection point(s).
\begin{exercise}
Since
\[
\frac{d^2}{dx^2}f(x)=\answer{1/x}
\]
$f''(x)$ is \wordChoice{\choice[correct]{positive}\choice{negative}} on $(0,\infty)$. 
\begin{exercise}
Since $f''(x)>0$ for all $x>0$, $f$ is \wordChoice{\choice[correct]{concave up}\choice{concave down}} on $(\answer{0},\answer{\infty})$.
\end{exercise}
\end{exercise}
\end{exercise}
\end{exercise}
\end{exercise}
\end{exercise}
\end{exercise}
\end{exercise}
\end{exercise}
\end{exercise}
\end{exercise}
\end{document}
