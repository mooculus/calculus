
%you
%\documentclass{ximera}
%
%\newcommand{\RR}{\mathbb R}
\renewcommand{\d}{\,d}
\newcommand{\dd}[2][]{\frac{d #1}{d #2}}
\renewcommand{\l}{\ell}
\newcommand{\ddx}{\frac{d}{dx}}
\newcommand{\dfn}{\textbf}
\newcommand{\eval}[1]{\bigg[ #1 \bigg]}

%
%\title[Refresh:]{Rational functions}
%
%\begin{document}
%\begin{abstract}
%  We review basic material for rational functions.
%\end{abstract}
%\maketitle
%
%\begin{problem}
%  The integration technique of partial fractions is a useful technique
%  when evaluating antiderivatives of rational expressions when the
%  degree of the numerator is less than the degree of the denominator.
%  
%  In order to use this​ technique, it is important to be able to
%  perform polynomial long division when the degree of the numerator is
%  greater than the degree of the denominator.  
%  
%  The following questions give practice with polynomial long division.
%  \begin{multipleChoice}
%    \choice[correct]{I understand.}
%    \choice{I do not understand.}
%  \end{multipleChoice}
%\end{problem}
%
%\begin{problem}
%  Divide:
%  \[
%  \frac{2x^2-3x-5}{x+3} = \answer{2x-9} + \frac{22}{x+3}
%  \]
%  \begin{hint}
%    Use long division.
%  \end{hint}
%\end{problem}
%
%\begin{problem}
%  Divide:
%  \[
%  \frac{4x^2+6x-1}{2x-1} = \answer{2x+4} + \frac{3}{2x-1}
%  \]
%  \begin{hint}
%    Use long division.
%  \end{hint}
%\end{problem}
%
%\begin{problem}
%  The technique of partial fractions requires that we are able to
%  factor the denominator of rational expressions.  We then have to
%  look for simpler rational functions whose sum or difference could
%  give the original function.  For​ instance, it would be hard to​
%  evaluate:
%  \[
%  \int \frac{1}{x^2+x}\d x
%  \]
%  ​However, it is true​ that:
%  \begin{align*}
%    \frac{1}{x^2+x} &= \frac{1}{x(x+1)}\\
%    &=\frac{1}{x}-\frac{1}{x+1}
%  \end{align*}
%  and it is easy to integrate the right-hand side of this expression.
%
%  
%  We will study how to obtain the right-hand side of this​ expression,
%  but a conceptual understanding of this procedure that​ ``undoes'' the
%  procedure of finding a common denominator is greatly aided by
%  remembering how to add rational functions.
%  \begin{multipleChoice}
%    \choice[correct]{I understand.}
%    \choice{I do not understand.}
%  \end{multipleChoice}
%\end{problem}
%
%\begin{problem}
%  Express as a single (reduced) fraction:
%  \[
%  \frac{7}{x-2}+\frac{5}{x} = \frac{\answer{-10+12x}}{\answer{-2x+x^2}}
%  \]
%\end{problem}
%
%
%\begin{problem}
%  Express as a single (reduced) fraction:
%  \[
%  \frac{x-7}{x^2-4} + \frac{1}{x-3} - \frac{2}{x-5} = \frac{\answer{-14x^2+75x-109}}{\answer{(-5 + x) (-3 + x) (-4 + x^2)}}
%  \]
%\end{problem}
%
%
%\begin{problem}
%  Express as a single (reduced) fraction:
%  \[
%  \frac{2}{x^2+7x+11} + \frac{5}{x^2-9} = \frac{\answer{37 + 35 x + 7 x^2}}{\answer{(-9 + x^2) (11 + 7 x + x^2)}}
%  \]
%\end{problem}
%
%\begin{problem}
%  ​L'H\^opital's rule is a very useful technique in evaluating the limits
%  that arise while solving problems about improper integrals.  The
%  following questions review and give practice using​ L'H\^opital's rule.
%  \begin{multipleChoice}
%    \choice[correct]{I understand.}
%    \choice{I do not understand.}
%  \end{multipleChoice}
%\end{problem}
%
%\begin{problem}
%  List the steps for computing a limit using L'H\^opital's rule for computing:
%  \[
%  \lim_{x\to a} \frac{f(x)}{g(x)}
%  \]
%  \begin{multipleChoice}
%    \choice{Take the derivative of the numerator and denominator.}
%    \choice[correct]{Compute the limits: $\lim_{x\to a}f(x)$ and $\lim_{x\to a}g(x)$.}
%    \choice{Use the quotient rule.}
%  \end{multipleChoice}
%  \begin{problem}
%    If
%    \[
%    \lim_{x\to a}f(x) = 0 = \lim_{x\to a}g(x),
%    \]
%    then
%    \begin{multipleChoice}
%      \choice{The limit is zero.}
%      \choice{The limit is of the form \zeroOverZero, and hence is indeterminant.}
%      \choice[correct]{The limit is of the form \zeroOverZero, and we may apply L'H\^opital's rule.}
%    \end{multipleChoice}
%    \begin{problem}
%      Now we
%      \begin{multipleChoice}
%        \choice[correct]{Take the derivatives of $f$ and $g$.}
%        \choice{Use the quotient rule.}
%      \end{multipleChoice}
%      \begin{problem}
%        Finally, we are done provided that
%        \[
%        \lim_{x\to a} \frac{f'(x)}{g'(x)}
%        \]
%        \begin{multipleChoice}
%          \choice[correct]{The limit above exists.}
%          \choice{The limit above is of the form \zeroOverZero.}
%        \end{multipleChoice}
%        \begin{problem}
%          If the final limit is of the form \zeroOverZero,
%          \begin{multipleChoice}
%            \choice{The limit does not exist.}
%            \choice{L'H\^opital's rule, has failed and this limit cannot be evaluated.}
%            \choice[correct]{We can use L'H\^opital's rule again.}
%          \end{multipleChoice}
%        \end{problem}
%      \end{problem}
%    \end{problem}
%  \end{problem}
%\end{problem}
%
%
%\begin{problem}
%  Suppose the limit
%  \[
%  \lim_{x\to a} f(x) \cdot g(x)
%  \]
%  has the form \zeroTimesInfty.  List the steps for for computing this
%  limit.  Select all that apply:
%  \begin{selectAll}
%    \choice{The determinant form  \zeroTimesInfty tells us that the limit is zero.}
%    \choice{The determinant form  \zeroTimesInfty tells us that the limit is $1$.}
%    \choice{The determinant form  \zeroTimesInfty tells us that the limit does not exist.}
%    \choice[correct]{Rewrite as  $\lim_{x\to a} \frac{f(x)}{g(x)^{-1}}$ and then use L'H\^optial's rule.}
%    \choice[correct]{Rewrite as  $\lim_{x\to a} \frac{g(x)}{f(x)^{-1}}$ and then use L'H\^optial's rule.}
%  \end{selectAll}
%\end{problem}
%
%
%\begin{problem}
%  Suppose the limit
%  \[
%  \lim_{x\to a} f(x)^{g(x)}
%  \]
%  has the form \oneToInfty.  List the steps for for computing this
%  limit.  Select all that apply:
%  \begin{selectAll}
%    \choice{The determinant form  \oneToInfty tells us that the limit is zero.}
%    \choice{The determinant form  \oneToInfty tells us that the limit is $1$.}
%    \choice{The determinant form  \oneToInfty tells us that the limit does not exist.}
%    \choice[correct]{Consider the limit $\lim_{x\to a} e^{\ln(f(x)^{g(x)})}$.}
%  \end{selectAll}
%  \begin{problem}
%    Now we may use log rules to write
%    \[
%    \ln(f(x)^{g(x)}) = g(x)\ln(f(x)).
%    \]
%    In this case $\lim_{x\to a}$ will be of the form:
%    \begin{multipleChoice}
%      \choice{\numOverZero}
%      \choice{\zeroOverZero}
%      \choice[correct]{\zeroTimesInfty}
%    \end{multipleChoice}
%  \end{problem}
%\end{problem}
%
%
%\begin{problem}
%  Use L'H\^opital's rule \textit{twice} to compute:
%  \[
%  \lim_{x\to 0} \frac{1-\cos(3x)}{4x^2} = \answer{9/8}
%  \]
%\end{problem}
%
%
%\begin{problem}
%  Use L'H\^opital's rule to compute:
%  \[
%  \lim_{x\to 0^+} x^{13x}= \answer{1}
%  \]
%\end{problem}
%
%
%\begin{problem}
%  Use L'H\^opital's rule to compute:
%  \[
%  \lim_{x\to \infty}\left(1+\frac{a}{x}\right)^x = \answer{e^a}
%  \]
%\end{problem}
%
%\begin{problem}
%  Use L'H\^opital's rule to compute:
%  \[
%  \lim_{x\to 0}(7x+\cos(x))^{1/x} = \answer{e^7}
%  \]
%\end{problem}
%
%
%\end{document}
=======
\documentclass{ximera}

\newcommand{\RR}{\mathbb R}
\renewcommand{\d}{\,d}
\newcommand{\dd}[2][]{\frac{d #1}{d #2}}
\renewcommand{\l}{\ell}
\newcommand{\ddx}{\frac{d}{dx}}
\newcommand{\dfn}{\textbf}
\newcommand{\eval}[1]{\bigg[ #1 \bigg]}


\title[Refresh:]{Rational functions}

\begin{document}
\begin{abstract}
  We review basic material for rational functions.
\end{abstract}
\maketitle

\begin{problem}
  The integration technique of partial fractions is a useful technique
  when evaluating antiderivatives of rational expressions when the
  degree of the numerator is less than the degree of the denominator.
  
  In order to use this technique, it is important to be able to
  perform polynomial long division when the degree of the numerator is
  greater than the degree of the denominator.

  
  The following questions give practice with polynomial long division.
  \begin{multipleChoice}
    \choice[correct]{I understand.}
    \choice{I do not understand.}
  \end{multipleChoice}
\end{problem}

\begin{problem}
  Divide:
  \[
  \frac{2x^2-3x-5}{x+3} = \answer{2x-9} + \frac{22}{x+3}
  \]
  \begin{hint}
    Use long division.
  \end{hint}
\end{problem}

\begin{problem}
  Divide:
  \[
  \frac{4x^2+6x-1}{2x-1} = \answer{2x+4} + \frac{3}{2x-1}
  \]
  \begin{hint}
    Use long division.
  \end{hint}
\end{problem}

\begin{problem}
  The technique of partial fractions requires that we are able to
  factor the denominator of rational expressions.  We then have to
  look for simpler rational functions whose sum or difference could
  give the original function. For instance, it would be hard to evaluate:
  \[
  \int \frac{1}{x^2+x}\d x
  \]
  However, it is true that:
  \begin{align*}
    \frac{1}{x^2+x} &= \frac{1}{x(x+1)}\\
    &=\frac{1}{x}-\frac{1}{x+1}
  \end{align*}
  and it is easy to integrate the right-hand side of this expression.

  
  We will study how to obtain the right-hand side of this expression, but a conceptual understanding of this procedure that ``undoes'' the
  procedure of finding a common denominator is greatly aided by
  remembering how to add rational functions.
  \begin{multipleChoice}
    \choice[correct]{I understand.}
    \choice{I do not understand.}
  \end{multipleChoice}
\end{problem}

\begin{problem}
  Express as a single (reduced) fraction:
  \[
  \frac{7}{x-2}+\frac{5}{x} = \frac{\answer{-10+12x}}{\answer{-2x+x^2}}
  \]
\end{problem}


\begin{problem}
  Express as a single (reduced) fraction:
  \[
  \frac{x-7}{x^2-4} + \frac{1}{x-3} - \frac{2}{x-5} = \frac{\answer{-14x^2+75x-109}}{\answer{(-5 + x) (-3 + x) (-4 + x^2)}}
  \]
\end{problem}


\begin{problem}
  Express as a single (reduced) fraction:
  \[
  \frac{2}{x^2+7x+11} + \frac{5}{x^2-9} = \frac{\answer{37 + 35 x + 7 x^2}}{\answer{(-9 + x^2) (11 + 7 x + x^2)}}
  \]
\end{problem}

\begin{problem}
  L'H\^opital's rule is a very useful technique in evaluating the limits
  that arise while solving problems about improper integrals.  The
  following questions review and give practicu using L'H\^opital's rule.
  \begin{multipleChoice}
    \choice[correct]{I understand.}
    \choice{I do not understand.}
  \end{multipleChoice}
\end{problem}

\begin{problem}
  List the steps for computing a limit using L'H\^opital's rule for computing:
  \[
  \lim_{x\to a} \frac{f(x)}{g(x)}
  \]
  \begin{multipleChoice}
    \choice{Take the derivative of the numerator and denominator.}
    \choice[correct]{Compute the limits: $\lim_{x\to a}f(x)$ and $\lim_{x\to a}g(x)$.}
    \choice{Use the quotient rule.}
  \end{multipleChoice}
  \begin{problem}
    If
    \[
    \lim_{x\to a}f(x) = 0 = \lim_{x\to a}g(x),
    \]
    then
    \begin{multipleChoice}
      \choice{The limit is zero.}
      \choice{The limit is of the form \zeroOverZero, and hence is indeterminant.}
      \choice[correct]{The limit is of the form \zeroOverZero, and we may apply L'H\^opital's rule.}
    \end{multipleChoice}
    \begin{problem}
      Now we
      \begin{multipleChoice}
        \choice[correct]{Take the derivatives of $f$ and $g$.}
        \choice{Use the quotient rule.}
      \end{multipleChoice}
      \begin{problem}
        Finally, we are done provided that
        \[
        \lim_{x\to a} \frac{f'(x)}{g'(x)}
        \]
        \begin{multipleChoice}
          \choice[correct]{The limit above exists.}
          \choice{The limit above is of the form \zeroOverZero.}
        \end{multipleChoice}
        \begin{problem}
          If the final limit is of the form \zeroOverZero,
          \begin{multipleChoice}
            \choice{The limit does not exist.}
            \choice{L'H\^opital's rule, has failed and this limit cannot be evaluated.}
            \choice[correct]{We can use L'H\^opital's rule again.}
          \end{multipleChoice}
        \end{problem}
      \end{problem}
    \end{problem}
  \end{problem}
\end{problem}


\begin{problem}
  Suppose the limit
  \[
  \lim_{x\to a} f(x) \cdot g(x)
  \]
  has the form \zeroTimesInfty.  List the steps for for computing this
  limit.  Select all that apply:
  \begin{selectAll}
    \choice{The determinant form  \zeroTimesInfty tells us that the limit is zero.}
    \choice{The determinant form  \zeroTimesInfty tells us that the limit is $1$.}
    \choice{The determinant form  \zeroTimesInfty tells us that the limit does not exist.}
    \choice[correct]{Rewrite as  $\lim_{x\to a} \frac{f(x)}{g(x)^{-1}}$ and then use L'H\^optial's rule.}
    \choice[correct]{Rewrite as  $\lim_{x\to a} \frac{g(x)}{f(x)^{-1}}$ and then use L'H\^optial's rule.}
  \end{selectAll}
\end{problem}


\begin{problem}
  Suppose the limit
  \[
  \lim_{x\to a} f(x)^{g(x)}
  \]
  has the form \oneToInfty.  List the steps for for computing this
  limit.  Select all that apply:
  \begin{selectAll}
    \choice{The determinant form  \oneToInfty tells us that the limit is zero.}
    \choice{The determinant form  \oneToInfty tells us that the limit is $1$.}
    \choice{The determinant form  \oneToInfty tells us that the limit does not exist.}
    \choice[correct]{Consider the limit $\lim_{x\to a} e^{\ln(f(x)^{g(x)})}$.}
  \end{selectAll}
  \begin{problem}
    Now we may use log rules to write
    \[
    \ln(f(x)^{g(x)}) = g(x)\ln(f(x)).
    \]
    In this case $\lim_{x\to a}$ will be of the form:
    \begin{multipleChoice}
      \choice{\numOverZero}
      \choice{\zeroOverZero}
      \choice[correct]{\zeroTimesInfty}
    \end{multipleChoice}
  \end{problem}
\end{problem}


\begin{problem}
  Use L'H\^opital's rule \textit{twice} to compute:
  \[
  \lim_{x\to 0} \frac{1-\cos(3x)}{4x^2} = \answer{9/8}
  \]
\end{problem}


\begin{problem}
  Use L'H\^opital's rule to compute:
  \[
  \lim_{x\to 0^+} x^{13x}= \answer{1}
  \]
\end{problem}


\begin{problem}
  Use L'H\^opital's rule to compute:
  \[
  \lim_{x\to \infty}\left(1+\frac{a}{x}\right)^x = \answer{e^a}
  \]
\end{problem}

\begin{problem}
  Use L'H\^opital's rule to compute:
  \[
  \lim_{x\to 0}(7x+\cos(x))^{1/x} = \answer{e^7}
  \]
\end{problem}


\end{document}
>>>>>>> 692131c596a9f13bc16c06f5c70be8e1fae83f17
