\documentclass{ximera}

\newcommand{\RR}{\mathbb R}
\renewcommand{\d}{\,d}
\newcommand{\dd}[2][]{\frac{d #1}{d #2}}
\renewcommand{\l}{\ell}
\newcommand{\ddx}{\frac{d}{dx}}
\newcommand{\dfn}{\textbf}
\newcommand{\eval}[1]{\bigg[ #1 \bigg]}


\author{Jim Talamo}
\license{Creative Commons 3.0 By-bC}


\outcome{}


\begin{document}
\begin{exercise}
This exercise explores some important concepts.  Please pay attention to how the concepts are related.

Consider the sequence $\{a_n\}_{n=1}$ given by $a_n = \frac{2n+2}{3n+1}$ and let $s_n = \sum_{k=1}^n a_k$.

The \emph{sequence} $a_n$:

\begin{multipleChoice}
\choice{converges to $0$.}
\choice[correct]{converges to $\frac{2}{3}$.}
\choice{diverges by the divergence test.}
\choice{converges by the divergence test.}
\choice{might converge or diverge; the divergence test is inconclusive.}
\end{multipleChoice}

\begin{feedback}
Note that this is a question about the \emph{sequence} $a_n$, NOT the series $\sum_{k=1}^{\infty} a_k$.  The divergence test applies to \emph{series}, not sequences.
\end{feedback}

The series $\sum_{k=1}^{\infty} a_k$:

\begin{multipleChoice}
\choice{converges to $0$.}
\choice[correct]{diverges by the divergence test.}
\choice{converges by the divergence test.}
\choice{might converge or diverge; the divergence test is inconclusive.}
\end{multipleChoice}

\begin{feedback}
The divergence test is applicable for this problem because a \emph{series} is being considered.  Since $\lim_{n \to \infty} a_n =  \frac{2}{3} \neq 0$, the series diverges by the divergence test.
\end{feedback}



The limit of the \emph{sequence} $s_n$:

\begin{multipleChoice}
\choice{exists but more analysis is needed to determine its value.}
\choice{exists and is $0$.}
\choice[correct]{does not exist.}
\end{multipleChoice}
\begin{feedback}
Since $\sum_{k=1}^{\infty} a_k$ diverges, $\lim_{n \to \infty} s_n$ does not exist by definition.  Note that the divergence test is used to conclude that the \emph{series} $\sum_{k=1}^{\infty} a_k$ diverges, but the definition of what a convergence series is is used to determine that $\lim_{n \to \infty} s_n$ does not exist.
\end{feedback}
\end{exercise}

\end{document}
