\documentclass{ximera}

\newcommand{\RR}{\mathbb R}
\renewcommand{\d}{\,d}
\newcommand{\dd}[2][]{\frac{d #1}{d #2}}
\renewcommand{\l}{\ell}
\newcommand{\ddx}{\frac{d}{dx}}
\newcommand{\dfn}{\textbf}
\newcommand{\eval}[1]{\bigg[ #1 \bigg]}


\title[Dig-In:]{What is a limit?}
\begin{document}
\begin{abstract}
  Here we introduce limits.
\end{abstract}
\maketitle

\begin{question}
Consider the function:
\[
f(x) = \frac{\sin(x)}{x}
\]
While $f(x)$ is undefined at $x=0$, we can still plot $f(x)$ at other
values, %see Figure~\ref{plot:(x^2 - 3x + 2)/(x-2)}.

%\begin{image}
%\begin{tikzpicture}
%	\begin{axis}[
%            domain=-2:4,
%            axis lines =middle, xlabel=$x$, ylabel=$y$,
%            every axis y label/.style={at=(current axis.above origin),anchor=south},
%            every axis x label/.style={at=(current axis.right of origin),anchor=west},
%            grid=both,
%            grid style={dashed, gridColor},
%            xtick={-2,...,4},
%            ytick={-3,...,3},
%          ]
%	  \addplot [very thick, penColor, smooth] {x-1};
%          \addplot[color=penColor,fill=background,only marks,mark=*] coordinates{(2,1)};  %% open hole
%        \end{axis}
%\end{tikzpicture}
%% \caption{A plot of $f(x)=\protect\frac{x^2 - 3x + 2}{x-2}$.}
%% \label{plot:(x^2 - 3x + 2)/(x-2)}
%\end{image}

Create a table of values of $f$ below:

\begin{tableanswer}
 x & f(x) \\ \hline
 1 &  \answer{sin(1)} \\
 0.1&  \answer{10*sin(0.1)} \\
 0.01 &  \answer{100*sin(0.01)}\\
 0.001 &   \answer{1000*sin(0.001)} \\
 0 & \text{undefined}\\
 -0.001 &  \answer{1000*sin(0.001)}
 -0.01&   \answer{100*sin(0.01)}\\
 -0.1&   \answer{10*sin(0.1)}\\
 -1 &   \answer{sin(1)} \\
\end{tableanswer}

It appears that $f$ is approaching the value $\answer{1}$ as $x$ gets closer and closer to $0$.
\end{question}

This example leads to the following intuitive concept.  We will meet a more precise formulation of this definition in an optional section (``the $\epsilon - \delta$ definition of a limit'').

\begin{definition}[Intuitive ``definition'' of a limit]
Let $f$ be a function.  If $f(x)$ can be made arbitrarily close to $L$ by making $x$ sufficiently close, but not equal to, $a$, then we say that the \textbf{limit} of $f(x)$ as $x$ approaches $a$ is $L$.  This is written symbolically as

\[
\lim_{x\to a} f(x) = L
\]
\end{definition}

\begin{question}

Select the appropriate way to verbalize the symbolic expression 

\[
\lim_{x \to 4} \sin(x)
\]

\begin{multiple-choice}
\item[correct] ``The limit of sine of x as x approaches $4$''.
\item ``Lim x to 4 sine x''
\item ``Limit of x approaches 4 of sine''
\end{multiple-choice}
\end{question}

\begin{question}
``If $f(3) = 4$ then $\displaystyle \lim_{x \to 3} f(x) = 4$''

\begin{multiple-choice}
\item[correct] False
\item True
\end{multiple-choice}

\feedback{For instance, if $f(x) = \begin{cases} 4 \text{if $x=3$} \\ 0 \text{else} \end{cases}$ then $f(3)=4$, but $\displaystyle \lim_{x \to 3} f(x) = 0$}
\end{question}

\begin{question}
Using the graph of $f$ below, does it appear that $\lim_{x \to 1} f(x)$ exists?  

%graph

\begin{multiple-choice}
\choice[correct]{Yes}
\choice{No}
\end{multiple-choice}

%only reveal after correct answer

\begin{question}
	It appears that $\lim_{x\to 1} f(x) = $ \answer{4} 
\end{question}


\end{question}

%replicate last question with  a removable discontinuity,  a continuous function, and finally a jump discontinuity.

%Jump discontinuity model:

\begin{question}
Using the graph of $f$ below, does it appear that $\lim_{x \to 1} f(x)$ exists?  

%graph

\begin{multiple-choice}
\choice{Yes}
\choice[correct]{No}
\end{multiple-choice}

%only reveal after correct answer

\begin{question}
	Even though $\lim_{x \to 1} f(x)$ does not exist, values of $f(x)$ appear to be approaching \answer{3} when $x \to 1$ through numbers greater than $1$, and values of $f(x)$ appear to be approaching \answer{6} when $x \to 1$ through numbers less than $1$.
\end{question}

\end{question}

The last question motivates the following:

\begin{definition}[Intuitive ``definition'' of a one sided limit]
Let $f$ be a function.  If $f(x)$ can be made arbitrarily close to $L$ by making $x>a$ sufficiently close, but not equal to, $a$, then we say that the \textbf{limit} of $f(x)$ as $x$ approaches $a$ from the right is $L$.  This is written symbolically as

\[
\lim_{x\to a^+} f(x) = L
\]

Similarly, if $f(x)$ can be made arbitrarily close to $L$ by making $x<a$ sufficiently close, but not equal to, $a$, then we say that the \textbf{limit} of $f(x)$ as $x$ approaches $a$ from the left is $L$.  This is written symbolically as

\[
\lim_{x\to a^-} f(x) = L
\]

\end{definition}

% oracle one sided limits here

\begin{theorem}
	$\lim_{x \to a} f(x)$ exists if and only if  
	\begin{itemize}
	   \item $\lim_{x \to a^-} f(x)$ exists
	   \item $\lim_{x \to a^+} f(x)$ exists
	   \item $\lim_{x \to a^-} f(x) = \lim_{x \to a^+} f(x)$
	\end{itemize}
	
	In this case, $\lim_{x \to a} f(x)$ is equal to the common value of the two one sided limits.
\end{theorem}









\end{document}
