\documentclass{ximera}

\newcommand{\RR}{\mathbb R}
\renewcommand{\d}{\,d}
\newcommand{\dd}[2][]{\frac{d #1}{d #2}}
\renewcommand{\l}{\ell}
\newcommand{\ddx}{\frac{d}{dx}}
\newcommand{\dfn}{\textbf}
\newcommand{\eval}[1]{\bigg[ #1 \bigg]}


\title[Dig-In:]{Approximating area with rectangles}

\begin{document}
\begin{abstract}
  We introduce the basic idea of using rectangles to approximate the
  area under a curve.
\end{abstract}
\maketitle

\section{Limits and areas}

Here is the idea. We want to compute the area between the curve and
the horizontal axis that is bounded by $x=a$ and $x=b$.

Consider the function $f(x) = x^2 +1$ and suppose we wanted to know
the area between the curve $y=f(x)$ and the $x$-axis bounded by $x=-1$
and $x=1$:
\begin{image}
  \begin{tikzpicture}
	\begin{axis}[
            domain=-1.2:1.2, xmin =-1.2,xmax=1.2,ymax=2.2,ymin=-.2,
            axis lines=center, xlabel=$x$, ylabel=$y$,
            every axis y label/.style={at=(current axis.above origin),anchor=south},
            every axis x label/.style={at=(current axis.right of origin),anchor=west},
            %xtick={1,1.25,1.5,1.75,2}, ytickmin=4, ytickmax=1,
            axis on top,
          ]
          \addplot [draw=none,fill=fillp,domain=-1:1, smooth] {x^2+1} \closedcycle;
          \addplot [very thick,penColor, smooth] {x^2+1};
%          \addplot [color=penColor,fill=penColor,only marks,mark=*] coordinates{(1,1)};  %% closed hole         
%          \addplot [color=penColor,fill=penColor,only marks,mark=*] coordinates{(1.25,1.06)};  %% closed hole       
%          \addplot [color=penColor,fill=penColor,only marks,mark=*] coordinates{(1.5,1.25)};  %% closed hole  
%          \addplot [color=penColor,fill=penColor,only marks,mark=*] coordinates{(1.75,1.56)};  %% closed hole       
          
          %\node at (axis cs:1,1) [textColor,above] {$f(x_0^*)$};
          %\node at (axis cs:1.25,1.06) [textColor,above] {$f(x_1^*)$};
          %\node at (axis cs:1.5,1.3) [textColor,above] {$f(x_2^*)$};
          %\node at (axis cs:1.75,1.6) [textColor,above] {$f(x_3^*)$};
        \end{axis}
\end{tikzpicture}
\end{image}
One way to do this would be to approximate the area with rectangles.
\begin{image}
  \begin{tikzpicture}
	\begin{axis}[
            domain=-1.2:1.2, xmin =-1.2,xmax=1.2,ymax=2.2,ymin=-.2,
            axis lines=center, xlabel=$x$, ylabel=$y$,
            every axis y label/.style={at=(current axis.above origin),anchor=south},
            every axis x label/.style={at=(current axis.right of origin),anchor=west},
            %xtick={1,1.25,1.5,1.75,2}, ytickmin=4, ytickmax=1,
            axis on top,
          ]
          \foreach \rectnumber in {1,2,...,8}
                   {
                     \addplot [draw=penColor,fill=fillp]
                     plot coordinates {({-1+(\rectnumber - 1) * 2/8},{(-1+\rectnumber * 2/8)^2+1} ) ({-1+(\rectnumber) * 2/8},{(-1+\rectnumber * 2/8)^2+1 })} \closedcycle;
                     }
          \addplot [very thick,penColor, smooth] {x^2+1};
%          \addplot [color=penColor,fill=penColor,only marks,mark=*] coordinates{(1,1)};  %% closed hole         
%          \addplot [color=penColor,fill=penColor,only marks,mark=*] coordinates{(1.25,1.06)};  %% closed hole       
%          \addplot [color=penColor,fill=penColor,only marks,mark=*] coordinates{(1.5,1.25)};  %% closed hole  
%          \addplot [color=penColor,fill=penColor,only marks,mark=*] coordinates{(1.75,1.56)};  %% closed hole       
          
          %\node at (axis cs:1,1) [textColor,above] {$f(x_0^*)$};
          %\node at (axis cs:1.25,1.06) [textColor,above] {$f(x_1^*)$};
          %\node at (axis cs:1.5,1.3) [textColor,above] {$f(x_2^*)$};
          %\node at (axis cs:1.75,1.6) [textColor,above] {$f(x_3^*)$};
        \end{axis}
\end{tikzpicture}
\end{image}
We could find the area exactly if we could compute the limit as the
width of the rectangles goes to zero and the number of rectangles goes
to infinity.

\begin{definition}
  If we are apporoximating the area between a curve and the $x$-axis
  on $[a,b]$ with $n$ rectangles of width $\Delta x$, then
  \[
  \Delta x = \frac{b-a}{n}.
  \]
\end{definition}



\begin{definition}
  When approximating an area with rectangles, a \dfn{grid point} is a
  point $x$ that determines the edge of a rectangle.
\begin{image}
  \begin{tikzpicture}
	\begin{axis}[
            domain=-1.2:1.2, xmin =-1.09545,xmax=1.09545,ymax=2.2,ymin=-0.2,
            clip=false,
            axis lines=center, xlabel=$x$, ylabel=$y$,
            every axis y label/.style={at=(current axis.above origin),anchor=south},
            every axis x label/.style={at=(current axis.right of origin),anchor=west},
            %xtick={1,1.25,1.5,1.75,2}, ytickmin=4, ytickmax=1,
            axis on top,
          ]
          \foreach \rectnumber in {1,2,...,8}
          {
          \addplot [draw=penColor,fill=fillp]
          plot coordinates {({-1+(\rectnumber -1)*2/8},{(-1+(\rectnumber -.5)*2/8)^2+1}) ({-1+(\rectnumber)*2/8},{(-1+(\rectnumber -.5)*2/8)^2+1})}
          \closedcycle;
%          \pgfmathparse{-1+(\rectnumber -1)*2/8}\let\a\pgfmathresult
%          \draw [->,very thick] (axis cs:0,-.5) -- (axis cs:{\aa},-0.5);
          }
          \addplot [very thick,penColor, smooth] {x^2+1};

          
%          \addplot [color=penColor,fill=penColor,only marks,mark=*] coordinates{(1,1)};  %% closed hole         
%          \addplot [color=penColor,fill=penColor,only marks,mark=*] coordinates{(1.25,1.06)};  %% closed hole       
%          \addplot [color=penColor,fill=penColor,only marks,mark=*] coordinates{(1.5,1.25)};  %% closed hole  
%          \addplot [color=penColor,fill=penColor,only marks,mark=*] coordinates{(1.75,1.56)};  %% closed hole       
          
          %\node at (axis cs:1,1) [textColor,above] {$f(x_0^*)$};
          %\node at (axis cs:1.25,1.06) [textColor,above] {$f(x_1^*)$};
          %\node at (axis cs:1.5,1.3) [textColor,above] {$f(x_2^*)$};
          %\node at (axis cs:1.75,1.6) [textColor,above] {$f(x_3^*)$};
        \end{axis}
\end{tikzpicture}
\end{image}
Note, if we are approximating the area between a curve and the
horizontal axis for $a\le x\le b$ with $n$ rectangles, then it is
always the case that
\[
x_0=a\qquad\text{and}\qquad x_n = b.
\]
\end{definition}

\begin{question}
  If we are approximating the area between a curve and the horizontal
  axis with $11$ rectangles, how many grid points will we have?
  \begin{hint}
    You can draw it!
  \end{hint}
  \begin{prompt}
    We'll have $\answer{12}$ grid points.
  \end{prompt}
\end{question}





\section{But which set of rectangles?}

If we are going to try and actually use many small rectangles to
compute the area under a curve, we should decide on exactly
\textit{which} rectangles we want to use. Here are three options that
we consider:

\paragraph{Rectangles defined by left endpoints}

We can set the rectangles up so that the left endpoint touches the
curve.

\begin{image}
  \begin{tikzpicture}
	\begin{axis}[
            domain=-1.2:1.2, xmin =-1.2,xmax=1.2,ymax=2.2,ymin=-.2,
            axis lines=center, xlabel=$x$, ylabel=$y$,
            every axis y label/.style={at=(current axis.above origin),anchor=south},
            every axis x label/.style={at=(current axis.right of origin),anchor=west},
            %xtick={1,1.25,1.5,1.75,2}, ytickmin=4, ytickmax=1,
            axis on top,
          ]
          \foreach \rectnumber in {1,2,...,8}
                   {
                     \addplot [draw=penColor,fill=fillp]
                     plot coordinates {({-1+(\rectnumber - 1) * 2/8},{(-1+(\rectnumber-1) * 2/8)^2+1} ) ({-1+(\rectnumber) * 2/8},{(-1+(\rectnumber-1) * 2/8)^2+1 })} \closedcycle;
                     }
          \addplot [very thick,penColor, smooth] {x^2+1};
%          \addplot [color=penColor,fill=penColor,only marks,mark=*] coordinates{(1,1)};  %% closed hole         
%          \addplot [color=penColor,fill=penColor,only marks,mark=*] coordinates{(1.25,1.06)};  %% closed hole       
%          \addplot [color=penColor,fill=penColor,only marks,mark=*] coordinates{(1.5,1.25)};  %% closed hole  
%          \addplot [color=penColor,fill=penColor,only marks,mark=*] coordinates{(1.75,1.56)};  %% closed hole       
          
          %\node at (axis cs:1,1) [textColor,above] {$f(x_0^*)$};
          %\node at (axis cs:1.25,1.06) [textColor,above] {$f(x_1^*)$};
          %\node at (axis cs:1.5,1.3) [textColor,above] {$f(x_2^*)$};
          %\node at (axis cs:1.75,1.6) [textColor,above] {$f(x_3^*)$};
        \end{axis}
\end{tikzpicture}
\end{image}

In the graph above, every rectangle is touching the curve at its
left-endpoint.


\paragraph{Rectangles defined by right endpoints}

We can set the rectangles up so that the right endpoint touches the
curve.

\begin{image}
  \begin{tikzpicture}
	\begin{axis}[
            domain=-1.2:1.2, xmin =-1.2,xmax=1.2,ymax=2.2,ymin=-.2,
            axis lines=center, xlabel=$x$, ylabel=$y$,
            every axis y label/.style={at=(current axis.above origin),anchor=south},
            every axis x label/.style={at=(current axis.right of origin),anchor=west},
            %xtick={1,1.25,1.5,1.75,2}, ytickmin=4, ytickmax=1,
            axis on top,
          ]
          \foreach \rectnumber in {1,2,...,8}
                   {
                     \addplot [draw=penColor,fill=fillp]
                     plot coordinates {({-1+(\rectnumber - 1) * 2/8},{(-1+\rectnumber * 2/8)^2+1} ) ({-1+(\rectnumber) * 2/8},{(-1+\rectnumber * 2/8)^2+1 })} \closedcycle;
                     }
          \addplot [very thick,penColor, smooth] {x^2+1};
%          \addplot [color=penColor,fill=penColor,only marks,mark=*] coordinates{(1,1)};  %% closed hole         
%          \addplot [color=penColor,fill=penColor,only marks,mark=*] coordinates{(1.25,1.06)};  %% closed hole       
%          \addplot [color=penColor,fill=penColor,only marks,mark=*] coordinates{(1.5,1.25)};  %% closed hole  
%          \addplot [color=penColor,fill=penColor,only marks,mark=*] coordinates{(1.75,1.56)};  %% closed hole       
          
          %\node at (axis cs:1,1) [textColor,above] {$f(x_0^*)$};
          %\node at (axis cs:1.25,1.06) [textColor,above] {$f(x_1^*)$};
          %\node at (axis cs:1.5,1.3) [textColor,above] {$f(x_2^*)$};
          %\node at (axis cs:1.75,1.6) [textColor,above] {$f(x_3^*)$};
        \end{axis}
\end{tikzpicture}
\end{image}


In the graph above, every rectangle is touching the curve at its
right-endpoint.

\paragraph{Rectangles defined by midpoints}


\begin{image}
  \begin{tikzpicture}
	\begin{axis}[
            domain=-1.2:1.2, xmin =-1.2,xmax=1.2,ymax=2.2,ymin=-.2,
            axis lines=center, xlabel=$x$, ylabel=$y$,
            every axis y label/.style={at=(current axis.above origin),anchor=south},
            every axis x label/.style={at=(current axis.right of origin),anchor=west},
            %xtick={1,1.25,1.5,1.75,2}, ytickmin=4, ytickmax=1,
            axis on top,
          ]
          \foreach \rectnumber in {1,2,...,8}
                   {
                     \addplot [draw=penColor,fill=fillp]
                     plot coordinates {({-1+(\rectnumber - 1) * 2/8},{(-1+(\rectnumber-.5) * 2/8)^2+1} ) ({-1+(\rectnumber) * 2/8},{(-1+(\rectnumber -.5)* 2/8)^2+1 })} \closedcycle;
                   }
          \addplot [very thick,penColor, smooth] {x^2+1};
%          \addplot [color=penColor,fill=penColor,only marks,mark=*] coordinates{(1,1)};  %% closed hole         
%          \addplot [color=penColor,fill=penColor,only marks,mark=*] coordinates{(1.25,1.06)};  %% closed hole       
%          \addplot [color=penColor,fill=penColor,only marks,mark=*] coordinates{(1.5,1.25)};  %% closed hole  
%          \addplot [color=penColor,fill=penColor,only marks,mark=*] coordinates{(1.75,1.56)};  %% closed hole       
          
          %\node at (axis cs:1,1) [textColor,above] {$f(x_0^*)$};
          %\node at (axis cs:1.25,1.06) [textColor,above] {$f(x_1^*)$};
          %\node at (axis cs:1.5,1.3) [textColor,above] {$f(x_2^*)$};
          %\node at (axis cs:1.75,1.6) [textColor,above] {$f(x_3^*)$};
        \end{axis}
\end{tikzpicture}
\end{image}


In the graph above, every rectangle is touching the curve at its
midpoint.

\section{Computing these approximations}

Once we know how to identify our rectangles, we can compute some
approximate areas.

\begin{definition}
  When approximating an area with rectangles, a \dfn{sample point} is
  the $x$-coordinate that determines the relevant height of our
  rectangles. We denote a sample point as:
  \[
  x_k^*
  \]
\end{definition}

From this definition we see that if we are approximating area with $n$ rectangles, then 
\begin{align*}
  \text{Area} &\approx \sum_{k=1}^n (\text{width of $k$th rectangle})\times (\text{height of $k$th rectangle}) \\
  &=\sum_{k=1}^n \Delta x \times f(x_k^*) \\
  &= \Delta x \times f(x_1^*) + \Delta x \times f(x_2^*) +  \Delta x \times f(x_3^*) + \dots +  \Delta x \times f(x_n^*) 
\end{align*}



\paragraph{Rectangles defined by a left endpoint}

\begin{image}
  \begin{tikzpicture}
	\begin{axis}[
            domain=-1.2:1.2, xmin =-1.2,xmax=1.2,ymax=2.2,ymin=-.2,
            axis lines=center, xlabel=$x$, ylabel=$y$,
            every axis y label/.style={at=(current axis.above origin),anchor=south},
            every axis x label/.style={at=(current axis.right of origin),anchor=west},
            %xtick={1,1.25,1.5,1.75,2}, ytickmin=4, ytickmax=1,
            axis on top,
          ]
          \foreach \rectnumber in {1,2,...,8}
                   {
                     \addplot [draw=penColor,fill=fillp]
                     plot coordinates {({-1+(\rectnumber - 1) * 2/8},{(-1+(\rectnumber-1) * 2/8)^2+1} ) ({-1+(\rectnumber) * 2/8},{(-1+(\rectnumber-1) * 2/8)^2+1 })} \closedcycle;
                   }
          \addplot [very thick,penColor, smooth] {x^2+1};
%          \addplot [color=penColor,fill=penColor,only marks,mark=*] coordinates{(1,1)};  %% closed hole         
%          \addplot [color=penColor,fill=penColor,only marks,mark=*] coordinates{(1.25,1.06)};  %% closed hole       
%          \addplot [color=penColor,fill=penColor,only marks,mark=*] coordinates{(1.5,1.25)};  %% closed hole  
%          \addplot [color=penColor,fill=penColor,only marks,mark=*] coordinates{(1.75,1.56)};  %% closed hole       
          
          %\node at (axis cs:1,1) [textColor,above] {$f(x_0^*)$};
          %\node at (axis cs:1.25,1.06) [textColor,above] {$f(x_1^*)$};
          %\node at (axis cs:1.5,1.3) [textColor,above] {$f(x_2^*)$};
          %\node at (axis cs:1.75,1.6) [textColor,above] {$f(x_3^*)$};
        \end{axis}
\end{tikzpicture}
\end{image}


\[
\begin{array}{c|c|c|c}
  k &  x_k  & x^*_k & f(x^*_k) \\ \hline
  0 & -1    & \text{NA} & \text{NA}  \\
  1 & -0.75 & -1    &   2      \\
  2 & -0.5  & -0.75 & 1.5625   \\
  3 & -0.25 & -0.5  & 1.25     \\
  4 &  0    & -0.25 & 1.0625   \\
  5 &  0.25 &  0    & 1        \\
  6 &  0.5  &  0.25 & 1.0625   \\
  7 &  0.75 &  0.5  & 1.25     \\
  8 &  1    &  0.75 & 1.5625   
\end{array}
\]

\paragraph{Rectangles defined by a right endpoint}


\paragraph{Rectangles defined by a midpoint}

\section{Riemann sums}



\end{document}
