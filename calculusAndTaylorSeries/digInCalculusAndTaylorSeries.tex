\documentclass{ximera}

\newcommand{\RR}{\mathbb R}
\renewcommand{\d}{\,d}
\newcommand{\dd}[2][]{\frac{d #1}{d #2}}
\renewcommand{\l}{\ell}
\newcommand{\ddx}{\frac{d}{dx}}
\newcommand{\dfn}{\textbf}
\newcommand{\eval}[1]{\bigg[ #1 \bigg]}


\outcome{Use Taylor series to read-off derivatives of a function.}
\outcome{Use Taylor series to solve differential equations.}
\outcome{Use Taylor series to compute integrals.}

\title[Dig-In:]{Calculus and Taylor series}

\begin{document}
\begin{abstract}
  Power series interact nicely with other calculus concepts.
\end{abstract}
\maketitle

\section{Reading derivatives from Taylor series}

Recall that if $f(x)$ has derivatives of all orders at $x=c$, then the
Taylor series centered at $c$ for $f$ is
\[
\sum_{n=0}^\infty \frac{f^{(n)}(c)}{n!}(x-c)^n.
\]
On the other hand, suppose we give you a series, that we claim is the
Taylor series for a function $f$. Given just the series, you can
quickly evaluate $f(c)$, $f'(c)$, $f''(c)$, \dots, and so on. Let's
see an example.

\begin{example}
  The Taylor series for $f(x) = \arctan(x)$ is
  \[
  \sum_{n=0}^\infty \frac{(-1)^n x^{2n+1}}{2n+1} = x - \frac{x^3}{3} + \frac{x^5}{5} - \frac{x^7}{7} +\cdots.
  \]
  Compute $f'(0)$, $f'''(0)$, $f^{(5)}(0)$, $f^{(7)}(0)$, and give a
  general description $f^{(n)}(0)$.
  \begin{explanation}
    With the Taylor series in hand, we can simply look at the
    coefficients and ``read-off'' the derivatives we need. Comparing this series
    \[
    x - \frac{x^3}{3} + \frac{x^5}{5} - \frac{x^7}{7} +\cdots
    \]
    to the general form of a Taylor (Maclaurin) series, we see
    \begin{align*}
    f'(0) &= \answer[given]{1}\\
    f'''(0) &= \answer[given]{-2}\\
    f^{(5)}(0) &= \answer[given]{4!}\\
    f^{(7)}(0) &= \answer[given]{-6!}
    \end{align*}
    In general, if $n$ is even,
    \[
    f^{(n)} = 0
    \]
    and if $n$ is odd,
    \[
    f^{(n)} = \pm(n-1)! 
    \]
    where the sign alternates, starting with $+$.
  \end{explanation}
\end{example}

\begin{question}
  Let
  \[
  -1 + x - 2x^2 + 3x^3 - 5x^4 + 8x^5-13x^6+21x^7 -34x^8 + \cdots  
  \]
  be the Taylor series for some function. What are the values of the
  function and the first eight derivatives when evaluated at zero?
  \begin{hint}
    You shouldn't be using any derivative rules, instead, you should
    just ``read-off'' the derivatives from the series.
  \end{hint}
  \begin{prompt}
    \begin{align*}
      f(0) &= \answer{-1}\\
      f'(0) &= \answer{1}\\
      f''(0) &= \answer{-4}\\
      f'''(0) &= \answer{18}\\
      f^{(4)}(0) &= \answer{-120}\\
      f^{(5)}(0) &= \answer{960}\\
      f^{(6)}(0) &= \answer{-9360}\\
      f^{(7)}(0) &= \answer{105840}\\
      f^{(8)}(0) &= \answer{-1370880}
    \end{align*}
  \end{prompt}
\end{question}




\section{Solving differential equations using power series}

If we have a differential equation we can frequently use Taylor series
to obtain an approximate solution, which will be (hopefully) converge
on some interval.

In his study of optics, George Biddell Airy developed the so-called
\textit{Airy function}, a function that solves the differential
equation
\[
y'' = xy
\]
for initial conditions $y(0) = a_0$, and $y'(0) = a_1$.  As innocent as
this differential equation seems, it is impossible to find a closed
form solution! Nevertheless, Taylor series will rescue us.

\begin{example}
  Approximate a solution to the differential equation
  \[
  y'' = x\cdot y
  \]
  where $y(0) = 1$ and $y'(0) = -1$, using the first $6$ nonzero terms
  of a Maclaurin series.
  \begin{explanation}
    Differential equations give us a recipe for computing
    derivatives. We'll use this to produce a Maclaurin series
    representing the solution of this differential equation.  Start by
    writing down:
    \begin{align*}
    y(0) &= \answer[given]{1}\\
    y'(0) &= \answer[given]{-1}
    \end{align*}
    To compute higher derivatives, simply differentiate the
    differential equation:
    \[
    \begin{array}{lcl}
      y'' = x\cdot y & \Rightarrow & y''(0) = \answer[given]{0}\\
      y''' = y + x\cdot y' & \Rightarrow & y'''(0) = \answer[given]{1}\\
      y^{(4)} = 2y' + x\cdot y'' & \Rightarrow & y^{(4)}(0) = \answer[given]{-2}\\
      y^{(5)} = 3y'' + x\cdot y''' & \Rightarrow & y^{(5)}(0) =\answer[given]{0}\\
      y^{(6)} = 4y''' + x\cdot y^{(4)} & \Rightarrow & y^{(6)}(0) =\answer[given]{4}\\
      y^{(7)} = 5y^{(4)} + x\cdot y^{(5)} & \Rightarrow & y^{(7)}(0) =\answer[given]{-10}
    \end{array}
    \]
    We can immediately write down the first $5$ terms of the Maclaurin series
    \[
    1 -x + \frac{x^3}{3!} -\frac{2x^4}{4!} + \frac{4x^6}{6!} -\frac{10x^7}{7!} +\cdots
    \]
  \end{explanation}
\end{example}

\section{Integration}

Just as we can differentiate term by term, we can also integrate term
by term. This allows us to approximate many functions where we cannot
find a ``closed-form'' formula. Recall the following theorem:

\begin{theorem}[Derivatives and Indefinite Integrals of Power Series]\index{power series!derivatives and integrals}
  Let
  \[
  f(x) = \sum_{n=0}^\infty a_n(x-c)^n
  \]
  be a function defined by a power series, with radius of convergence $R$.
  \begin{itemize}
  \item $f(x)$ is continuous and differentiable on $(c-R,c+R)$.
  \item	$f'(x) = \sum_{n=1}^\infty a_n\cdot n\cdot (x-c)^{n-1}$, with radius of convergence $R$.
  \item	$\int f(x) \d x = C+\sum_{n=0}^\infty a_n\frac{(x-c)^{n+1}}{n+1}$, with radius of convergence $R$.
  \end{itemize}
\end{theorem}
%% https://gowers.wordpress.com/2014/02/22/differentiating-power-series/
A few notes about the theorem above:
\begin{itemize}
\item The theorem states that differentiation and integration do not
  change the radius of convergence. It does not state anything about
  the \textit{interval} of convergence. They are not always the same.
\item Notice how the summation for $f'(x)$ starts with $n=1$. This is
  because the constant term $a_0$ of $f(x)$ goes to $0$.
\item Differentiation and integration are simply calculated
  term-by-term using the power rule.
\end{itemize}

We'll use this idea to investigate the function 
\[
\si(x) = \int_0^x \frac{\sin(t)}{t} \d t
\]
an important function in signal analysis.

\begin{example}
  Find a power series for $\si(x)$ centered at zero and give the
  radius and interval of convergence.
\begin{explanation}
  We can find a series expression for $\sin(x)/x$ using division, write with me
  \[
  \sin(x) = x-\frac{x^3}{3!}+\frac{x^5}{5!}-\dots = \sum_{n=0}^\infty \frac{\answer[given]{(-1)^n x^{2n+1}}}{(2n+1)!} 
  \]
  So 
  \[
  \frac{\sin(x)}{x} =1-\frac{x^2}{3!}+\frac{x^4}{5!}-\dots = \sum_{n=0}^\infty \frac{\answer[given]{(-1)^n x^{2n}}}{(2n+1)!} 
  \]
  Now we integrate term-by-term to find $\si(x)$:
  \begin{align*}
  \int &\frac{\sin(x)}{x} \d x = C+x-\frac{x^3}{3 \cdot 3!}+\frac{x^5}{5 \cdot 5!}-\cdots \\
  &= \sum_{n=0}^\infty \frac{\answer[given]{(-1)^nx^{2n+1}}}{\answer[given]{(2n+1)} \cdot (2n+1)!} 
  \end{align*}
  $\si(x)$ is the antiderivative of $f(x)$ with $\si(0) = 0$, so  $C=0$, and 
  \[
  \si(x) = \sum_{n=0}^\infty \frac{(-1)^nx^{2n+1}}{(2n+1) \cdot (2n+1)!}.
  \]
  To find the radius of convergence, use the ratio test. Write with me
  \begin{align*}
    \lim_{n\to\infty} &\frac{x^{2(n+1)+1}}{(2(n+1)+1) \cdot (2(n+1)+1)!} \cdot \frac{(2n+1) \cdot (2n+1)!}{x^{2n+1}}\\
    &=\lim_{n\to\infty}\frac{x^{2n+3}}{(2n+3) \cdot (2n+3)!} \cdot \frac{(2n+1) \cdot (2n+1)!}{x^{2n+1}}\\
    &=\lim_{n\to\infty}\frac{x^{2}\cdot (2n+1)}{(2n+3) \cdot (2n+3)\cdot(2n+2)}
  \end{align*}
  and for any fixed $x$ this limit is zero. Hence our series converges
  on $(-\infty,\infty)$, with radius of convergence $R=\infty$.
 \end{explanation}
\end{example}


\begin{example}
  If
  \[
  \si(x) = \sum_{n=0}^\infty \frac{(-1)^n x^{2n+1}}{(2n+1) \cdot (2n+1)!} 
  \]
  which $n$ is needed to approximate $\si(2)$ with an accuracy
  of $\frac{1}{10}$?
  \begin{explanation}
    By the alternating series estimation test, we would need
    \begin{align*}
    \frac{2^{2(n+1)+1}}{(2(n+1)+1) \cdot (2(n+1)+1)!}  <  \frac{1}{10}\\
    \frac{ 2^{2n+3}}{(2n+3) \cdot (2n+3)!} &< \frac{1}{10}
    \end{align*}
    We can experiment a bit to find that the least integer which makes this true is 
    \[
    n= \answer[given]{1}
    \]
    Thus $\si(2) \approx 2-\frac{8}{3 \cdot 3!} = 1.\bar{5}$ should be
    accurate to within $\frac{1}{10}$.  Wolfram alpha reports that $Si(2)
    \approx 1.60541 $, we this appears to be accurate!
  \end{explanation}
\end{example}



\end{document}
