\documentclass{ximera}

\newcommand{\RR}{\mathbb R}
\renewcommand{\d}{\,d}
\newcommand{\dd}[2][]{\frac{d #1}{d #2}}
\renewcommand{\l}{\ell}
\newcommand{\ddx}{\frac{d}{dx}}
\newcommand{\dfn}{\textbf}
\newcommand{\eval}[1]{\bigg[ #1 \bigg]}


\title[Dig-In:]{Divergence theorem}

\outcome{State and use the divergence theorem.}

\begin{document}
\begin{abstract}
  We introduce the divergence theorem.
\end{abstract}
\maketitle




\section{The divergence theorem}

The divergence theorem states that certain volume integrals are equal
to certain surface integrals. Let's see the statement.

\begin{theorem}[Divergence Theorem]\index{divergence theorem}
  Suppose that the components of $\vec{F}:\R^3\to\R^3$ have continuous
  partial derivatives. If $R$ is a solid bounded by a surface
  $\partial R$ oriented with the normal vectors pointing outside, then:
  \[
  \iiint_R \divergence \vec{F}  \d V =   \oiint_{\partial R} \vec{F}\dotp\uvec{n}\d S
  \]
\end{theorem}

Integrals of the type above arise any time we wish to understand
``fluid flow'' through a surface.  The ``fluid'' in question could be
a real fluid like air or water, or it could be an electormagnetic
field, or something else entirely. Unfortunately, many of the ``real''
applications of the divergence theorem require a deeper understand of
the specific context that the integral aries. For our part, we will
focus on using the divergence theorm as a tool for transforming one
intgeral into another (hopefully easier!) integral.


\begin{example}
  Let
  \[
  R = \{(x,y,z):-2\le x\le1, 1\le y\le 3, -5\le z\le 1\}
  \]
  and
  \[
  \vec{F} = \vector{\sin(z),y^2,e^x},
  \]
  compute:
  \[
  \oiint_{\partial R} \vec{F}\dotp \uvec{n} \d S
  \]
  \begin{explanation}
    To compute this integral, we'll use the divergence theorem. Since
    our region is a box, the limits of the triple integral will be
    easy to work with. Moreover, computing the divergence of $\vec{F}$ we see
    \[
    \divergence \vec{F}(x,y,z) = \answer[given]{2y}.
    \]
    So by the divergence theorem, we have
    \begin{align*}
      \oiint_{\partial R} \vec{F}\dotp \uvec{n} \d S &= \iiint_R \divergence \vec{F}  \d V \\
      &= \int_{-2}^{\answer[given]{1}} \int_{-5}^{\answer[given]{1}} \int_{1}^{\answer[given]{3}}
      \answer[given]{2y} \d \answer[given]{y}\d \answer[given]{z}\d \answer[given]{x}\\
      &= \answer[given]{144}.
    \end{align*}
  \end{explanation}
\end{example}

Now let's see another example:



\begin{example}
  Let
  \begin{align*}
  R = \{(x,y,z):&0\le x\le1, \\
  &0\le y\le 2-2x, \\
  &0\le z\le 3-3x-3y/2\}
  \end{align*}
  and
  \[
  \vec{F} = \vector{x,y,z},
  \]
  compute:
  \[
  \oiint_{\partial R} \vec{F}\dotp \uvec{n} \d S
  \]
  \begin{explanation}
    To compute this integral, we'll use the divergence theorem. This
    time our region is a tetrahedron, we'll work in cartesian
    coordiantes. Moreover, computing the divergence of $\vec{F}$ we
    see
    \[
    \divergence \vec{F}(x,y,z) = \answer[given]{3}.
    \]
    So by the divergence theorem, we have
    \begin{align*}
      \oiint_{\partial R} \vec{F}\dotp \uvec{n} \d S &= \iiint_R \divergence \vec{F}  \d V \\
      &= \int_{\answer[given]{0}}^{\answer[given]{1}} \int_{\answer[given]{0}}^{\answer[given]{2-2x}} \int_{\answer[given]{0}}^{\answer[given]{3-3x-3y/2}}
      \answer[given]{3} \d z \d y \d x\\
      &= \answer[given]{3}.
    \end{align*}
  \end{explanation}
\end{example}

With our next two examples, we cannot help but flex our mathematical muscles a bit:


\begin{example}
  Let
  \begin{align*}
    R = \{(x,y,z):&-1\le x\le1, \\
    &1\le y\le 1,\\
    &0\le z\le \sqrt{1-x^2-y^2}\}
  \end{align*}
  and
  \[
  \vec{F} = \vector{x,2y,3z},
  \]
  compute:
  \[
  \oiint_{\partial R} \vec{F}\dotp \uvec{n} \d S
  \]
  \begin{explanation}
    To compute this integral, we'll use the divergence theorem. Since
    our region is the top of a sphere, the limits of the triple
    integral will be easy to work with in spherical
    coordinates. Moreover, computing the divergence of $\vec{F}$ we
    see
    \[
    \divergence \vec{F}(x,y,z) = \answer[given]{6}.
    \]
    So by the divergence theorem, we have
    \begin{align*}
      \oiint_{\partial R} \vec{F}\dotp \uvec{n} \d S &= \iiint_R \divergence \vec{F}  \d V \\
      &= \int_{\answer[given]{0}}^{\answer[given]{2\pi}} \int_{\answer[given]{0}}^{\answer[given]{\pi/2}} \int_{\answer[given]{0}}^{\answer[given]{1}}
      \answer[given]{6} \rho^2\sin(\phi)\d\rho\d\phi\d\theta\\
      &= \answer[given]{4\pi}.
    \end{align*}
  \end{explanation}
\end{example}











\section{A new fundamental theorem of calculus}

How is the divergence theorem a fundamental theorem of calculus? Well
consider this:
\begin{image}
  \begin{tikzpicture}
    \draw[ultra thick, gray!50!black] plot [smooth cycle] coordinates {(-1.5,2) (.5,1) (1.5,2) (.5,3) (-1.5,3)};
    \shade[ball color=gray!50!white] plot [smooth cycle] coordinates {(-1.5,2) (.5,1) (1.5,2) (.5,3) (-1.5,3)};
    \node[inner sep=0pt] at (0,0) {$\iiint_R \divergence\vec{F}\d V\quad =\quad \oiint_{\partial R} \vec{F}\dotp\uvec{n}\d S$};

    \node at (-1.7,-.7) {$\underbrace{\hspace{7em}}$};

    \node at (1.7,-.7) {$\underbrace{\hspace{6.5em}}$};

    \node[below,inner sep=0pt,text width=4cm,scale=.5] at (-1.7,-1)
         {To compute the triple integral of $\divergence \vec{F}$ over a
           solid $V\subseteq\R^3$, };

    \node[below,inner sep=0pt,text width=4cm,scale=.5] at (1.8,-1) {we can compute the accumulation of $\vec{F}$ across a boundary surface $\partial R$.};
  \end{tikzpicture}
\end{image}


Are there more fundamental theorems of calculus? Absolutely, and we're
ready for the last one of this course. Read on young mathematician!



\end{document}
