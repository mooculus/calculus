\documentclass{ximera}

\newcommand{\RR}{\mathbb R}
\renewcommand{\d}{\,d}
\newcommand{\dd}[2][]{\frac{d #1}{d #2}}
\renewcommand{\l}{\ell}
\newcommand{\ddx}{\frac{d}{dx}}
\newcommand{\dfn}{\textbf}
\newcommand{\eval}[1]{\bigg[ #1 \bigg]}


\outcome{Use ``shortcut'' rules to find derivatives}


\title[Break-Ground:]{Sine is the key}

\begin{document}
\begin{abstract}
Two young mathematicians think about derivatives of trigonometric functions.
\end{abstract}
\maketitle

Check out this dialogue between two calculus students (based on a true
story):

\begin{dialogue}
\item[Devyn] Riley! I think we know more derivatives than we currently know.
\item[Riley] Oh you must tell me what you are thinking.
\item[Devyn] Well, it seems to me that almost all trig functions are
  basically built from sine.
\item[Riley] Right,
  \[
  \tan(x) = \frac{\sin(x)}{\cos(x)}
  \]
  and I'm thinking for everything else, we can use a triangle:
  
  
\end{dialogue}

The pattern
\[
  \text{if} \qquad f(x) = x^n\quad\text{then}\qquad f'(x) = n\cdot x^{n-1}
\]
holds whenever $n$ is a constant. Explaining why it works in
generality will take some time. For now, let's see if we can use the
problem to squash some derivatives with ease.

\begin{problem}
  Using the pattern found above, compute:
  \[
  \ddx x^{101} \begin{prompt}= \answer{101 x^{100}}\end{prompt}
  \]
\end{problem}

\begin{problem}
  Using the pattern found above, compute:
  \[
  \ddx \frac{1}{x^{77}} \begin{prompt}= \answer{-77 x^{-78}}\end{prompt}
  \]
\end{problem}


\begin{problem}
  Using the pattern found above, compute:
  \[
  \ddx \sqrt[5]{x} \begin{prompt}= \answer{x^{-4/5}/5}\end{prompt}
  \]
\end{problem}

\begin{problem}
  Using the pattern found above, compute:
  \[
  \ddx x^e  \begin{prompt}= \answer{e x^{e-1}}\end{prompt}
  \]
\end{problem}

%% \begin{xarmaBoost}
%%   Write down at least \textbf{five} questions for this lecture. After
%%   you have your questions, label them as ``Level 1,'' ``Level 2,'' or
%%   ``Level 3'' where:
%% \begin{description}
%% \item[Level 1] Means you know the answer, or know exactly how to do
%%   this problem.
%% \item[Level 2] Means you think you know how to do the problem.
%% \item[Level 3] Means you have no idea how to do the problem.
%% \end{description}
%% \begin{freeResponse}
%% \end{freeResponse}
%% \end{xarmaBoost}


\end{document}
