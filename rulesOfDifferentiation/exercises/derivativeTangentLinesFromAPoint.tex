\documentclass{ximera}

\newcommand{\RR}{\mathbb R}
\renewcommand{\d}{\,d}
\newcommand{\dd}[2][]{\frac{d #1}{d #2}}
\renewcommand{\l}{\ell}
\newcommand{\ddx}{\frac{d}{dx}}
\newcommand{\dfn}{\textbf}
\newcommand{\eval}[1]{\bigg[ #1 \bigg]}

\outcome{Equation of the tangent line.}
\outcome{Finding an equation of the tangent line from a given point to the curve.}

\author{Nela Lakos}

\begin{document}
\begin{exercise}

The figure below shows the graphs of $f$, where $f(x)=-x^2$ and the point $P(0,3)$.  Find the equations of all tangent lines to the curve $y=f(x)$ that pass through the point $P$.


\begin{image}
  \begin{tikzpicture}
    \begin{axis}[
        xmin=-4.3,xmax=4.3,ymin=-4.3,ymax=4.3,
        clip=true,
        unit vector ratio*=1 1 1,
        axis lines=center,
        grid = major,
      ytick={-4,-3,...,4},
    xtick={-4,-3,...,4},
        xlabel=$x$, ylabel=$y$,
        every axis y label/.style={at=(current axis.above origin),anchor=south},
        every axis x label/.style={at=(current axis.right of origin),anchor=west},
      ]
      \addplot[thick,blue,domain=-2.3:2.3,samples=50] plot{-(x)^2};
        \addplot[penColor,only marks,mark=*] coordinates{(0,3)};
      \node at (axis cs:2.8,-3) [blue] {$y=f(x)$};
       \node at (axis cs:0.4,3) [blue] {$P$};
      
      \end{axis}`
  \end{tikzpicture}
\end{image}

We will solve this problem by completing several steps.

STEP 1

Any such tangent line will pass through the point $P(0,3)$, and will have a slope, call it $m$. Therefore, its equation is given by
\[
y-\answer{3}=m(x-0)
\]
STEP 2

Now we have to find an expression for the slope $m$.
\begin{image}
  \begin{tikzpicture}
    \begin{axis}[
        xmin=-4.3,xmax=4.3,ymin=-4.3,ymax=4.3,
        clip=true,
        unit vector ratio*=1 1 1,
        axis lines=center,
        grid = major,
      ytick={-4,-3,...,4},
    xtick={-4,-3,...,4},
        xlabel=$x$, ylabel=$y$,
        every axis y label/.style={at=(current axis.above origin),anchor=south},
        every axis x label/.style={at=(current axis.right of origin),anchor=west},
      ]
      \addplot[thick,blue,domain=-2.3:2.3,samples=50] plot{-(x)^2};
       %\addplot[thin,red,domain=-2.3:2.3,samples=50] plot{3-2*sqrt(3)*x};
        \addplot[thin,red,domain=-2.3:2.3,samples=50] plot{3+2*sqrt(3)*x};
        \addplot[penColor,only marks,mark=*] coordinates{(0,3)};
        % \addplot[penColor2,only marks,mark=*] coordinates{(sqrt(3),-3)};
          \addplot[penColor2,only marks,mark=*] coordinates{(-sqrt(3),-3)};
      \node at (axis cs:2.8,-3) [blue] {$y=f(x)$};
       \node at (axis cs:-2.8,-3) [red] {$(a,-a^2)$};
        \node at (axis cs:0.4,3) [blue] {$P$};
      
      \end{axis}`
  \end{tikzpicture}
\end{image}
 Since the line is a tangent line, it will intersect the curve $y=f(x)=-x^2$ at some point, say $(a,f(a))=(a,-a^2)$, and , therefore, the slope $m$ is given by
\begin{align*}
m &=f'(a)\\
 m&= \answer{-2a}\\
\end{align*}



Therefore, an equation of a tangent line to the curve $y=f(x)$ that passes through the point $P(0,3)$ is given by
\[
y-3=\answer{-2a}x,
\]
where $a$ is some real number.

We have to determine this real number $a$!

Observe that the point $(a,-a^2)$ lies on both the tangent line and on the curve $y=f(x)$. Therefore, the point $(a,-a^2)$ satisfies the equation of the tangent line!
Plugging in $x=a$ and $y=-a^2$ into the equation of the line, we obtain a quadratic equation which can be solved for $a$
\[
-a^2-3=\answer{-2a}a
\]



Solving the quadratic equation, we obtain  two solutions (written in increasing order)


\[
a=\answer{-\sqrt{3}}\text {   and    } a=\answer{\sqrt{3}}
\]

This means  that there are two tangent lines passing through the point $P$, (written in increasing order)

 one with the slope $m=\answer{-2\sqrt{3}}$ and the other with the slope $m=\answer{2\sqrt{3}}$.
 
Check the picture!
\begin{image}
  \begin{tikzpicture}
    \begin{axis}[
        xmin=-4.3,xmax=4.3,ymin=-4.3,ymax=4.3,
        clip=true,
        unit vector ratio*=1 1 1,
        axis lines=center,
        grid = major,
      ytick={-4,-3,...,4},
    xtick={-4,-3,...,4},
        xlabel=$x$, ylabel=$y$,
        every axis y label/.style={at=(current axis.above origin),anchor=south},
        every axis x label/.style={at=(current axis.right of origin),anchor=west},
      ]
      \addplot[thick,blue,domain=-2.3:2.3,samples=50] plot{-(x)^2};
       \addplot[thin,red,domain=-2.3:2.3,samples=50] plot{3-2*sqrt(3)*x};
        \addplot[thin,red,domain=-2.3:2.3,samples=50] plot{3+2*sqrt(3)*x};
        \addplot[penColor,only marks,mark=*] coordinates{(0,3)};
         \addplot[penColor2,only marks,mark=*] coordinates{(sqrt(3),-3)};
          \addplot[penColor2,only marks,mark=*] coordinates{(-sqrt(3),-3)};
      \node at (axis cs:2.8,-3) [red] {\tiny$(\sqrt{3},-3)$};
       \node at (axis cs:-2.8,-3) [red] {\tiny$(-\sqrt{3},-3)$};
        \node at (axis cs:0.4,3) [blue] {$P$};
      
      \end{axis}`
  \end{tikzpicture}
\end{image}
There are exactly two lines that  are both tangent to the curve $y=f(x)$ and pass through the point $P$. The first line has a positive slope, and the second line has the negative slope, and their equations, written in that order, are:
\begin{align*}
y &= \answer{2\sqrt{3}x+3}\\ 
y &= \answer{-2\sqrt{3}x+3}\\
\end{align*}
\end{exercise}
\end{document}