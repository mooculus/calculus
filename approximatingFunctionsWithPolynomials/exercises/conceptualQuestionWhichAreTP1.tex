\documentclass{ximera}

\newcommand{\RR}{\mathbb R}
\renewcommand{\d}{\,d}
\newcommand{\dd}[2][]{\frac{d #1}{d #2}}
\renewcommand{\l}{\ell}
\newcommand{\ddx}{\frac{d}{dx}}
\newcommand{\dfn}{\textbf}
\newcommand{\eval}[1]{\bigg[ #1 \bigg]}


\author{Jim Talamo}
\license{Creative Commons 3.0 By-bC}


\outcome{Understand the relationship between a Taylor Polynomials and the function it represents}


\begin{document}
\begin{exercise}
Given that $f(x)$ is an infinitely differentiable function at $x=2$ and that $f''(2) =4$, which of the following polynomials is a possibility for the third degree Taylor Polynomial for $f(x)$ centered at $x=2$?

\begin{selectAll}
\choice{$p_3(x) = 2-3(x-2)+4(x-2)^2-3(x-2)^3$}
\choice{$p_3(x) = 3-5(x-2)-4(x-2)^2+4(x-2)^3$}
\choice[correct]{$p_3(x) = 4(x-2)+2(x-2)^2+5(x-2)^3$}
\choice{$p_3(x) = 6-3(x-2)-2(x-2)^2-(x-2)^3$}
\choice[correct]{$p_3(x) = 2+7(x-2)+2(x-2)^2+9(x-2)^3$}
\end{selectAll}

\begin{hint}
The Taylor Polynomial in question will be of the form:

\[
p_3(x) = a_0+a_1(x-2)+a_2(x-2)^2+a_3(x-2)^3
\]

To attack this, note that since we only have information about $f''(2)$, we can only determine the coefficient:

\begin{multipleChoice}
\choice{$a_0$}
\choice{$a_1$}
\choice[correct]{$a_2$}
\choice{$a_3$}
\end{multipleChoice}
\end{hint}

Since the coefficients and the values of the derivatives of the function are related via the formula:

\[
a_k = \frac{f^{(k)}(2)}{k!}
\]
We find that $a_2$ must be $\answer{2}$.  
\end{exercise}
\end{document}
