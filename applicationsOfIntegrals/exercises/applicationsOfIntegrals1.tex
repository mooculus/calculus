\documentclass{ximera}

\newcommand{\RR}{\mathbb R}
\renewcommand{\d}{\,d}
\newcommand{\dd}[2][]{\frac{d #1}{d #2}}
\renewcommand{\l}{\ell}
\newcommand{\ddx}{\frac{d}{dx}}
\newcommand{\dfn}{\textbf}
\newcommand{\eval}[1]{\bigg[ #1 \bigg]}


%\outcome{Given a velocity function, calculate displacement and distance traveled.}
%\outcome{Given a velocity function, find the position function.}
%\outcome{Given an acceleration function, find the velocity function.}
%\outcome{Understand the difference between displacement and distance traveled.}
%\outcome{Understand the relationship between position, velocity and acceleration.}

\author{Nela Lakos \and Kyle Parsons}

\begin{document}
\begin{exercise}

Consider a particle moving along a line.  Its acceleration and initial velocity are given by
\begin{align*}
a(t) &= 2t-4\\
v(0) &= 3
\end{align*}
for $0\leq t\leq4$.

The velocity of the particle at time $t$ is
\[
v(t) = \answer{t^2-4t+3}.
\]

The total \emph{distance} the particle travels on the interval $[0,4]$ is
\[
\int_0^4 \left|v(t)\right| \d t = \answer{4}.
\]
\begin{hint}
Note
\[
v(t)=(t-1)(t-3).
\]
Therefore $v$ is negative on the interval $(1,3)$.

\end{hint}
\begin{hint}
We have to compute the integral
\[
\int_0^4 \left|v(t)\right| \d t =\int_0^1 \left|v(t)\right| \d t+\int_1^3 \left|v(t)\right| \d t+\int_3^4 \left|v(t)\right| \d t.
\]
Therefore,
\[
\int_0^4 \left|v(t)\right| \d t =\int_0^1 v(t) \d t-\int_1^3v(t)\d t+\int_3^4 v(t) \d t.
\]

\end{hint}
\end{exercise}
\end{document}