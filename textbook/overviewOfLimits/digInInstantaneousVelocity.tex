\documentclass{ximera}

\newcommand{\RR}{\mathbb R}
\renewcommand{\d}{\,d}
\newcommand{\dd}[2][]{\frac{d #1}{d #2}}
\renewcommand{\l}{\ell}
\newcommand{\ddx}{\frac{d}{dx}}
\newcommand{\dfn}{\textbf}
\newcommand{\eval}[1]{\bigg[ #1 \bigg]}


\title[Dig-In:]{Instantaneous velocity}

\begin{document}
\begin{abstract}
\end{abstract}
\maketitle

When one computes average velocity, we look at 
\[
\frac{\text{change in displacement}}{\text{change in time}}.
\]
To obtain the (instantaneous) velocity, we want the change in time to
``go to'' zero. By this point we should know that ``go to'' is a
buzz-word for a \textit{limit}. The change in time is often given as
an interval whose length goes to zero.  However, intervals must always
be written
\[
[a,b] \qquad\text{where $a < b$.}
\]


Given $I = [a, a+h]$, we see that $h$ cannot be
negative, or else it violates the notation for intervals. Hence, if we
want smaller, and smaller, intervals around a point $a$, and we want
$h$ to be able to be negative, we write
\[
I_h = 
\begin{cases}
  [a+h,a]  & \text{if $h<0$}, \\ %% note this is MORE correct than std books
  [a,a+h]  & \text{if $0<h$}.     %% in the content section, we can explain this in detail
\end{cases}
\]
\todo{An interactive for this interval would be nice.}
Regardless of the value of $h$, the average velocity is computed by
\[
\frac{\text{change in displacement}}{\text{change in time}} =
\frac{s(t+h)-s(t)}{h}.
\]

Let's put this together by working an example.

\begin{example}
The \textit{MOOCulus Team} recently took a road trip from Columbus
Ohio to Urbana-Champaign Illinois in the \textit{MOOCulus-Mobile}. The
position of the MOOCulus-Mobile is roughly modeled by
\[
s(t) = 36t^2 - 4.8t^3 \qquad\text{(miles West of Columbus)} %% note the model is wrong
\]
on the interval $[0,5]$, where $t$ is measured in hours. The average
velocity on the interval $[0,5]$ is $\answer[given]{60}$ miles per hour. On the
other hand, consider the interval
\[
I_h = 
\begin{cases}
  [1+h,1]  & \text{if $h<0$}, \\ %% note this is MORE correct than std books
  [1,1+h]  & \text{if $0<h$}.     %% in the content section, we can explain this in detail
\end{cases}
\]
When $h = 0.1$, the average velocity is $\answer[given]{59.712}$. On the other
hand, when $h=-0.1$, the average velocity is $\answer[given]{55.392}$.
\end{example}


By using limits we may now work with instantaneous velocity directy.


Recall that we started this discussion by thinking about the
\textit{MOOCulus-Mobile}.

\begin{example}
  If the position west of Columbus of the MOOCulus-Mobile is modeled by 
  \[
  s(t) = 36t^2 - 4.8t^3 \qquad \text{for $0\le t\le 5$,}
  \] 
  how do we find a formula for the velocity of the MOOCulus-Mobile?
\begin{explanation}
  To solve this problem, we use the derivative. This is because the
  derivative computes the change in position with respect to time, and
  this is precisely the definition of velocity! Let's do this.

  Start by writing out the limit definition of the derivative where
  $s(t) = 36t^2 - 4.8t^3$.
  \begin{freeResponse}[given]
    \[
    \dd t s(t) = \lim_{h\to 0}\frac{36(t+h)^2 - 4.8(t+h)^3 - \left(36t^2 - 4.8t^3\right)}{h}
    \]
  \end{freeResponse}
  Now expand the numerator of the fraction.
  \begin{freeResponse}[given]
    \[
    \dd t s(t) = \lim_{h\to 0}\frac{36t^2 + 72th + 36h^2 - 4.8t^3 - 14.4t^2h - 14.4th^2 - 4.8h^3 - 36t^2 + 4.8t^3}{h}
    \]
  \end{freeResponse}
  Now combine like-terms.
  \begin{freeResponse}[given]
    \[
    \dd t s(t) = \lim_{h\to 0}\frac{72th + 36h^2 - 14.4t^2h - 14.4th^2 - 4.8h^3}{h}
    \]
  \end{freeResponse}
  Factor an $h$ from every term in the numerator.
  \begin{freeResponse}[given]
    \[
    \dd t s(t) = \lim_{h\to 0}\frac{h\left(72t + 36h - 14.4t^2 - 14.4th - 4.8h^2\right)}{h}
    \]
  \end{freeResponse}
  Cancel $h$ from the numerator and denominator.
  \begin{freeResponse}[given]
    \[
    \dd t s(t) = \lim_{h\to 0}\left(72t + 36h - 14.4t^2 - 14.4th - 4.8h^2\right)
    \]
  \end{freeResponse}
  Take the limit as $h$ goes to $0$. 
  \begin{freeResponse}[given]
    \[
    \dd t s(t) = 72t-14.4t^2
    \]
  \end{freeResponse}
  Behold the beauty of $s(t)$ and $s'(t)$:
  \begin{image}
    \begin{tikzpicture}
      \begin{groupplot}[group style={group size=2 by 1,horizontal sep=2cm}, width=0.5\textwidth]
        \nextgroupplot[
          clip=false,
          domain=0:5,
          xmax=5,
        xmin=0,
        ymax=300,
        ymin=0,
        axis lines =middle, 
        y label style={at={(axis description cs:-0.1,0.5)},rotate=90,anchor=south},
        x label style={at={(axis description cs:0.5,-.2)}},
        xlabel=hours into road-trip, ylabel=distance traveled in miles,
        ]
        \addplot [very thick, penColor, smooth] {36*x^2 - 4.8*x^3};
          \node at (axis cs:3,230) {\color{penColor}$s(t)$}; 
          \nextgroupplot[
            clip=false,
            domain=0:5,
            xmax=5,
            xmin=0,
            ymax=100,
            ymin=0,
            axis lines =middle, 
            y label style={at={(axis description cs:-0.1,0.5)},rotate=90,anchor=south},
            x label style={at={(axis description cs:0.5,-.2)}},
            xlabel=hours into road-trip, ylabel=velocity in miles per hour]
          \addplot [very thick, penColor2, smooth] {72*x-14.4*x^2};
          \node at (axis cs:3,95) {\color{penColor2}$s'(t)$};
      \end{groupplot}
    \end{tikzpicture}
  \end{image}
\end{explanation}
\end{example}







Now let's change gears a bit. Displacement and velocity are good
metaphors for concepts in calculus; however, the ``slope of a tangent line at a point'' is another powerful metaphore for concepts in calculus.



\end{document}
