\documentclass{ximera}

\newcommand{\RR}{\mathbb R}
\renewcommand{\d}{\,d}
\newcommand{\dd}[2][]{\frac{d #1}{d #2}}
\renewcommand{\l}{\ell}
\newcommand{\ddx}{\frac{d}{dx}}
\newcommand{\dfn}{\textbf}
\newcommand{\eval}[1]{\bigg[ #1 \bigg]}


\title[Dig-In:]{Unit tangent and unit normal vectors}

\outcome{Define normal vectors.}
\outcome{Define unit tangent and unit normal vectors.} 
\outcome{Define principal unit normal vectors.}

\begin{document}
\begin{abstract}
  We introduce two important unit vectors. 
\end{abstract}
\maketitle


Given a smooth vector-valued function $\vec{p}(t)$, \textit{any}
vector parallel to $\vec{p}'(t_0)$ is \textit{tangent} to the graph of
$\vec{p}(t)$ at $t=t_0$. It is often useful to consider just the
\textit{direction} of $\vec{p}'(t)$ and not its magnitude. Therefore we are
interested in the unit vector in the direction of $\vec{p}'(t)$. This
leads to a definition.
\begin{definition}
Let $\vec{p}(t)$ be a smooth function on an open interval $I$. The
\dfn{unit tangent vector} $\uvec{t}(t)$ is \index{unit tangent
  vector!definition} \index{unit vector!unit tangent vector}
\[
\utan(t) = \frac{\vec{p}'(t)}{|\vec{p}'(t)|}.
\]
\end{definition}

\begin{question}
  Let $\vec{p}(t) = \vector{3\cos(t), 3\sin(t), 4t}$. Find $\utan(t)$.
  \begin{prompt}
    \[
    \utan(t) = \vector{\answer{\frac{-3}{5}\sin(t)},\answer{\frac{3}{5}\cos(t)},\answer{\frac{4}{5}}}
    \]
    \begin{feedback}
      The unit tangent vector $\utan(t)$ \textbf{always} has a constant
      magnitude of $1$.
    \end{feedback}
  \end{prompt}
\end{question}

Just as knowing the direction tangent to a path is important, knowing
a direction orthogonal to a path is important. When dealing with
real-valued functions, one defined the \dfn{normal line} at a point to
the be the line through the point that was perpendicular to the
tangent line at that point. We can do a similar thing with
vector-valued functions. Given $\vec{p}(t)$ in $\R^2$, we have $2$
directions perpendicular to the tangent vector
\begin{image}
  \begin{tikzpicture}
    \begin{axis}%
      [width=175pt,tick label style={font=\scriptsize},axis on top,
	axis lines=center,
	view={115}{25},
	name=myplot,
	%xtick={-3,3},minor tick num=2,
	%ytick={-3,3},
	%ztick={-3,3},
	ymin=-3.5,ymax=3.5,
	xmin=-3.5,xmax=3.5,
	zmin=-15.9, zmax=15.9,
	every axis x label/.style={at={(axis cs:\pgfkeysvalueof{/pgfplots/xmax},0,0)},xshift=-3pt,yshift=-3pt},
	xlabel={\scriptsize $x$},
	every axis y label/.style={at={(axis cs:0,\pgfkeysvalueof{/pgfplots/ymax},0)},xshift=0pt,yshift=-5pt},
	ylabel={\scriptsize $y$},
	every axis z label/.style={at={(axis cs:0,0,\pgfkeysvalueof{/pgfplots/zmax})},xshift=0pt,yshift=4pt},
	zlabel={\scriptsize $z$}
      ]
      
      \addplot3[domain=-3.14:3.14,,thick,smooth,samples y=0,penColor,samples=30,] ({3*cos(x*180/3.14)},{3*sin(x*180/3.14)},{4*x});

      
      \draw[thick,->,penColor2] (axis cs: 3,0,0) -- (axis cs: 3,.6,.8);
      \draw[thick,->,penColor2] (axis cs: 1.6,2.5,4) -- (axis cs: 1.12,2.85,4.8);
    \end{axis}
  \end{tikzpicture}
\end{image}
The young mathematician wonders ``Is one of these two directions
preferable over the other?''  This question only gets harder in higher
dimensions.  Given $\vec{p}(t)$ in $\R^3$, there are infinite vectors
orthogonal to the tangent vector at a given point. Again, we might
wonder ``Is one of these infinite choices preferable over the others?
Is one of these the `right' choice?''

The answer in both $\R^2$ and $\R^3$ is ``Yes, there is one vector
that is preferable, and it is the `right' one to choose!'' Recall:
\begin{quote}
If $\vec{p}(t)$ has constant length, then $\vec{p}(t)$ is orthogonal
to $\vec{p}'(t)$ for all $t$.
\end{quote}
Since $\utan(t)$, the unit tangent vector, it necessarily has
constant length. Therefore
\[
\utan(t)\text{ is orthogonal to }\utan'(t).
\]

The vector-valued function $\utan'(t)$ is more than just a
convenient choice of vector that is orthogonal to $\vec{p}'(t)$;
rather, it is the ``right'' choice. We will use this to construct our
\textit{unit normal vector}:

\begin{definition}
Let $\vec{p}(t)$ be a vector-valued function where the unit tangent
vector, $\utan(t)$, is smooth on an open interval $I$. The
\dfn{unit normal vector} $\unormal(t)$ is \index{unit normal
  vector!definition}\index{unit vector!unit normal vector}
\[
\unormal(t) = \frac{\utan'(t)}{|\utan'(t)|}.
\]
Some folks call this the \dfn{principal unit normal vector}.
\end{definition}
\begin{warning}
  Even though $\utan(t)$ is a unit vector, this \textbf{does not}
  imply that $\utan'(t)$ is also a unit vector.
\end{warning}

\begin{question}
  Let $\vec{p}(t) = \vector{3\cos t, 3\sin t, 4t}$ as before. Find
  $\unormal(t)$.
  \begin{prompt}
    \[
    \unormal(t) =\vector{\answer{-\cos(t)},\answer{-\sin(t)},\answer{0}}.
    \]
    \begin{feedback}
      As a gesture of friendship, we present you with the following
      graph of the situation.
      \begin{image}
        \begin{tikzpicture}
          \begin{axis}%
            [width=175pt,tick label style={font=\scriptsize},axis on top,
	      axis lines=center,
	      view={135}{25},
	      name=myplot,
	      %xtick={-3,3},minor tick num=2,
	      %ytick={-3,3},
	      %ztick={-3,3},
	      ymin=-3.5,ymax=3.5,
	      xmin=-3.5,xmax=3.5,
	      zmin=-15.9, zmax=15.9,
	      every axis x label/.style={at={(axis cs:\pgfkeysvalueof{/pgfplots/xmax},0,0)},xshift=-3pt,yshift=-3pt},
	      xlabel={\scriptsize $x$},
	      every axis y label/.style={at={(axis cs:0,\pgfkeysvalueof{/pgfplots/ymax},0)},xshift=0pt,yshift=-5pt},
	      ylabel={\scriptsize $y$},
	      every axis z label/.style={at={(axis cs:0,0,\pgfkeysvalueof{/pgfplots/zmax})},xshift=0pt,yshift=4pt},
	      zlabel={\scriptsize $z$}
	    ]
            \addplot3[domain=-3.14:3.14,,thick,smooth,samples y=0,penColor,samples=30,] ({3*cos(x*180/3.14)},{3*sin(x*180/3.14)},{4*x});

            
            \draw[thick,->,penColor2] (axis cs: 0,3,6.28) -- (axis cs: -.6,3,7.08);
            \draw[thick,->,penColor2] (axis cs: 0,3,6.28) -- (axis cs: 0,2,6.28);
          \end{axis}
        \end{tikzpicture}
      \end{image}
    \end{feedback}
  \end{prompt}
\end{question}

\end{document}
