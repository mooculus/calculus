\documentclass[]{ximera}
%handout:  for handout version with no solutions or instructor notes
%handout,instructornotes:  for instructor version with just problems and notes, no solutions
%noinstructornotes:  shows only problem and solutions

%% handout
%% space
%% newpage
%% numbers
%% nooutcomes

%I added the commands here so that I would't have to keep looking them up
%\newcommand{\RR}{\mathbb R}
%\renewcommand{\d}{\,d}
%\newcommand{\dd}[2][]{\frac{d #1}{d #2}}
%\renewcommand{\l}{\ell}
%\newcommand{\ddx}{\frac{d}{dx}}
%\everymath{\displaystyle}
%\newcommand{\dfn}{\textbf}
%\newcommand{\eval}[1]{\bigg[ #1 \bigg]}

%\begin{image}
%\includegraphics[trim= 170 420 250 180]{Figure1.pdf}
%\end{image}

%add a ``.'' below when used in a specific directory.

\newcommand{\RR}{\mathbb R}
\renewcommand{\d}{\,d}
\newcommand{\dd}[2][]{\frac{d #1}{d #2}}
\renewcommand{\l}{\ell}
\newcommand{\ddx}{\frac{d}{dx}}
\newcommand{\dfn}{\textbf}
\newcommand{\eval}[1]{\bigg[ #1 \bigg]}




\author{Tom Needham}

\outcome{Use the divergence test to conclude that a series diverges.}
\outcome{Use the integral test to determine whether a series converges or diverges.}

\title[]{The Divergence Test and The Integral Test}

\begin{document}
\begin{abstract}
\end{abstract}
\maketitle

\vspace{-0.9in}

\section{Discussion Questions}

\begin{problem}
Determine whether the following statements are true or false, and explain your answer.

I. If the sequence $\{a_n\}$ satisfies $\lim_{n\rightarrow \infty} a_n = 0$, then the series $\sum_{n=1}^\infty a_n$ converges.

II. If the sequence $\{a_n\}$ satisfies $\lim_{n \rightarrow \infty} a_n \neq 0$, then the series $\sum_{n=1}^\infty a_n$ diverges.

III. If the series $\sum_{n=1}^\infty a_n$ converges, then $\lim_{n\rightarrow \infty} a_n = 0$. 

IV. If the series $\sum_{n=1}^\infty a_n$ diverges, then $\lim_{n\rightarrow \infty} a_n \neq 0$. 

\begin{solution}
I. This statement is false. For example, the harmonic series $\sum \frac{1}{n}$ diverges, while $\lim_{n \rightarrow \infty} \frac{1}{n} = 0$. 

II. This is The Divergence Test, and is therefore true.

III. This is the contrapositive of, and is therefore logically equivalent to, The Divergence Test, so it is true.

IV. This statement is the contrapositive of the statement in I., and is false.
\end{solution}

\end{problem}


\begin{problem}
Suppose that the series $\sum_{k=1}^\infty a_k$ has sequence of partial sums $\{s_n\}$ given by the explicit formula
$$
s_n = \frac{2n+1}{4n+3}.
$$
Student A claims that the series $\sum_{k=1}^\infty a_k$ converges because $\lim_{n \rightarrow \infty} s_n = \frac{1}{2}$. Student B claims that $\lim_{n\rightarrow \infty} s_n = \frac{1}{2} \neq 0$, so the divergence test implies that the series $\sum_{k=1}^\infty a_k$ diverges. Which (if either) student is correct?

\begin{solution}
Student A is correct, by the definition of convergence. Student B is incorrect; the divergence test deals with the limit of the sequence of  terms $a_k$, not the limit of the sequence of partial sums.
\end{solution}
\end{problem}

\section{Group Work}

\begin{problem}
Determine whether the series
$$
\sum_{k=1}^\infty \frac{k^3 - \ln(k+10) + k^5}{(2k+100)^5}
$$
converges or diverges.

\begin{solution}
Comparing the growth rates of the dominant terms in the numerator and denominator, we have 
$$
\lim_{k \rightarrow \infty} \frac{k^3 - \ln(k+10) + k^5}{(2k+100)^5} = \lim_{k\rightarrow \infty} \frac{k^5}{2^5 k^5} = \frac{1}{32} .
$$
Since the limit is nonzero, the series diverges by The Divergence Test.
\end{solution}
\end{problem}


\begin{problem}
Determine whether the series
$$
\sum_{k=2}^\infty \frac{1}{k \ln (k)}
$$
converges or diverges.

\begin{solution}
Since 
$$
\lim_{k \rightarrow \infty} \frac{1}{k \ln (k)} = 0,
$$
The Divergence Test is inconclusive, and we will attempt to solve the problem via The Integral Test. We compute the improper integral
\begin{align*}
\int_2^\infty \frac{1}{x \ln (x)} \d x
\end{align*}
by making the substitution $u = \ln (x)$, so that $\d u = \frac{1}{x} \d x$, so that 
\begin{align*}
\int_2^\infty \frac{1}{x \ln (x)} \d x &= \lim_{b \rightarrow \infty} \int_{\ln (2)}^b \frac{1}{u} \d u \\
&= \lim_{b \rightarrow \infty} \eval{\ln(u)}_{\ln(2)}^b \\
&= \lim_{b \rightarrow \infty} \ln(b) - \ln 2 \\
&= \infty.
\end{align*}
The Integral Test therefore implies that the series diverges.
\end{solution}
\end{problem}

\begin{problem}
For each part, determine which of the following properties each of the given sequences $\{a_k\}$ and $\{s_n\}$ has: monotone increasing, monotone decreasing, bounded above, bounded below. 

I. Let $\{a_k\}_{k=1}^\infty$ be the sequence defined by $a_k = \frac{1}{k^3}$ and let $\{s_n\}_{n=1}^\infty$ denote its sequence of partial sums. 

II. Let $\{a_k\}_{k=1}^\infty$ be the sequence defined by $a_k = \frac{2^k}{2^k +1}$ and let $\{s_n\}_{n=1}^\infty$ denote its sequence of partial sums.

\begin{solution}
I. The sequence $\{a_k\}$ is monotone decreasing and bounded (above and below), by inspection. Since the terms of $\{a_k\}$ are strictly positive, the sequence $\{s_n\}$ must be monotone increasing. It is straightforward to show via the Integral Test that the series $\sum \frac{1}{k^3}$ converges. It follows by the definition of convergence that the sequence $\{s_n\}$ has a finite limit and is therefore bounded.

II. The sequence $\{a_k\}$ is monotone increasing and bounded, with limit $\lim_{k\rightarrow \infty} a_k = 1$. The Divergence Test implies that the series $\sum a_k$ diverges. It follows that the sequence $\{s_n\}$ is monotone increasing and bounded below (because the terms of $a_k$ are positive), but not bounded above (because $\lim_{n \rightarrow \infty} s_n = \infty$). 
\end{solution}
\end{problem}

\end{document}
