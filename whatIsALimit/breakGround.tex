\documentclass{ximera}

\newcommand{\RR}{\mathbb R}
\renewcommand{\d}{\,d}
\newcommand{\dd}[2][]{\frac{d #1}{d #2}}
\renewcommand{\l}{\ell}
\newcommand{\ddx}{\frac{d}{dx}}
\newcommand{\dfn}{\textbf}
\newcommand{\eval}[1]{\bigg[ #1 \bigg]}


\outcome{Consider  values of a function at inputs approaching a given point.}

\title[Break-Ground:]{Stars and functions}

\begin{document}
\begin{abstract}
Two young mathematicians discuss stars and functions.
\end{abstract}
\maketitle

Check out this dialogue between two calculus students (based on a true
story):

\begin{dialogue}
\item[Devyn] Riley, did you know I like looking at the stars at night?
\item[Riley] Stars are freaking awesome balls of nuclear fire whose
  light took thousands of years to reach us.
\item[Devyn] I know! But did you know that the best way to see a very
  dim star is to look \textbf{near} it but \textbf{not exactly at} it? It's
  because then you can use the ``rods'' in your eye, which work better
  in low light than the ``cones'' in your eyes.
\item[Riley] That's amazing! Hey, that reminds me of when we were talking about the two functions
  \[
  f(x) = \frac{x^2-3x+2}{x-2}\qquad\text{and}\qquad g(x)= x-1,
  \]
  which we now know are completely different functions.
\item[Devyn] Whoa. How are you seeing a connection here?
\item[Riley] If we want to understand what is happening with the
  function
  \[
  f(x) = \frac{x^2-3x+2}{x-2},
  \]
  at $x=2$, we can't do it by setting $x=2$. Instead we need to look
  \textbf{near} $x=2$ but \textbf{not exactly at} $x=2$.
  \item[Devyn] Ah ha! Because if we are \textbf{not exactly at} $x=2$,
    then
    \[
    \frac{x^2-3x+2}{x-2} = x-1.
    \]
\end{dialogue}

\begin{problem}
  Let $f(x) = \frac{x^2-3x+2}{x-2}$ and $g(x) = x-1$. Which of the following is true?
  \begin{multipleChoice}
  \choice{$f(x) = g(x)$  for every value of $x$.}
  \choice{There is no $x$-value where $f(x) = g(x)$.}
  \choice[correct]{$f(x) = g(x)$ when $x\ne 2$.}
  \end{multipleChoice}
\end{problem}

\begin{problem}
  When you evaluate
  \[
  f(x) = \frac{x^2-3x+2}{x-2},
  \]
  at $x$-values approaching (but not equal to) $2$, what happens to the value of $f(x)$?
  \begin{prompt}
    The value of $f(x)$ approaches $\answer{1}$.
  \end{prompt}
  \begin{problem}
    Just from checking some values, can you be absolutely certain that your
    answer to the previous problem is correct?
    \begin{multipleChoice}
      \choice{yes}
      \choice[correct]{no}
    \end{multipleChoice}
    \begin{feedback}
      Here you only have information about a few specific points on
      the graph.  There are infinitely many $x$-values close to, but
      not equal to, $x = 2$. Hence we cannot be completely certain.
    \end{feedback}
  \end{problem}
\end{problem}

% reconcile two experiences: we say limits are about "closer and closer" but in practice we just compute algebraically.  How do you reconcile these two experiences? <--- Better in the next section.

%%% \begin{xarmaBoost}
%%   Write down at least \textbf{five} questions for this lecture. After
%%   you have your questions, label them as ``Level 1,'' ``Level 2,'' or
%%   ``Level 3'' where:
%% \begin{description}
%% \item[Level 1] Means you know the answer, or know exactly how to do
%%   this problem.
%% \item[Level 2] Means you think you know how to do the problem.
%% \item[Level 3] Means you have no idea how to do the problem.
%% \end{description}
%% \begin{freeResponse}
%% \end{freeResponse}
%% \end{xarmaBoost}



\end{document}
