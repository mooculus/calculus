\documentclass{ximera}

\newcommand{\RR}{\mathbb R}
\renewcommand{\d}{\,d}
\newcommand{\dd}[2][]{\frac{d #1}{d #2}}
\renewcommand{\l}{\ell}
\newcommand{\ddx}{\frac{d}{dx}}
\newcommand{\dfn}{\textbf}
\newcommand{\eval}[1]{\bigg[ #1 \bigg]}


\title{Limit laws and indeterminant forms}

\begin{document}
\begin{abstract}
  Here we look for patterns for computing limits.
\end{abstract}
\maketitle



On the other hand, there is a 


In this section, we present a handful of tools to compute many limits
without explicitly working with the definition of limit. Each of these
could be proved directly as we did in the previous section.

\begin{theorem}[Limit Laws]\index{limit laws}\label{theorem:limit-laws}
Suppose that $\lim_{x\to a}f(x)=L$, $\lim_{x\to a}g(x)=M$, $k$
is some constant, and $n$ is a positive integer.
\begin{itemize}
\item[\textbf{Constant Law}] $\lim_{x\to a} kf(x) = k\lim_{x\to a}f(x)=kL$.
\item[\textbf{Sum Law}] $\lim_{x\to a} (f(x)+g(x)) = \lim_{x\to a}f(x)+\lim_{x\to a}g(x)=L+M$.  
\item[\textbf{Product Law}] $\lim_{x\to a} (f(x)g(x)) = \lim_{x\to a}f(x)\cdot\lim_{x\to a}g(x)=LM$. 
\item[\textbf{Quotient Law}] $\lim_{x\to a} \frac{f(x)}{g(x)} =
  \frac{\lim_{x\to a}f(x)}{\lim_{x\to a}g(x)}=\frac{L}{M}$, if $M\ne0$.
\item[\textbf{Power Law}] $\lim_{x\to a} f(x)^n = \left(\lim_{x\to a}f(x)\right)^n=L^n$.
\item[\textbf{Root Law}] $\lim_{x\to a} \sqrt[n]{f(x)} = \sqrt[n]{\lim_{x\to
    a}f(x)}=\sqrt[n]{L}$ provided if $n$ is even, then $f(x)\ge 0$
  near $a$.
\item[\textbf{Composition Law}] If $\lim_{x\to a}g(x)=M$ and
  $\lim_{x\to M}f(x) = f(M)$, then $\lim_{x\to a} f(g(x)) = f(M)$.
\end{itemize}
\label{thm:limit laws}
\end{theorem}

Roughly speaking, these rules say that to compute the limit of an
algebraic expression, it is enough to compute the limits of the
``innermost bits'' and then combine these limits. This often means
that it is possible to simply plug in a value for the variable, since
$\lim_{x\to a} x =a$.


\begin{example}
Compute $\lim_{x\to 1}{x^2-3x+5\over x-2}$. 
\end{example}
\begin{solution}
Using limit laws, 
\begin{align*}
\lim_{x\to 1}{x^2-3x+5\over x-2}&=
\dfrac{\lim_{x\to 1}x^2-3x+5}{\lim_{x\to1}(x-2)} \\
&=\frac{\lim_{x\to 1}x^2-\lim_{x\to1}3x+\lim_{x\to1}5}{\lim_{x\to1}x-\lim_{x\to1}2} \\
&=\dfrac{\left(\lim_{x\to 1}x\right)^2-3\lim_{x\to1}x+5}{\lim_{x\to1}x-2} \\
&=\dfrac{1^2-3\cdot1+5}{1-2} \\
&=\dfrac{1-3+5}{-1} = -3.
\end{align*}
\end{solution}


It is worth commenting on the trivial limit $\lim_{x\to1}5$. From one
point of view this might seem meaningless, as the number 5 can't
``approach'' any value, since it is simply a fixed number. But 5 can,
and should, be interpreted here as the function that has value 5
everywhere, $f(x)=5$, with graph a horizontal line. From this point of
view it makes sense to ask what happens to the height of the function
as $x$ approaches 1.

We're primarily interested in limits that aren't so easy, namely
limits in which a denominator approaches zero. The basic idea is to
``divide out'' by the offending factor. This is often easier said than
done---here we give two examples of algebraic tricks that work on many
of these limits.


\begin{example}
Compute $\lim_{x\to1}{x^2+2x-3\over x-1}$. 
\end{example}
\begin{solution}
We can't simply plug in $x=1$ because that makes the denominator zero.
However, when taking limits we assume $x\ne 1$:
\begin{align*}
\lim_{x\to1}{x^2+2x-3\over x-1}&=\lim_{x\to1}{(x-1)(x+3)\over x-1} \\
&=\lim_{x\to1}(x+3)=4
\end{align*}
\end{solution}
%\marginnote[-1in]{Limits allow us to examine functions where they are not defined.}

\begin{example}
Compute $\lim_{x\to-1} {\sqrt{x+5}-2\over x+1}$.
\end{example}
\begin{solution} 
Using limit laws,
\begin{align*}
\lim_{x\to-1} {\sqrt{x+5}-2\over x+1}&=
\lim_{x\to-1} {\sqrt{x+5}-2\over x+1}{\sqrt{x+5}+2\over \sqrt{x+5}+2} \\
&=\lim_{x\to-1} {x+5-4\over (x+1)(\sqrt{x+5}+2)} \\
&=\lim_{x\to-1} {x+1\over (x+1)(\sqrt{x+5}+2)} \\
&=\lim_{x\to-1} {1\over \sqrt{x+5}+2}={1\over4}.
\end{align*}
\end{solution}
%\marginnote[-1.5in]{Here we are rationalizing the numerator by multiplying by the conjugate.}



investigate limits of
\textit{indeterminate form}.

\begin{definition}[List of Indeterminate Forms]\hfil
\begin{itemize}
\item[\textbf{0/0}] This refers to a limit of the form $\lim_{x\to a}
  \frac{f(x)}{g(x)}$ where $f(x)\to 0$ and $g(x)\to 0$ as $x\to a$.
\item[\textbf{$\pmb\infty$/$\pmb\infty$}] This refers to a limit of the form $\lim_{x\to a}
  \frac{f(x)}{g(x)}$ where $f(x)\to \infty$ and $g(x)\to \infty$ as $x\to a$.
\item[\textbf{0\,$\pmb{\cdot\infty}$}] This refers to a limit of the form $\lim_{x\to a}
  \left(f(x)\cdot g(x)\right)$ where $f(x)\to 0$ and $g(x)\to \infty$ as $x\to a$.
\item[\textbf{$\pmb\infty$--$\pmb\infty$}] This refers to a limit of the form $\lim_{x\to a}\left(
  f(x)-g(x)\right)$ where $f(x)\to \infty$ and $g(x)\to \infty$ as $x\to a$.

\item[\textbf{1$^{\pmb\infty}$}] This refers to a limit of the form $\lim_{x\to a}
  f(x)^{g(x)}$ where $f(x)\to 1$ and $g(x)\to \infty$ as $x\to a$.
\item[\textbf{0$^\text{0}$}] This refers to a limit of the form $\lim_{x\to a}
  f(x)^{g(x)}$ where $f(x)\to 0$ and $g(x)\to 0$ as $x\to a$.
\item[\textbf{$\pmb\infty^\text{0}$}] This refers to a limit of the form $\lim_{x\to a}
  f(x)^{g(x)}$ where $f(x)\to \infty$ and $g(x)\to 0$ as $x\to a$.
\end{itemize}
 
\end{definition}

In each of these cases, the value of the limit is \textbf{not}
immediately obvious.  For example, if we have a limit of the form
$\textbf{0$^\text{0}$}$, then the base of the exponent is ``driving the
limit to $0$,'' while the exponent is ``driving the limit to $1$.''
In this tug of war, it is not clear which of the two combatants will
triumph, or (if neither ``wins'') to what value they will reach a
compromise. Hence, a careful analysis is required.




\end{document}
