\documentclass[noauthor,handout]{ximera}
%handout:  for handout version with no solutions or instructor notes
%handout,instructornotes:  for instructor version with just problems and notes, no solutions
%noinstructornotes:  shows only problem and solutions

%% handout
%% space
%% newpage
%% numbers
%% nooutcomes

%I added the commands here so that I would't have to keep looking them up
%\newcommand{\RR}{\mathbb R}
%\renewcommand{\d}{\,d}
%\newcommand{\dd}[2][]{\frac{d #1}{d #2}}
%\renewcommand{\l}{\ell}
%\newcommand{\ddx}{\frac{d}{dx}}
%\everymath{\displaystyle}
%\newcommand{\dfn}{\textbf}
%\newcommand{\eval}[1]{\bigg[ #1 \bigg]}

%\begin{image}
%\includegraphics[trim= 170 420 250 180]{Figure1.pdf}
%\end{image}

%add a ``.'' below when used in a specific directory.

%\newcommand{\RR}{\mathbb R}
\renewcommand{\d}{\,d}
\newcommand{\dd}[2][]{\frac{d #1}{d #2}}
\renewcommand{\l}{\ell}
\newcommand{\ddx}{\frac{d}{dx}}
\newcommand{\dfn}{\textbf}
\newcommand{\eval}[1]{\bigg[ #1 \bigg]}

\newcommand{\RR}{\mathbb R}
\renewcommand{\d}{\,d}
\newcommand{\dd}[2][]{\frac{d #1}{d #2}}
\renewcommand{\l}{\ell}
\newcommand{\ddx}{\frac{d}{dx}}
\newcommand{\dfn}{\textbf}
\newcommand{\eval}[1]{\bigg[ #1 \bigg]}





\author{Jim Talamo and Tom Needham}

\outcome{Compute dot products between $2$ and $3$-dimensional vectors.}
\outcome{Answer conceptual questions about dot products.}
\outcome{Find orthogonal decompositions of vectors.}

\title[Collaborate:]{Dot Products}

\begin{document}
\begin{abstract}
\end{abstract}
\maketitle

\section{Discussion Questions}

\begin{problem}
Let $\vec{u}$ and $\vec{v}$ be nonzero $3$-dimensional vectors. Determine whether the following statements are true or false. 

\begin{enumerate}[label=(\alph*)]
\item $\mathrm{scal}_{\vec{v}} \vec{u} \leq \left|\vec{u}\right|$.
\item If $\vec{w}$ is parallel to $\vec{v}$, then $\mathrm{proj}_\vec{v} \vec{u} = \mathrm{proj}_\vec{w} \vec{u}$.
\item $(\vec{u} - \mathrm{proj}_\vec{v} \vec{u}) \cdot \vec{v} = 0$ .
\item $\mathrm{scal}_\vec{u} \vec{u} = 0$ .
\item If $\mathrm{proj}_\vec{v} \vec{u} = \mathrm{proj}_\vec{u} \vec{v}$, then $\vec{u} = \vec{v}$ .
\item If $\mathrm{scal}_\vec{v} \vec{u} = \mathrm{scal}_\vec{u} \vec{v}$, then $\vec{u} = \vec{v}$ .
\end{enumerate}
\begin{freeResponse}
\begin{enumerate}[label=(\alph*)]
\item This statement is true. We have 
$$
\mathrm{scal}_{\vec{v}} \vec{u} = \frac{\vec{u} \cdot \vec{v}}{\left|\vec{v}\right|} = \left|\vec{u}\right| \cos \theta,
$$
where $\theta$ is the angle between $\vec{u}$ and $\vec{v}$. Since the range of the cosine function is $[-1,1]$, the conclusion follows.
\item This statement is true. Assume that $\vec{v} = a \vec{w}$ for some scalar $a$. Then
$$
\mathrm{proj}_\vec{v} \vec{u} = \frac{\vec{u} \cdot \vec{v}}{\left|\vec{v}\right|^2} \vec{v} = \frac{\vec{u} \cdot a \vec{w}}{\left|a \vec{w}\right|^2} a \vec{w} = \frac{a^2}{a^2} \frac{\vec{u} \cdot \vec{w}}{\left|\vec{w}\right|^2} \vec{w} = \mathrm{proj}_\vec{w} \vec{u}.
$$
\item This statement is true. We have
$$
(\vec{u} - \mathrm{proj}_\vec{v} \vec{u}) \cdot \vec{v} = \vec{u} \cdot \vec{v} - \frac{\vec{u} \cdot \vec{v}}{\left|\vec{v}\right|^2} \vec{v} \cdot \vec{v} = \vec{u} \cdot \vec{v} - \frac{\vec{u} \cdot \vec{v}}{\left|\vec{v}\right|^2} \left|\vec{v}\right|^2 = \vec{u} \cdot \vec{v} - \vec{u} \cdot \vec{v} = 0.
$$
\item This statement is false. For any nonzero $\vec{u}$, we have
$$
\mathrm{scal}_\vec{u} \vec{u} = \frac{\vec{u} \cdot \vec{u}}{\left|\vec{u}\right|} = \frac{\left|\vec{u}\right|^2}{\left|\vec{u}\right|} = \left|\vec{u}\right| \neq 0.
$$
\item This statement is false; a counterexample is given by any pair of nonzero, orthogonal vectors. 
\item This statement is false, with the same counterexample used in the previous statement. 
\end{enumerate}
\end{freeResponse}
\end{problem}

%%%%%%%%%%%%%%%%%%%%%%%%%%%%%%%%%%%%%%%%%%%%%%%%%%%%%

\section{Group Work}

\begin{problem}
Let $\vec{u} = \left<1,0,2\right>$, $\vec{v} = \left<-1,1,2\right>$ and $\vec{w} = \left<2,2,0\right>$. 
\begin{itemize}
\item[I.] Which pairs of vectors listed above are orthogonal?
\item[II.] Which pairs of vectors have interior angle between them less than $\frac{\pi}{2}$?
\end{itemize}
\begin{freeResponse}
Calculating dot products explicitly, we have
\begin{align*}
\vec{u} \cdot \vec{v} &=  1 \cdot -1 + 0 \cdot 1 + 2 \cdot 2 = 3 \neq 0\\
\vec{u} \cdot \vec{w} &= 1 \cdot 2 + 0 \cdot 2 + 2 \cdot 0 = 2 \neq 0 \\
\vec{v} \cdot \vec{w} &= -1 \cdot 2 + 1 \cdot 2 + 2 \cdot 0 = 0.
\end{align*}
It follows that $\vec{v}$ and $\vec{w}$ are the unique pair which are orthogonal.

II. A pair of vectors has interior angle less than $\frac{\pi}{2}$ if and only if their dot product is positive. Therefore the pairs $\vec{u}$ and $\vec{v}$ and $\vec{u}$ and $\vec{w}$ have this property.
\end{freeResponse}
\end{problem}

%%%%%%%%%%%%%%%%%%%%%%%%%%%%%%%%%%%%%%%%%%%%%%%%%%%%%

\begin{problem}
Let $\vec{u} = \left<2,a,6\right>$ and $\vec{v} = \left<-1,2,a\right>$ for some number $a$. 
\begin{itemize}
\item[I.] Find $a$ so that $\vec{u}$ and $\vec{v}$ are parallel or explain why $a$-value exists.
\item[II.] Find $a$ so that $\vec{u}$ and $\vec{v}$ are orthogonal or explain why $a$-value exists.
\end{itemize}

\begin{freeResponse}
I. Since the first coordinate of $\vec{u}$ is $2$ and the first coordinate of $\vec{v}$ is $-1$, the vectors could only be parallel if $\vec{u} = -2 \cdot \vec{v}$. Considering the second and third coordinates of the vectors, this condition would force $-2 \cdot 2 = a$ and $-2 \cdot a = 6$. Since these equations cannot be satisfied simultaneously, it must be that there is no number $a$ which would make the vectors $\vec{u}$ and $\vec{v}$ parallel.

II. We have
$$
\vec{u} \cdot \vec{v} = 2 \cdot (-1) + a \cdot 2 + 6 \cdot a = -2 + 8a.
$$
For the vectors to be orthogonal, we need
$$
-2 + 8 a = 0.
$$
The number $a = \frac{1}{4}$ produces orthogonal vectors $\vec{u}$ and $\vec{v}$. 
\end{freeResponse}
\end{problem}

%%%%%%%%%%%%%%%%%%%%%%%%%%%%%%%%%%%%%%%%%%%%%%%%%%%%%

\begin{problem}
Let $\vec{u} = \left<-2,2\right>$ and $\vec{v} = \left<1,4\right>$. 
\begin{itemize}
\item[I.] Sketch $\vec{u}$, $\vec{v}$ and $\mathrm{proj}_{\vec{v}} \vec{u}$ on the axes below.  Then, compute $\mathrm{proj}_\vec{v} \vec{u}$. 

\resizebox {6cm} {!} {   \begin{tikzpicture}  
    \begin{axis}[  
        xmin=-5,  
        xmax=5,  
        ymin=-5,  
        ymax=5,  
        axis lines=center,  
        xlabel=$x$,  
        ylabel=$y$,  
        every axis y label/.style={at=(current axis.above origin),anchor=south},  axis on top
        every axis x label/.style={at=(current axis.right of origin),anchor=west},  axis on top
      ]  
      
            \end{axis}  
  \end{tikzpicture}  }



\item[II.] Find a vector $\vec{p}$ parallel to $\vec{v}$ and a vector $\vec{n}$ orthogonal to $\vec{v}$ so that $\vec{u} = \vec{p} + \vec{n}$ and sketch $\vec{u}$, $\vec{v}$, $\vec{p}$ and $\vec{n}$ on the axes below.


\resizebox {6cm} {!} {   \begin{tikzpicture}  
    \begin{axis}[  
        xmin=-5,  
        xmax=5,  
        ymin=-5,  
        ymax=5,  
        axis lines=center,  
        xlabel=$x$,  
        ylabel=$y$,  
        every axis y label/.style={at=(current axis.above origin),anchor=south},  axis on top
        every axis x label/.style={at=(current axis.right of origin),anchor=west},  axis on top
      ]  
      
            \end{axis}  
  \end{tikzpicture}  }

\end{itemize}

\begin{freeResponse}
I. The projection vector is 
$$
\mathrm{proj}_\vec{v} \vec{u} = \frac{\left<-2,2\right> \cdot \left<1,4\right>}{\left|\left<1,4\right>\right|^2} \left<1,4\right> = \frac{-2 + 8}{1^2 + 4^2} \left<1,4\right>  = \frac{6}{17}\left<1,4\right>.
$$

II. Let $\vec{p} = \mathrm{proj}_\vec{v} \vec{u}$, as calculated above, and let 
$$
\vec{n} = \vec{u} - \vec{p} = \left<-2,2\right> - \frac{6}{17} \left<1,4\right> = \left<-\frac{40}{17},\frac{10}{17}\right>.
$$
One can check by a calculation that $\vec{p}$ is parallel to $\vec{v}$ and $\vec{n}$ is perpendicular to $\vec{v}$. This also follows by general principles: $\vec{p}$ is parallel to $\vec{v}$ by definition, and $\vec{n}$ is perpendicular to $\vec{v}$ by Problem 1.
\end{freeResponse}
\end{problem}

\begin{problem}
Determine necessary and sufficient conditions on vectors $\vec{u}$ and $\vec{v}$ for $\mathrm{proj}_\vec{v} \vec{u} = \mathrm{proj}_\vec{u} \vec{v}$. (Hint: consider the case where $\vec{u}$ and $\vec{v}$ are orthogonal separately.)

\begin{freeResponse}
If $\vec{u}$ and $\vec{v}$ are orthogonal, then
$$
\mathrm{proj}_\vec{v} \vec{u} = \mathrm{proj}_\vec{u} \vec{v} = \vec{0}.
$$
In fact, orthogonality of $\vec{u}$ and $\vec{v}$ is both necessary and sufficient for one (and hence both) projections to be $\vec{0}$. 

Now suppose that $\vec{u}$ and $\vec{v}$ are not orthogonal. Then $\mathrm{proj}_\vec{v} \vec{u} = \mathrm{proj}_\vec{u} \vec{v}$ holds if and ony if 
$$
\frac{\vec{u} \cdot \vec{v}}{\left|\vec{v}\right|^2} \vec{v} = \frac{\vec{v} \cdot \vec{u}}{\left|\vec{u}\right|^2} \vec{u}.
$$
Since we have assumed that $\vec{u} \cdot \vec{v} \neq 0$, this equation holds if and only if
$$
\frac{\vec{u}}{\left|\vec{u}\right|^2} = \frac{\vec{v}}{\left|\vec{v}\right|^2}.
$$
\end{freeResponse}
\end{problem}

\end{document}
