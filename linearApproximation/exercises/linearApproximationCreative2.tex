\documentclass{ximera}
\newcommand{\RR}{\mathbb R}
\renewcommand{\d}{\,d}
\newcommand{\dd}[2][]{\frac{d #1}{d #2}}
\renewcommand{\l}{\ell}
\newcommand{\ddx}{\frac{d}{dx}}
\newcommand{\dfn}{\textbf}
\newcommand{\eval}[1]{\bigg[ #1 \bigg]}

\author{Steven Gubkin}
\license{Creative Commons 3.0 By-NC}

\outcome{Define linear approximation as an application of the tangent to a curve.}
\outcome{Find the linear approximation to a function at a point and use it to approximate the function value.}
\outcome{Identify when a linear approximation can be used.}

\begin{document}
\begin{exercise}
In Einstein's theory of relativity, we can derive that

$$E(v) = \frac{mc^2}{\sqrt{1-\frac{v^2}{c^2}}}$$

where $E(v)$ is the energy of an object with ``Rest mass'' $m$ and velocity $v$.

Let us analyze this more closely.

First find the linear approximation to the function $f(u) = \frac{mc^2}{\sqrt{1-u}}$ at $u=0$.

Using this approximation, and substituting $u = \frac{v^2}{c^2}$, we can obtain an approximation for $E(v)$ which is valid for small velocities $v$.

The approximation you obtain should have two terms.  One of which is the famous $E = mc^2$ (representing the resting energy) and the other should be the classical kinetic energy of the object.

\begin{prompt}
	The local linearization of $\frac{m c^2}{\sqrt{1-u}}$ at $u=0$ is $\answer{m c^2 +\frac{m c^2}{2}u}$.

	So we get that $E(v) \approx \answer{mc^2+\frac{1}{2}mv^2}$.
\end{prompt}

\end{exercise}
\end{document}
