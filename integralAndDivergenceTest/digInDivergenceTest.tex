\documentclass{ximera}

\newcommand{\RR}{\mathbb R}
\renewcommand{\d}{\,d}
\newcommand{\dd}[2][]{\frac{d #1}{d #2}}
\renewcommand{\l}{\ell}
\newcommand{\ddx}{\frac{d}{dx}}
\newcommand{\dfn}{\textbf}
\newcommand{\eval}[1]{\bigg[ #1 \bigg]}


\outcome{Use the divergence test to determine that a series diverges.}
\outcome{Recognize known convergent or divergent series.}

\title[Dig-In:]{The divergence test}

\begin{document}
\begin{abstract}
If an infinite sum converges, then its terms must tend to zero.
\end{abstract}
\maketitle


As one contemplates the behavior of series, a few facts become clear.
In order to add an infinite list of nonzero numbers and get a finite
result, ``most'' of those numbers must be ``very near'' $0$.  Think of 
this in the opposite sense: what happens if you try to sum $\sum_{n=1}^\infty 2$?

If a series diverges, it means that the sum of an infinite list of
numbers is not finite (it may approach $\pm \infty$ or it may
oscillate), and:
\begin{itemize}
\item The series will still diverge if the first term is removed.
\item The series will still diverge if the first $10$ terms are
  removed.
\item The series will still diverge if the first $1000000$ terms
  are removed.
\item The series will still diverge if \textbf{any finite number} of terms
  from anywhere in the series are removed.
\end{itemize}

These concepts are very important and lie at the heart of the next
theorems.

\begin{theorem}[Divergence test]
  Consider the series
  \[
  \sum_{n=1}^\infty a_n.
  \]
\begin{enumerate}
\item If $\sum_{n=1}^\infty a_n$ converges, then $\lim_{n\to\infty}a_n
  =0$.
\item If $\lim_{n\to\infty}a_n \neq 0$, then $\sum_{n=1}^\infty a_n$
  diverges.
\end{enumerate}
\end{theorem}
Note that the two statements above are really the same. In order to
converge, the limit of the terms of the sequence must approach $0$; if
they do not, the series will not converge.
\begin{warning}
  This theorem \emph{does not state} that if $\lim_{n\to\infty} a_n =
  0$ then $\sum_{n=1}^\infty a_n$ converges.
\end{warning}

The standard example of a sequence whose terms go to zero, and yet does
not converge, is the harmonic series. The Harmonic sequence,
$(1/n)$, converges to $0$ while the Harmonic Series,
\[
\sum_{n=1}^\infty\frac{1}{n}\qquad\text{diverges.}
\]

Let's see if you've digested what we've been saying:

\begin{question}
Which of the following statements are true?  Mark all that apply.
\begin{selectAll}
  \choice[correct]{If $\sum_{k=0}^\infty a_k$ is convergent, then $\lim_{k \to \infty} a_k = 0$ }
  \choice{If $a_k \to 0$ as $k \to \infty$, then $\sum_{k=0}^\infty a_k$ is convergent}
  \choice{If $\sum_{k=0}^\infty a_k$ is divergent, then $\lim_{k \to \infty} a_k \neq 0$ }
  \choice[correct]{If $\lim_{k \to \infty} a_k \neq 0$, then $\sum_{k=0}^\infty a_k$ is divergent}
\end{selectAll}
\end{question}


\begin{question}
  We say that a series ``passes the divergence test'' if its sequence
  of terms tends to zero.  Which of the following series pass the
  divergence test?
\begin{selectAll}
	\choice[correct]{$\sum_{k=3}^\infty \frac{1}{\ln{ k }}$}
	\choice{$\sum_{k=0}^\infty \sin(k)$}
	\choice[correct]{$\sum_{k=0}^\infty \frac{\sin(k)}{k^2}$}
	\choice{$\sum_{k=5}^\infty \frac{k+7}{k+6}$}
	\choice{$\sum_{k=0}^\infty \frac{2k}{k - 5}$}
\end{selectAll}
\end{question}

Restating this point again (because it is very important): passing the divergence test 
means that a series has a chance to converge.  The divergence test cannot tell us 
whether a series converges.


\section{Some questions}
%% Jim Talamo's questions

\begin{question}
  Suppose $(a_n)_{n \geq 1}$ is a sequence and $\sum^{\infty}_{n= 1}
  a_n$ converges to $L>0$.  Let $S_n = \sum^n_{k=1} a_k$. Select all
  statements that must be true:
  \begin{selectAll}
    \choice{$\lim_{n \to \infty} a_n = L$}
    \choice[correct]{$\lim_{n \to \infty} a_n = 0$}
    \choice{$\lim_{n \to \infty} S_n = 0$}
    \choice[correct]{$\lim_{n \to \infty} S_n = L$}
    \choice[correct]{$\sum^{\infty}_{n=1} S_n$ must diverge.}
    \choice{$\sum^{\infty}_{n=1} (a_n+1) = L+1$}
    \choice{The divergence test tells us $\sum^{\infty}_{n= 1} a_n$ converges to $L$.}
  \end{selectAll}
\end{question}

\begin{question}
  Suppose that $\seq{a_n}_{n \geq 1}$ is a \emph{decreasing} sequence.
  Let $S_n = \sum^n_{k=1} a_k$ and suppose $\lim_{n \to \infty} S_n$
  does not exist. Select all statements that must be true:
  \begin{selectAll}
    \choice{$\lim_{n \to \infty} a_n$ does not exist.}
    \choice{$\sum^{\infty}_{k=1} a_k$ could converge.}
    \choice[correct]{$\sum^{\infty}_{n=1} S_n$ must diverge.} 
    \choice{$\seq{S_n}$ must be monotonic.}
    \choice{$\seq{S_n}$ must be bounded.}
    \choice{$\lim_{n \to \infty} S_n = -\infty$}
    \choice{The divergence test applied to $\sum^{\infty}_{k= 1} a_k$ would
      guarantee that $\sum^{\infty}_{k=1} a_k$ diverges.}
  \end{selectAll}
\end{question}


\begin{question}
  Suppose that $\seq{a_n}_{n \geq 1}$ is a sequence with $a_n > 0$ for all $n \geq 1$.
  Let $S_n = \sum^n_{k=1} a_k$ and suppose $\lim_{n \to \infty} S_n =
  L$. Select all statements that must be true:
  \begin{selectAll}
    \choice[correct]{$\sum^{\infty}_{k=1} a_k = L$}
    \choice[correct]{$\lim_{n \to \infty} a_n = 0$}
    \choice[correct]{$\seq{S_n}$ must be monotonic}
    \choice[correct]{$\seq{S_n}$ must be bounded}
    \choice{$\sum^{\infty}_{n=1} (a_n-L) = 0$}
    \choice[correct]{$\sum^{\infty}_{n=1} S_n$ must diverge}
    \choice{The divergence test applied to $\sum^{\infty}_{k= 1} a_k$ would
      guarantee that $\sum^{\infty}_{k=1} a_k$ converges.}
  \end{selectAll}
\end{question}

It's a great idea at this point to stop and compare the previous two questions.









\end{document}


