\documentclass{ximera}

\newcommand{\RR}{\mathbb R}
\renewcommand{\d}{\,d}
\newcommand{\dd}[2][]{\frac{d #1}{d #2}}
\renewcommand{\l}{\ell}
\newcommand{\ddx}{\frac{d}{dx}}
\newcommand{\dfn}{\textbf}
\newcommand{\eval}[1]{\bigg[ #1 \bigg]}


\author{Jason Miller}
\license{Creative Commons 3.0 By-NC}


\outcome{ Recognize the patterns that appear in trigonometric integrals and use appropriate substitutions 
to compute them.}


\begin{document}
\begin{exercise}
Determine the integral

\[
\int \sec^{2}(x) \tan(x) \d x
\]

We solve this integral in two different ways


First let $u=\tan(x)$. Then $\d u=\answer{ \sec^{2}(x)} \d x$. 

Then the integral in terms of $u$ is
\[
\int \answer{u} \d u
\]

Integrating and going back to $x$ we get:
\[
\int \sec^{2}(x)\tan(x) \d x= \answer{  \frac{ \tan^{2}(x)}{2}  + C}
\]
Use $C$ for the constant of integration. 

\begin{exercise}

Now let us do a different substitution to find the integral. Let $u=\sec(x)$. Then $\d u=\answer{ \sec(x) \tan(x)} \d x$

Then the integral in terms of $u$ becomes:

\[
\int \answer{u} \d u 
\]

Integrating and going back to $x$ gives us:

\[
\int \sec^{2}(x) \tan(x) \d x= \answer{  \frac{ \sec^{2}(x) }{2}+ C}
\]
Use $C$ for the constant of integration. 

Note that the answer you obtained from the second substitution seems to differ from the answer we obtained using the first substitution. Recall there is a theorem that all antiderivatives 
of a given function can only differ by a constant. Can you explain this apparent discrepancy?
\end{exercise}



\end{exercise}
\end{document}
