\documentclass{ximera}

\newcommand{\RR}{\mathbb R}
\renewcommand{\d}{\,d}
\newcommand{\dd}[2][]{\frac{d #1}{d #2}}
\renewcommand{\l}{\ell}
\newcommand{\ddx}{\frac{d}{dx}}
\newcommand{\dfn}{\textbf}
\newcommand{\eval}[1]{\bigg[ #1 \bigg]}


\outcome{Define linear approximation as an application of the tangent to a curve.}
\outcome{Find the linear approximation to a function at a point and use it to approximate the function value.}
%\outcome{Identify when a linear approximation can be used.}
%\outcome{Label a graph with the appropriate quantities used in linear approximation.}
%\outcome{Find the error of a linear approximation.}
\outcome{Compute differentials.}
%\outcome{Use the second derivative to discuss whether the linear approximation over or underestimates the actual function value.}
%\outcome{Contrast the notation and meaning of \d{y} versus \Delta y.}
%\outcome{Understand that the error shrinks faster than the displacement in the input.}
%\outcome{Justify the chain rule via the composition of linear approximations.}

\author{Nela Lakos \and Kyle Parsons}

\begin{document}
\begin{exercise}

The total number of people $N$ who have contracted a common cold by a time $t$ days after its outbreak on an island is given by
\[
N(t) = \frac{200000}{1+100e^{-0.1t}},\,t\geq0.
\]

We can express the relationship between a small change in $t$ and the corresponding change in $N$ in the form $\d{N} = N'(t)\d{t}$ giving us
\[
\d{N} = \frac{\answer{2000000}e^{-0.1t}}{\left(1+100e^{-0.1t}\right)^2}\d{t}.
\]

Assuming that 45 days have passed since the outbreak started, we use the above formula to estimate the number of people (rounded to the nearest 1 person) that will fall sick in the next day. That number is approximately $\answer{4986}$.

Assuming that 100 days have passed since the outbreak started, we use the above formula to estimate the number of people (rounded to the nearest 1 person) that will fall sick in the next day. That number is approximately $\answer{90}$.

\end{exercise}
\end{document}