\documentclass{ximera}

\newcommand{\RR}{\mathbb R}
\renewcommand{\d}{\,d}
\newcommand{\dd}[2][]{\frac{d #1}{d #2}}
\renewcommand{\l}{\ell}
\newcommand{\ddx}{\frac{d}{dx}}
\newcommand{\dfn}{\textbf}
\newcommand{\eval}[1]{\bigg[ #1 \bigg]}


\author{Jim Talamo}
\license{Creative Commons 3.0 By-NC}


\outcome{Answer conceptual questions about sequences}
\outcome{Understand the relationship between the terms of a sequence, its sequence of partial sums, and its sequence of remainders}

\begin{document}

\begin{exercise}

Suppose that $a_n = \frac{\ln(n)}{n}$ for $n \geq 2$. Circle all that apply for each sequence listed below.

$\bullet$ The sequence $\{a_n\}_{n=1}$ is eventually
\begin{selectAll}
\choice{increasing}
\choice[correct]{decreasing}
\choice[correct]{bounded above}
\choice[correct]{bounded below}
\end{selectAll}

$\bullet$ The sequence $\{s_n\}_{n=1}$ is
\begin{selectAll}
\choice[correct]{increasing}
\choice{decreasing}
\choice{bounded above}
\choice[correct]{bounded below}
\end{selectAll}

$\bullet$ The sequence $\{r_n\}_{n=1}$ is
\begin{selectAll}
\choice[correct]{not defined}
\choice{increasing}
\choice{decreasing}
\choice{bounded above}
\choice{bounded below}
\end{selectAll}

\begin{hint}
We can use the integral test to check whether $\sum_{k=1}^{\infty} a_k$ converges.  This allows us to answer the questions about $\{s_n\}_{n=1}$ and $\{r_n\}_{n=1}$.
\end{hint}

\begin{exercise}
Which of the sequences has a limit?
\begin{selectAll}
\choice[correct]{$\{a_n\}_{n=1}$}
\choice[correct]{$\{s_n\}_{n=1}$}
\choice[correct]{$\{r_n\}_{n=1}$}
\end{selectAll}
\end{exercise}


\end{exercise}
\end{document}