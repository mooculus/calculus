\documentclass{ximera}

\newcommand{\RR}{\mathbb R}
\renewcommand{\d}{\,d}
\newcommand{\dd}[2][]{\frac{d #1}{d #2}}
\renewcommand{\l}{\ell}
\newcommand{\ddx}{\frac{d}{dx}}
\newcommand{\dfn}{\textbf}
\newcommand{\eval}[1]{\bigg[ #1 \bigg]}


\title[Dig-In:]{Divergence theorem}

\outcome{State and use the divergence theorem.}

\begin{document}
\begin{abstract}
  We introduce the divergence theorem.
\end{abstract}
\maketitle




\section{The divergence theorem}

The divergence theorem states that certain volume integrals are equal
to certain surface integrals. Let's see the statement.

\begin{theorem}[Divergence Theorem]\index{divergence theorem}
  Suppose that the components of $\vec{F}:\R^3\to\R^3$ have continuous
  partial derivatives. If $R$ is a solid bounded by a surface
  $\partial R$ oriented with the normal vectors pointing outside, then:
  \[
  \iiint_R \divergence \vec{F}  \d V =   \oiint_{\partial R} \vec{F}\dotp\uvec{n}\d S
  \]
\end{theorem}

Integrals of the type above arise any time we wish to understand
``fluid flow'' through a surface.  The ``fluid'' in question could be
a real fluid like air or water, or it could be an electromagnetic
field, or something else entirely. Unfortunately, many of the ``real''
applications of the divergence theorem require a deeper understanding
of the specific context where the integral arises. For our part, we
will focus on using the divergence theorem as a tool for transforming
one integral into another (hopefully easier!) integral.


\begin{example}
  Let
  \[
  R = \{(x,y,z):-2\le x\le1, 1\le y\le 3, -5\le z\le 1\}
  \]
  and
  \[
  \vec{F}(x,y,z) = \vector{\sin(z),y^2,e^x},
  \]
  compute:
  \[
  \oiint_{\partial R} \vec{F}\dotp \uvec{n} \d S
  \]
  \begin{explanation}
    To compute this integral, we'll use the divergence theorem. Since
    our region is a box, the limits of the triple integral will be
    easy to work with. Moreover, computing the divergence of $\vec{F}$ we see
    \[
    \divergence \vec{F}(x,y,z) = \answer[given]{2y}.
    \]
    So by the divergence theorem, we have
    \begin{align*}
      \oiint_{\partial R} \vec{F}\dotp \uvec{n} \d S &= \iiint_R \divergence \vec{F}  \d V \\
      &= \int_{-2}^{\answer[given]{1}} \int_{-5}^{\answer[given]{1}} \int_{1}^{\answer[given]{3}}
      \answer[given]{2y} \d \answer[given]{y}\d \answer[given]{z}\d \answer[given]{x}\\
      &= \answer[given]{144}.
    \end{align*}
  \end{explanation}
\end{example}

Now let's see another example:



\begin{example}
  Let
  \begin{align*}
  R = \{(x,y,z):&0\le x\le1, \\
  &0\le y\le 2-2x, \\
  &0\le z\le 3-3x-3y/2\}
  \end{align*}
  and
  \[
  \vec{F}(x,y,z) = \vector{x,y,z},
  \]
  compute:
  \[
  \oiint_{\partial R} \vec{F}\dotp \uvec{n} \d S
  \]
  \begin{explanation}
    To compute this integral, we'll use the divergence theorem. This
    time our region is a tetrahedron, we'll work in cartesian
    coordinates. Moreover, computing the divergence of $\vec{F}$ we
    see
    \[
    \divergence \vec{F}(x,y,z) = \answer[given]{3}.
    \]
    So by the divergence theorem, we have
    \begin{align*}
      \oiint_{\partial R} \vec{F}\dotp \uvec{n} \d S &= \iiint_R \divergence \vec{F}  \d V \\
      &= \int_{\answer[given]{0}}^{\answer[given]{1}} \int_{\answer[given]{0}}^{\answer[given]{2-2x}} \int_{\answer[given]{0}}^{\answer[given]{3-3x-3y/2}}
      \answer[given]{3} \d z \d y \d x\\
      &= \answer[given]{3}.
    \end{align*}
  \end{explanation}
\end{example}

With our next two examples, we cannot help but flex our mathematical muscles a bit:


\begin{example}
  Let
  \begin{align*}
    R = \{(x,y,z):&-1\le x\le1, \\
    &1\le y\le 1,\\
    &0\le z\le \sqrt{1-x^2-y^2}\}
  \end{align*}
  and
  \[
  \vec{F}(x,y,z) = \vector{x,2y,3z},
  \]
  compute:
  \[
  \oiint_{\partial R} \vec{F}\dotp \uvec{n} \d S
  \]
  \begin{explanation}
    To compute this integral, we'll use the divergence theorem.
    Computing the divergence of $\vec{F}$ we see
    \[
    \divergence \vec{F}(x,y,z) = \answer[given]{6}.
    \]
    So by the divergence theorem, we have
    \begin{align*}
      \oiint_{\partial R} \vec{F}\dotp \uvec{n} \d S &= \iiint_R \divergence \vec{F}  \d V \\
      &= \iiint_R \answer[given]{6}  \d V \\
      &= \answer[given]{6} \iiint_R \d V
    \end{align*}
    But $\iiint_R \d V$ is just the volume of the hemisphere of radius
    $1$, which we know is $\answer[given]{2\pi/3}$. Hence:
    \[
    \oiint_{\partial R} \vec{F}\dotp \uvec{n} \d S = \answer[given]{4\pi}
    \]
  \end{explanation}
\end{example}


Above, we used the fact that we \textit{know} that the volume of a
sphere is $4\pi r^3/3$. When you know the volume that's great! If not
you have to compute the integral.

\begin{example}
  Let $R$ be the cone whose base is a disk of radius $2$ in the plane
  $z=1$ and whose vertex is a the origin. Compute the flux of
  \[
  \vec{F}(x,y,z) = \vector{x-\sin(y), 2y +e^x,  \cos(xy)-3z}
  \]
  across the boundary of $R$.
  \begin{explanation}
    Again, we will use the divergence theorem. Computing the
    divergence of $\vec{F}$ we see:
    \[
    \divergence\vec{F}(x,y,z) = \answer[given]{-1}
    \]
    Now write with me
    \begin{align*}
      \oiint_{\partial R} \vec{F}\dotp \uvec{n} \d S &= \iiint_R \divergence \vec{F}  \d V \\
      &= \iiint_R \answer[given]{-1} \d V \\
      &= \answer[given]{-1} \iiint_R \d V
    \end{align*}
    If you know the volume of the cone described above, you can be
    done! If not, do not despair, we'll simply use cylindrical
    coordinates to compute it. Write with me:
    \begin{align*}
    \iiint_R \d V &= \int_{\answer[given]{0}}^{\answer[given]{2\pi}}
    \int_{\answer[given]{0}}^{\answer[given]{2}}
    \int_{\answer[given]{r/2}}^{\answer[given]{1}} r \d z \d r \d\theta\\
    &=\int_{\answer[given]{0}}^{\answer[given]{2\pi}}
    \int_{\answer[given]{0}}^{\answer[given]{2}} \answer[given]{r - r^2/2}\d r \d\theta\\
    &=\int_{\answer[given]{0}}^{\answer[given]{2\pi}} \answer[given]{2/3}  \d\theta\\
    &=\answer[given]{4\pi/3}
    \end{align*}
    Hence:
    \[
    \oiint_{\partial R} \vec{F}\dotp \uvec{n} \d S = \answer[given]{-4\pi/3}
    \]
  \end{explanation}
\end{example}












\section{A new fundamental theorem of calculus}

How is the divergence theorem a fundamental theorem of calculus? Well
consider this:
\begin{image}
  \begin{tikzpicture}
    \draw[ultra thick, gray!50!black] plot [smooth cycle] coordinates {(-1.5,2) (.5,1) (1.5,2) (.5,3) (-1.5,3)};
    \shade[ball color=gray!50!white] plot [smooth cycle] coordinates {(-1.5,2) (.5,1) (1.5,2) (.5,3) (-1.5,3)};
    \node[inner sep=0pt] at (0,0) {$\iiint_R \divergence\vec{F}\d V\quad =\quad \oiint_{\partial R} \vec{F}\dotp\uvec{n}\d S$};

    \node at (-1.7,-.7) {$\underbrace{\hspace{7em}}$};

    \node at (1.7,-.7) {$\underbrace{\hspace{6.5em}}$};

    \node[below,inner sep=0pt,text width=4cm,scale=.5] at (-1.7,-1)
         {To compute the triple integral of $\divergence \vec{F}$ over a
           solid $V\subseteq\R^3$, };

    \node[below,inner sep=0pt,text width=4cm,scale=.5] at (1.8,-1) {we can compute the accumulation of $\vec{F}$ across a boundary surface $\partial R$.};
  \end{tikzpicture}
\end{image}


Are there more fundamental theorems of calculus? Absolutely, and we're
ready for the last one of this course. Read on young mathematician!



\end{document}
