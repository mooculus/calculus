\documentclass{ximera}

\newcommand{\RR}{\mathbb R}
\renewcommand{\d}{\,d}
\newcommand{\dd}[2][]{\frac{d #1}{d #2}}
\renewcommand{\l}{\ell}
\newcommand{\ddx}{\frac{d}{dx}}
\newcommand{\dfn}{\textbf}
\newcommand{\eval}[1]{\bigg[ #1 \bigg]}


\author{Jim Talamo}

\license{Creative Commons 3.0 By-SA}
\outcome{Review properties of lines}

\begin{document}

\begin{exercise}
The following exercise explores how geometric information about lines can be obtained from vector-valued functions that describe them.

First, recall that once we start discussing lines in $\R^3$, the notion of ``slope'' needs to be revisited.  A key observation is that lines with different ``slopes'' in the $xy$-plane actually point in different \emph{directions}, so it is a natural idea to replace  ``slope'' with a vector parallel to the line.

\begin{fact}
Given a vector $\vec{v}$ and a point $P_0$, a parameterization of the line that is parallel to $\vec{v}$ that passes through $P_0$ is

\[
\vec{l}(t) = \vec{v}t+\vec{P}_0, t \in \R.
\]

\end{fact}
\begin{exercise}
Find a parametrization of the line that is parallel to $\vec{v} = \vector{3,-2,5}$ that passes through $(2,9,4)$.

\[
\vec{l}(t) = \vector{\answer{3},\answer{-2},\answer{5}}t+\vector{\answer{2},\answer{9},\answer{4}} = \vector{\answer{3t+2},\answer{-2t+9},\answer{5t+4}}
\]

\begin{exercise}

Now, note that the line exists independently of the parameterization that we use to describe it; different choices of the parameter will trace out the line in different fashions, but they trace out the same line. 

Consider the line in $\R^2$ given by $y=4x+5$.

\begin{itemize}
\item If we require that $x(t)=t$, then $y(t) = \answer{4t+5}$, and a parameterization of the line is

\[\vec{l}_1(t) = \vector{x(t),y(t)} = \vector{\answer{t},\answer{4t+5}}, t \in \R .\]

\item If we require that $y(t)=t$, then $x(t) = \answer{\frac{1}{4}t-\frac{5}{4}}$, and a parameterization of the line is

\[\vec{l}_2(t) = \vector{x(t),y(t)} = \vector{\answer{\frac{1}{4}t-\frac{5}{4}},\answer{t}}, t \in \R .\]

\end{itemize}

Note that the line passes through the point $(x,y) = (2,13)$.  The vector-valued function $\vec{l}_1(t)$ will trace through this point at $t=\answer{2}$, while the vector-valued function $\vec{l}_2(t)$ will trace through this point at $t=\answer{13}$.

\begin{exercise}
Given a parametric description of a line, we can extract a vector parallel to the line without too much difficulty.

To find a vector parallel to the line $\vec{l}(t) = \vector{2t+3,4-5t,2}$, note that we can write the line in the form 

\[
\vec{l}(t) = \vec{v}t+\vec{P}_0 = \vector{\answer{2},\answer{-5},\answer{0}}t+\vector{\answer{3},\answer{4},\answer{2}}
\]

A vector parallel to the line is thus $\vector{\answer{2},\answer{-5},\answer{0}}$.

\begin{exercise}
To explore this a little more deeply, note that we can find a vector parallel to the line from two points on the line.  Note that $\vec{l}(0) = \vector{3,4,2}$ and $\vec{l}(1) = \vector{\answer{5},\answer{-1},\answer{2}}$.

A vector $\vec{v}$ parallel to the line can be found from the corresponding points

\[
\vec{v} = \vector{5-3,-1-4,2-2} = \vector{\answer{2},\answer{-5},\answer{0}}.
\]

\begin{exercise}
Now that we have a quick way to find a vector parallel to a given line, we can use these to determine if two lines described by vector-valued functions are parallel.

Let $\vecl_{1}(t)=\vector{2t+4,t+5,4t+8}$ and $\vecl_{2}(t)=\vector{4t+1,2t,8t-3}$. 

Then $\vecl_{1}$ and $\vecl_{2}$ are 
\begin{multipleChoice}
\choice[correct]{parallel.}
\choice{intersecting.}
\choice{non-intersecting.}
\choice{the same line.}
\end{multipleChoice}


\end{exercise}
\end{exercise}
\end{exercise}
\end{exercise}
\end{exercise}
\end{exercise}

\end{document}
