\documentclass{ximera}

\newcommand{\RR}{\mathbb R}
\renewcommand{\d}{\,d}
\newcommand{\dd}[2][]{\frac{d #1}{d #2}}
\renewcommand{\l}{\ell}
\newcommand{\ddx}{\frac{d}{dx}}
\newcommand{\dfn}{\textbf}
\newcommand{\eval}[1]{\bigg[ #1 \bigg]}


\title[Refresh:]{Square-roots}

\begin{document}
\begin{abstract}
  Remember our facts about square-roots.
\end{abstract}
\maketitle

\begin{problem}
  Square-roots come up by necessity in many of the applications in
  this course, and they are traditionally troublesome to
  handle. Perhaps the most fundamental error that arises is the claim
  that square-roots split-up over addition and subtraction. Note that:
  \[
  \sqrt{a^2+b^2} \ne a+ b
  \]
  \begin{warning}
    \textbf{If you make this error in any problem, you will receive NO
      credit for the rest of the problem, and you may additionally be
      penalized for this error! DO NOT DO THIS!}
  \end{warning}
  \begin{multipleChoice}
    \choice[correct]{I understand.}
    \choice{I do not understand.}
  \end{multipleChoice}
\end{problem}

\begin{problem}
  To verify in general that $\sqrt{a^2+b^2} \ne a+b$, note that if the equality were true, then the left- and right-hand sides must be equal no matter what values we plug in for $a$ and $b$.
  
  Let $a = 4$ and $b = 3$.
  Then $\sqrt{a^2 + b^2} = \answer{5}$.
  (Simplify your answer completely.)
  However, $a + b = \answer{7}$.
  
  Hence, $\sqrt{a^2 + b^2} \ne a + b$ since equality does not hold for any choice of $a$ and $b$.
\end{problem}

\begin{problem}
  Simplify the following expression completely or state that it cannot be simplified further.
  \[
    \sqrt{x^2 + 16}.
  \]
  \begin{multipleChoice}
    \choice{$x + 4$}
    \choice{$|x| + 4$}
    \choice[correct]{This cannot be simplified further.}
  \end{multipleChoice}
\end{problem}

\begin{problem}
  At this point in the course there are very few types of expressions involving square roots that we can integrate.  The common types of integrals that arise in these sections include integrals whose integrand​ involves:
  \begin{itemize}
    \item
      square roots of linear functions of $x$
      
    \item
      square roots where the expression underneath the square root is a perfect square
    
    \item
      a $u$-substitution.
  \end{itemize}
  
  The next three problems deal with these.
  
  \begin{multipleChoice}
    \choice[correct]{I understand}
    \choice{I don't understand}
  \end{multipleChoice}
\end{problem}

\begin{problem}[Square roots of linear expressions in $x$]
  Evaluate $\int \sqrt{4x + 5}$ by filling out the following argument:
  \begin{enumerate}
    \item Let $u = \answer{4x + 5}$, then $\d u = \answer{4} \d x$.
    \item Making these substitutions into the integral gives: $\int \sqrt{4x + 5} \d x = \int \answer{\frac{\sqrt{u}}{4}} \d u$.
    \item After computing the antiderivative in $u$ and substituting, we find:
    \[
      \int \sqrt{4x + 5} \d x = \answer{\frac{1}{6}(4x+5)^{3/2}} + C
    \]
  \end{enumerate}
\end{problem}

\begin{problem}[Perfect squares]
  Expressions of the form $ax^2 + bx + c$ can be written as a sum or difference of squares.
  The following problem reviews how to do this.
  
  Write $x^2 + 6x - 2$ as a sum or difference of squares.
  
  \subsection{Procedure}
  \begin{enumerate}
    \item Take half of the coefficient in front of $x$
    \item Square it and add and subtract it from the expression

      The result so far is: $x^2 + 6x + 9 - 9 - 2$.
      
    \item The first three terms can be be rewritten as 
      \[
        x^2 + 6x + 9 = (x + 3)^2
      \]
      so the expression can be written as
      \[
        x^2 + 6x - 2 = (x+3)^2 - 11.
      \]
  \end{enumerate}
  
 
    Following this example, write $x^2 + 8x + 3$ as a sum or difference of squares:
    \[
      x^2 + 8x + 3 = (x + \answer{4})^2 + \answer{-13}
    \]  
\end{problem}

\begin{problem}
  An integral that will arise in the context of a specific type of problem we will study is
  \[
    \int \sqrt{1 + \left(x - \frac{1}{4x} \right)^2} \d x, \text{ $x > 0$}
  \]
  This is the actual form you will get when answer this type of problem.
  You will be expected to simplify the integrand then compute the antiderivative!
  
  \begin{enumerate}
    \item The expression in the integrand can be simplified.
      Simplify this completely and in your final answer, write all powers of $x$ in the denominator as negative powers of $x$ in the numerator.
      $1 + \left(x - \frac{1}{4x}\right)^2 = \answer{x^2 + \frac{1}{2} + \frac{1}{16} x^{-2}}$
    \item Simplify the expression $\left( x + \frac{1}{4x} \right)^2$
    \[
       \left( x + \frac{1}{4x} \right)^2 = \answer{x^2 + \frac{1}{2} + \frac{1}{16x^2}}
    \]
    \item From this you should see that the expression under the square root is actually a perfect square!
      Using this and the fact that $x > 0$, compute the antiderivative:
      \[
        \int \sqrt{1 + \left( x - \frac{1}{4x} \right)^2} \d x = \answer{\frac{1}{2}x^2 + \frac{1}{4} \ln(x)} + C
      \]
  \end{enumerate}
\end{problem}

\begin{problem}
   A student is asked to compute $\int x \sqrt{1 - x^2} \d x$ on a midterm and provides the following solution:
   \[
     \int x \sqrt{1 - x^2} \d x = \int x(1 - x) \d x = \int (x - x^2) \d x = \frac{1}{2}x^2 - \frac{1}{3} x^3 + C.
   \]
   
   Determine if the student is correct.
   If the student is not correct, choose the most appropriate response below.
   \begin{multipleChoice}
     \choice{The student is correct.}
     \choice{The student is incorrect: $\int x(1 - x) \d x = \frac{1}{2}x^2\left( x - \frac{1}{2}x^2 \right) + C$.}
     \choice[correct]{The student is incorrect: $\sqrt{1 - x^2} \ne 1 - x$.}
   \end{multipleChoice}
   \begin{problem}
     To compute this antiderivative, the student should perform a
     $u$-substitution with $u = \answer{1 - x^2}$.  Upon making this
     substitution, the correct antiderivative of $x \sqrt{1 - x^2}$ is
     $\answer{\frac{-1}{3} (1 - x^2)^{3/2}} + C$.
   \end{problem}
\end{problem}

\begin{problem}
  Here is an example of an application where working correctly with square roots�� arises.
  
  Consider the curve given by:
  \[
    y = \frac{1}{27} (9x^2 + 6)^{3/2} \text{ from $x = 0$ to $x = 3$}
  \]
  \begin{enumerate}
    \item Calculate the derivative $y'$.
      Simplify your answer as completely as possible.
      \[
        \frac{\d y}{\d x} = \answer{x \sqrt{9x^2 + 6}}
      \]
    
    \item Calculate $(\d y/ \d x)^2$.
      Simplify your answer completely.
      \[
        \left( \frac{\d y}{\d x} \right)^2 = \answer{9x^4 + 6x^2}
      \]
      
    \item The length of this curve is given by the formula:
    \[
      L = \int_0^3 \sqrt{1 + \left( \frac{\d y}{\d x} \right)^2} \d x
    \]
    
    Substitute the expression for $(\d y/ \d x)^2$ into the expression above.
    You should notice that the expression in the integrand is a perfect square!
    Indeed,
    \[
      1 + \left( \frac{\d y}{\d x} \right)^2 = (\answer{3x^2 + 1})^2
    \]
    so
    $\sqrt{1 + \left( \frac{\d y}{\d x} \right)^2} = (\answer{3x^2 + 1})$ (simplify completely).
    
    Evaluating the integral $L \int_0^3 \sqrt{1 + \left( \frac{\d y}{\d x} \right)^2} \d x$ now gives the length of the curve is $\answer{30}$.
    (Type an exact answer, using radicals as needed.)
  \end{enumerate}
\end{problem}


\end{document}
