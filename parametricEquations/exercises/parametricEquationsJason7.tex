\documentclass{ximera}

\newcommand{\RR}{\mathbb R}
\renewcommand{\d}{\,d}
\newcommand{\dd}[2][]{\frac{d #1}{d #2}}
\renewcommand{\l}{\ell}
\newcommand{\ddx}{\frac{d}{dx}}
\newcommand{\dfn}{\textbf}
\newcommand{\eval}[1]{\bigg[ #1 \bigg]}


\author{Jason Miller}
\license{Creative Commons 3.0 By-bC}


\outcome{}

\begin{document}
\begin{exercise}

Find a parametrization for the line segment connecting the point $(-2, 5)$ to the point $(3,-1)$. 

First, let's think of a general way to construct a parametrization for the line segment from a point $(x_{1}, y_{1})$ to a point $(x_{2}, y_{2})$. 

We construct a parametrization that starts at $(x_{1}, y_{1})$ at  $t=0$ and ends at $(x_{2}, y_{2})$ when $t=1$. 

Consider the parametrization

\begin{align*}
 x(t)&=x_{1}+t(x_{2}-x_{1}) \\
y(t)&=y_{1}+t(y_{2}-y_{1})
\end{align*}
where $0\leq t \leq 1$.

We check that when $t=0$ we get the point $(x_{1}, y_{1})$ and when $t=1$ we get the point $(x_{2}, y_{2})$. As $t$ increases from $0$ to $1$, these parametric curves trace out a line segment from the first point to the second. 

Now using this method, find a parametrization for the line segment connecting the point $(-2, 5)$ to the point $(3, -1)$. 

\begin{align*}
x(t)&=\answer{-2 +5t    } \\
y(t)&=\answer{ 5-6t   }
\end{align*}
for $0 \leq t \leq 1$. 

Note that other parametrizations are possible. We could trace the line segment in the backwards direction. We could alter the speed at which the segment is traced out in time, etc. 

\begin{exercise}

Let's think about parametrizations of the circle. 

First, think about parametrizing the circle of radius 1 centered at the origin $x^2+y^2=1$. 

One way to construct a parametrization is as follows:
\begin{align*}
x(t)&=\cos(t)\\
y(t)&=\sin(t)
\end{align*}
for $0 \leq t < 2\pi$. 

This parametrization starts at the point $(1,0)$ and traces the curve out counterclockwise. 
Now suppose we want to parametrize the circle of radius $a$ centered at the origin. So we just adjust our parametrization to be:

\begin{align*}
x(t)&=a\cos(t)\\
y(t)&=a\sin(t)
\end{align*}
for $0 \leq t < 2\pi$. 

We can only change the center of the circle to be any point $(x_{0}, y_{0})$ instead of $(0,0)$. 

\begin{align*}
x(t)&=x_{0} + a\cos(t)\\
y(t)&=y_{0}+ a\sin(t)
\end{align*}
for $0 \leq t < 2\pi$. 
This has the effect of translating the origin to the point $(x_{0}, y_{0})$ and the given parametric equations will trace out a circle of radius $a$ in a 
counterclockwise fashion around the new center, namely the circle $(x-x_{0})^2+(y-y_{0})^2=a^2$. 

We can also trace out the circle in a clockwise fashion by just using the same parametric equations but replacing $t$ with $-t$. If  you think of $t$ as effectively the angle coordinate, this means we are letting the angle become more and more negative.

We can also trace out the circle more quickly. We altered the radius of the circle by messing with the amplitude of the cosine and sine terms. We can alter how 
rapidly the circle is traced out by messing with frequency. 

For example, 

\begin{align*}
x(t)&=a\cos(2t)\\
y(t)&=a\sin(2t)
\end{align*}
for $0\leq t <\pi$

will trace out the curve twice as quickly as 

\begin{align*}
x(t)&=a\cos(t)\\
y(t)&=a\sin(t)
\end{align*}
for $0 \leq t < 2\pi$. 


Notice that if we wish to trace out the circle exactly once then if we trace it out more rapidly by using $2t$ in the argument of the cosine and sine then we have to shorten the time interval to $[0, \pi)$ instead of $[0, 2\pi)$ for the original case. 



Now consider again the following parametrization:

\begin{align*}
x(t)&=\cos(t)\\
y(t)&=\sin(t)
\end{align*}
for $0 \leq t < 2\pi$. 

Let's call the speed with which we trace out the circle with this parametrization the unit speed. 




Using all of the above information, find a parametrization of the circle $(x+13)^2+(y-7)^2=15$ that traces it out exactly once clockwise at half of unit speed. 

\begin{align*}
x(t)&=\answer{ -13+\sqrt{15}\cos\left(\frac{-t}{2}\right) }\\
y(t)&=\answer{ 7+\sqrt{15}\sin\left( \frac{-t}{2} \right) }
\end{align*}
for $0 \leq t < \answer{ 4 \pi    }$.


\end{exercise}
\end{exercise}
\end{document}
