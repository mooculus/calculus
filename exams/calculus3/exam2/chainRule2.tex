\documentclass{ximera}

\newcommand{\RR}{\mathbb R}
\renewcommand{\d}{\,d}
\newcommand{\dd}[2][]{\frac{d #1}{d #2}}
\renewcommand{\l}{\ell}
\newcommand{\ddx}{\frac{d}{dx}}
\newcommand{\dfn}{\textbf}
\newcommand{\eval}[1]{\bigg[ #1 \bigg]}


\author{Jim Fowler}
\license{Creative Commons 4.0 By-SA}

\begin{document}

\begin{exercise}
  
Suppose  $f : \R^2 \to \R$  is a function with 
$(\nabla f) \cdot \langle 1, 0 \rangle > 0$  and
$(\nabla f) \cdot \langle 0, 1 \rangle < 0$.

Suppose  $g : \R^2 \to \R$  is a function with  
$(\nabla g) \cdot \langle 1, 0 \rangle > 0$  and
$(\nabla g) \cdot \langle 0, 1 \rangle > 0$.


Suppose $h : \R^2 \to \R$  is a function with      
$(\nabla h) \cdot \langle 1, 0 \rangle < 0$ and
$(\nabla h) \cdot \langle 0, 1 \rangle > 0$.

Then
$\displaystyle\frac{\partial}{\partial s} f(g(s,t),h(s,t))$ is
\begin{multipleChoice}
  \choice{negative}
  \choice{zero}
  \choice[correct]{positive}
  \choice{unknowable from the given data.}      
\end{multipleChoice}

\begin{hint}
By the chain rule,
\begin{align*}
\frac{\partial}{\partial x} f(g(x,y),h(x,y))
&= \frac{\partial f}{\partial x_1} \frac{\partial g}{\partial x_1} + \frac{\partial f}{\partial x_2} \frac{\partial h}{\partial x_1} \\
&= \mbox{positive} \cdot \mbox{positive} + \mbox{negative} \cdot \mbox{negative} \\
&= \mbox{positive} + \mbox{positive} \\
&= \mbox{positive},
\end{align*}
\end{hint}
\begin{hint}
We conclude $\displaystyle\frac{\partial}{\partial x} f(g(x,y),h(x,y)) > 0$.
\end{hint}

\end{exercise}
\end{document}