\documentclass[handout,noauthor,nooutcomes]{ximera}

\author{Bart Snapp}

\newcommand{\RR}{\mathbb R}
\renewcommand{\d}{\,d}
\newcommand{\dd}[2][]{\frac{d #1}{d #2}}
\renewcommand{\l}{\ell}
\newcommand{\ddx}{\frac{d}{dx}}
\newcommand{\dfn}{\textbf}
\newcommand{\eval}[1]{\bigg[ #1 \bigg]}


\title[Collaborate:]{Quadric surfaces}

\begin{document}
\begin{abstract}
  We investigate quadric surfaces.
\end{abstract}
\maketitle

\textbf{Work in groups of 3--4, writing your answers on a separate
  sheet of paper.}


Consider $F(x,y)= 39 - 30 x + 5 x^2 + 2 y + y^2$. Here is a plot of
the quadric surface $z=F(x,y)$:
\begin{image}[3in]
  \includegraphics{surfacePlot1.jpg}
\end{image}
Here we see a contour plot for $z = F(x,y)$ with the same domain:
\begin{image}[4in]
  \includegraphics{contours1.jpg}
\end{image}
\begin{problem}
Here we see a table of values for $z= F(x,y)$ with the same domain:
\[
\renewcommand*{\arraystretch}{1.5}
\begin{array}{c|c|c|c|c|c|c|c|}\hline
\rule{.5cm}{0cm} & 77 & 42 & 17 & 2 & -3 & 2 & 17 \\ \hline
 & 74 & 39 & 14 & -1 & -6 & -1 & 14 \\ \hline
 & 73 & 38 & 13 & -2 & -7 & -2 & 13 \\ \hline
 & 74 & 39 & 14 & -1 & -6 & -1 & 14 \\ \hline
 & 77 & 42 & 17 & 2 & -3 & 2 & 17 \\ \hline
 & 82 & 47 & 22 & 7 & 2 & 7 & 22 \\ \hline
 & 89 & 54 & 29 & 14 & 9 & 14 & 29 \\ \hline
 &    &    &    &    &   &    & 
\end{array}
\]
The $x$-coordinates should run along the bottom of this table and the
$y$ coordinates should run along the left-hand side. Fill them in.
\end{problem}

\begin{problem}
Estimate the height of each of the level curves in the contour plot
for $F(x,y)$ above. \textbf{Write the heights directly on the contour
  plot.}
\end{problem}

\begin{problem}
Find $(x,y)$ such that $\grad F(x,y) = \vec{0}$. Mark this position on the
contour plot and on the table above. Call this point $\vec{c}$.
\end{problem}


\begin{problem}
  Compute the second degree Taylor polynomial for $F(x,y)$ centered at
  the point $\vec{c}$ found above.
\end{problem}



\begin{problem}
  Identify the surface described by $z = F(x,y)$ and find its
  extrema (if any). Confirm with the second derivative test.
\end{problem}





\newpage


Consider $G(x,y)= -7 - 2 x + x^2 - 12 y - 3 y^2$. Here is a plot of
the quadric surface $z=G(x,y)$:
\begin{image}[3in]
  \includegraphics{surfacePlot2.jpg}
\end{image}
Here we see a contour plot for $z = G(x,y)$ with the same domain:
\begin{image}[4in]
  \includegraphics{contours2.jpg}
\end{image}

\newpage


\begin{problem}
Here we see a table of values for $z= G(x,y)$ with the same domain:

\[
\renewcommand*{\arraystretch}{1.5}
\begin{array}{c|c|c|c|c|c|c|c|}\hline
\rule{.5cm}{0cm}& -19 & -22 & -23 & -22 & -19 & -14 & -7 \\\hline
& -4 & -7 & -8 & -7 & -4 & 1 & 8 \\\hline
& 5 & 2 & 1 & 2 & 5 & 10 & 17 \\\hline
& 8 & 5 & 4 & 5 & 8 & 13 & 20 \\\hline
& 5 & 2 & 1 & 2 & 5 & 10 & 17 \\\hline
& -4 & -7 & -8 & -7 & -4 & 1 & 8 \\\hline
& -19 & -22 & -23 & -22 & -19 & -14 & -7 \\\hline
  &    &    &    &    &   &    & 
\end{array}
\]

The $x$-coordinates should run along the bottom of this table and the
$y$ coordinates should run along the left-hand side. Fill them in.
\end{problem}

\begin{problem}
Estimate the height of each of the level curves in the contour plot
for $G(x,y)$ above. \textbf{Write the heights directly on the contour
  plot.}
\end{problem}

\begin{problem}
Find $(x,y)$ such that $\grad G(x,y) = \vec{0}$. Mark this position on the
contour plot and on the table above. Call this point $\vec{c}$.
\end{problem}


\begin{problem}
  Compute the second degree Taylor polynomial for $G(x,y)$ centered at
  the point $\vec{c}$ found above.
\end{problem}



These next problems are more difficult. As you learn more, you should
return to them and see if you can finish them off.


\begin{problem}
  Identify the surface described by $z = G(x,y)$ and find its
  extrema (if any). Confirm with the second derivative test.
\end{problem}


\begin{problem}
  Find three surfaces where the second derivative test is
  inconclusive. One where there is a local maximum, one with a local
  minimum, and one with a saddle.
\end{problem}


\end{document}
