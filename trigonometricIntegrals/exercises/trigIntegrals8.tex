\documentclass{ximera}

\newcommand{\RR}{\mathbb R}
\renewcommand{\d}{\,d}
\newcommand{\dd}[2][]{\frac{d #1}{d #2}}
\renewcommand{\l}{\ell}
\newcommand{\ddx}{\frac{d}{dx}}
\newcommand{\dfn}{\textbf}
\newcommand{\eval}[1]{\bigg[ #1 \bigg]}


\author{Jason Miller}
\license{Creative Commons 3.0 By-NC}


\outcome{ Recognize the patterns that appear in trigonometric integrals and use appropriate substitutions 
to compute them.}


\begin{document}
\begin{exercise}
Determine the integral

\[
\int \tan^{2}(x) \sec^{6}(x) \d x
\]

We should perform a substitution with $u=\answer{ \tan(x)}$. Hence $\d u= \answer{ \sec^{2}(x)} \d x$. 

\begin{exercise}
Rewriting our integral in terms of $u$ gives us 

\[
\int \answer{u^{2} (1+ u^{2})^{2} } \d u
\]
\begin{exercise}
Integrating and going back to $x$ we get:
\[
\int \tan^{2}(x) \sec^{6}(x) \d x= \answer{  \frac{\tan^{3}(x)}{3} +\frac{ \tan^{7}(x)}{7} + \frac{ 2 \tan^{5}(x)}{5} +C    }
\]
(Use $C$ for the constant of integration)
\end{exercise}
\end{exercise}
\end{exercise}
\end{document}
