\documentclass{ximera}

\newcommand{\RR}{\mathbb R}
\renewcommand{\d}{\,d}
\newcommand{\dd}[2][]{\frac{d #1}{d #2}}
\renewcommand{\l}{\ell}
\newcommand{\ddx}{\frac{d}{dx}}
\newcommand{\dfn}{\textbf}
\newcommand{\eval}[1]{\bigg[ #1 \bigg]}


\title[Refresh:]{Inequalities}
\author{Bart Snapp and Jim Talamo}
\begin{document}
\begin{abstract}
  Remember our facts about inequalities.
\end{abstract}
\maketitle

\begin{exercise}
Another way to interpret inequalities is more algebraic.

\begin{example}
Write the inequality $|2x-3|<9$ without using the absolute value sign.

\begin{explanation}
The end points of the interval will be the $x$-values for which either $2x-3 = 9$ or $2x-3=-9$.  Since we want the points for which $|2x-3|<9$, we want the points in between both endpoints, that is, we want all $x$ for which

\[
-9 < 2x -3 <9.
\]

We can solve this now.

\begin{align*}
-6 < 2x < 12 \\
-3 < x < 6 
\end{align*}

Note that when $x=-3$, we have that $|2x-3| = \answer{9}$ and when $x=6$, we have that $|2x-3| = \answer{9}$.

You can check for yourself that any $x$-value in between $x=-3$ and $x=6$ will also satisfy the inequality.

\end{explanation}
\end{example}

\begin{problem}
  Write the inequality $|3x - 5| \le 1$ without using the absolute value sign.
  \begin{multipleChoice}
    \choice{$x \ge 4/3$ or $x \le 2$}
    \choice[correct]{$4/3 \le x \le 2$}
    \choice{$3x - 5 = \pm 1$}
    \choice{$x \le 4/3$ or $x \ge 2$}
  \end{multipleChoice}
\end{problem}

\begin{problem}
  Write the inequality $|x + 2| > 6$ without using the absolute value sign.
  \begin{multipleChoice}
    \choice{$x + 2 > 6$}
    \choice{$x + 2 = \pm 6$}
    \choice[correct]{$x + 2 < -6$ or $x + 2 > 6$}
    \choice{$x > -8$ or $x < 4$}
  \end{multipleChoice}
\end{problem}

\end{exercise}

\end{document}
