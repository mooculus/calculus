\documentclass{ximera}

\newcommand{\RR}{\mathbb R}
\renewcommand{\d}{\,d}
\newcommand{\dd}[2][]{\frac{d #1}{d #2}}
\renewcommand{\l}{\ell}
\newcommand{\ddx}{\frac{d}{dx}}
\newcommand{\dfn}{\textbf}
\newcommand{\eval}[1]{\bigg[ #1 \bigg]}



%\outcome{Find tangent lines to curves defined parametrically}
\author{Alex Beckwith}

\begin{document}
\begin{exercise}

Suppose that a curve $C$ is defined by the parametric equations:
\[
\begin{cases}
x(t) &=  t^3-3t+1\\
y(t) &= 3t^2+5t
\end{cases} , \textrm{ for } -\infty < t < \infty
\]
Find the derivative $\frac{dy}{dx}$ in terms of $t$.

\[
\frac{dy}{dx} = \frac{\answer{6t+5}}{\answer{3t^2-3}}
\]

\begin{hint}
Using the chain rule allows us to write $\frac{dy}{dx} = \frac{dy/dt}{dx/dt}$.
\end{hint}

%%%Horizontal Tangent Lines%%%%%%
\begin{exercise}
How many vertical tangent lines does the curve $C$ have?
\begin{multipleChoice}
\choice{$0$}
\choice{$1$}
\choice[correct]{$2$}
\choice{$3$}
\end{multipleChoice}

The vertical tangent lines occur when $t=\answer{-1}$ and $t=\answer{1}$.

\begin{exercise}
When $t=-1$, the vertical tangent line is $x=\answer{3}$.

When $t=1$, the vertical tangent line is $x=\answer{-1}$.
\end{exercise}

%%%vertical Tangent Lines%%%%%%
\begin{exercise}
Next we want the equation of the tangent line to $C$ at $(1,0)$. The value of $t$ for which $(x(t),y(t))=(1,0)$ is $t=\answer{0}$, and the equation of the tangent line there is $y= \answer{-\frac{5}{3}}x+\answer{\frac{5}{3}}$.
\end{exercise}
\end{exercise}
\end{exercise}
\end{document}
