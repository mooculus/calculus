\documentclass{ximera}

\newcommand{\RR}{\mathbb R}
\renewcommand{\d}{\,d}
\newcommand{\dd}[2][]{\frac{d #1}{d #2}}
\renewcommand{\l}{\ell}
\newcommand{\ddx}{\frac{d}{dx}}
\newcommand{\dfn}{\textbf}
\newcommand{\eval}[1]{\bigg[ #1 \bigg]}


\outcome{Identify determinate and indeterminate forms.}
\outcome{Distinguish between determinate and indeterminate forms.}


\title[Break-Ground:]{Arithmetic of large and small}

\begin{document}
\begin{abstract}
Two young mathematicians investigate the arithmetic of large
and small numbers.
\end{abstract}
\maketitle


Check out this dialogue between two calculus students (based on a true
story):


HERE IT IS 0/0 means anything.


\begin{dialogue}
\item[Devyn] Hey Riley, remember when we did stuff with
  \textit{numbers} in math class?
\item[Riley] Oh, the glory years! Now all we have are $a$, $b$, $c$, and $x$, $y$, $z$.
\item[Devyn] I know! Here's a crazy idea, let's replace those $a$,
  $b$, $c$'s, and $x$, $y$, $z$'s with ``large'' and ``small.''
\item[Riley] This is a new level of strange. Wow. So you could do things like:
  \[
  \text{large}+\text{small}\qquad\text{and}\qquad \frac{\text{small}}{\text{large}}
  \]
\item[Devyn] And maybe
  \[
  \text{large}+\text{small} = \text{large}
  \]
  because the answer is still ``large.''
\item[Riley] Right! And maybe
  \[
  \frac{\text{small}}{\text{large}} = \text{small}
  \]
  because a small number divided by a large number is even smaller, and so it will still be ``small.''
\item[Devyn] And now
  \[
  \frac{\text{large}}{\text{large}} = \dots
  \]
  hmmmm.
  \item[Riley] Yeah, I'm not sure here. Because even if two numbers are
    ``large'' one may be way larger than the other.
\end{dialogue}



\begin{problem}
  Using the context above, 
  \[
  \text{small}+\text{small} = ?
  \]
  \begin{multipleChoice}
  \choice{``large''}
  \choice[correct]{``small''}
  \choice{impossible to say}
  \end{multipleChoice}
\end{problem}


\begin{problem}
  Using the context above, 
  \[
  \text{large}\times\text{large} = ?
  \]
  \begin{multipleChoice}
  \choice[correct]{``large''}
  \choice{``small''}
  \choice{impossible to say}
  \end{multipleChoice}
\end{problem}

\begin{problem}
  Using the context above, 
  \[
  \frac{\text{large}}{\text{small}} = ?
  \]
  \begin{multipleChoice}
  \choice[correct]{``large''}
  \choice{``small''}
  \choice{impossible to say}
  \end{multipleChoice}
\end{problem}

\begin{problem}
  Using the context above, 
  \[
  \frac{\text{small}}{\text{small}} = ?
  \]
  \begin{multipleChoice}
  \choice{``large''}
  \choice{``small''}
  \choice[correct]{impossible to say}
  \end{multipleChoice}
\end{problem}

%% \begin{xarmaBoost}
%%   Write down at least \textbf{five} questions for this lecture. After
%%   you have your questions, label them as ``Level 1,'' ``Level 2,'' or
%%   ``Level 3'' where:
%% \begin{description}
%% \item[Level 1] Means you know the answer, or know exactly how to do
%%   this problem.
%% \item[Level 2] Means you think you know how to do the problem.
%% \item[Level 3] Means you have no idea how to do the problem.
%% \end{description}
%% \begin{freeResponse}
%% \end{freeResponse}
%% \end{xarmaBoost}



\end{document}
