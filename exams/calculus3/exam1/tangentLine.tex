\documentclass{ximera}

\newcommand{\RR}{\mathbb R}
\renewcommand{\d}{\,d}
\newcommand{\dd}[2][]{\frac{d #1}{d #2}}
\renewcommand{\l}{\ell}
\newcommand{\ddx}{\frac{d}{dx}}
\newcommand{\dfn}{\textbf}
\newcommand{\eval}[1]{\bigg[ #1 \bigg]}


\author{Bart Snapp}

\begin{document}
Consider the paraboloid $z = x^2+3y^2$ and the plane $z=x+y+4$:
\begin{image}[1.5in]
  \includegraphics[width=1.5in]{planeAndPara.jpg}
\end{image}
These surfaces intersect at a curve that we will call $\vec{p}(t)$.
\begin{problem}
  Find a vector \textbf{normal to the plane} at the point
  $(2,1,7)$. Justify your answer.
  \begin{prompt}
    Vector $\vec{n} = \vector{\answer{1},\answer{1},-1}$ is normal to
    the plane. We can read this off the coefficients for the equation
    of the plane.
  \end{prompt}
\end{problem}


\begin{problem}
  Find a vector \textbf{normal to the paraboloid} at the point
  $(2,1,7)$. Justify your answer.
  \begin{prompt}
    Viewing our surface as the level surface
    \[
    0 = \answer{x^2+3y^2}-z
    \]
    we can simply take the gradient of
    \[
    F(x,y,z) = \answer{x^2+3y^2}-z
    \]
    to find $\grad F(x,y,z) =
    \vector{\answer{2x},\answer{6y},\answer{-1}}$. And at $(2,1,7)$ we have another normal vector
      \[
      \vec{m} = \vector{\answer{4},\answer{6},\answer{-1}}.
      \]
  \end{prompt}
\end{problem}

\begin{problem}
  Use your work above to find a \textbf{vector} tangent to the curve
  $\vec{p}(t)$ at the point $(2,1,7)$.
  \begin{prompt}
    Compute $\vec{m}\cross\vec{n}$ to find:
    \[
    \vec{t} = \vector{\answer{5},-3,\answer{2}}
    \]
  \end{prompt}
\end{problem}

\begin{problem}
  Write a vector-valued function that describes a \textbf{line}
  tangent to $\vec{p}(t)$ at $(2,1,7)$.
  \begin{prompt}
    \[
    \vecl(t) = \vector{\answer{2+5t},1-3t,\answer{7+2t}}
    \]
  \end{prompt}
\end{problem}
\end{document}
  
