\documentclass{ximera}

\newcommand{\RR}{\mathbb R}
\renewcommand{\d}{\,d}
\newcommand{\dd}[2][]{\frac{d #1}{d #2}}
\renewcommand{\l}{\ell}
\newcommand{\ddx}{\frac{d}{dx}}
\newcommand{\dfn}{\textbf}
\newcommand{\eval}[1]{\bigg[ #1 \bigg]}


\author{Jim Talamo}
\license{Creative Commons 3.0 By-bC}


\outcome{}


\begin{document}
\begin{exercise}

Given a sequence $\{a_n\}_{n=1}$, there are two limits that can be constructed from it that play an important role later on.  This exercise gives practice constructing and computing them.

Consider the sequence $\{a_n \}_{n=1}$, where $a_n =n \cdot 2^n$.  Then:
\[
\lim_{n \to \infty} \frac{a_{n+1}}{a_n} = \answer{2}
\]

\begin{hint}
The limit that must be computed is:

\[
\lim_{n \to \infty} \frac{a_{n+1}}{a_n} = \lim_{n \to \infty} \frac{\answer{(n+1) 2^{n+1}}}{\answer{n 2^n}}
\]

\end{hint}

\[
\lim_{n \to \infty} \sqrt[n]{a_n} = \answer{2}
\]
\begin{hint}
The limit that must be computed is:

\[
\lim_{n \to \infty} \sqrt[n]{a_n} = \lim_{n \to \infty} \sqrt[n]{n \cdot 2^n} 
\]
Simplifying this gives:
\[
\lim_{n \to \infty} \sqrt[n]{a_n} = \lim_{n \to \infty} \answer{2} \sqrt[n]{\answer{n}} 
\]

To compute the limit, set $L = \lim_{n \to \infty} \sqrt[n]{a_n}$, take the natural logarithm of both sides and use the properties of logarithms to bring the limit into a form where L'Hopital's rule applies.
\end{hint}
\end{exercise}
\end{document}
