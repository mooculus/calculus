\documentclass{ximera}

\newcommand{\RR}{\mathbb R}
\renewcommand{\d}{\,d}
\newcommand{\dd}[2][]{\frac{d #1}{d #2}}
\renewcommand{\l}{\ell}
\newcommand{\ddx}{\frac{d}{dx}}
\newcommand{\dfn}{\textbf}
\newcommand{\eval}[1]{\bigg[ #1 \bigg]}


\outcome{Explain the relationship between differentiability and continuity.}
\outcome{Determine whether a piecewise function is differentiable.}
\title[Dig-In:]{Differentiability implies continuity}

\begin{document}
\begin{abstract}
We see that if a function is differentiable at a point, then it must
be continuous at that point.
\end{abstract}
\maketitle

There are connections between continuity and differentiability.

\begin{theorem}[Differentiability Implies Continuity]\index{differentiability implies continuity}
If $f$ is a differentiable function at $x = a$, then $f$ is continuous
at $x=a$.
\begin{explanation}
To explain why this is true, we are going to use the following
definition of the derivative
\[
f'(a) = \lim_{x\to a} \frac{f(x)-f(a)}{x-a}.
\]

  Assuming that $f'(a)$ exists, we want to show that $f(x)$ is
continuous at $x=a$, hence we must show that
\[
\lim_{x\to a} f(x) = f(a).
\]
Starting with
\[
\lim_{x\to a} \left(f(x) - f(a)\right)
\]
we multiply and divide by $(x-a)$ to get
\begin{align*}
  &= \lim_{x\to a} \left((x-a)\frac{f(x) - f(a)}{x-a}\right) \\
  &= \left(\lim_{x\to a} (x-a) \right) \left(\lim_{x\to a}\frac{f(x) - f(a)}{x-a}\right) &\text{Limit Law.} \\
  &= \answer[given]{0}\cdot f'(a) = \answer[given]{0}.
\end{align*}
Since 
\[
\lim_{x\to a}\left(f(x) - f(a)\right) = 0 
\]
we see that $\lim_{x\to a} f(x) = f(a)$, and so $f$ is continuous at
$x=a$.
\end{explanation}
\end{theorem}

This theorem is often written as its contrapositive:
\begin{quote}
If $f(x)$ is not continuous at $x=a$, then $f(x)$ is not
differentiable at $x=a$.
\end{quote}


Thus from the theorem above, we see that all differentiable functions
on $\RR$ are continuous on $\RR$. Nevertheless there are continuous
functions on $\RR$ that are not differentiable on $\RR$.

\begin{question}
  Which of the following functions are continuous but not
  differentiable on $\RR$?
  \begin{selectAll}
    \choice{$x^2$}
    \choice{$\lfloor x \rfloor$}
    \choice[correct]{$|x|$}
    \choice{$\frac{\sin(x)}{x}$}
  \end{selectAll}
\end{question}

\begin{example}
  Consider
  \[
  f(x) = \begin{cases}
          x^2 &\text{if $x<3$,}\\
          mx+b &\text{if $x\ge 3$.}
         \end{cases}
  \]
  What values of $m$ and $b$ make $f$ differentiable at $x=3$?
  \begin{explanation}
    To start, we know that we must make $f$ both continuous and
    differentiable. Hence, we must ensure that the value of both
    pieces of $f$ agree at $x=3$. Write with me
    \begin{align*}
      \eval{x^2}_{x=3} &= \eval{mx+b}_{x=3}\\
      9 &= m\cdot 3 + b.
    \end{align*}
    Now we must ensure that the derivatives of each piece of $f$ agree
    at $x=3$. Write with me
    \begin{align*}
    \ddx x^2 &= \lim_{h\to 0}\frac{(x+h)^2-x^2}{h}\\
    &= \lim_{h\to 0}\frac{x^2 + 2xh + h^2 - x^2}{h}\\
    &= \lim_{h\to 0}\frac{2xh + h^2}{h}\\
    &= \lim_{h\to 0}\left(2x + h\right)\\
    &=2x.
    \end{align*}
    Moreover,
    \[
    \ddx(mx+ b) = m
    \]
    by the definition of a tangent line. Hence we must have
    \begin{align*}
      \eval{\ddx x^2}_{x=3} &= \eval{\ddx(mx+b)}_{x=3}\\
      \eval{2x}_{x=3} &= \eval{m}_{x=3}\\
      6 &= m.
    \end{align*}
    Ah! So now
    \begin{align*}
      9 &= m\cdot 3 + b\\
      9 &= 6\cdot 3 + b\\
      9 &= 18 + b,
    \end{align*}
    so $b=-9$. Thus setting $m=6$ and $b=-9$ will give us a continuous
    and differentiable piecewise function.
  \end{explanation}
\end{example}


\end{document}
