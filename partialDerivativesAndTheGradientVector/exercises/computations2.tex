\documentclass{ximera}

\newcommand{\RR}{\mathbb R}
\renewcommand{\d}{\,d}
\newcommand{\dd}[2][]{\frac{d #1}{d #2}}
\renewcommand{\l}{\ell}
\newcommand{\ddx}{\frac{d}{dx}}
\newcommand{\dfn}{\textbf}
\newcommand{\eval}[1]{\bigg[ #1 \bigg]}



\author{Jim Talamo}

\outcome{Compute the partial derivative of an expression.}
\newcommand{\Pp}[2]{\frac{\partial #1}{\partial #2}}

\begin{document}

This exercise gives mechanical practice calculating partial derivatives for exponentials, logarithms, and trigonometric functions.
\begin{exercise}


If $f(x,y) = \ln(2x+3y^2)$, then $\Pp{f}{x} = \answer{\frac{2}{2x+3y^2}}$.

\begin{hint}
To compute the partial derivative with respect to $x$, we will treat all other variables as constants and differentiate expressions that explicitly depend on $x$ the same way we would before.

\[
\Pp{f}{x} = \pp{x}\left[ \ln(2x+3y^2) \right] =\frac{1}{2x+3y^2} \cdot \pp{x}\left[ 2x+3y^2 \right] =\frac{1}{2x+3y^2} \cdot \left( \answer{2} \right) .
\]
\end{hint}

\end{exercise}

%%%%%%%%%%%%%%%%%%%%%

\begin{exercise}
If $f(x,y) = \sin\left(\frac{x}{y}\right)$, then 

\begin{itemize}
\item $f_x(x,y) = \answer{\cos\left(\frac{x}{y}\right) \cdot \frac{1}{y}}$
\item $f_y(x,y) = \answer{- \frac{x}{y^2} \cdot \cos\left(\frac{x}{y}\right) }$
\end{itemize}

\end{exercise}

%%%%%%%%%%%%%%%%%%%%%

\begin{exercise}
If $f(x,y) = e^{4x^2+3y}$, then 

\begin{itemize}
\item $f_x(x,y) = \answer{8xe^{4x^2+3y}}$
\item $f_y(x,y) = \answer{3e^{4x^2+3y}}$
\end{itemize}


\end{exercise}

\end{document}
