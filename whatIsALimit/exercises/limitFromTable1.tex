\documentclass{ximera}

\newcommand{\RR}{\mathbb R}
\renewcommand{\d}{\,d}
\newcommand{\dd}[2][]{\frac{d #1}{d #2}}
\renewcommand{\l}{\ell}
\newcommand{\ddx}{\frac{d}{dx}}
\newcommand{\dfn}{\textbf}
\newcommand{\eval}[1]{\bigg[ #1 \bigg]}


\outcome{Consider values of a function at inputs approaching a given point.}
\outcome{Understand the concept of a limit.}
\outcome{Identify when a limit does not exist.}

\author{Nela Lakos \and Kyle Parsons}

\begin{document}
\begin{exercise}

Complete the following table.  Use \textbf{exact} values.

\[
\begin{array}{|c|c|c|c|c|c|c|c|c|}
\hline
x & \frac{1}{3} & \frac{1}{30} & \frac{1}{300} & \frac{1}{301} & \frac{1}{1000} & \frac{1}{1001} & \frac{1}{50000} & \frac{1}{100004}\\\hline
\frac{1}{x} & 3 & 30 & 300 & 301 & 1000 & 1001 & 50000 & 100004 \\\hline
\end{array}
\]

\begin{exercise}
Based on the table above, make a conjecture about $\lim_{x\to 0^{+}}\frac{1}{x}$.  Does it exist? Explain.
\begin{freeResponse}
It seems like the limit does not exist.  The values $\frac{1}{x}$ don't seem to approach any number, even when $x$ is very close to 0.  It seems that these values just grow and grow and keep growing.
\end{freeResponse}
\end{exercise}

\end{exercise}
\end{document}