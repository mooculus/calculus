\documentclass{ximera}

\newcommand{\RR}{\mathbb R}
\renewcommand{\d}{\,d}
\newcommand{\dd}[2][]{\frac{d #1}{d #2}}
\renewcommand{\l}{\ell}
\newcommand{\ddx}{\frac{d}{dx}}
\newcommand{\dfn}{\textbf}
\newcommand{\eval}[1]{\bigg[ #1 \bigg]}


\author{Jason Miller and Jim Talamo}
\license{Creative Commons 3.0 By-NC}


\outcome{Evaluate integrals using integration by parts, including mulitple iterations}


\begin{document}
\begin{exercise}
Determine the definite integral 

\[
\int_0^{\pi} x\sin(x) \cos(x) \d x = \answer{-\frac{1}{4} \pi}
\]

\begin{hint}
Since it is generally nice to let $u$ be the polynomial when we have one, let $u = x$ and $\d v = \sin(x) \cos(x)$.  We need to integrate $\sin(x) \cos(x)$, and this can be done at least two different ways.

\begin{itemize}
\item One way is to let $w=\sin(x)$.  Then, $\d w = \answer{\cos(x)} \d x$ and $\int \sin(x) \cos(x) \d x = \int \answer{w} \d w$.  Working through this substitution gives

\[
\int \sin(x) \cos(x) \d x = \frac{1}{2}\sin^2(x)+C.
\]
 
(Note: a similar substitution with $w=\cos(x)$ also will work.)

Using integration by parts, the \emph{indefinite} integral $\int x\sin(x) \cos(x) \d x$ can be found.

\[
\int x\sin(x) \cos(x) \d x =\frac{1}{2}x\sin^2(x) - \int \frac{1}{2} \sin^2(x) \d x. 
\]

To compute the integral on the righthand side, the trig identity

\[
\sin^2(x) = \frac{1}{2} - \frac{1}{2} \cos(2x)
\] 

will be helpful.  Once you find the indefinite integral, you may use Fundamental Theorem of Calculus to find $\int_0^{\pi} x\sin(x) \cos(x) \d x $.

 \item Another way is to use the identity $\sin(2x) = \frac{1}{2} \sin(x)\cos(x)$ to write $\int  \sin(x)\cos(x) \d x = \int \frac{1}{2} \sin(2x) \d x = \answer{-\frac{1}{4}\cos(2x)}$.

Using integration by parts, the \emph{indefinite} integral $\int x\sin(x) \cos(x) \d x$ can be found.

\[
\int x\sin(x) \cos(x) \d x =-\frac{1}{4}x\cos(2x) + \int \frac{1}{4} \cos(2x) \d x. 
\]

Once you find the indefinite integral, you may use Fundamental Theorem of Calculus to find $\int_0^{\pi} x\sin(x) \cos(x) \d x $.

\end{itemize}
 
\end{hint}


\end{exercise}
\end{document}
