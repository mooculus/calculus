\documentclass{ximera}

\newcommand{\RR}{\mathbb R}
\renewcommand{\d}{\,d}
\newcommand{\dd}[2][]{\frac{d #1}{d #2}}
\renewcommand{\l}{\ell}
\newcommand{\ddx}{\frac{d}{dx}}
\newcommand{\dfn}{\textbf}
\newcommand{\eval}[1]{\bigg[ #1 \bigg]}


\begin{document}
\begin{exercise}

\outcome{Use limits to find the slope of the tangent line at a point.}
\outcome{Understand the definition of the derivative at a point.}
Using the \textbf{definition of the derivative}, fill in the blanks:
  \[
  f'(a)=
  \lim_{h\to\answer[format=integer]{0}}
  \frac{f(\answer{a+h})-f(\answer{a})}{\answer{h}}
  \]

  \begin{validator}[(first*first*first - second*second*second)/third==3*a^2+3*a*h+h^2]
  \[
  f'(a)=
  \lim_{h\to 0}
  \frac{f(\answer[id=first]{a+h})-f(\answer[id=second]{a})}{\answer[id=third]{h}}
  \]
  \end{validator}
  %% TO DO THIS CORRECTLY, SEE https://ximera.osu.edu/course/kisonecat/ximeraSample/sample
    
  %% \lim_{\answer1{} \to \answer2{}} should be such that \answer1
  %% records the variable entered and validates when it is NOT an
  %% integer. The second answer should then validate if 0 or a is
  %% entered. An ``if'' statement must occur, detecting whether 0 or a
  %% was entered. In the first case, the following answers check
  %% whether the inputs are in the obvious form and in the latter it
  %% detects if the next inputs are in the obvious form
\end{exercise}

  \begin{exercise}
    Let $f(x)=\sqrt{5+x^2}$. Use the limit definition of the derivative
    above to find $f'(2)$.
    \[
    f'(2)=\answer{\frac{2}{3}}
    \]
  \end{exercise}

\end{document}
