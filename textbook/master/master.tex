\documentclass{ximera}
%% handout
%% nohints
%% newpage

\newcommand{\RR}{\mathbb R}
\renewcommand{\d}{\,d}
\newcommand{\dd}[2][]{\frac{d #1}{d #2}}
\renewcommand{\l}{\ell}
\newcommand{\ddx}{\frac{d}{dx}}
\newcommand{\dfn}{\textbf}
\newcommand{\eval}[1]{\bigg[ #1 \bigg]}


\usepackage[framemethod=TikZ]{mdframed}

\usepackage{geometry}
\geometry{landscape, 
  left=1in,
  top=1in,
  textheight=6.5in,
  textwidth=5in,
  marginparsep=0.5in,
  marginparwidth=2.5in}


%% %% Attempting to add a ``notes'' in margin
%% %\usepackage[some]{background} %% remove the [some] to have the header
%% \usepackage{background} %% remove the [some] to have the header
%% %% Code for the header
%% \backgroundsetup{
%%   opacity=1,
%%   color=black,
%%   position={19,-8cm},
%%   angle=0,scale=1,
%%   firstpage=true, 
%%   contents={
%%     \begin{tikzpicture}[x=1cm, y=1cm]
%%       \node at (5, 16.5) {\Large Notes:};
%%       \draw[step=5mm, line width=0.1mm, black!40!white] (0,0) grid (10,16);
%%      \end{tikzpicture}
%%      }}
%% %% End code of notes






%% Allows for a wide text/regular text
\renewcommand{\fullwidth}{\newgeometry{textwidth=10cm,textheight=10cm}}
\renewcommand{\normalwidth}{\restoregeometry}
%% End


%% Allows us to combine documents with document classes
\let\masterdocument\document
\let\endmasterdocument\enddocument
\let\otherinput\input %% Numbering starts a page too soon without this
\newcommand{\masterinput}[1]{\otherinput{#1}\clearpage} %% Numbering starts a page too soon without this
%% End combo hack



%% This is the code that will allow us to redefine the title with
%% \let\maketitle\makesectiontitle

%% \makeatletter
%% \newcommand\makesectiontitle{
%%   \addtocounter{titlenumber}{1}\addcontentsline{toc}{subsection}{\thetitlenumber~\@title} %% puts titles in the toc
%%   {\flushleft\large\bfseries \@pretitle\par\vspace{-1em}}%
%%   {\flushleft\LARGE\bfseries {\ifnumbers\thetitlenumber\fi}{\ifnumbers\hspace{1em}\else\hspace{0em}\fi}\@title \par }
%%   \vskip .6em\noindent\textit\theabstract\setcounter{problem}{0}\setcounter{subsection}{0}\par\vspace{2em}
%%   \ifnooutcomes\else\ifhandout\else\let\thefootnote\relax\footnote{Learning outcomes: \theoutcomes}\fi\fi
%%   \aftergroup\@afterindentfalse
%%   \aftergroup\@afterheading}
%% \makeatother




\begin{masterdocument}
%% Redefines "explanation"
\renewmdenv[outerlinewidth=2,topline=false, bottomline=false, leftline=true, rightline=false, 
leftmargin=40,innertopmargin=0pt,innerbottommargin=0pt,skipbelow=\baselineskip,
outerlinecolor=textColor,fontcolor=textColor,backgroundcolor=background]{explanation}%
%%
  


\begin{titlepage}
% \NoBgThispage %(Makes it barf currently)

    \vspace*{1in}
    \begin{center}
      \Huge \bfseries Textbook\\[0.5cm]
      \normalsize This document was typeset on \today.
    \end{center}
    %\vspace*{\fill}
\end{titlepage}

  \tableofcontents
  \setcounter{page}{2}

\cleardoublepage
  
\renewcommand{\documentclass}{\setbox0\vbox}
\renewcommand{\input}{\setbox0\vbox}
\renewenvironment{document}{\renewcommand{\input}{\masterinput}
}{\renewcommand{\input}{\setbox0\vbox}}
%% Understanding Functions
%% BreakGround
\masterinput{../understandingFunctions/breakGround.tex}\def\theoutcomes{}
%% DigIns
\masterinput{../understandingFunctions/digInForEachInputExactlyOneOutput.tex}\def\theoutcomes{}
\masterinput{../understandingFunctions/digInOperationsAndCompositionsOfFunctions.tex}\def\theoutcomes{}
\masterinput{../understandingFunctions/digInInversesOfFunctions.tex}\def\theoutcomes{}
\masterinput{../understandingFunctions/digInAWordOnNotation.tex}\def\theoutcomes{}



%% What is a limit
\masterinput{../whatIsALimit/breakGround.tex}\def\theoutcomes{}
\masterinput{../whatIsALimit/digInWhatIsALimit.tex}\def\theoutcomes{}

%% Limit Laws
\masterinput{../limitLaws/breakGround.tex}\def\theoutcomes{}
\masterinput{../limitLaws/digInLimitLaws.tex}\def\theoutcomes{}
\masterinput{../limitLaws/digInContinuity.tex}\def\theoutcomes{}
\masterinput{../limitLaws/digInTheSqueezeTheorem.tex}\def\theoutcomes{}

%% Indeterminant forms
\masterinput{../indeterminateForms/breakGround.tex}\def\theoutcomes{}

%% Using limits to detect asymptotes

%% The intermediate value theorem
\masterinput{../continuity/breakGround.tex}\def\theoutcomes{}
\masterinput{../continuity/digInTheIntermediateValueTheorem.tex}\def\theoutcomes{}

%% Overview of limits
\masterinput{../overviewOfLimits/breakGround.tex}\def\theoutcomes{}
\masterinput{../overviewOfLimits/digInInstantaneousVelocity.tex}\def\theoutcomes{}
\masterinput{../overviewOfLimits/digInThePreciseDefinitionOfALimit.tex}\def\theoutcomes{}


%% Definition of the derivative
\masterinput{../definitionOfTheDerivative/breakGround.tex}\def\theoutcomes{}
\masterinput{../definitionOfTheDerivative/digInTheDerivativeViaLimits.tex}\def\theoutcomes{}
\masterinput{../definitionOfTheDerivative/digInDifferentiabilityImpliesContinuity.tex}\def\theoutcomes{}


%% Product and quotient rules
\masterinput{../productAndQuotientRules/breakGround.tex}\def\theoutcomes{}

%% Applied optimization
\masterinput{../appliedOptimization/breakGround.tex}\def\theoutcomes{}

%% Derivatives of inverse functions

\masterinput{../derivativesOfInverseFunctions/breakGround.tex}\def\theoutcomes{}


\masterinput{../moreThanOneRate/breakGround.tex}\def\theoutcomes{}
\masterinput{../lhopitalsRule/breakGround.tex}\def\theoutcomes{}

\end{masterdocument}
