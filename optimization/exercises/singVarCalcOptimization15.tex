\documentclass{ximera}
\newcommand{\RR}{\mathbb R}
\renewcommand{\d}{\,d}
\newcommand{\dd}[2][]{\frac{d #1}{d #2}}
\renewcommand{\l}{\ell}
\newcommand{\ddx}{\frac{d}{dx}}
\newcommand{\dfn}{\textbf}
\newcommand{\eval}[1]{\bigg[ #1 \bigg]}

\author{Bart Snapp}
\license{Creative Commons 3.0 By-NC}
\acknowledgement{https://www.whitman.edu/mathematics/calculus/}
  \outcome{Interpret an optimization problem as the procedure used to make a system or design as effective or functional as possible.}
  \outcome{Set up an optimization problem by identifying the objective function and appropriate constraints.}
  \outcome{Solve optimization problems by finding the appropriate absolute extremum.}
  \outcome{Solve basic word problems involving maxima or minima.}
\begin{document}
\begin{exercise}

 A trough is constructed with the following dimensions:
  \begin{image}
    \begin{tikzpicture}[scale=.5]
      \draw[penColor, very thick] (-2,4) -- (2,3.5) -- (0,0)--cycle;
      \draw[penColor, very thick] (0,0) -- (9,1) -- (11,4.5) --(2,3.5);
      \draw[penColor,very thick] (-2,4) -- (7,5) -- (7.45,4.11);
      \draw[penColor,very thick] (11,4.5) -- (7,5);
      \draw[penColor,very thick,dashed] (7.45,4.11) -- (9,1);
      \node at (9,5.1) {$x$};
      \node at (3,5) {$64$};
      \node at (-1.5,2) {$2$};
      \node at (1.5,1.8) {$2$};
    \end{tikzpicture}
  \end{image}
  Find $x$ that maximizes the volume.
  \begin{hint}
  Let V be the volume of the trough. Then
  
  V=(Area of the isosceles triangle)$\cdot 64$
  \end{hint}
  \begin{hint}
  The two bases of the trough have the shape of an isosceles triangle.
   \begin{image}
    \begin{tikzpicture}[scale=.5]
      \draw[penColor,  thick] (-2,4) -- (2,4) -- (0,0)--cycle;
    \draw[penColor2, dotted,thin] (0,0) -- (0,4) --(0,0)--cycle;
      \node at (0,4.2) {$x$};
       \node at (0.2,2.5) {$h$};
      \node at (-1.5,2) {$2$};
      \node at (1.5,1.8) {$2$};
    \end{tikzpicture}
  \end{image}

   Let A be the area of the isosceles triangle. Then
  $A=\frac{1}{2}\cdot x\cdot h$,
  
  and
  
  
$A=\frac{1}{2}\cdot x\cdot\sqrt{4-\frac{x^2}{4}}$
  \end{hint}
  \begin{prompt}
  \[
  x = \answer{2\sqrt{2}}
  \]
  \end{prompt}
\end{exercise}
\end{document}
