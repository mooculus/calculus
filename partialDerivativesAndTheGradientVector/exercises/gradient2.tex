\documentclass{ximera}

\newcommand{\RR}{\mathbb R}
\renewcommand{\d}{\,d}
\newcommand{\dd}[2][]{\frac{d #1}{d #2}}
\renewcommand{\l}{\ell}
\newcommand{\ddx}{\frac{d}{dx}}
\newcommand{\dfn}{\textbf}
\newcommand{\eval}[1]{\bigg[ #1 \bigg]}



\author{Jim Talamo}

\outcome{Calculate the gradient.}
\outcome{Practice mechanics for finding critical points.}

\begin{document}

\begin{exercise}
Suppose that $F(x,y) =x^3+3xy$.  Then,
\[
\grad{F}(x,y) = \vector{\answer{3x^2+3y},\answer{3x}}.
\]

\begin{exercise}
How many distinct points $(x,y)$ are there for which $\grad{F}(x,y) = \vec{0}$?

\begin{multipleChoice}
\choice{none}
\choice[correct]{one}
\choice{two}
\choice{more than two, but finitely many}
\choice{infinitely many}
\end{multipleChoice}

\begin{hint}
If $\grad{F}(x,y) = \vec{0}$, we must have that the $x$ and $y$
components are zero simultaneously.  Note that the only way the
$y$-component can be zero occurs when $x=0$.  How many options does
this leave for the $x$-component?
\end{hint}

\begin{exercise}
The point $(x,y)$ for which $\grad{F}(x,y) = \vec{0}$ is $(x,y) = \left(\answer{0},\answer{0} \right)$


\begin{remark}
As it turns out, the points for which $\grad{F}(x,y) = \vec{0}$ will
be very important when we try to find relative extrema of a function
of two variables.  This exercise gives a little practice with
analyzing the gradient to that end.
\end{remark}
\end{exercise}
\end{exercise}
\end{exercise}

\end{document}
