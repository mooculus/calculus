\documentclass{ximera}

\newcommand{\RR}{\mathbb R}
\renewcommand{\d}{\,d}
\newcommand{\dd}[2][]{\frac{d #1}{d #2}}
\renewcommand{\l}{\ell}
\newcommand{\ddx}{\frac{d}{dx}}
\newcommand{\dfn}{\textbf}
\newcommand{\eval}[1]{\bigg[ #1 \bigg]}


\author{Jim Talamo}
\license{Creative Commons 3.0 By-NC}


\outcome{Understand remainder estimates for alternating series.}

\begin{document}

\begin{exercise}
A student is asked to determine $\sum_{k=1}^{\infty} \frac{(-1)^k}{\sqrt[5]{k}}$ to within $.001$ of its actual value.  The student decides to use one million terms to give the approximation, correctly uses computational software to compute $\sum_{k=1}^{1000000} \frac{(-1)^k}{\sqrt[5]{k}} = -.5124$ to four decimal places and claims that the answer should be sufficiently close since so many terms were used.  According to the alternating series remainder results, how close should the student actually expect the answer to be to the exact value of the series? 

The error bound using the remainder results for alternating series gives that the upper bound for the error is $\answer[tolerance=.0002]{.0631}$.  Thus, the student \wordChoice{\choice{should}\choice[correct]{should not}} believe that the approximation is necessarily within $.001$ of the exact value of the series.


\begin{hint}
First, note that the series $\sum_{k=1}^{\infty} \frac{(-1)^k}{\sqrt[5]{k}}$ will converge by the alternating series test.  As such, the series \emph{has} a value, and we can now go about trying to approximate it, and the remainder results for alternating series give an error bound of $|r_n| \leq a_{n+1}$.  Here, we want to find $r_{1000000}$, and we thus have that 

\[
|r_{1000000}| \leq \frac{1}{\sqrt[5]{1000001}} \approx \answer{.0631} \qquad \textrm{(to four decimal places)}
\] 

\end{hint}

\begin{exercise}
According the alternating series results, what should the smallest value of $N$ be so we know that $\sum_{k=1}^N \frac{(-1)^k}{\sqrt[5]{k}}$  will be within $.001$ of the actual value of $\sum_{k=1}^{\infty} \frac{(-1)^k}{\sqrt[5]{k}}$ ?

We find that $N \geq \answer[tolerance=.0001]{(1000)^5-1} $

\begin{hint}
We need $N$ so $|r_N| \leq \frac{1}{\sqrt[5]{N+1}} \leq .001$.

\begin{align*}
\frac{1}{\sqrt[5]{N+1}} &\leq .001 \\
\sqrt[5]{N+1} &\geq 1000 \\
N+1 &\geq (1000)^5 \\
N &\geq (1000)^5 -1\\
\end{align*}

\end{hint}

\begin{feedback}
Notice that it's not just enough to use technology to sum ``many'' terms for a series and claim that the result is approximately equal to the actual value of the series.  To be confident in our approximation, we have to have a sense as to how ``bad'' our error is.  To get within $.001$ of the actual value of the series in question here, we need almost one quadrillion terms!  
\end{feedback}
\end{exercise}
\end{exercise}

\end{document}
