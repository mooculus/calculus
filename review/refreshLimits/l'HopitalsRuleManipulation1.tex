\documentclass{ximera}

\newcommand{\RR}{\mathbb R}
\renewcommand{\d}{\,d}
\newcommand{\dd}[2][]{\frac{d #1}{d #2}}
\renewcommand{\l}{\ell}
\newcommand{\ddx}{\frac{d}{dx}}
\newcommand{\dfn}{\textbf}
\newcommand{\eval}[1]{\bigg[ #1 \bigg]}

\author{Jim Talamo}
\title[Refresh:]{ L'H\^{o}pital's Rule}

\begin{document}
\begin{abstract}
Review  L'H\^{o}pital's Rule.
\end{abstract}
\maketitle

\begin{problem}
Sometimes, some manipulation is required in order to bring a limit to a form where L'H\^{o}pital's Rule can be used.  

\begin{example}
Compute $\lim_{x \to 0^+} x\csc(x)$.

\begin{explanation}
Since $\sin(x) \to 0^+$ (meaning arbitrarily close to $0$ and positive) as $x \to 0^+$, we have $\csc(x) \to +\infty$ as $x \to 0^+$.  Notice that direct evaluation gives the indeterminate form $0 \times \infty$, so we cannot use L'H\^{o}pital's Rule yet.  However, since 

\[
\csc(x) = \frac{1}{\sin(x)}
\]

we can rewrite the original expression as

\[
\lim_{x \to 0^+} x\csc(x) = \lim_{x \to 0^+} \frac{x}{\sin(x)}.
\]
Now, direct evaluation gives the indeterminate form $\frac{0}{0}$, so we can use L'H\^{o}pital's Rule to conclude

\[
\lim_{x \to 0^+} \frac{x}{\sin(x)} = \lim_{x \to 0^+} \frac{1}{\cos(x)} =1.
\]
\end{explanation}
\end{example}

This example is one in which the indeterminate form $0 \times \pm \infty$ arise.  When this happens, we must manipulate the original expression by either putting 

\begin{itemize}  
\item the expression whose limit is $0$ in the denominator or
\item the expression whose limit is $\pm \infty$ in the denominator.
\end{itemize}

Try your hand at these.  If a limit does not exist, use $\infty$ or $-\infty$ as appropriate or write ``DNE'' otherwise.

\begin{exercise}
\[
\lim_{x \to 0^+} x \ln(x) = \answer{0}  
\]
\end{exercise}

\begin{exercise}
\[
\lim_{x \to \infty} xe^{1/x} = \answer{\infty}  
\]
\end{exercise}

\end{problem}
\end{document}
