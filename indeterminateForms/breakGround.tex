\documentclass{ximera}

\newcommand{\RR}{\mathbb R}
\renewcommand{\d}{\,d}
\newcommand{\dd}[2][]{\frac{d #1}{d #2}}
\renewcommand{\l}{\ell}
\newcommand{\ddx}{\frac{d}{dx}}
\newcommand{\dfn}{\textbf}
\newcommand{\eval}[1]{\bigg[ #1 \bigg]}



\outcome{Calculate limits of the form zero over zero.}


\title[Break-Ground:]{Could it be anything?}

\begin{document}
\begin{abstract}
Two young mathematicians investigate the arithmetic of large
and small numbers.
\end{abstract}
\maketitle


Check out this dialogue between two calculus students (based on a true
story):


\begin{dialogue}
\item[Devyn] Hey Riley, remember
  \[
  \lim_{\theta\to 0}\frac{\sin(\theta)}{\theta}?
  \]
\item[Riley] It is equal to $1$!
\item[Devyn] But was that crazy proof with all the triangles really
  necessary? I mean, just plug in zero. 
  \[
  \eval{\frac{\sin(\theta)}{\theta}}_{\theta=0} = \frac{\sin(0)}{0}=\frac{0}{0}\dots
  \]
  \item[Riley] You were going to say ``$1$,'' right? 
  \item[Devyn] Yeah, but now I'm not sure I was right.
  \item[Riley] Dividing by zero is usually a bad idea.
  \item[Devyn] You are right. I will never do it again! Also, don't
    tell anyone about this conversation.
  \item[Riley] What conversation?
  \item[Devyn] Exactly.
\end{dialogue}



\begin{problem}
  Consider the function
  \[
  f(x) = \frac{x}{x}.
  \]
  \[
  f(0) = \answer{DNE}\qquad\lim_{x\to 0} f(x) = \answer{1}.
  \]
\end{problem}

\begin{problem}
  Consider the function
  \[
  f(x) = \frac{4x}{x}.
  \]
  \[
  f(0) = \answer{DNE}\qquad\lim_{x\to 0} f(x) = \answer{4}.
  \]
\end{problem}

\begin{problem}
  Consider the function
  \[
  f(x) = \frac{x}{-3x}.
  \]
  \[
  f(0) = \answer{DNE}\qquad\lim_{x\to 0} f(x) = \answer{-1/3}.
  \]
\end{problem}

%%% \begin{xarmaBoost}
%%   Write down at least \textbf{five} questions for this lecture. After
%%   you have your questions, label them as ``Level 1,'' ``Level 2,'' or
%%   ``Level 3'' where:
%% \begin{description}
%% \item[Level 1] Means you know the answer, or know exactly how to do
%%   this problem.
%% \item[Level 2] Means you think you know how to do the problem.
%% \item[Level 3] Means you have no idea how to do the problem.
%% \end{description}
%% \begin{freeResponse}
%% \end{freeResponse}
%% \end{xarmaBoost}



\end{document}
