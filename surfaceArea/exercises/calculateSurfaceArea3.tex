\documentclass{ximera}

\newcommand{\RR}{\mathbb R}
\renewcommand{\d}{\,d}
\newcommand{\dd}[2][]{\frac{d #1}{d #2}}
\renewcommand{\l}{\ell}
\newcommand{\ddx}{\frac{d}{dx}}
\newcommand{\dfn}{\textbf}
\newcommand{\eval}[1]{\bigg[ #1 \bigg]}


\author{Jim Talamo and Nicholas Hemleben}
\license{Creative Commons 3.0 By-NC}


\outcome{Set up an integral that gives the area of a surface of revolution}
\outcome{Find  the area of a surface of revolution}

\begin{document}
\begin{exercise}

The portion of the curve $x=\sqrt{2y+1}$ from $x=1$ to $x=3$ about the $y$-axis.

 \begin{image}
      \begin{tikzpicture}
        \begin{axis}[
            xmin=-.3, xmax=3.6,
            domain=-1:1,
            ymin=-.8, ymax=4.5,
            clip=false,
            xtick = {-1,1,2,3},
            ytick = {-1,1,2,3,4},
            axis lines =center,
            xlabel=$x$, ylabel=$y$, every axis y label/.style={at=(current axis.above origin),anchor=south},
            every axis x label/.style={at=(current axis.right of origin),anchor=west},
            axis on top,
          ]
                              
         \addplot [penColor,thick,smooth,domain=0:3.1]{1/2*x^2-1/2};
          
         % ds and points
          	\addplot[color=penColor2,fill=penColor2,only marks,mark=*] coordinates{(2,1.5)};
		\addplot[color=penColor2,fill=penColor2,only marks,mark=*] coordinates{(2.1,1.705)};
		\addplot[ultra thick, penColor2] plot coordinates {(2,1.5) (2.1,1.705)};
          	\node[anchor=north, penColor2] at (axis cs:2.2,1.7) {$\Delta s$};
	
          %r and point 
           	\addplot[thick, penColor2] plot coordinates {(0,1.6) (2.05,1.65)};
		\node[anchor=north, penColor2] at (axis cs:1,1.5) {$r$};
          
          %curve segment
        \addplot[color=penColor,fill=penColor,only marks,mark=*] coordinates{(1,0)};
	\addplot[color=penColor,fill=penColor,only marks,mark=*] coordinates{(3,4)};
          
          \node[penColor] at (axis cs:1.8,3) {$x=\sqrt{2y+1}$};
        \end{axis}
      \end{tikzpicture}
    \end{image}
 

Set up an integral with respect to $y$ that gives the area of the surface of revolution:  

\[
SA=\int_{y=\answer{0}}^{y=\answer{4}} \answer{2 \pi \sqrt{2y+1}}\answer{ \sqrt{1+ \frac{1}{2y+1}}} \d y
\]



\begin{exercise}
Evaluating this integral, we find that the surface area of the surface of revolution is $\answer{\frac{2\pi}{3} ((10)^{3/2} -(2)^{3/2})}$ square units. 


\begin{hint}
We can simplify the integrand algebraically by using the laws of exponents to write $\sqrt{2y+1} \sqrt{1+ \frac{1}{2y+1}} =\sqrt{\answer{2y+2}}$. 
\end{hint}

\end{exercise}



Set up an integral with respect to $x$ that gives the area of the surface of revolution:  

\[
SA=\int_{x=\answer{1}}^{x=\answer{3}} \answer{2 \pi x} \answer{ \sqrt{1+ x^2}} \d x
\]



\begin{exercise}
Evaluating this integral, we find that the surface area of the surface of revolution is $\answer{\frac{2\pi}{3} ((10)^{3/2} -(2)^{3/2})}$ square units. 

\end{exercise}

By evaluating both of these integrals, 
\begin{multipleChoice}
\choice[correct]{The answers match.}
\choice{The answers are different.}
\end{multipleChoice}

\end{exercise}






\end{document}
