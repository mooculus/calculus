\documentclass{ximera}
\newcommand{\RR}{\mathbb R}
\renewcommand{\d}{\,d}
\newcommand{\dd}[2][]{\frac{d #1}{d #2}}
\renewcommand{\l}{\ell}
\newcommand{\ddx}{\frac{d}{dx}}
\newcommand{\dfn}{\textbf}
\newcommand{\eval}[1]{\bigg[ #1 \bigg]}

\author{Steven Gubkin}
\license{Creative Commons 3.0 By-NC}

\outcome{Define accumulation functions.}
\outcome{Calculate and evaluate accumulation functions.}
\outcome{State the First Fundamental Theorem of Calculus.}
\outcome{Take derivatives of accumulation functions using the First Fundamental Theorem of Calculus.}
\outcome{Use accumulation functions to find information about the original function.}
\outcome{Understand the relationship between the function and the derivative of its accumulation function.}

\begin{document}
\begin{exercise}

True or False:  If  $f$ is continuous on $\mathbb{R}$,  $g(x)= \int_1^{x} f(t) \d t$ and  $h(x)= \int_2^x f(t) \d t$, then $g-h$ is a constant function

\begin{hint}
For all $x$,


 $g(x)-h(x)=\int_1^{x} f(t) \d t- \int_2^x f(t) \d t=\int_1^{x} f(t) \d t+ \int_x^2 f(t) \d t=\int_1^{2} f(t) \d t$
 
 OR
 
 
 For all $x$,
 
 
 $(g-h)'(x)=g'(x)-h'(x)=f(x)-f(x)=0$.
 
 
 The derivative of the function $g-h$ is zero, for all real numbers. Therefore, the function is constant.
\end{hint}
\begin{prompt}
	\begin{multipleChoice}
		\choice[correct]{True}
		\choice{False}
	\end{multipleChoice}
\end{prompt}

\end{exercise}
\end{document}