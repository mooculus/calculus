\documentclass{ximera}

\newcommand{\RR}{\mathbb R}
\renewcommand{\d}{\,d}
\newcommand{\dd}[2][]{\frac{d #1}{d #2}}
\renewcommand{\l}{\ell}
\newcommand{\ddx}{\frac{d}{dx}}
\newcommand{\dfn}{\textbf}
\newcommand{\eval}[1]{\bigg[ #1 \bigg]}



\outcome{Interpert the product of rate and time as area.}
\outcome{Approximate position from velocity.}
\outcome{Recognize Riemann sums.}

\author{Nela Lakos \and Kyle Parsons}

\begin{document}
\begin{exercise}

A function $v$ describes the velocity of a car (in mi/hr) moving along a straight highway for a 4 hour interval.  The graph of $v$ is given below.

\begin{image}
  \begin{tikzpicture}
    \begin{axis}[
        xmin=-0.1,xmax=4.1,ymin=-0.8,ymax=25.8,
        clip=true,
        unit vector ratio*=8 1 1,
        axis lines=center,
        grid = major,
        ytick={0,5,...,36},
        xtick={0,1,...,6},
        xlabel=$t$, ylabel=$v$,
        every axis y label/.style={at=(current axis.above origin),anchor=south},
        every axis x label/.style={at=(current axis.right of origin),anchor=west},
      ]
      \addplot[ultra thick,penColor,domain=0:4] plot{5*x};     
       
      %\node at (axis cs:4.5,30) {$v(t)$};
      \end{axis}`
  \end{tikzpicture}
\end{image}

Using geometry, we find that the displacement of the car over the 4 hour period is
\[
s(4)-s(0) = \answer{40}\text{mi}.
\]

Using a Left Riemann sum with $n=2$ subintervals we can approximate this displacement as
\[
s(4)-s(0)\approx\answer{20}\text{mi}.
\]

We can also use a Right Riemann sum with $n=4$ subintervals to approximate this displacement as
\[
s(4)-s(0)\approx\answer{50}\text{mi}.
\]

Finally, we can use a Midpoint Riemann sum with $n=2$ subintervals to approximate this displacement as
\[
s(4)-s(0)\approx\answer{40}\text{mi}.
\]

Using geometry, we can find a formula for the displacement up to time $t$ for $0\leq t\leq4$ as
\[
s(t)-s(0) = \answer{\frac{5t^2}{2}}\text{mi}.
\]

Now, also consider the initial value problem
\begin{align*}
s'(t) &= v(t)\\
s(0) &= 7.
\end{align*}
Then
\[
s(t) = \answer{\frac{5t^2}{2}+7}.
\]

Compare these final two parts of the problem.  Do they lead you to any conjecture about the relationship between displacement (area under a curve) and any antiderivative of $v$?

\end{exercise}
\end{document}
