\documentclass{ximera}

\newcommand{\RR}{\mathbb R}
\renewcommand{\d}{\,d}
\newcommand{\dd}[2][]{\frac{d #1}{d #2}}
\renewcommand{\l}{\ell}
\newcommand{\ddx}{\frac{d}{dx}}
\newcommand{\dfn}{\textbf}
\newcommand{\eval}[1]{\bigg[ #1 \bigg]}


%\outcome{Given a velocity function, calculate displacement and distance traveled.}
%\outcome{Given a velocity function, find the position function.}
%\outcome{Given an acceleration function, find the velocity function.}
%\outcome{Understand the difference between displacement and distance traveled.}
%\outcome{Understand the relationship between position, velocity and acceleration.}

\author{Nela Lakos \and Kyle Parsons}

\begin{document}
\begin{exercise}

The graph of a function $f$ on the interval $[1,9]$ is given below.

\begin{image}
  \begin{tikzpicture}
    \begin{axis}[
        xmin=-0.3,xmax=10.3,ymin=-0.3,ymax=6.3,
        clip=true,
        unit vector ratio*=1 1 1,
        axis lines=center,
        grid = major,
        ytick={-1,0,...,36},
        xtick={0,1,...,10},
        xlabel=$x$, ylabel=$y$,
        every axis y label/.style={at=(current axis.above origin),anchor=south},
        every axis x label/.style={at=(current axis.right of origin),anchor=west},
      ]
      \addplot[ultra thick,penColor,domain=1:5,samples=2] {x+1};    
      \addplot[ultra thick,penColor,domain=5:9,samples=2] {11-x};
        
      \node at (axis cs:1.5,5.5) {$y=f(x)$};
      \end{axis}`
  \end{tikzpicture}
\end{image}

Using geometry we can evaluate $\int_1^9f(x)\d x$ as
\[
\int_1^9f(x)\d x = \answer{32}.
\]
With this we can calculate the average value of $f$ on $[1,9]$ is $\answer{4}$.

The rectangle with base $[1,9]$ that has area $\int_1^9f(x)\d x$ has height $\answer{4}$.

\begin{image}
  \begin{tikzpicture}
    \begin{axis}[
        xmin=-0.3,xmax=10.3,ymin=-0.3,ymax=6.3,
        clip=true,
        unit vector ratio*=1 1 1,
        axis lines=center,
        grid = major,
        ytick={-1,36},
        xtick={0,1,...,10},
        xlabel=$x$, ylabel=$y$,
        every axis y label/.style={at=(current axis.above origin),anchor=south},
        every axis x label/.style={at=(current axis.right of origin),anchor=west},
      ]
      \addplot[ultra thick,penColor,domain=1:5,samples=2] {x+1};    
      \addplot[ultra thick,penColor,domain=5:9,samples=2] {11-x};
      \filldraw[penColor2,opacity=0.3] (axis cs:1,0) rectangle (axis cs:9,4);
        
      \node at (axis cs:1.5,5.5) {$y=f(x)$};
      \end{axis}`
  \end{tikzpicture}
\end{image}

The points in $[1,9]$ where $f(c) = \int_1^9f(x)\d x$ are $c=\answer{3}$ and $c=\answer{7}$ (answer from left to right).

\end{exercise}
\end{document}