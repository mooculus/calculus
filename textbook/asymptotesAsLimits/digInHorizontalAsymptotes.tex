\documentclass{ximera}

\newcommand{\RR}{\mathbb R}
\renewcommand{\d}{\,d}
\newcommand{\dd}[2][]{\frac{d #1}{d #2}}
\renewcommand{\l}{\ell}
\newcommand{\ddx}{\frac{d}{dx}}
\newcommand{\dfn}{\textbf}
\newcommand{\eval}[1]{\bigg[ #1 \bigg]}


\title[Dig-In:]{Horizontal asymptotes}

\begin{document}
\begin{abstract}
We explore functions that behave like horizontal lines as the input
grows without bound.
\end{abstract}
\maketitle



Consider the function:
\[
f(x) = \frac{6x-9}{x-1}
\]
\begin{image}
\begin{tikzpicture}
	\begin{axis}[
            domain=1:4,
            ymax=20,
            ymin=-10,
            samples=100,
            axis lines =middle, xlabel=$x$, ylabel=$y$,
            every axis y label/.style={at=(current axis.above origin),anchor=south},
            every axis x label/.style={at=(current axis.right of origin),anchor=west}
          ]
	  \addplot [very thick, penColor, smooth, domain=(0:.9)] {(6*x-9)/(x-1)};
          \addplot [very thick, penColor, smooth, domain=(1.1:3)] {(6*x-9)/(x-1)};
          \addplot [textColor, dashed] plot coordinates {(1,-10) (1,20)};
        \end{axis}
\end{tikzpicture}
%% \caption{A plot of $f(x)=\protect\frac{6x-9}{x-1}$.}
%% \label{plot:(6x-9)/(x-1)}
\end{image}
As $x$ approaches infinity, it seems like $f(x)$ approaches a specific
value. This is a \textit{limit at infinity}.

\begin{definition}\label{def:limitAtInfty}\index{limit!at infinity}
If $f(x)$ becomes arbitrarily close to a specific value $L$ by making
$x$ sufficiently large, we write
\[
\lim_{x\to \infty} f(x) = L
\]
and we say, the \dfn{limit at infinity} of $f(x)$ is $L$.  

If $f(x)$ becomes arbitrarily close to a specific value $L$ by making
$x$ sufficiently large and negative, we write
\[
\lim_{x\to -\infty} f(x) = L
\]
and we say, the \dfn{limit at negative infinity} of $f(x)$ is $L$.  
\end{definition}

\begin{example} Compute
\[
\lim_{x\to\infty} \frac{6x-9}{x-1}.
\]
\begin{explanation}
Write
\begin{align*}
\lim_{x\to\infty}\frac{6x-9}{x-1} &= \lim_{x\to\infty}\frac{6x-9}{x-1} \frac{1/x}{1/x}\\
&=\lim_{x\to\infty}\frac{\frac{6x}{x} - \frac{9}{x}}{\frac{x}{x} - \frac{1}{x}}\\
&= \lim_{x\to\infty} \frac{6}{1}\\
&= 6.
\end{align*}
\end{explanation}
\end{example}

Sometimes one must be careful, consider this example.

\begin{example}
Compute
\[
\lim_{x\to -\infty} \frac{x+1}{\sqrt{x^2}}
\]
\begin{explanation}
In this case we multiply the numerator and denominator by $-1/x$,
which is a positive number as since $x\to -\infty$, $x$ is a negative
number.
\begin{align*}
\lim_{x\to -\infty} \frac{x+1}{\sqrt{x^2}} &= \lim_{x\to -\infty} \frac{x+1}{\sqrt{x^2}} \cdot \frac{-1/x}{-1/x}\\
&= \lim_{x\to -\infty} \frac{-1-1/x}{\sqrt{x^2/x^2}}\\
&= -1.
\end{align*}
\end{explanation}
\end{example}


Here is a somewhat different example of a limit at infinity.

\begin{example}
Compute
\[
\lim_{x\to \infty} \frac{\sin(7x)}{x}+4.
\]

\begin{image}
\begin{tikzpicture}
	\begin{axis}[
            domain=2:20,
            ymax=5,
            ymin=3,
            samples=100,
            axis lines =middle, xlabel=$x$, ylabel=$y$,
            every axis y label/.style={at=(current axis.above origin),anchor=south},
            every axis x label/.style={at=(current axis.right of origin),anchor=west}
          ]
	  \addplot [very thick, penColor, smooth] {(1/x) * sin(deg(7*x))+4};
        \end{axis}
\end{tikzpicture}
%% \caption{A plot of $f(x)=\frac{\sin(7x)}{x}+4$.}
%% \label{plot:sin7x/x+4}
\end{image}

\begin{explanation}
We can bound our function
\[
-1/x + 4 \le \frac{\sin(7x)}{x}+4 \le 1/x + 4.
\]
Since 
\[
\lim_{x\to \infty} -1/x + 4 = 4 = \lim_{x\to \infty}1/x + 4
\] 
we conclude by the Squeeze Theorem, Theorem~\ref{theorem:squeeze},
$\lim_{x\to\infty}\frac{\sin(7x)}{x}+4 = 4$.
\end{explanation}
\end{example}






\begin{definition}\label{def:horiz asymptote}\index{asymptote!horizontal}\index{horizontal asymptote}
If  
\[
\lim_{x\to \infty} f(x) = L \qquad\text{or}\qquad \lim_{x\to -\infty} f(x) = L,
\]
then the line $y=L$ is a \dfn{horizontal asymptote} of $f(x)$.
\end{definition}

\begin{example} 
Give the horizontal asymptotes of
\[
f(x) = \frac{6x-9}{x-1}
\]
\begin{explanation}
From our previous work, we see that $\lim_{x\to \infty} f(x) = 6$, and
upon further inspection, we see that $\lim_{x\to -\infty} f(x) =
6$. Hence the horizontal asymptote of $f(x)$ is the line $y=6$.
\end{explanation}
\end{example}


It is a common misconception that a function cannot cross an
asymptote. As the next example shows, a function can cross an
asymptote, and in this case this occurs an infinite number of times!

\begin{example}
Give a horizontal asymptote of
\[
f(x) = \frac{\sin(7x)}{x}+4.
\]
\begin{explanation}
Again from previous work, we see that $\lim_{x\to \infty} f(x) =
4$. Hence $y=4$ is a horizontal asymptote of $f(x)$.
\end{explanation}
\end{example}


We conclude with an infinite limit at infinity.

\begin{example}
Compute
\[
\lim_{x\to \infty} \ln(x)
\]
\begin{image}
\begin{tikzpicture}
	\begin{axis}[
            domain=0:20,
            ymax=4,
            ymin=-1,
            samples=100,
            axis lines =middle, xlabel=$x$, ylabel=$y$,
            every axis y label/.style={at=(current axis.above origin),anchor=south},
            every axis x label/.style={at=(current axis.right of origin),anchor=west}
          ]
	  \addplot [very thick, penColor, smooth] {ln(x)};
        \end{axis}
\end{tikzpicture}
%% \caption{A plot of $f(x)=\ln(x)$.}
%% \label{plot:lnx}
\end{image}
\begin{explanation}
The function $\ln(x)$ grows very slowly, and seems like it may have a
horizontal asymptote, see Figure~\ref{plot:lnx}. However, if we
consider the definition of the natural log
\[
\ln(x) = y \qquad \Leftrightarrow\qquad e^y =x
\]
Since we need to raise $e$ to higher and higher values to obtain
larger numbers, we see that $\ln(x)$ is unbounded, and hence
$\lim_{x\to\infty}\ln(x)=\infty$.
\end{explanation}
\end{example}


\end{document}
