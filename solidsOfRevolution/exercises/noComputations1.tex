\documentclass{ximera}
\newcommand{\RR}{\mathbb R}
\renewcommand{\d}{\,d}
\newcommand{\dd}[2][]{\frac{d #1}{d #2}}
\renewcommand{\l}{\ell}
\newcommand{\ddx}{\frac{d}{dx}}
\newcommand{\dfn}{\textbf}
\newcommand{\eval}[1]{\bigg[ #1 \bigg]}

\author{Jim Talamo}
\license{Creative Commons 3.0 By-NC}
\outcome{Think about Washer and Shell Method conceptually}
\begin{document}


\begin{exercise}
 The region bounded by $y=x$, $y=2x$, and $x=6$ is revolved about the line $x=9$.  If an integral or sum of integrals with respect to $x$ is used to compute the volume of the solid, which method should be used?
 
\begin{multipleChoice}
\choice{Washer Method}
\choice[correct]{Shell Method} 
\end{multipleChoice}

\end{exercise}

\begin{exercise}
 The region bounded by $y=\frac{1}{x+1}$, $x=0$, and $x=2$ is revolved about the line $y=-1$.  If the Washer Method is used to calculate the volume or the resulting solid, we must:
 
\begin{multipleChoice}
\choice[correct]{integrate with respect to $x$.}
\choice{integrate with respect to $y$.} 
\end{multipleChoice}

\end{exercise}

\begin{exercise}
 The region $R$ is bounded by $x=9-y^2$ and $x=-2$.  The Shell Method can be used to set up an integral with respect to $y$ that computes the resulting solid is $R$ is revolved about:
 
 
\begin{multipleChoice}
\choice{$x=10$}
\choice[correct]{$y=10$} 
\end{multipleChoice}

\end{exercise}

\begin{exercise}
 The region bounded by $y=3x$, $y=8-2x$, and $y=0$ is revolved about the line $x=-2$.  Which method should be used to express the volume with a single integral?
 
\begin{multipleChoice}
\choice[correct]{Washer Method}
\choice{Shell Method} 
\choice{both methods require a single integral}
\choice{both methods require more than one integral}
\end{multipleChoice}

\end{exercise}

\end{document}