\documentclass{ximera}

\newcommand{\RR}{\mathbb R}
\renewcommand{\d}{\,d}
\newcommand{\dd}[2][]{\frac{d #1}{d #2}}
\renewcommand{\l}{\ell}
\newcommand{\ddx}{\frac{d}{dx}}
\newcommand{\dfn}{\textbf}
\newcommand{\eval}[1]{\bigg[ #1 \bigg]}


\outcome{Find the domain and range of a function.}

\author{Nela Lakos \and Kyle Parsons}

\begin{document}
\begin{exercise}

The (entire) graph of $f$ is given below.

\begin{image}
  \begin{tikzpicture}
    \begin{axis}[
        xmin=-4.8,xmax=4.8,ymin=-4.8,ymax=4.8,
        clip=true,
        unit vector ratio*=1 1 1,
        axis lines=center,
        grid = major,
        ytick={-4,-3,...,4},
    xtick={-4,-3,...,4},
        xlabel=$x$, ylabel=$y$,
        every axis y label/.style={at=(current axis.above origin),anchor=south},
        every axis x label/.style={at=(current axis.right of origin),anchor=west},
      ]
      \draw[very thick,penColor] (axis cs:-4,4) -- (axis cs:-1,1);
      \draw[very thick,penColor] (axis cs:1,-2) -- (axis cs:4,0);
      
      \addplot[penColor,only marks,mark=*] coordinates{(-4,4) (-1,1) (1,-2) (4,0)};
      
      \node at (axis cs:2.5,2.5) [penColor] {$y=f(x)$};
      \end{axis}`
  \end{tikzpicture}
\end{image}

The domain of $f$ in interval notation (from left to right) is
\[
\left[\answer{-4},\answer{-1}\right]\cup\left[\answer{1},\answer{4}\right].
\]

The range of $f$ in interval notation (from bottom to top) is
\[
\left[\answer{-2},\answer{0}\right]\cup\left[\answer{1},\answer{4}\right].
\]

The domain of $f^{-1}$ in interval notation (from left to right) is
\[
\left[\answer{-2},\answer{0}\right]\cup\left[\answer{1},\answer{4}\right].
\]

\begin{align*}
f(4) &= \answer{0}\\
f(1) &= \answer{-2}\\
f^{-1}(4) &= \answer{-4}\\
f^{-1}(1) &= \answer{-1}\\
f'(-3) &= \answer{-1}\\
f'(2) &= \answer{\frac{2}{3}}
\end{align*}

\end{exercise}
\end{document}