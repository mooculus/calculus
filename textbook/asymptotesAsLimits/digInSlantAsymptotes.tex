\documentclass{ximera}

\newcommand{\RR}{\mathbb R}
\renewcommand{\d}{\,d}
\newcommand{\dd}[2][]{\frac{d #1}{d #2}}
\renewcommand{\l}{\ell}
\newcommand{\ddx}{\frac{d}{dx}}
\newcommand{\dfn}{\textbf}
\newcommand{\eval}[1]{\bigg[ #1 \bigg]}


\title[Dig-In:]{Slant asymptotes}

\begin{document}
\begin{abstract}
We explore functions that ``shoot to infinity'' at certain points in
their domain.
\end{abstract}
\maketitle

If we think of an asymptote as a ``line that a function looks like
when the input is large,'' then there are three types of asymptotes:
\begin{enumerate}
\item Vertical Asymptotes.
\item Horizontal Aymptotes.
\item Slant Asymptotes.
\end{enumerate}

Recall, a function $f$ has a vertical asymptote at $x=a$ if

\[
\lim_{x\to a} f(x) = \pm\infty, \qquad \lim_{x\to a+} f(x) = \pm\infty, \qquad\text{or}\qquad \lim_{x\to a-} f(x) = \pm\infty,
\]
In this case, the asymptote is the vertical line
\[
x = a.
\]
On the other hand, a function $f$ has a horizontal asymptote if
\[
\lim{x\to \infty} f(x) = L \qquad\text{or}\qquad \lim_{x\to -\infty}
f(x) = L.
\]
In this case, the asymptote is the line
\[
\l(x) = L.
\]
A \textit{slant asymptote} is a somewhat different beast.

\begin{definition}\index{slant asymptote}\index{asymptote!slant}
  If there is a line $\l$ such that
  \[
  \lim_{x\to \infty}\left(f(x) - \l(x)\right) = 0,
  \]
  then $\l$ is a \textbf{slant asymptote} for $f(x)$.
\end{definition}
\begin{question}
  image with slant asymptote
\end{question}

To analytically find slant asymptotes, one must find the required
information to determine a line, namely the slope and $y$-intercept.
While there are several ways to do this, we wil give a method that is
fairly general. 


\begin{example}
  Find the slant asymptote of
  \[
f(x) = \frac{x^2-3x+2}{x+2}.
  \]
  \begin{explanation}
    We are looking to see if there is a line $\l$ such that
    \[
    \lim_{x\to \infty}\left(f(x) - \l(x)\right) = 0.
    \]
    We will imagine that we have such a line
    \[
    \l(x) = mx + b
    \]
    and attempt to find such a line. We see
    \begin{align*}
      m &=\lim_{x\to\infty}\frac{\frac{x^2-3x+2}{x+2}}{x}\\
      &= \lim_{x\to\infty}\frac{\frac{x^2-3x+2}{x+2}}{x}\\
      &= \lim_{x\to\infty}\frac{x^2-3x+2}{x^2+2x}\\
      &= \lim_{x\to\infty}\frac{x^2-3x+2}{x^2+2x}\cdot\frac{1/x^2}{1/x^2}\\
      &= \lim_{x\to\infty}\frac{1-3/x+2/x^2}{1+2/x}\\
      &= 1.
    \end{align*}
    So $m=1$. To find the $y$-intercept, write
    \begin{align*}
      b &=\lim_{x\to\infty} \left(\frac{x^2-3x+2}{x+2} - x\right)\\
      &=\lim_{x\to\infty} \frac{x^2-3x+2-x^2}{x+2}\\
      &=\lim_{x\to\infty} \frac{-3x+2}{x+2}\\
      &=\lim_{x\to\infty} \frac{-3x+2}{x+2}\cdot\frac{1/x}{1/x}\\
      &=\lim_{x\to\infty} \frac{-3+2/x}{1+2/x}\\
      &= -3.
    \end{align*}
    Hence our slant asymptote is $\l(x) = x-3$.
  \end{explanation}
\end{example}



\end{document}
