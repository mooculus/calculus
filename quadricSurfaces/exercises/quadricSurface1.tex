\documentclass{ximera}

\newcommand{\RR}{\mathbb R}
\renewcommand{\d}{\,d}
\newcommand{\dd}[2][]{\frac{d #1}{d #2}}
\renewcommand{\l}{\ell}
\newcommand{\ddx}{\frac{d}{dx}}
\newcommand{\dfn}{\textbf}
\newcommand{\eval}[1]{\bigg[ #1 \bigg]}

\usepackage{currfile}
\makeatletter
\ifxake
% The code below the \else is executed in a sagecell on the Ximera
% server, so \makerandom doesn't have to do anything when run under
% xake.
\newcommand{\makerandom}{}
\else
\newcommand {\ST@wsf }[1]{\immediate \write \ST@sf {##1}}
\newcommand{\makerandom}{%
  \ST@wsf{jobname="\currfilebase"}%
  \ST@wsf{import hashlib}%
  \ST@wsf{set_random_seed(int(hashlib.sha256(jobname.encode('utf-8')).hexdigest(), 16))}%
}
\fi
\makeatother


\outcome{Use the second derivative test to identify quadric surfaces.}

\begin{document}
\makerandom

\begin{sagesilent}
  x = var('x')
  y = var('y')
  ax = randint(-4,4)
  ay = randint(-4,4)  
  f = expand(randint(1,3)*(x-ax)**2 + randint(1,3)*(y-ay)**2)
  disc = derivative(f,x,x)(x=ax,y=ay)*derivative(f,y,y)(x=ax,y=ay) - derivative(f,x,y)(x=ax,y=ay)**2
\end{sagesilent}

\begin{exercise}
  Define a function $F : \R^2 \to \R$ by the rule $F(x,y) = \sage{f}$,
  and consider its graph, which consists of points $(x,y,z) \in \R^3$
  satisfying $z = F(x,y)$.  The graph of this function is a quadric
  surface.
  
  Consider the quadric surface $z= F(x,y) = \sage{f}$.
  
  At the point $\vec{c} = \vector{\sage{ax},\sage{ay}}$, we have that:
  \[
  \grad F(\vec{c}) = \vector{\answer{0},\answer{0}}%% it will always be zero
  \]
  Computing
  \begin{exercise}
    \[
    D(\vec{c}) = \answer{\sage{disc}}
    \]
    \begin{exercise}
      we see this quadric surface is a(n):
      \begin{multipleChoice}
        \choice[correct]{elliptic parabolid}
        \choice{hyperbolic parabolid}
      \end{multipleChoice}
      \begin{feedback}
        Confirm your answer by plotting the surface in a \link[3D Surface Plotter]{https://academo.org/demos/3d-surface-plotter/}.
      \end{feedback}
    \end{exercise}
  \end{exercise}
\end{exercise}
\end{document}
