\documentclass{ximera}

\newcommand{\RR}{\mathbb R}
\renewcommand{\d}{\,d}
\newcommand{\dd}[2][]{\frac{d #1}{d #2}}
\renewcommand{\l}{\ell}
\newcommand{\ddx}{\frac{d}{dx}}
\newcommand{\dfn}{\textbf}
\newcommand{\eval}[1]{\bigg[ #1 \bigg]}


\title{Approximating the area under a curve}

\begin{document}

\begin{abstract}
%Stuff can go here later if we want!
\end{abstract}

\maketitle

\begin{sectionOutcomes}

After completing this section, students should be able to do the following.

\begin{itemize}
	\item Express the sum of n terms using sigma notation.
	\item Apply the properties of sums when working with sums in sigma notation.
	\item Understand the relationship between area under a curve and sums of areas of rectangles.
	\item Approximate area of the region under a curve.
	\item Compute left, right, and midpoint Riemann sums with 10 or fewer rectangles.
	\item Understand how Riemann sums with n rectangles are computed and how the exact value of the area is obtained by taking the limit as $n\to\infty$ .
\end{itemize}

\end{sectionOutcomes}

\end{document}
