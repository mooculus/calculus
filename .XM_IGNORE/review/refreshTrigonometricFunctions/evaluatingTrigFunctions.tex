\documentclass{ximera}

%\newcommand{\RR}{\mathbb R}
\renewcommand{\d}{\,d}
\newcommand{\dd}[2][]{\frac{d #1}{d #2}}
\renewcommand{\l}{\ell}
\newcommand{\ddx}{\frac{d}{dx}}
\newcommand{\dfn}{\textbf}
\newcommand{\eval}[1]{\bigg[ #1 \bigg]}

\newcommand{\RR}{\mathbb R}
\renewcommand{\d}{\,d}
\newcommand{\dd}[2][]{\frac{d #1}{d #2}}
\renewcommand{\l}{\ell}
\newcommand{\ddx}{\frac{d}{dx}}
\newcommand{\dfn}{\textbf}
\newcommand{\eval}[1]{\bigg[ #1 \bigg]}


\author{Jim Talamo}
\license{Creative Commons 3.0 By-NC}
\outcome{}
\begin{document}

\begin{exercise}


  Many integrals involving powers of expressions of the forms: $u(x)^2 + a^2$, $u(x)^2 - a^2$, or $a^2 - u(x)^2$.
  These forms respectively require the use of a trigonometric substitution of the form: $u = a \tan(\theta)$, $u = a \sec(\theta)$, or $u = a \sin(\theta)$.
 
  As with all substitutions, the substitution into the integral is made so we end up with an antiderivative in the new variable that we know how to compute.
  After finding the antiderivative in terms of $\theta$ we must express the trigonometric functions in terms of the original variable $x$ (keep this in mind as you are working out problems in lecture and recitation). 
  
  We will have one of the basic trigonometric functions written in terms of $x$, but a fundamental step in these problems is the use of right triangles to express all of the other trigonometric functions in terms of $x$.
  As such, the following exercises will give you practice doing this.
  
  \begin{multipleChoice}
    \choice[correct]{I understand}
    \choice{I do not understand}
  \end{multipleChoice}  

\begin{problem}
  Use the relationship between trigonometric functions, and drawing the appropriate right-angle triangles, to find the \emph{exact values} of the remaining trigonometric functions if $\sec(\alpha) = \frac{3}{\sqrt{2}}$ and $0 < \alpha < \pi/2$.
  
    \begin{align*}
      \sin(\alpha) &= \answer{\frac{\sqrt{7}}{3}}  \\
      \cos(\alpha) &= \answer{\frac{\sqrt{2}}{3}}  \\
      \tan(\alpha) &= \answer{\sqrt{\frac{7}{2}}}  \\
      \cot(\alpha) &= \answer{\sqrt{\frac{2}{7}}}  \\
      \csc(\alpha) &= \answer{\frac{3}{\sqrt{7}}}
    \end{align*}  
\end{problem}

\begin{problem}
   Use the relationship between trigonometric functions, and drawing the appropriate right-angle triangles, to find the \emph{exact values} of the remaining trigonometric functions if $\sin(\theta) = \frac{-\sqrt{5}}{3}$ and $\theta$ is in quadrant IV.
  
    \begin{align*}
      \cos(\alpha) &= \answer{\frac{2}{3}}  \\
      \csc(\alpha) &= \answer{\frac{-3}{\sqrt{5}}}  \\
      \sec(\alpha) &= \answer{\frac{3}{2}}  \\
      \tan(\alpha) &= \answer{\frac{-\sqrt{5}}{2}}  \\
      \cot(\alpha) &= \answer{\frac{2}{-\sqrt{5}}}
    \end{align*}  
\end{problem}

\begin{problem}
  Using the interpretation of the output of an inverse trigonometric function is an angle and drawing the appropriate right-angle triangle, find an equivalent algebraic expression for the following composite function:
  \[
    \sin(\arccos(x)) = \answer{\sqrt{1-x^2}}
  \]
\end{problem}

\begin{problem}
  If $3x = 5 \cos(\theta)$, express $\tan(\theta)$ in terms of $x$:
  
  \[
    \tan(\theta) = \answer{\frac{\sqrt{25- 9x^2}}{3x}}
  \]
\end{problem}

\end{exercise}
\end{document}
