\documentclass{ximera}

\newcommand{\RR}{\mathbb R}
\renewcommand{\d}{\,d}
\newcommand{\dd}[2][]{\frac{d #1}{d #2}}
\renewcommand{\l}{\ell}
\newcommand{\ddx}{\frac{d}{dx}}
\newcommand{\dfn}{\textbf}
\newcommand{\eval}[1]{\bigg[ #1 \bigg]}


%\outcome{Understand the statement of the Extreme Value Theorem.}
\outcome{Understand the statement of the Mean Value Theorem.}
%\outcome{Sketch pictures to illustrate why the Mean Value Theorem is true.}
\outcome{Determine whether Rolle's Theorem or the Mean Value Theorem can be applied.}
%\outcome{Find the values guaranteed by Rolle's Theorem or the Mean Value Theorem.}
%\outcome{Use the Mean Value Theorem to solve word problems.}
%\outcome{Compare and contrast the Intermediate Value Theorem, the Mean Value Theorem, and Rolle's Theorem.}
%\outcome{Identify calculus ideas which are consequences of the Mean Value Theorem.}
%\outcome{Use the Mean Value Theorem to bound the error in linear approximation.}

\author{Nela Lakos \and Kyle Parsons}

\begin{document}
\begin{exercise}

Suppose $g$ is a function that is continuous on $[3,5]$ and differentiable on $(3,5)$.  Further suppose that $f(3)=2$, $f(5)=8$, and $f'(x)>0$ for all $x$ in $(3,5)$.

Answer the following true-false questions.

$0<g'(x)\leq2$ for all $x$ in $(3,5)$.
\begin{multipleChoice}
\choice{True}
\choice[correct]{False}
\end{multipleChoice}

$0<g(x)\leq8$ for all $x$ in $(3,5)$.
\begin{multipleChoice}
\choice[correct]{True}
\choice{False}
\end{multipleChoice}

$g'(c)=3$ for some number $c$ in $(3,5)$.
\begin{multipleChoice}
\choice[correct]{True}
\choice{False}
\end{multipleChoice}

$g$ does not satisfy the conditions of the Mean Value Theorem on $[3,5]$.
\begin{multipleChoice}
\choice{True}
\choice[correct]{False}
\end{multipleChoice}

\end{exercise}
\end{document}