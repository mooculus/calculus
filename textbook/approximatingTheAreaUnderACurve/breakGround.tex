\documentclass{ximera}

\newcommand{\RR}{\mathbb R}
\renewcommand{\d}{\,d}
\newcommand{\dd}[2][]{\frac{d #1}{d #2}}
\renewcommand{\l}{\ell}
\newcommand{\ddx}{\frac{d}{dx}}
\newcommand{\dfn}{\textbf}
\newcommand{\eval}[1]{\bigg[ #1 \bigg]}


\outcome{Define area.}

\title[Break-Ground:]{Approximating area under the curve}

\begin{document}
\begin{abstract}
Two young mathematicians discuss connections between
area and velocity.
\end{abstract}
\maketitle

% Introduce Riemann sums as an approximation of antiderivatives

% Think more broadly than area under the curve

% Think about multiplication: we don't just do an "area" or "array" model for multiplication

Check out this dialogue between two calculus students (based on a true
story):

\begin{dialogue}
\item[Devyn] Riley, I've have a troubling question.
\item[Riley] Yes? 
\item[Devyn] What is area?
\item[Riley] Oh, for a rectangle it is just
  \[
  \text{width}\times \text{height}
  \]
\item[Devyn] Right. But shouldn't it mean more?
\end{dialogue}

Area is a funny thing. What is it really? Well the idea is this:
\begin{quote}
Define something as having a ``unit'' area, and see how many times it
``covers'' something.
\end{quote}
The most obivious thing to use for our ``unit'' area is a unit
square. From this we can quickly move on to find the area of any
rectangle as
  \[
  \text{width}\times \text{height}
  \]
\begin{problem}
  Sometimes area is described as ``how much paint will cover a surface.''
  Is this accurate? What do you think?
  \begin{freeResponse}
  \end{freeResponse}
\end{problem}
%% \begin{xarmaBoost}
%%   Write down at least \textbf{five} questions for this lecture. After
%%   you have your questions, label them as ``Level 1,'' ``Level 2,'' or
%%   ``Level 3'' where:
%% \begin{description}
%% \item[Level 1] Means you know the answer, or know exactly how to do
%%   this problem.
%% \item[Level 2] Means you think you know how to do the problem.
%% \item[Level 3] Means you have no idea how to do the problem.
%% \end{description}
%% \begin{freeResponse}
%% \end{freeResponse}
%% \end{xarmaBoost}


\end{document}
