\documentclass{ximera}

% THIS NEEDS TO BE FIXED WHEN PUT INTO XIMERA! %
\newcommand{\RR}{\mathbb R}
\renewcommand{\d}{\,d}
\newcommand{\dd}[2][]{\frac{d #1}{d #2}}
\renewcommand{\l}{\ell}
\newcommand{\ddx}{\frac{d}{dx}}
\newcommand{\dfn}{\textbf}
\newcommand{\eval}[1]{\bigg[ #1 \bigg]}

\author{Hans Parshall}
% \newcommand{\RR}{\mathbb R}
\renewcommand{\d}{\,d}
\newcommand{\dd}[2][]{\frac{d #1}{d #2}}
\renewcommand{\l}{\ell}
\newcommand{\ddx}{\frac{d}{dx}}
\newcommand{\dfn}{\textbf}
\newcommand{\eval}[1]{\bigg[ #1 \bigg]}


\begin{document}
\begin{exercise}
Many statements are \emph{conditional}; that is, they can be written as ``If $P$, then $Q$.''  We call $P$ the \emph{hypothesis} and $Q$ the \emph{conclusion}.  

The convergence tests for series can be formulated as conditional statements.  Understanding the general logic associated to conditional statements can be very helpful to understand what the tests allow us to conclude as well as what they do not allows us to conclude.

The order of the hypothesis and the conclusion matters.  For instance, consider the following two statements.
\begin{enumerate}
\item If it is raining, then Laura brings her umbrella.
\item If Laura brings her umbrella, then it is raining.
\end{enumerate}
In the first statement, the hypothesis of
\wordChoice{
	\choice[correct]{``it is raining''}
	\choice{``Laura brings her umbrella''}
}
leads to the conclusion of
\wordChoice{
	\choice{``it is raining''}
	\choice[correct]{``Laura brings her umbrella''}
}.
Notice this statement gives us no information about what happens when it is not raining.  Maybe she always brings her umbrella, or maybe she is reacting to the rain.

In the second statement, the hypothesis of
\wordChoice{
	\choice{``it is raining''}
	\choice[correct]{``Laura brings her umbrella''}
}
leads to the conclusion of
\wordChoice{
	\choice[correct]{``it is raining''}
	\choice{``Laura brings her umbrella''}
}.
This describes a different situation, in which Laura brings her umbrella only in the case when it is raining.  We have no information about the weather when she doesn't bring her umbrella; maybe she frequently forgets her umbrella when it is raining.

Observe that, despite being built from essentially the same types of information, the truth of either statement is independent of the other.  This is a typical feature of converse statements.

\begin{definition}
Given a conditional statement ``If $P$, then $Q$.'' we define its \emph{converse} as the statement ``If $Q$, then $P$.''
\end{definition}

In general, the truth of a statement and the truth of its converse are \emph{unrelated}.  For an example relevant to our work on series, recall that the divergence test is an example of a conditional statement:


\begin{quote}
If $\lim_{k \rightarrow \infty} a_k \neq 0$, then $\sum_{k = 1}^\infty a_k$ diverges.
\end{quote}

Which of the following is the converse to the statement above?
\begin{multipleChoice}
\choice{If $\lim_{k \rightarrow \infty} a_k \neq 0$, then $\sum_{k = 1}^\infty a_k$ diverges.}
\choice{If $\lim_{k \rightarrow \infty} a_k = 0$, then $\sum_{k = 1}^\infty a_k$ converges.}
\choice{If $\sum_{k = 1}^\infty a_k$ converges, then $\lim_{k \rightarrow \infty} a_k = 0$.}
\choice[correct]{If $\sum_{k = 1}^\infty a_k$ diverges, then $\lim_{k \rightarrow \infty} a_k \neq 0$.}
\end{multipleChoice}

We know that, in general, this converse is
\wordChoice{
\choice{true}
\choice[correct]{false}
}.
For example, the harmonic series satisfies the hypothesis ``$\sum_{k = 1}^\infty \frac{1}{k}$ diverges'' but the conclusion ``$\lim_{k \rightarrow \infty} \frac{1}{k} \neq 0$'' is
\wordChoice{
\choice{true}
\choice[correct]{false}
}.


\begin{exercise}
It is usually more helpful to rearrange a conditional statement into a logically equivalent form.  For instance, consider the following two statements:

\begin{enumerate}
	\item If it is raining, then Laura brings her umbrella.
	\item If Laura does not bring her umbrella, then it is not raining.
\end{enumerate}

We saw the first statement in the previous example.  In the second statement, the hypothesis of ``Laura does not bring her umbrella'' leads to the conclusion of ``it is not raining.''  This contains the same content as the first statement.  That is, knowing that the rain leads Laura to bring her umbrella, we also have information about the weather when Laura does not bring her umbrella.  This is an example of contrapositive statements.

\begin{definition}
Given a conditional statement ``If $P$, then $Q$.'' we define its \emph{contrapositive} as the statement ``If $\neg Q$, then $\neg P$).''  We are writing $\neg P$ for the \emph{negation} of $P$, which has the opposite meaning of $P$.  In general, the truth of a statement and the truth of its contrapositive are the \emph{same}, and we say they are \emph{logically equivalent}.
\end{definition}

Recall again the conditional statement of the divergence test:

\begin{quote}
If $\lim_{k \rightarrow \infty} a_k \neq 0$, then $\sum_{k = 1}^\infty a_k$ diverges.
\end{quote}

We will write out its contrapositive.  The negation $\neg$($\lim_{k \rightarrow \infty} a_k \neq 0$) can be rewritten as $\lim_{k \rightarrow \infty} a_k = 0$.  The negation $\neg$($\sum_{k = 1}^\infty a_k$ diverges) can be rewritten as $\sum_{k = 1}^\infty a_k$ converges.  So the contrapositive to the statement of the divergence test above can be written as:

\begin{multipleChoice}
\choice{If $\lim_{k \rightarrow \infty} a_k \neq 0$, then $\sum_{k = 1}^\infty a_k$ diverges.}
\choice{If $\lim_{k \rightarrow \infty} a_k = 0$, then $\sum_{k = 1}^\infty a_k$ converges.}
\choice[correct]{If $\sum_{k = 1}^\infty a_k$ converges, then $\lim_{k \rightarrow \infty} a_k = 0$.}
\choice{If $\sum_{k = 1}^\infty a_k$ diverges, then $\lim_{k \rightarrow \infty} a_k \neq 0$.}
\end{multipleChoice}

This contrapositive form is typically more useful if we want to apply knowledge about the series $\sum_{k = 1}^\infty a_k$ to learn about the limit of the sequence $\{a_k\}_{k = 1}^\infty$.

Suppose $\{a_k\}_{k = 1}^\infty$ is a sequence for which $\lim_{k \rightarrow \infty} \left\lvert \frac{a_{k + 1}}{a_k} \right\rvert$ exists.  Recall that one case of the ratio test tells us that

\begin{quote}
If $\lim_{k \rightarrow \infty} \left\lvert \frac{a_{k + 1}}{a_k} \right\rvert < 1$, then $\sum_{k = 1}^\infty a_k$ converges.
\end{quote}

What is the negation of the hypothesis $\lim_{k \rightarrow \infty} \left\lvert \frac{a_{k + 1}}{a_k} \right\rvert < 1$?

\begin{multipleChoice}
\choice{$\lim_{k \rightarrow \infty} \left\lvert \frac{a_{k + 1}}{a_k} \right\rvert < 1$}
\choice[correct]{$\lim_{k \rightarrow \infty} \left\lvert \frac{a_{k + 1}}{a_k} \right\rvert \geq 1$}
\choice{$\lim_{k \rightarrow \infty} \left\lvert \frac{a_{k + 1}}{a_k} \right\rvert > 1$}
\choice{$\lim_{k \rightarrow \infty} \left\lvert \frac{a_{k + 1}}{a_k} \right\rvert \leq 1$}
\end{multipleChoice}

What is the contrapositive of the case of the ratio test above?
\begin{multipleChoice}
\choice{If $\lim_{k \rightarrow \infty} \left\lvert \frac{a_{k + 1}}{a_k} \right\rvert < 1$, then $\sum_{k = 1}^\infty a_k$ converges.}
\choice{If $\lim_{k \rightarrow \infty} \left\lvert \frac{a_{k + 1}}{a_k} \right\rvert \geq 1$, then $\sum_{k = 1}^\infty a_k$ diverges.}
\choice{If $\sum_{k = 1}^\infty a_k$ converges, then $\lim_{k \rightarrow \infty} \left\lvert \frac{a_{k + 1}}{a_k} \right\rvert < 1$.}
\choice[correct]{If $\sum_{k = 1}^\infty a_k$ diverges, then $\lim_{k \rightarrow \infty} \left\lvert \frac{a_{k + 1}}{a_k} \right\rvert \geq 1$.}
\end{multipleChoice}

What is the converse of the case of the ratio test above?
\begin{multipleChoice}
\choice{If $\lim_{k \rightarrow \infty} \left\lvert \frac{a_{k + 1}}{a_k} \right\rvert < 1$, then $\sum_{k = 1}^\infty a_k$ converges.}
\choice{If $\lim_{k \rightarrow \infty} \left\lvert \frac{a_{k + 1}}{a_k} \right\rvert \geq 1$, then $\sum_{k = 1}^\infty a_k$ diverges.}
\choice[correct]{If $\sum_{k = 1}^\infty a_k$ converges, then $\lim_{k \rightarrow \infty} \left\lvert \frac{a_{k + 1}}{a_k} \right\rvert < 1$.}
\choice{If $\sum_{k = 1}^\infty a_k$ diverges, then $\lim_{k \rightarrow \infty} \left\lvert \frac{a_{k + 1}}{a_k} \right\rvert \geq 1$.}
\end{multipleChoice}

Since we know the ratio test is true, which of the following \emph{must} be true?
\begin{multipleChoice}
\choice{The converse must be true.}
\choice[correct]{The contrapositive must be true.}
\choice{Both must be true.}
\choice{Neither must be true.}
\end{multipleChoice}

\begin{feedback}
A good example that demonstrates explicitly that the converse to the case of the ratio test is not true is any convergent $p$-series; if $p>1$, then $\sum_{k=1}^{\infty} \frac{1}{k^p}$ converges, but $\lim_{k \to \infty} \frac{k^{p+1}}{k^p} =1$.
\end{feedback}

\end{exercise}
\end{exercise}
\end{document}