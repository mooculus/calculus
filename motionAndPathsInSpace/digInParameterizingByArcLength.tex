\documentclass{ximera}

\newcommand{\RR}{\mathbb R}
\renewcommand{\d}{\,d}
\newcommand{\dd}[2][]{\frac{d #1}{d #2}}
\renewcommand{\l}{\ell}
\newcommand{\ddx}{\frac{d}{dx}}
\newcommand{\dfn}{\textbf}
\newcommand{\eval}[1]{\bigg[ #1 \bigg]}


\title[Dig-In:]{Parameterizing by arc length}

\outcome{Reparametrize a curve.}
\outcome{Parameterize a curve in terms of arc length.}

\begin{document}
\begin{abstract}
  We find a new description of curves that trivializes arc length
  computations.
\end{abstract}
\maketitle

Recall that if $\vec{f}$ is a vector-valued function where
\begin{itemize}
\item $\vec{f}'$ is continuous.
\item The curve defined by $\vec{f}(t)$ is traversed once for $a\le
  t\le b$.
\end{itemize}
  The arc length of the curve from
  \[
  \vec{f}(a)\quad\text{to}\quad\vec{f}(b)
  \]
  is given by
  \[
  \text{arc length} = \int_a^b |\vec{f}'(t)|\d t.
  \]
  This is all good and well; however, the integral
  \[
  \int_a^b |\vec{f}'(t)|\d t
  \]
  could be quite difficult to compute. In this section, we see a new
  description of the curve drawn by $\vec{f}(t)$, we'll call it
  $\vec{g}(s)$ where the \textbf{same} curve is drawn by both $\vec{f}$ and
  $\vec{g}$ and we have that
  \[
  s =  \int_0^s |\vec{g}'(t)|\d t.
  \]
  This is called an \dfn{arc length parameterization}. It is nice to
  work with functions parameterized by arc length, because computing
  the arc length is easy. If $g$ is parameterized by arc length, then
  the length of $g(s)$ when $a\le s\le b$, is simply $b-a$. No
  integral computations need to be done.  Consider the following
  example:

  \begin{example}
    Let $\vec{f}(t) = \vector{\cos(t), \sin(t)}$ for $0\le t<
    2\pi$. Show that $\vec{f}$ is parameterized by arc length.
    \begin{explanation}
      Here we need to show that
      \[
      s = \int_0^s |\vec{f}'(t)| \d t.
      \]
      We'll just compute the right-hand side of the equation above and
      see what happens. Write with me,
      \[
      \vec{f}'(t) = \vector{\answer[given]{-\sin(t)},\answer[given]{\cos(t)}}
      \]
      and so
      \begin{align*}
      |\vec{f}'(t)| &= \sqrt{\vec{f}'(t)\dotp \vec{f}'(t)}\\
      &= \sqrt{\vector{\answer[given]{-\sin(t)},\answer[given]{\cos(t)}}\dotp\vector{\answer[given]{-\sin(t)},\answer[given]{\cos(t)}}}\\
      &= \answer[given]{1}.
      \end{align*}
      Now our integral becomes:
      \begin{align*}
        \int_0^s  |\vec{f}'(t)| \d t &= \int_0^s \d t\\
        &= \answer[given]{s}.
      \end{align*}
      Hence $\vec{f}$ is parameterized by arc length.
    \end{explanation}
  \end{example}

  From your own experience and the work above, we think the next
  theorem should be quite sensible.

  \begin{theorem}
    A vector-valued function $\vec{f}:\R\to \R^2$ is parameterized by
    arc length if and only if $|\vec{f}'| = 1$.
  \end{theorem}

  \begin{question}
    Which of the following vector-valued functions are parameterized
    by arc length?
    \begin{selectAll}
      \choice[correct]{$t\vector{11/61,60/61}$}
      \choice[correct]{$\vector{3\sin(t/3),3\cos(t/3)}$}
      \choice{$t\vector{16/113,112/113}$}
      \choice[correct]{$\vector{7,t17/145,t144/145}$}
      \choice{$\vector{3\cos(t),3\sin(t)}$}
    \end{selectAll}
  \end{question}

  We proceed by discussing several special cases, and then by giving a
  general method.

  \section{Disguised lines}

  Sometimes you have a vector-valued function that is merely a line in
  disguise.\index{line!in disguise} How could this be? Well consider
  the vector-valued function:
  \[
  \vec{f}(t) = \vector{2-3\sin(t),1+4\sin(t)}\quad \text{for $-\pi/2\le t\le \pi/2$}
  \]
  This doesn't look very much like a line, for one thing it has the
  function $\sin(t)$ in each component. On the other hand, if we look
  at $\vec{f}'$, we see
  \[
  \vec{f}'(t) = \vector{-3\cos(t),4\cos(t)}
  \]
  Ah, we can now factor a $\cos(t)$ out of each component to get:
  \[
  \underbrace{\cos(t)}_{\text{scalar function}}\cdot \overbrace{\vector{-3,4}}^{\text{constant vector}}
  \]
  this is a scalar-function times a constant vector. The fact that we
  can ``pull-out'' the scalar function, and are left with a constant
  vector tells us that the line segment plotted by $\vec{f}$ for
  $-\pi/2\le t\le \pi/2$ is identical to the line segment plotted by:
  \[
  \vec{g}(s) = \vector{2-3s,1+4s}\quad -1\le s\le 1
  \]
  \begin{question}
    Which of the following are line segments in disguise?
    \begin{selectAll}
      \choice[correct]{$\vector{2+e^t,4,-2-e^t}$ for $0\le t\le 2$}
      \choice{$\vector{3+e^t,-1+2e^t,2-e^{2t}}$ for $0\le t\le 1$}
      \choice[correct]{$\vector{5-3\cos(t),4+2\cos(t),1+\cos(t)}$ for $0\le t\le 2\pi$}
      \choice{$\vector{-1 -\sin(t),3+\sin(t),2-\cos(t)}$ for $0\le t\le \pi/2$}
      \choice{$\vector{2+5t,3t^2}$ for $-1\le t\le 1$}
      \choice[correct]{$\vector{3-t^3,1+2t^3}$ for $-1\le t\le 1$}
    \end{selectAll}
  \end{question}
  Once we identify a vector-valued function as a disguised line, we
  can rewrite it as
  \[
  \text{point}+ s \cdot \text{unit vector}
  \]
  and we have an arc length parameterization.
  \begin{example}
    Consider $\vec{f}(t) = \vector{3t^2,4t^2}$ for $0\le t\le
    1$. Parameterize this curve by arc length.
    \begin{explanation}
      If we think about $\vec{f}$ we see that the variable $t$ only
      appears in the expression as $t^2$. This means as $t$ grows, it
      will grow \textit{identically} in each component of $\vec{f}$.
      \begin{onlineOnly}
        Indeed a quick check with a graph will show that:
        \[
        \graph{(3t, 4t)}
        \]
        \[
        \graph{(3t^2,4t^2)}
        \]
        produce the same graph.
      \end{onlineOnly}
      Now, by our theorem above, $\vec{g}$ is parameterized by arc
      length if and only if
      \[
      |\vec{g}'(t)| = \answer[given]{1}
      \]
      Hence we need to find a unit vector in the the same direction as
      the line drawn by $\vec{f}$. Write with me,
      \begin{align*}
        \frac{\vec{f}'(t)}{|\vec{f}'(t)|} &= \frac{\vector{\answer[given]{6t},\answer[given]{8t}}}{\answer[given]{10t}}\\
        &=\vector{\answer[given]{3/5},\answer[given]{4/5}}.
      \end{align*}
      Hence $\vec{g}(s) =
      s\vector{\answer[given]{3/5},\answer[given]{4/5}}$ draws the
      same curve as $\vec{f}$ and is parameterized by arc length.
    \end{explanation}
  \end{example}

  \begin{question}
    Give an arc length parameterization of $\vec{f}(t) =
    \vector{3-4t^3,2+t^3,5-t^3}$ for $0\le t\le 1$.
    \begin{prompt}
      \[
      \vec{g}(s) =
      \vector{\answer{3-4s/\sqrt{18}},\answer{2+s/\sqrt{18}},\answer{5-s/\sqrt{18}}}
      \]
    \end{prompt}
  \end{question}

  Try your hand at this one now:

    \begin{question}
    Give an arc length parameterization of $\vec{f}(t) =
    \vector{1-e^t,3+e^t,5}$ for $0\le t$.
    \begin{hint}
      Check the values of $\vec{f}(0)$ and $\vec{f}(1)$.
    \end{hint}
    \begin{prompt}
      \[
      \vec{g}(s) =
      \vector{\answer{-s/\sqrt{2}},\answer{4+s/\sqrt{2}},\answer{5}}
      \]
    \end{prompt}
  \end{question}

  
  \section{Disguised circles}

  Sometimes the curve we are given is a circle in disguise.
  \index{circle!in disguise}

  \begin{example}
    Consider $\vec{f}(t) = \vector{\sin(2\pi t^2),\cos(2\pi t^2)}$ for
    $0\le t\le 1$. Parameterize this curve by arc length.
    \begin{explanation}
      Here, we should recognize this curve a unit circle, being drawn
      in a counterclockwise fashion, starting (when $t=0$) at the
      point $\left(\answer[given]{0},\answer[given]{1}\right)$. Ah! So
      an arc length parameterization is given by
      \[
      \vec{g}(s) = \vector{\answer[given]{\sin(s)},\answer[given]{\cos(s)}}.
      \]
    \end{explanation}
  \end{example}


  \begin{question}
    Consider $\vec{f}(t) = \vector{3\sin(a t),t/2,3\cos(a t)}$ for
    $0\le t$. Find $a$ that makes this parameterized by arc length.
    \begin{hint}
      Set $|\vec{f}'(t)| = 1$ and solve for $a$.
    \end{hint}
    \begin{prompt}
      \[
      a = \pm \answer{\frac{1}{2\sqrt{3}}}
      \]
    \end{prompt}
  \end{question}
  

  
  \section{A general method}

  While we are about to present a general method for finding
  representations of functions parameterized by arc length, one must
  not overestimate its strength.

  Regardless, if you want an arc length parameterization of $\vec{f}(t)$
  starting at $t=a$ here is the idea:
  \begin{enumerate}
  \item Compute
    \[
    L(t)  = \int_a^t |\vec{f}'(u)| \d u
    \]
  \item Now write
    \[
    s = L(t)
    \]
    and solve for $t$. In this case you will have
    \[
    t = L^{-1}(s)
    \]
  \item The function
  \[
  \vec{g}(s) = \vec{f}(L^{-1}(s))
  \]
  will be parameterized by arc length.
  \end{enumerate}

  Try your hand at it.

  \begin{example}
    Parameterize $\vec{f}(t) = \vector{6t^3,8t^3,3}$ for $t\ge 0$ by
    arc length.
    \begin{explanation}
      Write with me,
      \begin{align*}
      |\vec{f}'(t)| &= \sqrt{\answer[given]{18^2t^4+24^2t^4}}\\
      &=\answer[given]{30t^2}
      \end{align*}
      \begin{align*}
      L(t) &= \int_0^t 30 u^2 \d u\\
      &= \eval{\answer[given]{10 u^3}}_0^t\\
      &= \answer[given]{10t^3}.
      \end{align*}
      So now set $s = \answer[given]{10t^3}$ and solve for $t$.
      \[
      t = \sqrt[3]{\answer[given]{s/10}}
      \]
      Our function parameterized by arc length is
      \[
      \vec{g}(s) = \vector{\answer[given]{6s/10},\answer[given]{8s/10},\answer[given]{3}}.
      \]
    \end{explanation}
  \end{example}


  
\end{document}
