\documentclass{ximera}

\newcommand{\RR}{\mathbb R}
\renewcommand{\d}{\,d}
\newcommand{\dd}[2][]{\frac{d #1}{d #2}}
\renewcommand{\l}{\ell}
\newcommand{\ddx}{\frac{d}{dx}}
\newcommand{\dfn}{\textbf}
\newcommand{\eval}[1]{\bigg[ #1 \bigg]}


\outcome{Understand the statement of the Extreme Value Theorem.}

\title[Dig-In:]{The Extreme Value Theorem}

\begin{document}
\begin{abstract}
  We examine a fact about continuous functions.
\end{abstract}
\maketitle

\begin{definition}\hfil\index{maximum/minimum!absolute}
\begin{enumerate}
\item A function $f$ has an \dfn{global maximum} at $x=a$, if $f(a)\ge
  f(x)$ for every $x$ in the domain of the function.
\item A function $f$ has an \dfn{global minimum} at $x=a$, if $f(a)\le
  f(x)$ for every $x$ in the domain of the function.
\end{enumerate} 
A \dfn{global extremum}\index{extremum!global} is either a
global maximum or a global minimum.
\end{definition}

If we are working on an finite closed interval, then we have the
following theorem.

\begin{theorem}[Extreme Value Theorem]\label{theorem:evt}\index{Extreme Value Theorem}
If $f$ is a continuous function for all $x$ in the closed interval
$[a,b]$, then there are points $c$ and $d$ in $[a,b]$, such that
$(c,f(c))$ is a global maximum and $(d,f(d))$ is a global
minimum on $[a, b]$.

Below, we see a geometric interpretation of this theorem.
\begin{image}
\begin{tikzpicture}
	\begin{axis}[
            domain=0:6, xmin=0, xmax=6, ymin=0, ymax=2.5,
            axis lines =left, xlabel=$x$, ylabel=$y$,
            every axis y label/.style={at=(current axis.above origin),anchor=south},
            every axis x label/.style={at=(current axis.right of origin),anchor=west},
            xtick={1,2,4,5}, ytick={.2,2.2},
            xticklabels={$a$,$c$,$d$,$b$}, yticklabels={$f(d)$,$f(c)$},
            axis on top,
          ]
          \addplot [draw=none, fill=fill1, domain=(1:5)] {2.5} \closedcycle;

          \addplot [textColor,dashed] plot coordinates {(0,2.2) (2,2.2)};
          \addplot [textColor,dashed] plot coordinates {(0,.2) (4,.2)};
          \addplot [textColor,dashed] plot coordinates {(2,0) (2,2.2)};
          \addplot [textColor,dashed] plot coordinates {(4,0) (4,.2)};

          \addplot [fill1,very thick] plot coordinates {(1,0) (1,2.5)};
          \addplot [fill1,very thick] plot coordinates {(5,0) (5,2.5)};

          \addplot [very thick,penColor, smooth,domain=(1.5:2.5)] {sin(deg(x*1.57-1.57)) + 1.2};%max
          \addplot [very thick,penColor, smooth,domain=(3.5:4.5)] {sin(deg(x*1.57-1.57)) + 1.2};%min
          \addplot [very thick,dashed,penColor!50!background, smooth,domain=(2.5:3.5)] {sin(deg(x*1.57 - 1.57)) + 1.2}; 
          \addplot [very thick,dashed,penColor!50!background, smooth,domain=(1:1.5)] {sin(deg(x*1.57 - 1.57)) + 1.2}; 
          \addplot [very thick,dashed,penColor!50!background, smooth,domain=(4.5:5)] {sin(deg(x*1.57 - 1.57)) + 1.2}; 
          
          \addplot [color=penColor,fill=penColor,only marks,mark=*] coordinates{(2,2.2)};  %% closed hole          
          \addplot [color=penColor,fill=penColor,only marks,mark=*] coordinates{(4,.2)};  %% closed hole          
        \end{axis}
\end{tikzpicture}
%% \caption{A geometric interpretation of the Extreme Value Theorem. A
%%   continuous function $f(x)$ attains both an global maximum and an
%%   global minimum on an interval $[a,b]$. Note, it may be the case
%%   that $a=c$, $b=d$, or that $d<c$.}
%% \label{figure:extreme-value}
%% \end{marginfigure}
\end{image}
\end{theorem}
\begin{question}
  Would this theorem hold if we were working on an open interval?
  \begin{multipleChoice}
    \choice{yes}
    \choice[correct]{no}
  \end{multipleChoice}
  \begin{hint}
    Consider $\tan(\theta)$ for $-\pi/2 < \theta < \pi/2$. Does this function achieve its maximum and minimum? 
  \end{hint}
\end{question}

\end{document}
