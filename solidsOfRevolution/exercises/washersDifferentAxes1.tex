\documentclass{ximera}
\newcommand{\RR}{\mathbb R}
\renewcommand{\d}{\,d}
\newcommand{\dd}[2][]{\frac{d #1}{d #2}}
\renewcommand{\l}{\ell}
\newcommand{\ddx}{\frac{d}{dx}}
\newcommand{\dfn}{\textbf}
\newcommand{\eval}[1]{\bigg[ #1 \bigg]}

\author{Jim Talamo }
\license{Creative Commons 3.0 By-NC}
\outcome{Set up a volume integral using the Washer Method}
\begin{document}
\begin{exercise}

	Let $R$ be the region in the $xy$-plane bounded between $y=\sqrt[3]{x}$, $x=2$, and the $x$-axis.  This exercise gives practice using the Washer Method to set up an integral that gives the volume of the solid of revolution obtained when $R$ is rotated about several different axes.
	


When using the Washer Method, the slices must be:
\begin{multipleChoice}
\choice{parallel to the axis of rotation.}
\choice[correct]{perpendicular to the axis of rotation.}
\end{multipleChoice}



\begin{exercise}
Suppose that $R$ is revolved about the $y$-axis.  To use the Washer Method, the slices must be:

\begin{multipleChoice}
\choice{vertical}
\choice[correct]{horizontal}
\end{multipleChoice}
\begin{exercise}

We thus have a helpful version of the picture of the region $R$ below:

  \begin{image}
            \begin{tikzpicture}
            	\begin{axis}[
            		domain=-.9:2.5, ymax=2.5,xmax=2.4, ymin=-0.5, xmin=-.4,
            		axis lines =center, xlabel=$x$, ylabel=$y$,
            		every axis y label/.style={at=(current axis.above origin),anchor=south},
            		every axis x label/.style={at=(current axis.right of origin),anchor=west},
            		axis on top,
            		]
                      
            	\addplot [draw=penColor2,very thick,smooth,samples=100,domain=0:.1] {x^(1/3)};
	        \addplot [draw=penColor2,very thick,smooth,domain=.1:2.4] {x^(1/3)};
		\addplot [draw=penColor3,very thick,smooth] {0};
		\addplot [draw=penColor,very thick,smooth] coordinates {(2,-10)(2,10)};
		\addplot [draw=penColor5,very thick,dashed] coordinates {(0,-10)(0,10)};
		                       
            	\addplot [name path=A,domain=0:2,draw=none,samples=200] {x^(1/3)};   
            	\addplot [name path=B,domain=0:2,draw=none] {0};
            	\addplot [fillp] fill between[of=A and B];
	                
		\node at (axis cs:1.75,2.25) [penColor] {$x=2$};
		\node at (axis cs:1.2,1.3) [penColor2] {$x=y^3$};  
		\addplot [draw=penColor, fill = gray!50] plot coordinates {(.22,.5) (1.995,.5) (1.995,.6) (.22,.6) (.22,.5)};
          
          %Draw R and r
          \addplot [draw=red,very thick] coordinates {(0,1)(2,1)};
          \node at (axis cs:.7,1.15) [red] {$R$};
          
	 \addplot [draw=blue,very thick] coordinates {(0,.75)(.42,.75)};

	 \node at (axis cs:.25,.85)  [blue]  {$r$};
                      
                  	\end{axis}
            \end{tikzpicture}
  \end{image}
            
 We see from the picture that both $R$ and $r$ are:
 \begin{multipleChoice}
 \choice{vertical distances}
 \choice[correct]{horizontal distances}
 \end{multipleChoice}           
            
\begin{exercise}
Since $R$ is the distance from the axis of rotation to the outer curve, and this is a horizontal distance, we find $R = x_{right}-x_{left}$.
\begin{multipleChoice}
 \choice[correct]{$x_{right} = 2$}
 \choice{$x_{right} = y^3$}
  \choice{$x_{right} = 0$}
\end{multipleChoice}       

\begin{multipleChoice}
 \choice{$x_{left} = 2$}
 \choice{$x_{left} = y^3$}
  \choice[correct]{$x_{left} = 0$}
\end{multipleChoice}   

So, $R= \answer{2}$.

 
 \begin{exercise}

Since $r$ is the distance from the axis of rotation to the inner curve, and this is a horizontal distance, we find $r = x_{right}-x_{left}$.
\begin{multipleChoice}
 \choice{$x_{right} = 2$}
 \choice[correct]{$x_{right} = y^3$}
  \choice{$x_{right} = 0$}
\end{multipleChoice}       

\begin{multipleChoice}
 \choice{$x_{left} = 2$}
 \choice{$x_{left} = y^3$}
  \choice[correct]{$x_{left} = 0$}
\end{multipleChoice} 

So, $r= \answer{y^3}$.

\begin{exercise}
Using the formula $V = \int_{y=c}^{y=d} \pi (R^2-r^2) \d y$, an integral that gives the volume of this solid of revolution is:

\[
V = \int_{y=0}^{y=8} \pi \answer{(2^2-(y^3)^2)} \d y
\]

\end{exercise}
 \end{exercise}
\end{exercise}    
\end{exercise}
\end{exercise}
\end{exercise}




 
%%%%%
\begin{exercise}
Suppose that $R$ is revolved about the line $y=-2$.


\begin{exercise}

\begin{multipleChoice}
\choice[correct]{vertical}
\choice{horizontal}
\end{multipleChoice}

\begin{exercise}

We thus have a helpful version of the picture of the region $R$ below:

  \begin{image}
            \begin{tikzpicture}
            	\begin{axis}[
            		domain=-.9:2.5, ymax=2.5,xmax=2.4, ymin=-2.5, xmin=-.4,
            		axis lines =center, xlabel=$x$, ylabel=$y$,
            		every axis y label/.style={at=(current axis.above origin),anchor=south},
            		every axis x label/.style={at=(current axis.right of origin),anchor=west},
            		axis on top,
            		]
                      
            	\addplot [draw=penColor2,very thick,smooth,samples=100,domain=0:.1] {x^(1/3)};
	        \addplot [draw=penColor2,very thick,smooth,domain=.1:2.4] {x^(1/3)};
		\addplot [draw=penColor3,very thick,smooth] {0};
		\addplot [draw=penColor,very thick,smooth] coordinates {(2,-10)(2,10)};
		\addplot [draw=penColor5,very thick,dashed] coordinates {(-10,-2)(10,-2)};
		                       
            	\addplot [name path=A,domain=0:2,draw=none,samples=200] {x^(1/3)};   
            	\addplot [name path=B,domain=0:2,draw=none] {0};
            	\addplot [fillp] fill between[of=A and B];
	         \node at (axis cs:1.25,1.4) [penColor2] {$y=\sqrt[3]{x}$};       
		\node at (axis cs:1.75,2.25) [penColor] {$x=2$};
	
		\addplot [draw=penColor, fill = gray!50] plot coordinates {(.74,.01) (.81,.01) (.81,.88) (.74,.88) (.74,.01)};
          
          %Draw R and r
          \addplot [draw=red,very thick] coordinates {(.65,-2)(.65,.865)};
          \node at (axis cs:.75,-.8) [red] {$R$};
          
	 \addplot [draw=blue,very thick] coordinates {(.9,-2)(.9,0)};

	 \node at (axis cs:1,-1.0)  [blue]  {$r$};
                      
                  	\end{axis}
            \end{tikzpicture}
  \end{image}
            
 We see from the picture that both $R$ and $r$ are:
 \begin{multipleChoice}
 \choice[correct]{vertical distances}
 \choice{horizontal distances}
 \end{multipleChoice}           
            
\begin{exercise}
Since $R$ is the distance from the axis of rotation to the outer curve, and this is a vertical distance, we find $R = y_{top}-y_{bot}$.
\begin{multipleChoice}
 \choice{$y_{top} = -2$}
 \choice{$y_{top} = 0$}
  \choice[correct]{$y_{top} = \sqrt[3]{x}$}
\end{multipleChoice}       

\begin{multipleChoice}
 \choice[correct]{$y_{bot} = -2$}
 \choice{$y_{bot} = 0$}
  \choice{$y_{bot} = \sqrt[3]{x}$}
\end{multipleChoice}   

So, $R= \answer{x^{1/3}-(-2)}$.
 \end{exercise}
 
 \begin{exercise}

Since $r$ is the distance from the axis of rotation to the inner curve, and this is a vertical distance, we find $r = y_{top}-y_{bot}$.
\begin{multipleChoice}
 \choice{$y_{top} =-2$}
 \choice[correct]{$y_{top} = 0$}
  \choice{$y_{top} = \sqrt[3]{x}$}
\end{multipleChoice}       

\begin{multipleChoice}
 \choice[correct]{$y_{bot} = -2$}
 \choice{$y_{bot} =0$}
  \choice{$y_{bot} = \sqrt[3]{x}$}
\end{multipleChoice} 

So, $r= \answer{0-(-2)}$.


\begin{exercise}
Using the formula $V = \int_{x=a}^{x=b} \pi (R^2-r^2) \d x$, an integral that gives the volume of this solid of revolution is:

\[
V = \int_{x=0}^{x=2} \pi \answer{((x^{1/3}+2)^2-(2)^2)} \d y
\]

\end{exercise}
\end{exercise}    
\end{exercise}
\end{exercise}
\end{exercise}

%%%%%%%%%%%%%%%
%%%%%%%%%%%%%%%%%%%%%%%

\begin{exercise}
Suppose that $R$ is revolved about the line $x=5$.


\begin{exercise}



We thus have a helpful version of the picture of the region $R$ below:

 \begin{image}
            \begin{tikzpicture}
            	\begin{axis}[
            		domain=-.9:5.5, ymax=2.5,xmax=5.4, ymin=-0.5, xmin=-.4,
            		axis lines =center, xlabel=$x$, ylabel=$y$,
            		every axis y label/.style={at=(current axis.above origin),anchor=south},
            		every axis x label/.style={at=(current axis.right of origin),anchor=west},
            		axis on top,
            		]
                      
            	\addplot [draw=penColor2,very thick,smooth,samples=100,domain=0:.1] {x^(1/3)};
	        \addplot [draw=penColor2,very thick,smooth,domain=.1:5.4] {x^(1/3)};
		\addplot [draw=penColor3,very thick,smooth] {0};
		\addplot [draw=penColor,very thick,smooth] coordinates {(2,-10)(2,10)};
		\addplot [draw=penColor5,very thick,dashed] coordinates {(5,-10)(5,10)};
		                       
            	\addplot [name path=A,domain=0:2,draw=none,samples=200] {x^(1/3)};   
            	\addplot [name path=B,domain=0:2,draw=none] {0};
            	\addplot [fillp] fill between[of=A and B];
	                
		\node at (axis cs:1.5,2.25) [penColor] {$x=2$};
		\node at (axis cs:4.5,2) [penColor5] {$x=5$};
		\node at (axis cs:1.2,1.3) [penColor2] {$x=y^3$};  
		\addplot [draw=penColor, fill = gray!50] plot coordinates {(.22,.5) (1.995,.5) (1.995,.6) (.22,.6) (.22,.5)};
          
          %Draw R and r
          \addplot [draw=red,very thick] coordinates {(1,1)(5,1)};
          \node at (axis cs:3,1.15) [red] {$R$};
          
	 \addplot [draw=blue,very thick] coordinates {(2,.75)(5,.75)};

	 \node at (axis cs:3.3,.85)  [blue]  {$r$};
                      
                  	\end{axis}
            \end{tikzpicture}
  \end{image}
                 
\begin{exercise}
From the picture, we find that $R = \answer{5-y^3}$ and $r= \answer{5-2 }$.

\begin{exercise}
Thus, an integral that gives the volume of this solid of revolution is:

\[
V = \int_{y=0}^{y=8} \pi \answer{(5-y^3)^2-(3)^2)} \d y
\]
\end{exercise}


\end{exercise}
\end{exercise}












\end{exercise}
\end{document}