\documentclass{ximera}

%\newcommand{\RR}{\mathbb R}
\renewcommand{\d}{\,d}
\newcommand{\dd}[2][]{\frac{d #1}{d #2}}
\renewcommand{\l}{\ell}
\newcommand{\ddx}{\frac{d}{dx}}
\newcommand{\dfn}{\textbf}
\newcommand{\eval}[1]{\bigg[ #1 \bigg]}

\newcommand{\RR}{\mathbb R}
\renewcommand{\d}{\,d}
\newcommand{\dd}[2][]{\frac{d #1}{d #2}}
\renewcommand{\l}{\ell}
\newcommand{\ddx}{\frac{d}{dx}}
\newcommand{\dfn}{\textbf}
\newcommand{\eval}[1]{\bigg[ #1 \bigg]}


\author{}
\license{Creative Commons 3.0 By-NC}
\outcome{}
\begin{document}

\begin{exercise}

  When evaluating integrals of trigonometric functions with the same argument (i.e. inner function...the argument of $\sin(3x)$ would be $3x$, it is often necessary to convert powers of one trigonometric function into powers of the other.
    This is most easily done via the Pythagorean identities: $\sin^2(x) + \cos^2(x) = 1$, $\tan^2(x) + 1 = \sec^2(x)$, and $1 + \cot^2(x) = \csc^2(x)$.
    
    Using these formulas, it is \emph{easy} to express even powers of one of the trigonometric functions into powers of the complementary function in the above formulas.
    The following problems give you practice with this.
  \begin{multipleChoice}
    \choice[correct]{I understand.}
    \choice{I do not understand.}
  \end{multipleChoice}

\begin{problem}
  Using the appropriate Pythagorean trigonometric identity, express $\sin^2(x) \cos^2(x)$ as a sum of powers of $\cos(x)$:
  \[
    \sin^2(x) \cos^2(x) = \answer{\cos^2(x) - \cos^4(x)}
  \]
\end{problem}

\begin{problem}
  Using the appropriate Pythagorean trigonometric identity, express $\sec^4(x) \tan^2(x)$ as:
  \begin{enumerate}
    \item A sum of powers of $\sec(x)$
      \[
        \sec^4(x) \tan^2(x) = \answer{\sec^6(x) - \sec^4(x)}
      \]
      (Simplify your final answer)
    \item A sum of powers of $\tan(x)$:
      \[
        \sec^4(x) \tan^2(x) = \answer{\tan^6(x) + 2\tan^4(x) + \tan^2(x)}
      \]
      (Simplify your final answer)
  \end{enumerate}
\end{problem}

\begin{problem}
  For the following expressions, fill in the half-angle formulas:
  $\sin^2(x) = \answer{\frac{1 - \cos(2x)}{2}}$ and $\cos^2(x) = \answer{\frac{1 + \cos(2x)}{2}}$.  
  
  \begin{hint}
    Look these up.
  \end{hint}
\end{problem}


\begin{problem}
  Sometimes when evaluating integrals of trigonometric functions, it becomes necessary to use the double angle formulas for sines and cosines
  \[
    \sin^2(x) = \frac{1}{2} - \frac{1}{2}\cos(2x)
  \]
  \[
    \cos^2(x) = \frac{1}{2} + \frac{1}{2}\cos(2x)
  \]
  to write even powers of sines and cosines as sums of cosines of linear functions of $x$.
  For instance:
  \begin{align*}
    \cos^4(x) &= (\cos^2(x))^2 \\
    &= \left(\frac{1}{2} + \frac{1}{2}\cos(2x)\right)^2 \\
    &= \frac{1}{4} + \frac{1}{2}\cos(2x) + \frac{1}{4}\cos^2(2x)
  \end{align*}
  
  We can then use the formula on $\cos^2(2x)$ to write $\cos^2(2x) = \frac{1}{2} + \frac{1}{2}\cos(4x)$, and substitute into the above to obtain:
  \begin{align*}
    \cos^4(x) &= (\cos^2(x))^2\\
    &= \left(\frac{1}{2} + \frac{1}{2}\cos(2x)\right)^2 \\
    &= \frac{1}{4} + \frac{1}{2}\cos(2x) + \frac{1}{4}\left( \frac{1}{2} + \frac{1}{2}\cos(4x)\right) \\
    &= \frac{3}{8} + \frac{1}{2}\cos(2x) + \frac{1}{8}\cos(4x)
  \end{align*}
  
  For practice, write the following in terms of sums of cosines of
  linear powers in $x$:
  
  \begin{enumerate}
    \item $\sin^4(x) = \answer{\frac{3}{8} - \frac{1}{2}\cos(2x) + \frac{1}{8}\cos(4x)}$

    \item $\sin^2(x)\cos^2(x) = \answer{\frac{1}{8} - \frac{1}{8}\cos(4x)}$
  \end{enumerate}
\end{problem}

\end{exercise}
\end{document}
