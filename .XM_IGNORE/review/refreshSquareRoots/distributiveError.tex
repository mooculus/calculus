\documentclass{ximera}
\newcommand{\RR}{\mathbb R}
\renewcommand{\d}{\,d}
\newcommand{\dd}[2][]{\frac{d #1}{d #2}}
\renewcommand{\l}{\ell}
\newcommand{\ddx}{\frac{d}{dx}}
\newcommand{\dfn}{\textbf}
\newcommand{\eval}[1]{\bigg[ #1 \bigg]}


\author{}
\license{Creative Commons 3.0 By-NC}
\outcome{}
\begin{document}

\begin{exercise}

  Square-roots come up by necessity in many of the applications in
  this course, and they are traditionally troublesome to
  handle. Perhaps the most fundamental error that arises is the claim
  that square-roots split-up over addition and subtraction. Note that:
  \[
  \sqrt{a^2+b^2} \ne a+ b
  \]
  \begin{warning}
    \textbf{This type of error can significantly change problems.  If you make this error in any problem, you will generally receive NO
      credit for the rest of the problem. Please do not ever do this!}
  \end{warning}


\begin{problem}
  To verify in general that $\sqrt{a^2+b^2} \ne a+b$, note that if the equality were true, then the left- and right-hand sides must be equal no matter what values we plug in for $a$ and $b$.
  
  Let $a = 4$ and $b = 3$.
  Then $\sqrt{a^2 + b^2} = \answer{5}$.
  (Simplify your answer completely.)
  However, $a + b = \answer{7}$.
  
  Hence, $\sqrt{a^2 + b^2} \ne a + b$ since equality does not hold for any choice of $a$ and $b$.
\end{problem}

\begin{problem}
  Simplify the following expression completely or state that it cannot be simplified further.
  \[
    \sqrt{x^2 + 16}.
  \]
  \begin{multipleChoice}
    \choice{$x + 4$}
    \choice{$|x| + 4$}
    \choice[correct]{This cannot be simplified further.}
  \end{multipleChoice}
\end{problem}

\end{exercise}
\end{document}