\documentclass{ximera}

\newcommand{\RR}{\mathbb R}
\renewcommand{\d}{\,d}
\newcommand{\dd}[2][]{\frac{d #1}{d #2}}
\renewcommand{\l}{\ell}
\newcommand{\ddx}{\frac{d}{dx}}
\newcommand{\dfn}{\textbf}
\newcommand{\eval}[1]{\bigg[ #1 \bigg]}


\author{Jim Talamo and Nicholas Hemleben}
\license{Creative Commons 3.0 By-NC}


\outcome{Set up an integral that gives the area of a surface of revolution}
\outcome{Find  the area of a surface of revolution}

\begin{document}
\begin{exercise}

The portion of the curve $y=2x^3$ from $x=0$ to $x=1$ is shown below:

 \begin{image}
      \begin{tikzpicture}
        \begin{axis}[
            xmin=-.3, xmax=1.2,
            domain=-1:1,
            ymin=-.3, ymax=2.5,
            clip=false,
            xtick = {-1,1},
            ytick = {-1,1,2},
            axis lines =center,
            xlabel=$x$, ylabel=$y$, every axis y label/.style={at=(current axis.above origin),anchor=south},
            every axis x label/.style={at=(current axis.right of origin),anchor=west},
            axis on top,
          ]
                              
         \addplot [penColor,thick,smooth,domain=0:1.1]{2*x^3};
          
         % ds and points
          	\addplot[color=penColor2,fill=penColor2,only marks,mark=*] coordinates{(.4,.128)};
		\addplot[color=penColor2,fill=penColor2,only marks,mark=*] coordinates{(.6,.44)};
		\addplot[ultra thick, penColor2] plot coordinates {(.4,.128) (.6,.44)};
          	\node[anchor=north, penColor2] at (axis cs:.45,.6) {$\Delta s$};
	
          %r and point 
           	\addplot[thick, penColor2] plot coordinates {(.5,0) (.5,.25)};
		\node[anchor=north, penColor2] at (axis cs:.55,.25) {$r$};
          
          
        \addplot[color=penColor,fill=penColor,only marks,mark=*] coordinates{(0,0)};
	\addplot[color=penColor,fill=penColor,only marks,mark=*] coordinates{(1,2)};
          
          \node[penColor] at (axis cs:.6,1.4) {$y=2x^3$};
        \end{axis}
      \end{tikzpicture}
    \end{image}

 The curve is then revolved about the $x$-axis to form a surface of revolution.  The resulting frustum resulting from revolving the slice is shown below:
    
      \begin{image}
      \begin{tikzpicture}
        \begin{axis}[
            xmin=-.3, xmax=1.2,
            domain=-1:1,
            ymin=-2.1, ymax=2.1,
            clip=false,
            xtick = {-1,1},
            ytick = {-1,1,2},
            axis lines =center,
            xlabel=$x$, ylabel=$y$, every axis y label/.style={at=(current axis.above origin),anchor=south},
            every axis x label/.style={at=(current axis.right of origin),anchor=west},
            axis on top,
          ]
                              
         \addplot [penColor,thick,smooth,domain=0:1]{2*x^3};
          
         % ds and points
          	\addplot[color=penColor2,fill=penColor2,only marks,mark=*] coordinates{(.4,.128)};
		\addplot[color=penColor2,fill=penColor2,only marks,mark=*] coordinates{(.6,.44)};
		\addplot[thick, penColor2] plot coordinates {(.4,.128) (.6,.44)};
		\addplot[thick, penColor2] plot coordinates {(.4,-.128) (.6,-.44)};
          	\node[anchor=north, penColor2] at (axis cs:.45,.6) {$\d s$};
          
        \addplot[color=penColor,fill=penColor,only marks,mark=*] coordinates{(0,0)};
	\addplot[color=penColor,fill=penColor,only marks,mark=*] coordinates{(1,2)};
          
          \node[penColor] at (axis cs:.6,1.4) {$y=2x^3$};
          
           %ellipses
           \addplot [penColor2,thick,smooth,domain=.39:.4,samples=100]{sqrt(.016-.016/.0001*(x-.4)^2)};
           \addplot [penColor2,thick,smooth,domain=.39:.4,samples=100]{-sqrt(.016-.016/.0001*(x-.4)^2)};
            \addplot [penColor2,thick,smooth,domain=.59:.61,samples=100]{sqrt(.194-.194/.0001*(x-.6)^2)};
           \addplot [penColor2,thick,smooth,domain=.59:.61,samples=100]{-sqrt(.194-.194/.0001*(x-.6)^2)};
           
           
        \end{axis}
      \end{tikzpicture}
    \end{image}    
    
The function and its limits are in terms of $x$.  Also, note that if we solve for $x$, we have $x= \sqrt[3]{\frac{y}{2}}$, which is not differentiable at $y=0$.  Thus, we should:

\begin{multipleChoice}
\choice[correct]{integrate with respect to $x$.}
\choice{integrate with respect to $y$.}
\end{multipleChoice}



\begin{exercise}
Since we have chosen this, we use the result:

\[ SA = \int_{x=a}^{x=b} 2 \pi r \d s\]

and we must express $r$ in terms of $x$ and $\d s$ in terms of $x$ and $\d x$.  

\begin{exercise}
For $\d s$, we use $\d s \sqrt{1+\left(\frac{\d y}{\d x}\right)^2} \d x$.

Here, $\frac{\d y}{\d x}=\answer{6x^2}$ so $\d s= \answer{\sqrt{1+ 36 x^4}} \d x$. 
\end{exercise}

\begin{exercise}
Note that is $r$ is the distance from the axis to the curve. This is a:

\begin{multipleChoice}
\choice[correct]{vertical distance}
\choice{horizontal distance}
\end{multipleChoice} 
Thus $r=y_{top}-y_{bot}$.  

We note that the slice is at a location $(x,y)$, which happens to be on the curve.  This allows us to use the curve to express $x$ in terms of $y$ or $y$ in terms of $x$ if necessary.  

The quantity $2x^3$ expresses:
\begin{multipleChoice}
\choice[correct]{The $y$-value on the curve if the $x$-value of the slice is specified.}
\choice{The $x$-value on the curve.}
\end{multipleChoice} 

The quantity $\sqrt[3]{\frac{y}{2}}$ expresses:
\begin{multipleChoice}
\choice[correct]{The $x$-value on the curve if the $y$-value of the slice is specified.}
\choice{The $y$-value on the curve.}
\end{multipleChoice} 

Since we have to express $r$ in terms of $x$, and we note that $y_{top}$ is on the curve, we must express it in terms of $x$.  Hence, $y_{top} = \answer{2x^3}$.

Since $y_{bot}$ lies on the $x$-axis, $y_{bot} = \answer{0}$, and $r= \answer{2x^3}$.

\end{exercise}

\begin{exercise}
Now we see that an integral that gives the surface area is: 
\[
SA= \int_{x=\answer{0}}^{x=\answer{1}} 2 \pi r \d s = \int_0^1 \answer{4 \pi x^3} \answer{\sqrt{1+36x^4}} \d x 
\]

\begin{exercise}
Evaluating this integral, we find that the surface area of the surface of revolution is $\answer{\frac \pi {54} ((37)^{3/2} -1)}$ square units. 

\begin{hint} You can use a substitution to evaluate the integral with $u= \answer{1+36x^4}$.  Then, $du= \answer{144x^3} dx$. 

For the limits of integration,  when $x=0$, $u= \answer{1}$ and when $x=1$, $u=\answer{37}$.

Thus:

\begin{align*}
SA= \int_{x=0}^{x=1} 4 \pi x^3 \sqrt{1+36x^4} \d x &=  \int_{u=\answer{1}}^{u=\answer{37}} \answer{\frac{\pi}{36}} \answer{u^{1/2}} \d u \\
&=  \eval{\answer{\frac{\pi}{54} u^{3/2 } }}_{u=1}^{u=37}
&= \answer{\frac \pi {54} ((37)^{3/2} -1)}
\end{align*}
\end{hint}


\end{exercise}
\end{exercise}
\end{exercise}
\end{exercise}





\end{document}
