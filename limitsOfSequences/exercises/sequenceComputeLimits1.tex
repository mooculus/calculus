\documentclass{ximera}

\newcommand{\RR}{\mathbb R}
\renewcommand{\d}{\,d}
\newcommand{\dd}[2][]{\frac{d #1}{d #2}}
\renewcommand{\l}{\ell}
\newcommand{\ddx}{\frac{d}{dx}}
\newcommand{\dfn}{\textbf}
\newcommand{\eval}[1]{\bigg[ #1 \bigg]}


\author{Jim Talamo}
\license{Creative Commons 3.0 By-bC}


\outcome{}


\begin{document}
\begin{exercise}


Consider$\{a_n \}_{n=1}$ where $a_n = \left(1-\frac{2}{n}\right)^{2n}$.  Then:
\[
\lim_{n \to \infty} a_n = \answer{e^{-4}}
\]

\begin{hint}
This limit has indeterminate form $\answer{1}^{\answer{\infty}}$
(Use ``$\infty$" or ``$-\infty"$ where appropriate)
Set $L = \lim_{n \to \infty} a_n$.  Taking the logarithm of both sides gives:

\[
\ln L = \ln  \left( \lim_{n \to \infty} \left(1-\frac{2}{n}\right)^{2n} \right) = \lim_{n \to \infty} \ln \left(1-\frac{2}{n}\right)^{2n} 
\]

Using the properties of logarithms:
\[
\ln L = \lim_{n \to \infty} \answer{2n} \ln \left(1-\frac{2}{n}\right) 
\]
\begin{question}
This limit has indeterminate form $\answer{\infty} \cdot \answer{0}$. To convert this, we can move one of the terms into the denominator of the denominator:

\[
\ln L = \lim_{n \to \infty} \frac{2 \ln \left(1-\frac{2}{n}\right)}{\answer{\frac{1}{n}}} 
\]

\begin{question}
Now, the limit has the indeterminate form: $\frac{\answer{0}}{\answer{0}}$ and we can use L'Hopital's Rule to write:
\[
\ln L = \lim_{n \to \infty} \frac{2 \ln \left(1-\frac{2}{n}\right)}{\answer{\frac{1}{n}}} =  \lim_{n \to \infty} \frac{\answer{\frac{2}{1-\frac{2}{n}} \left( \frac{2}{n^2}\right)}}{\answer{-\frac{1}{n^2}}} 
\]

Evaluating this gives: $\ln L = -4$.  Hence, $L = \answer{e^{-4}}$.



\end{question}
\end{question}
\end{hint}
\end{exercise}
\end{document}
