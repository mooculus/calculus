\documentclass{ximera}

\newcommand{\RR}{\mathbb R}
\renewcommand{\d}{\,d}
\newcommand{\dd}[2][]{\frac{d #1}{d #2}}
\renewcommand{\l}{\ell}
\newcommand{\ddx}{\frac{d}{dx}}
\newcommand{\dfn}{\textbf}
\newcommand{\eval}[1]{\bigg[ #1 \bigg]}


\title[Dig-In:]{Approximating area with rectangles}

\begin{document}
\begin{abstract}
  We introduce the basic idea of using rectangles to approximate the
  area under a curve.
\end{abstract}
\maketitle

\section{Limits and areas}





\section{But which set of rectangles?}

If we are going to try an actually use many small rectangles to
compute the area under a curve, we should decide on exactly
\textit{which} rectangles we want to use. Here are three options that
we consider:

\paragraph{Rectangles defined by a left endpoint}


\paragraph{Rectangles defined by a right endpoint}


\paragraph{Rectangles defined by a midpoint}


\section{Computing these approximations}


\paragraph{Rectangles defined by a left endpoint}


\paragraph{Rectangles defined by a right endpoint}


\paragraph{Rectangles defined by a midpoint}



\end{document}
