\documentclass{ximera}

\newcommand{\RR}{\mathbb R}
\renewcommand{\d}{\,d}
\newcommand{\dd}[2][]{\frac{d #1}{d #2}}
\renewcommand{\l}{\ell}
\newcommand{\ddx}{\frac{d}{dx}}
\newcommand{\dfn}{\textbf}
\newcommand{\eval}[1]{\bigg[ #1 \bigg]}


\author{Jim Talamo}
\license{Creative Commons 3.0 By-bC}


\outcome{}


\begin{document}
\begin{exercise}
In many exercises, it is only necessary to work with a partial sum for a Taylor series.  There are two common ways to specify how many terms to use.

\begin{itemize}
\item Specifying a Taylor polynomial of a given degree
\item Specifying how many terms of the series to use
\end{itemize}

To make this a bit more explicit, consider the function $f(x)=\sin(x)$.  Which of the following is the series in summation notation?

\begin{multipleChoice}
\choice{$\sum_{k=0}^{\infty} \frac{(-1)^k}{(2k)!}x^{2k}$}
\choice{$\sum_{k=0}^{\infty} \frac{(-1)^k}{k!}x^{k}$}
\choice[correct]{$\sum_{k=0}^{\infty} \frac{(-1)^k}{(2k+1)!}x^{2k+1}$}
\choice{$\sum_{k=0}^{\infty} \frac{(-1)^{2k+1}}{(2k+1)!}x^{2k+1}$}
\end{multipleChoice}

%%%%%%%%%%%%%
\begin{exercise}
\begin{exercise}
Suppose we want to find the fifth degree Taylor polynomial for $f(x)=\sin(x)$ centered at $x=0$.  We should:

\begin{multipleChoice}
\choice{Substitute $k=0,1,2,3,4,$ and $5$ into the summand above and write out the sum of those terms}
\choice{Write out the first five powers of $x$ in the series above that have a nonzero coefficient}
\choice[correct]{Write out terms in the series above until we exhibit the $x^5$ term}
\end{multipleChoice}

Recall that the $n$-th degree Taylor polynomial is a polynomial of degree $n$ whose coefficients are chosen so the values of both the original function and the approximation as well as their first $n$ derivatives agree!  Despite what the values of the coefficients are, this problem specifically requires that you write out all of the terms up to and including the $x^5$ term.

The fifth degree Taylor Polynomial for $f(x) = \sin(x)$ centered at $x=0$ is:

\[
p_5(x) = \answer{x-\frac{1}{3!}x^3+\frac{1}{5!}x^5}
\]

\end{exercise}
%%%%%%%%%%%%%

%%%%%%%%%%%%%
\begin{exercise}
Suppose we want to find the sum of the first five nonzero terms in the Taylor series for $f(x)=\sin(x)$ centered at $x=0$.  We should:

\begin{multipleChoice}
\choice{Substitute $k=0,1,2,3,4,$ and $5$ into the summand above and write out the sum of those terms}
\choice[correct]{Write out the first five powers of $x$ in the series above that have a nonzero coefficient}
\choice{Write out terms in the series above until we exhibit the $x^5$ term}
\end{multipleChoice}

The sum of the first five nonzero terms in the Taylor series for $f(x)=\sin(x)$ centered at $x=0$ is:

\[
\sin(x) = \answer{x-\frac{1}{3!}x^3+\frac{1}{5!}x^5-\frac{1}{7!}x^7+\frac{1}{9!}x^9}+ \ldots
\]

\end{exercise}
%%%%%%%%%%%%%
In some applications, we will want to study a specific Taylor polynomial.  For instance, if we want to approximate a function to within a prescribed degree of accuracy, we can use Taylor's Remainder Theorem to find a sufficient Taylor polynomial.  

Other times, we will only need the first several terms in order to draw the desired conclusion.  Calculating limits using Taylor series is a good example of this.
  
\end{exercise}

\end{exercise}
\end{document}
