\documentclass{ximera}

\newcommand{\RR}{\mathbb R}
\renewcommand{\d}{\,d}
\newcommand{\dd}[2][]{\frac{d #1}{d #2}}
\renewcommand{\l}{\ell}
\newcommand{\ddx}{\frac{d}{dx}}
\newcommand{\dfn}{\textbf}
\newcommand{\eval}[1]{\bigg[ #1 \bigg]}

\author{Jim Talamo}
\title[Refresh:]{ L'H\^{o}pital's Rule}

\begin{document}
\begin{abstract}
Review  L'H\^{o}pital's Rule.
\end{abstract}
\maketitle

\begin{problem}
Sometimes, some manipulation is required in order to bring a limit to a form where L'H\^{o}pital's Rule can be used.  

\begin{example}
Compute $\lim_{x \to 0^+} x^x$.

\begin{explanation}
Direct evaluation gives the form $0^0$, which is an indeterminate form; the $0$ in the base wants the expression to tend to $0$, but the $0$ in the exponent wants the expression to tend to $1$.  We cannot use L'H\^{o}pital's Rule yet, but we can note that by setting $L = \lim_{x \to 0^+} x^x$, we have

\begin{align*}
\ln(L)  &= \ln\left[ \lim_{x \to 0} x^x\right] \\
\ln(L)  &=   \lim_{x \to 0} \left[ \ln(x^x) \right] \\
\ln(L)  &=   \lim_{x \to 0} x \ln(x)  \\
\end{align*} 

We now have the form $0 \times -\infty$, so we are a bit closer.  We must either move the $x$ or $\ln(x)$ to the denominator.  It seems like moving the $x$ to the denominator will be easier to handle, so we try it.


\begin{align*}
\ln(L)  =   \lim_{x \to 0} x \ln(x)  = \lim_{x \to 0} \frac{ \ln(x)}{1/x}  \\
\end{align*} 

We now have the form $-\frac{\infty}{\infty}$, so we can apply L'H\^{o}pital's Rule.

\begin{align*}
\ln(L)  =  \lim_{x \to 0} \frac{1/x}{-1/x^2}  \\
\end{align*} 

After some algebra, we have $\ln(L) = 0$. and we conclude that $L = \lim_{x \to 0^+} x^x = 1$.
\end{explanation}
\end{example}

This example is one in which the exponential indeterminate form $0^0$ arises.  The other exponential indeterminate forms that arise are $1^{\infty}$ and $\infty^0$.  When this happens, the procedure outlined above is a good one; taking logarithms of both sides (if necessary) will allow the exponent to be converted into a product.  From there, further manipulation can be done to write the limit in a form where  L'H\^{o}pital's Rule can be applied.

Try your hand at these.  If a limit does not exist, use $\infty$ or $-\infty$ as appropriate or write ``DNE'' otherwise.

\begin{exercise}
\[
\lim_{x \to \infty} \ln\left(x^{1/x}\right) = \answer{0}  
\]
\begin{hint}
Using the rule $\ln(a^b) = b \ln(a)$ should produce a quick solution.
\end{hint}
\end{exercise}

\begin{exercise}
\[
\lim_{x \to \infty} \left(1+\frac{a}{x}\right)^x = \answer{e^a}  
\]
(leave your answer in terms of $a$ if appropriate)
\end{exercise}

\begin{exercise}
\[
\lim_{x \to \infty} \left(1+\frac{a}{x^2}\right)^x = \answer{1}  
\]
(leave your answer in terms of $a$ if appropriate)
\end{exercise}

\begin{exercise}
\[
\lim_{x \to \infty} \left(1+\frac{a}{\ln(x)}\right)^x = \answer{\infty}  
\]
(leave your answer in terms of $a$ if appropriate)
\end{exercise}

\end{problem}
\end{document}
