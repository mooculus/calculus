\documentclass{ximera}

\newcommand{\RR}{\mathbb R}
\renewcommand{\d}{\,d}
\newcommand{\dd}[2][]{\frac{d #1}{d #2}}
\renewcommand{\l}{\ell}
\newcommand{\ddx}{\frac{d}{dx}}
\newcommand{\dfn}{\textbf}
\newcommand{\eval}[1]{\bigg[ #1 \bigg]}


\outcome{}


\title[Break-Ground:]{Slope of a curve}

\begin{document}
\begin{abstract}
Two young mathematicians discuss the novel idea of the ``slope of a curve.''
\end{abstract}
\maketitle


Check out this dialogue between two calculus students (based on a true
story):

\begin{dialogue}
\item[Devyn] Riley, do you remember `slope?'
\item[Riley] Most definitely. ``Rise over run.''
\item[Devyn] You know it.
\item[Riley] ``Change in $y$ over change in $x$.''
\item[Devny] That's right.  
\item[Riley] ``m.''
\item[Devny] Enough! My important question, is could we define
  ``slope'' for a curve?
\item[Riley] Well maybe if we ``zoom in'' on a curve, it would look
  like a line, and then we could call it ``the slope at that point.''
\item[Devyn] Ah! And this ``zoom in'' idea sounds like a limit!
\item[Riley] This is so awesome. We just made math!
\end{dialogue}

The concept introduced above, of the ``slope of a curve at a point,''
is one it in fact is one of the central concepts of calculus. It will,
in course, be completely explaind.  Let's see if we can clarify Devyn
and Riley's ideas.


\begin{problem}
To find the ``slope of a curve at a point,'' Devyn and Riley spoke of
``zooming in'' on a curve until it looks like a line.
  \begin{freeResponse}
  \end{freeResponse}
\end{problem}


%% \begin{xarmaBoost}
%%   Write down at least \textbf{five} questions for this lecture. After
%%   you have your questions, label them as ``Level 1,'' ``Level 2,'' or
%%   ``Level 3'' where:
%% \begin{description}
%% \item[Level 1] Means you know the answer, or know exactly how to do
%%   this problem.
%% \item[Level 2] Means you think you know how to do the problem.
%% \item[Level 3] Means you have no idea how to do the problem.
%% \end{description}
%% \begin{freeResponse}
%% \end{freeResponse}
%% \end{xarmaBoost}



\end{document}
