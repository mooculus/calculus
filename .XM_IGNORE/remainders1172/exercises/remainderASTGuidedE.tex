\documentclass{ximera}

%\usepackage{todonotes}

\newcommand{\todo}{}

\usepackage{esint} % for \oiint
\ifxake%%https://math.meta.stackexchange.com/questions/9973/how-do-you-render-a-closed-surface-double-integral
\renewcommand{\oiint}{{\large\bigcirc}\kern-1.56em\iint}
\fi

\def\xmNotExpandableAsAccordion{true}

\graphicspath{
  {./}
  {ximeraTutorial/}
  {basicPhilosophy/}
  {functionsOfSeveralVariables/}
  {normalVectors/}
  {lagrangeMultipliers/}
  {vectorFields/}
  {greensTheorem/}
  {shapeOfThingsToCome/}
  {dotProducts/}
  {../productAndQuotientRules/exercises/}
  {../normalVectors/exercisesParametricPlots/}
  {../continuityOfFunctionsOfSeveralVariables/exercises/}
  {../partialDerivatives/exercises/}
  {../chainRuleForFunctionsOfSeveralVariables/exercises/}
  {../commonCoordinates/exercisesCylindricalCoordinates/}
  {../commonCoordinates/exercisesSphericalCoordinates/}
  {../greensTheorem/exercisesCurlAndLineIntegrals/}
  {../greensTheorem/exercisesDivergenceAndLineIntegrals/}
  {../shapeOfThingsToCome/exercisesDivergenceTheorem/}
  {../greensTheorem/}
  {../shapeOfThingsToCome/}
}

\newcommand{\mooculus}{\textsf{\textbf{MOOC}\textnormal{\textsf{ULUS}}}}

\usepackage{tkz-euclide}\usepackage{tikz}
\usepackage{tikz-cd}
\usetikzlibrary{arrows}
\tikzset{>=stealth,commutative diagrams/.cd,
  arrow style=tikz,diagrams={>=stealth}} %% cool arrow head
\tikzset{shorten <>/.style={ shorten >=#1, shorten <=#1 } } %% allows shorter vectors

\usetikzlibrary{backgrounds} %% for boxes around graphs
\usetikzlibrary{shapes,positioning}  %% Clouds and stars
\usetikzlibrary{matrix} %% for matrix
\usepgfplotslibrary{polar} %% for polar plots
\usepgfplotslibrary{fillbetween} %% to shade area between curves in TikZ
%\usetkzobj{all}
%\usepackage[makeroom]{cancel} %% for strike outs
%\usepackage{mathtools} %% for pretty underbrace % Breaks Ximera
%\usepackage{multicol}
\usepackage{pgffor} %% required for integral for loops



%% http://tex.stackexchange.com/questions/66490/drawing-a-tikz-arc-specifying-the-center
%% Draws beach ball
\tikzset{pics/carc/.style args={#1:#2:#3}{code={\draw[pic actions] (#1:#3) arc(#1:#2:#3);}}}



\usepackage{array}
\setlength{\extrarowheight}{+.1cm}   
\newdimen\digitwidth
\settowidth\digitwidth{9}
\def\divrule#1#2{
\noalign{\moveright#1\digitwidth
\vbox{\hrule width#2\digitwidth}}}





\newcommand{\RR}{\mathbb R}
\newcommand{\R}{\mathbb R}
\newcommand{\N}{\mathbb N}
\newcommand{\Z}{\mathbb Z}

\newcommand{\sagemath}{\textsf{SageMath}}


%\renewcommand{\d}{\,d\!}
\renewcommand{\d}{\mathop{}\!d}
\newcommand{\dd}[2][]{\frac{\d #1}{\d #2}}
\newcommand{\pp}[2][]{\frac{\partial #1}{\partial #2}}
\renewcommand{\l}{\ell}
\newcommand{\ddx}{\frac{d}{\d x}}

\newcommand{\zeroOverZero}{\ensuremath{\boldsymbol{\tfrac{0}{0}}}}
\newcommand{\inftyOverInfty}{\ensuremath{\boldsymbol{\tfrac{\infty}{\infty}}}}
\newcommand{\zeroOverInfty}{\ensuremath{\boldsymbol{\tfrac{0}{\infty}}}}
\newcommand{\zeroTimesInfty}{\ensuremath{\small\boldsymbol{0\cdot \infty}}}
\newcommand{\inftyMinusInfty}{\ensuremath{\small\boldsymbol{\infty - \infty}}}
\newcommand{\oneToInfty}{\ensuremath{\boldsymbol{1^\infty}}}
\newcommand{\zeroToZero}{\ensuremath{\boldsymbol{0^0}}}
\newcommand{\inftyToZero}{\ensuremath{\boldsymbol{\infty^0}}}



\newcommand{\numOverZero}{\ensuremath{\boldsymbol{\tfrac{\#}{0}}}}
\newcommand{\dfn}{\textbf}
%\newcommand{\unit}{\,\mathrm}
\newcommand{\unit}{\mathop{}\!\mathrm}
\newcommand{\eval}[1]{\bigg[ #1 \bigg]}
\newcommand{\seq}[1]{\left( #1 \right)}
\renewcommand{\epsilon}{\varepsilon}
\renewcommand{\phi}{\varphi}


\renewcommand{\iff}{\Leftrightarrow}

\DeclareMathOperator{\arccot}{arccot}
\DeclareMathOperator{\arcsec}{arcsec}
\DeclareMathOperator{\arccsc}{arccsc}
\DeclareMathOperator{\si}{Si}
\DeclareMathOperator{\scal}{scal}
\DeclareMathOperator{\sign}{sign}


%% \newcommand{\tightoverset}[2]{% for arrow vec
%%   \mathop{#2}\limits^{\vbox to -.5ex{\kern-0.75ex\hbox{$#1$}\vss}}}
\newcommand{\arrowvec}[1]{{\overset{\rightharpoonup}{#1}}}
%\renewcommand{\vec}[1]{\arrowvec{\mathbf{#1}}}
\renewcommand{\vec}[1]{{\overset{\boldsymbol{\rightharpoonup}}{\mathbf{#1}}}}
\DeclareMathOperator{\proj}{\vec{proj}}
\newcommand{\veci}{{\boldsymbol{\hat{\imath}}}}
\newcommand{\vecj}{{\boldsymbol{\hat{\jmath}}}}
\newcommand{\veck}{{\boldsymbol{\hat{k}}}}
\newcommand{\vecl}{\vec{\boldsymbol{\l}}}
\newcommand{\uvec}[1]{\mathbf{\hat{#1}}}
\newcommand{\utan}{\mathbf{\hat{t}}}
\newcommand{\unormal}{\mathbf{\hat{n}}}
\newcommand{\ubinormal}{\mathbf{\hat{b}}}

\newcommand{\dotp}{\bullet}
\newcommand{\cross}{\boldsymbol\times}
\newcommand{\grad}{\boldsymbol\nabla}
\newcommand{\divergence}{\grad\dotp}
\newcommand{\curl}{\grad\cross}
%\DeclareMathOperator{\divergence}{divergence}
%\DeclareMathOperator{\curl}[1]{\grad\cross #1}
\newcommand{\lto}{\mathop{\longrightarrow\,}\limits}

\renewcommand{\bar}{\overline}

\colorlet{textColor}{black} 
\colorlet{background}{white}
\colorlet{penColor}{blue!50!black} % Color of a curve in a plot
\colorlet{penColor2}{red!50!black}% Color of a curve in a plot
\colorlet{penColor3}{red!50!blue} % Color of a curve in a plot
\colorlet{penColor4}{green!50!black} % Color of a curve in a plot
\colorlet{penColor5}{orange!80!black} % Color of a curve in a plot
\colorlet{penColor6}{yellow!70!black} % Color of a curve in a plot
\colorlet{fill1}{penColor!20} % Color of fill in a plot
\colorlet{fill2}{penColor2!20} % Color of fill in a plot
\colorlet{fillp}{fill1} % Color of positive area
\colorlet{filln}{penColor2!20} % Color of negative area
\colorlet{fill3}{penColor3!20} % Fill
\colorlet{fill4}{penColor4!20} % Fill
\colorlet{fill5}{penColor5!20} % Fill
\colorlet{gridColor}{gray!50} % Color of grid in a plot

\newcommand{\surfaceColor}{violet}
\newcommand{\surfaceColorTwo}{redyellow}
\newcommand{\sliceColor}{greenyellow}




\pgfmathdeclarefunction{gauss}{2}{% gives gaussian
  \pgfmathparse{1/(#2*sqrt(2*pi))*exp(-((x-#1)^2)/(2*#2^2))}%
}


%%%%%%%%%%%%%
%% Vectors
%%%%%%%%%%%%%

%% Simple horiz vectors
\renewcommand{\vector}[1]{\left\langle #1\right\rangle}


%% %% Complex Horiz Vectors with angle brackets
%% \makeatletter
%% \renewcommand{\vector}[2][ , ]{\left\langle%
%%   \def\nextitem{\def\nextitem{#1}}%
%%   \@for \el:=#2\do{\nextitem\el}\right\rangle%
%% }
%% \makeatother

%% %% Vertical Vectors
%% \def\vector#1{\begin{bmatrix}\vecListA#1,,\end{bmatrix}}
%% \def\vecListA#1,{\if,#1,\else #1\cr \expandafter \vecListA \fi}

%%%%%%%%%%%%%
%% End of vectors
%%%%%%%%%%%%%

%\newcommand{\fullwidth}{}
%\newcommand{\normalwidth}{}



%% makes a snazzy t-chart for evaluating functions
%\newenvironment{tchart}{\rowcolors{2}{}{background!90!textColor}\array}{\endarray}

%%This is to help with formatting on future title pages.
\newenvironment{sectionOutcomes}{}{} 



%% Flowchart stuff
%\tikzstyle{startstop} = [rectangle, rounded corners, minimum width=3cm, minimum height=1cm,text centered, draw=black]
%\tikzstyle{question} = [rectangle, minimum width=3cm, minimum height=1cm, text centered, draw=black]
%\tikzstyle{decision} = [trapezium, trapezium left angle=70, trapezium right angle=110, minimum width=3cm, minimum height=1cm, text centered, draw=black]
%\tikzstyle{question} = [rectangle, rounded corners, minimum width=3cm, minimum height=1cm,text centered, draw=black]
%\tikzstyle{process} = [rectangle, minimum width=3cm, minimum height=1cm, text centered, draw=black]
%\tikzstyle{decision} = [trapezium, trapezium left angle=70, trapezium right angle=110, minimum width=3cm, minimum height=1cm, text centered, draw=black]


\author{Jim Talamo}
\license{Creative Commons 3.0 By-NC}


\outcome{Understand the alternating series test estimates.}

\begin{document}

\begin{exercise}
Consider the series $\sum_{k=0}^{\infty} \frac{(-1)^k}{k!}$.  

Does the alternating series test apply?

\begin{multipleChoice}
\choice[correct]{Yes.}
\choice{No.}
\end{multipleChoice}

What does the alternating series test tell us?
\begin{multipleChoice}
\choice{The series converges to $0$.}
\choice[correct]{The series converges, but we do not know its value yet.}
\choice{The series diverges.}
\end{multipleChoice}

\begin{feedback}
Note that the series is of the form $\sum_{k=0}^{\infty} (-1)^k a_k$, where $a_k = \frac{1}{k!}>0$.  Since $\lim_{n \to \infty} \frac{1}{n!} =0$ and $a_{n+1} \leq a_n$ for all $n$, we conclude that the series converges by the alternating series test.  
\end{feedback}

\begin{exercise}
Recall that once we have determined that a series converges, we are often interested in the following two questions.

\begin{itemize}
\item[1.] How bad is the error made when we approximate a convergent infinite series by its first several terms?
%In other words, if we specify $N$, how close is $\sum_{k=n_0}^{N} a_k$ to the exact value of $\sum_{k=n_0}^{\infty} a_k$?
\item[2.] How many terms should we specify if we want to know the value of a convergent series to obtain a desired precision?
%Said another way, given an acceptable value for the error, what value should we pick for $N$ so $\sum_{k=n_0}^{N} a_k$ approximates $\sum_{k=n_0}^{\infty} a_k$ that accurately?
\end{itemize}

We now want to approximate the value of $\sum_{k=0}^{\infty} \frac{(-1)^k}{k!}$.  Since we do not have a way to find an explicit formula for $s_n=\sum_{k=1}^n \frac{(-1)^k}{k!}$, we need to turn to the remainder estimate that comes with alternating series.

\begin{theorem}[Alternating Series Remainder Estimates]
If $\{a_n\}_{n=n_0}$ be a sequence whose terms are positive and decreasing and
$\lim_{n\to\infty} a_n=0$. Then,  
\[
\big| r_n \big| \leq a_{n+1} \qquad \textrm{ for all } n \geq n_0,
\]
where $r_n = \sum_{k=n+1}^{\infty} a_k$.
\end{theorem}

Unlike with the integral test, we will use $s_n$ as our approximation to $\sum_{k=0}^{\infty} \frac{(-1)^k}{k!}$ since the sequence $\{s_n\}_{n=1}$ is \wordChoice{\choice{eventually}\choice[correct]{never eventually}} monotonic.  This is also why we consider the \emph{magnitude} of the error; whether $s_n$ is an overestimate or underestimate will depend on the choice of $n$. 

%%%%%%%%%%%%%%%%%%%%%%%%%%%%%%%%%%%%%%%%%%%%%
\begin{exercise}
Calculate $\sum_{k=0}^{4} \frac{(-1)^k}{k!}$ to four decimal places.  

To four decimal places, we find that $\sum_{k=0}^{4} \frac{(-1)^k}{k!} = \answer[tolerance=.0001]{.3750}$.  

We use the alternating series results to find the maximum possible error made if $\sum_{k=0}^{4} \frac{(-1)^k}{k!}$ is used to approximate $\sum_{k=0}^{\infty} \frac{(-1)^k}{k!}$. 

First, note that for our series, $a_n =$ \wordChoice{\choice{$\frac{(-1)^n}{n!}$}\choice[correct]{$\frac{1}{n!}$}}.  Since we are using $\sum_{k=0}^{4} \frac{(-1)^k}{k!}$ for our approximation, we should choose $n=\answer{4}$.  What do the remainder results tell us that the magnitude of the error?

\begin{multipleChoice}
\choice{$\big|r_4 \big| \leq \frac{1}{3!}$}
\choice{$\big|r_4 \big| \leq \frac{1}{4!}$}
\choice[correct]{$\big|r_4 \big| \leq \frac{1}{5!}$}
\end{multipleChoice}

To four decimal places, what is the maximum possible error made if we use $\sum_{k=0}^{4} \frac{(-1)^k}{k!}$ as our approximation?

The maximum possible error made is $\answer[tolerance=.0001]{.0083}$.

\begin{feedback}
We thus conclude that $\sum_{k=0}^{\infty} \frac{(-1)^k}{k!} \approx .3750$ and that this approximation is accurate to within $.0083$.
\end{feedback}

%%%%%%%%%%%%%%%%%%%%%%%%%%%%%%%%%%%%%%%%%%%%%
\begin{exercise}
Suppose that we want to approximate $\sum_{k=0}^{\infty} \frac{(-1)^k}{k!}$ to within $.00001$ of its actual value.  We will do so by finding a value of $N$ for which we are guaranteed that $\big|r_N\big| \leq .00001$, then use $s_N$ to provide the approximation.

We do not have a formula for $r_n$, so we must turn to the remainder results.  Note that we have $\big|r_n\big| \leq a_{n+1}$, so if we ensure that $a_{n+1} \leq .00001$, we know that the magnitude of the error will be no more than $.00001$.

\[
\big|r_n\big| \leq \frac{1}{(n+1)!} \leq .00001
\]

There isn't a nice general method for solving equations with factorials, but we can check if the above inequality holds for various $n$.

\begin{itemize}
\item If we choose $n=5$, then to seven decimal places $\frac{1}{(n+1)!} = \frac{1}{6!} \approx \answer{.0013889}$, so $n=5$ is \wordChoice{\choice[correct]{too small}\choice{an acceptable value for $n$}} .
\item If we choose $n=7$, then to seven decimal places $\frac{1}{(n+1)!} = \frac{1}{8!} \approx \answer{.0000248}$, so $n=7$ is \wordChoice{\choice[correct]{too small}\choice{an acceptable value for $n$}} .
\item If we choose $n=8$, then to seven decimal places $\frac{1}{(n+1)!} = \frac{1}{9!} \approx .000000276$, so $n=8$ will work.
\end{itemize}

We thus compute $\sum_{k=1}^8 \frac{(-1)^k}{k!} \approx \answer[tolerance=.000003]{.367882}$ to six decimal places and note that this will approximate $\sum_{k=1}^\infty \frac{(-1)^k}{k!}$ to within $.00001$ of its true value.
\end{exercise}

\end{exercise}
%%%%%%%%%%%%%%%%%%%%%%%%%%%%%%%%%%%%%%%%%%%%%

\end{exercise}
\end{exercise}
\end{document}
