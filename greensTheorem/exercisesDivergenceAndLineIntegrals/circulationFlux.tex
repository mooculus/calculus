\documentclass{ximera}

\newcommand{\RR}{\mathbb R}
\renewcommand{\d}{\,d}
\newcommand{\dd}[2][]{\frac{d #1}{d #2}}
\renewcommand{\l}{\ell}
\newcommand{\ddx}{\frac{d}{dx}}
\newcommand{\dfn}{\textbf}
\newcommand{\eval}[1]{\bigg[ #1 \bigg]}

\usepackage{currfile}
\makeatletter
\ifxake
% The code below the \else is executed in a sagecell on the Ximera
% server, so \makerandom doesn't have to do anything when run under
% xake.
\newcommand{\makerandom}{}
\else
\newcommand {\ST@wsf }[1]{\immediate \write \ST@sf {##1}}
\newcommand{\makerandom}{%
  \ST@wsf{jobname="\currfilebase"}%
  \ST@wsf{import hashlib}%
  \ST@wsf{set_random_seed(int(hashlib.sha256(jobname.encode('utf-8')).hexdigest(), 16))}%
}
\fi
\makeatother


\author{Jim Fowler \and Bart Snapp}

\outcome{Compute circulation: the flow of a vector field along a curve.}
\outcome{Compute flux: the flow of a vector field across a curve.}

\begin{document}
\makerandom

\begin{sagesilent}
  variables = [var('x'), var('y')]
  var('t')
  m = sum([randint(-1,1) * prod([v**randint(0,2) for v in variables]) for _ in range(2)])
  n = sum([randint(-1,1) * prod([v**randint(0,2) for v in variables]) for _ in range(2)])
  f=[m,n]
\end{sagesilent}

\begin{exercise}

  Let $\vec{F}(x,y) = \vector{\sage{m},\sage{n}}$ and $C$ be the unit
  circle centered at the origin drawn in a counterclockwise fashion as
  $t$ runs from $0$ to $2\pi$. Set up an integral that will compute
  the flow along a curve $C$ (circulation) and an integral that will
  compute the flow across a curve $C$ (flux).
  \begin{prompt}
    \[
    \oint_C \vec{F}\dotp\d\vec{p}
    = \int_{\answer{0}}^{\answer{2\pi}}\left( \sage{f[0](x=cos(t),y=sin(t))*(-sin(t)) + f[1](x=cos(t),y=sin(t))*cos(t)} \right)\d t 
    \]
    \[
    \oint_C \vec{F}\dotp\d\vec{n} = \int_{\answer{0}}^{\answer{2\pi}} \left(\sage{f[0](x=cos(t),y=sin(t))*(cos(t)) + f[1](x=cos(t),y=sin(t))*sin(t)} \right)\d t 
    \]
  \end{prompt}
\end{exercise}
\end{document}
