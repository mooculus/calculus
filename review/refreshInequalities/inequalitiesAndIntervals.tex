\documentclass{ximera}

\newcommand{\RR}{\mathbb R}
\renewcommand{\d}{\,d}
\newcommand{\dd}[2][]{\frac{d #1}{d #2}}
\renewcommand{\l}{\ell}
\newcommand{\ddx}{\frac{d}{dx}}
\newcommand{\dfn}{\textbf}
\newcommand{\eval}[1]{\bigg[ #1 \bigg]}


\title[Refresh:]{Inequalities}

\author{Jim Talamo}
\begin{document}
\begin{abstract}
  Remember our facts about inequalities.
\end{abstract}
\maketitle

\begin{exercise}
When finding the radius and open interval of convergence for Taylor series, it is important to be comfortable working with absolute values and how they can be used/interpreted to construct intervals.  This assignment reviews some of the basic properties.
  
One of the most common mistakes that students make is that they treat the absolute value as a linear function.  In practice, this leads to mistakes like the following.
 
\begin{quote}
A student claims that if $|x - 2| < 3$, then $|x| < 5$. Is the student correct?
\end{quote}
 
One way to explore this is to draw the intervals represented by each $|x-2|<3$ and $|x|<5$.  

To start, note that when $x=5$, we have $|x-2|=|5-2|=3$, so $x=5$ is one endpoint of the interval represented by $|x-2|<5$.  

By considering $|x-2|=3$ exactly as written, the other endpoint of the interval is $x= \answer{-1}$. 

Now, select the interval below that correctly shows all points $x$ such that $|x - 2| < 3$.
  \begin{multipleChoice}
    \choice[correct]{\includegraphics[scale = 0.25]{inequalitiesImages/ineqProblem-2-Fig-A.png}}
    \choice{\includegraphics[scale = 0.25]{inequalitiesImages/ineqProblem-2-Fig-B.png}}
    \choice{\includegraphics[scale = 0.25]{inequalitiesImages/ineqProblem-2-Fig-C.png}}
    \choice{\includegraphics[scale = 0.25]{inequalitiesImages/ineqProblem-2-Fig-D.png}}
  \end{multipleChoice}

\begin{exercise}

On the other hand, $|x|<5$ also represents an interval on the real line.  Select the interval below that correctly shows all point $x$ such that $|x| < 5$.
  \begin{multipleChoice}
    \choice{\includegraphics[scale = 0.25]{inequalitiesImages/ineqProblem-3-Fig-A.png}}
    \choice{\includegraphics[scale = 0.25]{inequalitiesImages/ineqProblem-3-Fig-B.png}}
    \choice{\includegraphics[scale = 0.25]{inequalitiesImages/ineqProblem-3-Fig-C.png}}
    \choice[correct]{\includegraphics[scale = 0.25]{inequalitiesImages/ineqProblem-3-Fig-D}}
  \end{multipleChoice}
  
The expression $|x - c| < r$ can be thought of in a more geometric way as well; it represents the collection of all points that are a distance $r$ units from a central point $c$.
  
    \begin{itemize}
  \item The expression $|x-c|$ measures the distance a particular $x$-value is away from the point $x=c$.
  \item The inequality ``$<r$'' shows us that we are looking for only those $x$-values that are at most $r$ units away from $x=c$.
  \end{itemize}
  
  
  For example, if we consider the number line corresponding to the expression $|x - 4| < 2$
\begin{image}
  \includegraphics[scale = 0.5]{inequalitiesImages/ineqProblem-10-Fig}
\end{image}
  \begin{itemize}
  \item The center of this interval is at $x = \answer{4}$.
  \item The endpoints $x = 2$ and $x = \answer{6}$ are precisely a distance of $\answer{2}$ away from the center $x = 4$. 
  \item All points that satisfy the inequality are within a distance of 2 from the center.
\end{itemize}
  
  This observation is particularly helpful when constructing the interval of convergence of a Taylor series.

\end{exercise}
\end{exercise}

\end{document}
