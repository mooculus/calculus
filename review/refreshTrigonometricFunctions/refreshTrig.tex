\documentclass{ximera}

\newcommand{\RR}{\mathbb R}
\renewcommand{\d}{\,d}
\newcommand{\dd}[2][]{\frac{d #1}{d #2}}
\renewcommand{\l}{\ell}
\newcommand{\ddx}{\frac{d}{dx}}
\newcommand{\dfn}{\textbf}
\newcommand{\eval}[1]{\bigg[ #1 \bigg]}


\title[Refresh:]{Trigonometry}

\begin{document}
\begin{abstract}
  Remember our facts about trigonometry.
\end{abstract}
\maketitle

\begin{problem}
  When working problems involving integrals of products of trigonometric functions, it is necessary to use trigonometric identities to manipulate the expressions in the integrand.
  
  While all of the necessary trigonometric identities will be provided on exams and quizzes, you will need to know how to use them correctly to evaluate antiderivatives involving trigonometric functions.  As such, the following problems will give you some practice.
  
  \begin{multipleChoice}
    \choice[correct]{I understand.}
    \choice{I do not understand.}
  \end{multipleChoice}
\end{problem}

\begin{problem}
  Write the following expression in terms of either sine or cosine:
  \[
    \sec(x) \cot(x) = \answer{\frac{1}{\sin(x)}}
  \]
\end{problem}

\begin{problem}
  Use the trigonometric identities to simplify $\frac{\tan(x)}{\sec(x)}$   
  \[
    \frac{\tan(x)}{\sec(x)} = \answer{\sin(x)}
  \]
\end{problem}

\begin{problem}
  Combine and simplify the following trigonometric expression:
  \[
    \frac{1}{\cos^2(x)} + \frac{1}{\sin^2(x)} = \answer{\csc^2(x)\sec^2(x)}
  \]  
\end{problem}


\begin{problem}
  When working problems involving integrals of products of trigonometric functions, it is necessary to use trigonometric identities to manipulate the expressions in the integrand.
  
  While all of the necessary trigonometric identities will be provided on exams and quizzes, you will need to know how to use them correctly to evaluate antiderivatives involving trigonometric functions.  As such, the following problems will give you some practice.
  
  \begin{multipleChoice}
    \choice[correct]{I understand.}
    \choice{I do not understand.}
  \end{multipleChoice}
\end{problem}

\begin{problem}
  When evaluating integrals of trigonometric functions with the same argument (i.e. inner function...the argument of $\sin(3x)$ would be $3x$, it is often necessary to convert powers of one trigonometric function into powers of the other.
    This is most easily done via the Pythagorean identities: $\sin^2(x) + \cos^2(x) = 1$, $\tan^2(x) + 1 = \sec^2(x)$, and $1 + \cot^2(x) = \csc^2(x)$.
    
    Using these formulas, it is \emph{easy} to express even powers of one of the trigonometric functions into powers of the complementary function in the above formulas.
    The following problems give you practice with this.
  \begin{multipleChoice}
    \choice[correct]{I understand.}
    \choice{I do not understand.}
  \end{multipleChoice}
\end{problem}

\begin{problem}
  Using the appropriate Pythagorean trigonometric identity, express $\sin^2(x) \cos^2(x)$ as a sum of powers of $\cos(x)$:
  \[
    \sin^2(x) \cos^2(x) = \answer{\cos^2(x) - \cos^4(x)}
  \]
\end{problem}

\begin{problem}
  Using the appropriate Pythagorean trigonometric identity, express $\sec^4(x) \tan^2(x)$ as:
  \begin{enumerate}
    \item A sum of powers of $\sec(x)$
      \[
        \sec^4(x) \tan^2(x) = \answer{\sec^6(x) - \sec^4(x)}
      \]
      (Simplify your final answer)
    \item A sum of powers of $\tan(x)$:
      \[
        \sec^4(x) \tan^2(x) = \answer{\tan^6(x) + 2\tan^4(x) + \tan^2(x)}
      \]
      (Simplify your final answer)
  \end{enumerate}
\end{problem}

\begin{problem}
  For the following expressions, fill in the half-angle formulas:
  $\sin^2(x) = \answer{\frac{1 - \cos(2x)}{2}}$ and $\cos^2(x) = \answer{\frac{1 + \cos(2x)}{2}}$.  
  
  \begin{hint}
    Look these up.
  \end{hint}
\end{problem}


\begin{problem}
  Sometimes when evaluating integrals of trigonometric functions, it becomes necessary to use the double angle formulas for sines and cosines
  \[
    \sin^2(x) = \frac{1}{2} - \frac{1}{2}\cos(2x)
  \]
  \[
    \cos^2(x) = \frac{1}{2} + \frac{1}{2}\cos(2x)
  \]
  to write even powers of sines and cosines as sums of cosines of linear functions of $x$.
  For instance:
  \begin{align*}
    \cos^4(x) &= (\cos^2(x))^2 \\
    &= \left(\frac{1}{2} + \frac{1}{2}\cos(2x)\right)^2 \\
    &= \frac{1}{4} + \frac{1}{2}\cos(2x) + \frac{1}{4}\cos^2(2x)
  \end{align*}
  
  We can then use the formula on $\cos^2(2x)$ to write $\cos^2(2x) = \frac{1}{2} + \frac{1}{2}\cos(4x)$, and substitute into the above to obtain:
  \begin{align*}
    \cos^4(x) &= (\cos^2(x))^2\\
    &= \left(\frac{1}{2} + \frac{1}{2}\cos(2x)\right)^2 \\
    &= \frac{1}{4} + \frac{1}{2}\cos(2x) + \frac{1}{4}\left( \frac{1}{2} + \frac{1}{2}\cos(4x)\right) \\
    &= \frac{3}{8} + \frac{1}{2}\cos(2x) + \frac{1}{8}\cos(4x)
  \end{align*}
  
  For practice, write the following in terms of sums of cosines of
  linear powers in $x$:
  
  \begin{enumerate}
    \item $\sin^4(x) = \answer{\frac{3}{8} - \frac{1}{2}\cos(2x) + \frac{1}{8}\cos(4x)}$

    \item $\sin^2(x)\cos^2(x) = \answer{\frac{1}{8} - \frac{1}{8}\cos(4x)}$
  \end{enumerate}
\end{problem}


\begin{problem}
  Many integrals involving powers of expressions of the forms: $u(x)^2 + a^2$, $u(x)^2 - a^2$, or $a^2 - u(x)^2$.
  These forms respectively require the use of a trigonometric substitution of the form: $u = a \tan(\theta)$, $u = a \sec(\theta)$, or $u = a \sin(\theta)$.
 
  As with all substitutions, the substitution into the integral is made so we end up with an antiderivative in the new variable that we know how to compute.
  After finding the antiderivative in terms of $\theta$, we must express the trigonometric functions in terms of the original variable $x$ (keep this in mind as you are working out problems in lecture and recitation). 
  
  We will have one of the basic trigonometric functions written in terms of $x$, but a fundamental step in these problems is the use of right triangles to express all of the other trigonometric functions in terms of $x$.
  As such, the following exercises will give you practice doing this.
  
  \begin{multipleChoice}
    \choice[correct]{I understand}
    \choice{I do not understand}
  \end{multipleChoice}  
\end{problem}

\begin{problem}
  Use the relationship between trigonometric functions, and drawing the appropriate right-angle triangles, to find the \emph{exact values} of the remaining trigonometric functions if $\sec(\theta) = \frac{3}{\sqrt{2}}$ and $0 < \theta < \pi/2$.
  
    \begin{align*}
      \sin(\alpha) &= \answer{\frac{\sqrt{7}}{3}}  \\
      \cos(\alpha) &= \answer{\frac{\sqrt{2}}{3}}  \\
      \tan(\alpha) &= \answer{\sqrt{\frac{7}{2}}}  \\
      \cot(\alpha) &= \answer{\sqrt{\frac{2}{7}}}  \\
      \csc(\alpha) &= \answer{\frac{3}{\sqrt{7}}}
    \end{align*}  
\end{problem}

\begin{problem}
   Use the relationship between trigonometric functions, and drawing the appropriate right-angle triangles, to find the \emph{exact values} of the remaining trigonometric functions if $\sin(\theta) = \frac{-\sqrt{5}}{3}$ and $\theta$ is in quadrant IV.
  
    \begin{align*}
      \cos(\alpha) &= \answer{\frac{2}{3}}  \\
      \csc(\alpha) &= \answer{\frac{-3}{\sqrt{5}}}  \\
      \sec(\alpha) &= \answer{\frac{3}{2}}  \\
      \tan(\alpha) &= \answer{\frac{-\sqrt{5}}{2}}  \\
      \cot(\alpha) &= \answer{\frac{2}{-\sqrt{5}}}
    \end{align*}  
\end{problem}

\begin{problem}
  Using the interpretation of the output of an inverse trigonometric function is an angle and drawing the appropriate right-angle triangle, find an equivalent algebraic expression for the following composite function:
  \[
    \sin(\arccos(x)) = \answer{\sqrt{1-x^2}}
  \]
\end{problem}

\begin{problem}
  If $3x = 5 \cos(\theta)$, express $\tan(\theta)$ in terms of $x$:
  
  \[
    \tan(\theta) = \answer{\frac{\sqrt{25- 9x^2}}{3x}}
  \]
\end{problem}
\end{document}
