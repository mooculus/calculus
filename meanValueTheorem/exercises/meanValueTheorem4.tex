\documentclass{ximera}

\newcommand{\RR}{\mathbb R}
\renewcommand{\d}{\,d}
\newcommand{\dd}[2][]{\frac{d #1}{d #2}}
\renewcommand{\l}{\ell}
\newcommand{\ddx}{\frac{d}{dx}}
\newcommand{\dfn}{\textbf}
\newcommand{\eval}[1]{\bigg[ #1 \bigg]}


%\outcome{Understand the statement of the Extreme Value Theorem.}
\outcome{Understand the statement of the Mean Value Theorem.}
%\outcome{Sketch pictures to illustrate why the Mean Value Theorem is true.}
\outcome{Determine whether Rolle's Theorem or the Mean Value Theorem can be applied.}
%\outcome{Find the values guaranteed by Rolle's Theorem or the Mean Value Theorem.}
\outcome{Use the Mean Value Theorem to solve word problems.}
%\outcome{Compare and contrast the Intermediate Value Theorem, the Mean Value Theorem, and Rolle's Theorem.}
%\outcome{Identify calculus ideas which are consequences of the Mean Value Theorem.}
%\outcome{Use the Mean Value Theorem to bound the error in linear approximation.}

\author{Nela Lakos \and Kyle Parsons}

\begin{document}
\begin{exercise}

The population (in millions of cells) of a culture of bacteria is given by the function $P(t)$ where $t$ is measured in weeks after some start date.

Below is a partial table of values of $P(t)$.
\[
\begin{array}{|c|c|c|c|c|}
\hline
t & 0 & 1 & 3 & 4\\\hline
P(t) & 0 & 50 & 75 & 80\\\hline
\end{array} 
\]

The average rate of change of the population during the time interval $[0,4]$ is $\answer{20}$ million cells per week.

Suppose that $P(t)$ is continuous on $[0,\infty)$ and differentiable on $(0,\infty)$. Then we know that $P$ \wordChoice{\choice[correct]{satisfies}\choice{does not satisfy}} the conditions of the Mean Value Theorem so \wordChoice{\choice[correct]{there exists}\choice{there does not necessarily exist}} a $c>0$ such that $f'(c)$ equals the above calculated average rate of change of the population.

\end{exercise}
\end{document}