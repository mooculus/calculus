\documentclass{ximera}

\newcommand{\RR}{\mathbb R}
\renewcommand{\d}{\,d}
\newcommand{\dd}[2][]{\frac{d #1}{d #2}}
\renewcommand{\l}{\ell}
\newcommand{\ddx}{\frac{d}{dx}}
\newcommand{\dfn}{\textbf}
\newcommand{\eval}[1]{\bigg[ #1 \bigg]}


\author{Bart Snapp}

\outcome{Use iterated integrals to compute multiple integrals.}
\outcome{Apply Fubini's Theorem.}

\begin{document}
\begin{exercise}
  Use a triple integral to compute the volume of the tetrahedron found
  in the $1$st octant bounded by $2x+3y+5z=30$.
    
  Set-up the iterated integrals that will compute the volume of this
  tetrahedron when integrating with respect to $x$, $y$, and $z$ in
  the orders listed below:
  \begin{prompt}
    \[
    \int_{\answer{0}}^{\answer{15}} \int_{\answer{0}}^{\answer{10-2x/3}} \int_{\answer{0}}^{\answer{\frac{30-2x-3y}{5}}}\d z \d y \d x
    \]
    \[
    \int_{\answer{0}}^{\answer{10}} \int_{\answer{0}}^{\answer{15-3y/2}} \int_{\answer{0}}^{\answer{\frac{30-2x-3y}{5}}}\d z \d x \d y
    \]
    \[
    \int_{\answer{0}}^{\answer{15}} \int_{\answer{0}}^{\answer{6-2x/5}}  \int_{\answer{0}}^{\answer{\frac{30-2x-5z}{3}}}\d y \d z \d x
    \]
    \[
    \int_{\answer{0}}^{\answer{6}}  \int_{\answer{0}}^{\answer{15-5z/2}} \int_{\answer{0}}^{\answer{\frac{30-2x-5z}{3}}}\d y \d x \d z
    \]
    \[
    \int_{\answer{0}}^{\answer{6}}  \int_{\answer{0}}^{\answer{10-5z/3}} \int_{\answer{0}}^{\answer{\frac{30-3y-5z}{2}}}\d x \d y \d z
    \]
    \[
    \int_{\answer{0}}^{\answer{10}} \int_{\answer{0}}^{\answer{6-3y/5}}  \int_{\answer{0}}^{\answer{\frac{30-3y-5z}{2}}}\d x \d z \d y
    \]
  \end{prompt}
\end{exercise}
\end{document}
