\documentclass{ximera}
\newcommand{\RR}{\mathbb R}
\renewcommand{\d}{\,d}
\newcommand{\dd}[2][]{\frac{d #1}{d #2}}
\renewcommand{\l}{\ell}
\newcommand{\ddx}{\frac{d}{dx}}
\newcommand{\dfn}{\textbf}
\newcommand{\eval}[1]{\bigg[ #1 \bigg]}

\author{Jim Talamo}
\license{Creative Commons 3.0 By-NC}
\outcome{Determine if a function is linear}
%THIS EXERCISE SHOULD COME BEFORE ``functionsoflinearfunctions", which is designed for students to compute indefinite integrals of functions of linear functions without performing an explicit substitution
\begin{document}
\begin{exercise}
We say that a function $f(x)$ is \emph{linear in x} if we can write $f(x)$ in the form $f(x) = ax+b$ for constants $a$ and $b$.  If a function is not linear, we say that it is \emph{nonlinear in x}.

Determine if the following functions are linear or nonlinear:

For the function $f(x) = 5x+3$, $f(x)$ is \wordChoice{\choice[correct]{linear} \choice{nonlinear}} in $x$.

For the function $f(x) = \sin(2x+1)$, $f(x)$ is \wordChoice{\choice{linear} \choice[correct]{nonlinear}} in $x$.

For the function $f(x) = 2x^2+3$, $f(x)$ is \wordChoice{\choice{linear} \choice[correct]{nonlinear}} in $x$.

For the function $f(x) = (2x+1)^2-4x^2$, $f(x)$ is \wordChoice{\choice[correct]{linear} \choice{nonlinear}} in $x$.

For the function $f(x) = (\sqrt{x}+1)^2-2\sqrt{x}$, $f(x)$ is \wordChoice{\choice[correct]{linear} \choice{nonlinear}} in $x$.

For the function $f(x) = \frac{\sin(x)+5}{3\sin(x)}$, $f(x)$ is \wordChoice{\choice[correct]{linear} \choice{nonlinear}} in $x$.

\end{exercise}
\end{document}