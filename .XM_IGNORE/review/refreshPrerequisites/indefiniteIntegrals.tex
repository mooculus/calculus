\documentclass{ximera}

%\newcommand{\RR}{\mathbb R}
\renewcommand{\d}{\,d}
\newcommand{\dd}[2][]{\frac{d #1}{d #2}}
\renewcommand{\l}{\ell}
\newcommand{\ddx}{\frac{d}{dx}}
\newcommand{\dfn}{\textbf}
\newcommand{\eval}[1]{\bigg[ #1 \bigg]}

\newcommand{\RR}{\mathbb R}
\renewcommand{\d}{\,d}
\newcommand{\dd}[2][]{\frac{d #1}{d #2}}
\renewcommand{\l}{\ell}
\newcommand{\ddx}{\frac{d}{dx}}
\newcommand{\dfn}{\textbf}
\newcommand{\eval}[1]{\bigg[ #1 \bigg]}


\author{}
\license{Creative Commons 3.0 By-NC}
\outcome{}
\begin{document}

\begin{exercise}

In this course, we will learn several techniques for evaluating tricky indefinite integrals. A prerequisite is proficiency with the basic rules for evaluating indefinite integrals. In this exercise, we evaluate several indefinite integrals using basic antiderivative rules. For each problem, please use $C$ as the arbitrary constant in your solution.

\begin{exercise} Calculate:
\[
\int x^3 +6 x + 1 \d x = \answer{x^4/4 + 3x^2 + x+ C}
\]

\end{exercise}


\begin{exercise} Calculate:
\[
\int \frac{5}{x^5} + \frac{4}{x^4} \d x = \answer{\frac{-5}{4 x^4}-\frac{4}{3 x^3}+C}
\]

\end{exercise}

\begin{exercise} Calculate:
\[
\int \cos(x) - 2x^2 \d x = \answer{\sin(x) - 2x^3/3+C}
\]

\end{exercise}

\begin{exercise} Calculate:
\[
\int e^x - e - x \d x = \answer{e^x-ex-x^2/2+C}
\]

\end{exercise}



\begin{exercise} Calculate:
\[
\int 16 \d u = \answer{16u+C}
\]

\end{exercise}


\begin{exercise} Calculate:
\[
\int 7 \d z = \answer{7z+C}
\]

\end{exercise}

\begin{exercise} Calculate:
\[
\int  \d t = \answer{t+C}
\]

\end{exercise}


\end{exercise}
\end{document}