\documentclass{ximera}

\newcommand{\RR}{\mathbb R}
\renewcommand{\d}{\,d}
\newcommand{\dd}[2][]{\frac{d #1}{d #2}}
\renewcommand{\l}{\ell}
\newcommand{\ddx}{\frac{d}{dx}}
\newcommand{\dfn}{\textbf}
\newcommand{\eval}[1]{\bigg[ #1 \bigg]}



\outcome{Answer conceptual questions about sequences.}
\outcome{Answer conceptual questions about series.}
\outcome{Determine which convergence tests are applicable to a given series.}
\outcome{Determine an efficient test to use to determine whether a given series converges or diverges.}


\title[Dig-In:]{Putting it all together}

\begin{document}
\begin{abstract}
The various concepts associated with sequences and series are reviewed.
\end{abstract}
\maketitle

All of the series convergence tests we have used require that the
underlying sequence $\seq{a_n}$ be a positive sequence. We can
actually relax this and state that there must be an $N>0$ such that
$a_n>0$ for all $n>N$; that is, $\seq{a_n}$ is positive for all but a
finite number of values of $n$.  We've also stated this by saying that 
the tail of the sequence must have positive terms. In this section we explore series
whose summation includes negative terms.


\section{Alternating series test}

We start with a very specific form of series, where the terms of the
summation alternate between being positive and negative.

\begin{definition}
Let $\seq{a_n}$ be a positive sequence. An \dfn{alternating series} is
a series of either the form 
\[
\sum_{n=1}^\infty (-1)^na_n\qquad \text{or}\qquad \sum_{n=1}^\infty (-1)^{n+1}a_n.
\]
\end{definition}

In essence, the signs of the terms of $\seq{a_n}$ alternate between 
positive and negative.

Recall that the terms of the harmonic series come from the harmonic sequence
$\seq{a_n} = \seq{1/n}$. An important alternating series is the
\dfn{alternating harmonic series}:
\[
\sum_{n=1}^\infty \frac{(-1)^{n+1}}{n} = 1-\frac12+\frac13-\frac14+\frac15-\frac16+\cdots
\]

Geometric series are also alternating series when $r<0$. For
instance, if $r=-1/2$, the geometric series is
\[
\sum_{n=0}^\infty \left(\frac{-1}{2}\right)^n = 1-\frac12+\frac14-\frac18+\frac1{16}-\frac1{32}+\cdots
\]

We know that geometric series converge when $|r|<1$ and have the sum:
\[
\sum_{n=0}^\infty r^n = \frac1{1-r}.
\]
When $r=-1/2$ as above, we find
\[
\sum_{n=0}^\infty \left(\frac{-1}{2}\right)^n = \frac1{1-(-1/2)} = \frac 1{3/2} = \frac23.
\]

A powerful convergence theorem exists for other alternating series
that meet a few conditions.

\begin{theorem}[Alternating Series Test]\index{alternating series test}
Let $\seq{a_n}$ be a positive, nonincreasing sequence where
$\lim_{n\to\infty}a_n=0$. Then
\[
\sum_{n=1}^\infty (-1)^{n}a_n \qquad \text{and}\qquad \sum_{n=1}^\infty (-1)^{n+1}a_n 
\]
converge.
\end{theorem}

%% The basic idea behind the theorem above is illustrated below
%% \[
%% BADBAD%\ref{fig:alt_series_converge}
%% \]
%% A positive, decreasing sequence $\seq{a_n}$ is shown along with the
%% partial sums
%% \[
%% S_n = \sum_{i=1}^n(-1)^{i+1}a_i =a_1-a_2+a_3-a_4+\cdots+(-1)^na_n.
%% \]
%% Because $\seq{a_n}$ is decreasing, the amount by which $S_n$ bounces
%% up/down decreases. Moreover, the odd terms of $S_n$ form a decreasing,
%% bounded sequence, while the even terms of $S_n$ form an increasing,
%% bounded sequence. Since bounded, monotonic sequences converge and the
%% terms of $\seq{a_n}$ approach $0$, one can show the odd and even terms
%% of $S_n$ converge to the same common limit $L$, the sum of the series.

\begin{question}
  Does the alternating series test apply to the series
  \[
  \sum_{n=1}^\infty \frac{(-1)^{n+1}}{n} ?
  \]
  \begin{prompt}
    \begin{multipleChoice}
      \choice[correct]{yes}
      \choice{no}
    \end{multipleChoice}
    \begin{feedback}
      This is the alternating harmonic series as seen previously. The
      underlying sequence is $\seq{a_n} = \seq{1/n}$, which is
      positive, decreasing, and approaches 0 as
      $n\to\infty$. Therefore we can apply the alternating series test
      and conclude this series converges.
      
      %While the test does not state what the series converges to, we
      %will see later that $\sum_{n=1}^\infty
      %(-1)^{n+1}\frac1n=\ln2.$
    \end{feedback}
  \end{prompt}
  \begin{question}
    Does the alternating series test apply to the series
    \[
    \sum_{n=1}^\infty \frac{(-1)^n\ln n}{n}?
    \]
    \begin{prompt}
      \begin{multipleChoice}
        \choice[correct]{yes}
        \choice{no}
      \end{multipleChoice}
      \begin{feedback}
        The underlying sequence is $\seq{a_n} = \seq{\ln n/n}$. This
        is positive and approaches $0$ as $n\to\infty$ (use
        L'H\^opital's Rule). However, the sequence is not decreasing
        for all $n$. It is straightforward to compute $a_1=0$,
        $a_2\approx0.347$, $a_3\approx 0.366$, and $a_4\approx 0.347$:
        the sequence is increasing for at least the first $3$ terms.
      
        However, we do not ``give-up'' and immediately conclude that we
        cannot apply the alternating series test. Rather, consider the
        long term behavior of $\seq{a_n}$. Treating $a_n=a(n)$ as a
        continuous function of $n$ defined on $(1,\infty)$, we can take
        its derivative:
        \[
        a'(n) = \frac{1-\ln n}{n^2}.
        \]
        The derivative is negative for all $n\geq 3$ (actually, for
        all $n>e$), meaning $a(n)=a_n$ is decreasing on
        $(3,\infty)$. We can now apply the alternating series test to
        the series when we start with $n=3$ and conclude that
        $\sum_{n=3}^\infty(-1)^n\frac{\ln n}{n}$ converges; adding the
        terms with $n=1$ and $n=2$ do not change the convergence.
        
        The important lesson here is that as before, if a series fails
        to meet the criteria of the alternating series test on only a
        finite number of terms, we can still apply the test.
      \end{feedback}
    \end{prompt}
    \begin{question}
      Does the alternating series test apply to the series
      \[
      \sum_{n=1}^\infty (-1)^{n+1}\frac{|\sin n|}{n^2}?
      \]
      \begin{prompt}
        \begin{multipleChoice}
          \choice{yes}
          \choice[correct]{no}
        \end{multipleChoice}
        \begin{feedback}
          The underlying sequence is $\seq{a_n} = |\sin n|/n$. This
          sequence is positive and approaches $0$ as
          $n\to\infty$. However, it is not a decreasing sequence; the
          value of $|\sin n|$ oscillates between $0$ and $1$ as
          $n\to\infty$. We cannot remove a finite number of terms to
          make $\seq{a_n}$ decreasing, therefore we cannot apply the
          alternating series test.
	  
          Keep in mind that this does not mean we conclude the series
          diverges; in fact, it does converge. We are just unable to
          conclude this based on the alternating series test.
        \end{feedback}
      \end{prompt}
    \end{question}
  \end{question}
\end{question}




\section{Approximating alternating series}


While there are many factors involved when studying rates of
convergence, the alternating structure of an alternating series gives
us a powerful tool when approximating the sum of a convergent series.

\begin{theorem}[Alternating Series Approximation]\index{alternating approximation theorem}
Let $\seq{a_n}$ be a sequence that satisfies the hypotheses of the
alternating series test, let $S_n$ be the $n$th partial
sum, and let
\[
L = \sum_{n=1}^\infty (-1)^{n}a_n\qquad\text{or}\qquad L=\sum_{n=1}^\infty (-1)^{n+1}a_n.
\]
Then
\begin{itemize}
\item $|L-S_n| \leq a_{n+1}$, and
\item $L$ is between $S_n$ and $S_{n+1}$.
\end{itemize}
In this case, $R_n=L-S_n$ is called the $n$th \dfn{remainder} of the
series.
\end{theorem}

Here is the basic idea behind this theorem.  Say we have an alternating sequence, 
$\seq{a_n}$.  Let's assume the first term is positive, so the second is negative, and 
so on.  We add the first two numbers and get some number $S_2$.  Now $S_2$ is 
smaller than $S_1 = a_1$, because we subtracted something from $a_1$.  Next, we 
add on the third term, $a_3$, to get the partial sum $S_3$.  This $S_3$ is bigger than 
$S_2$, because the sequence is alternating, but is smaller than $S_1$, because the
sequence is decreasing.

If we know the series converges to some $L$, we can see that we must be bouncing 
back and forth around $L$ as we add and subtract terms.  At one point, 
$S_n$ is larger than $L$, and then subtracting off the next term makes the partial sum 
smaller than $L$.  In other words, the 
true limit $L$ must be between $S_n$ and $S_{n+1}$.  Imagine plotting $\seq{a_n}$, 
$\seq{S_n}$, and $L$ on a number line.  (Or, try it yourself with the alternating harmonic 
series!)  can be no further from $S_n$ than whatever the next term in the sequence 
is.  How do we get from $S_n$ to $S_{n+1}$?  By adding (or subtracting) $a_{n+1}$, which 
takes us back ``across'' $L$ again.  In other words, the distance between $L$ and $S_n$ can 
be no more than $a_{n+1}$.

See if you can use these same ideas to prove the alternating series test!

Let's see an example of approximating an alternating series.

\begin{example}
Approximate the sum of the alternating harmonic series with an error less than $10^{-2}$.
\begin{explanation}
  Look at the $(n+1)$th term.
	\begin{align*}
	\frac{1}{n+1} &< 10^{-2}\\
	n+1&> 100\\
	n&> 99 
	\end{align*}
	Using a computer, we can see that
        \[
        S_{99} = 0.69849462\dots
        \]
        We will see later that the true value of this series is $\ln(2)$. Comparing
        \begin{align*}
          \ln(2) &= 0.69314718\dots \qquad \text{and}\\
          S_{99} &= 0.69849462\dots
        \end{align*}
         we have our desired accuracy.
\end{explanation}
\end{example}

%% BADBAD
%% Which partial sum gives a remainder less than....
%% It would be nice to have quick questions about this




\section{Absolute convergence versus conditional convergence}


It is an interesting result that the harmonic series,
\[
\sum_{n=1}^\infty \frac1n
\]
diverges, yet the alternating harmonic series,
\[
\sum_{n=1}^\infty (-1)^{n+1}\frac1n,
\]
converges. The notion that simply alternating the signs of the terms in a
series can change a series from divergent to convergent leads us to the following
definitions.

\begin{definition}\index{absolutley convergent}\index{conditionally convergent}\hfil
\begin{itemize}
\item A series $\sum_{n=1}^\infty a_n$ \dfn{converges absolutely} if $\sum_{n=1}^\infty |a_n|$ converges.
\item A series $\sum_{n=1}^\infty a_n$ \dfn{converges conditionally} if $\sum_{n=1}^\infty a_n$ converges but $\sum_{n=1}^\infty |a_n|$ diverges.
\end{itemize}
\end{definition}

Note, in the definition above, $\sum_{n=1}^\infty a_n$ is not
necessarily an alternating series; it may just have some negative
terms.

\begin{question}
  Does the series
  \[
  \sum_{n=1}^\infty (-1)^n\frac{n+3}{n^2+2n+5}
  \]
  converge absolutely, converge conditionally, or diverge?
  \begin{prompt}
    \begin{multipleChoice}
      \choice[correct]{The series converges conditionally.}
      \choice{The series converges absolutely.}
      \choice{The series diverges.}
    \end{multipleChoice}
    \begin{feedback}
      We can show the series
      \[
      \sum_{n=1}^\infty \left|(-1)^n\frac{n+3}{n^2+2n+5}\right|=
      \sum_{n=1}^\infty \frac{n+3}{n^2+2n+5}
      \]
      diverges using the limit comparison test, comparing with $1/n$.
      The series
      \[
      \sum_{n=1}^\infty (-1)^n\frac{n+3}{n^2+2n+5}
      \]
      converges using the alternating series test; we conclude the
      series converges conditionally.
    \end{feedback}
  \end{prompt}
  \begin{question}
    Does the series
    \[
    \sum_{n=1}^\infty (-1)^n\frac{n^2+2n+5}{2^n}
    \]
    converge absolutely, converge conditionally, or diverge?
    \begin{prompt}
      \begin{multipleChoice}
        \choice{The series converges conditionally.}
        \choice[correct]{The series converges absolutely.}
        \choice{The series diverges.}
      \end{multipleChoice}
      \begin{feedback}
        We can show the series
        \[
        \sum_{n=1}^\infty \left|(-1)^n\frac{n^2+2n+5}{2^n}\right|=\sum_{n=1}^\infty \frac{n^2+2n+5}{2^n}
        \]
        converges using the ratio test.  Therefore we conclude
        \[
        \sum_{n=1}^\infty (-1)^n\frac{n^2+2n+5}{2^n}
        \]
        converges absolutely.
      \end{feedback}
    \end{prompt}
    \begin{question}
      Does the series
      \[
      \sum_{n=3}^\infty (-1)^n\frac{3n-3}{5n-10}
      \]
      converge absolutely, converge conditionally, or diverge?
      \begin{prompt}
        \begin{multipleChoice}
          \choice{The series converges conditionally.}
          \choice{The series converges absolutely.}
          \choice[correct]{The series diverges.}
        \end{multipleChoice}
        \begin{feedback}
	  The series
          \[
          \sum_{n=3}^\infty \left|(-1)^n\frac{3n-3}{5n-10}\right| =
          \sum_{n=3}^\infty \frac{3n-3}{5n-10}
          \]
          diverges using the divergence test, so it does not converge
          absolutely.  The series
          \[
          \sum_{n=3}^\infty (-1)^n\frac{3n-3}{5n-10}
          \]
          fails the conditions of the alternating series test as
          $(3n-3)/(5n-10)$ does not approach $0$ as $n\to\infty$. We
          can state further that this series diverges; as
          $n\to\infty$, the series effectively adds and subtracts
          $3/5$ over and over. This causes the sequence of partial
          sums to oscillate and not converge.  Therefore the series
          diverges.
        \end{feedback}
      \end{prompt}
    \end{question}
  \end{question}
\end{question}

Knowing that a series converges absolutely allows us to make two
important statements. The first, given in the following theorem, is
that absolute convergence is ``stronger'' than regular
convergence. That is, just because
\[
\sum_{n=1}^\infty a_n
\]
converges, we cannot conclude that
\[
\sum_{n=1}^\infty |a_n|
\]
will converge, but knowing a series converges absolutely tells us that
$\sum_{n=1}^\infty a_n$ will converge.

One reason this is important is that our convergence tests all require
that the underlying sequence of terms be positive. By taking the
absolute value of the terms of a series where not all terms are
positive, we are often able to apply an appropriate test and determine
absolute convergence. This, in turn, determines that the series we are
given also converges.

The second statement relates to \dfn{rearrangements} of series. When
dealing with a finite set of numbers, the sum of the numbers does not
depend on the order in which they are added. (So $1+2+3 = 3+1+2$.) One
may be surprised to find out that when dealing with an infinite set of
numbers, the same statement does not always hold true: some infinite
lists of numbers may be rearranged in different orders to achieve
different sums. The theorem states that the terms of an absolutely
convergent series can be rearranged in any way without affecting the
sum.

\begin{theorem}[Absolute Convergence]\index{absolute convergence theorem}
  Let $\sum_{n=1}^\infty a_n$ be a series that converges absolutely.
  Let $\seq{b_n}$ be any rearrangement of the sequence
  $\seq{a_n}$. Then
  \[
  \sum_{n=1}^\infty b_n = \sum_{n=1}^\infty a_n.
  \]
\end{theorem}

This theorem states that rearranging the terms of an absolutely
convergent series does not affect its sum. Making such a statement implies that perhaps
the sum of a conditionally convergent series can change based on the
arrangement of terms. Indeed, it can. The \textit{Riemann
  rearrangement theorem} (named after Bernhard Riemann) states that
any conditionally convergent series can have its terms rearranged so
that the sum is any desired value, including $\infty$!


As an example, consider the alternating harmonic series once more. We
have stated that
\[
\sum_{n=1}^\infty \frac{(-1)^{n+1}}{n} =1-\frac12+\frac13-\frac14+\frac15-\frac16+\frac17\cdots = \ln 2.
\]
Consider the rearrangement where every positive term is followed by two negative terms:
\[
1-\frac12-\frac14+\frac13-\frac16-\frac18+\frac15-\frac1{10}-\frac1{12}\cdots
\]
(Convince yourself that these are exactly the same numbers as appear
in the alternating harmonic series, just in a different order.) Now
group some terms and simplify:
\begin{align*}
\left(1-\frac12\right)-\frac14+\left(\frac13-\frac16\right)-\frac18+\left(\frac15-\frac1{10}\right)-\frac1{12}+\cdots &= \\
\frac12-\frac14+\frac16-\frac18+\frac1{10}-\frac{1}{12}+\cdots &= \\
\frac12\left(1-\frac12+\frac13-\frac14+\frac15-\frac16+\cdots\right) & = \frac12\ln 2.
\end{align*}
By rearranging the terms of the series, we have arrived at a different
sum!
\begin{quote}
  One could \textit{try} to argue that the alternating harmonic series
  does not actually converge to $\ln 2$, and here is an example of such 
  an argument. According to the
  alternating series test, we know that this series converges to some number $L$. 
   If, as our intuition tells us should be true, the rearrangement does not change the sum,
  then we have just seen that $L = L/2$.  The only possibility for $L$ is then $L=0$. But the alternating series
  approximation theorem quickly shows that $L>0$. The only conclusion
  is that the rearrangement \textit{did}, contrary to our intuition, change the sum.
\end{quote}
The fact that conditionally convergent series can be rearranged to
equal any number is really an incredible result.



While series are worthy of study in and of themselves, our ultimate
goal within calculus is the study of \textit{power series}, which we
will consider in the next section. We will use power series to create
functions where the output is the result of an infinite summation.

%A special type of power series is something called a Taylor Series; in the context of Taylor Series we will finally show the Alternating Harmonic Series converges to $\ln 2$.

\end{document}
