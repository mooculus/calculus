\documentclass{ximera}

\newcommand{\RR}{\mathbb R}
\renewcommand{\d}{\,d}
\newcommand{\dd}[2][]{\frac{d #1}{d #2}}
\renewcommand{\l}{\ell}
\newcommand{\ddx}{\frac{d}{dx}}
\newcommand{\dfn}{\textbf}
\newcommand{\eval}[1]{\bigg[ #1 \bigg]}


\title[Dig-In:]{Second Fundamental Theorem of Calculus}

\begin{document}
\begin{abstract}
Accumulated area is computed by undoing the derivative. 
\end{abstract}
\maketitle




There is a another common form of the Fundamental Theorem of Calculus:

Here the notation
\[
F(x) \Biggr|_a^b
\]
means that one should evaluate $F(x)$ at $b$ and then subtract from this $F(x)$ evaluated at $a$. Hence
\[
F(x) \Biggr|_a^b = F(b)-F(a).
\]  

\begin{theorem}[Second Fundamental Theorem of Calculus]
  Suppose that $f(x)$ is continuous on the interval $[a,b]$. If $F(x)$
  is any antiderivative of $f(x)$, then
  \[
  \int_a^b f(x)\d x = \eval{F(x)}_a^b = F(b)-F(a).
  \]
\end{theorem}

\begin{proof}
We know from Theorem~\ref{thm:fundamental_theorem_I} 
\[
  G(x)=\int_a^x f(t)\d t
\]
is an antiderivative of $f(x)$, and therefore any antiderivative
$F(x)$ of $f(x)$ is of the form $F(x)=G(x)+k$. Then 
\begin{align*}
  F(b)-F(a) &=G(b)+k-(G(a)+k) 
  &= G(b)-G(a) \\
  &=\int_a^b f(t)\d t-\int_a^a f(t)\d t.
\end{align*}
It is not hard to see that $\int_a^a f(t)\d t=0$, so this means that
\[
  F(b)-F(a)=\int_a^b f(t)\d t,
\]
which is exactly what Theorem~\ref{thm:fundamental_theorem_II} says.
\end{proof}

From this you should see that the two versions of the Fundamental
Theorem are very closely related. To avoid confusion, some people call
the two versions of the theorem ``The Fundamental Theorem of
Calculus---Version I'' and ``The Fundamental Theorem of
Calculus---Version II'', although unfortunately there is no universal
agreement as to which is ``Version I'' and which ``Version II''. Since
it really is the same theorem, differently stated, people often simply
call them both ``The Fundamental Theorem of Calculus.''

Let's see an example of the fundamental theorem in action.

\begin{example}
Compute
\[
\int_1^2\left(x^9 + \frac{1}{x}\right) \d x
\]
\end{example}

\begin{solution}
Here we start by finding an antiderivative of 
\[
x^9 + \frac{1}{x}.
\]
The correct choice is $\frac{x^{10}}{10} + \ln(x)$, one could verify this by
taking the derivative. Hence
\begin{align*}
\int_1^2\left(x^9 + \frac{1}{x}\right) \d x &= \left(\frac{x^{10}}{10} + \ln(x)\right)\Bigg|_1^2 \\
&= \frac{2^{10}}{10} + \ln(2) - \frac{1}{10}.
\end{align*}
\end{solution}


When we compute a definite integral, we first find an antiderivative
and then substitute. It is convenient to first display the
antiderivative and then do the substitution; we need a notation
indicating that the substitution is yet to be done. A typical solution
would look like this:
\[
  \left.\int_1^2 x^2\d x={x^3\over 3}\right|_1^2 = 
  {2^3\over3}-{1^3\over3}={7\over3}.
\]
The vertical line with subscript and superscript is used to indicate
the operation ``substitute and subtract'' that is needed to finish the
evaluation. 

Now we know that to solve certain kinds of problems, those that lead
to a sum of a certain form, we ``merely'' find an antiderivative and
substitute two values and subtract. Unfortunately, finding
antiderivatives can be quite difficult. While there are a small number
of rules that allow us to compute the derivative of any common
function, there are no such rules for antiderivatives. There are some
techniques that frequently prove useful, but we will never be able to
reduce the problem to a completely mechanical process.

\end{document}
