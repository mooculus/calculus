\documentclass{ximera}

%\newcommand{\RR}{\mathbb R}
\renewcommand{\d}{\,d}
\newcommand{\dd}[2][]{\frac{d #1}{d #2}}
\renewcommand{\l}{\ell}
\newcommand{\ddx}{\frac{d}{dx}}
\newcommand{\dfn}{\textbf}
\newcommand{\eval}[1]{\bigg[ #1 \bigg]}

\newcommand{\RR}{\mathbb R}
\renewcommand{\d}{\,d}
\newcommand{\dd}[2][]{\frac{d #1}{d #2}}
\renewcommand{\l}{\ell}
\newcommand{\ddx}{\frac{d}{dx}}
\newcommand{\dfn}{\textbf}
\newcommand{\eval}[1]{\bigg[ #1 \bigg]}


\author{}
\license{Creative Commons 3.0 By-NC}
\outcome{}
\begin{document}

\begin{exercise}

  Here is an example of an application where working correctly with square roots arises.
  
  Consider the curve given by:
  \[
    y = \frac{1}{27} (9x^2 + 6)^{3/2} \text{ from $x = 0$ to $x = 3$}
  \]
  \begin{enumerate}
    \item Calculate the derivative $y'$.
      Simplify your answer as completely as possible.
      \[
        \frac{\d y}{\d x} = \answer{x \sqrt{9x^2 + 6}}
      \]
    
    \item Calculate $(\d y/ \d x)^2$.
      Simplify your answer completely.
      \[
        \left( \frac{\d y}{\d x} \right)^2 = \answer{9x^4 + 6x^2}
      \]
      
    \item The length of this curve is given by the formula:
    \[
      L = \int_0^3 \sqrt{1 + \left( \frac{\d y}{\d x} \right)^2} \d x
    \]
    
    Substitute the expression for $(\d y/ \d x)^2$ into the expression above.
    You should notice that the expression in the integrand is a perfect square!
    Indeed,
    \[
      1 + \left( \frac{\d y}{\d x} \right)^2 = (\answer{3x^2 + 1})^2
    \]
    so
    $\sqrt{1 + \left( \frac{\d y}{\d x} \right)^2} = (\answer{3x^2 + 1})$ (simplify completely).
    
    Evaluating the integral $L \int_0^3 \sqrt{1 + \left( \frac{\d y}{\d x} \right)^2} \d x$ now gives the length of the curve is $\answer{30}$.
    (Type an exact answer, using radicals as needed.)
  \end{enumerate}


\end{exercise}
\end{document}