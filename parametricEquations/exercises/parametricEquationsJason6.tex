\documentclass{ximera}

\newcommand{\RR}{\mathbb R}
\renewcommand{\d}{\,d}
\newcommand{\dd}[2][]{\frac{d #1}{d #2}}
\renewcommand{\l}{\ell}
\newcommand{\ddx}{\frac{d}{dx}}
\newcommand{\dfn}{\textbf}
\newcommand{\eval}[1]{\bigg[ #1 \bigg]}


\author{Jason Miller}
\license{Creative Commons 3.0 By-bC}


\outcome{}

\begin{document}
\begin{exercise}



Consider the parametric equations 
\begin{align*}
x(t) &= 2\sin(2t)\\
y(t) &= 2\sin(t)
\end{align*}

For $-\pi \leq t < \pi$.

The curve traced out by these parametric equations is pictured below. 

\begin{image}  
  \begin{tikzpicture}  
    \begin{axis}[  
        xmin=-2.2,  
        xmax=2.2,  
        ymin=-2.2,  
        ymax=2.2,  
        axis lines=center,  
        xlabel=$x$,  
        ylabel=$y$,  
        every axis y label/.style={at=(current axis.above origin),anchor=south},  
        every axis x label/.style={at=(current axis.right of origin),anchor=west},  
      ]  
      \addplot [ultra thick, penColor, smooth, samples=100, domain=-pi: pi] ({2*sin(deg(2*x))},{2*sin(deg(x))});
      \end{axis}  
  \end{tikzpicture}  
\end{image} 

Determine the tangent line to the curve when $t=\frac{\pi}{6} $. 

The tangent line is given by $y-\answer{1}=\answer{ \frac{\sqrt{3}}{2} } (x-\answer{\sqrt{3}})$


\begin{image}  
  \begin{tikzpicture}  
    \begin{axis}[  
        xmin=-2.2,  
        xmax=2.5,  
        ymin=-2.2,  
        ymax=2.2,  
        axis lines=center,  
        xlabel=$x$,  
        ylabel=$y$,  
        every axis y label/.style={at=(current axis.above origin),anchor=south},  
        every axis x label/.style={at=(current axis.right of origin),anchor=west},  
      ]  
      \addplot [ultra thick, penColor, smooth, samples=100, domain=-pi: pi] ({2*sin(deg(2*x))},{2*sin(deg(x))});
       \addplot [ultra thick, penColor4, smooth, samples=100, domain=-pi:pi] { .866*x-.5};
      \end{axis}  
  \end{tikzpicture}  
\end{image} 


\begin{exercise}

Find all the $t$ values for which the tangent line is horizontal. 

The horizontal tangent line that occurs when $t$ is positive is $t=\answer{ \frac{\pi}{2}}$. The equation of this tangent line is $\answer{ y=2}$.  

The horizontal tangent line that occurs when $t$ is negative is  $t=\answer{ \frac{-\pi}{2}}$. The equation of this tangent line is $\answer {y=-2}$. 

Find all the $t$ values for which the tangent line is horizontal. 

Enter the $t$ values from smallest to largest. Remember that our $t$ value lie in the interval $[-\pi, \pi)$. 

The $t$ values where the tangent line is vertical, in order, are $\answer{-\frac{3\pi}{4}}$, $\answer{ -\frac{\pi}{4}}$, $\answer{ \frac{\pi}{4}}$, and $\answer{ \frac{ 3\pi}{4}}$. 

These give rise to two vertical tangent lines $x=2$ and $x=-2$ which are shown below. 

\begin{image}  
  \begin{tikzpicture}  
    \begin{axis}[  
        xmin=-2.2,  
        xmax=2.5,  
        ymin=-2.2,  
        ymax=2.2,  
        axis lines=center,  
        xlabel=$x$,  
        ylabel=$y$,  
        every axis y label/.style={at=(current axis.above origin),anchor=south},  
        every axis x label/.style={at=(current axis.right of origin),anchor=west},  
      ]  
      \addplot [ultra thick, penColor, smooth, samples=100, domain=-pi: pi] ({2*sin(deg(2*x))},{2*sin(deg(x))});
      \addplot +[ultra thick, mark=none] coordinates {(-2, -2) (-2, 2)};
       \addplot +[ultra thick, mark=none] coordinates {(2, -2) (2, 2)};
      \end{axis}  
  \end{tikzpicture}  
\end{image} 

\end{exercise}
\end{exercise}
\end{document}
