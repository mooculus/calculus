\documentclass{ximera}
\newcommand{\RR}{\mathbb R}
\renewcommand{\d}{\,d}
\newcommand{\dd}[2][]{\frac{d #1}{d #2}}
\renewcommand{\l}{\ell}
\newcommand{\ddx}{\frac{d}{dx}}
\newcommand{\dfn}{\textbf}
\newcommand{\eval}[1]{\bigg[ #1 \bigg]}

\author{Steven Gubkin}
\license{Creative Commons 3.0 By-NC}

\outcome{Find the values guaranteed by Rolle's Theorem or the Mean Value Theorem.}
\outcome{Understand the statement of the Mean Value Theorem.}
\outcome{Determine whether Rolle's Theorem or the Mean Value Theorem can be applied.}
\begin{document}
\begin{exercise}

Let $f(x) = Ax^2+Bx+C$.  Find the constant $c$, known to exist by the mean value theorem, for which $f'(c) = \frac{f(b)-f(a)}{b-a}$, where $a$ and $b$ are any two distinct real numbers.  The result is a very interesting property of quadratics.  Can you explain this result more conceptually, and less algebraically?

\begin{prompt}
	$$c = \answer{\frac{b+a}{2}}$$
\end{prompt}

\end{exercise}
\end{document}