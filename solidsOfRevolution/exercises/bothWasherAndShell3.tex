\documentclass{ximera}

\newcommand{\RR}{\mathbb R}
\renewcommand{\d}{\,d}
\newcommand{\dd}[2][]{\frac{d #1}{d #2}}
\renewcommand{\l}{\ell}
\newcommand{\ddx}{\frac{d}{dx}}
\newcommand{\dfn}{\textbf}
\newcommand{\eval}[1]{\bigg[ #1 \bigg]}


\author{ Jason Miller}
\license{Creative Commons 3.0 By-NC}


\outcome{Find the bounded area between two curves}
\outcome{Find volume of solid described using cross sections}

\begin{document}
\begin{exercise}

The region bounded by the curves $y=x-1$, $y=\ln(x)$, and $y=1$ is revolved about the line $y=1$. 

To use the Washer Method to set up an integral or sum of integrals that would give the volume of the solid: 

  \begin{multipleChoice}
    \choice[correct]{we should integrate with respect to $x$.}
    \choice{we should integrate with respect to $y$.}
  \end{multipleChoice}

How many integrals will we need to express the volume of the solid using the Washer Method: $\answer{2}$

\begin{exercise}

Express the volume of the solid using the Washer Method method: 
\[
\int_{\answer{1}}^{\answer{2}} \answer{ \pi \left( 1 - \ln(x) \right)^{2} - \pi \left( 2-x \right)^{2} } \d x + \int_{\answer{2}}^{\answer{e}} \answer{ \pi \left( 1- \ln(x) \right)^{2} } \d x
\]
\end{exercise}
\end{exercise}

\begin{exercise}

To use the Shell Method to set up an integral or sum of integrals that would give the volume of the solid: 

  \begin{multipleChoice}
    \choice{we should integrate with respect to $x$.}
    \choice[correct]{we should integrate with respect to $y$.}
  \end{multipleChoice}

How many integrals will we need to express the volume of the solid using the Shell Method: $\answer{1}$. 


\begin{exercise} 
The integral that gives the volume of $S$ is: 
\[
\int_{\answer{0}}^{\answer{1}} \answer{ 2\pi \left( 1-y \right) \left( e^{y}-y-1 \right)} \d y
\] 

\begin{hint}
You should notice that the curves $y=\ln(x)$ and $y=x-1$ intersect when $x=1$ and the corresponding $y$-value will be the lower limit of integration.
\end{hint}
\end{exercise}
\end{exercise}
\end{document}

