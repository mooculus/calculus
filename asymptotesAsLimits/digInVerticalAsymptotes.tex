\documentclass{ximera}

\newcommand{\RR}{\mathbb R}
\renewcommand{\d}{\,d}
\newcommand{\dd}[2][]{\frac{d #1}{d #2}}
\renewcommand{\l}{\ell}
\newcommand{\ddx}{\frac{d}{dx}}
\newcommand{\dfn}{\textbf}
\newcommand{\eval}[1]{\bigg[ #1 \bigg]}


\title[Dig-In:]{Vertical asymptotes}

\outcome{Recognize when a limit is indicating there is a vertical asymptote.}
\outcome{Evaluate the limit as $x$ approaches a point where there is a vertical asymptote.}
\outcome{Match graphs of functions with their equations based on vertical asymptotes.}
\outcome{Discuss what it means for a limit to equal $\infty$.}
\outcome{Define a vertical asymptote.}
\outcome{Understand the relationship between limits and vertical asymptotes.}
\outcome{Find vertical asymptotes of famous functions.}
\outcome{Find vertical asymptotes by looking at a graph.}


\begin{document}
\begin{abstract}
We explore functions that ``shoot to infinity'' near certain points in
their domain.
\end{abstract}
\maketitle

Consider the function
\[
f(x) = \frac{1}{(x+1)^2}.
\]
\begin{image}
\begin{tikzpicture}
	\begin{axis}[
            domain=-2:1,
            ymax=100,
            samples=100,
            axis lines =middle, xlabel=$x$, ylabel=$y$,
            every axis y label/.style={at=(current axis.above origin),anchor=south},
            every axis x label/.style={at=(current axis.right of origin),anchor=west}
          ]
	  \addplot [very thick, penColor, smooth, domain=(-2:-1.1)] {1/(x+1)^2};
          \addplot [very thick, penColor, smooth, domain=(-.9:1)] {1/(x+1)^2};
          \addplot [textColor, dashed] plot coordinates {(-1,0) (-1,100)};
        \end{axis}
\end{tikzpicture}
%% \caption{A plot of $f(x)=\protect\frac{1}{(x+1)^2}$.}
%% \label{plot:1/(x+1)^2}
\end{image}
While the $\lim_{x\to -1} f(x)$ does not exist, something can still be
said.

\begin{definition}\label{def:inflimit}\index{limit!infinite}\index{infinite limit}
If $f(x)$ grows arbitrarily large as $x$ approaches $a$, we write
\[
\lim_{x\to a} f(x) = \infty
\]
and say that the limit of $f(x)$ is equal to \dfn{ infinity} as $x$
goes to $a$.

If $|f(x)|$ grows arbitrarily large as $x$ approaches $a$ and $f(x)$ is
negative, we write
\[
\lim_{x\to a} f(x) = -\infty
\]
and say that the limit of $f(x)$ is equal to \dfn{ negative infinity}
as $x$ goes to $a$.
\end{definition}

\begin{question}
  Which of the following are correct?
  \begin{selectAll}
    \choice[correct]{$\lim_{x\to -1} \frac{1}{(x+1)^2} = \infty$}
    \choice{$\lim_{x\to -1} \frac{1}{(x+1)^2} \to \infty$}
    \choice{$f(x) = \frac{1}{(x+1)^2}$, so $f(-1) = \infty$}
    \choice[correct]{$f(x) = \frac{1}{(x+1)^2}$, so as $x\to -1$,
      $f(x) \to \infty$}
  \end{selectAll}
\end{question}

On the other hand, consider the function 
\[
f(x) = \frac{1}{(x-1)}.
\]
\begin{image}
\begin{tikzpicture}
	\begin{axis}[
            domain=-1:2,
            ymax=50,
            ymin=-50,
            samples=100,
            axis lines =middle, xlabel=$x$, ylabel=$y$,
            every axis y label/.style={at=(current axis.above origin),anchor=south},
            every axis x label/.style={at=(current axis.right of origin),anchor=west}
          ]
	  \addplot [very thick, penColor, smooth, domain=(1.02:2)] {1/(x-1)};
          \addplot [very thick, penColor, smooth, domain=(-1:.98)] {1/(x-1)};
          \addplot [textColor, dashed] plot coordinates {(1,-50) (1,50)};
        \end{axis}
\end{tikzpicture}
%% \caption{A plot of $f(x)=\protect\frac{1}{x-1}$.}
%% \label{plot:1/(x-1)}
\end{image}
While the two sides of the limit as $x$ approaches $1$ do not agree, we can still consider the one-sided
limits.  We see $\lim_{x\to 1^+} f(x) = \infty$ and $\lim_{x\to 1^-} f(x) =
-\infty$.


\begin{definition}\label{def:vert asymptote}\index{asymptote!vertical}\index{vertical asymptote}
If at least one of the following hold:
\begin{itemize}
\item $\lim_{x\to a} f(x) = \pm\infty$,
\item $\lim_{x\to a^+} f(x) = \pm\infty$,
\item $\lim_{x\to a^-} f(x) = \pm\infty$,
\end{itemize}
then the line $x=a$ is a \dfn{vertical asymptote} of $f$.
\end{definition}


\begin{example}
Find the vertical asymptotes of 
\[
f(x) = \frac{x^2-9x+14}{x^2-5x+6}.
\]

\begin{explanation}
Since $f$ is a rational function, it is continuous on its domain. So the only points where the function can possibly have a vertical asymptote are zeros of the denominator. \\
Start by factoring both the numerator and the denominator:
\[
\frac{x^2-9x+14}{x^2-5x+6} = \frac{(x-2)(x-7)}{(x-2)(x-3)}
\]
Using limits, we must investigate what happens with $ f(x)$ when $x\to 2$ and $x\to 3$, since $2$ and $3$ are the only zeros of the denominator. Write
\begin{align*}
\lim_{x\to 2} \frac{(x-2)(x-7)}{(x-2)(x-3)} &= \lim_{x\to 2} \frac{(x-7)}{(x-3)}\\
&= \frac{-5}{-1}\\
&=5.
\end{align*}
Now write
\begin{align*}
\lim_{x\to 3} \frac{(x-2)(x-7)}{(x-2)(x-3)} &= \lim_{x\to 3} \frac{(x-7)}{(x-3)}\\
&= \lim_{x\to 3}\frac{-4}{x-3}.
\end{align*}
Consider the one-sided limits separately.  Since $\lim_{x\to 3^+}
(x-3)$ approaches $0$ from the right and the numerator is negative,
$\lim_{x\to 3^+} f(x) = -\infty$. Since $\lim_{x\to 3^-} (x-3)$
approaches $0$ from the left and the numerator is negative,
$\lim_{x\to 3^-} f(x) = \infty$.
\begin{image}
\begin{tikzpicture}
	\begin{axis}[
            domain=1:4,
            ymax=50,
            ymin=-50,
            samples=100,
            axis lines =middle, xlabel=$x$, ylabel=$y$,
            every axis y label/.style={at=(current axis.above origin),anchor=south},
            every axis x label/.style={at=(current axis.right of origin),anchor=west}
          ]
	  \addplot [very thick, penColor, smooth, domain=(3.02:4)] {(x-7)/(x-3)};
          \addplot [very thick, penColor, smooth, domain=(1:2.98)] {(x-7)/(x-3)};
          \addplot [textColor, dashed] plot coordinates {(3,-50) (3,50)};
          \addplot[color=penColor,fill=background,only marks,mark=*] coordinates{(2,5)};  %% open hole
        \end{axis}
\end{tikzpicture}
%% \caption{A plot of $f(x)=\protect\frac{x^2-9x+14}{x^2-5+6}$.}
%% \label{plot:(x^2-9x+14)/(x^2-5x+6)}
\end{image}
Hence we have a vertical asymptote at $x=3$.
\end{explanation}
\end{example}



\end{document}
