\documentclass{ximera}
\newcommand{\RR}{\mathbb R}
\renewcommand{\d}{\,d}
\newcommand{\dd}[2][]{\frac{d #1}{d #2}}
\renewcommand{\l}{\ell}
\newcommand{\ddx}{\frac{d}{dx}}
\newcommand{\dfn}{\textbf}
\newcommand{\eval}[1]{\bigg[ #1 \bigg]}

\author{Steven Gubkin}
\license{Creative Commons 3.0 By-NC}

\outcome{Understand the statement of the Mean Value Theorem.}
\outcome{Determine whether Rolle's Theorem or the Mean Value Theorem can be applied.}
\outcome{Identify calculus ideas which are consequences of the Mean Value Theorem.}
\begin{document}
\begin{exercise}

We will use the mean value theorem to show that $\sin x<x$ for $x>0$.

First we consider $0<x<2\pi$.  We apply the mean value theorem to $\sin$ on the interval $[0,x]$ to find there exists a $0<c<x$ with 
\[
\frac{\answer{\sin{x}}-\sin{0}}{\answer{x}-0} = \answer{\cos{c}}.
\]
Now we solve for $\sin{x}$ in the above expression to find
\[
\sin{x} = \answer{x\cos{c}}.
\]
Since $0<c<2\pi$ we know that 
\[
\cos{c} < \answer{1}.
\]
Using this bound we find that 
\[
\sin{x} < \answer{x}.
\]

Next for $x=2\pi$, $\sin{x}=\answer{0}$ and so $\sin{x}<x$.

Finally since $\sin$ is periodic with period $\answer{2\pi}$, for $x>2\pi$ there exists some $0<\tilde{x}\leq2\pi$ with $\sin{x}=\sin{\tilde{x}}$.  With this in mind
\[
\sin{x}=\sin{\tilde{x}}<\answer{\tilde{x}}<\answer{x}.
\]

Thus we have shown for all $x>0$ that $\sin{x}<x$.

\end{exercise}
\end{document}