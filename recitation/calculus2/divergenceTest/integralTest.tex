\documentclass[noauthor]{ximera}
%handout:  for handout version with no solutions or instructor notes
%handout,instructornotes:  for instructor version with just problems and notes, no solutions
%noinstructornotes:  shows only problem and solutions

%% handout
%% space
%% newpage
%% numbers
%% nooutcomes

%I added the commands here so that I would't have to keep looking them up
%\newcommand{\RR}{\mathbb R}
%\renewcommand{\d}{\,d}
%\newcommand{\dd}[2][]{\frac{d #1}{d #2}}
%\renewcommand{\l}{\ell}
%\newcommand{\ddx}{\frac{d}{dx}}
%\everymath{\displaystyle}
%\newcommand{\dfn}{\textbf}
%\newcommand{\eval}[1]{\bigg[ #1 \bigg]}

%\begin{image}
%\includegraphics[trim= 170 420 250 180]{Figure1.pdf}
%\end{image}

%add a ``.'' below when used in a specific directory.

\newcommand{\RR}{\mathbb R}
\renewcommand{\d}{\,d}
\newcommand{\dd}[2][]{\frac{d #1}{d #2}}
\renewcommand{\l}{\ell}
\newcommand{\ddx}{\frac{d}{dx}}
\newcommand{\dfn}{\textbf}
\newcommand{\eval}[1]{\bigg[ #1 \bigg]}




\author{Tom Needham and Jim Talamo}

\outcome{Use the divergence test to conclude that a series diverges.}
\outcome{Use the integral test to determine whether a series converges or diverges.}

\title[]{The Integral Test and $p$-series}

\begin{document}
\begin{abstract}
\end{abstract}
\maketitle

\vspace{-0.9in}

\section{Discussion Questions}

\begin{problem}
For which of the following series can the integral test be applied?  If the integral test cannot be applied, state which assumption is not met.

\begin{center}
\begin{tabular}{lll}
I. $\sum_{k=1}^{\infty} k \sin(k^2)$ \hspace{10mm} & II. $\sum_{k=1}^{\infty} e^{-k^2}$ \hspace{10mm} & III.  $\sum_{k=3}^{\infty} \frac{k}{k^2+4}$
\end{tabular}
\end{center}

\begin{freeResponse}
I. The integral test cannot be applied since the function $f(x) = x \sin\left(x^2\right)$ is neither eventually positive nor eventually decreasing.

II. The integral test can be applied since the function $f(x) = e^{-x^2}$ is continuous, positive, and decreasing for $x \geq 1$.

III. The integral test can be applied since the function $f(x) = \frac{x}{x^2+1}$ is continuous, positive, and eventually decreasing. To show that the function is decreasing, note that 

\[
f'(x) = \frac{x^2+1 - 2x^2}{\left(x^2+1\right)^2} = -\frac{-x^2+1}{\left(x^2+1\right)^2}.
\]
So, $f'(x)<0$ for $x \geq 1$, so $f(x)$ is decreasing for all $x >1$.


\end{freeResponse}
\end{problem}

\begin{problem}
Which of the following are $p$-series?

\begin{center}
\begin{tabular}{llll}
I. $\sum_{k=1}^{\infty} \frac{(-1)^k}{k^3}$ \hspace{5mm} & II. $\sum_{k=1}^{\infty} \frac{2}{\sqrt[3]{k}}$ \hspace{5mm} & III.  $\sum_{k=1}^{\infty} \frac{2^k}{k^3}$ \hspace{5mm} & IV. $\sum_{p=1}^{\infty} \left(\frac{1}{3}\right)^p$
\end{tabular}
\end{center}

\begin{freeResponse}
The only $p$-series is II.
\end{freeResponse}
\end{problem}

\section{Group Work}

\begin{problem}
Determine whether the series
$$
\sum_{k=2}^\infty \frac{1}{k \ln (k)}
$$
converges or diverges.

\begin{freeResponse}
Since 
$$
\lim_{k \rightarrow \infty} \frac{1}{k \ln (k)} = 0,
$$
The Divergence Test is inconclusive, and we will attempt to solve the problem via The Integral Test. We compute the improper integral
\begin{align*}
\int_2^\infty \frac{1}{x \ln (x)} \d x
\end{align*}
by making the substitution $u = \ln (x)$, so that $\d u = \frac{1}{x} \d x$, so that 
\begin{align*}
\int_2^\infty \frac{1}{x \ln (x)} \d x &= \lim_{b \rightarrow \infty} \int_{\ln (2)}^b \frac{1}{u} \d u \\
&= \lim_{b \rightarrow \infty} \eval{\ln(u)}_{\ln(2)}^b \\
&= \lim_{b \rightarrow \infty} \ln(b) - \ln 2 \\
&= \infty.
\end{align*}
The Integral Test therefore implies that the series diverges.
\end{freeResponse}
\end{problem}

\begin{problem}
For each part, determine which of the following properties each of the given sequences $\{a_n\}_{n=1}$ and $\{s_n\}_{n=1}$ has: monotone increasing, monotone decreasing, bounded above, bounded below. 

I. Let $\{a_k\}_{k=1}^\infty$ be the sequence defined by $a_k = \frac{1}{k^2+1}$ and let $\{s_n\}_{n=1}^\infty$ denote its sequence of partial sums. 

II. Let $\{a_k\}_{k=1}^\infty$ be the sequence defined by $a_k = \frac{k}{e^k}$ and let $\{s_n\}_{n=1}^\infty$ denote its sequence of partial sums.

\begin{freeResponse}
I. The sequence $\{a_k\}$ is monotone decreasing and bounded (above and below), by inspection. Since the terms of $\{a_k\}$ are strictly positive, the sequence $\{s_n\}$ must be monotone increasing. It is straightforward to show via the Integral Test that the series $\sum \frac{1}{k^3}$ converges. It follows by the definition of convergence that the sequence $\{s_n\}$ has a finite limit and is therefore bounded.

II. The sequence $\{a_k\}$ is monotone increasing and bounded, with limit $\lim_{k\rightarrow \infty} a_k = 1$. The Divergence Test implies that the series $\sum a_k$ diverges. It follows that the sequence $\{s_n\}$ is monotone increasing and bounded below (because the terms of $a_k$ are positive), but not bounded above (because $\lim_{n \rightarrow \infty} s_n = \infty$). 
\end{freeResponse}
\end{problem}

\end{document}
