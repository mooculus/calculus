\documentclass{ximera}

\newcommand{\RR}{\mathbb R}
\renewcommand{\d}{\,d}
\newcommand{\dd}[2][]{\frac{d #1}{d #2}}
\renewcommand{\l}{\ell}
\newcommand{\ddx}{\frac{d}{dx}}
\newcommand{\dfn}{\textbf}
\newcommand{\eval}[1]{\bigg[ #1 \bigg]}


\author{Jim Talamo}
\license{CC-By-SA-NC}


\outcome{Apply the multivariable chain rule.}


\begin{document}
\begin{exercise}

The following exercise explores an application of the multivariable chain rule in the context of a well-known fact from physics.

Take $h=0$ to be the height associated to the ground, and suppose that a ball is dropped from an initial height of $y=h_0$ .  Initially, the ball has only potential energy, which depends on the height of the ball only and is given by 

\[
PE(y) = mgy,
\]
where $m$ is the mass of the ball, $g$ is the acceleration due to gravity, and $y$ is the height of the ball.

As the ball falls, its velocity increases, and it gains kinetic energy, which depends only on the velocity of the ball and is given by

\[
KE(v) = \frac{1}{2}mv^2.
\]

The total energy of the ball during its fall is given by 

\[
E=E(y,v) = \frac{1}{2} mv^2 +mgy.
\]

Now, as the ball falls, $v$ and $y$ will depend on time, so we may think of them as functions of time and write $v(t)$ and $y(t)$ to make this dependence explicit.  Note that as the ball continues to fall, $v$ \wordChoice{\choice[correct]{increases}\choice{decreases}} and $y$  \wordChoice{\choice{increases}\choice[correct]{decreases}}.



We can thus think of the total energy as a function of time and write

\[
E\big(v(t),h(t)\big)) = \frac{1}{2} m \big[v(t)\big]^2 +mg \big[y(t)\big].
\]

We can use the chain rule to compute $\dd[E]{t}$.

Select the correct expression for $\dd[E]{t}$.

\begin{multipleChoice}
\choice{$\dd[E]{t} = mv+mg \dd[y]{t}$}
\choice{$\dd[E]{t} = \dd[E]{v} \dd[v]{t} + \dd[E]{y} \dd[y]{t}$.}
\choice[correct]{$\dd[E]{t} = \pp[E]{v} \dd[v]{t} + \pp[E]{y} \dd[y]{t}$.}
\end{multipleChoice}

\begin{exercise}
Note that $\pp[E]{v} = \answer{mv}$ and $\pp[E]{y} = \answer{mg}$, so the chain rule tells us that 

\[
\dd[E]{t} = mv \dd[v]{t} + mg \dd[y]{t}.
\]

In the absence of air resistance or any other external forces, the work-energy theorem guarantees that the total energy of the ball is conserved; thus $E(t)$ is constant although the amount of kinetic and potential energy changes in time.

Since $E(t)$ is constant in time, we must have that $\dd[E]{t} = \answer{0}$ since the derivative of a constant is $\answer{0}$.  

\begin{exercise}
Substituting this gives

\[
0 = mv \dd[v]{t} + mg \dd[y]{t}.
\]

Writing $\dd[v]{t}=a(t)$ and $\dd[y]{t} = v(t)$ gives 

\[
0 = mv(t) \cdot a(t) + mg \cdot v(t) = v(t) \cdot \big[ ma(t) +mg \big],
\]

from which we conclude for any time $t$ during the fall, $ma(t) = -mg$.

This should not be surprising; the only force acting on the ball is gravity, and this force is directed downwards.  Hence, Newton's second law $F_{net} = ma$ produces the result

\[
ma = -mg,
\]

which s exactly what we found before!
\end{exercise}
  
\end{exercise}
\end{exercise}
\end{document}
