\documentclass{ximera}

\newcommand{\RR}{\mathbb R}
\renewcommand{\d}{\,d}
\newcommand{\dd}[2][]{\frac{d #1}{d #2}}
\renewcommand{\l}{\ell}
\newcommand{\ddx}{\frac{d}{dx}}
\newcommand{\dfn}{\textbf}
\newcommand{\eval}[1]{\bigg[ #1 \bigg]}


%\outcome{Find tangent lines to parametric curves}
\author{Jim Talamo }

\begin{document}
\begin{exercise}

Select all of the following that provide an alternate description for the polar coordinates $(r, \theta) = \left(3,\frac{\pi}{3}\right)$:

\begin{selectAll}
\choice[correct]{$(r, \theta) = \left(3,-\frac{5\pi}{3} \right)$}
\choice{$(r, \theta) = \left(-3,\frac{5\pi}{3} \right)$}
\choice[correct]{$(r, \theta) = \left(-3,-\frac{2\pi}{3} \right)$}
\choice{$(r, \theta) = \left(-3,-\frac{5\pi}{3} \right)$}
\choice[correct]{$(r, \theta) = \left(3,\frac{7\pi}{3} \right)$}
\choice[correct]{$(r, \theta) = \left(-3,-\frac{4\pi}{3} \right)$}
\end{selectAll}

\begin{hint}
One way to do this is to convert all of the points to Cartesian coordinates.  A better way is to remember that to graph a point in polar coordinates:

\begin{itemize}
\item If $r>0$, start along the positive $x$-axis.
\item If $r<0$, start along the negative $x$-axis.
\item If $\theta>0$, rotate counterclockwise.
\item If $\theta<0$, rotate clockwise.
\end{itemize}
\end{hint}
\end{exercise}
\end{document}