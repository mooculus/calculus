\documentclass[]{ximera}
%handout:  for handout version with no solutions or instructor notes
%handout,instructornotes:  for instructor version with just problems and notes, no solutions
%noinstructornotes:  shows only problem and solutions

%% handout
%% space
%% newpage
%% numbers
%% nooutcomes

%I added the commands here so that I would't have to keep looking them up
%\newcommand{\RR}{\mathbb R}
%\renewcommand{\d}{\,d}
%\newcommand{\dd}[2][]{\frac{d #1}{d #2}}
%\renewcommand{\l}{\ell}
%\newcommand{\ddx}{\frac{d}{dx}}
%\everymath{\displaystyle}
%\newcommand{\dfn}{\textbf}
%\newcommand{\eval}[1]{\bigg[ #1 \bigg]}

%\begin{image}
%\includegraphics[trim= 170 420 250 180]{Figure1.pdf}
%\end{image}

%add a ``.'' below when used in a specific directory.

\newcommand{\RR}{\mathbb R}
\renewcommand{\d}{\,d}
\newcommand{\dd}[2][]{\frac{d #1}{d #2}}
\renewcommand{\l}{\ell}
\newcommand{\ddx}{\frac{d}{dx}}
\newcommand{\dfn}{\textbf}
\newcommand{\eval}[1]{\bigg[ #1 \bigg]}




\author{Tom Needham}

\outcome{Use the ratio test to determine whether a series converges or diverges.}
\outcome{Use the root test to determine whether a series converges or diverges.}

\title[]{The Ratio and Root Tests}

\begin{document}
\begin{abstract}
\end{abstract}
\maketitle

\vspace{-0.5in}

\section{Discussion Questions}

\begin{problem}
Suppose that we want to determine whether the series
$$
\sum_{n=1}^\infty \frac{n^2+6n}{5n^3 + 2n^2 + 1}
$$
converges or diverges. Would either the Ratio Test or the Root Test be a useful tool for this task?

\begin{solution}
We can tell immediately that neither the Ratio Test nor the Root Test would be conclusive for this example. This follows because, for any polynomials $p(n)$ and $q(n)$, it holds that
$$
\lim_{n\rightarrow \infty} \frac{p(n+1)/q(n+1)}{p(n)/q(n)} = 1
$$
and 
$$
\lim_{n\rightarrow \infty} \sqrt[n]{p(n)} = 1.
$$
\end{solution}
\end{problem}

\begin{problem}
Suppose that $\{a_k\}$ is a sequence with positive terms and 
$$
\lim_{k \rightarrow \infty} \frac{a_{k+1}}{a_k} = \frac{1}{2}.
$$
Is it possible to determine whether the following series converge or diverge?
\begin{center}
\begin{tabular}{ll}
I. $\sum_{k=1}^\infty a_k$ \hspace{1in} II. $\sum_{k=1}^\infty \frac{a_{k+1}}{a_k}$.
\end{tabular}
\end{center}
A student claims that $\sum_{k=1}^\infty a_k$ converges to $\frac{1}{2}$. Is the student correct?

\begin{solution}
The sequence in I. converges by the Ratio Test and the sequence in II. diverges by the Divergence Test. The student is not correct; the Ratio Test can only tell us that a series converges, not what it converges to. To give a concrete counterexample to the student's claim, consider the series 
$$
\sum_{k=1}^\infty \frac{1}{2^k}.
$$
The terms $a_k = 1/2^k$ of this series satisfy $\lim_{k \rightarrow \infty} \frac{a_{k+1}}{a_k} = \frac{1}{2}$, but in this case we can determine the value of the series exactly:
$$
\sum_{k=1}^\infty \frac{1}{2^k} = \sum_{k=0}^\infty \frac{1}{2^k} - 1 = \frac{1}{1-1/2} - 1 = 1 \neq \frac{1}{2}.
$$
\end{solution}
\end{problem}

\begin{problem}
Suppose that the sequence $\{a_k\}_{k=1}^\infty$ has sequence of partial sums $\{s_n\}_{n=1}^\infty$ given by the formula
$$
s_n = \frac{3^n}{n!}.
$$
Determine the convergence properties of the series 
$$
\sum_{n=1}^\infty s_n \;\; \mbox{ and } \;\; \sum_{k=1}^\infty a_k. 
$$
The possible answers for each case are: ``converges to a known value", ``converges to some value that we are unable to determine", or ``diverges".

\begin{solution}
By definition,
$$
\sum_{k=1}^\infty a_k = \lim_{n\rightarrow \infty} s_n = \lim_{n\rightarrow \infty} \frac{3^n}{n!} = 0,
$$
where the limit is computed by comparing growth rates. That is, the sequence $\sum a_k$ converges to a known value.

The series $\sum s_n$ converges, by the Ratio Test, since
\begin{align*}
\lim_{n \rightarrow \infty} \frac{s_{n+1}}{s_n} &= \lim_{n \rightarrow \infty} \frac{3^{n+1}/(n+1)!}{3^n}{n!} \\
&= \lim_{n \rightarrow \infty} \frac{3^{n+1}}{(n+1)!} \frac{n!}{3^n} \\
&= \lim_{n \rightarrow \infty} \frac{3}{n+1} \\
&= 0.
\end{align*}
That is, the series $\sum s_n$ converges, but we don't have the tools necessary to determine its value. 
\end{solution}
\end{problem}

\section{Group Work}

\begin{problem}
Let $\{a_k\}_{k=1}^\infty$ be the sequence defined by $a_k = \frac{k^5}{k^k}$ and let $\{s_n\}$ denote its sequence of partial sums. Determine whether the respective sequences $\{a_k\}$ and $\{s_n\}$ are bounded and/or monotonic.

\begin{solution}
The first few terms of $a_k$ are $1, 2^5/2^2 = 8, 3^5/3^3 = 9, 4^5/4^4 = 4$, so we see that the sequence is neither monotonically increasing nor decreasing (although the sequence will ``eventually" become monotonically decreasing). It does have a limit (which can be seen by comparing growth rates), and it follows that the sequence is bounded. The terms $a_k$ are all positive, so the sequence $\{s_n\}$ is monotonically increasing. We claim that the sequence $\{s_n\}$ is bounded. To see this, we will show that the series $\sum a_k$ converges. This follows by the Root Test, as 
$$
\lim_{k\rightarrow \infty} \sqrt[k]{\frac{k^5}{k^k}} = \lim_{k\rightarrow \infty} \frac{(k^{1/k})^5}{k}.
$$
We have previously seen that $\lim_{k\rightarrow \infty} k^{1/k} = 1$, so we have 
$$
\lim_{k\rightarrow \infty} \frac{(k^{1/k})^5}{k} = 0 < 1.
$$
\end{solution}
\end{problem}


\begin{problem}
Determine whether the following series converge or diverge.
\begin{center}
\begin{tabular}{lll}
I. $\sum_{n=1}^\infty \left(\frac{3n^2+2}{n^2-10}\right)^{2n}$ \hspace{.3in} II. $\sum_{n=1}^\infty \frac{2n^2+10}{n+3n^2}$ \hspace{.3in} III. $\sum_{n=1}^\infty \frac{(n!)^2}{(2n)!}$
\end{tabular}
\end{center}

\begin{solution}
I. Since each term in the series is raised to the power $2n$, this series is amenable to the Root Test. We have 
$$
\lim_{n \rightarrow \infty} \sqrt[n]{\left(\frac{3n^2+2}{n^2-10}\right)^{2n}} = \lim_{n \rightarrow \infty} \left(\frac{3n^2+2}{n^2-10}\right)^{2} = \lim_{n \rightarrow \infty} \frac{9 n^4}{n^4} = 9 > 1,
$$
where the second equality follows by identifying the dominant terms in the numerator and denominator. It follows that the series diverges.

II. This series diverges by the Divergence Test. Remember that the first step in determining whether a series converges or diverges should be to quickly check whether the Divergence Test is applicable.

III. Because the terms in the series involve factorials, the Ratio Test should be useful here. We have 
\begin{align*}
\lim_{n \rightarrow \infty} \frac{((n+1)!)^2}{(2(n+1))!} \frac{(2n)!}{(n!)^2} &= \lim_{n \rightarrow \infty} \frac{(n+1)! \cdot (n+1)!}{(2n+2)!} \frac{(2n)!}{n! \cdot n!} \\
&= \lim_{n \rightarrow \infty} \frac{(2n)!}{(2n+2)(2n+1)(2n)!} \frac{(n+1)! \cdot (n+1)!}{n! \cdot n!} \\
&= \lim_{n \rightarrow \infty} \frac{1}{(2n+2)(2n+1)} \frac{(n+1)^2}{1} \\
&= \lim_{n \rightarrow \infty} \frac{n^2}{4n^2} \\
&= \frac{1}{4} < 1.
\end{align*}
It follows that the series converges.
\end{solution}
\end{problem}


\end{document}
