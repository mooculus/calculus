\documentclass{ximera}

\newcommand{\RR}{\mathbb R}
\renewcommand{\d}{\,d}
\newcommand{\dd}[2][]{\frac{d #1}{d #2}}
\renewcommand{\l}{\ell}
\newcommand{\ddx}{\frac{d}{dx}}
\newcommand{\dfn}{\textbf}
\newcommand{\eval}[1]{\bigg[ #1 \bigg]}


\author{Jim Talamo}
\license{Creative Commons 3.0 By-NC}


\outcome{Understand the relationship between sequences, sequences of partial sums, and remainders.}

\begin{document}

\begin{exercise}

Suppose that $\{a_n\}_{n=1}$ is a sequence and it is known that $s_n = \frac{2n^2+4}{n^2}$.

Is it possible to define remainders for $\sum_{k=1}^{\infty} a_k$?

\begin{multipleChoice}
\choice[correct]{Yes, because $\sum_{k=1}^{\infty} a_k$ converges.}
\choice{No, because $\sum_{k=1}^{\infty} a_k$ diverges.}
\end{multipleChoice}

\begin{exercise}
Calculate $\sum_{k=1}^{\infty} a_k$, then fill in the table below.

We find that $\sum_{k=1}^{\infty} a_k = \answer{2}$.

\begin{hint}
Recall that $\sum_{k=1}^{\infty} a_k$ and $\lim_{n \to \infty} s_n$ represent the same quantity!
\end{hint}

\begin{center}
\begin{tabular}{c | c | c | c | c }
n& $1$ & $2$ & $3$ & $4$ \\ [2 ex]
\hline
$a_n$ & $ \answer{6}$ &$ \answer{-3}$ & $ \answer{-\frac{5}{9}}$ & $ \answer{-\frac{7}{36}}$  \\ [2 ex]
\hline
$s_n$ & $ \answer{6}$ &$ \answer{3}$ & $ \answer{\frac{22}{9}}$ & $ \answer{\frac{9}{4}}$  \\ [2 ex]
\hline
$r_n$ & $ \answer{-4}$ & $ \answer{-1}$ & $ \answer{-\frac{4}{9}}$ & $ \answer{-\frac{1}{4}}$ 
\end{tabular}
\end{center}

\begin{hint}
To fill out the table, first compute $s_n$ from the given formula for $n=1,2,3,4$.  Then, use the fact that

\begin{itemize}
\item $a_n = s_n-s_{n-1}$ (since $s_n = a_1+a_2+\ldots+a_{n-1}+a_n = s_{n-1}+a_n$)
\item Since  $\sum_{k=1}^{\infty} a_k = \answer{2}$, we have that $2=s_n+r_n$ for every $n$.
\end{itemize}
\end{hint}

\end{exercise}
\end{exercise}
\end{document}