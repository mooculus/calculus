\documentclass{ximera}

\newcommand{\RR}{\mathbb R}
\renewcommand{\d}{\,d}
\newcommand{\dd}[2][]{\frac{d #1}{d #2}}
\renewcommand{\l}{\ell}
\newcommand{\ddx}{\frac{d}{dx}}
\newcommand{\dfn}{\textbf}
\newcommand{\eval}[1]{\bigg[ #1 \bigg]}


\outcome{Compute limits of families of functions.} 
\outcome{Compute average velocity.}
\outcome{Approximate instantaneous velocity.}
\outcome{Compare average and instantaneous velocity.}
\outcome{Plot difference quotients for varying approximations of the instantaneous rate of change.}

\title[Dig-In:]{Instantaneous velocity}

\begin{document}
\begin{abstract}
We use limits to compute instantaneous velocity.
\end{abstract}
\maketitle

When we compute average velocity, we look at 
\[
\frac{\text{change in position}}{\text{change in time}}.
\]
To obtain the (instantaneous) velocity, we want the change in time to
``go to'' zero. By this point we should know that ``go to'' is a
buzz-word for a \emph{limit}. The change in time is often given as
the length of an interval, and this length goes to zero.

The average velocity on the (time) interval $[a,b]$ is given by
\[
v_{\text{av}} = 
\frac{\text{change in position}}{\text{change in time}} =
\frac{s(b)-s(a)}{b-a}.
\]
Here $s(t)$ denotes the position, at the time $t$, of an object moving along a line.

Let's put all of this together by working an example.

\begin{example}
A young mathematician throws a ball straight into the air with 
a velocity of 40ft/sec. Its height (in feet) after $t$ seconds 
is given by
\[
s(t) = 40t-16t^2 \qquad\text{(feet above the ground)} .
\]

When will the ball hit the ground?

\begin{explanation}
To determine when the ball hits the ground we need to solve the
equation
\[
s(t)=0
\]
for t.  That is,
\begin{align*}
40t-16t^2 &= 0\\
t(40-16t) &= 0.
\end{align*}
This has solutions $t=0$
seconds and $t=\answer[given]{2.5}$ seconds.  Since the ball hits
the ground \wordChoice{\choice{before}\choice[correct]{after}} it's
thrown, we know that the ball hits the ground at $t=\answer[given]{2.5}$
seconds.
\end{explanation}

What is the height of the ball after $2$ seconds?

\begin{explanation}
To find the height of the ball after $2$ seconds we simply need 
to plug $2$ into the equation for $s(t)$ to find
\[
s(2) = 40\cdot\answer[given]{2} - 16\cdot\answer[given]{2}^2 = 
\answer[given]{16}\unit{ft.}
\]

\end{explanation}

Consider the following points lying along the $s$ axis.
\begin{image}
\begin{tikzpicture}
    \begin{axis}[
        ymin=-10.3,ymax=25.3,xmin=-20,xmax=20,
        clip=false,
        unit vector ratio*=1 1 1,
        axis lines=center,
        ytick={-10,-5,...,25},
        hide x axis,
        ylabel=$s$,
        every axis y label/.style={
          at={(ticklabel* cs:1)},
          anchor=south},
      ]
        \addplot[only marks,very thick,penColor,mark=*]
	        coordinates{(0,-6)};
	    \node at (axis cs:0,-6) [right] {$A$};
	    
        \addplot[only marks,very thick,penColor,mark=*]
	        coordinates{(0,0)};
	    \node at (axis cs:0,0) [right] {$B$};
	    
	    \addplot[only marks,very thick,penColor,mark=*]
	        coordinates{(0,40-16)};
	    \node at (axis cs:0,40-16) [right] {$D$};
	    
        \addplot[only marks,very thick,penColor,mark=*]
	        coordinates{(0,40*2-16*2^2)};
	        \node at (axis cs:0,40*2-16*2^2) [right] {$C$};
        \addplot[only marks,very thick,white,mark=*]
	coordinates{(19,0)};
        \addplot[only marks,very thick,white,mark=*]
	        coordinates{(-19,0)};
    \end{axis}`
\end{tikzpicture}
\end{image}

Which points correspond to the height of the ball at times $0$, $1$ and $2$? 
\begin{explanation}
The point that corresponds to $s(0)$, the position (height) of the 
ball at $t=0$, is \wordChoice{\choice{A}\choice[correct]{B}\choice{C}\choice{D}}.

The point that corresponds to $s(1)$, the position (height) of the 
ball at $t=1$, is \wordChoice{\choice{A}\choice{B}\choice{C}\choice[correct]{D}}.

The point that corresponds to $s(2)$, the position (height) of the 
ball at $t=2$, is \wordChoice{\choice{A}\choice{B}\choice[correct]{C}\choice{D}}.

\end{explanation}

Next let's consider the average velocity of the ball. What is the average velocity of the ball on the interval $[1,2]$?

\begin{explanation}

In order to find the average velocity of the ball on the 
interval $[1,2]$ we recall that the average velocity on 
the interval $[a,b]$ is given by
\[
v_{\text{av}} = 
\frac{s(b) - s\left(\answer[given]{a}\right)}
{\answer[given]{b} - \answer[given]{a}}.
\]
Plugging in $a=1$ and $b=2$ we find that 
\[
v_{\text{av}} = \frac{s(2)-s(1)}{2-1} = 
\frac{\answer[given]{16} - \answer[given]{24}}{1} =
\answer[given]{-8}\unit{ft}/\unit{s}.
\] 

\end{explanation}

What is the average velocity of the ball on the interval
$[t,2]$ for $0<t<2$?.  

\begin{explanation}

We use the same formula we used to find the average velocity on
the interval $[1,2]$ to find the average velocity on the interval
$[t,2]$ for $0<t<2$.
\begin{align*}
v_{\text{av}} &= 
\frac{s(2)-s\left(\answer[given]{t}\right)}
{\answer[given]{2}-\answer[given]{t}}\\
&= \frac{16 - (40t-16t^2)}{2-t}\\
&= \frac{8(2t^2-5t+2)}{2-t}\\
&= \frac{8(2t-1)(t-2)}{-(t-2)}\\
&= -8(2t-1)\\
&= (8-16t)\unit{ft}/\unit{s}
\end{align*}
for $0<t<2$.

\end{explanation}

What is the average velocity of the ball on the interval $[2,t]$ for
$2<t<2.5$?

\begin{explanation}

To calculate the average velocity on the interval $[2,t]$ for 
$2<t<2.5$ we will use our average velocity formula one more time
to find
\[
v_{\text{av}} = \frac{s(t)-s(2)}{t-2} = \frac{s(2)-s(t)}{2-t}. 
\]
However, this is exactly the same expression we got when calculating
the average velocity on the interval $[t,2]$ for $0<t<2$.  So the 
average velocity on the interval $[2,t]$ for $2<t<2.5$ is given by
\[
v_{\text{av}} =\answer[given]{(8-16t)}\unit{ft}/\unit{s}.
\]
\end{explanation}
\end{example}

In our previous example, we computed \textit{average velocity} on
several different intervals. If we let the size of the interval go to
zero, we get \dfn{instantaneous velocity}. Limits will allow us to
compute instantaneous velocity.  Let's use the same setting as before.

\begin{example}
The height of a ball above the ground between $0$ and $2.5$ seconds
is given by
\[
s(t) = 40t - 16t^2.
\] 
Find the instantaneous velocity of the ball $2$ seconds after it
is thrown.
\begin{explanation}
From the previous example, we know that the average velocity of the
ball on the interval $[t,2]$ for $0<t<2$ and the average velocity
on the interval $[2,t]$ for $2<t<2.5$ are both given by
\[
v_{\text{av}} =  \frac{s(t)-s(2)}{t-2}= \answer[given]{(8-16t)}\unit{ft}/\unit{s}.
\]
All we need to do to find the instantaneous velocity at $t=2$ is 
take the limit as $t$ goes to $\answer[given]{2}$ of the expression
above.  Doing so we find
\begin{align*}
v = \lim_{t\to2}v_{\text{av}}
=\lim_{t\to2}  \frac{s(t)-s(2)}{t-2}
&= \lim_{t\to2}(8-16t)\\
&= \answer[given]{-24}\unit{ft}/\unit{s}.
\end{align*}
\end{explanation}
\end{example}
\end{document}
