\documentclass{ximera}

\newcommand{\RR}{\mathbb R}
\renewcommand{\d}{\,d}
\newcommand{\dd}[2][]{\frac{d #1}{d #2}}
\renewcommand{\l}{\ell}
\newcommand{\ddx}{\frac{d}{dx}}
\newcommand{\dfn}{\textbf}
\newcommand{\eval}[1]{\bigg[ #1 \bigg]}


\outcome{}


\title[Break-Ground:]{From Galileo to Apollo}

\begin{document}

\begin{abstract}

\end{abstract}
\maketitle


In the 1600's, Galileo popularized the idea that falling bodies fell
at the same velocity regardless of their mass. While this is something
we know on an intellectual level, you also know from experience that
hammers fall quickly and feathers fall slowly. The reason for this is
\textit{air resistance}.  Let's take a minute to investiagte the
effects of air resistance. The velocity of a falling object with air
resistance can be modeled by
\[
v = \frac{mg(1-e^{-kt/m})}{k}\qquad\text{meters per second}
\]
where $m$ is the mass of the object in kilograms, $g$ is the
acceleration due to gravity in meter per second-squared, $t$ is the
amount of time in seconds that the object is falling, and $k$ is a
positive constant that will vary per object and represents the
resistance caused by the atmosphere.

\begin{problem}
Given that the velocity of a falling object in an atmosphere is 
\[
v = \frac{mg(1-e^{-kt/m})}{k}\qquad\text{meters per second,}
\]
what happens as $t$ goes to infinity?
\begin{hint}
Recall that $e^{-x}$ goes to zero as $x$ goes to infinity.
\end{hint}
\begin{prompt}
As $t$ goes to infinity, $v$ goes to \answer{m*g/k}.
\end{prompt}
\end{problem}


Now suppose we want the effect of the air resistance to go to
zero. Remember this is controlled by the coefficient $k$, so we want
$k$ to go to zero. This amounts to computing the following limit
\[
\lim_{k\to 0+} \frac{mg(1-e^{-kt/m})}{k}.
\]
However, this limit is not so easy to compute!

\begin{problem}
Consider the following video: \youtube{http://youtu.be/TVAiVVxD-ng}
Based on the video, can you guess the value of:
\[
\lim_{k\to 0+} \frac{mg(1-e^{-kt/m})}{k}
\]
must be?
\begin{hint}
First note there is no air resistance on the Moon, and hence $k$ is
zero.
\end{hint}
\begin{hint}
Next note that both the hammer and the feather fall at the same rate.
Hence the mass of the falling object must not affect its velocity.
\end{hint}
\begin{prompt}
$\lim_{k\to 0+}\frac{mg(1-e^{-kt/m})}{k} = $\answer{g*t}.
\end{prompt}
\end{problem}

\begin{xarmaBoost}
  Write down at least \textbf{five} questions for this lecture. After
  you have your questions, label them as ``Level 1,'' ``Level 2,'' or
  ``Level 3'' where:
\begin{description}
\item[Level 1] Means you know the answer, or know exactly how to do
  this problem.
\item[Level 2] Means you think you know how to do the problem, or will
  soon learn how to do the problem.
\item[Level 3] Means you have no idea how to do the problem.
\end{description}
\begin{freeResponse}
\end{freeResponse}
\end{xarmaBoost}

\end{document}
