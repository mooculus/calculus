\documentclass{ximera}
\newcommand{\RR}{\mathbb R}
\renewcommand{\d}{\,d}
\newcommand{\dd}[2][]{\frac{d #1}{d #2}}
\renewcommand{\l}{\ell}
\newcommand{\ddx}{\frac{d}{dx}}
\newcommand{\dfn}{\textbf}
\newcommand{\eval}[1]{\bigg[ #1 \bigg]}

\author{Bart Snapp}
\license{Creative Commons 3.0 By-NC}
\acknowledgement{https://www.whitman.edu/mathematics/calculus/}
  \outcome{Interpret an optimization problem as the procedure used to make a system or design as effective or functional as possible.}
  \outcome{Set up an optimization problem by identifying the objective function and appropriate constraints.}
  \outcome{Solve optimization problems by finding the appropriate absolute extremum.}
  \outcome{Solve basic word problems involving maxima or minima.}
\begin{document}
\begin{exercise}

  You want to make cylindrical containers to hold 1 liter (1000$\text{cm}^3$) using the
  least amount of construction material.  The side is made from a
  rectangular piece of material, and this can be done with no material
  wasted.  However, the top and bottom are cut from squares of side
  $2r$, so that $2(2r)^2=8r^2$ of material is needed (rather than
  $2\pi r^2$, which is the total area of the top and bottom).  Find
  the dimensions of the container using the least amount of material.
  \begin{prompt}
  \[
  \text{radius}=\answer{5}\text{cm}\qquad\text{height}=\answer{40/\pi}\text{cm}
  \]
  \end{prompt}
\end{exercise}
\end{document}
