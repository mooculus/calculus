\documentclass{ximera}

\newcommand{\RR}{\mathbb R}
\renewcommand{\d}{\,d}
\newcommand{\dd}[2][]{\frac{d #1}{d #2}}
\renewcommand{\l}{\ell}
\newcommand{\ddx}{\frac{d}{dx}}
\newcommand{\dfn}{\textbf}
\newcommand{\eval}[1]{\bigg[ #1 \bigg]}


\author{Jim Talamo}
\license{Creative Commons 3.0 By-NC}


\outcome{Understand remainder estimates for alternating series.}

\begin{document}

\begin{problem}
A student is asked to determine $\sum_{k=1}^{\infty} \frac{(-1)^k}{\sqrt{k}}$ to within $.001$ of its actual value.  The student correctly uses computational software to compute $\sum_{k=1}^{10000} \frac{(-1)^k}{\sqrt{k}} = -.5999$ to four decimal places and claims that the answer is sufficiently close.  Evaluate the students' line of reasoning.

\begin{freeResponse}
First, note that the series $\sum_{k=1}^{\infty} \frac{(-1)^k}{\sqrt{k}}$ will converge by the alternating series test.  As such, the series \emph{has} a value, and we can now go about trying to approximate it.  

First, recall the important result below.

\begin{theorem}[Alternating Series Remainder Estimates]
If $\{a_n\}_{n=n_0}$ be a sequence whose terms are positive and nonincreasing  and
$\lim_{n\to\infty} a_n=0$. Then,  
\[
\big| r_n \big| \leq a_{n+1} \qquad \textrm{ for all } n \geq n_0,
\]
where $r_n = \sum_{k=n+1}^{\infty} a_k$.
\end{theorem}

According to the above, the student should only expect that the error in their approximation to be given by

\[
\big|r_{10000}\big| < a_{10001} = \frac{1}{\sqrt{10001}} \approx .0099 \quad \textrm{ (to four decimal places). }
\]

The student therefore should use more terms to be confident that the obtained approximation is within $.001$ of the actual value of the series. 
\end{freeResponse}

\end{problem}

\end{document}