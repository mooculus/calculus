\documentclass{ximera}

\newcommand{\RR}{\mathbb R}
\renewcommand{\d}{\,d}
\newcommand{\dd}[2][]{\frac{d #1}{d #2}}
\renewcommand{\l}{\ell}
\newcommand{\ddx}{\frac{d}{dx}}
\newcommand{\dfn}{\textbf}
\newcommand{\eval}[1]{\bigg[ #1 \bigg]}


\author{Jim Talamo}
\license{CC-By-SA-NC}


\outcome{Calculate a directional derivative.}


\begin{document}
\begin{exercise}
The following exercises serves as a guided walkthrough for computing the rate of change of a function in the direction of a given line.

Suppose that $f(x,y) = \cos(2x^2y) +3x-y^2+1$ and suppose that we want to find the rate of change of $f(x,y)$ at $(0,2)$ in the direction of $\uvec{u}$, where $\uvec{u}$ is a vector in the positive $x$-direction that is parallel to the line $x-2y=4$.

We want to use the result

\[
D_{\uvec{u}}f(a,b) = \grad{f}(a,b) \dotp \uvec{u}.
\]

To do so, we must calculate the gradient and find a unit vector $\uvec{u}$ in the direction of the line with positive $x$-component.

\begin{itemize}
\item To compute the gradient, note that $f(x,y) = \cos(2x^2y) +3x-y^2+1$.

\begin{align*}
f_x(x,y) &= \answer{-4xy \sin(2x^2y)+3} \textrm{ so } f_x(0,2) = \answer{3} \\
f_y(x,y) &= \answer{-2x^2\sin(2x^2y) -2y} \textrm{ so } f_y(0,2) = \answer{-4} \\
\end{align*}

Thus, $\grad{f}(0,2) = \vector{\answer{3},\answer{-4}}$.

\item To find the unit vector $\uvec{u}$, we first will find a vector that is in the appropriate direction, then scale it appropriately.

We can do this two different ways.  Both are presented below, and you can choose your favorite way in future problems.

\begin{itemize}
\item \textbf{Method 1}: Use the equation of the line to find a \emph{normal} vector. 

From the equation of the line $x-2y=4$, a \emph{normal} vector to the line is $\vector{1,-2}$.  To find a vector \emph{parallel} to the line, we can flip the components and negate one of them (which force the dot product of this vector and the original to be $0$).  Our options doing this are thus $\vec{u} = \vector{-2,1}$ or $\vec{u} = \vector{2,1}$.  The one that points in the appropriate direction is \wordChoice{\choice{$\vector{-2,1}$}\choice[correct]{$\vector{2,1}$}}.


\item \textbf{Method 2}: Parameterize the line, then extract a parallel vector. 

From the equation of the line $x-2y=4$, we can solve for $x$ easily to find $x= \answer{2y+4}$.  Setting $y(t) = t$, we find $x(t) = \answer{2t+4}$.  Thus, a parametric description of the line is

\[
\vec{l}(t) = \vector{\answer{2t+4}, t}.
\]

If we write this in the form $\vec{l}(t) = \vec{u}t+\vec{P}_0$, we can use $\vec{u}$ to find $\uvec{u}$.  We find that $\vec{u} = \vector{\answer{2},\answer{1}}$.
\end{itemize}

Hence, a \emph{unit} vector that points in the appropriate direction is thus 

\[
\uvec{u} = \vector{\answer{ \frac{2}{\sqrt{5}}}, \answer{\frac{1}{\sqrt{5}} }}.
\]

We now can use the formula $D_{\uvec{u}}f(a,b) = \grad{f}(a,b) \dotp \uvec{u}$ to compute $D_{\uvec{u}}f(2,0)$ and find

\[
D_{\uvec{u}}f(2,0) = \answer{\frac{2}{\sqrt{5}}}.
\]

From this, we see that at $(2,0)$, $f(x,y)$ \wordChoice{\choice[correct]{increases}\choice{decreases}\choice{does not change}} in the $\uvec{u}$-direction.
\end{itemize}

\end{exercise}
\end{document}
