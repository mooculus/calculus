\documentclass{ximera}

\newcommand{\RR}{\mathbb R}
\renewcommand{\d}{\,d}
\newcommand{\dd}[2][]{\frac{d #1}{d #2}}
\renewcommand{\l}{\ell}
\newcommand{\ddx}{\frac{d}{dx}}
\newcommand{\dfn}{\textbf}
\newcommand{\eval}[1]{\bigg[ #1 \bigg]}


\author{Jim Talamo}
\license{Creative Commons 3.0 By-bC}


\outcome{}


\begin{document}
\begin{exercise}
Consider the power series $\sum_{k=0}^{\infty} \frac{(-1)^k 5^k }{2k+1}x^{2k+1}$.

We want to determine the radius and interval of convergence for this power series. 

First, we use the Ratio Test to determine the radius of convergence. 

We need to determine the limit:

\[
L(x) = \lim_{k \to \infty} \left| \answer{\frac{5 (2k+1)}{2k+3}x^2} \right| = 5|x|^2 \lim_{k \to \infty}  \answer{\frac{2k+1}{2k+3}}
\]

Evaluating this limit gives:

\[
L(x) = 5|x|^2 \cdot \answer{1}
\]

\begin{hint}

To do this, we'll think of the power series as a sum of functions of $x$ by writing: 

\[
\sum_{k=0}^{\infty} \frac{(-1)^k 5^k }{2k+1}x^{2k+1} = \sum_{k=0}^{\infty} b_k(x)
\]

We need to determine the limit $L(x) = \lim_{k \to \infty} \left| \frac{b_{k+1}(x)}{b_k(x)}\right|$, where we have explicitly indicated here that this limit likely depends on the $x$-value we choose. 



We calculate $b_{k+1}(x)=\answer{\frac{(-1)^{k+1} 5^{k+1}}{2(k+1)+1}} \cdot x^{\answer{2k+3}}$ and $b_k(x)=\answer{ \frac{(-1)^k 5^k}{2k+1}} \cdot x^{\answer{2k+1}}$. 

Simplifying the ratio $\left|\frac{b_{k+1}(x)}{b_k(x)}\right|$ gives us $\left|\frac{b_{k+1}}{b_k}\right|=\answer{ \frac{5 (2k+1)}{2k+3}x^2}$. 
\end{hint}





\begin{exercise}

The radius of convergence for the power series is $\answer{\frac{1}{\sqrt{5}}}$.


\begin{hint}

Recall the Ratio Test tells us that a series $\sum^{\infty}_{k=1} b_k$ converges if $L <1$ where $L=\lim_{k \to \infty}\left| \frac{b_{k+1}}{b_k}\right|$. 

Thus, in order to determine the set of $x$ for which our power series converges, we need to determine those $x$ which satisfy 
the inequality $5|x|^2 <\answer{1}$. 

Rewriting this inequality we obtain $|x|^2< \answer{\frac{1}{5}}$. 

We can take the square root of both sides of the inequality to obtain $|x|<\answer{\frac{1}{\sqrt{5}}}$. Note that taking the square root preserves the inequality due to the fact that the square root function is an increasing function.

\end{hint}








\begin{exercise}

The lefthand endpoint of the interval of convergence is $x=\answer{\frac{-1}{\sqrt{5}}}$ and the righthand endpoint is $x= \answer{\frac{1}{\sqrt{5}}}$ and the interval of convergence of the power series $\sum_{k=0}^{\infty} \frac{(-1)^k 5^k }{2k+1}x^{2k+1}$ is:
\begin{multipleChoice}
\choice{$\left(-\frac{1}{\sqrt{5}},\frac{1}{\sqrt{5}}\right)$}
\choice{$\left(-\frac{1}{\sqrt{5}},\frac{1}{\sqrt{5}}\right]$}
\choice{$\left[-\frac{1}{\sqrt{5}},\frac{1}{\sqrt{5}}\right)$}
\choice[correct]{$\left[-\frac{1}{\sqrt{5}},\frac{1}{\sqrt{5}}\right]$}
\end{multipleChoice}


\begin{hint}



 Now we need to determine if our power series converges at the endpoints $x=\frac{-1}{\sqrt{5}}$ and $x=\frac{1}{\sqrt{5}}$. 

First we test convergence at $x=\frac{-1}{\sqrt{5}}$. 

We obtain the series

\[
\sum^{\infty}_{k=0} \answer{\frac{(-1)^{k+1}}{(2k+1)\sqrt{5}}}
\]


This series \wordChoice{\choice[correct]{converges}\choice{diverges}} by the 

\begin{multipleChoice}
\choice{comparison test}
\choice{ratio test}
\choice{root test}
\choice{integral test}
\choice[correct]{alternating series test}
\choice{geometric series}
\end{multipleChoice}


Now we test convergence at the point  $x=\frac{1}{\sqrt{5}}$. We obtain the series

\[
\sum^{\infty}_{k=0} \frac{(-1)^k}{\sqrt{5}(2k+1)}
\]


This series \wordChoice{\choice[correct]{converges}\choice{diverges}} by the 

\begin{multipleChoice}
\choice{comparison test}
\choice{ratio test}
\choice{root test}
\choice{integral test}
\choice[correct]{alternating series test}
\choice{geometric series}
\end{multipleChoice}

\end{hint}




\end{exercise}
\end{exercise}
\end{exercise}
\end{document}
