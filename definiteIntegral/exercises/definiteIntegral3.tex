\documentclass{ximera}

\newcommand{\RR}{\mathbb R}
\renewcommand{\d}{\,d}
\newcommand{\dd}[2][]{\frac{d #1}{d #2}}
\renewcommand{\l}{\ell}
\newcommand{\ddx}{\frac{d}{dx}}
\newcommand{\dfn}{\textbf}
\newcommand{\eval}[1]{\bigg[ #1 \bigg]}



\outcome{Use integral notation for both antiderivatives and definite integrals.}
\outcome{Compute definite integrals using geometry.}
\outcome{Compute definite integrals using the properties of integrals.}
%\outcome{Justify the properties of definite integrals using algebra or geometry.}
%\outcome{Understand how Riemann sums are used to find exact area.}
%\outcome{Define net area.}
%\outcome{Approximate net area.}
\outcome{Split the area under a curve into several pieces to aid with calculations.}
\outcome{Use symmetry to calculate definite integrals.}
\outcome{Explain geometrically why symmetry of a function simplifies calculation of some definite integrals.}

\author{Nela Lakos \and Kyle Parsons}

\begin{document}
\begin{exercise}

Given that $\int_{-2}^1 f(x) \d x = -7$ and $\int_1^3 f(x) \d x = 5$ and that

 $f(x) < 0$ on $[-2,1]$ and $f(x) > 0$ on $[1,3]$,  compute the following integrals or write \verb|NEI| if there is not enough information to compute the integral.

\begin{align*}
\int_{-2}^3 f(x) \d x &= \answer{-2}\\
\int_{-2}^3 \left|f(x)\right| \d x &= \answer{12}\\
\int_{-2}^2 f(x) \d x + \int_2^3 f(x) \d x &= \answer{-2}\\
\int_2^2 f(x) \d x &= \answer{0}\\
\int_{-2}^2 \left|f(x)\right| \d x &= \answer{NEI}
\end{align*}

\end{exercise}
\end{document}