\documentclass{ximera}

\newcommand{\RR}{\mathbb R}
\renewcommand{\d}{\,d}
\newcommand{\dd}[2][]{\frac{d #1}{d #2}}
\renewcommand{\l}{\ell}
\newcommand{\ddx}{\frac{d}{dx}}
\newcommand{\dfn}{\textbf}
\newcommand{\eval}[1]{\bigg[ #1 \bigg]}


\author{Gregory Hartman \and Matthew Carr}
\license{Creative Commons 3.0 By-NC}
\acknowledgement{https://github.com/APEXCalculus}
\outcome{Write equations representing lines in space.}
\begin{document}
\begin{exercise}
Find the equation representing the line orthogonal to $\vec{d}_{1}=\vector{1,0,1}$ and $\vec{d}_{2}=\vector{2,0,3}$ that passes through $P=(5,1,9)$. Express your answer as a vector and as parametric equations.

\begin{prompt}
\[
\text{Vector:  } \vecl(t)=\vector{\answer{5},\answer{1},\answer{9}}+t\vector{\answer{0},-1,\answer{0}}
\]
\end{prompt}
\begin{prompt}
\[
\text{Parametric:  } x(t)=\answer{5},\ y(t)=\answer{1-t},\ z(t)=\answer{9}
\]
\end{prompt}

\begin{hint}
  Let's first find the direction that the line is heading in.
\end{hint}

\begin{hint}
  We are told that the line is orthogonal to $\vec{d}_{1}=\vector{1,0,1}$ and $\vec{d}_{2}=\vector{2,0,3}$.  We can use the cross product to find a vector orthogonal to these two vectors.
\end{hint}

\begin{hint}
  $\vector{1,0,1} \cross \vector{2,0,3} = \vector{0,-1,0}$.
\end{hint}

\begin{hint}
   So $\vecl(t) = \mbox{point} + t\vector{0,-1,0}$.
 \end{hint}

 \begin{hint}
   Since we are told that the line passes through $P$, we then may conclude
   $\vecl(t) = \vector{5,1,9}  + t\vector{0,-1,0}$.
 \end{hint}

 \begin{hint}
   Adding together corresponding entries reveals that $x(t)=5$.
 \end{hint}
 
 \begin{hint}
   Moreover $y(t)=1-t$.
 \end{hint}

 \begin{hint}
   Finally $z(t)=9.$
 \end{hint}
 
\end{exercise}
\end{document}
