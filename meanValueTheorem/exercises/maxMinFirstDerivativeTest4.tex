\documentclass{ximera}
\newcommand{\RR}{\mathbb R}
\renewcommand{\d}{\,d}
\newcommand{\dd}[2][]{\frac{d #1}{d #2}}
\renewcommand{\l}{\ell}
\newcommand{\ddx}{\frac{d}{dx}}
\newcommand{\dfn}{\textbf}
\newcommand{\eval}[1]{\bigg[ #1 \bigg]}

\author{Steven Gubkin \and Nela Lakos}
\license{Creative Commons 3.0 By-NC}
\outcome{ Define a critical point.}
\outcome{ Find critical points.}
\outcome{ Define absolute maximum and absolute minimum.}
\outcome{ Find the absolute max or min of a continuous function on a closed interval.}
\outcome{ Define local maximum and local minimum.}
\outcome{ Compare and contrast local and absolute maxima and minima.}
\outcome{ Identify situations in which an absolute maximum or minimum is guaranteed.}
\outcome{ Classify critical points.}
\outcome{ State the First Derivative Test.}
\outcome{ Apply the First Derivative Test.}
\begin{document}

\begin{exercise}

Consider the function $f(x) =\frac{x^2+2x+6}{x-1}$. \\
Find the x-coordinates of the global maximum and global minimum of $f$ on the interval $ [-3,0]$ .
\begin{hint}
First, compute the derivative of $f$.
$$
f'(x) = \frac{\answer{x^2-2x-8}}{(x-1)^2}
$$
The function $f$ has two critical points in the set $(-\infty,1)\cup (1,\infty)$ (note that we have to exclude 1 since it is not in the domain so $f$ is not defined there). 
 If we call these points $a$ and $b$ and order them such
that $a < b $, then
$$
a = \answer{-2}
$$

$$
b=\answer{4}
$$
\end {hint}

\begin{hint}

Next, we evaluate $f$ at the end points and at  the only critical point in the interval $ [-3,0]$.
$$
f(-3) = \answer{-9/4}
$$
$$
f(0)=\answer{-6}
$$
$$
f(a)=\answer{-2}
$$
\end {hint}

The function $f$ attains its global maximum at	
\begin{prompt}
$x=\answer{-2}$.
\end{prompt}
The function $f$ attains its global minimum at	
\begin{prompt}
$x=\answer{0}$.
\end{prompt}
\end{exercise}
\end{document}
