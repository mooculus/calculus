\documentclass{ximera}

\newcommand{\RR}{\mathbb R}
\renewcommand{\d}{\,d}
\newcommand{\dd}[2][]{\frac{d #1}{d #2}}
\renewcommand{\l}{\ell}
\newcommand{\ddx}{\frac{d}{dx}}
\newcommand{\dfn}{\textbf}
\newcommand{\eval}[1]{\bigg[ #1 \bigg]}


\author{Gregory Hartman \and Matthew CarrMathew Carr\and Nela Lakos}
\license{Creative Commons 3.0 By-NC}
\acknowledgement{https://github.com/APEXCalculus}

\outcome{Add up a large number of terms quickly using sigma notation.}
\outcome{Compute left, right, and midpoint Riemann Sums with many rectangles.}
\outcome{Understand the relationship between area under a curve and sums of rectangles.}
\outcome{Approximate area under a curve.}
\outcome{Understand how the area under a curve is related to the antiderivative.}
\outcome{Use limits of Riemann sums to find the exact area under a curve.}
\outcome{Understand how Riemann sums are used to find exact area.}

\begin{document}
\begin{exercise}


Answer the following using the formulas $\sum_{k=1}^{n}k=\frac{n(n+1)}{2}$ and $\sum_{k=1}^{n}C=nC$.
\begin{enumerate}
\item Find the Right Riemann Sum for $f(x)=3x+1$  on the interval $[0,4]$ using $n$ rectangles.
\begin{hint}
\[
\Delta x= \frac{\answer{4}}{n}
\]
\end{hint}
\begin{hint}
Let's determine the grid points, for $0\le k\le n$
\[
 x_k= \answer{k}\cdot\frac{\answer{4}}{n}
\]
\end{hint}
\begin{hint}
\[
 x_k^*= \answer{k}\cdot\frac{\answer{4}}{n}
\]
\[
 f(x_k^*)=f\left( \answer{k}\cdot\frac{\answer{4}}{n}\right)=3\cdot\left(\answer{k}\cdot\frac{\answer{4}}{n}\right)+1
\]

\end{hint}
\begin{hint}
Now, lets write the right Riemann sum
\[
\sum_{k=1}^n  f(x_k^*)\Delta x=\sum_{k=1}^n  \left(3\cdot\left(\answer{k}\cdot\frac{\answer{4}}{n}\right)+1\right)\cdot \frac{\answer{4}}{n}
\]

\end{hint}
\begin{hint}
Now, lets simplify the sum
\[
\sum_{k=1}^n  f(x_k^*)\Delta x= \frac{\answer{4}}{n}\sum_{k=1}^n  \left(3\cdot\left(\answer{k}\cdot\frac{\answer{4}}{n}\right)+1\right)= \frac{\answer{4}}{n}\left(\sum_{k=1}^n  3\cdot\left(\answer{k}\cdot\frac{\answer{4}}{n}\right)+\sum_{k=1}^n 1\right)
\]

\end{hint}
\begin{hint}
Now, lets simplify more
\[
\sum_{k=1}^n  f(x_k^*)\Delta x= \frac{\answer{4}}{n}\left(3\cdot\frac{\answer{4}}{n}\sum_{k=1}^n \answer{k}+\sum_{k=1}^n 1\right)
\]

\end{hint}
\begin{hint}
Now, apply the given formulas
\[
\sum_{k=1}^n  f(x_k^*)\Delta x= \frac{\answer{4}}{n}\left(3\cdot\frac{\answer{4}}{n}\frac{n(n+1)}{2}+n\right)
\]
\end{hint}
\begin{prompt}
\[
\sum_{k=1}^n  f(x_k^*)\Delta x=28+\frac{\answer{24}}{n}
\]
\end{prompt}
\item Compute the limit as $n\to\infty$ of your answer above.
 \begin{prompt} 
\[
\lim_{n\to\infty}\sum_{k=1}^n  f(x_k^*)\Delta x=\answer{28}
\]
\end{prompt}
\item Compute the area between the curve $y=f(x)$ and the interval $[0,4]$ using geometry.
\begin{hint}
\begin{image}
  \begin{tikzpicture}[
      declare function = {f(\x) = 3*x+1;} ]
	\begin{axis}[
            domain=-.2:4, xmin =-.2,xmax=4,ymax=13.3,ymin=-.2,
            width=6in,
            height=3in,
            xtick={0,1,2,3,4}, ytick={0,...,13},
            xticklabels={$0$,$1$,$2$, $3$,$4$},
           yticklabels={0,,2,,4,,6,,8,,10,,12},
            axis lines=center, xlabel=$x$, ylabel=$y$,
            every axis y label/.style={at=(current axis.above origin),anchor=south},
            every axis x label/.style={at=(current axis.right of origin),anchor=west},
            axis on top,
          ]
          \addplot [draw=none,fill=fillp,domain=0:4, smooth] {f(x)} \closedcycle;
          \addplot [very thick,penColor, smooth] {f(x)};
        \end{axis}
\end{tikzpicture}
\end{image}
Find the x-intercept first. 
\end{hint}
\begin{prompt}
 \[
A=\answer{28}
\]
\end{prompt}
\end{enumerate}
\end{exercise}
\end{document}
