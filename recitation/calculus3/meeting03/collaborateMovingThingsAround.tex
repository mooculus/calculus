\documentclass[handout,hints,noauthor,nooutcomes]{ximera}

\newcommand{\RR}{\mathbb R}
\renewcommand{\d}{\,d}
\newcommand{\dd}[2][]{\frac{d #1}{d #2}}
\renewcommand{\l}{\ell}
\newcommand{\ddx}{\frac{d}{dx}}
\newcommand{\dfn}{\textbf}
\newcommand{\eval}[1]{\bigg[ #1 \bigg]}


\title[Collaborate:]{Moving things around}

\begin{document}
\begin{abstract}
  We move things around in space.
\end{abstract}
\maketitle

\textbf{Work in groups of 3--4, writing your answers on a separate
  sheet of paper.}


\section{Moving parametric graphs around}

We have (at least!) two ways of writing a parametric formula for a
curve in space
\[
\vector{x(t),y(t),z(t)}\quad\text{and}\quad x(t) \veci + y(t) \vecj + z(t) \veck
\]
This second way, using $\veci$, $\vecj$, and $\veck$ is what we're
going to think about today.

\begin{problem}
  Compare and contrast the following vector-valued functions in
  $\R^3$:
  \begin{align*}
    \vec{a}(t) &= \veci \cos(t) + \vecj \sin(t)\\
    \vec{b}(t) &= \veci \cos(t) + \veck \sin(t)\\
    \vec{c}(t) &= \vecj \cos(t) + \veck \sin(t)
  \end{align*}
\end{problem}

\begin{problem}
  Consider vectors:
  \begin{align*}
    \uvec{v} &= \vector{\frac{1}{\sqrt{6}},\frac{-2}{\sqrt{6}},\frac{1}{\sqrt{6}}}\\
    \uvec{w} &= \vector{\frac{1}{\sqrt{2}},0,\frac{-1}{\sqrt{2}}}
  \end{align*}
  Verify the following:
  \begin{enumerate}
  \item $|\uvec{v}| = |\uvec{w}| = 1$.
  \item $\uvec{v}$ is orthogonal to $\uvec{w}$.
  \item The lines $\vecl(t) = t \uvec{v}$ and $\vec{m}(t) = t
    \uvec{w}$ both lie on the plane $x+y+z = 0$.
  \end{enumerate}
\end{problem}

\begin{problem}
  Give a vector-valued formula for a circle of radius $1$ that lies in
  the plane $x+y+z=0$.
\end{problem}


\begin{problem}
  Consider the equations:
  \begin{align*}
    x+y &= 6\\
    x^2+y^2+z^2 &= 18
  \end{align*}
  How many solutions for $x$, $y$, and $z$ do you expect to find?
  \begin{hint}
    Draw a picture showing the geometry of this situation.
  \end{hint}
\end{problem}

\begin{problem}
  Consider the equations:
  \begin{align*}
    x+y &= 0\\
    x^2+y^2+z^2 &= 18
  \end{align*}
    Parameterize a curve giving the solutions to these equations.
    \begin{hint}
      Draw a picture showing the geometry of this situation.
    \end{hint}
\end{problem}


\begin{problem}
  Consider the equations:
  \begin{align*}
    x+y &= 2\\
    x^2+y^2+z^2 &= 18
  \end{align*}
    Parameterize a curve giving the solutions to these equations.
    \begin{hint}
      Draw a picture showing the geometry of this situation.
    \end{hint}
\end{problem}


\begin{problem}
  Let $a$ and $b$ be fixed real numbers. Consider the equations:
  \begin{align*}
    x+y &= a\\
    x^2+y^2+z^2 &= b
  \end{align*}
  Give a general solution in terms of $a$ and $b$.
  \begin{hint}
    Draw a picture showing the geometry of this situation.
  \end{hint}
\end{problem}







\section{Moving graphs around}

We have just worked with vector-valued functions. The intrepid young
mathematician who wishes to further expand their mind, might wish to
press-on, and work with implicit equations as well.


\begin{problem}
  Explain how graphing
  \[
  y= x^2
  \]
  is related to all vectors $\vec{x}= \vector{x,y}$ such that
  \[
  \scal_\vecj(\vec{x}) = \scal_\veci(\vec{x})^2.
  \]
\end{problem}

\begin{problem}
  Consider $\vec{v} = \vector{1,1}$, $\vec{w} = \vector{-1,1}$, and $\vec{x}
  = \vector{x,y}$. What will the graph of 
  \[
  \scal_\vec{w}(\vec{x}) = \scal_\vec{v}(\vec{x})^2
  \]
  look like? Confirm your answer by using something like
  \textit{Desmos}, \textit{GeoGebra}, or \textit{WolframAlpha}.
\end{problem}

\begin{problem}
  Consider $\vec{v} = \vector{1,-2}$, $\vec{w} = \vector{2,1}$, and $\vec{x}
  = \vector{x-3,y+4}$. What will the graph of 
  \[
  \scal_\vec{w}(\vec{x}) = \scal_\vec{v}(\vec{x})^2
  \]
  look like? Confirm your answer by using something like
  \textit{Desmos}, \textit{GeoGebra}, or \textit{WolframAlpha}.
\end{problem}

\begin{problem}
  Consider $\uvec{v} = \vector{\cos(\theta),\sin(\theta)}$, $\uvec{w} =
    \vector{-\sin(\theta),\cos(\theta)}$, and $\vec{x} =
    \vector{x,y}$. For any fixed value of $\theta$, what will the
    graph of
  \[
  \scal_\uvec{w}(\vec{x}) = f(\scal_\uvec{v}(\vec{x}))
  \]
  look like? Confirm your answer for different values of $\theta$ by
  choosing a reasonable function $f$ and using something like
  \textit{Desmos}, \textit{GeoGebra}, or \textit{WolframAlpha}.
\end{problem}


\end{document}
