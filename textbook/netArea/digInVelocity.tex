\documentclass{ximera}

\newcommand{\RR}{\mathbb R}
\renewcommand{\d}{\,d}
\newcommand{\dd}[2][]{\frac{d #1}{d #2}}
\renewcommand{\l}{\ell}
\newcommand{\ddx}{\frac{d}{dx}}
\newcommand{\dfn}{\textbf}
\newcommand{\eval}[1]{\bigg[ #1 \bigg]}


\title[Dig-In:]{Relating velocity and position}


\begin{document}

A central theme of this course has been that we can often gain a
better understanding of a function by looking at its derivative, and
then working backwards.  This has been our approach to max/min
problems, curve sketching, linear approximation, etc.  So
antiderivatives have really been important to us all along.

We have a graphical interpretation of the derivative as the slope of a
tangent line at a point.  We have not yet found a graphical
interpretation of the antiderivative.

  \begin{question}
   Let $f$ be the constant function $4$. Let $F$ be any antiderivative
   of $f$.  Then $F(5)-F(2) = \answer{12}$.
   \begin{hint}
    This is just a fancy way of saying that $f$ is a line of slope
    $4$, and we are looking for the rise corresponding to a run of
    $5-2= 3$
   \end{hint}
   \begin{hint}
    $F(5) - F(2) = 4(5-2) = 12$
   \end{hint}
  \end{question}
  
  \begin{question}
    The area of the region bounded by the grpah of $f$, the horizontal
    axis, and the vertical lines $x=2$ and $x=5$ is \answer{12}.
    
    \begin{hint}
      \begin{image}
  \begin{tikzpicture}[
      declare function = {f(\x) = 4;} ]
	\begin{axis}[
            domain=-.2:7, xmin =-.2,xmax=7,ymax=5,ymin=-.2,
            width=4in,
            height=2in,
            xtick={2,5}, 
	   ytick={4},
            xticklabels={$2$,$5$},
            yticklabels={$4$},
            axis lines=center, xlabel=$x$, ylabel=$y$,
            every axis y label/.style={at=(current axis.above origin),anchor=south},
            every axis x label/.style={at=(current axis.right of origin),anchor=west},
            axis on top,
          ]
          \addplot [draw=none,fill=fillp,domain=2:5, smooth] {f(x)} \closedcycle;
          \addplot [very thick,penColor, smooth, domain=0:7] {f(x)};
        \end{axis}
\end{tikzpicture}
\end{image}
    \end{hint}
    \begin{hint}
      This is a rectangle with height $4$ and width $3$, so the area is $12$
    \end{hint}
  \end{question}


The fact that these two answers are the same is the germ of one of the
most ``fundamental'' ideas in all of calculus.

\begin{idea}[Two models of multiplication]
  There are two basic models of multiplication: A "rate times time"
  perpective and an area perspective.  For instance, we could
  interpret $3*4$ as an answer to the question "If I am going $3
  \textrm{mph}$ for $4$ hours, how far have I traveled?" or as the
  answer to the question "What is the area of a rectangle with length
  $3$ and width $4$?"
\end{idea}

We can apply this basic idea to the problem of approximating
antiderivatives.

  Let $f$ be a given function, like this one:

\begin{image}
  \begin{tikzpicture}[
      declare function = {f(\x) = cos((\x-1.1)*360/6.28);} ]
	\begin{axis}[
            domain=-.2:4, xmin =-.2,xmax=4,ymax=1,ymin=-1,
            width=4in,
            height=2in,
            xtick={1,3}, 
            xticklabels={$1$,$3$},
            ytick style={draw=none},
            yticklabels={},
            axis lines=center, xlabel=$x$, ylabel=$y$,
            every axis y label/.style={at=(current axis.above origin),anchor=south},
            every axis x label/.style={at=(current axis.right of origin),anchor=west},
            axis on top,
          ]

          \addplot [very thick,penColor, smooth] {f(x)};
          \node at (axis cs:3.5,4.5) [anchor=west] {$f$};  


        \end{axis}
\end{tikzpicture}
\end{image}

 and let $F$ be an antiderivative of $f$.  We want to approximate $F(3) - F(1)$.  Let's get an approximation using $4$ subintervals.
  
  \begin{align*}F(3) - F(1) &= (F(3) - F(2.5))+(F(2.5) - F(2))+(F(2) - F(1.5))+(F(1.5) - F(1))\\
  &\approx F'(3)\frac{1}{2}+F'(2.5)\frac{1}{2}+F'(2)\frac{1}{2}+F'(1.5)\frac{1}{2}\\
  &=f(3)\frac{1}{2}+f(2.5)\frac{1}{2}+f(2)\frac{1}{2}+f(1.5)\frac{1}{2}
  \end{align*}
  
  We can visualize this as the sum of the areas of $4$ rectangles:

\begin{image}
  \begin{tikzpicture}[
      declare function = {f(\x) = cos((\x-1.1)*360/6.28);} ]
	\begin{axis}[
            domain=-.2:4, xmin =-.2,xmax=4,ymax=1,ymin=-1,
            width=4in,
            height=2in,
            xtick={1,1.5,2,2.5,3}, 
            xticklabels={$1$,$1.5$,$2$, , $ \hphantom{22}3$},
            ytick style={draw=none},
            yticklabels={},
            axis lines=center, xlabel=$x$, ylabel=$y$,
            every axis y label/.style={at=(current axis.above origin),anchor=south},
            every axis x label/.style={at=(current axis.right of origin),anchor=west},
            axis on top,
          ]
        
\foreach \rectnumber in {1,2,3,4}
               {
                 \addplot [draw=penColor,fill=fillp] plot coordinates
                          {({1+(\rectnumber - 1) * 0.5},{f(1+\rectnumber * 0.5)})
                            ({1+(\rectnumber) * 0.5},{f(1+\rectnumber * 0.5) })} \closedcycle;

          
               };

	
         \addplot [very thick,penColor, smooth] {f(x)};
          \node at (axis cs:3.5,4.5) [anchor=west] {$f$};  
        \end{axis}
\end{tikzpicture}
\end{image}
  
By approximating $f$ by a piecewise constant function, we are approximating $F$ by a piecewise linear function:

\begin{image}
  \begin{tikzpicture}[
      declare function = {f(\x) = \x^3/3+1;} ]
	\begin{axis}[
            domain=-.2:4, xmin =-.2,xmax=4,ymax=11,ymin=-.2,
            width=4in,
            height=2in,
            xtick={1,1.5,2,2.5,3}, 
            xticklabels={$1$,$1.5$,$2$, $2.5$, $3$},
            ytick style={draw=none},
            yticklabels={},
            axis lines=center, xlabel=$x$, ylabel=$y$,
            every axis y label/.style={at=(current axis.above origin),anchor=south},
            every axis x label/.style={at=(current axis.right of origin),anchor=west},
            axis on top,
          ]

          \addplot [very thick,penColor2, smooth, domain=1:1.5] {f(1.5)*(x-1)};
          \addplot [very thick,penColor2, smooth, domain=1.5:2] {f(2)*(x-1.5)+f(1.5)*0.5};
          \addplot [very thick,penColor2, smooth, domain=2:2.5] {f(2.5)*(x-2)+f(2)*0.5+f(1.5)*0.5};
          \addplot [very thick,penColor2, smooth, domain=2.5:3] {f(3)*(x-2.5)+f(2.5)*(0.5)+f(2)*0.5+f(1.5)*0.5};

  \addplot [draw=penColor] plot coordinates {(1,0) (1.5,0) (1.5, {f(1.5)*(0.5)}) (2, {f(1.5)*(0.5)}) (2, {f(2)*0.5+f(1.5)*0.5}) (2.5, {f(2)*0.5+f(1.5)*0.5}) (2.5, {f(2.5)*(0.5)+f(2)*0.5+f(1.5)*0.5}) (3, {f(2.5)*(0.5)+f(2)*0.5+f(1.5)*0.5})  (3, { f(3)*0.5+f(2.5)*(0.5)+f(2)*0.5+f(1.5)*0.5})};

          \node at (axis cs:3.1,10) [anchor=west] {$F$};  
        \end{axis}
\end{tikzpicture}
\end{image}


  
  On the one hand, we can interpret each $F'(x)\Delta x$ as an approximation in the change in the height of $F$ over the subinterval.  On the other hand, since $F' =f$, we can think of $F'(x)\Delta x = f(x)\Delta x$ as the area of a rectangle based on the subinterval, and with hieght $f(x)$.  So the total change in $F$ over $[1,3]$ can be reinterpreted as the sum of the areas of the rectangles under $f$.
  
  If we make our subintervals smaller, the repeated linear approximation should get more precise, and the sum of the areas of the rectangles gets closer to the area under the curve:

\begin{image}
  \begin{tikzpicture}[
      declare function = {f(\x) = \x^2/3+1;} ]
	\begin{axis}[
            domain=-.2:4, xmin =-.2,xmax=4,ymax=7,ymin=-.2,
            width=4in,
            height=2in,
            xtick={1,3}, 
            xticklabels={$1$,$3$},
            ytick style={draw=none},
            yticklabels={},
            axis lines=center, xlabel=$x$, ylabel=$y$,
            every axis y label/.style={at=(current axis.above origin),anchor=south},
            every axis x label/.style={at=(current axis.right of origin),anchor=west},
            axis on top,
          ]
        
\foreach \rectnumber in {1,2,...,20}
               {
                 \addplot [draw=penColor,fill=fillp] plot coordinates
                          {({1+(\rectnumber - 1) * 0.1},{f(1+\rectnumber * 0.1)})
                            ({1+(\rectnumber) * 0.1},{f(1+\rectnumber * 0.1) })} \closedcycle;

                 \addplot[decoration={brace,mirror,raise=.2cm},decorate,thin] plot coordinates
                          {(\rectnumber+.1,0)
                            (1+\rectnumber-.1,0)};
               };
          \addplot [very thick,penColor, smooth] {f(x)};
          \node at (axis cs:3.5,4.5) [anchor=west] {$f$};  
        \end{axis}
\end{tikzpicture}
\end{image}


  
\end{document}
