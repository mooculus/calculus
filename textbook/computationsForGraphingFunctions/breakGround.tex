\documentclass{ximera}

\newcommand{\RR}{\mathbb R}
\renewcommand{\d}{\,d}
\newcommand{\dd}[2][]{\frac{d #1}{d #2}}
\renewcommand{\l}{\ell}
\newcommand{\ddx}{\frac{d}{dx}}
\newcommand{\dfn}{\textbf}
\newcommand{\eval}[1]{\bigg[ #1 \bigg]}


\outcome{Determine how the graph of a function looks without using a calculator.}

% Synthesis of everything we've learned

% We've learned so much stuff!
% Can't we put this all together somehow?

% Let's make a concept map of everything we've learned so far.

% Holes, point discontinuities, review from limits at the beginning


\title[Break-Ground:]{Wanted: graphing procedure}

\begin{document}
\begin{abstract}
Two young mathematicians discuss how to sketch the graphs of functions.
\end{abstract}
\maketitle

Check out this dialogue between two calculus students (based on a true
story):

\begin{dialogue}
\item[Devyn] Riley, OK I know how to plot something if I'm given a description.
\item[Riley] Yes, it's kinda fun right?
\item[Devyn] I know!  But now I'm not sure how to get the information I need.
\item[Riley] You know, I'd like to make up a procedure based on all
  these facts, that would tell me what the graph of any function would look like.
\item[Devyn] Me too! Let's get to work!
\end{dialogue}



\begin{problem}%% BADBAD This is a good problem. How do we do it?
  Below is a list of features of a graph of a function.
  \begin{enumerate}
  \item Find any vertical asymptotes, these are points $x=a$ where
    $f(x)$ goes to infinity as $x$ goes to $a$ (from the right, left,
    or both).    
  \item Find the critical points (the points where $f'(x) = 0$ or
    $f'(x)$ is undefined).
  \item Identify inflection points and concavity.
  \item Determine an interval that shows all relevant behavior.
  \item Find the candidates for inflection points, the points where
    $f''(x) = 0$ or $f''(x)$ is undefined.
  \item Compute $f'(x)$ and $f''(x)$.
  \item Find the $y$-intercept, this is the point $(0,f(0))$. Place this
    point on your graph.
  \item Use either the first or second derivative test to identify local extrema and/or
    find the intervals where your function is increasing/decreasing.
  \item If possible, find the $x$-intercepts, the points where $f(x) =
    0$. Place these points on your graph.
  \item Analyze end behavior:  as $x \to \pm \infty$, what happens to the graph of $f$?  Does it  have horizontal asymptotes, increase or decrease without bound, or have some other kind of behavior?
 \end{enumerate}
  In what order should we take these steps? For example, one must compute
   $f'(x)$ before computing $f''(x)$. Also, one must compute $f'(x)$ before 
   finding the critical points. There is more than one correct answer.
  \begin{freeResponse}
  Here is one possible answer to this question.  Compare it with yours!
  \begin{enumerate}
  \item Compute $f'(x)$ and $f''(x)$.
  \item Find the $y$-intercept, this is the point $(0,f(0))$. Place this
    point on your graph.
  \item Find any vertical asymptotes, these are points $x=a$ where
    $f(x)$ goes to infinity as $x$ goes to $a$ (from the right, left,
    or both).
  \item If possible, find the $x$-intercepts, the points where $f(x) =
    0$. Place these points on your graph.
  \item Analyze end behavior:  as $x \to \pm \infty$, what happens to the graph of $f$?  Does it  have horizontal asymptotes, increase or decrease without bound, or have some other kind of behavior?.
  \item Find the critical points (the points where $f'(x) = 0$ or
    $f'(x)$ is undefined).
  \item Use either the first or second derivative test to identify local extrema and/or
    find the intervals where your function is increasing/decreasing.
  \item Find the candidates for inflection points, the points where
    $f''(x) = 0$ or $f''(x)$ is undefined.
  \item Identify inflection points and concavity.
  \item Determine an interval that shows all relevant behavior
  \end{enumerate}
  \end{freeResponse}
\end{problem}

%%
%% %% \begin{xarmaBoost}
%%   Write down at least \textbf{five} questions for this lecture. After
%%   you have your questions, label them as ``Level 1,'' ``Level 2,'' or
%%   ``Level 3'' where:
%% \begin{description}
%% \item[Level 1] Means you know the answer, or know exactly how to do
%%   this problem.
%% \item[Level 2] Means you think you know how to do the problem.
%% \item[Level 3] Means you have no idea how to do the problem.
%% \end{description}
%% \begin{freeResponse}
%% \end{freeResponse}
%% \end{xarmaBoost}

%%
\end{document}
