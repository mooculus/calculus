\documentclass{ximera}

\newcommand{\RR}{\mathbb R}
\renewcommand{\d}{\,d}
\newcommand{\dd}[2][]{\frac{d #1}{d #2}}
\renewcommand{\l}{\ell}
\newcommand{\ddx}{\frac{d}{dx}}
\newcommand{\dfn}{\textbf}
\newcommand{\eval}[1]{\bigg[ #1 \bigg]}


\title{Buckeye VR}

\begin{document}
\begin{abstract}
  You'll learn Buckeye VR.
\end{abstract}
\maketitle

Buckeye VR will allow us plot parametric curves and surfaces.
You can install Buckeye VR by scanning this QR code:
\begin{image}
  \includegraphics{QRBVRPlot.png}
\end{image}

Or by visiting these links:
\begin{itemize}
\item For Android: \link[Here]{https://play.google.com/store/apps/details?id=edu.Buckeyevr.osu.BuckeyeVR_3D_Plot_Viewer}
\item For Mac: \link[Here]{https://itunes.apple.com/us/app/buckeyevr-3d-plot-viewer/id1280551694?mt=8}
\end{itemize}



\section{A curve}

First, we'll look at an example of a curve:

\begin{verbatim}
//Example Curve.bvr

// Parametric curve
helix=parametricCurve
helix.setMinMax=t,-10,10
helix.setEquation=x,(cos(t))
helix.setEquation=y,(sin(t))
helix.setEquation=z,(.3*t)
helix.setSegmentCount=70
helix.setWidth=0.2,0.2
helix.color=red
helix.spawn
\end{verbatim}

You can see this plotted in Buckeye VR
\link[here]{http://go.osu.edu/bvr1}.
If you have the app you can scan the following QR code:
\begin{image}
\includegraphics{bvrQR1.png}  
\end{image}



\section{A surface}

Second, we'll look at a surface:

\begin{verbatim}
// Example Surface.bvr

// Parametric surface
cyl=parametricSurface
cyl.setMinMax=s,-10,10
cyl.setMinMax=t,0,6.6
cyl.setEquation=x,(cos(t))
cyl.setEquation=y,(sin(t))
cyl.setEquation=z,(.3*s)
cyl.setSegmentCount=20
cyl.color=blue
cyl.spawn
\end{verbatim}

You can see this plotted in Buckeye VR
\link[here]{http://go.osu.edu/bvr2}.
If you have the app you can scan the following QR code:
\begin{image}
\includegraphics{bvrQR2.png}  
\end{image}

\section{Combined}

You can combine plots simply by listing them after each other.

\begin{verbatim}
// Example Combined.bvr

// Parametric curve
helix=parametricCurve
helix.setMinMax=t,-10,10
helix.setEquation=x,(cos(t))
helix.setEquation=y,(sin(t))
helix.setEquation=z,(.3*t)
helix.setSegmentCount=70
helix.setWidth=0.2,0.2
helix.color=red
helix.spawn

// Parametric surface
cyl=parametricSurface
cyl.setMinMax=s,-10,10
cyl.setMinMax=t,0,6.6
cyl.setEquation=x,(cos(t))
cyl.setEquation=y,(sin(t))
cyl.setEquation=z,(.3*s)
cyl.setSegmentCount=20
cyl.color=blue
cyl.spawn
\end{verbatim}

You can see this plotted in Buckeye VR
\link[here]{http://go.osu.edu/bvr3}.
If you have the app you can scan the following QR code:
\begin{image}
\includegraphics{bvrQR3.png}  
\end{image}

\section{Play around with it}
You can play with Buckeye VR 
\link[here]{https://buckeyevr.osu.edu/plot/}.

Some things to note: 
\begin{itemize}
\item You should use a \textbf{text editor}, not a word processor to
  do your work. \textit{TextEdit} is a good text editor for Macs, and
  \textit{Notepad} is a good text editor for Windows. If they produce
  \texttt{*.rtf} files, you may need to adjust your preferences.
\item You must \textbf{start} your file with
\begin{verbatim}
\\ NAME.bvr
\end{verbatim}
  \item You must \textit{name} each object in your graph. For example,
    the curve above is named \verb|helix| and the surface is named
    \verb|cyl|. \textbf{Each object should have a distinct name.}
  \item I suggest you only use the variable $t$ for parametric curves
    and the variables $s$ and $t$ for parametric surfaces. Full
    disclosure, the variables $x$, $y$, and $z$ will also work.
  \item All expressions for the components of your curve or surface
    must be contained in \verb|(| and \verb|)|.
  \item Often the domain will need to be enlarged to make a curve or
    surface with no holes. Note, in the parametric surface example
    above, $t$ \textit{should} run from $0$ to $2\pi\approx
    6.28$. However in the code, it runs from \verb|0| to
    \verb|6.6|. You may have to resort to similar tricks.
  \item You can use the following colors for curves and surfaces: black, blue,
    cyan, aqua, grey/gray, green, magenta, fuchsia, red, white,
    yellow. 
\end{itemize}
\end{document}
