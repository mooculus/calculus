\documentclass{ximera}

\newcommand{\RR}{\mathbb R}
\renewcommand{\d}{\,d}
\newcommand{\dd}[2][]{\frac{d #1}{d #2}}
\renewcommand{\l}{\ell}
\newcommand{\ddx}{\frac{d}{dx}}
\newcommand{\dfn}{\textbf}
\newcommand{\eval}[1]{\bigg[ #1 \bigg]}


\author{Jim Talamo}
\license{Creative Commons 3.0 By-bC}


\outcome{}


\begin{document}
\begin{exercise}

Determine whether the series:

\[
\sum_{k=1}^{\infty} (-1)^k \frac{2k^7+3k^5}{4^k}
\]

converges absolutely, converges conditionally, or diverges by completing the exercise that follows.

\begin{hint}
For the series $\sum_{k=1}^{\infty}(-1)^k \frac{2k^7+3k^5}{4^k}$, is the series geometric or a $p$-series?

\begin{multipleChoice}
\choice{The series is a geometric series.}
\choice{The series is a $p$-series.}
\choice[correct]{The series is neither a geometric series nor a $p$-series.}
\end{multipleChoice}

Let's think about the other most useful convergence tests.  Which of the following tests are \emph{applicable}?

\begin{selectAll}
\choice[correct]{divergence test}
\choice{comparison test}
\choice{limit comparison test}
\choice{ratio test}
\choice{root test}
\choice[correct]{alternating series test}
\end{selectAll}

Unfortunately, the divergence test is not conclusive since $\lim_{n \to \infty} \frac{\sin(n)}{n^2} = \answer{0}$, and no other test applies since the summand is not strictly positive.  The alternating series test does apply here, but note that you should be able to justify why $\frac{2k^7+3k^5}{4^k}$ is decreasing if you are asked.

Since we have to classify the convergence, we will check for absolute convergence; i.e. whether the series:

\[
\sum_{k=1}^{\infty} \left|(-1)^k \frac{2k^7+3k^5}{4^k} \right| = \sum_{k=1}^{\infty} \frac{2k^7+3k^5}{4^k}
\]

converges.  Now that the summand is positive, we can use more tests!  A solution can be now provided using either Ratio, Root, or the Limit Comparison Test.
\end{hint}

\end{exercise}
\end{document}
