\documentclass{ximera}

\newcommand{\RR}{\mathbb R}
\renewcommand{\d}{\,d}
\newcommand{\dd}[2][]{\frac{d #1}{d #2}}
\renewcommand{\l}{\ell}
\newcommand{\ddx}{\frac{d}{dx}}
\newcommand{\dfn}{\textbf}
\newcommand{\eval}[1]{\bigg[ #1 \bigg]}


\outcome{}

\title[Dig-in:]{The derivative as a function}

\begin{document}
\begin{abstract}
\end{abstract}
\maketitle

We know that to find the derivative of a function at a point $x=a$ we write
\[
f'(a) = \lim_{h\to 0}\frac{f(a+h)-f(a)}{h}.
\]
However, if we replace the given number $a$ with a varible $x$, we now
have
\[
f'(x) = \lim_{h\to 0}\frac{f(x+h)-f(x)}{h}.
\]
This tells us the instantaneous rate of change at any given point
$x$.
\begin{warning}
  The notation
  \[
  \text{$f'(a)$ means take the derivative of $f$ first, then evaluate
    at $x=a$.}
    \]
\end{warning}
Given a function $f$ from the real numbers to the real numbers, the
derivative $f'$ is also a function from the real numbers to the real
numbers. The upshot is that we may study a function by examining the
graph of its derivative and vice versa.




\begin{example}
Here are plots of four functions.
\[
\begin{array}{cccc}
\begin{tikzpicture}
	\begin{axis}[
            domain=-3:3,
            width=2in,
            ymax=4,
            ymin=-4,
            %samples=100,
            axis lines =middle, xlabel=$x$, ylabel=$y$,
            every axis y label/.style={at=(current axis.above origin),anchor=south},
            every axis x label/.style={at=(current axis.right of origin),anchor=west}
          ]
          \addplot [very thick, penColor, smooth,domain=(-3:3)] {x^4-4*x^3+4*x^2};
        \end{axis}
\end{tikzpicture} & 
\begin{tikzpicture}
	\begin{axis}[
            domain=-3:3,
            width=2in,
            ymax=4,
            ymin=-4,
            %samples=100,
            axis lines =middle, xlabel=$x$, ylabel=$y$,
            every axis y label/.style={at=(current axis.above origin),anchor=south},
            every axis x label/.style={at=(current axis.right of origin),anchor=west}
          ]
          \addplot [very thick, penColor, smooth,domain=(-3:3)] {3*x^2-6*x};
        \end{axis}
\end{tikzpicture} & 
\begin{tikzpicture}
	\begin{axis}[
            domain=-3:3,
            width=2in,
            ymax=4,
            ymin=-4,
            %samples=100,
            axis lines =middle, xlabel=$x$, ylabel=$y$,
            every axis y label/.style={at=(current axis.above origin),anchor=south},
            every axis x label/.style={at=(current axis.right of origin),anchor=west}
          ]
          \addplot [very thick, penColor, smooth,domain=(-3:3)] {x^3-3*x^2+2};
        \end{axis}
\end{tikzpicture} & 
\begin{tikzpicture}
	\begin{axis}[
            domain=-3:3,
            width=2in,
            ymax=4,
            ymin=-4,
            %samples=100,
            axis lines =middle, xlabel=$x$, ylabel=$y$,
            every axis y label/.style={at=(current axis.above origin),anchor=south},
            every axis x label/.style={at=(current axis.right of origin),anchor=west}
          ]
          \addplot [very thick, penColor, smooth,domain=(-3:3)] {4*x^3-12*x^2 + 8*x};
        \end{axis}
\end{tikzpicture} \\
\end{array}
\]
\[
p(x) & q(x) & r(x) & s(x)
\]
\end{fullwidth}
Two of these functions are the derivatives of the other two, identify
which functions are the derivatives of the others.
\begin{answer}
$p'(x) = s(x)$ and $r'(x) = q(x)$
\end{answer}

    \begin{explanation}
    From this graph, we can read off
    \end{explanation}
\end{example}


graphs of derivatvies

(postive -> Increaseing)


\end{document}

