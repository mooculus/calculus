\documentclass{ximera}

\newcommand{\RR}{\mathbb R}
\renewcommand{\d}{\,d}
\newcommand{\dd}[2][]{\frac{d #1}{d #2}}
\renewcommand{\l}{\ell}
\newcommand{\ddx}{\frac{d}{dx}}
\newcommand{\dfn}{\textbf}
\newcommand{\eval}[1]{\bigg[ #1 \bigg]}


\author{Jim Talamo}
\license{Creative Commons 3.0 By-bC}

\outcome{Compute derivatives of Taylor series}


\begin{document}

\begin{exercise}
Find the seventh degree Taylor polynomial centered at $x=0$ for the function $g(x) = \int_0^x e^{-t^2} \d t$

\[
p_7(x) = \answer{x-\frac{1}{3}x^3+\frac{1}{10}x^5-\frac{1}{42}x^7}
\]

\begin{hint}
We begin by exhibiting several nonzero terms in the series centered at $t=0$ for $e^t$:

\[
e^{t} = \answer{1}+\answer{1}t+\answer{\frac{1}{2}}t^2+\answer{\frac{1}{3!}}t^3+\ldots
\]
We may need more terms, but we can exhibit them later if necessary.

\begin{question}
Using the rule of compositions, we can find the series for $e^{-t^2}$:
\[
e^{-t^2} =1+\left(\answer{-t^2}\right)+\frac{1}{2}\left(\answer{-t^2}\right)^2+\frac{1}{3!}\left(\answer{-t^2}\right)^3+\ldots
\]
Simplifying, we have:
\[
e^{-t^2} = \answer{1-t^2+\frac{1}{2}t^4-\frac{1}{6}t^6}+\ldots (\textrm{ your answer should include up to the } t^6 \textrm{ term})
\]

\begin{question}
We can now integrate the expression we wrote down above to find:

\begin{align*}
g(x) = \int_0^x e^{-t^2} \d t &= \int_0^x \left( 1-t^2+\frac{1}{2}t^4-\frac{1}{6}t^6  +\ldots\right) \d t \\
&= \eval{\answer{ t-\frac{1}{3}t^3+\frac{1}{10}t^5-\frac{1}{42}t^7  } + \ldots }_0^x \\
&= \answer{x-\frac{1}{3}x^3+\frac{1}{10}x^5-\frac{1}{42}x^7} + \ldots 
\end{align*}

We can now extract the sixth degree Taylor polynomial centered at $x=0$ for $g(x)$:
\[
p_7(x) = \answer{x-\frac{1}{3}x^3+\frac{1}{10}x^5-\frac{1}{42}x^7}
\]

\end{question}
\end{question}
\end{hint}

Note that we have no way to write down the antiderivatives for $e^{-t^2}$, but we can use the above series to approximate values for definite integrals of the form $ \int_0^x e^{-t^2} \d t$.

Using a calculator or program of your choice $ \int_0^{.5} e^{-t^2} \d t = \answer[tolerance=.000001]{.461281} $ to $6$ decimal places.

We can approximate $\int_0^{.5} e^{-t^2} \d t$ by $p_7\left(\answer{.5}\right)$ and find: $\int_0^{.5} e^{-t^2} \d t \approx \answer[tolerance=.000001]{.461272} $ to $6$ decimal places.

\end{exercise}
\end{document}
