\documentclass[handout]{ximera}
%handout:  for handout version with no solutions or instructor notes
%handout,instructornotes:  for instructor version with just problems and notes, no solutions
%noinstructornotes:  shows only problem and solutions

%% handout
%% space
%% newpage
%% numbers
%% nooutcomes

%I added the commands here so that I would't have to keep looking them up
%\newcommand{\RR}{\mathbb R}
%\renewcommand{\d}{\,d}
%\newcommand{\dd}[2][]{\frac{d #1}{d #2}}
%\renewcommand{\l}{\ell}
%\newcommand{\ddx}{\frac{d}{dx}}
%\everymath{\displaystyle}
%\newcommand{\dfn}{\textbf}
%\newcommand{\eval}[1]{\bigg[ #1 \bigg]}

%\begin{image}
%\includegraphics[trim= 170 420 250 180]{Figure1.pdf}
%\end{image}

%add a ``.'' below when used in a specific directory.

\newcommand{\RR}{\mathbb R}
\renewcommand{\d}{\,d}
\newcommand{\dd}[2][]{\frac{d #1}{d #2}}
\renewcommand{\l}{\ell}
\newcommand{\ddx}{\frac{d}{dx}}
\newcommand{\dfn}{\textbf}
\newcommand{\eval}[1]{\bigg[ #1 \bigg]}




\author{Tom Needham and Jim Talamo}

\outcome{Use definite integrals to compute volumes of solids of revolution.}
\outcome{Determine whether the Washer or Shell method is better to compute the volume of a solid.}
\outcome{Determine the appropriate variable of integration to compute the volume of a solid.}
\outcome{Set up integrals with respect to both $x$ and $y$ that give the volume of the solid.}

\title[]{Volume by Slicing}

\begin{document}
\begin{abstract}
\end{abstract}
\maketitle

\vspace{-0.9in}

\section{Discussion Questions}

\begin{problem}
\begin{enumerate}
\item[I.] The region bounded by $y=x$, $y=3x$, and $x=2$ is revolved about the line $x=5$. We wish to compute the volume of the resulting solid by integrating with respect to $x$. Which method (Washer or Shell) should be used to compute the volume?

\item[II.]  The region bounded by $y=\sqrt{x}$, $y=0$, $x=1$ and $x=4$ is revolved about the line $x=8$.  We wish to compute the volume of the resulting solid using the Washer Method. Which variable should we integrate with respect to in order to compute the volume?

\item[III.] The region bounded by $y=-e^x$, $y=0$, $x=1$ and $x=2$ is rotated about either a horizontal or a vertical line. Suppose that the volume of the resulting solid can be calculated using the Shell method, integrating with respect to $y$. Was the axis of rotation horizontal or vertical?
\end{enumerate}
\end{problem}

\begin{freeResponse}

\end{freeResponse}

\begin{problem}
Let $R$ be the region bounded by $y=4-x^2$ and $2x+y=1$.  
\begin{center}
\resizebox {6cm} {!} {
\begin{tikzpicture}
		\begin{axis}[
			domain=-2:4, ymax=6,xmax=4, ymin=-6, xmin=-2,
			axis lines =center, xlabel=$x$, ylabel=$y$,
            		every axis y label/.style={at=(current axis.above origin),anchor=south},
            		every axis x label/.style={at=(current axis.right of origin),anchor=west},
            		axis on top,
            		]
                      
            	\addplot [draw=penColor,very thick,smooth] {4-x^2};
            	\addplot [draw=penColor2,very thick,smooth] {1-2*x};
                       
            	\addplot [name path=A,domain=-1:3,draw=none] {4-x^2};   
            	\addplot [name path=B,domain=-1:3,draw=none] {1-2*x};
            	\addplot [fillp] fill between[of=A and B];

		
		\node at (axis cs:1.75,3.75) [penColor] {$y=4-x^2$};
		\node at (axis cs:1.2,-4) [penColor2] {$2x+y=1$};
                      
            	\end{axis}
	\end{tikzpicture}}
	\end{center}
	
\begin{enumerate}
\item[I.] If $R$ is revolved about the line $x=-2$, what is the minimum number of integrals required to express the volume of the resulting solid if the Washer method is used?  What if the Shell method is used?

\item[II.] If $R$ is revolved about the line $y=5$, what is the minimum number of integrals required to express the volume of the resulting solid if the Washer method is used?  What if the Shell method is used?
\end{enumerate}
\end{problem}

\begin{freeResponse}
\end{freeResponse}



\section{Group Work}

\begin{problem}
The region $R$ bounded by $y=\ln(x)$, $y=0$, $y=2$, and $x=1$ is revolved about the $y$-axis. Calculate the volume of the resulting solid using any method you like.
\end{problem}

\begin{freeResponse}

\end{freeResponse}

\begin{problem}
The region $R$ in the upper half plane, bounded by the curves $x^2+y^2 = 1$ and $y=0$, is rotated around the line $y=2$. Set up (but don't evaluate) an integral or some of integrals with respect to $y$ which could be used to compute the volume of the solid.
\end{problem}

\begin{freeResponse}

\end{freeResponse}

\begin{problem}
Let $R$ be the region bounded by $y=\sin (x)$, $y=0$, $x=\frac{\pi}{4}$ and $x = \frac{\pi}{2}$. For each of the following cases, set up (but don't evaluate) integrals which could be used to calculate the volume of the given solid using both the Shell method and the Washer method.

\begin{enumerate}
\item[I.] The solid is formed by rotating $R$ around the line $x = -1$.
\item[II.] The solid is formed by rotating $R$ around the line $y=-1$.
\end{enumerate}
\end{problem}

\begin{freeResponse}

\end{freeResponse}


\end{document}
