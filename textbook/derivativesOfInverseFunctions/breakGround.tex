\documentclass{ximera}

\newcommand{\RR}{\mathbb R}
\renewcommand{\d}{\,d}
\newcommand{\dd}[2][]{\frac{d #1}{d #2}}
\renewcommand{\l}{\ell}
\newcommand{\ddx}{\frac{d}{dx}}
\newcommand{\dfn}{\textbf}
\newcommand{\eval}[1]{\bigg[ #1 \bigg]}


\outcome{}

\title[Break-Ground:]{Derivatives of inverse functions}

\begin{document}
\begin{abstract}
Two young mathematicians discuss the derivative of inverse functions.
\end{abstract}
\maketitle

Check out this dialogue between two calculus students:

\begin{dialogue}
\item[Devyn] Riley, I've been thinking about calculus.
\item[Riley] Calculus is the greatest.
\item[Devyn] I know! Here is my idea: If I know the derivative of some
  function $f(x)$, I think I should be able to find the derivative of
  $f^{-1}(x)$.
\item[Riley] Hmmm say more.
\item[Devyn] I'll do much better than that, I'll draw pictures. Think
  about it this way, we know that the plot of $f^{-1}(x)$ is simply
  the plot of $f(x)$ flipped over the line $y=x$ right?
  \begin{image}
  PLOT FUNCTION AND INVERSE  
  \end{image}
\item[Riley] Ah! So if you know the slope of the tangent line to
  $y=f(x)$, then by ``flipping'' we should somehow be able to find the
  slope of the tangent line to $y=f^{-1}(x)$.
  \begin{image}
  PLOT FUNCTION AND INVERSE slopes 
  \end{image}
\item[Devyn] Exactly!
  
\item[Devyn] I'm going to differentiate $\sin^{-1} x$
\item[Riley] Let's use the chain rule.
\item[Devyn] Now I apply the power rule.
\item[Riley] Is that right?
\end{dialogue}

% This gets at what the notation sin^{-1} x means, and what the inverse function theorem is saying

% Contrast it with a geometric interpretation that since the inverse is the fxn flipped over the line y=x, we should have this info once we know f'

% becaue we want to actually have two answers that both seem reasonable and demand that they student resolve the contradiction in mathematics

%% \begin{xarmaBoost}
%%   Write down at least \textbf{five} questions for this lecture. After
%%   you have your questions, label them as ``Level 1,'' ``Level 2,'' or
%%   ``Level 3'' where:
%% \begin{description}
%% \item[Level 1] Means you know the answer, or know exactly how to do
%%   this problem.
%% \item[Level 2] Means you think you know how to do the problem.
%% \item[Level 3] Means you have no idea how to do the problem.
%% \end{description}
%% \begin{freeResponse}
%% \end{freeResponse}
%% \end{xarmaBoost}


\end{document}
