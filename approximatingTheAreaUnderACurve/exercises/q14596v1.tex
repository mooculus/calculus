\documentclass{ximera}
\newcommand{\RR}{\mathbb R}
\renewcommand{\d}{\,d}
\newcommand{\dd}[2][]{\frac{d #1}{d #2}}
\renewcommand{\l}{\ell}
\newcommand{\ddx}{\frac{d}{dx}}
\newcommand{\dfn}{\textbf}
\newcommand{\eval}[1]{\bigg[ #1 \bigg]}

\author{Emma Smith Zbarsky \and Bobby Ramsey}
\license{Creative Commons Attribution 3.0 Unported}
\acknowledgement{https://quadbase.org/questions/q14596v1}
\begin{document}

\begin{exercise}
% Updating wording to match textbook
Estimate the area under the curve $y=\sin^2(x)$ between $x=0$ and
$x=\pi$ using $n=3$ rectangles and using midpoints as sample points.


\begin{hint}
	This is a Riemann sum computation using a small number of rectangles. Write out the intervals
	and sample points, then compute.
\end{hint}


\begin{hint}
	We are computing a Riemann sum with $\Delta x = \frac{\pi}{3}$ over the
	interval $[0,\pi]$. Therefore the subintervals are:
	\[[0,\pi/3], [\pi/3,2\pi/3], [2\pi/3,\pi].\] The midpoints of these subintervals are $x_1^* = \dfrac{\pi}{6}$, $x_2^* = \dfrac{\pi}{2}$, 
	and $x_3^* = \dfrac{5\pi}{6}$.
	The Riemann sum is:
	\begin{align*}
		f(x_1^*)\Delta x + f(x_2^*)\Delta x + f(x_3^*)\Delta x 
			&= \sin^2\left(x_1^*\right) \Delta x + \sin^2\left(x_2^*\right) \Delta x + \sin^2\left(x_3^*\right) \Delta x\\
			&= \sin^2\left(\dfrac{\pi}{6}\right) \dfrac{\pi}{3} + \sin^2\left(\dfrac{\pi}{2}\right) \dfrac{\pi}{3} 
				+ \sin^2\left(\dfrac{5\pi}{6}\right) \dfrac{\pi}{3}\\
			&= \left( \dfrac{1}{2}\right)^2 \dfrac{\pi}{3} + \left( 1\right)^2 \dfrac{\pi}{3} + \left( -\dfrac{1}{2}\right)^2 \dfrac{\pi}{3}\\
			&= \dfrac{\pi}{3}\left( \dfrac{1}{4} + 1 + \dfrac{1}{4} \right)
	\end{align*}
%	\begin{align*}
%		\sum_{k=1}^3 f\left(\frac{x_k+x_{k-1}}{2}\right) \Delta x 
%		&= \sum_{k=1}^3 \sin^2\left(\frac{x_k+x_{k-1}}{2}\right) \left(\frac{\pi}{3}\right) \\
%		&= \left(\frac{\pi}{3}\right)\left(\sin^2\left(\frac{1}{2}\left(0+\frac{\pi}{3}\right)\right)+\sin^2\left(\frac{1}{2}\left(\frac{\pi}{3}+\frac{2\pi}{3}\right)\right) + \sin^2\left(\frac{1}{2}\left(\frac{2\pi}{3}+\pi\right)\right)\right) \\
%		&= \left(\frac{\pi}{3}\right)\left(\sin^2\left(\frac{\pi}{6}\right)+\sin^2\left(\frac{\pi}{2}\right) + \sin^2\left(\frac{5\pi}{6}\right)\right) \\
%		&= \left(\frac{\pi}{3}\right)\left(\left(\frac{1}{2}\right)^2 + \left(1\right)^2 + \left(-\frac{1}{2}\right)^2\right) \\
%		&= \frac{\pi}{3}\left(\frac{1}{4}+1+\frac{1}{4}\right) \\
%		&= \boxed{\frac{\pi}{2}}
%	\end{align*}
\end{hint}


\begin{multipleChoice}
	\choice{$1$}
	\choice{$\frac{2}{3}$}
	\choice[correct]{$\frac{\pi}{2}$}
	\choice{$0$}
	\choice{$\frac{4}{3}$}
\end{multipleChoice}

\end{exercise}
\end{document}
