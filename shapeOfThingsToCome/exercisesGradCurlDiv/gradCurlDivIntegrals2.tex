\documentclass{ximera}

\newcommand{\RR}{\mathbb R}
\renewcommand{\d}{\,d}
\newcommand{\dd}[2][]{\frac{d #1}{d #2}}
\renewcommand{\l}{\ell}
\newcommand{\ddx}{\frac{d}{dx}}
\newcommand{\dfn}{\textbf}
\newcommand{\eval}[1]{\bigg[ #1 \bigg]}


\author{Jim Fowler and Bart Snapp}

\outcome{Compute integrals in space by invoking an appropriate choice of Stokes' theorem.}

\begin{document}

Let $B$ be the sphere of radius $3$ centered at the point $(0,0,7)$, i.e.,
\[
  B = \{ (x,y,z) \in \R^3 : x^2 + y^2 + (z-7)^2 = 3^2 \}
\]
and let $H$ be the northern hemisphere meaning
\[
  H = \{ (x,y,z) \in B : z \geq 7 \}.
\]
In both cases, arrange the orientation so that the positive-flux direction at the north pole points in the direction of the positive $z$-axis.

Define a vector field $\vec{F} : \R^3 \to \R^3$ by the rule
\[
  \vec{F}(x,y,z) = \vector{z,x+y+z,y+2z}.
\]

%%%%%%%%%%%%%%%%%%%%%%%%%%%%%%%%%%%%%%%%%%%%%%%%%%%%%%%%%%%%%%%%
\begin{exercise}
  $\displaystyle\oiint_B (\curl\vec{F})\dotp\uvec{n}\d S = \answer{0}$.

  \begin{hint}
    This equals, by Stokes' theorem, $\displaystyle\oint_{\partial B} \vec{F}\dotp \d \vec{p}$.
  \end{hint}

  \begin{hint}
    But $\partial B$ is empty, so the integral vanishes.
  \end{hint}
\end{exercise}

%%%%%%%%%%%%%%%%%%%%%%%%%%%%%%%%%%%%%%%%%%%%%%%%%%%%%%%%%%%%%%%%
\begin{exercise}
  $\curl \vec{F} = \vector{\answer{0},\answer{1},\answer{1}}$.
\end{exercise}

%%%%%%%%%%%%%%%%%%%%%%%%%%%%%%%%%%%%%%%%%%%%%%%%%%%%%%%%%%%%%%%%
\begin{exercise}
  $\displaystyle\iint_H (\curl\vec{F})\dotp\uvec{n}\d S = \answer{9\pi}$

  \begin{hint}
    Let $E$ be the equator meaning
    \[
      E = \{ (x,y,z) \in B : z = 7 \},
    \]
    so that $E$ is a circle of radius $3$.
  \end{hint}

  \begin{hint}
    By Stokes' theorem, this integral is $\oint_E \vec{F} \dotp \d \vec{p}$.
  \end{hint}

  \begin{hint}
    Let $D$ be the equatorial disk
    \[
      D = \{ (x,y,z) \in \R^3 : x^2 + y^2 \leq 3^2 \mbox{ and } z = 7 \}.
    \]
  \end{hint}

  \begin{hint}
    By Stokes' theorem, these integrals are also equal to $\displaystyle\iint_D (\curl\vec{F})\dotp\uvec{n}\d S$ provided $D$ is oriented appropriately---in this case, with $\uvec{n} = \vector{0,0,1}$.
  \end{hint}

  \begin{hint}
    Since $(\curl\vec{F})\dotp\uvec{n} = 1$, these integrals are the area of $D$, which is $9\pi$.
  \end{hint}
\end{exercise}

%%%%%%%%%%%%%%%%%%%%%%%%%%%%%%%%%%%%%%%%%%%%%%%%%%%%%%%%%%%%%%%%
\begin{exercise}
  $\displaystyle\oiint_B \vec{F}\dotp\uvec{n}\d S = \answer{108\pi}$.

  \begin{hint}
    By the divergence theorem, the given integral is the same as the integral of $\divergence \vec{F}$ over the enclosed volume.
  \end{hint}

  \begin{hint}
    But $\divergence \vec{F} = 3$ and the sphere of radius $3$ has volume $36\pi$.
  \end{hint}

  \begin{hint}
    So the integral is $3 \cdot 36\pi$.
  \end{hint}
\end{exercise}

%%%%%%%%%%%%%%%%%%%%%%%%%%%%%%%%%%%%%%%%%%%%%%%%%%%%%%%%%%%%%%%%
\begin{exercise}
  $\displaystyle\iint_H \vec{F}\dotp\uvec{n} \d S = \answer{180\pi}$

  \begin{hint}
    Let $D$ be the equatorial disk
    \[
      D = \{ (x,y,z) \in \R^3 : x^2 + y^2 \leq 3^2 \mbox{ and } z = 7\}.
    \]
  \end{hint}

  \begin{hint}
    Then $H$ and $D$ (suitably oriented) together bound an solid hemisphere.
  \end{hint}

  \begin{hint}
    By the divergence theorem, the difference between the flux through $H$ and $D$ equals $\divergence \vec{F}$ integrated over that solid hemisphere.
  \end{hint}

  \begin{hint}
    But $\divergence \vec{F} = 3$ and the hemisphere of radius $3$ has volume $18\pi$.
  \end{hint}

  \begin{hint}
    So the difference between the flux through $H$ and $D$ is $54\pi$.
  \end{hint}

  \begin{hint}
    The flux through $D$ can be computed by hand: in that case, $\uvec{n} = \vector{0,0,-1}$ but then on $D$ the quantity $\vec{F}\dotp\uvec{n} = -y-14$.
  \end{hint}

  \begin{hint}
    By symmetry, $\iint_D -y \d A = 0$ and  $\iint_D -14 \d A = -126\pi$.
  \end{hint}

  \begin{hint}
    So the desired integral is $54\pi + 126\pi$.
  \end{hint}
\end{exercise}



\end{document}
