\documentclass{ximera}

\newcommand{\RR}{\mathbb R}
\renewcommand{\d}{\,d}
\newcommand{\dd}[2][]{\frac{d #1}{d #2}}
\renewcommand{\l}{\ell}
\newcommand{\ddx}{\frac{d}{dx}}
\newcommand{\dfn}{\textbf}
\newcommand{\eval}[1]{\bigg[ #1 \bigg]}


\outcome{Consider function values nearer and nearer to a given input value.}
\outcome{Understand the concept of a limit.}
\outcome{Limits as understanding local behavior of functions.}
\outcome{Calculate limits from a graph (or state that the limit does not exist).}
\outcome{Define a one-sided limit.}


\title[Dig-In:]{What is a limit?}
\begin{document}
\begin{abstract}
  We introduce limits.
\end{abstract}
\maketitle

\begin{question}
Consider the function
\[
f(x) = \frac{\sin(x)}{x}.
\]
While $f(x)$ is undefined at $x=0$, we can still plot $f(x)$ at other
values near $x = 0$. Create a table of values of $f$ below:
\[
\begin{tchart}{c|c}
  x & f(x) \\ \hline
  1 &  \answer{\sin(1)} \\
  0.1 &  \answer{10\cdot\sin(0.1)} \\
  0.01 &  \answer{100\cdot\sin(0.01)} \\
  0.001 &   \answer{1000\cdot\sin(0.001)} \\
  0 & \text{undefined}\\
  -0.001 &  \answer{1000\cdot\sin(0.001)}\\
 -0.01&   \answer{100\cdot\sin(0.01)}\\
 -0.1&   \answer{10\cdot\sin(0.1)}\\
 -1 &   \answer{\sin(1)}
\end{tchart}
\]
It appears that $f$ is approaching the value $\answer[given]{1}$ as $x$ gets
closer and closer to $0$.
\end{question}

This example leads to the following intuitive concept.  We will meet a more precise formulation of this definition in an optional section (``the $\epsilon - \delta$ definition of a limit'').

\begin{definition}[Intuitive ``definition'' of a limit]
Let $f$ be a function.  If $f(x)$ can be made arbitrarily close to $L$ by making $x$ sufficiently close, but not equal to, $a$, then we say that the \textbf{limit} of $f(x)$ as $x$ approaches $a$ is $L$.  This is written symbolically as

\[
\lim_{x\to a} f(x) = L.
\]
\end{definition}

\begin{question}

Select the appropriate way to verbalize the symbolic expression 

\[
\lim_{x \to 4} \sin(x)
\]

\begin{multipleChoice}
\choice[correct] {``The limit of sine of $x$ as $x$ approaches $4$''.}
\choice {``Lim $x$ to $4$ sine $x$''}
\choice {``Limit of $x$ approaches $4$ of sine''}
\end{multipleChoice}
\end{question}

Let's begin to explore some of the meaning of this definition.  Remember 
that a limit tells us what is happening to function values {\em near} some value of $x$.

\begin{question}
Is the following statement true for any function $f$?

``If $f(3) = 4$ then $\displaystyle \lim_{x \to 3} f(x) = 4$''

\begin{multipleChoice}
\choice[correct]{False}
\choice{True}
\end{multipleChoice}

\begin{feedback}
  For instance, if
  \[
  f(x) =
  \begin{cases}
    4 &\text{if $x=3$} \\
    0 &\text{else}
  \end{cases}
  \]
  then $f(3)=4$, but $\lim_{x \to 3} f(x) =0$.
\end{feedback}
\end{question}

The last question points to a very important concept.  Just because we know the 
behavior of a function at the point in question does not tell us anything about the 
limit value of the function.  In the previous question, we know that at $x = 3$, we 
have $f(x) = 4$.  However, this statement doesn't tell us anything about the behavior 
of the function near to $x = 3$.

Similarly, knowing the limit value does not tell us anything about the function value. 
In the feedback to the previous question, we looked at the function 
  \[
  f(x) =
  \begin{cases}
    4 &\text{if $x=3$} \\
    0 &\text{else}.
  \end{cases}
  \]
In this case, if we knew only that $\lim_{x \to 3} f(x) = 0$, we would not be able 
to predict that $f(3) = 4$.

\begin{question}
Using the graph of $f$ below, does it appear that $\lim_{x \to 1} f(x)$ exists?  

%graph

\begin{multipleChoice}
\choice[correct]{Yes}
\choice{No}
\end{multipleChoice}

%only reveal after correct answer

\begin{feedback}
	It appears that $\lim_{x\to 1} f(x) =  \answer{4}$ 
\end{feedback}


\end{question}

%replicate last question with  a removable discontinuity,  a continuous function, and finally a jump discontinuity.

%Jump discontinuity model:

\begin{question}
Using the graph of $f$ below, does it appear that $\lim_{x \to 1} f(x)$ exists?  

%graph

\begin{multipleChoice}
\choice{Yes}
\choice[correct]{No}
\end{multipleChoice}

%only reveal after correct answer

\begin{feedback}
  Even though $\lim_{x \to 1} f(x)$ does not exist, values of
  $f(x)$ appear to be approaching \answer{3} when $x \to 1$
  through numbers greater than $1$, and values of $f(x)$ appear
  to be approaching \answer{6} when $x \to 1$ through numbers
  less than $1$.
\end{feedback}

\end{question}

The last question motivates the following:

\begin{definition}[Intuitive ``definition'' of a one sided limit]
  Let $f$ be a function.  If $f(x)$ can be made arbitrarily close to $L$ by making $x>a$ sufficiently close, but not equal to, $a$, then we say that the \textbf{limit} of $f(x)$ as $x$ approaches $a$ from the right is $L$.  This is written symbolically as

  \[
\lim_{x\to a^+} f(x) = L
\]

Similarly, if $f(x)$ can be made arbitrarily close to $L$ by making
$x<a$ sufficiently close, but not equal to, $a$, then we say that the
\textbf{limit} of $f(x)$ as $x$ approaches $a$ from the left is $L$.
This is written symbolically as

\[
\lim_{x\to a^-} f(x) = L
\]

\end{definition}

% oracle one sided limits here

\begin{theorem}
	$\lim_{x \to a} f(x)$ exists if and only if  
	\begin{itemize}
	   \item $\lim_{x \to a^-} f(x)$ exists
	   \item $\lim_{x \to a^+} f(x)$ exists
	   \item $\lim_{x \to a^-} f(x) = \lim_{x \to a^+} f(x)$
	\end{itemize}
	In this case, $\lim_{x \to a} f(x)$ is equal to the common
        value of the two one sided limits.
\end{theorem}









\end{document}
