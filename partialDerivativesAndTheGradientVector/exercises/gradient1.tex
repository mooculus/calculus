\documentclass{ximera}

\newcommand{\RR}{\mathbb R}
\renewcommand{\d}{\,d}
\newcommand{\dd}[2][]{\frac{d #1}{d #2}}
\renewcommand{\l}{\ell}
\newcommand{\ddx}{\frac{d}{dx}}
\newcommand{\dfn}{\textbf}
\newcommand{\eval}[1]{\bigg[ #1 \bigg]}



\author{Jim Talamo}

\outcome{Calculate the gradient.}
\newcommand{\Pp}[2]{\frac{\partial #1}{\partial #2}}

\begin{document}

\begin{exercise}
Suppose that $f(x,y) = 4x^2+3xy+5$.  Then,

\begin{align*}
\grad{f}(x,y) &= \vector{\answer{8x+3y},\answer{3x}} \\
\grad{f}(1,2) &= \vector{\answer{14},\answer{3}}
\end{align*}

\begin{hint}
The gradient is a vector-valued function whose first component is the partial derivative of the function with respect to $x$ and whose second component is the partial derivative of the function with respect to $y$; that is

\[
\grad{f} = \vector{f_x(x,y) , f_y(x,y)}.
\]

To compute it, we just calculate the partial derivatives and insert them in the appropriate component.

\end{hint}
\end{exercise}

\end{document}
