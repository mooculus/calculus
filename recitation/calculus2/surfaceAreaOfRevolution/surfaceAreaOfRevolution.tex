\documentclass[handout]{ximera}
%handout:  for handout version with no solutions or instructor notes
%handout,instructornotes:  for instructor version with just problems and notes, no solutions
%noinstructornotes:  shows only problem and solutions

%% handout
%% space
%% newpage
%% numbers
%% nooutcomes

%I added the commands here so that I would't have to keep looking them up
%\newcommand{\RR}{\mathbb R}
%\renewcommand{\d}{\,d}
%\newcommand{\dd}[2][]{\frac{d #1}{d #2}}
%\renewcommand{\l}{\ell}
%\newcommand{\ddx}{\frac{d}{dx}}
%\everymath{\displaystyle}
%\newcommand{\dfn}{\textbf}
%\newcommand{\eval}[1]{\bigg[ #1 \bigg]}

%\begin{image}
%\includegraphics[trim= 170 420 250 180]{Figure1.pdf}
%\end{image}

%add a ``.'' below when used in a specific directory.

\newcommand{\RR}{\mathbb R}
\renewcommand{\d}{\,d}
\newcommand{\dd}[2][]{\frac{d #1}{d #2}}
\renewcommand{\l}{\ell}
\newcommand{\ddx}{\frac{d}{dx}}
\newcommand{\dfn}{\textbf}
\newcommand{\eval}[1]{\bigg[ #1 \bigg]}


\author{Jim Talamo}

\outcome{Use definite integrals to find areas of surfaces of revolution.}


\title[]{Surface Area of Revolution}

\begin{document}
\begin{abstract}
\end{abstract}
\maketitle

\vspace{-0.9in}

\section{Discussion Questions}

\begin{problem}
The segment of the curve $y=x^3$ from $x=0$ to $x=2$ is revolved about the $y$-axis.  

\begin{enumerate}
\item[I.] Sketch the curve. 
\item[II.] Show that the the length element $ds$ in terms of $x$ and $\d x$ is: \[ds = \sqrt{1+9x^4} \d x.\]
\item[III.] Write an expression for the length element $ds$ in terms of $y$ and $\d y$.
\item[IV.] Which of the following integrals gives the area of the surface of revolution?

\begin{tabular}{ll}
A. $\int_0^2 2\pi x \sqrt{1+9x^4} \d x$ & B. $\int_0^2 2\pi x^3 \sqrt{1+9x^4} \d x$  \\[4ex]
C. $\int_0^8 2\pi y \sqrt{1+\frac{1}{9}y^{-4/3}} \d y$ \qquad \qquad & D. $\int_0^8 2\pi y^{1/3} \sqrt{1+\frac{1}{9}y^{-4/3}} \d y$
\end{tabular}
\end{enumerate}
\end{problem}

\begin{freeResponse}

\end{freeResponse}

\section{Group Work}

\begin{problem}
The curve $C$ is the segment of $y=\cos(2x)$ from $x=0$ to $x=\pi/4$.  Set up an integral with respect to $x$ and an integral with respect to $y$ that gives the area of the surface of revolution when the curve is revolved about the following axes: 

\begin{tabular}{lll}
I. The $x$-axis. \qquad \qquad II. The line $x=-4$.  \qquad \qquad III. The line $y=5$.
\end{tabular}

\begin{freeResponse}

\end{freeResponse}

\end{problem}

\begin{problem}
Find the surface area of the surface generated by revolving the curve given by
	\begin{enumerate}
			\item  $x = 2y^3$ from $\left( 0, 0 \right)$ to $\left( 2, 1 \right)$ about the $y$-axis.
		\begin{freeResponse}
		The formula for the surface area is
			\[
			\text{{\color{red} Surface Area}} = \int_0^{1} 2 \pi f(y) \sqrt{1+f'(y)^2} \d y.
			\]
		Since $x = f(y) = 2y^3$, 
		we know that $f'(y) = 6y^2$.  
		Note that
			\begin{align*}
			\sqrt{1+f'(y)^2} \d y  &= \sqrt{1+ \left( 6y^2 \right)^2}  \\
			&=  \sqrt{1+ 36y^4}  \\
			\end{align*}
		and so
			\begin{align*}
			\text{{\color{red} Surface Area}} &= \int_0^{1} 2 \pi \left(2y^3\right) \left(  \sqrt{1+ 36y^4} \right) \d y  \\
			&= \int_0^{1} 4 \pi y^3  \sqrt{1+ 36y^4}  \d y \\
			& \\
			u&=1+ 36y^4 \\
			du &= 144y^3dy \\
			\frac{du}{144} &= y^3dy \\
			& \\
			u(0)&=1+36(0)^4 = 1\\
			u(1)&= 1+ 36(1)^4) = 37 \\
			& \\
			&= \frac{4 \pi}{144} \int_1^{37}  \sqrt{u}  \d u \\
			&= \frac{4 \pi}{144} \eval{\frac{2}{3}u^{\frac{3}{2}}}_1^{37}\\
			&= \frac{4 \pi}{144} \left[ \left( \frac{2}{3}(37)^{\frac{3}{2}} \right) - \left( \frac{2}{3}(1)^{\frac{3}{2}} \right) \right]  \\
			&= \frac{(37)^{\frac{3}{2}} -1 }{54}\pi
			\end{align*}
	
		\end{freeResponse}
	
	
		\item  $y = \frac{1}{6} x^3 + \frac{1}{2x}$ from $\left( 2, \frac{19}{12} \right)$ to $\left( 3, \frac{14}{3} \right)$ about the $x$-axis.
		\begin{freeResponse}
		The formula for the surface area is
			\[
			\text{{\color{red} Surface Area}} = \int_2^3 2 \pi f(x) \sqrt{1+f'(x)^2} \d x.
			\]
		Since $y = f(x) = \frac{1}{6} x^3 + \frac{1}{2x}$, we know that $f'(x) = \frac{1}{2} x^2 - \frac{1}{2} x^{-2}$.  
		Note that
			\begin{align*}
			\sqrt{1+f'(x)^2} &= \sqrt{1+ \left( \frac{1}{2}x^2 - \frac{1}{2}x^{-2} \right)^2}  \\
			&= \sqrt{1+ \left( \frac{1}{4}x^4 - \frac{1}{2} + \frac{1}{4}x^{-4} \right)}  \\
			&= \sqrt{\frac{1}{4}x^4 + \frac{1}{2} + \frac{1}{4}x^{-4}}  \\
			&= \sqrt{\left( \frac{1}{2}x^2 + \frac{1}{2}x^{-2} \right)^2}  \\
			&= \left( \frac{1}{2}x^2 + \frac{1}{2}x^{-2} \right)
			\end{align*}
		and so
			\begin{align*}
			\text{{\color{red} Surface Area}} &= \int_2^3 2 \pi \left( \frac{1}{6} x^3 + \frac{1}{2} x^{-1} \right) \left( \frac{1}{2} x^2 + \frac{1}{2} x^{-2} \right) \d x  \\
			&= 2 \pi \int_2^3 \left( \frac{1}{12} x^5 + \frac{1}{12} x + \frac{1}{4} x + \frac{1}{4} x^{-3} \right) \d x  \\
			&= 2 \pi \int_2^3 \left( \frac{1}{12} x^5 + \frac{1}{3} x + \frac{1}{4} x^{-3} \right) \d x  \\
			&= 2 \pi \eval{\frac{1}{72}x^6 + \frac{1}{6}x^2 - \frac{1}{8}x^{-2}}_2^3  \\
			&= 2\pi \left[ \left( \frac{81}{8} + \frac{3}{2} - \frac{1}{72} \right) - \left( \frac{8}{9} + \frac{2}{3} - \frac{1}{32} \right) \right]  \\
			&= 2\pi \left( \frac{2916 + 432 - 4 - 256 - 192 + 9}{288} \right)  \\
			&= \frac{2905 \pi}{144}.
			\end{align*}
		\end{freeResponse}
		
		
		

	\end{enumerate}
	
\end{problem}

\end{document}
