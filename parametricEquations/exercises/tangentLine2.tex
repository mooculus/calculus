\documentclass{ximera}

\newcommand{\RR}{\mathbb R}
\renewcommand{\d}{\,d}
\newcommand{\dd}[2][]{\frac{d #1}{d #2}}
\renewcommand{\l}{\ell}
\newcommand{\ddx}{\frac{d}{dx}}
\newcommand{\dfn}{\textbf}
\newcommand{\eval}[1]{\bigg[ #1 \bigg]}


%\outcome{Find tangent lines to parametric curves}
\author{Alex Beckwith}

\begin{document}
\begin{exercise}

Suppose that a curve $C$ is defined by the parametric equations:

\[
\begin{cases}
x(t) &= (t-1)^3 \\
y(t) &= (t+1)^2(t-2)
\end{cases} , \textrm{ for } -\infty < t < \infty 
\]

Find the derivative $\frac{dy}{dx}$ in terms of $t$.

\[
\frac{dy}{dx} = \answer{\frac{2(t+1)\cdot(t-2) +(t+1)^2}{3(t-1)^2}}
\]

\begin{hint}
Using the chain rule allows us to write $\frac{dy}{dx} = \frac{dy/dt}{dx/dt}$.
\end{hint}

%%%Horizontal Tangent Lines%%%%%%
\begin{exercise}
How many horizontal tangent lines does the curve $C$ have?
\begin{multipleChoice}
\choice{$0$}
\choice[correct]{$1$}
\choice{$2$}
\choice{$3$}
\end{multipleChoice}

The horizontal tangent line occurs when $t=\answer{-1}$.

\begin{exercise}
When $t=-1$, the horizontal tangent line is $y=\answer{0}$ and occurs at the point $(x,y) = \left(\answer{-8},\answer{0}\right)$.
\end{exercise}

%%%vertical Tangent Lines%%%%%%
\begin{exercise}
How many vertical tangent lines does the curve $C$ have?
\begin{multipleChoice}
\choice{$0$}
\choice[correct]{$1$}
\choice{$2$}
\choice{$3$}
\end{multipleChoice}

The vertical tangent line occurs when $t=\answer{1}$.

When $t=1$, the vertical tangent line is $x=\answer{0}$ and occurs at the point $(x,y) = \left(\answer{0},\answer{-4}\right)$.
\end{exercise}
\end{exercise}
\end{exercise}


\end{document}
