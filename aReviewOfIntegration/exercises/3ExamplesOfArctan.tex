\documentclass{ximera}
\newcommand{\RR}{\mathbb R}
\renewcommand{\d}{\,d}
\newcommand{\dd}[2][]{\frac{d #1}{d #2}}
\renewcommand{\l}{\ell}
\newcommand{\ddx}{\frac{d}{dx}}
\newcommand{\dfn}{\textbf}
\newcommand{\eval}[1]{\bigg[ #1 \bigg]}

\author{Jim Talamo}
\license{Creative Commons 3.0 By-NC}
\outcome{Explore the integrals that involve inverse tangents}
\begin{document}
\begin{exercise}
The following exercise gives practice working with the rule: 

\[\int \frac{1}{x^2+a^2} \d x = \frac{1}{a} \arctan\left(\frac{x}{a}\right) +C, \]

which will recur frequently in the course.  Using this formula and the rules for antidifferentiation, find the following:

\begin{prompt} (Use $C$ for the constant of integration) \end{prompt}

\begin{align*}
\int \frac{4}{x^2+1} \d x &= \answer{4\arctan(x)+C}\\
\int \frac{1}{x^2+4} \d x &= \answer{\frac{1}{2}\arctan\left(\frac{x}{2}\right)+C}\\
\int \frac{1}{4x^2+1} \d x &= \answer{\frac{1}{2} \arctan(2x)+C}\\
\end{align*}

Now, try one a little more complicated:

\[ \int \frac{3}{9x^2+16} \d x = \answer{\frac{1}{4}\arctan\left(\frac{3x}{4}\right)+C} \]

\begin{prompt} Use $C$ for the constant of integration. \end{prompt}
\end{exercise}
\end{document}