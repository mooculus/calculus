\documentclass{ximera}

\newcommand{\RR}{\mathbb R}
\renewcommand{\d}{\,d}
\newcommand{\dd}[2][]{\frac{d #1}{d #2}}
\renewcommand{\l}{\ell}
\newcommand{\ddx}{\frac{d}{dx}}
\newcommand{\dfn}{\textbf}
\newcommand{\eval}[1]{\bigg[ #1 \bigg]}


\title[Refresh:]{Tangent lines}

\begin{document}
\begin{abstract}
  Tangent lines are fundamental for understanding calculus.
\end{abstract}
\maketitle

\begin{problem}
  Several of the sections in this portion of the course deal with
  finding derivatives and equations of tangent lines.  A few points to remember:
  \begin{quote}
    The slope of a tangent line to a curve is always given by
    $\dd[y]{x}$.
  \end{quote}
  
  When we are working with polar or parametric representations of curves, we must figure out how to compute $\dd[y]{x}$ using the information we are given rather than converting to a Cartesian description of the curve.
  
  \begin{quote}
    If we have a point $(x_0, y_0)$ on a curve, the equation of the tangent line (if it exists) at that point in Cartesian coordinates is:
    \[
      y - y_0 = m_\text{tan}(x - x_0)
    \]
    where $m_\text{tan}$ is the slope of the tangent line at that point.
  \end{quote}
  We will find various different ways to find this slope depending on how we describe the curve!
    
    The purpose of this assignment is to review some of the procedures and concepts related to tangent lines from a first course in calculus that will be necessary to answer questions from the upcoming sections.
    
  \begin{multipleChoice}
    \choice[correct]{I understand}
    \choice{I do not understand}
  \end{multipleChoice}
\end{problem}

\begin{problem}
  The slope of the tangent line is given by
  \[
  m_\text{tan} = \lim_{x\to a} \frac{f(x) - f(a)}{x-a}.
  \]
  Given two points, $(a, f(a))$ and $(x,f(x))$ where $x\ne a$, the
  slope of the \textbf{secant} line that joins these points is given
  by:
  \[
  m_\text{sec} = \frac{f(x) - f(a)}{x-a}
  \]
  The limit (assuming it exists) of this quotient as $x$ approaches
  $\answer{a}$ is the slope of the \textbf{tangent} line.
\end{problem}

\begin{problem}
  Given a function $f$, 
  \begin{multipleChoice}
    \choice{$f'(x)$ is the slope of the tangent line.}
    \choice[correct]{$f'(x)$ gives a formula for the slope of the
      tangent line for any value of $x$ in the domain of $f$.}
    \choice{$f'(x)$ gives a formula for the tangent line for any value
      of $x$ in the domain of $f$.}
    \choice{$f'(x)$ is the instantaneous rate of change.}
  \end{multipleChoice}
\end{problem}

\begin{problem}
  Let's see if we can explain why the slope of the tangent line is the
  \textbf{instantaneous rate of change}:
  \begin{quote}
    The average rate of change on an interval $[a,x]$ is given by the
    \wordChoice{\choice[correct]{slope}\choice{derivative}} of the
    \wordChoice{\choice[correct]{secant line}\choice{tangent
        line},\choice{function}}. The limit of the average rate of
    change as $\answer{x}$ goes to $\answer{a}$ gives the
    \wordChoice{\choice{average}\choice[correct]{instantaneous}} rate
    of change.
  \end{quote}
\end{problem}




\begin{problem}
  Let $f(3) = 1$, $f'(3)=-4$, and $g(x) = f(x)+x^2$.
  Find the tangent line to $y= g(x)$ at $x=3$.
  \[
  y= \answer{2(x-3)+10}
  \]
\end{problem}


\begin{problem}
  Let $f(3) = 1$, $f'(3)=-4$, and $g(x) = x\cdot f(x)$.
  Find the tangent line to $y= g(x)$ at $x=3$.
  \[
  y= \answer{-11(x-3)+3}
  \]
  \begin{hint}
    Remember the product rule.
  \end{hint}
\end{problem}

\begin{problem}
  Let $f(3) = 1$, $f'(3)=-4$, and $g(x) = f(x)^2$.
  Find the tangent line to $y= g(x)$ at $x=3$.
  \[
  y= \answer{-8(x-3) + 1}
  \]
  \begin{hint}
    Remember the chain rule.
  \end{hint}
\end{problem}


\end{document}
