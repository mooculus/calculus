\documentclass{ximera}

\newcommand{\RR}{\mathbb R}
\renewcommand{\d}{\,d}
\newcommand{\dd}[2][]{\frac{d #1}{d #2}}
\renewcommand{\l}{\ell}
\newcommand{\ddx}{\frac{d}{dx}}
\newcommand{\dfn}{\textbf}
\newcommand{\eval}[1]{\bigg[ #1 \bigg]}


\author{Jim Talamo}
\license{Creative Commons 3.0 By-NC}


\outcome{Understand the relationship between the sequence of remainders and the convergence of the series.}

\begin{document}

\begin{exercise}

Consider the series $\sum_{k=2}^{\infty} \frac{1}{k \ln(k)}$.  

Does the integral test apply?

\begin{multipleChoice}
\choice[correct]{Yes.}
\choice{No.}
\end{multipleChoice}

Can we define remainders so we can approximate the value of the series?

\begin{multipleChoice}
\choice{Yes, but it is not necessary since the integral test tells us the value to which the series converges.}
\choice{Yes, because the series converges by the integral test.}
\choice[correct]{No, because the series diverges by the integral test.}
\end{multipleChoice}

\begin{feedback}
Note that the integral test does apply since $f(x) =\frac{1}{x \ln(x)}$ is continuous, positive, and decreasing for all $x \geq 2$.  Computing the vale of the necessary improper integral tells us whether the series converges or diverges.

\[
\int_2^\infty  \frac{1}{x \ln(x)} \d x = \lim_{b \to \infty} \eval{\int_2^b  \frac{1}{x \ln(x)} \d x} = \lim_{b \to \infty} \eval{\ln\big(\ln(x)\big)}_2^b = \lim_{b \to \infty} \eval{\ln\big(\ln(b)\big) - \ln\big(\ln(2)\big) }.
\]

Here, the substitution $u=\ln(x)$ is helpful to find the necessary antiderivative. We note that since $\ln(b) \rightarrow \infty$ as $b \rightarrow \infty$, $\ln\big(\ln(b)\big) \rightarrow \infty$ as well.   Hence, the improper integral diverges, and by the integral test, the series does too.  Since the series diverges, we cannot define a remainders. 
\end{feedback}

\end{exercise}
\end{document}
