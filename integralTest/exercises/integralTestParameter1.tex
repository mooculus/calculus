\documentclass{ximera}

\newcommand{\RR}{\mathbb R}
\renewcommand{\d}{\,d}
\newcommand{\dd}[2][]{\frac{d #1}{d #2}}
\renewcommand{\l}{\ell}
\newcommand{\ddx}{\frac{d}{dx}}
\newcommand{\dfn}{\textbf}
\newcommand{\eval}[1]{\bigg[ #1 \bigg]}


\author{Jim Talamo}
\license{Creative Commons 3.0 By-bC}


\outcome{}


\begin{document}
\begin{exercise}
Suppose that $p$ is a real number and consider the series $\sum_{k=2}^{\infty} \frac{1}{k\ln^p(k)}$.  For this part of the exercise, set $p=2$; that is, we consider the series 

\[
\sum_{k=2}^{\infty} \frac{1}{k\ln^2(k)}
\]

Which of the following should be done to check whether the series converges or diverges?
\begin{multipleChoice}
\choice{Recognize this as a geometric series then apply results for convergence of geometric series.}
\choice{Recognize this as a $p$-series then apply results for convergence of geometric series.}
\choice{Recognize that $\lim_{n \to \infty} \frac{1}{n\ln^2(n)} = 0$, hence the series converges.}
\choice[correct]{Apply the integral test.}
\end{multipleChoice}

\begin{exercise}
The integral test applies here since by letting $f(x) = \frac{2}{k\ln^2(k)}$:
\begin{selectAll}
\choice[correct]{$f(x)$ is decreasing.}
\choice[correct]{$f(x) \geq 0$ for all $x \geq 2$.}
\choice[correct]{$f(x)$ is continuous for $x \geq 2$.}
\end{selectAll}

To use the integral test, we must determine if the improper integral

\[
\int_2^{\infty} \frac{1}{x\ln^2(x)} \d x
\]
converges or diverges, and we find that the improper integral \wordChoice{\choice[correct]{converges}\choice{diverges}}.

\begin{hint}
The substitution $u=\ln(x)$ is helpful.
\end{hint}
\begin{exercise}

Hence, the series  $\sum_{k=2}^{\infty} \frac{1}{k\ln^2(k)}$
\begin{multipleChoice}
\choice[correct]{converges by the integral test.}
\choice{diverges by the integral test.}
\end{multipleChoice}

\begin{exercise}
Now, consider the series $\sum_{k=2}^{\infty} \frac{1}{k\ln^p(k)}$, where $p$ is a real number.  By applying the integral test, we can find all of the values for $p$ for which the series converges, and we find that $\sum_{k=2}^{\infty} \frac{1}{k\ln^p(k)}$ converges for $p$ \wordChoice{\choice{$<$}\choice{$\leq$}\choice[correct]{$>$}\choice{$\geq$}} $\answer{1}$.

\begin{hint}
By making the substitution $u= \ln(x)$, we find that $\int \frac{1}{x\ln^p(x)} \d x = \frac{(\ln(x))^{\answer{1-p}}}{1-p} +C$.  The integral  $\int_2^{\infty} \frac{1}{x\ln^p(x)} \d x $ will thus diverge when the exponent above is positive.
\end{hint}
\end{exercise}
\end{exercise}
\end{exercise}
\end{exercise}

\end{document}
