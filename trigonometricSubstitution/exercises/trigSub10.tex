\documentclass{ximera}

\newcommand{\RR}{\mathbb R}
\renewcommand{\d}{\,d}
\newcommand{\dd}[2][]{\frac{d #1}{d #2}}
\renewcommand{\l}{\ell}
\newcommand{\ddx}{\frac{d}{dx}}
\newcommand{\dfn}{\textbf}
\newcommand{\eval}[1]{\bigg[ #1 \bigg]}


\author{Jim Talamo and Jason Miller}
\license{Creative Commons 3.0 By-NC}


\outcome{}


\begin{document}
\begin{exercise}
Compute the definite integral. 

\[
\int_{7}^{10} \frac{\sqrt{x^{2}-8x+7}}{x-4} \d x
\]



In order to determine the integral, we try a trig substitution.  Note that to use this, we need to recognize the expression under the radical as a sum or difference of perfect squares, so our first step is to complete the square:

\[
x^2-8x+7 = \left(x - \answer{4}\right)^2+\answer{-9}
\]

\begin{exercise}
Thus, we should use $x=\answer{3\sec(\theta)+4}$ and $\d x= \answer{ 3\sec(\theta)\tan(\theta)} \d \theta$. 


\begin{exercise}
We have two possible options. We could determine the antiderivative in terms of $\theta$ 
and then convert back to $x$ and then evaluate using our given bounds. 
Another option is that we could transform the $x$ bounds on our original integral to $\theta$ bounds. 

Using our substitution, we can convert our bounds. Note that the bounds on our integral guarantee that $0\leq \theta \leq \frac{\pi}{2}$. 

When $x=7$, $\theta=\answer{0}$. 

When $x=10$, $\theta=\answer{\frac{\pi}{3}}$. 



\begin{exercise}


Now we can express our original definite integral as a trigonometric integral in terms of the variable $\theta$. 

\[
\int_{7}^{10} \frac{\sqrt{x^{2}-8x+7}}{x-4}  \d x= \int_{\answer{0}}^{\answer{\frac{\pi}{3}}}  \answer{3\tan^{2}(\theta)}   \d \theta
\]

\begin{exercise}
To proceed, we will need to compute $\int \tan^2(\theta) \d x$:

\[
\int \tan^2(\theta) \d x = \answer{\tan(\theta)-\theta}
\]

\begin{hint}
A particular Pythagorean trigonometric identity might be helpful!
\end{hint}

\begin{exercise}
Finally we calculate our original definite integral: 

\[
\int_{7}^{10} \frac{\sqrt{x^{2}-8x+7}}{x-4} \d x=\int_{0}^{\frac{\pi}{3}} 3\tan^{2}(\theta) \d \theta=\answer{3\sqrt{3}-\pi}
\]


\end{exercise}
\end{exercise}
\end{exercise}
\end{exercise}
\end{exercise}

\end{exercise}
\end{document}
