\documentclass{ximera}

\newcommand{\RR}{\mathbb R}
\renewcommand{\d}{\,d}
\newcommand{\dd}[2][]{\frac{d #1}{d #2}}
\renewcommand{\l}{\ell}
\newcommand{\ddx}{\frac{d}{dx}}
\newcommand{\dfn}{\textbf}
\newcommand{\eval}[1]{\bigg[ #1 \bigg]}


\outcome{Know the graphs and properties of ``famous'' functions.}

\title[Break-Ground:]{How crazy could it be?}

\begin{document}
\begin{abstract}
  Two young mathematicians think about the plots of functions.
\end{abstract}
\maketitle

Check out this dialogue between two calculus students (based on a true
story):

\begin{dialogue}
\item[Devyn] Riley, do you remember when we first starting graphing
  functions? Like with a ``T-chart?''
\item[Riley] I remember everything.
\item[Devyn] I used to get so excited to plot stuff! I would wonder:
  ``What crazy curve would be drawn this time? What crazy picture will
  I see?''
\item[Riley] Then we learned about the slope-intercept form of a
  line. Good-old
  \[
  y = mx +b.
  \]
\item[Devyn] Yeah, but lines are really boring. What about
  polynomials? What could you tell me about
  \[
  y= 5x^6-5x^5-5x^4+5x^3+x^2 -1
  \]
  just by looking at the equation?
\item[Riley] Hmmmm. I'm not sure\dots
\end{dialogue}

\begin{problem}
  When $x$ is a large number (furthest from zero), which term of
  $5x^6-5x^5-5x^4+5x^3+x^2 -1$ is largest (furthest from zero)?
  \begin{multipleChoice}
    \choice{$-1$}
    \choice{$x^2$}
    \choice{$5x^3$}
    \choice{$-5x^4$}
    \choice{$-5x^5$}
    \choice[correct]{$5x^6$}
  \end{multipleChoice}
\end{problem}

\begin{problem}
  When $x$ is a small number (near zero), which term of
  $5x^6-5x^5-5x^4+5x^3+x^2 -1$ is largest (furthest from zero)?
  \begin{multipleChoice}
    \choice[correct]{$-1$}
    \choice{$x^2$}
    \choice{$5x^3$}
    \choice{$-5x^4$}
    \choice{$-5x^5$}
    \choice{$5x^6$}
  \end{multipleChoice}
\end{problem}


\begin{problem}
  Very roughly speaking, what does the graph of
  $y=5x^6-5x^5-5x^4+5x^3+x^2 -1$ look like?
  \begin{multipleChoice}
    \choice{The graph starts in the lower left and ends in the upper
      right of the plane.}
    \choice{The graph starts in the lower right and ends in the upper
      left of the plane.}
    \choice[correct]{The graph looks something like the letter ``U.''}
    \choice{The graph looks something like an upside down letter ``U.''}
  \end{multipleChoice}
\end{problem}



%% \begin{xarmaBoost}
%%   Write down at least \textbf{five} questions for this lecture. After
%%   you have your questions, label them as ``Level 1,'' ``Level 2,'' or
%%   ``Level 3'' where:
%% \begin{description}
%% \item[Level 1] Means you know the answer, or know exactly how to do
%%   this problem.
%% \item[Level 2] Means you think you know how to do the problem.
%% \item[Level 3] Means you have no idea how to do the problem.
%% \end{description}
%% \begin{freeResponse}
%% \end{freeResponse}
%% \end{xarmaBoost}



\end{document}
