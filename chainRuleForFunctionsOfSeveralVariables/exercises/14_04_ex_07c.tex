\documentclass{ximera}

\newcommand{\RR}{\mathbb R}
\renewcommand{\d}{\,d}
\newcommand{\dd}[2][]{\frac{d #1}{d #2}}
\renewcommand{\l}{\ell}
\newcommand{\ddx}{\frac{d}{dx}}
\newcommand{\dfn}{\textbf}
\newcommand{\eval}[1]{\bigg[ #1 \bigg]}


\author{David Guichard \and Neal Koblitz \and H. Jerome Keisler \and Albert Scheller \and Barry Balof \and Mike Wills \and Matthew Carr \and Bart Snapp}
\license{CC-By-SA-NC}

\acknowledgement{https://www.whitman.edu/mathematics/multivariable/}

\outcome{Compute the derivative of the compostion of a function of
  several variables with a vector-valued function.}

\begin{document}
\begin{exercise}
Consider the ideal gas law, given by $PV=nRT$, relating pressure, $P$,
volume, $V$, and the temperature $T$ of $n$ moles of gas, where $R$ is
the ideal gas constant. We can view $P$, $V$ and $T$ each as functions
of the other two variables.

If the pressure of a gas is decreasing at a rate of $0.4\unit{Pa/min}$
and the volume is increasing at a rate of $3\unit{K/min}$, how fast is
the temperature changing? Express your answer in terms of $P$, $V$,
$n$ and $R$ alone.

\begin{prompt}
\[
\dd[T]{t}=\answer{\frac{3P-0.4 V}{n R}}
\]
\end{prompt}

\end{exercise}
\end{document}
