\documentclass{ximera}

\newcommand{\RR}{\mathbb R}
\renewcommand{\d}{\,d}
\newcommand{\dd}[2][]{\frac{d #1}{d #2}}
\renewcommand{\l}{\ell}
\newcommand{\ddx}{\frac{d}{dx}}
\newcommand{\dfn}{\textbf}
\newcommand{\eval}[1]{\bigg[ #1 \bigg]}


\author{Bart Snapp}

\outcome{Give the equation for a torus.}

\title[Dig-In:]{Drawing a torus}

\begin{document}
\begin{abstract}
  Learn how to draw a torus.
\end{abstract}
\maketitle

A common shape studied in mathematics is a \dfn{torus} or
\textit{donut}. To draw a torus by-hand like a pro is easy. Start by
drawing an ellipse:
\begin{image}
  \begin{tikzpicture}  
    \draw [ultra thick, penColor] (0,0) ellipse (3cm and 2cm);
  \end{tikzpicture}  
\end{image}

Now make that ellipse \textit{smile}!

\begin{image}
  \begin{tikzpicture}  
    \draw [ultra thick, penColor] (0,0) ellipse (3cm and 2cm);
    \draw [ultra thick, penColor] (-1.5,.5) arc (198:341:1.58);
  \end{tikzpicture}  
\end{image}

Finally, add in an upside down arc:

\begin{image}
  \begin{tikzpicture}  
    \draw [ultra thick, penColor] (0,0) ellipse (3cm and 2cm);
    \draw [ultra thick, penColor] (-1.5,.5) arc (198:341:1.58);
    \draw [ultra thick, penColor] (1.22,0) arc (39:141:1.57);
  \end{tikzpicture}  
\end{image}

And like magic, we have drawn a torus! On the other hand, if you want
to use a computer to draw a torus, perhaps you should use the
parametric formula:
\begin{align*}
  x(\theta,\phi) &= (R + r\cdot \cos(\phi))\cos(\theta)\\
  y(\theta,\phi) &= (R + r\cdot \cos(\phi))\sin(\theta)\\
  z(\theta,\phi) &= r\cdot \sin(\phi)
\end{align*}
where $R$ is the radius from the center of the torus to the center of
the ``tube,'' $r$ is the radius of the ``tube,'' $0\le \theta<2\pi$,
and $0\le \phi<2\pi$. However, listen, \textit{you} could have derived
a parametric formula using the techniques you've learned. Let's do it.

\begin{example}
  Use unit tangent vectors and unit normal vectors to derive a
  parametric formula for a torus.
  \begin{explanation}
    Imagine a circular curve in $\R^3$ that runs through the donut,
    shown in the diagram from the previous problem by the two ``dots.''
    \begin{image}
      \includegraphics{transdonut.jpg}
    \end{image}
    Give a formula for a vector-valued function $\vec{p}(\theta)$ that will
    draw a circle in the $(x,y)$-plane, centered at the origin, of radius
    $R$, as $\theta$ runs from $0$ to $2\pi$.
    \[
    \vec{p}(\theta) = \vector{\answer[given]{R\cos(\theta)},\answer[given]{R\sin(\theta)},\answer{0}}
    \]
    Compute $\utan(\theta)$, the function that will give the unit tangent
    vector for any value of $\theta$. \textbf{Simplify your answer.}
    \[
    \utan(\theta) = \vector{\answer[given]{-\sin(\theta)},\answer[given]{\cos(\theta)},\answer[given]{0}}
    \]
    Compute $\unormal(\theta)$, the function that will give the principal
    unit normal vector for any value of $\theta$. \textbf{Simplify your answer.}
    \[
    \unormal(\theta) =\vector{\answer[given]{-\cos(\theta)},\answer[given]{-\sin(\theta)},\answer[given]{0}}
    \]
    Compute $\ubinormal(\theta)$, the function that will give the 
    unit binormal vector for any value of $\theta$. \textbf{Simplify your answer.}
    \begin{align*}
      \ubinormal(\theta) &= \utan(\theta)\cross\unormal(\theta)\\
      &=\vector{\answer[given]{0},\answer[given]{0},\answer[given]{1}}
    \end{align*}
    Now if we put these together we can write our torus as $\vec{T}(\theta,\phi)$
    \[
    \vec{T}(\theta,\phi) = \vec{p}(\theta) + \answer[given]{r}\cdot \unormal(\theta)\cos(\answer[given]{\phi}) + \answer[given]{r}\cdot \ubinormal(\theta)\sin(\answer[given]{\phi})
    \]
    Simplifying and writing the components of this formula out we see:
    \begin{align*}
      x(\theta,\phi) &= \answer[given]{(R - r\cdot \cos(\phi))\cos(\theta)}\\
      y(\theta,\phi) &= \answer[given]{(R - r\cdot \cos(\phi))\sin(\theta)}\\
      z(\theta,\phi) &= \answer[given]{r\cdot \sin(\phi)}
    \end{align*}
    And done. We've given a parametric formula for a torus.
  \end{explanation}
\end{example}

\begin{question}
  If you're paying attention, you may notice that we now have a very
  similar formula to the one given above, except that we have some
  minus signs where before we had plus signs. What happened here?
  \begin{prompt}
    \begin{multipleChoice}
      \choice{We made a mistake in our work in the example above.}
      \choice{We lied to you when we give the initial parametric formula for the torus.}
      \choice{We just broke math.}
      \choice[correct]{Everybody wins. Both formulas draw a torus.}
    \end{multipleChoice}
    \begin{feedback}[correct]
      Both formulas are correct, the first one we gave was derived
      using ``outward'' pointing normal vectors. The second one we
      gave was derived using ``inward'' pointing normal vectors.
    \end{feedback}
  \end{prompt}
\end{question}


\end{document}
