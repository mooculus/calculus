\documentclass{ximera}

\newcommand{\RR}{\mathbb R}
\renewcommand{\d}{\,d}
\newcommand{\dd}[2][]{\frac{d #1}{d #2}}
\renewcommand{\l}{\ell}
\newcommand{\ddx}{\frac{d}{dx}}
\newcommand{\dfn}{\textbf}
\newcommand{\eval}[1]{\bigg[ #1 \bigg]}


\author{Jim Fowler}

\outcome{Use the product rule to explore why constant speed motion experience acceleration orthogonal to the direction of motion.}

\begin{document}

\begin{exercise}

  Suppose $\vec{p} : \R \to \R^3$ is a vector-valued function; its image is a curve in $\R^3$.

  Let $\vec{v}(t) = \vec{p}'(t)$ be the derivative, and $\vec{a}(t) = \vec{p}''(t)$ be the second derivative.

  Suppose for all $t \in \R$ that $|\vec{v}(t)| = 1$.

  Then $\vec{a}(t) \dotp \vec{v}(t) = \answer{0}$.

  \begin{hint}
    We are told that $\vec{v}(t) \dotp \vec{v}(t) = 1$.
  \end{hint}

  \begin{hint}
    Therefore $\dd{t} \left( \vec{v}(t) \dotp \vec{v}(t) \right) = 0$.
  \end{hint}

  \begin{hint}
    By the product rule, $\dd{t} \left( \vec{v}(t) \dotp \vec{v}(t) \right)$ is $\vec{a}(t) \dotp \vec{v}(t) + \vec{v}(t) \dotp \vec{a}(t)$.
  \end{hint}

  \begin{hint}
    By commutativity of dot product, this means that $\dd{t} \left( \vec{v}(t) \dotp \vec{v}(t) \right)$ is $2 \cdot \vec{a}(t) \dotp \vec{v}(t)$.
  \end{hint}

  \begin{hint}
    Since $\dd{t} \left( \vec{v}(t) \dotp \vec{v}(t) \right)$ vanishes, it must be that $\vec{a}(t) \dotp \vec{v}(t)$ vanishes as well.
  \end{hint}

  \begin{hint}
    Therefore we conclude $\vec{a}(t) \dotp \vec{v}(t) = 0$.
  \end{hint}  
  
\end{exercise}
\end{document}
