\documentclass{ximera}

\newcommand{\RR}{\mathbb R}
\renewcommand{\d}{\,d}
\newcommand{\dd}[2][]{\frac{d #1}{d #2}}
\renewcommand{\l}{\ell}
\newcommand{\ddx}{\frac{d}{dx}}
\newcommand{\dfn}{\textbf}
\newcommand{\eval}[1]{\bigg[ #1 \bigg]}


\author{Jim Fowler}

\outcome{Apply the sum rule for vector-valued derivatives.}

\begin{document}

\begin{exercise}

  Suppose $\vec{p} : \R \to \R^3$ and $\vec{q} : \R \to \R^3$ are vector-valued functions.

  Further assume $\vec{p}'(-1) = \vector{1,2,3}$ and $\vec{q}'(-1) = \vector{1,1,-2}$.

  Let $\vec{v}(t) = \vec{p}(t) + \vec{q}(t)$ be yet another vector-valued function.

  Then $\vec{v}'(-1) = \vector{\answer{2},\answer{3},\answer{1}}$.
  
\end{exercise}
\end{document}


%%% Local Variables:
%%% mode: latex
%%% TeX-master: t
%%% End:
