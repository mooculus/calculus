\documentclass{ximera}

\newcommand{\RR}{\mathbb R}
\renewcommand{\d}{\,d}
\newcommand{\dd}[2][]{\frac{d #1}{d #2}}
\renewcommand{\l}{\ell}
\newcommand{\ddx}{\frac{d}{dx}}
\newcommand{\dfn}{\textbf}
\newcommand{\eval}[1]{\bigg[ #1 \bigg]}


\author{Jim Talamo}

\outcome{Find domain for a piecewise function.}
\outcome{Expose students to algebra that is necessary for optimization problems.}

%This type of problem gets at some of my main goals - to have students analyze what they are given in hopes of developing some computational intuition and familiarity, to disseminate important algebraic skills they will need later (optimization and working with intersection conditions -which arises in a different algebraic context for multiple integrals), and to get them to have to think a bit in general.

\begin{document}
\begin{exercise}
Consider the function $f(x,y) =  \begin{cases} 2x+3y , & x+y^2 \neq 6 \\ \frac{x+2y}{x^2-y^2} , & x+y^2=6 \end{cases} $ . 

Find the following values.

\begin{align*}
f(1,5) &= \answer{17} \\
f(5,1) &= \answer{\frac{7}{24}}
\end{align*}

How many points $(x,y)$ are not in the domain of the function?

\begin{multipleChoice}
\choice{None; the domain is $\R^2$.}
\choice{One}
\choice{Two}
\choice{Three}
\choice[correct]{Four}
\choice{More than four, but finitely many}
\choice{Infinitely many}
\end{multipleChoice}

\begin{exercise}
List the points that are not in the domain of the function.

\[
\left( \answer{-3}, \answer{-3}\right) , \left( \answer{-3}, \answer{3} \right) , \left( \answer{2}, \answer{-2} \right) , \textrm{ and } \left(\answer{2}, \answer{2} \right)
\]

(type the points with the smaller $x$-values first.  If two points have the same $x$-value, type the one with the smaller $y$-value first)


\begin{hint}
Note that the top piece is defined for all $(x,y)$, and the bottom is undefined only when $x^2-y^2=0$.  Since the bottom piece is defined only for $(x,y)$ where $x+y^2=6$, we must find all points $(x,y)$ for which:

\begin{align}
x^2-y^2&=0 \\
x+y^2&=6.
\end{align}

From the first condition, we have $y^2=x^2$, so substituting this into the second one gives

\begin{align*}
x+x^2 &=6 \\
x^2+x-6 &= 0
\end{align*}

From this, $x=\answer{-3}$ or $x=\answer{2}$ (list the smaller $x$-value first).

Since $y^2=x^2$, when $x=-3$, $y= \answer{-3}$ or $y=\answer{3}$ (type the smaller $y$-value in first).

\end{hint}
\end{exercise}


\end{exercise}
\end{document}
