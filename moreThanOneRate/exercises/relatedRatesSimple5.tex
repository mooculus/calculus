\documentclass{ximera}
\newcommand{\RR}{\mathbb R}
\renewcommand{\d}{\,d}
\newcommand{\dd}[2][]{\frac{d #1}{d #2}}
\renewcommand{\l}{\ell}
\newcommand{\ddx}{\frac{d}{dx}}
\newcommand{\dfn}{\textbf}
\newcommand{\eval}[1]{\bigg[ #1 \bigg]}

\author{Steven Gubkin\and Nela Lakos}
\license{Creative Commons 3.0 By-NC}

\outcome{Solve basic related rates word problems.}
\outcome{Understand the process of solving related rates problems.}
\outcome{Calculate derivatives of expressions with multiple variables implicitly.}
\begin{document}

\begin{exercise}

A right circular cone is growing.  As it grows, its height remains
equal to its diameter. The diameter is growing at a constant rate of
$2 \frac{\textrm{units}}{\textrm{s}}$. At what rate is its volume
growing at the time the diameter is $6 \textrm{units}$?


SOLUTION: 

 Let $V$be the volume, $r$ the radius, $D$ the diameter,  and $h$ the height  of the cone.


The given rate is

 \[
 \dd[\answer{D}]{t}= 2 \frac{\textrm{units}}{\textrm{s}}.
  \]
  
  The rate to be determined is
  
   \[
 \Bigl[ \dd[\answer{V}]{t} \Bigr]_{D=6}
  \]

\begin{hint}
\begin{image}
\begin{tikzpicture}
%\draw[penColor,very thick] (0,4) ellipse (4 and 1);
\draw[very thick,penColor!20!background] (2,2) arc (0:180:2 and .5);% top half of ellipse
\draw[very thick,penColor] (-2,2) arc (180:360:2 and .5);% bottom half of ellipse
%\draw[penColor, very thick] (0,4) -- (4,4);
\draw[dashed,penColor!50!background, very thick] (-2,2) -- (2,2);
\draw[ penColor,very thick] (-2,2) -- (0,6.25);
\draw[ penColor,very thick] (2,2) -- (0,6.25);
\draw[ dashed,penColor2,very thick] (0,2) -- (0,6.25);
\node[above, penColor] at (0,1.62) {$D$ };
\node[right, penColor2] at (0,4.1) {$h$ };
\end{tikzpicture}
\end{image}
\end{hint}


\begin{hint}
We know that
\begin{align*}
	D &= 2r\\
	V &= \frac{1}{3} \pi r^2 h\\
	h &= D.\\
\end{align*}
\end{hint}
\begin{hint}
  While we could blindly differentiate, and everything would work out
  fine, it pays to do a little algebra first since we only care about
  the relationship between the volume and the diameter.  A little
  rearranging yields
\[
V = \frac{1}{3} \pi \left(\frac{D}{2}\right)^2D.
\]
So
\[
V = \frac{1}{12} \pi D^3.
\]
\end{hint}

\begin{hint}
  Differentiating with respect to time, we obtain
  \[
  \dd[V]{t} = \frac{1}{4} \pi D^2 \dd[D]{t}.
  \]
\end{hint}

\begin{hint}
  At the time of interest, we have
  \[
 \Bigl[ \dd[V]{t} \Bigr]_{D=6}= \frac{1}{4} \pi (6)^2 (2) = 18\pi\frac{\textrm{units}^3}{\textrm{s}}.
  \]
\end{hint}

\begin{prompt}
  \[
 \Bigl[ \dd[V]{t} \Bigr]_{D=6} = \answer{18\pi}\frac{\textrm{units}^3}{\textrm{s}}
  \]
\end{prompt}

\end{exercise}

\end{document}
