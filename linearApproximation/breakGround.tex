\documentclass{ximera}

\newcommand{\RR}{\mathbb R}
\renewcommand{\d}{\,d}
\newcommand{\dd}[2][]{\frac{d #1}{d #2}}
\renewcommand{\l}{\ell}
\newcommand{\ddx}{\frac{d}{dx}}
\newcommand{\dfn}{\textbf}
\newcommand{\eval}[1]{\bigg[ #1 \bigg]}


\outcome{}

\title[Break-Ground:]{Replacing curves with lines}

\begin{document}
\begin{abstract}
Two young mathematicians discuss linear approximation.
\end{abstract}
\maketitle

Check out this dialogue between two calculus students (based on a true
story):



\begin{dialogue}
\item[Devyn] Hmmmm. Riley, I just thought of something\dots
\item[Riley] What is it?
\item[Devyn] When we compute derivatives, we are looking at the slope
  of tangent lines right?
\item[Riley] You know it.
\item[Devyn] Well, I wonder: Instead of studying curves, could we just
  study ``zoomed-in'' lines?
\item[Riley] I'm not sure\dots
\end{dialogue}


You read someplace that
\[
\l(x) = \frac{1}{4}(x-4)+2
\]
is a good approximation for $f(x) = \sqrt{x}$ when $x$ is close to
$4$.

\begin{problem}
  Plot $\l(x)$ and $f(x)$. Explain how this shows that $\l(x)$ is a
  good approximation when $x$ is close to $4$.
  \begin{prompt}
  \begin{freeResponse}
  \end{freeResponse}
  \end{prompt}
\end{problem}

\begin{problem}
 Explain (if you can) using concepts of calculus to explain why
 $\l(x)$ is a good approximation for $f(x)$ when $x$ is close to $4$.
 \begin{prompt}
  \begin{freeResponse}
  \end{freeResponse}
 \end{prompt}
\end{problem}




%%% \begin{xarmaBoost}
%%   Write down at least \textbf{five} questions for this lecture. After
%%   you have your questions, label them as ``Level 1,'' ``Level 2,'' or
%%   ``Level 3'' where:
%% \begin{description}
%% \item[Level 1] Means you know the answer, or know exactly how to do
%%   this problem.
%% \item[Level 2] Means you think you know how to do the problem.
%% \item[Level 3] Means you have no idea how to do the problem.
%% \end{description}
%% \begin{freeResponse}
%% \end{freeResponse}
%% \end{xarmaBoost}


\end{document}
