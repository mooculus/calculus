\documentclass{ximera}

\newcommand{\RR}{\mathbb R}
\renewcommand{\d}{\,d}
\newcommand{\dd}[2][]{\frac{d #1}{d #2}}
\renewcommand{\l}{\ell}
\newcommand{\ddx}{\frac{d}{dx}}
\newcommand{\dfn}{\textbf}
\newcommand{\eval}[1]{\bigg[ #1 \bigg]}


\outcome{Define accumulation functions.}
\outcome{Calculate and evaluate accumulation functions.}
\outcome{State the First Fundamental Theorem of Calculus.}
\outcome{Take derivatives of accumulation functions using the First Fundamental Theorem of Calculus.}
\outcome{Use accumulation functions to find information about the original function.}
\outcome{Understand the relationship between the function and the derivative of its accumulation function.}

\author{Nela Lakos \and Kyle Parsons}

\begin{document}
\begin{exercise}

The graph of $g$ on $[0,6]$ is given below.

\begin{image}
  \begin{tikzpicture}
    \begin{axis}[
        xmin=-0.3,xmax=6.3,ymin=-2.3,ymax=2.3,
        clip=true,
        unit vector ratio*=1 1 1,
        axis lines=center,
        grid = major,
        ytick={-4,-3,...,36},
        xtick={0,1,...,10},
        xlabel=$t$, ylabel=$y$,
        every axis y label/.style={at=(current axis.above origin),anchor=south},
        every axis x label/.style={at=(current axis.right of origin),anchor=west},
      ]
      \addplot[ultra thick,penColor,domain=0:4] {2-x};
      \addplot[ultra thick,penColor,domain=4:6] {-2};
        
      \node at (axis cs:1.5,1.5) {$y=g(t)$};
      \end{axis}`
  \end{tikzpicture}
\end{image}

Let $A(x) = \int_0^x g(t) \d t$ for $0\leq x\leq6$.

Evaluate the following.
\begin{align*}
A(6) &= \answer{-4}\\
A'(4) &= \answer{-2}\\
\int_0^6 \left|g(t)\right| \d t &= \answer{8}
\end{align*}

\end{exercise}
\end{document}