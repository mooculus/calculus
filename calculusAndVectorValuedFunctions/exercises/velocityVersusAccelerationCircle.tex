\documentclass{ximera}

\newcommand{\RR}{\mathbb R}
\renewcommand{\d}{\,d}
\newcommand{\dd}[2][]{\frac{d #1}{d #2}}
\renewcommand{\l}{\ell}
\newcommand{\ddx}{\frac{d}{dx}}
\newcommand{\dfn}{\textbf}
\newcommand{\eval}[1]{\bigg[ #1 \bigg]}


\author{Jim Fowler}

\outcome{Use the product rule to explore why constant speed motion experience acceleration orthogonal to the direction of motion.}

\begin{document}

\begin{exercise}

  Suppose $\vec{p} : \R \to \R^2$ is the vector-valued function given by the rule
  \[
    \vec{p}(t) = \vector{\cos t, \sin t}.
  \]
  Let $\vec{v}(t) = \vec{p}'(t)$ be the derivative, and $\vec{a}(t) = \vec{p}''(t)$ be the second derivative.

  What is $\vec{a}(t) \dotp \vec{v}(t)$?  It is $\answer{0}$.

  \begin{hint}
    We compute $\vec{v}(t) = \vector{-\sin\left(t\right),\,\cos\left(t\right)}$.
  \end{hint}

  \begin{hint}
    We further compute $\vector{-\cos\left(t\right),\,-\sin\left(t\right)}$.
  \end{hint}

  \begin{hint}
    Therefore $\vec{a}(t) \dotp \vec{v}(t)$ equals $\vector{-\cos\left(t\right),\,-\sin\left(t\right)} \dotp \vector{-\sin\left(t\right),\,\cos\left(t\right)}$.
  \end{hint}

  \begin{hint}
    But this is $\cos\left(t\right) \cdot \sin\left(t\right) - \sin\left(t\right) \cos\left(t\right)$ which is zero.
  \end{hint}

  \begin{hint}
    Therefore we conclude $\vec{a}(t) \dotp \vec{v}(t) = 0$.
  \end{hint}
  
\end{exercise}
\end{document}


%%% Local Variables:
%%% mode: latex
%%% TeX-master: t
%%% End:
