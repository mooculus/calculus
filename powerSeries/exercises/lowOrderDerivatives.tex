\documentclass{ximera}

\newcommand{\RR}{\mathbb R}
\renewcommand{\d}{\,d}
\newcommand{\dd}[2][]{\frac{d #1}{d #2}}
\renewcommand{\l}{\ell}
\newcommand{\ddx}{\frac{d}{dx}}
\newcommand{\dfn}{\textbf}
\newcommand{\eval}[1]{\bigg[ #1 \bigg]}


\author{Jim Talamo}
\license{Creative Commons 3.0 By-bC}


\outcome{}


\begin{document}
\begin{exercise}
Let $f(x) = \sum_{k=0}^{\infty} \frac{4^k}{k!}(x-1)^{k+1}$.  Find $f(1)$, $f'(1)$, and $f''(1)$.

\[
f(1) = \answer{0} \qquad f'(1) = \answer{1} \qquad f''(1) = \answer{8}
\]

\begin{hint}
In order to work this, write out a few terms in the function that the summation notation represents:

\[
f(x) = \answer{0}+\answer{1} \cdot (x-1)+\answer{4}  \cdot (x-1)^2+\answer{8}  \cdot (x-1)^3+ \ldots
\]

This can be used to find that $f(1) = \answer{0}$ and can also be used to write down the first few derivatives.

\begin{question}
By differentiating $f(x)=(x-1)+4(x-1)^2+8(x-1)^3+\ldots$ , we find:

\[
f'(x) = \answer{1}+\answer{8} \cdot (x-1)+\answer{24}  \cdot (x-1)^2+ \ldots
\]
Thus, $f'(1) = \answer{1}$.

\begin{question}
By differentiating $f'(x)=1+8(x-1)+24(x-1)^2+\ldots$ , we find:

\[
f''(x) = \answer{8}+\answer{48} \cdot (x-1)+ \ldots
\]
Thus, $f''(1) = \answer{8}$.

\end{question}
\end{question}



\end{hint}
\end{exercise}
\end{document}
