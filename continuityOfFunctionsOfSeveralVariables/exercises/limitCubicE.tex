\documentclass{ximera}

\author{Jim Talamo}

\newcommand{\RR}{\mathbb R}
\renewcommand{\d}{\,d}
\newcommand{\dd}[2][]{\frac{d #1}{d #2}}
\renewcommand{\l}{\ell}
\newcommand{\ddx}{\frac{d}{dx}}
\newcommand{\dfn}{\textbf}
\newcommand{\eval}[1]{\bigg[ #1 \bigg]}


\outcome{Evaluate limits of functions of several variables.}
\newcommand{\point}[1]{\left(#1\right)} %this allows \point{ to be changed to \vector{ with a quick find and replace
\newcommand{\pt}[1]{\mathbf{#1}} %this allows \pt{ to be changed to \vec{ with a quick find and replace
\newcommand{\Lim}[2]{\lim_{\point{#1} \to \point{#2}}}

\begin{document}
\begin{exercise}
  Consider the function $F(x,y)= \frac{2x^2+4xy^3+6y^6}{x^2+3y^6}$.
  
  \begin{itemize}
  \item As $\point{x,y} \to \point{0,0}$ along $x=0$, $F(x,y)$ approaches $\answer{2}$.
  \item As $\point{x,y} \to \point{0,0}$ along $y=0$, $F(x,y)$ approaches $\answer{2}$.
  \end{itemize}
  
Is this enough to determine that $\Lim{x,y}{0,0} = 0$? \wordChoice{\choice{Yes}\choice[correct]{No}}
   
   \begin{exercise}
As $\point{x,y} \to \point{0,0}$ along $y=x$, $F(x,y)$ approaches $\answer{2}$.
   
  Is this enough to determine that $\Lim{x,y}{0,0} = 0$? \wordChoice{\choice{Yes}\choice[correct]{No}}
      
\begin{exercise}
Maybe we're a bit unlucky; let's analyze along $y=mx$.  

As $\point{x,y} \to \point{0,0}$ along $y=x$, $F(x,y)$ approaches $\answer{2}$.

\begin{hint}
Along $y=mx$, we have 

\[
F(x,y)= F\left(x,\answer{mx}\right) = \frac{2x^2+4x\left(mx\right)^3+6\left(mx\right)^6}{x^2+3\left(mx\right)^6} = \frac{x^2 \cdot \left(\answer{2+4m^3x^2+6m^2x^4} \right) }{x^2 \cdot \left(1 +3m^6x^4 \right) }
\]
\end{hint}
\begin{exercise}

What's happening here?  By choosing a straight line path, the only coefficient that actually matters are those in front of the lowest power of $x$.  One thing we can notice is that by looking at a different type of path, we can try to balance the powers of $x$ and $y$ so each term has the same exponent.  Let's try $x=y^3$.

As $\point{x,y} \to \point{0,0}$ along $x=y^3$, $F(x,y)$ approaches $\answer{4}$.

From this, we can conclude
\begin{multipleChoice}
\choice{$\Lim{x,y}{0,0} F(x,y)$ exists.}
\choice[correct]{$\Lim{x,y}{0,0} F(x,y)$ does not exist.}
\end{multipleChoice}

\begin{feedback}[correct]
Take a minute to reflect on this; in order to determine that a limit exists, we have to show that the function tends to the same value along \emph{any} path, not just check a few! Even if the function tends to the same value along every path of a certain type, you can \emph{never} use the result to determine that a limit exists.  Try to find part of a \emph{level curve} as we did here; that is find a type of path that makes each term have the same power so cancellation occurs. 
\end{feedback}
   \end{exercise}
   \end{exercise}
   \end{exercise}
   \end{exercise}
\end{document}
