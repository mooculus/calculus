\documentclass{ximera}

\newcommand{\RR}{\mathbb R}
\renewcommand{\d}{\,d}
\newcommand{\dd}[2][]{\frac{d #1}{d #2}}
\renewcommand{\l}{\ell}
\newcommand{\ddx}{\frac{d}{dx}}
\newcommand{\dfn}{\textbf}
\newcommand{\eval}[1]{\bigg[ #1 \bigg]}


\title{Inverses of functions}


\outcome{Find the domain and range of a function.}
\outcome{Determine if a function is one-to-one.}
\outcome{Perform basic operations and compositions on functions.}



\begin{document}
\begin{abstract}
  Here we ``undo'' functions.
\end{abstract}
\maketitle


If a function maps every ``input'' to exactly one ``output,'' an
inverse of that function maps every ``output'' to exactly one
``input.''  We need a more formal definition to actually say anything
with rigor.

\begin{definition}
  Let $f$ be a function with domain $A$ and range $B$:
  \begin{image}
    \begin{tikzpicture}
      \node[star,star points=7,star point ratio=2.5,draw] at (0,0) {$A$};
      \node[cloud, draw,cloud puffs=10,cloud puff arc=120, aspect=2, inner ysep=1em] at (5,0) {$B$};
      \node at (2.25,.3) {$f$};
      \draw[->] (1.5,0) to (3,0);
    \end{tikzpicture}
  \end{image}
  Let $g$ be a function with domain $B$ and range $A$:
  \begin{image}
    \begin{tikzpicture}
      \node[cloud, draw,cloud puffs=10,cloud puff arc=120, aspect=2, inner ysep=1em] at (-.5,0) {$B$};
      \node[star,star points=7,star point ratio=2.5,draw] at (4.5,0) {$A$};
      \node at (2.25,.3) {$g$};
      \draw[->] (1.5,0) to (3,0);
    \end{tikzpicture}
  \end{image}
  We say that $f$ and $g$ are \dfn{inverses} of each other if $f(g(b))
  = b$ for all $b$ in $B$, and also $g(f(a)) = a$ for all $a$ in $A$.
  Sometimes we write $g = f^{-1}$ in this case.
  \begin{image}
    \begin{tikzpicture}
      \node[cloud, draw,cloud puffs=10,cloud puff arc=120, aspect=2, inner ysep=1em] at (-.5,0) {$B$};
      \node[cloud, draw,cloud puffs=10,cloud puff arc=120, aspect=2, inner ysep=1em] at (5,0) {$B$};
      \node at (2.25,.3) {$f\circ f^{-1}$};
      \draw[->] (1.5,0) to (3,0);
    \end{tikzpicture}
  \end{image}
  and
  \begin{image}
    \begin{tikzpicture}
      \node[star,star points=7,star point ratio=2.5,draw] at (0,0) {$A$};
      \node[star,star points=7,star point ratio=2.5,draw] at (4.5,0) {$A$};
      \node at (2.25,.3) {$f^{-1}\circ f$};
      \draw[->] (1.5,0) to (3,0);
    \end{tikzpicture}
  \end{image}
  So, we could rephrase these conditions as
  \[
  f(f^{-1}(x)) = x\qquad\text{and}\qquad f^{-1}(f(x)) = x.
  \]
  \end{definition}
These two simple equations are somewhat more subtle than they
initially appear.

\begin{question}
  Let $f$ be a function.  If the point $(1,9)$ is on the graph of $f$,
  what point must be the the graph of $f^{-1}$?	
  \[
  \left( \answer[given]{9}, \answer[given]{1} \right)
  \]
  \begin{feedback}
    Since $f(1) = 9$, we must have $f^{-1}(f(1)) = 1$, so $f^{-1}(9) =
    1$.  Thus $(9,1)$ is on the graph of $f^{-1}$.  This is a general
    rule.  If $(a,b)$ is on the graph of $f$, then $(b,a)$ will be on the
    graph of $f^{-1}$.
  \end{feedback}
\end{question}

\begin{warning}
  This notation can be very confusing.  Keep a watchful eye:
  \begin{align*}
    f^{-1}(x) &= \text{the inverse function of $f(x)$.}\\
    f(x)^{-1} &= \text{$\frac{1}{f(x)}$.}
  \end{align*}
\end{warning}
\begin{question}
  Which of the following is notation for the inverse of the function
  $\sin(\theta)$ on the interval $[-\pi/2,\pi/2]$?
  \begin{multipleChoice}
    \choice[correct]{$\sin^{-1}(\theta)$}
    \choice{$\sin(\theta)^{-1}$}
  \end{multipleChoice}
\end{question}


So far, we've only dealt with abstract examples.  Let's see
if we can ground this in a real-life context.

\begin{example}
  The function
  \[
  f(t) = \left(\frac{9}{5}\right) t + 32
  \]	
  takes a temperature $t$ in degrees Celsius, and converts it into Fahrenheit.  
  The domain of this function is $-\infty < t < \infty$.  What does the inverse 
  of this function tell you? What is the inverse of this function?

  \begin{explanation}
    If $f$ converts Celsius measurements to Fahrenheit measurements of
    temperature, then $f^{-1}$ converts Fahrenheit measurements to
    Celsius measurements of temperature.
    
    To find the inverse function, first note that 
    \[
    f(f^{-1}(t)) = t \qquad \text{by the definition of inverse
      functions.}
    \]
    Now write out the left-hand side of the equation
    \[
    f(f^{-1}(t)) = \left(\frac{9}{5}\right) f^{-1}(t)+32\qquad\text{by the rule for $f$}
    \]
    and solve for $f^{-1}(t)$.
    \begin{align*}
      \left(\frac{9}{5}\right) f^{-1}(t)+32 &= t &&\text{by the rule for $f$}\\
      \left(\frac{9}{5}\right) f^{-1}(t)&= t -32\\
      f^{-1}(t) &= \left(\frac{5}{9}\right)(t - 32).
    \end{align*}
    So $f^{-1}(t) = \left(\frac{5}{9}\right)(t - 32)$ is the inverse
    function of $f$, which converts a Fahrenheit measurement back into
    a Celsius measurement.  The domain of this inverse function is $-\infty < t <\infty$.
    
    Finally, we could check our work again using the definition of inverse functions.
     We have already guaranteed that
    \[
    f(f^{-1}(t)) = t,
    \]
    since we solved for $f^{-1}$ in our calculation.  On the other hand, 
    \begin{align*}
    f^{-1}(f(t)) &=\left(\frac{5}{9}\right)(f(t) - 32)\\
    &= \left(\frac{5}{9}\right)(f(t) - 32)\\
    \end{align*}
    which you should simplify to check that $f^{-1}(f(t)) = t$.
  \end{explanation}
\end{example}



We have examined several functions in order to determine their inverse 
functions, but there is still more to this story.  Not every function has an inverse function, 
so we must learn how to check for this situation.


\begin{question}
  Let $f$ be a function, and imagine that the points $(2,3)$ and
  $(7,3)$ are both on its graph.  Could $f$ have an inverse function?
  \begin{multipleChoice}
    \choice{Yes}
    \choice[correct]{No}
  \end{multipleChoice}
  \begin{feedback}
    The function $f$ could \textbf{not} have an inverse function.
    Imagine that it did.  Then $f^{-1}(f(2)) = 2$ and $f^{-1}(f(7)) =
    7$.  Then we have both $f^{-1}(3) = 2$ and $f^{-1}(3) = 7$.  Since
    a \textbf{function} cannot send the same input to two different
    outputs, $f$ must not have an inverse function.
  \end{feedback}
\end{question}


Look again at the last question.  If two different inputs for a
function have the same output, there is no hope of that function
having an inverse function.  Why?  The inverse function must also be a
function, and a function can only have one output for each input.
More specifically, we have the next definition.

\begin{definition}
A function is called \dfn{one-to-one} if for every value in the range,
there is exactly one value in the domain.
\end{definition}

\begin{question}
  Which of the following are functions that are also one-to-one?
  \begin{selectAll}
    \choice{Relating words to their meaning in a dictionary.}
    \choice[correct]{Relating social security numbers of living
      people to actual living people.}
    \choice{Relating people to their birthday.}
    \choice{Relating mothers to their children.}
  \end{selectAll}
\end{question}

\begin{question}
Which of the following functions are one to one?  Select all that
apply.
\begin{selectAll}
\choice[correct]{$f(x) = x$}
\choice{$f(x) = x^2$}
\choice{$f(x) = x^3 - 4x$}
\choice[correct]{$f(x) = x^3+4$}
\end{selectAll}
\end{question}




You may recall that a plot gives $y$ as a function of $x$ if every
vertical line crosses the plot at most once, and we called this
the \dfn{vertical line test}. Similarly, a function is one-to-one
if every horizontal line crosses the plot at most once, and we call this 
 the \dfn{horizontal line test}. Below, we give a graph
of $f(x)=-5x^2+30x+60$. While this graph passes the vertical line test,
and hence represents $y$ as a function of $x$, it does not pass the
horizontal line test, so the function is not one-to-one.
\begin{image}
\begin{tikzpicture}
	\begin{axis}[
            clip=false, domain=0:7.58, axis lines =middle, xlabel=$t$,
            ylabel=$h$, every axis y label/.style={at=(current
              axis.above origin),anchor=south}, every axis x
            label/.style={at=(current axis.right of
              origin),anchor=west}, ] \addplot [very thick, penColor,
            smooth] {-5*x^2 +30*x+60}; \addplot [very thick,
            penColor2] {80}; \addplot [very thick, penColor4] plot
          coordinates {(5,0) (5,110)};
        \end{axis}
\end{tikzpicture}
\end{image}

The green line in the picture is an example of a vertical line.  You can see that any 
such vertical line will intersect the graph at most once.  The red line  
is an example of a horizontal line, and you can see that this horizontal 
line intersects the graph twice.

As we have discussed, we can only find an inverse of a function when 
it is one-to-one.  If a function is not one-to-one, but we still want an inverse, 
we must restrict the domain. Let's see what this means in
our next examples.

\begin{question}
	Consider the graph of the function $f$ below:

	\begin{image}
\begin{tikzpicture}
	\begin{axis}[
	  ticks=none,
            domain=-2.5:2.5,
            xmin=-1.5, xmax=1.5,
            ymin=-1, ymax=1,
            axis lines =middle,
          ]
          \addplot [very thick, penColor2, smooth, samples=100,domain=-2:2] {x^3-x-1/6};
          \node at (axis cs:1.3,1.5) [penColor2, anchor=west] {$f$};
	\node at (axis cs:-0.99,-0.01) [penColor, anchor=west] {\textbullet};
	 \node at (axis cs:-1,-0.1) [penColor, anchor=west] {$A$};
\node at (axis cs:-0.67,-0.01) [penColor, anchor=west] {\textbullet};
	 \node at (axis cs:-0.67,-0.1) [penColor, anchor=west] {$B$};
\node at (axis cs:-0.265,-0.01) [penColor, anchor=west] {\textbullet};
	 \node at (axis cs:-0.33,-0.1) [penColor, anchor=west] {$C$};
\node at (axis cs:0.47,-0.01) [penColor, anchor=west] {\textbullet};
\node at (axis cs:0.47,-0.1) [penColor, anchor=west] {$D$};
\node at (axis cs:0.98,-0.01) [penColor, anchor=west] {\textbullet};
\node at (axis cs:1.03,-0.1) [penColor, anchor=west] {$E$};
\node at (axis cs:1,0.6) [penColor2, anchor=west] {$f$};

        \end{axis}
\end{tikzpicture}
\end{image}

On which of the following intervals is $f$ one-to-one?
\begin{selectAll}
\choice[correct]{$[A,B]$}
\choice{$[A,C]$}
\choice[correct]{$[B,D]$}
\choice{$[C,E]$}
\choice[correct]{$[C,D]$}
\end{selectAll}

\end{question}

\begin{example}
Consider the function
\[
f(x) = x^2.
\]
Does $f$ have an inverse? If so, what is it? If not, attempt to
restrict the domain of $f$ and find an inverse on the restricted
domain.
\end{example}


\begin{explanation}
In this case $f$ is not one-to-one. However, it is one-to-one on
the interval $[0,\infty)$. Hence we can find an inverse of $f(x)=x^2$
  on this interval, and it is our familiar function $\sqrt{x}$.  
\begin{image}
\begin{tikzpicture}
	\begin{axis}[
            domain=-2:2,
            xmin=-2, xmax=2,
            ymin=-2, ymax=2,
            axis lines =middle, xlabel=$x$, ylabel=$y$,
            every axis y label/.style={at=(current axis.above origin),anchor=south},
            every axis x label/.style={at=(current axis.right of origin),anchor=west},
          ]
	  \addplot [very thick, penColor, smooth] {x^2};
          \addplot [very thick, penColor2, smooth, samples=100,domain=0:2] {sqrt(x)};
          \addplot [dashed, textColor] {x};
          \node at (axis cs:-1.2,.55) [penColor,anchor=west] {$f(x)$};
          \node at (axis cs:1.4,1) [penColor2, anchor=west] {$f^{-1}(x)$};
        \end{axis}
\end{tikzpicture}
%% \caption{A plot of $f(x)=x^2$ and $f^{-1}(x) = \sqrt{x}$. While
%%   $f(x)=x^2$ is not one-to-one on $\RR$, it is one-to-one on
%%   $[0,\infty)$.}
%% \label{plot:fxn and inverse x^2}
\end{image}
\end{explanation}




\begin{example}
Consider the function
\[
f(x) = x^3.
\]
Does $f(x)$ have an inverse? If so, what is it? If not, attempt to
restrict the domain of $f(x)$ and find an inverse on the restricted
domain.
\begin{explanation}
In this case $f(x)$ is one-to-one and $f^{-1}(x) = \sqrt[3]{x}$. For
your viewing pleasure we give a graph of $y=f(x)=x^3$ and
$y=f^{-1}(x)= \sqrt[3]{x}$. Note, the graph of $f^{-1}$ is the image
of $f$ after being flipped over the line $y=x$.
\begin{image}
\begin{tikzpicture}
	\begin{axis}[
            domain=-2:2,
            xmin=-2, xmax=2,
            ymin=-2, ymax=2,
            axis lines =middle, xlabel=$x$, ylabel=$y$,
            every axis y label/.style={at=(current axis.above origin),anchor=south},
            every axis x label/.style={at=(current axis.right of origin),anchor=west},
          ]
	  \addplot [very thick, penColor, smooth] {x^3};
          \addplot [very thick, penColor2, smooth, samples=100,domain=.01:2] {x^(1/3)};
          \addplot [very thick, penColor2, smooth, samples=100,domain=-2:-.01] {-abs(x)^(1/3)};
          \addplot [very thick, penColor2] plot coordinates {(.01,.215) (-.01,-.215)};
          \addplot [dashed, textColor] {x};
          \node at (axis cs:-1.2,-.42) [penColor,anchor=west] {$f(x)$};
          \node at (axis cs:1.2,.9) [penColor2, anchor=west] {$f^{-1}(x)$};
        \end{axis}
\end{tikzpicture}
%% \caption{A plot of $f(x)=x^3$ and $f^{-1}(x) = \sqrt[3]{x}$. Note
%%   $f^{-1}(x)$ is the image of $f(x)$ after being flipped over the line
%%   $y=x$.}
%% \label{plot:fxn and inverse x^3}
\end{image}
\end{explanation}
\end{example}




\end{document}

