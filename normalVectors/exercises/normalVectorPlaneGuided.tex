\documentclass{ximera}

\newcommand{\RR}{\mathbb R}
\renewcommand{\d}{\,d}
\newcommand{\dd}[2][]{\frac{d #1}{d #2}}
\renewcommand{\l}{\ell}
\newcommand{\ddx}{\frac{d}{dx}}
\newcommand{\dfn}{\textbf}
\newcommand{\eval}[1]{\bigg[ #1 \bigg]}


\author{Jim Talamo}

\outcome{Define unit tangent and unit normal vectors.}

\begin{document}
\begin{exercise}

Given a normal vector $\vec{n} = \vector{a,b,c}$ and a point $(x_0,y_0,z_0)$, the equation of the plane with normal vector $\vec{n}$ that passes through $(x_0,y_0,z_0)$ is given by 

\[
a(x-x_0)+b(y-y_0)+c(z-z_0) = 0.
\]

Find the equation of a plane with normal vector $\vector{2,-1,9}$ that passes through $(2,-3,4)$.  Express your final answer in the form $ax+by+cz=d$.  

\[
2x+\left(\answer{-1}\right)y+\left(\answer{9}\right)z = \answer{43}.
\]

\begin{exercise}
One important observation is that it's not too difficult to find a normal vector to a plane once we have an equation that describes it.

Which of the following vectors is a normal vector for the plane $2x-y+4z=7$?

\begin{multipleChoice}
\choice{$\vector{0,1,2}$}
\choice{$\vector{4,1,0}$}
\choice[correct]{$\vector{2,-1,4}$}
\choice{more than one of these}
\choice{none of these}
\end{multipleChoice}

\begin{exercise}
We can easily extract a normal vector from the coefficients of $x$, $y$, and $z$, but how do we verify that this vector is indeed normal to the plane?

\begin{multipleChoice}
\choice{We take the dot product of the $\vector{2,-1,4}$ and the plane.}
\choice{We take the dot product of the $\vector{2,-1,4}$ and a single vector parallel to the plane.}
\choice[correct]{We take the dot product of the $\vector{2,-1,4}$ and an arbitrary vector parallel to the plane.}
\end{multipleChoice}

To get started, let's find a vector parallel to the plane.  We can do this by finding two points on the plane.  Since the plane gives a single constraint between $x$, $y$, and $z$, we can freely choose any two of them, then use the equation of the plane to find the third.

For instance, if $x=2$, $y=1$, then $z= \answer{1}$, and if $y=1$ and $z=2$, then $x=\answer{0}$, so the points $\left(2,1,\answer{1}\right)$ and $\left(\answer{0}, 1, 2\right)$ lie on the plane.

Thus, a vector parallel to the plane is $\vec{v} = \vector{2,\answer{0},\answer{-1}}$.

Now, we check $\vec{n} \dotp \vec{v} = \answer{0}$.

\begin{exercise}
To establish that $\vec{n}$ is normal to the plane, we must establish that $\vec{n}$ is orthogonal to \emph{any} vector that is parallel to the plane. To do this, pick two points $(x_1,y_1,z_1)$ and $(x_2,y_2,z_2)$ that lie on the plane.

A vector that is parallel to the plane that starts at $(x_1,y_1,z_1)$ and ends at $(x_2,y_2,z_2)$ is $\vec{v} = \vector{x_2-x_1,y_2-y_1,z_2-z_1}$.

Now, $\vec{n} \dotp \vec{v} = \vector{x_2-x_1,y_2-y_1,z_2-z_1} \dotp \vector{2,-1,4} = 2(x_2-x_1) -(y_2-y_1) +4(z_2-z_1)$, and we must establish that this is $0$.

We can rearrange the equation.

\begin{align*}
2(x_2-x_1) -(y_2-y_1) +4(z_2-z_1) &= 2x_2-2x_1 -y_2+y_1 +4z_2-4z_1 \\
&= \left[2x_2 -y_2+4z_2\right]-\left[2x_1 -y_1 +4z_1\right]
\end{align*}

Now, since $(x_1,y_1,z_1)$ and ends at $(x_2,y_2,z_2)$  lie on the plane, $2x_2 -y_2+4z_2 = \answer{7}$ and $2x_1 -y_1+4z_1 = \answer{7}$, so 

\[
\vec{n} \dotp \vec{v} = \left[2x_2 -y_2+4z_2\right]-\left[2x_1 -y_1 +4z_1\right] = \answer{7} -\answer{7} = 0.
\]

Thus, $\vec{n}$ is orthogonal to \emph{any} arbitrary vector that is parallel to the plane, and thus $\vec{n}$ is a normal vector to the plane.
\end{exercise}
\end{exercise}
\end{exercise}
\end{exercise}
\end{document}
