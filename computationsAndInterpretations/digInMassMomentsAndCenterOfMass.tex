\documentclass{ximera}

\newcommand{\RR}{\mathbb R}
\renewcommand{\d}{\,d}
\newcommand{\dd}[2][]{\frac{d #1}{d #2}}
\renewcommand{\l}{\ell}
\newcommand{\ddx}{\frac{d}{dx}}
\newcommand{\dfn}{\textbf}
\newcommand{\eval}[1]{\bigg[ #1 \bigg]}


\outcome{Find the mass, moments, and the center of mass of an object.}

\title[Dig-In:]{Mass, moments, and center of mass}

\begin{document}
\begin{abstract}
  We use integrals to model mass.
\end{abstract}
\maketitle

\section{Mass}

We learned some time ago that if the density of an object is uniform,
\[
\text{mass} = \text{density}\times\text{volume}.
\]
When the density of an object is not uniform, we define a density
function $\rho(x,y,z)$ and mass $m$, to write:
\[
\d m = \rho(x,y,z) \d V
\]
Summing these together with an integral, we find the mass
is equal to
\[
\iiint_R \rho(x,y,z) \d V.
\]

\begin{example}
  Find the mass of the solid defined by the region
  \[
  R = \{(x,y,z):0\le x\le 3, 0\le y\le 6-2x, 0\le z<2-y/3-2x/3\}
  \]
  \begin{image}
    \begin{tikzpicture}
      \begin{axis}%
        [width=175pt,height=200pt,
          tick label style={font=\scriptsize},axis on top,
	  axis lines=center,
	  view={145}{35},
	  name=myplot,
	  %xtick={1,2,3,4},
	  %ytick={1,2,3,4,5,6},
	  %ztick=\empty,
	  %extra x ticks={1},
	  minor x tick num=1,
	  minor y tick num=4,
	  minor z tick num=1,
	  %extra x tick labels={$a$},
	  %extra y ticks={1},
	  %extra y tick labels={$a$},
	  %extra z ticks={1},
	  %extra z tick labels={$h$},
	  ymin=-.5,ymax=6.9,
	  xmin=-.5,xmax=4.9,
	  zmin=-.5, zmax=2.9,
	  every axis x label/.style={at={(axis cs:\pgfkeysvalueof{/pgfplots/xmax},0,0)},xshift=-1pt,yshift=-4pt},
	  xlabel={\scriptsize $x$},
	  every axis y label/.style={at={(axis cs:0,\pgfkeysvalueof{/pgfplots/ymax},0)},xshift=5pt,yshift=-3pt},
	  ylabel={\scriptsize $y$},
	  every axis z label/.style={at={(axis cs:0,0,\pgfkeysvalueof{/pgfplots/zmax})},xshift=0pt,yshift=4pt},
	  zlabel={\scriptsize $z$}
	]
        \draw [penColor, very thick] (axis cs:0,0,0) -- (axis cs:0,0,2)
	(axis cs:0,0,0) -- (axis cs:3,0,0)
	(axis cs:0,0,0) -- (axis cs:0,6,0);
	\draw [fill1,thin,fill=fill1,opacity=.6] (axis cs: 3,0,0) -- (axis cs: 0,6,0) -- (axis cs: 0,0,2)--cycle;
        \draw [penColor,thick] (axis cs: 3,0,0) -- (axis cs: 0,6,0) -- (axis cs: 0,0,2)--cycle;
      \end{axis}
    \end{tikzpicture}
  \end{image}
  with density function $\rho(x,y,z)=3x$.
  \begin{explanation}
    Write with me
    \begin{align*}
      M &= \iiint_R \rho(x,y,z)\d V \\
      &= \int_0^3\int_0^{\answer[given]{6-2x}}\int_0^{\answer[given]{2-y/3-2x/3}} \big(\answer[given]{3x}\big)\d z\d y\d x\\
      &= \int_0^3\int_0^{\answer[given]{6-2x}} (\answer[given]{6x-xy-2x^2})\d y\d x\\
      &= \int_0^3\eval{\answer[given]{6xy-xy^2/2-2x^2y}}_0^{\answer[given]{6-2x}}\d x\\
      &= \int_0^3 (\answer[given]{18x-12x^2+2x^3}) \d x\\
      &= \answer[given]{27/2}
    \end{align*}
  \end{explanation}
\end{example}


\section{Moments and center of mass}

A \textit{moment} is a scalar quantity describing how mass is distributed
in relation to a point, line, or plane.

\begin{definition}
  Let a region $R\subset\R^3$ define a solid with density function $\rho(x,y,z)$.
  \begin{itemize}
  \item The \dfn{moment} about the $(x,y)$-plane is given by
    \[
    M_{xy} = \iiint_R z \rho(x,y,z) \d V
    \]
  \item The \dfn{moment} about the $(x,z)$-plane is given by
    \[
    M_{xz} = \iiint_R y \rho(x,y,z) \d V
    \]
  \item The \dfn{moment} about the $(y,z)$-plane is given by
    \[
    M_{yz} = \iiint_R x \rho(x,y,z) \d V
    \]
  \end{itemize}
\end{definition}

The moments are directly related to the center of mass of an object.

\begin{definition}
  Let a region $R\subset\R^3$ define a solid with density function $\rho(x,y,z)$.
  The \dfn{center of mass} is defined to be the point
  \[
  (\bar{x},\bar{y},\bar{z}) = \left(\frac{M_{yz}}{M},\frac{M_{xz}}{M},\frac{M_{xy}}{M}\right).
  \]
\end{definition}



\begin{example}
  Consider the solid defined by the region
  \[
  R= \{ (x,y,z) : -1\le x \le 1, -\sqrt{1-x^2}<y<0, 0\le z\le -y\}
  \]
  \begin{image}
    \begin{tikzpicture}[>=stealth]
      \begin{axis}%
        [width=175pt,height=200pt,
          tick label style={font=\scriptsize},axis on top,
          axis lines=center,
	  view={115}{35},
	  name=myplot,
	  %xtick={1,2,3,4},
	  %ytick={1,2,3,4,5,6},
	  %ztick=\empty,
	  %extra x ticks={1},
	  %minor x tick num=1,
	  %minor y tick num=4,
	  %minor z tick num=1,
	  %extra x tick labels={$a$},
	  %extra y ticks={1},
	  %extra y tick labels={$a$},
	  %extra z ticks={1},
	  %extra z tick labels={$h$},
	  ymin=-1.1,ymax=.2,
	  xmin=-1.1,xmax=1.3,
	  zmin=-.1, zmax=1.5,
	  every axis x label/.style={at={(axis cs:\pgfkeysvalueof{/pgfplots/xmax},0,0)},xshift=-1pt,yshift=-4pt},
	  xlabel={\scriptsize $x$},
	  every axis y label/.style={at={(axis cs:0,\pgfkeysvalueof{/pgfplots/ymax},0)},xshift=5pt,yshift=-3pt},
	  ylabel={\scriptsize $y$},
	  every axis z label/.style={at={(axis cs:0,0,\pgfkeysvalueof{/pgfplots/zmax})},xshift=0pt,yshift=4pt},
	  zlabel={\scriptsize $z$},colormap/cool
	]
        \addplot3[domain=180:360,,y domain=0:1,mesh,samples=16,samples y=3,very thin,z buffer=sort] ({cos(x)},{sin(x)},{-y*sin(x)});
        
        \addplot3[domain=180:360,,smooth,%fill=white,
          penColor,samples=16,samples y=0,very thick] ({cos(x)},{sin(x)},{0});
        
        \addplot3[domain=180:360,,y domain=0:1,mesh,samples=16,samples y=5,very thin,z buffer=sort] ({y*cos(x)},{y*sin(x)},{-y*sin(x)});
        
        \addplot3[domain=180:360,,smooth,%fill=white,
          penColor,samples=16,samples y=0,very thick] ({cos(x)},{sin(x)},{-sin(x)});
        
        \draw [->] (axis cs:1,-.6,0) node[below,rotate=-15] {\scriptsize $x^2+y^2=1$} -- (axis cs: .7,-.6,.2);
        
        \draw [->] (axis cs:0,-.4,1.2) node[above,] {\scriptsize $z=-y$} -- (axis cs: 0,-.4,.6);
      \end{axis}
    \end{tikzpicture}
  \end{image}
  with density function
  \[
  \rho(x,y,z) = 10 + x^2 + 5y -5z.
  \]
  Find the center of mass of this solid.
  \begin{explanation}
    At this point we need to compute \textit{four} triple
    integrals. Each computation will require a number of careful
    steps. Get out several sheets of paper and take a deep breath.
    First we'll compute the mass. Write with me:
    \begin{align*}
      M &= \iiint_R \big(10+x^2+5y-5z\big)\d V \\
      &= \int_{-1}^1\int_{-\sqrt{1-x^2}}^0\int_0^{-y} \big(10+x^2+5y-5z\big)\d z\d y \d x\\
      &= \answer[given]{\frac{34}{5}-\frac{15\pi}{16}}
    \end{align*}
    Now we'll compute $M_{yz}$. Write with me:
    \begin{align*}
      M_{yz}	&= \iiint_R \answer[given]{x(10+x^2+5y-5z)}\d V \\%& M_{xy}	&= \iiint_R z\big(10+x^2+5y-5z\big)\ dV & M_{xz}\\
      &=\answer[given]{0}
    \end{align*}
    Now we'll compute $M_{xz}$. Write with me:
    \begin{align*}
      M_{xz} &= \iiint_R \answer[given]{y(10+x^2+5y-5z)}\d V\\
      &= \answer[given]{2-\frac{61\pi}{48}}
    \end{align*}
    Now we'll compute $M_{xy}$. Write with me:
  \begin{align*}
    M_{xy} &= \iiint_R \answer[given]{z(10+x^2+5y-5z)}\d V \\
    &= \answer[given]{\frac{61\pi}{96}-\frac{10}{9}}
  \end{align*}
  The center of mass is
  \[
  \begin{bmatrix}
    \bar{x}\\
    \bar{y}\\
    \bar{z}
  \end{bmatrix}
  =
  \begin{bmatrix}
    \answer[given]{0}\\
    \answer[given]{\left(2-\frac{61\pi}{48}\right)/\left(\frac{34}{5}-\frac{15\pi}{16}\right)}\\
    \answer[given]{\left(\frac{61\pi}{96}-\frac{10}{9}\right)/\left(\frac{34}{5}-\frac{15\pi}{16}\right)}
  \end{bmatrix}
  \]
  \end{explanation}
\end{example}

As stated before, there are many uses for triple integration beyond
finding volume. When $h(x,y,z)$ describes a rate of change function
over some space region $R$, then
\[
\iiint_R h(x,y,z)\d V
\]
gives the total change over $R$. Our example of this was computing
mass via a density function. Here a density function is simply a
``rate of mass change per volume'' function. Thus, integrating density
gives total mass.

While knowing \textit{how to integrate} is important, it is arguably
much more important to know \textit{how to set up} integrals. It takes
skill to create a formula that describes a desired quantity; modern
technology is very useful in evaluating these formulas quickly and
accurately.



\end{document}
