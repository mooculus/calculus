\documentclass{ximera}

\newcommand{\RR}{\mathbb R}
\renewcommand{\d}{\,d}
\newcommand{\dd}[2][]{\frac{d #1}{d #2}}
\renewcommand{\l}{\ell}
\newcommand{\ddx}{\frac{d}{dx}}
\newcommand{\dfn}{\textbf}
\newcommand{\eval}[1]{\bigg[ #1 \bigg]}


\title[Refresh:]{Factorials}
\author{Jim Talamo}
\begin{document}
\begin{abstract}
  Remember our facts about exponentials.
\end{abstract}
\maketitle


\begin{exercise}
Ratios of factorials simplify nicely as well.  To begin, we review some properties of factorials.

\section{Factorials}

\begin{problem}
  Given an integer $n$, the notation ``$n!$'' is defined as follows:
  \[
  n! = n(n-1)(n-2)\dots (3)(2)(1)
  \]
  For instance, $3!=3\cdot2\cdot 1 = 6$. 
  
Compute the following.
  \begin{align*}
    4! &= \answer{24}\\
    6! &= \answer{720}\\
    \frac{6!}{4!} &= \answer{30}
  \end{align*}
  
  \begin{problem}
    How did you compute $\frac{6!}{4!}$ in the last problem? You could
    certainly find $6!$ and $4!$ separately, then divide, but there is
    a nicer way to do this.
    \begin{align*}
      \frac{6!}{4!} &= \frac{6\cdot5\cdot4\cdot3\cdot2\cdot1}{4\cdot3\cdot2\cdot1}\\
      &=\frac{6\cdot5\cdot\not{4}\cdot\not{3}\cdot\not{2}\cdot\not{1}}{\not{4}\cdot\not{3}\cdot\not{2}\cdot\not{1}}\\
      &= 6\cdot 5.
    \end{align*}
    
    Note that we could make this even more notationally efficient by noting that $6! = 6 \cdot 5 \cdot 4!$ so we can write
    
    \[
    \frac{6!}{4!} = \frac{6 \cdot 5 \cdot \cancel{4!}}{\cancel{4!}} =30
    \]
    Ratios of factorials are always easiest to compute by canceling
    like terms! Compute the following, simplify your final answers.
    \begin{align*}
      \frac{5!}{4!} &= \answer{5}\\
      \frac{6!}{3!\cdot 3!} &= \answer{20}\\
      \frac{k!}{(k+2)!}  &=\answer{\frac{1}{(k+2)(k+1)}}
    \end{align*}
    
    
    
    
  \end{problem}
\end{problem}

\begin{problem}
  Often, factorials arise in sequences, and it is important to
  understand how to write out various terms. Suppose $a_k$ is a
  sequence whose $k$th term is given by:
  \[
  a_k = (2k+1)!
  \]
  \begin{enumerate}
  \item $a_2 = \answer{120}$
  \item Find an expression for $a_{k+1}$. Express your answer in the form $(ck+d)!$.
    \[
    a_{k+1} = \left(\answer{2} \cdot k + \answer{3}\right)!
    \]
  \item Calculate and simplify:
    \[
    \frac{a_{k+1}}{a_k} = \answer{(2k+3)(2k+2)}
    \]
  \end{enumerate}
\end{problem}

\begin{problem}
  Let $a_k$ be a sequence whose $k$-th term is given by
  \[
  a_k = \frac{(2k)!}{(k!)^2}.
  \]
  Calculate and simplify.
  \begin{align*}
    \frac{a_{k+1}}{a_k} &= \answer{\frac{4k+2}{k+1}}\\
    \lim_{k\to\infty} \frac{a_{k+1}}{a_k} &= \answer{4}
  \end{align*}
  
  \begin{hint}
  Note that $a_{k+1} = \frac{(2(k+1))!}{\big[(k+1)!\big]^2} = \frac{(2k+2)!}{(k+1)! \cdot (k+1)!}$.  Thus,
  
  \[
   \frac{a_{k+1}}{a_k} = \frac{(2k+2)!}{(k+1)! \cdot (k+1)!} \cdot \frac{k! \cdot k!}{(2k)!}.
  \]
  
  Rearranging to collect like terms gives
  
  \begin{align*}
   \frac{a_{k+1}}{a_k} &= \frac{(2k+2)!}{(2k)!} \cdot \frac{k!}{(k+1)!} \cdot \frac{k!}{(k+1)!} \\
   &= \frac{(2k+2)(2k+1)(2k)!}{(2k)!} \cdot \frac{k!}{(k+1) \cdot k!} \cdot \frac{k!}{(k+1) \cdot k!}.\\
  \end{align*}
  
  Now simplify.
  \end{hint}
\end{problem}

\end{exercise}

\end{document}
