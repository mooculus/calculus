\documentclass{ximera}

\newcommand{\RR}{\mathbb R}
\renewcommand{\d}{\,d}
\newcommand{\dd}[2][]{\frac{d #1}{d #2}}
\renewcommand{\l}{\ell}
\newcommand{\ddx}{\frac{d}{dx}}
\newcommand{\dfn}{\textbf}
\newcommand{\eval}[1]{\bigg[ #1 \bigg]}


\author{Jim Fowler}
\license{Creative Commons 4.0 By-SA}

\begin{document}

  Let $T$ be a triangle in the $yz$-plane, with vertices at $(0,0,0)$ and $(0,1,1)$ and $(0,-1,1)$.

  Define $C$ to  be the cone resulting from rotating $T$ around the $z$-axis.

  Let $S$ be the unit ball centered at $(0,0,0)$; that is, let
  \[
    S = \{ (x,y,z) \in \R^3 : x^2 + y^2 + z^2 \leq 1 \}.
  \]
  
  Finally define $R := C \cap S$, that is, $R$ is the intersection of $C$ and $S$.
  
  \begin{exercise}
    Using the usual cartesian coordinates, one could compute the volume of $R$ via
    \[
      \int_{\answer{-1/\sqrt{2}}}^{\answer{1/\sqrt{2}}} \int_{\answer{-\sqrt{1/2-x^2}}}^{\answer{\sqrt{1/2-x^2}}} \int_{\answer{\sqrt{x^2+y^2}}}^{\answer{\sqrt{1-x^2-y^2}}} \d z \d y \d x
    \]
  \end{exercise}

  \begin{exercise}
    Using the cylindrical coordinates, one could compute the volume of $R$ via
    \[
      \int_{\answer{0}}^{\answer{1/\sqrt{2}}} \int_{0}^{\answer{2\pi}} \int_{\answer{r}}^{\answer{\sqrt{1-r^2}}} r \d z \d \theta \d r
    \]
  \end{exercise}

  
\begin{exercise}
  Using spherical coordinates, one can compute the volume of $R$ using
  \[
    \int_{0}^{\answer{2\pi}} \int_{0}^{\answer{\pi/4}} \int_{\answer{0}}^{\answer{1}} \rho^2 \sin \varphi \d \rho \d \varphi \d\theta.
  \]

\end{exercise}

\end{document}
