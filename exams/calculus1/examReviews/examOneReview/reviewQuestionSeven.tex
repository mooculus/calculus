\documentclass{ximera}
\newcommand{\RR}{\mathbb R}
\renewcommand{\d}{\,d}
\newcommand{\dd}[2][]{\frac{d #1}{d #2}}
\renewcommand{\l}{\ell}
\newcommand{\ddx}{\frac{d}{dx}}
\newcommand{\dfn}{\textbf}
\newcommand{\eval}[1]{\bigg[ #1 \bigg]}

\outcome{Calculate limits using the limit laws.}
\author{Tom Dinitz and Nela Lakos}
\license{Creative Commons 3.0 By-NC}
\begin{document}

\begin{exercise}
\tag{limit}

The (entire) graph of a function $f$ is given in the figure below.

\begin{image}
           
\begin{tikzpicture}
    \begin{axis}[
            xmin=-10, xmax=10, ymin=-6,ymax=6,   
            unit vector ratio*=1 1 1,
            axis lines =middle, xlabel=$x$, ylabel=$y$,
            every axis y label/.style={at=(current axis.above origin),anchor=south},
            every axis x label/.style={at=(current axis.right of origin),anchor=west},
            xtick={-10,...,10}, ytick={-6,...,6},
            grid=major,width=4in,
            grid style={dashed, gridColor},
          ]
          \addplot[color=penColor,very thick] plot coordinates
                  {(-8,0) (0,-4)};
	  \addplot[color=penColor,very thick] plot coordinates
		  {(3,2) (8,0)};
	  \addplot[very thick, color=penColor, smooth, domain=(0:3)] {6-(2*x/3)^2};
          \addplot[color=penColor,fill=penColor,only marks,mark=*] coordinates{(-8,2)};  %% closed hole
          \addplot[color=penColor,fill=background,only marks,mark=*] coordinates{(-8,0)};  %% open hole
          \addplot[color=penColor,fill=background,only marks,mark=*] coordinates{(0,-4)};  %% open hole
  	  \addplot[color=penColor,fill=background,only marks,mark=*] coordinates{(0,6)};
	  \addplot[color=penColor,fill=background,only marks,mark=*] coordinates{(3,2)};
	  \addplot[color=penColor,fill=penColor,only marks,mark=*] coordinates{(3,0)};
	  \addplot[color=penColor,fill=background,only marks,mark=*] coordinates{(0,8)};
        \end{axis}
\end{tikzpicture}
\end{image}


(i) Find the domain and range of $f$. Write your answer in interval notation.

Domain of $f$: \begin{prompt}$\Big[\answer{-8},\answer{0}\Big)\cup \Big(\answer{0},\answer{8}\Big)$\end{prompt}

Range of $f$: \begin{prompt}$\Big(\answer{-4},\answer{6}\Big)$\end{prompt}

(ii) List the largest intervals of continuity for $f$: \begin{prompt}$\Big(\answer{-8},\answer{0}\Big)$ and $\Big(\answer{0},\answer{3}\Big)$ and $\Big(\answer{3},\answer{8}\Big)$\end{prompt}

(iii) Determine if there are any points $a$ in the interval $[-8,8]$ for which the following statements are true. If there are any such points, find all of them.

(a) $\lim\limits_{x\to a} f(x)$ exists, but the function $f$ is NOT continuous at $a$.
\begin{prompt}
\begin{multipleChoice}
\choice{There are no points}
\choice[correct]{There is at least one point}
\end{multipleChoice}

\begin{exercise}
$a=\answer{3}$
\end{exercise}
\end{prompt}

(b) Both limits $\lim\limits_{x\to a^+} f(x)$ and $\lim\limits_{x\to a^-}$ exist, but the limit $\lim\limits_{x\to a} f(x)$ does not exist.

\begin{prompt}
\begin{multipleChoice}
\choice{There are no points}
\choice[correct]{There is at least one point}
\end{multipleChoice}

\begin{exercise}
$a=\answer{0}$
\end{exercise}
\end{prompt}

(c) $\lim\limits_{x\to a} f(x)=-2$.

\begin{prompt}
\begin{multipleChoice}
\choice{There are no points}
\choice[correct]{There is at least one point}
\end{multipleChoice}

\begin{exercise}
$a=\answer{4}$
\end{exercise}
\end{prompt}

(d) $\lim\limits_{x\to a^+} f(x)=0$.
\begin{prompt}
\begin{multipleChoice}
\choice{There are no points}
\choice[correct]{There is at least one point}
\end{multipleChoice}

\begin{exercise}
$a=\answer{-8}$
\end{exercise}
\end{prompt}

(e) $\lim\limits_{x\to a^-} f(x)=0$.
\begin{prompt}
\begin{multipleChoice}
\choice{There are no points}
\choice[correct]{There is at least one point}
\end{multipleChoice}

\begin{exercise}
$a=\answer{8}$
\end{exercise}
\end{prompt}

(iv) Find the following value or say `DNE'
\begin{enumerate}
\item $f(0)=\begin{prompt}\answer[format=string]{DNE}\end{prompt}$

\item $f(3)=\begin{prompt}\answer{0}\end{prompt}$

\item  $f^{-1}(0)=\begin{prompt}\answer{3}\end{prompt}$

\item $f^{-1}(2)=\begin{prompt}\answer{-8}\end{prompt}$

\item $f^{-1}(-4)=\begin{prompt}\answer[format=string]{DNE}\end{prompt}$

\item $f^{-1}(-2)=\begin{prompt}\answer{-4}\end{prompt}$
\end{enumerate}
\end{exercise}
\end{document}
