\documentclass[12pt,numbers,handout]{xourse}

\usepackage{stix}

\newcommand{\RR}{\mathbb R}
\renewcommand{\d}{\,d}
\newcommand{\dd}[2][]{\frac{d #1}{d #2}}
\renewcommand{\l}{\ell}
\newcommand{\ddx}{\frac{d}{dx}}
\newcommand{\dfn}{\textbf}
\newcommand{\eval}[1]{\bigg[ #1 \bigg]}


\usepackage[framemethod=TikZ]{mdframed}

\usepackage{geometry}
\geometry{landscape, 
  left=1in,
  top=1in,}
%%  %% textheight=6.5in,
%%  %% textwidth=5in,
%%  %% marginparsep=0.5in,
%%   %% marginparwidth=2.5in}


%% %% Attempting to add a ``notes'' in margin
%% %\usepackage[some]{background} %% remove the [some] to have the header
%% \usepackage{background} %% remove the [some] to have the header
%% %% Code for the header
%% \backgroundsetup{
%%   opacity=1,
%%   color=black,
%%   position={19,-8cm},
%%   angle=0,scale=1,
%%   firstpage=true, 
%%   contents={
%%     \begin{tikzpicture}[x=1cm, y=1cm]
%%       \node at (5, 16.5) {\Large Notes:};
%%       \draw[step=5mm, line width=0.1mm, black!40!white] (0,0) grid (10,16);
%%      \end{tikzpicture}
%%      }}
%% %% End code of notes






%% Allows for a wide text/regular text
\renewcommand{\fullwidth}{\newgeometry{textwidth=10cm,textheight=10cm}}
\renewcommand{\normalwidth}{\restoregeometry}
%% End


%% This is the code that will allow us to redefine the title with
%% \let\maketitle\makesectiontitle

%% \makeatletter
%% \newcommand\makesectiontitle{
%%   \addtocounter{titlenumber}{1}\addcontentsline{toc}{subsection}{\thetitlenumber~\@title} %% puts titles in the toc
%%   {\flushleft\large\bfseries \@pretitle\par\vspace{-1em}}%
%%   {\flushleft\LARGE\bfseries {\ifnumbers\thetitlenumber\fi}{\ifnumbers\hspace{1em}\else\hspace{0em}\fi}\@title \par }
%%   \vskip .6em\noindent\textit\theabstract\setcounter{problem}{0}\setcounter{subsection}{0}\par\vspace{2em}
%%   \ifnooutcomes\else\ifhandout\else\let\thefootnote\relax\footnote{Learning outcomes: \theoutcomes}\fi\fi
%%   \aftergroup\@afterindentfalse
%%   \aftergroup\@afterheading}
%% \makeatother


\title{Calculus I}

\begin{document}
\maketitle
%% \setlength{\columnsep}{3cm}
%% \setlength{\columnseprule}{0.4pt}
%% Redefines "explanation"
\renewmdenv[outerlinewidth=2,topline=false, bottomline=false, leftline=true, rightline=false, 
leftmargin=40,innertopmargin=0pt,innerbottommargin=0pt,skipbelow=\baselineskip,
outerlinecolor=textColor,fontcolor=textColor,backgroundcolor=background]{explanation}%

  
%\tableofcontents

\cleardoublepage


\part{Functions, limits, and continuity}


%% Understanding Functions
%%Title Page
\activity{../understandingFunctions/titlePage.tex}
%% BreakGround
\activity{../understandingFunctions/breakGround.tex}
%% DigIns
\activity{../understandingFunctions/digInForEachInputExactlyOneOutput.tex}
\activity{../understandingFunctions/digInCompositionOfFunctions.tex}
\activity{../understandingFunctions/digInInversesOfFunctions.tex}

%%Review of famous functions
%%Title page
\activity{../reviewOfFamousFunctions/titlePage.tex}
%% DigIns
\activity{../reviewOfFamousFunctions/breakGround.tex}
\activity{../reviewOfFamousFunctions/digInPolynomialFunctions.tex}
\activity{../reviewOfFamousFunctions/digInRationalFunctions.tex}
\activity{../reviewOfFamousFunctions/digInTrigonometricFunctions.tex}
\activity{../reviewOfFamousFunctions/digInExponentialAndLogarithmeticFunctions.tex}


%% What is a limit
\activity{../whatIsALimit/titlePage.tex}
\activity{../whatIsALimit/breakGround.tex}
\activity{../whatIsALimit/digInWhatIsALimit.tex}

%% %% Limit Laws
\activity{../limitLaws/titlePage.tex}
\activity{../limitLaws/breakGround.tex}
\activity{../limitLaws/digInContinuity.tex}
\activity{../limitLaws/digInLimitLaws.tex}
\activity{../limitLaws/digInTheSqueezeTheorem.tex}

%% %% Indeterminant forms
\activity{../indeterminateForms/titlePage.tex}
\activity{../indeterminateForms/breakGround.tex}
\activity{../indeterminateForms/digInLimitsOfTheFormZeroOverZero.tex}
\activity{../indeterminateForms/digInLimitsOfTheFormNonZeroOverZero.tex}

%% %% Using limits to detect asymptotes
\activity{../asymptotesAsLimits/titlePage.tex}
\activity{../asymptotesAsLimits/digInVerticalAsymptotes.tex}
\activity{../asymptotesAsLimits/digInHorizontalAsymptotes.tex}
\activity{../asymptotesAsLimits/digInSlantAsymptotes.tex}


%% %% The intermediate value theorem
\activity{../continuity/titlePage.tex}
\activity{../continuity/breakGround.tex}
\activity{../continuity/digInContinuityOfPiecewiseFunctions.tex}
\activity{../continuity/digInTheIntermediateValueTheorem.tex}


%% %% An application of limits
\activity{../anApplicationOfLimits/titlePage.tex} 
\activity{../anApplicationOfLimits/breakGround.tex}
\activity{../anApplicationOfLimits/digInInstantaneousVelocity.tex}

%% %% %% Bonus: The precise definition of a limit
%% %% \activity{../preciseDefinitionOfALimit/titlePage.tex}
%% %% \activity{../preciseDefinitionOfALimit/digInThePreciseDefinitionOfALimit.tex}

\part{Derivatives}

%% %% Definition of the derivative
\activity{../definitionOfTheDerivative/titlePage.tex}
\activity{../definitionOfTheDerivative/breakGround.tex}
\activity{../definitionOfTheDerivative/digInTheDerivativeViaLimits.tex}

%% %% The derivative as a function
\activity{../derivativeAsAFunction/titlePage.tex}
\activity{../derivativeAsAFunction/breakGround.tex}
\activity{../derivativeAsAFunction/digInTheDerivativeAsAFunction.tex}
\activity{../derivativeAsAFunction/digInDifferentiabilityImpliesContinuity.tex}


%% %% Rules of differentiation
\activity{../rulesOfDifferentiation/titlePage.tex}
\activity{../rulesOfDifferentiation/breakGround.tex}
\activity{../rulesOfDifferentiation/digInBasicRulesOfDifferentiation.tex}
\activity{../rulesOfDifferentiation/digInTheDerivativeOfEToTheX.tex}
\activity{../rulesOfDifferentiation/digInTheDerivativeOfSine.tex}


%% %% Higher order derivatives and graphs
\activity{../higherOrderDerivativesAndGraphs/titlePage.tex}
\activity{../higherOrderDerivativesAndGraphs/breakGround.tex}
\activity{../higherOrderDerivativesAndGraphs/digInHigherOrderDerivativesAndGraphs.tex}
\activity{../higherOrderDerivativesAndGraphs/digInConcavity.tex}
\activity{../higherOrderDerivativesAndGraphs/digInPositionVelocityAndAcceleration.tex}


%% %% Product and quotient rules
\activity{../productAndQuotientRules/titlePage.tex}
\activity{../productAndQuotientRules/breakGround.tex}
\activity{../productAndQuotientRules/digInProductRuleAndQuotientRule.tex}

%% %% The chain rule
%% \activity{../chainRule/titlePage.tex}
%% \activity{../chainRule/breakGround.tex}
%% \activity{../chainRule/digInChainRule.tex}


%% %% Mean Value Theorem
%% \activity{../meanValueTheorem/titlePage.tex}

%% %% Linear Approximation
%% \activity{../linearApproximation/titlePage.tex}
%% %\activity{../linearApproximation/breakGround.tex}
%% \activity{../linearApproximation/digInLinearApproximation.tex}

%% %% Maximums and Minimums
%% \activity{../maximumsAndMinimums/titlePage.tex}

%% %%Optimization
%% \activity{../optimization/titlePage.tex}

%% %% Applied optimization
%% \activity{../appliedOptimization/titlePage.tex}
%% \activity{../appliedOptimization/breakGround.tex}

%% %% Implicit Differentiation
%% \activity{../implicitDifferentiation/titlePage.tex}
%% \activity{../implicitDifferentiation/breakGround.tex}
%% \activity{../implicitDifferentiation/digInImplicitDifferentiation.tex}

%% %% Derivatives of trigonometric functions
%% \activity{../trigonometricDerivatives/titlePage.tex}
%% \activity{../trigonometricDerivatives/breakGround.tex}
%% \activity{../trigonometricDerivatives/digInDerivativesOfTrigFunctions.tex}

%% %% Implicit differentiation
%% \activity{../implicitDifferentiation/titlePage.tex}

%% %% Derivatives of inverse functions
%% \activity{../derivativesOfInverseFunctions/titlePage.tex}
%% \activity{../derivativesOfInverseFunctions/breakGround.tex}

%% %% Logarithmic differentiation
%% \activity{../logarithmicDifferentiation/titlePage.tex}

%% %% Advanced graphing of functions
%% \activity{../advancedGraphingOfFunctions/titlePage.tex}
%% \activity{../advancedGraphingOfFunctions/breakGround.tex}
%% \activity{../advancedGraphingOfFunctions/digInSketchingThePlotOfAFunction.tex}

%% %% More than one rate
%% \activity{../moreThanOneRate/titlePage.tex}
%% \activity{../moreThanOneRate/breakGround.tex}
%% \activity{../moreThanOneRate/digInMoreThanOneRate.tex}

%% %% Applied related rates
%% \activity{../appliedRelatedRates/titlePage.tex}
%% \activity{../appliedRelatedRates/breakGround.tex}
%% \activity{../appliedRelatedRates/digInAppliedRelatedRates.tex}


%% \part{Antiderivatives, return of limits, and integrals}


%% %% Differential equations
%% \activity{../differentialEquations/titlePage.tex}
%% \activity{../differentialEquations/breakGround.tex}
%% \activity{../differentialEquations/digInDifferentialEquations.tex}

%% %% Antiderivatives
%% \activity{../antiderivatives/titlePage.tex}

%% %% L'Hosptial's Rule
%% \activity{../lhopitalsRule/titlePage.tex}
%% \activity{../lhopitalsRule/breakGround.tex}

%% %% Approximating the area under a curve
%% \activity{../approximatingTheAreaUnderACurve/titlePage.tex}

%% %% Sigma notation
%% \activity{../sigmaNotation/titlePage.tex}

%% %% Net area
%% \activity{../netArea/titlePage.tex}

%% %% The definite integral
%% \activity{../definiteIntegral/titlePage.tex}

%% %% The First Fundamental Theorem of Calculus
%% \activity{../firstFundamentalTheoremOfCalculus/titlePage.tex}

%% %% The Second Fundamental Theorem of Calculus
%% \activity{../secondFundamentalTheoremOfCalculus/titlePage.tex}

%% %% The idea of substitution
%% \activity{../substitution/titlePage.tex}
%% \activity{../substitution/breakGround.tex}
%% \activity{../substitution/digInIdeaOfSubstitution.tex}
%% %% Working with substitution
%% \activity{../workingWithSubstitution/titlePage.tex}
%% \activity{../workingWithSubstitution/breakGround.tex}
%% \activity{../workingWithSubstitution/digInWorkingWithSubstitution.tex}

%% %% Applications of integrals
%% \activity{../applicationsOfIntegrals/titlePage.tex}

\end{document}
