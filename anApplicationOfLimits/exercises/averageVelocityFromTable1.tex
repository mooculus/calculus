\documentclass{ximera}

\newcommand{\RR}{\mathbb R}
\renewcommand{\d}{\,d}
\newcommand{\dd}[2][]{\frac{d #1}{d #2}}
\renewcommand{\l}{\ell}
\newcommand{\ddx}{\frac{d}{dx}}
\newcommand{\dfn}{\textbf}
\newcommand{\eval}[1]{\bigg[ #1 \bigg]}


\outcome{Compute average velocity.}
\outcome{Approximate instantaneous velocity.}

\author{Nela Lakos \and Kyle Parsons}

\begin{document}
\begin{exercise}

Suppose $s(t)$ is the position (in feet) of an object moving along a line for time $t\geq 0$ (in seconds).  The average velocity of the object on the interval $[a,b]$ is
\[
v_{\text{av}} = \frac{s(b)-\answer{s(a)}}{\answer{b-a}}.
\]

\begin{exercise}

The table below gives the value of $s(t)$ at several times.

\[
\begin{array}{|c|c|c|c|c|c|c|c|}
\hline
t & 0 & 0.5 & 1 & 1.5 & 2 & 2.5 & 3 \\\hline
s(t) & 0 & 22 & 32 & 48 & 54 & 64 & 74 \\\hline
\end{array}
\]

The average velocity of the object on the interval $[1,3]$ is
\[
v_{\text{av}} = \answer{21}\text{ft/s.}
\]

\begin{exercise}

Based on the information in the table above we \wordChoice{\choice{can}\choice[correct]{cannot}} calculate the instantaneous velocity of the object at $t=1$.

\begin{exercise}

The best we can do is calculate the average velocity of the object over the intervals $[0.5,1]$ and $[1,1.5]$.  

The average velocity of the object on $[0.5,1]$ is 
\[
v_{\text{av}} = \answer{20}\text{ft/s.}
\]

The average velocity on the interval $[1,1.5]$ is
\[
v_{\text{av}} = \answer{32}\text{ft/s.}
\]

\end{exercise}
\end{exercise}
\end{exercise}
\end{exercise}
\end{document}