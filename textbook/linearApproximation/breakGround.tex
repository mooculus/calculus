\documentclass{ximera}

\newcommand{\RR}{\mathbb R}
\renewcommand{\d}{\,d}
\newcommand{\dd}[2][]{\frac{d #1}{d #2}}
\renewcommand{\l}{\ell}
\newcommand{\ddx}{\frac{d}{dx}}
\newcommand{\dfn}{\textbf}
\newcommand{\eval}[1]{\bigg[ #1 \bigg]}


\outcome{}

\title[Break-Ground:]{Linear approximation}

\begin{document}
\begin{abstract}
Here we see a dialogue where two young mathematicians discuss linear approximation.
\end{abstract}
\maketitle

Lost hiker?

Procedural questions: how to approximate functions.

What is square root of 2.1?

Differentials?

% Right before the linear approximation section is the mean value
% theorem.  So I think the main story should be the extent to which
% second derivatives (acceleration) move you away from being a
% straight line.

\begin{dialogue}
\item[Devyn] 
What's the point of all these derivatives?
\item[Riley]
What's the smallest positive number?
I'm going to optimize something on an open interval.  Oh so many problems.
\end{dialogue}


In this activity we will study \textit{linear approximations}. 

\begin{problem}
Smart Sally says that the line
\[
\l(x) = \frac{1}{4}(x-4)+2
\]
is a good approximation for $f(x) = \sqrt{x}$ when $x$ is close to
$4$. 
\begin{enumerate}
\item Plot $\l(x)$ and $f(x)$. Explain how this shows that she is
correct.
\item Use concepts of calculus to explain why Sally is correct. 
\end{enumerate}
\end{problem}

\begin{problem}
Consider $f(x) = \sqrt[3]{x}$. Find the equation of the line tangent
to $f(x)$ when $x= 27$. Use this to approximate
$\sqrt[3]{28}$. Explain your reasoning and sketch a plot of this
situation.
\end{problem}

\begin{problem}
Consider $f(x) = \sqrt[5]{x}$. Find the equation of the line tangent
to $f(x)$ when $x= 243$. Use this to approximate
$\sqrt[5]{250}$. Explain your reasoning and sketch a plot of this
situation.
\end{problem}

\begin{problem}
In the problems above, we've been working with \textit{linear
approximations}. Explain what this means and how linear approximations
work.
\end{problem}

\begin{problem}
Suppose you want to know $\sqrt[3]{10}$. Explain how to use a linear
approximation to estimate this value. Sketch a plot to clarify why
your method works.
\end{problem}


\begin{problem}
Calculus Cal attempts to use Smart Sally's method of computing linear
approximations to compute $\sqrt[3]{2}$ with
\[
\l(t) = \frac{1}{27}(x-27)+3.
\] 
Smart Sally laughs at him and says, ``That's never going to work!''
Why is she sure this won't work? Explain what Cal did right and what
Cal did wrong. Sketch a plot of this situation to enhance your
explanation.
\end{problem}

\begin{problem}
Calculus Cal attempts to use Smart Sally's method of computing linear
approximations to compute $11^5$ with
\[
\l(t) = 5\cdot 10^4(x-10)+100000.
\] 
Smart Sally laughs at him and says, ``That's never going to work!''
Why is she sure this won't work? Explain what Cal did right and what
Cal did wrong. Sketch a plot of this situation to enhance your
explanation.
\end{problem}

\begin{problem}
Explain the general procedure for finding linear approximations, and
what the potential pit-falls are for such a procedure.
\end{problem}





%% \begin{xarmaBoost}
%%   Write down at least \textbf{five} questions for this lecture. After
%%   you have your questions, label them as ``Level 1,'' ``Level 2,'' or
%%   ``Level 3'' where:
%% \begin{description}
%% \item[Level 1] Means you know the answer, or know exactly how to do
%%   this problem.
%% \item[Level 2] Means you think you know how to do the problem.
%% \item[Level 3] Means you have no idea how to do the problem.
%% \end{description}
%% \begin{freeResponse}
%% \end{freeResponse}
%% \end{xarmaBoost}


\end{document}
