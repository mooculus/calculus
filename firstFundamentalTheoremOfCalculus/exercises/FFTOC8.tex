\documentclass{ximera}

\newcommand{\RR}{\mathbb R}
\renewcommand{\d}{\,d}
\newcommand{\dd}[2][]{\frac{d #1}{d #2}}
\renewcommand{\l}{\ell}
\newcommand{\ddx}{\frac{d}{dx}}
\newcommand{\dfn}{\textbf}
\newcommand{\eval}[1]{\bigg[ #1 \bigg]}


\outcome{Define accumulation functions.}
\outcome{Calculate and evaluate accumulation functions.}
\outcome{State the First Fundamental Theorem of Calculus.}
\outcome{Take derivatives of accumulation functions using the First Fundamental Theorem of Calculus.}
\outcome{Use accumulation functions to find information about the original function.}
\outcome{Understand the relationship between the function and the derivative of its accumulation function.}

\author{Nela Lakos \and Kyle Parsons}

\begin{document}
\begin{exercise}

The graph of $f$ is given below.

\begin{image}
  \begin{tikzpicture}
    \begin{axis}[
        xmin=-0.3,xmax=6.3,ymin=-2.3,ymax=2.3,
        clip=true,
        unit vector ratio*=1 1 1,
        axis lines=center,
        grid = major,
        ytick={-2,-1,...,36},
        xtick={0,1,...,10},
        xlabel=$t$, ylabel=$y$,
        every axis y label/.style={at=(current axis.above origin),anchor=south},
        every axis x label/.style={at=(current axis.right of origin),anchor=west},
      ]
      \draw[ultra thick,penColor] (axis cs:0,-2) arc[radius=200,start angle=-90,end angle=0] (axis cs:2,0);
      \addplot[ultra thick,penColor,domain=2:4] {x-2};
      \addplot[ultra thick,penColor,domain=4:6] {2};    
        
      \node at (axis cs:1.5,1.5) {$y=f(t)$};
      \end{axis}`
  \end{tikzpicture}
\end{image}

Let $A(x) = \int_0^x f(t) \d t$.

Evaluate the following.
\begin{align*}
A(0) &= \answer{0}\\
A(2) &= \answer{-\pi}\\
A(4) &= \answer{2-\pi}\\
A(6) &= \answer{6-\pi}\\
\end{align*}

The graph of $A(x)$ is given below.

\begin{image}
  \begin{tikzpicture}
    \begin{axis}[
        xmin=-0.3,xmax=6.3,ymin=-4.3,ymax=4.3,
        clip=true,
        unit vector ratio*=1 1 1,
        axis lines=center,
        grid = major,
        ytick={-4,-3,...,36},
        xtick={0,1,...,10},
        xlabel=$x$, ylabel=$y$,
        every axis y label/.style={at=(current axis.above origin),anchor=south},
        every axis x label/.style={at=(current axis.right of origin),anchor=west},
      ]
      \addplot[ultra thick,penColor,domain=0:2] {-x*sqrt(4-x^2)/2 - 2*asin(x/2)*pi/180};
      \addplot[ultra thick,penColor,domain=2:4] {(x-2)^2/2-pi};
      \addplot[ultra thick,penColor,domain=4:6] {2*(x-4)-pi+2};    
        
      \node at (axis cs:1.5,1.5) {$y=A(x)$};
      \end{axis}`
  \end{tikzpicture}
\end{image}

Evaluate the following.
\begin{align*}
A'(2) &= \answer{0}\\
A'(5) &= \answer{2}
\end{align*}

Solve the initial value problem $y'(x) = f(x)$ and $y(0) = 6$ in terms of $A$.
\[
y(x) = \answer{A(x)+6}
\]

\end{exercise}
\end{document}