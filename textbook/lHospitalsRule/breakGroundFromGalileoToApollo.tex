\documentclass{ximera}

\newcommand{\RR}{\mathbb R}
\renewcommand{\d}{\,d}
\newcommand{\dd}[2][]{\frac{d #1}{d #2}}
\renewcommand{\l}{\ell}
\newcommand{\ddx}{\frac{d}{dx}}
\newcommand{\dfn}{\textbf}
\newcommand{\eval}[1]{\bigg[ #1 \bigg]}


\outcome{}


\title[Break-Ground:]{From Galileo to Apollo}

\begin{document}

\begin{abstract}
Here we investigate a limit that arises when considering falling
bodies in an atmosphere.
\end{abstract}
\maketitle


\todo{Outcomes}

In the 1600's, Galileo popularized the idea that \href{http://en.wikipedia.org/wiki/Galileo%27s_Leaning_Tower_of_Pisa_experiment}{falling bodies fell
at the same velocity regardless of their mass}. While this is something
we know on an intellectual level, you also know from experience that
hammers fall quickly and feathers fall slowly. The reason for this is
\textit{air resistance}.  Let's take a minute to investigate the
effects of air resistance. The velocity of a falling object with air
resistance can be modeled by
\[
v = \frac{mg(1-e^{-kt/m})}{k}\qquad\text{meters per second}
\]
where $m$ is the mass of the object in kilograms, $g$ is the
acceleration due to gravity in meter per second-squared, $t$ is the
amount of time in seconds that the object is falling, and $k$ is a
positive constant that will vary per object and represents the
resistance caused by the atmosphere.

\begin{problem}
Given that the velocity of a falling object in an atmosphere is 
\[
v = \frac{mg(1-e^{-kt/m})}{k}\qquad\text{meters per second,}
\]
what happens as $t$ goes to infinity?
\begin{hint}
Recall that $e^{-x}$ goes to zero as $x$ goes to infinity.
\end{hint}
\begin{prompt}
As $t$ goes to infinity, $v$ goes to \answer{m*g/k}.
\end{prompt}
\end{problem}


Now suppose we want the effect of the air resistance to go to
zero. Remember this is controlled by the coefficient $k$. Hence, we
want $k$ to go to zero. This amounts to computing the following limit
\[
\lim_{k\to 0+} \frac{mg(1-e^{-kt/m})}{k}.
\]
However, this limit is not so easy to compute!

%I am concerned about this example for the following reasons:

%1. The number of variables.  For a first example, I would prefer something which is more "atomically" focused on 
%the relevant content.  The extra variables are a distraction here. Tying in with the number of variables, we really
%have a family of functions V_{k,m}(t) depending on the mass and air resistence, and we want to know the pointwise limit of this
%function as $k \to 0$.  That is just conceptually at a higher level than I think most of these students can deal with.
%We should cover such situations, but not in introductions. 

%This is (perhaps) a major philoshophical difference between the two of us: 
%you seem to prefer "complex" while I prefer "simple" introductions to a topic.

%2. I think the following subtlely false solution is likely from many students:
%Since the limit function should be independant of mass, we must be able to take $m=k$.  Then the expression for velocity
%reduces to g(1-e^{-t}), which is also (coincidentally) independant of $k$.  So the limit must be $g(1-e^{-t})$.
%The reasons this solution is false are certainly beyond a first brush with calculus, and are usually addressed in
%a first analysis course.

%3. This limit could actually be evaluated by students given their currently available technology:
% it is the definition of the derivative of mge^{-xt/m} at the point $x=0$.  So this doesn't "show L'hopital's rule is needed".

%4.  The intended solution is apparently just to recognize that, in a vacuum, the only force acting on the object is gravity
%and the acceleration due to gravity is given by $g$.  So the velocity at time $t$ should  $gt$.   I think this really
%assumes too much physical intuition.  This also takes some "oompf" out of the solution.  We didn't actually 
%use "math" to find the answer.  The student does not have a class of limits called "l'hopitals rule" limits in
%their heads yet, so I think this will just register as a strange example which is not easily generalizable to anything else.

 
\begin{problem}
Consider the following video: \youtube{http://youtu.be/TVAiVVxD-ng}
Based on the video, can you guess the value of:
\[
\lim_{k\to 0+} \frac{mg(1-e^{-kt/m})}{k}
\]
must be?
\begin{hint}
First note there is no air resistance on the Moon, and hence $k$ is
zero.
\end{hint}
\begin{hint}
Next note that both the hammer and the feather fall at the same rate.
Hence the mass of the falling object must not affect its velocity.
\end{hint}
\begin{hint}
Finally remember that if acceleration $a$ is constant, then velocity
after time $t$ is given by $a\cdot t$.
\end{hint}
\begin{prompt}
$\lim_{k\to 0+}\frac{mg(1-e^{-kt/m})}{k} = $\answer{g*t}.
\end{prompt}
\end{problem}

Soon you will learn a technique that will allow you to squash limits
like this with ease.

\begin{xarmaBoost}
  Write down at least \textbf{five} questions for this lecture. After
  you have your questions, label them as ``Level 1,'' ``Level 2,'' or
  ``Level 3'' where:
\begin{description}
\item[Level 1] Means you know the answer, or know exactly how to do
  this problem.
\item[Level 2] Means you think you know how to do the problem, or will
  soon learn how to do the problem.
\item[Level 3] Means you have no idea how to do the problem.
\end{description}
\begin{freeResponse}
\end{freeResponse}
\end{xarmaBoost}

\end{document}
