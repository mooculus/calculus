\documentclass{ximera}

\newcommand{\RR}{\mathbb R}
\renewcommand{\d}{\,d}
\newcommand{\dd}[2][]{\frac{d #1}{d #2}}
\renewcommand{\l}{\ell}
\newcommand{\ddx}{\frac{d}{dx}}
\newcommand{\dfn}{\textbf}
\newcommand{\eval}[1]{\bigg[ #1 \bigg]}


\author{Bart Snapp}

\outcome{Use Green's Theorem as a planimeter.}

\begin{document}
\begin{exercise}
  Consider an analemma:
  \begin{image}
    \begin{tikzpicture}
      \begin{axis}[
          xmin=-1.2,xmax=1.2,ymin=-1.2,ymax=1.2,
          axis lines=center,
          xlabel=$x$, ylabel=$y$,
          unit vector ratio*=1 1 1,
          every axis y label/.style={at=(current axis.above origin),anchor=south},
          every axis x label/.style={at=(current axis.right of origin),anchor=west},
        ]        
        \addplot [very thick, penColor, smooth, domain=(0:360)] ({sin(x)},{sin(2*x)});
      \end{axis}
\end{tikzpicture}
  \end{image}
  The left loop of this curve is drawn in a counterclockwise fashion by
  \[
  \vec{a}(t) = \vector{\sin(t), \sin(2t)}
  \]
  as $t$ runs from $\pi$ to $2\pi$. Let $R$ be the region enclosed by
  this curve.
  
\begin{itemize}
\item Set $\vec{F}(x,y) = \vector{0,x}$ and use Green's theorem to
  set-up, but do not evaluate, a line integral computing the area of
  $R$.
\begin{prompt}
  \[
  \iint_R\d A = \int_{\answer{\pi}}^{\answer{2\pi}} \answer{2\sin(t)\cos(2t)} \d t
  \]
\end{prompt}
\item Set $\vec{G}(x,y) = \vector{-y,0}$ and use Green's theorem to
  set-up, but do not evaluate, a line integral computing the area of
  $R$.
  \begin{prompt}
  \[
  \iint_R\d A = \int_{\answer{\pi}}^{\answer{2\pi}} \answer{-2\sin(t)\cos^2(t)} \d t
  \]
  \end{prompt}
\item Set $\vec{H}(x,y) = \vector{-y/2,x/2}$ and use Green's theorem
  to set-up, but do not evaluate, a line integral computing the area
  of $R$.
  \begin{prompt}
  \[
  \iint_R\d A = \int_{\answer{\pi}}^{\answer{2\pi}} \answer{-\sin^3(t)} \d t
  \]
  \end{prompt}
\end{itemize}
Use one of these integrals to compute the area of one loop of the analemma.
\begin{prompt}
  \[
  \text{area} = \answer{4/3}
  \]
\end{prompt}
\end{exercise}
\end{document}
