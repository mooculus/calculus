\documentclass{ximera}

\newcommand{\RR}{\mathbb R}
\renewcommand{\d}{\,d}
\newcommand{\dd}[2][]{\frac{d #1}{d #2}}
\renewcommand{\l}{\ell}
\newcommand{\ddx}{\frac{d}{dx}}
\newcommand{\dfn}{\textbf}
\newcommand{\eval}[1]{\bigg[ #1 \bigg]}


%\outcome{Find tangent lines to parametric curves}
\author{Jim Talamo}

\begin{document}
\begin{exercise}
Desmos is a very useful tool and can be used to develop good intuition about polar curves.  The steps below show you how to use Desmos to:

\begin{itemize}
\item draw a polar curve
\item  animate a parametric representation of a polar curve in the context of a specific example.
\end{itemize}

Suppose that a curve $C$ is represented by the relationship $r=4 \sin(3\theta)$.

To draw this curve:

\begin{itemize}
\item[1.] Go to https://www.desmos.com/calculator
\item[2.] In Line 1, type  type ``r=4sin(3 theta)" \emph{exactly} as written.
\end{itemize}


You should see the word ``theta'' change to the symbol ``$\theta$'' automatically after you type it.

Now, answer the following questions about this curve:

Does the curve intersect the origin?

\begin{multipleChoice}
\choice[correct]{Yes}
\choice{No}
\end{multipleChoice}


At how many distinct points does the curve have a vertical tangent line?  $\answer{4}$

The minimum $y$-value on the curve is $y=\answer{-4}$.

\begin{exercise}
We have seen that given a polar representation of curve in the form $r=f(\theta)$, we can use the relationships:
\[\begin{cases}
x&=r \cos(\theta) \\
y&=r \sin(\theta)
\end{cases}\]

to create a parametric description of the curve that uses $\theta$ as a parameter by substituting the expression $f(\theta)$ in for $r$.  Here $r= 4 \sin(3\theta)$, so:

\[\begin{cases}
x(\theta) &=r(\theta) \cos(\theta) = 4 \sin(3\theta) \cos(\theta) \\
y(\theta) &=r(\theta) \sin(\theta) = \answer{4 \sin(3\theta) \sin(\theta)}
\end{cases}\]

To see how this curve is drawn as the parameter $\theta$ increases, follow the steps below:

\begin{itemize}
\item[1.] Go to https://www.desmos.com/calculator
\item[2.] Type the following expressions \emph{exactly} as written below:
\begin{itemize}
\item In Line 1, type ``R(t) = 4 sin(3t)''
\item In Line 2, type ``X(t) = R(t)cos(3t)".  
\item In Line 3, type ``Y(t) = R(t)sin(t)"
\item Then, uncheck the boxes left of the expressions to make the graphs disappear.
\item In Line 4, type ``$(X(t),Y(t))$'' and select $0\leq t \leq2\pi$ when the bounds for $t$ arise.  Check on the box next to what you typed, and you should see an interesting curve.
\end{itemize}
\item[3.] Now, uncheck the box next to ``$(X(t),Y(t))$''.
\item[4.] In Line 5, type ''$(X(a),Y(a))$''.  When ``add slider'' pops up as an option, click on it, and choose the same range of values for $a$ as you did for $t$.  That is, select $0 \leq a \leq 2\pi$.
\item[5.] Go back to Line 3 and add ``$\{t<a\}$" right after $Y(t)$ (before the final parenthesis).  

Line 4 should now read $(X(t),Y(t)\{t<a\} )$.

\item[6.] Now click on the Play button next to $a$.  You should see an animation for the curve.  
\begin{itemize}
\item Experiment with the speed of the animation by by clicking on the expression in the ``$<< \quad >>$'' box (which appears only while the curve is being animated)
\item Experiment with the direction in which the curve is traced out by clicking on the arrows that appear to the left of the ``$<< \quad >>$'' box.  When both arrow point right, this shows the \emph{positive orientation} (the direction in which the curve is traversed as $t$ increases).
\end{itemize}
\end{itemize}

Watch the animation of the curve and answer the following questions:

When $\theta$ ranges from $0$ to $2\pi$, how many times in the curve traversed? $\answer{2}$

The curve starts in Quadrant I when $\theta$ initially increases from $\theta =0$.  

Once it  passes through the origin for the first time (for the smallest nonzero value of $\theta$), the curve continues into:
\begin{multipleChoice}
\choice{Quadrant I}
\choice{Quadrant II}
\choice[correct]{Quadrant III}
\choice{Quadrant IV}
\end{multipleChoice}

Once it passes through the origin for the second time (for the second smallest nonzero value of $\theta$), the curve continues into:
\begin{multipleChoice}
\choice{Quadrant I}
\choice[correct]{Quadrant II}
\choice{Quadrant III}
\choice{Quadrant IV}
\end{multipleChoice}

 \begin{remark}
When you are working problems from these exercises, you should check your work using Desmos!
\end{remark}

\end{exercise}
\end{exercise}
\end{document}
