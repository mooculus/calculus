\documentclass{ximera}

\newcommand{\RR}{\mathbb R}
\renewcommand{\d}{\,d}
\newcommand{\dd}[2][]{\frac{d #1}{d #2}}
\renewcommand{\l}{\ell}
\newcommand{\ddx}{\frac{d}{dx}}
\newcommand{\dfn}{\textbf}
\newcommand{\eval}[1]{\bigg[ #1 \bigg]}


\author{Jim Talamo}
\license{Creative Commons 3.0 By-bC}

\outcome{Compute derivatives of Taylor series}


\begin{document}

\begin{exercise}
Suppose that we want to find the Taylor series centered at $x=0$ for $\arctan(x)$.  

Let's take some derivatives:

\begin{align*}
f'(x) &= \answer{\frac{1}{1+x^2}} \\
f''(x) &= \answer{-\frac{2x}{(1+x^2)^2}}\\
\vdots
\end{align*}
 
Further derivatives will involve product, quotient, and chain rule!  

A better way to find the series in summation notation for this is to note that since:

\[
\int \arctan(x) = \frac{1}{1+x^2} \textrm{ then } \arctan(x) +C_1 = \int \frac{1}{1+x^2} \d x
\]

First, let's find the Taylor series centered at $x=0$ for $\frac{1}{1+x^2}$:

\begin{exercise}
To begin, note that $\frac{1}{1-x} = \sum_{k=0}^{\infty} \answer{x^k}$ for $|x| < \answer{1}$.  Since:

\[
\frac{1}{1+x^2} = \frac{1}{1- (\answer{-x^2})}
\] 
we can immediately write the Taylor series centered at $x=0$ for $\frac{1}{1+x^2}$ by replacing $x$ in the above result with $-x^2$:

\[
\frac{1}{1+x^2} = \sum_{k=0}^{\infty} \answer{(-x^2)^k} \textrm{ for } |\answer{-x^2}|<1
\] 

\begin{exercise}
Simplifying this by using the fact that $(-x^2)^k = ((-1) \cdot x^2)^k = (-1)^k \cdot x^{2k}$ gives:

\[
\frac{1}{1+x^2} = \sum_{k=0}^{\infty} (-1)^k x^{2k} \textrm{ for } |x|<1
\] 

We can now integrate:


\[
\int \frac{1}{1+x^2} \d x =\int \left[\sum_{k=0}^{\infty}  (-1)^k x^{2k}  \right] \d x = \sum_{k=0}^{\infty} \left[ \int  (-1)^k x^{2k}  \d x \right]
\]
(interchange $\sum$ and $\int$)

We can interchange the constants and the integral:

\[
\sum_{k=0}^{\infty} \left[ \int (-1)^k x^{2k}  \d x \right] = \sum_{k=0}^{\infty} \left[ \answer{(-1)^k}\int x^{2k} \d x  \right]
\]

(For each $k$, the expression in front of the power of $x$ is constant, so it can be pulled outside of the derivative)

Now, all we have to do is differentiate powers of $x$.  Indeed:

\[
\sum_{k=0}^{\infty} \left[ (-1)^k \int x^{2k} \d x  \right] = \sum_{k=0}^{\infty} \answer{\frac{(-1)^k}{2k+1}}x^{2k+1} +C_2
\]

\begin{exercise}
On one hand, we have that:

\[ \int \frac{1}{1+x^2} \d x = \arctan(x) +C_1 \]
and on the other, we have: 
\[\int \frac{1}{1+x^2} \d x = \sum_{k=0}^{\infty} \frac{(-1)^k}{2k+1}x^{2k+1} +C_2.\]

The conclusion to draw is:

\[
\arctan(x) = \sum_{k=0}^{\infty} \frac{(-1)^k}{2k+1}x^{2k+1} +C 
\]

where $C=C_1-C_2$; note that both integrals produce an arbitrary constant, so these will not necessarily be equal!

Here, we have found $\arctan(x)$ up to a constant, and we know from prior math experience that $\arctan(x)$ certainly is not unspecified up to an arbitrary constant. We can find this constant by noting that since the series has terms just with nonzero powers of $x$, by substituting $x=0$ into both sides:

\[
\arctan(0) = \answer{0} +C
\]

Thus $C= \answer{0}$.

\begin{exercise}
Thus, we have:

\[
\arctan(x) = \sum_{k=0}^{\infty} \frac{(-1)^k}{2k+1}x^{2k+1} 
\]

Since integrating does not change the radius of convergence of a series, and the radius of convergence for $\frac{1}{1+x^2}$ is $\answer{1}$, the radius of convergence for the series for $\arctan(x)$ is $\answer{1}$.

\end{exercise}
\end{exercise} 






\end{exercise}
\end{exercise}
\end{exercise}
\end{document}
