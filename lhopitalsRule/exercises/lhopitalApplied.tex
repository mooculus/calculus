\documentclass{ximera}

\newcommand{\RR}{\mathbb R}
\renewcommand{\d}{\,d}
\newcommand{\dd}[2][]{\frac{d #1}{d #2}}
\renewcommand{\l}{\ell}
\newcommand{\ddx}{\frac{d}{dx}}
\newcommand{\dfn}{\textbf}
\newcommand{\eval}[1]{\bigg[ #1 \bigg]}


\outcome{Recall how to find limits for forms that are not indeterminate.}
\outcome{Define an indeterminate form.}
\outcome{Determine if a form is indeterminate.}
\outcome{Convert indeterminate forms to the form zero over zero or infinity over infinity.}
\outcome{Define l’Hopital’s Rule and identify when it can be used.}
\outcome{Use l’Hopital’s Rule to find limits.}

\author{Nela Lakos}

\begin{document}
\begin{exercise}
The charge in the LCR circuit, with no resistance ($R=0$), with external voltage $E(t)=\sin{(w t)}$, 

and no charge and no current initially is given by

\[
Q_{w}(t)=\frac{w\sin{(w_{0} t)-w_{0}\sin{(w t)}}}{w_{0}(w^2-w_{0}^2)}
\]

where $t$ is time in seconds and $w_{0}=\frac{1}{2}$ is a "natural frequency" of the circuit.

The charge at resonance, $Q(t)$, is defined by the limit

\[
Q(t)=\lim_{w\to w_{0}} Q_{w}(t)=\lim_{w\to w_{0}} \frac{w\sin{(w_{0} t)-w_{0}\sin{(w t)}}}{w_{0}(w^2-w_{0}^2)}
\]

(a) What is the form of the limit?

\[
FORM:\answer{\frac{0}{0}}
\]

(b)  Evaluate the limit to find the expression for $Q(t)$. 

You may use l’Hopital’s Rule.


Keep in mind that $w_{0}=\frac{1}{2}$ when writing your answer. 

\[
Q(t)=\lim_{w\to w_{0}} Q_{w}(t)=\lim_{w\to w_{0}} \frac{w\sin{(w_{0} t)-w_{0}\sin{(w t)}}}{w_{0}(w^2-w_{0}^2)}=2\sin{\left(\frac{ t}{2}\right)-t\cdot\answer{\cos{\left(\frac{t}{2}\right)}}}
\]



\end{exercise}
\end{document}