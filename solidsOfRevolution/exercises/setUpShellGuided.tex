\documentclass{ximera}
\newcommand{\RR}{\mathbb R}
\renewcommand{\d}{\,d}
\newcommand{\dd}[2][]{\frac{d #1}{d #2}}
\renewcommand{\l}{\ell}
\newcommand{\ddx}{\frac{d}{dx}}
\newcommand{\dfn}{\textbf}
\newcommand{\eval}[1]{\bigg[ #1 \bigg]}

\author{Jim Talamo and Alex Beckwith}
\license{Creative Commons 3.0 By-NC}
\outcome{Set up a volume integral using the Washer Method}
\begin{document}
\begin{exercise}

	Let $R$ be the region in the $xy$-plane bounded by $y=0$, $y=\ln x$, $y=2$, and $x=1$. This exercise will walk you through setting up an integral using the Shell Method that will give the volume of the solid generated when $R$ is revolved about the line $x=-1$.
            \begin{image}
            \begin{tikzpicture}
            	\begin{axis}[
            		domain=-2.5:8.5, ymax=2.5,xmax=8.4, ymin=-0.5, xmin=-2.4,
            		axis lines =center, xlabel=$x$, ylabel=$y$,
            		every axis y label/.style={at=(current axis.above origin),anchor=south},
            		every axis x label/.style={at=(current axis.right of origin),anchor=west},
            		axis on top,
            		]
                      
            	\addplot [draw=penColor,very thick,smooth] {2};
            	\addplot [draw=penColor2,very thick,smooth] {ln(x)};
		\addplot [draw=penColor3,very thick,smooth] {0};
		\addplot [draw=penColor4,very thick,smooth] coordinates {(1,-4)(1,12)};
		\addplot [draw=penColor5,very thick,dotted] coordinates {(-1,-0.5)(-1,2.5)};
                       
            	\addplot [name path=A,domain=1:7.4,draw=none] {2};   
            	\addplot [name path=B,domain=1:7.4,draw=none] {ln(x)};
            	\addplot [fillp] fill between[of=A and B];
	                
            	\node at (axis cs:5,1.2) [penColor2] {$y=\ln(x)$};
		\node at (axis cs:4.5,2.15) [penColor] {$y=2$};
		\node at (axis cs:1.8,2.3) [penColor4] {$x=1$};
            	\end{axis}
            \end{tikzpicture}
            \end{image}

Since we are using the Shell Method, the slices must be:
\begin{multipleChoice}
\choice[correct]{parallel}
\choice{perpendicular}
\end{multipleChoice}
to the axis of rotation.  

Slices that are parallel to the axis of rotation $x=-1$ are:
\begin{multipleChoice}
\choice[correct]{vertical}
\choice{horizontal}
\end{multipleChoice}

Since the slices are vertical, we must: 
\begin{multipleChoice}
\choice[correct]{integrate with respect to $x$.}
\choice{integrate with respect to $y$.}
\end{multipleChoice}

Since we must integrate with respect to $x$, we will use the result:

\[V = \int_{x=a}^{x=b}2\pi \rho h \d x \]

to set up the volume.  We must now find the limits of integration as express the radius $\rho$ and the height $h$ in terms of the variable of integration $x$. 

\begin{exercise}
The limits of integration are: $a= \answer{1}$ and $b = \answer{e^2}$. 
\end{exercise}

\begin{exercise}

We thus have a helpful version of the picture of the region $R$ below:

  \begin{image}
            \begin{tikzpicture}
            	\begin{axis}[
            		domain=-2.5:8.5, ymax=2.5,xmax=8.4, ymin=-0.5, xmin=-2.4,
            		axis lines =center, xlabel=$x$, ylabel=$y$,
            		every axis y label/.style={at=(current axis.above origin),anchor=south},
            		every axis x label/.style={at=(current axis.right of origin),anchor=west},
            		axis on top,
            		]
                      
            	\addplot [draw=penColor,very thick,smooth] {2};
            	\addplot [draw=penColor2,very thick,smooth] {ln(x)};
		\addplot [draw=penColor3,very thick,smooth] {0};
		\addplot [draw=penColor4,very thick,smooth] coordinates {(1,-4)(1,12)};
		\addplot [draw=penColor5,very thick,dotted] coordinates {(-1,-0.5)(-1,2.5)};
                       
            	\addplot [name path=A,domain=1:7.4,draw=none] {2};   
            	\addplot [name path=B,domain=1:7.4,draw=none] {ln(x)};
            	\addplot [fillp] fill between[of=A and B];
	                
            	\node at (axis cs:3,.5) [penColor2] {$y=\ln(x)$};
		\node at (axis cs:4.5,2.15) [penColor] {$y=2$};
		\node at (axis cs:1.8,2.3) [penColor4] {$x=1$};
	
		\addplot [draw=penColor, fill = gray!50] plot coordinates {(2.7, 1.1) (3,1.1) (3,2) (2.7,2) (2.7, 1.1)};
          
          %Draw R and r
          \addplot [draw=black!30!red,very thick] coordinates {(-1,1.5)(2.7,1.5)};
          \node at (axis cs:1.7,1.35) [black!30!red] {$\rho$};
          

	 \draw[decoration={brace,raise=.1cm,mirror},decorate,thin] (axis cs:3.05,1.1)--(axis cs:3.05,2);

	 \node at (axis cs:3.6,1.55)  [black!30!blue]  {$h$};
                      
                  	\end{axis}
            \end{tikzpicture}
  \end{image}
            
 We see from the picture that $\rho$ is a:
 \begin{multipleChoice}
 \choice{vertical distance}
 \choice[correct]{horizontal distance}
 \end{multipleChoice}           
            
\begin{exercise}
Since $\rho$ is the distance from the axis of rotation to the slice, and this is a horizontal distance, we find $\rho = x_{right}-x_{left}$.
\begin{multipleChoice}
 \choice[correct]{$x_{right} = x$}
 \choice{$x_{right} = \ln(x)$}
  \choice{$x_{right} = -1$}
\end{multipleChoice}       

\begin{multipleChoice}
 \choice{$x_{left} = \ln(x)$}
 \choice{$x_{left} = 1$}
  \choice[correct]{$x_{left} = -1$}
\end{multipleChoice}   

\begin{hint}
Remember that the process of find the volume requires us to express the volume of each shell in terms of the $x$-value where the shell is located.  An arbitrary shell is located at $x$!
\end{hint}

So, $\rho= \answer{x-(-1)}$.
 \end{exercise}
 
  We see from the picture that $h$ is a:
 \begin{multipleChoice}
 \choice[correct]{vertical distance}
 \choice{horizontal distance}
 \end{multipleChoice}           
 
 \begin{exercise}
Since $h$ is the height of the slice, and this is a vertical distance, we find $h = y_{top}-y_{bot}$.
\begin{multipleChoice}
 \choice[correct]{$y_{top} = 2$}
 \choice{$y_{top} = \ln(x)$}
\end{multipleChoice}       

\begin{multipleChoice}
 \choice{$y_{bot} =2$}
 \choice[correct]{$y_{bot} =  \ln(x)$}
\end{multipleChoice}   

So, $h= \answer{2-\ln(x)}$.
 \end{exercise}
           
\begin{exercise}

Using \[V = \int_{x=a}^{x=b} 2\pi \rho h \d x, \] we find that an integral that gives the volume of the solid of revolution is:            
	\[
	V= \int_{x=\answer{1}}^{x=\answer{e^2}}
	\answer{2\pi(x+1)(2-\ln(x))}\d x
	\]
\end{exercise}
\end{exercise}
\end{exercise}

\end{document}