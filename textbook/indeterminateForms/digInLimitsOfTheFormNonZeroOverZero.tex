\documentclass{ximera}

\newcommand{\RR}{\mathbb R}
\renewcommand{\d}{\,d}
\newcommand{\dd}[2][]{\frac{d #1}{d #2}}
\renewcommand{\l}{\ell}
\newcommand{\ddx}{\frac{d}{dx}}
\newcommand{\dfn}{\textbf}
\newcommand{\eval}[1]{\bigg[ #1 \bigg]}


\title[Dig-In:]{Limits of the form nonzero over zero}

\begin{document}
\begin{abstract}
  We want to solve limits that have the form nonzero over zero.
\end{abstract}

\maketitle

%This chapter will consist of three dig in's 1. The Almost Equal Theorem 2. Limits of the form non-zero over zero 3. Determinate and indeterminate forms 

Let's cut to the chase:
\begin{definition}
  A limit
  \[
  \lim_{x\to a} \frac{f(x)}{g(x)}
  \]
  is said to be of the form \numOverZero\ if
  \[
  \lim_{x\to a} f(x) = k\qquad\text{and}\qquad \lim_{x\to a} g(x) =0.
  \]
  where $k$ is some nonzero constant.
\end{definition}

What can we say about limits which have this form?
Let's start by investigating a simple example.  Let's consider
\[
\lim_{x\to 0^+} \frac{1}{x}.
\]
Since
\[
\lim_{x\to 0^+} 1 = 1\qquad\text{and}\qquad\lim_{x\to0^+} x = 0 
\]
this limit is of the form \numOverZero.  In order to understand what
is happening, let's plug in some numbers that get closer and closer to
$0$ from the right side:
\begin{align*}
f(1) &= \frac{1}{1}=1\\
f\left(\frac{1}{10}\right) &= \frac{1}{\frac{1}{10}}=10\\
f\left(\frac{1}{100}\right) &= \frac{1}{\frac{1}{100}}=100\\
f\left(\frac{1}{1000}\right) &= \frac{1}{\frac{1}{1000}}=1000\\ 
\end{align*}


We can see that the closer to zero the number we put in for $x$, the larger the $f(x)$ value is.  In fact, you can make the $f(x)$- value as large as you want if you put in an $x$-value close enough to zero.  We can also see this as we look at the graph of the function.

%\begin{figure}
\begin{tikzpicture}
	\begin{axis}[
            domain=-1:1,
            ymax=100,
            samples=100,
            axis lines =middle, xlabel=$x$, ylabel=$y$,
            every axis y label/.style={at=(current axis.above origin),anchor=south},
            every axis x label/.style={at=(current axis.right of origin),anchor=west}
          ]
	  \addplot [very thick, penColor, smooth, domain=(-1:-.1)] {1/x};
          \addplot [very thick, penColor, smooth, domain=(.01:1)] {1/x};
          \addplot [textColor, dashed] plot coordinates {(0,0) (0,100)};
        \end{axis}
\end{tikzpicture}
%\caption{A plot of $f(x)=\protect\frac{1}{x^2}$.}
%\label{plot:1/x^2}
%\end{figure}

\begin{definition}\label{def:inflimit}\index{limit!infinite}\index{infinite limit}
If $f(x)$ grows arbitrarily large as $x$ approaches $a$, we write
\[
\lim_{x\to a} f(x) = \infty
\]
and say that the limit of $f(x)$ \textbf{approaches infinity} as $x$
goes to $a$.

If $|f(x)|$ grows arbitrarily large as $x$ approaches $a$ and $f(x)$ is
negative, we write
\[
\lim_{x\to a} f(x) = -\infty
\]
and say that the limit of $f(x)$ \textbf{approaches negative infinity}
as $x$ goes to $a$.
\end{definition}

We might want to ask, what happens to $\lim_{x\to 0^-} \frac{1}{x}  \sim \frac{\#}{0}$?  Considering points which are less than zero and approaching zero, we see:

$f(-1) = \frac{1}{-1}=-1$\\ \vspace{.1in}
$f\left(-\frac{1}{10}\right) = \frac{1}{-\frac{1}{10}}=-10$\\ \vspace{.1in}
$f\left(-\frac{1}{100}\right) = \frac{1}{-\frac{1}{100}}=-100$\\ \vspace{.1in}
$f\left(-\frac{1}{1000}\right) = \frac{1}{-\frac{1}{1000}}=-1000$\\ 

It appears that $\lim_{x\to 0^-} \frac{1}{x} = -\infty$.  

What if we consider the general limit as $x$ approaches zero, $\lim_{x\to 0} \frac{1}{x}$?  Then, we have that the limit from the left side does not equal the limit from the right side.  Therefore, the limit does not exist and we write $\lim_{x\to 0} \frac{1}{x} = DNE$.

\begin{theorem}[Limits of the Form Non-Zero Over Zero]
Let $\lim_{x\to a} f(x) \sim \frac{\#}{0}$.  Then, the $\lim_{x\to a^+} f(x) = \pm \infty$ and $\lim_{x\to a^-} f(x) = \pm \infty$, depending on the sign of $f(x)$.  If the limit from the left and the limit from the right agree, then $\lim_{x\to a} f(x)$ is said to have the same limit.  If the limit from the left and the limit from the right do not agree, then $\lim_{x\to a} f(x) = DNE$.  Special note: Since $\infty$ is not a number, none of these limits exist.  Writing that a limit equals $\infty$ or $-\infty$ is a more specific answer than just saying the limit generally does not exist.
\end{theorem}

\begin{example}
Find $\lim_{x\to -1} \frac{1}{(x+1)^2} = \answer[given]{\infty}$
\begin{explanation}
First, we look at the form of this limit.  We see that $\lim_{x\to -1} \frac{1}{(x+1)^2} \sim \frac{\#}{0}$.  This tells us that we need to consider the left and right hand limits separately.  For each of the left and right hand limits, we need to determine whether the limit will equal $+\infty$ or $-\infty$.  We will do this by looking at the sign of the nearby numbers.  When we write $a^+$, we will mean that we are thinking about numbers just a little bit bigger than $a$.  When we write, $a^-$, we will mean that we are thinking about numbers just a little bit smaller than $a$.\\
\begin{equation}
\lim_{x\to -1^+} \frac{1}{(x+1)^2} = \frac{1}{((-1)^+ +1)^2}
 = \frac{1}{(0^+)^2}
 = \frac{1}{(0^+)}
 =\frac{+}{+} = +
\end{equation}

Therefore, $\lim_{x\to -1^+} \frac{1}{(x+1)^2} = \infty$.

Note that in the computation above, we are only trying to determine the sign of the solution.  Otherwise, this reasoning would not make sense.  Let's look at the left sided limit now.
\begin{equation}
\lim_{x\to -1^-} \frac{1}{(x+1)^2} = \frac{1}{((-1)^- +1)^2}
 = \frac{1}{(0^-)^2}
 = \frac{1}{(0^+)}
 =\frac{+}{+} = +
\end{equation}

Therefore, $\lim_{x\to -1^-} \frac{1}{(x+1)^2} = \infty$.

Since both sides agree, we can conclude that $\lim_{x\to -1} \frac{1}{(x+1)^2} = \infty$.  Looking at the graph of this function, we see that this answer makes sense.\\
\begin{tikzpicture}
	\begin{axis}[
            domain=-2:1,
            ymax=100,
            samples=100,
            axis lines =middle, xlabel=$x$, ylabel=$y$,
            every axis y label/.style={at=(current axis.above origin),anchor=south},
            every axis x label/.style={at=(current axis.right of origin),anchor=west}
          ]
	  \addplot [very thick, penColor, smooth, domain=(-2:-1.1)] {1/(x+1)^2};
          \addplot [very thick, penColor, smooth, domain=(-.9:1)] {1/(x+1)^2};
          \addplot [textColor, dashed] plot coordinates {(-1,0) (-1,100)};
        \end{axis}
\end{tikzpicture}

\end{explanation}
\end{example}



\begin{example}
Find $\lim_{x\to 2} \frac{x^2-9x+14}{x^2-5x+6} = \answer[given]{5}$ and  $\lim_{x\to 3} \frac{x^2-9x+14}{x^2-5x+6}= \answer[given]{DNE} $

\begin{explanation}
Start by factoring both the numerator and the denominator:
\[
\frac{x^2-9x+14}{x^2-5x+6} = \frac{(x-2)(x-7)}{(x-2)(x-3)}
\]
This will allow us to quickly see the forms of the limits.  Let's first look at $\lim_{x\to 2} \frac{(x-2)(x-7)}{(x-2)(x-3)} \sim \frac{0}{0}$.  Recall that $\frac{0}{0}$ is an indeterminate form, and therefore we cannot tell the answer from just looking at the form.  We can easily see from our factored expression how to come up with an almost equal function which was can use to find this limit.  We just cancel the $(x-2)$ term from the numerator and denominator of this fraction.
\begin{align*}
\lim_{x\to 2} \frac{(x-2)(x-7)}{(x-2)(x-3)} &= \lim_{x\to 2} \frac{(x-7)}{(x-3)}\\
&= \frac{-5}{-1}\\
&=5.
\end{align*}

Now consider $\lim_{x\to 3} \frac{(x-2)(x-7)}{(x-2)(x-3)} \sim \frac{-4}{0} \sim \frac{\#}{0}$.   We can still use the almost equal function $\frac{(x-7)}{(x-3)}$ to find this limit if we wish, since we know these two functions are equal on all points near 3.  Since this limit has the form $\frac{\#}{0}$, we need to look at both the right and left limits and consider the sign of the solution to determine whether each limit is $+\infty$ or $-\infty$.

\begin{align*}
\lim_{x\to 3^+} \frac{(x-2)(x-7)}{(x-2)(x-3)} &= \lim_{x\to 3^+} \frac{(x-7)}{(x-3)}\\
&= \lim_{x\to 3^+}\frac{-4}{3^+ -3}\\
&= \lim_{x\to 3^+}\frac{-4}{0^+}\\
&= \lim_{x\to 3^+}\frac{-}{+} = -\\.
\end{align*}

Therefore, $\lim_{x\to 3^+} \frac{(x-2)(x-7)}{(x-2)(x-3)} = \infty$.

\begin{align*}
\lim_{x\to 3^-} \frac{(x-2)(x-7)}{(x-2)(x-3)} &= \lim_{x\to 3^-} \frac{(x-7)}{(x-3)}\\
&= \lim_{x\to 3^-}\frac{-4}{3^- -3}\\
&= \lim_{x\to 3^-}\frac{-4}{0^-}\\
&= \lim_{x\to 3^+}\frac{-}{-} = +\\.
\end{align*}

Therefore, $\lim_{x\to 3^-} \frac{(x-2)(x-7)}{(x-2)(x-3)} = -\infty$.

Since the limits from both sides do not agree, we say $\lim_{x\to 3} \frac{(x-2)(x-7)}{(x-2)(x-3)} = DNE$.

Looking at the graph of this function, we can see that our answers makes sense.

\begin{tikzpicture}
	\begin{axis}[
            domain=1:4,
            ymax=50,
            ymin=-50,
            samples=100,
            axis lines =middle, xlabel=$x$, ylabel=$y$,
            every axis y label/.style={at=(current axis.above origin),anchor=south},
            every axis x label/.style={at=(current axis.right of origin),anchor=west}
          ]
	  \addplot [very thick, penColor, smooth, domain=(3.02:4)] {(x-7)/(x-3)};
          \addplot [very thick, penColor, smooth, domain=(1:2.98)] {(x-7)/(x-3)};
          \addplot [textColor, dashed] plot coordinates {(3,-50) (3,50)};
          \addplot[color=penColor,fill=background,only marks,mark=*] coordinates{(2,5)};  %% open hole
        \end{axis}
\end{tikzpicture}

\end{explanation}
\end{example}


\end{document}
