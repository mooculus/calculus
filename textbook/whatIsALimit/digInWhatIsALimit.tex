\documentclass{ximera}

\newcommand{\RR}{\mathbb R}
\renewcommand{\d}{\,d}
\newcommand{\dd}[2][]{\frac{d #1}{d #2}}
\renewcommand{\l}{\ell}
\newcommand{\ddx}{\frac{d}{dx}}
\newcommand{\dfn}{\textbf}
\newcommand{\eval}[1]{\bigg[ #1 \bigg]}


\title[Dig-In:]{What is a limit?}
\begin{document}
\begin{abstract}
  Here we introduce limits.
\end{abstract}
\maketitle


Consider the function:
\[
f(x) = \frac{x^2 - 3x + 2}{x-2}
\]
While $f(x)$ is undefined at $x=2$, we can still plot $f(x)$ at other
values, see Figure~\ref{plot:(x^2 - 3x + 2)/(x-2)}. Examining
Table~\ref{table:(x^2 - 3x + 2)/(x-2)}, we see that as $x$ approaches
$2$, $f(x)$ approaches $1$. We write this: 
\[
\text{As $x \to 2$, $f(x) \to 1$}\qquad\text{or}\qquad \lim_{x\to 2} f(x) = 1.
\]
Intuitively, $\lim_{x\to a} f(x) = L$ when the value of $f(x)$ can
be made arbitrarily close to $L$ by making $x$ sufficiently close, but
not equal to, $a$.  This leads us to the formal definition of a
\textit{limit}.

\begin{image}
\begin{tikzpicture}
	\begin{axis}[
            domain=-2:4,
            axis lines =middle, xlabel=$x$, ylabel=$y$,
            every axis y label/.style={at=(current axis.above origin),anchor=south},
            every axis x label/.style={at=(current axis.right of origin),anchor=west},
            grid=both,
            grid style={dashed, gridColor},
            xtick={-2,...,4},
            ytick={-3,...,3},
          ]
	  \addplot [very thick, penColor, smooth] {x-1};
          \addplot[color=penColor,fill=background,only marks,mark=*] coordinates{(2,1)};  %% open hole
        \end{axis}
\end{tikzpicture}
%% \caption{A plot of $f(x)=\protect\frac{x^2 - 3x + 2}{x-2}$.}
%% \label{plot:(x^2 - 3x + 2)/(x-2)}
\end{image}

\[
\begin{tchart}{ll}
 x & f(x) \\ \hline
 1.7 &  0.7 \\
 1.9 &  0.9 \\
 1.99 &  0.99 \\
 1.999 &  0.999 \\
  2 &  \text{undefined}
\end{tchart}\qquad
\begin{tchart}{ll}
 x & f(x) \\ \hline
  2 & \text{undefined}\\
 2.001&  1.001\\
 2.01&  1.01\\
 2.1 &  1.1 \\
 2.3 &  1.3 \\
\end{tchart}
\]
%% \caption{Values of $f(x)=\protect\frac{x^2 - 3x + 2}{x-2}$.}
%% \label{table:(x^2 - 3x + 2)/(x-2)}

\end{document}
