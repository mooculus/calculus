\documentclass{ximera}

\newcommand{\RR}{\mathbb R}
\renewcommand{\d}{\,d}
\newcommand{\dd}[2][]{\frac{d #1}{d #2}}
\renewcommand{\l}{\ell}
\newcommand{\ddx}{\frac{d}{dx}}
\newcommand{\dfn}{\textbf}
\newcommand{\eval}[1]{\bigg[ #1 \bigg]}


\begin{document}
\begin{problem}
  Sketch the region bounded by the $(x,y)$-plane, the $(y,z)$-plane,
  the $(x,z)$-plane and the plane $x+2y + 3z = 6$. \textbf{Label
    relevant points.}
    \[
    \begin{tikzpicture}[x=0.7cm,y=0.7cm,z=0.5cm]
      % The axes %%FAKED
      \draw[->] (xyz cs:x=-1) -- (xyz cs:x=5) node[above] {$y$};
      \draw[->] (xyz cs:y=-1) -- (xyz cs:y=4) node[right] {$z$};
      \draw[->] (xyz cs:z=1) -- (xyz cs:z=-7) node[above] {$x$};
      % The thin ticks
      \foreach \coo in {1,0,...,-6} %%x
               {
                 \draw (xyz cs:y=-0.15pt,z=\coo) -- (xyz cs:y=0.15pt,z=\coo);
               }
      \foreach \coo in {-1,0,...,4}%% y
               {
                 \draw (\coo,-1.5pt) -- (\coo,1.5pt);
               }
      \foreach \coo in {-1,0,...,3}
               {
                 \draw (-1.5pt,\coo) -- (1.5pt,\coo);
               }         
    \end{tikzpicture}
    \]
    \begin{prompt}
      \begin{multipleChoice}
        \choice[correct]{I've drawn this.}
      \end{multipleChoice}
      \begin{feedback}[correct]
        \begin{image}
          \includegraphics{scan1.jpg}
        \end{image}
      \end{feedback}
    \end{prompt}
\end{problem}

\begin{problem}
  Let $F(x,y) = \frac{6-x-2y}{3}$. Let
  \[
  R =\{(x,y,z): 0\le x \le 6, 0 \le y \le \frac{-x}{2} + 3\}
  \]
  Compute: $\iint_R \d S$. Give a very brief explanation.
  \begin{prompt}
    \[
    \mathrm{Area} = \answer{3\sqrt{14}}
    \]
    \begin{feedback}[correct]
      One could either use half the magnitude cross product of
      something like $|\vector{-6,0,2}\cross\vector{-6,3,0}|$, or one
      could compute an interated integral like:
      \[
      \int_0^6\int_0^{-x/2+3} \sqrt{14} \d y \d x
      \]
    \end{feedback}
  \end{prompt}
  \vfill
\end{problem}




\hrule

\begin{problem}
  \begin{selectAll}
    \choice[correct]{The function $F(x,y) = x^6+y^4$ has exactly one minimum.}
    \choice{If $F^{(1,0)}(1,2)=F^{(0,1)}(1,2)=0$, $F^{(2,0)}(1,2) >
      0$, $F^{(0,2)}(1,2)>0$, and $F^{(1,1)}(1,2) <0$, then $(1,2)$ is
      a local minimum for $F$.}
    \choice[correct]{If $F^{(1,0)}(3,4)=0$, $F^{(0,1)}(3,4)=0$,
      $F^{(2,0)}(3,4) =0$, $F^{(0,2)}(3,4)>0$, and $F^{(1,1)}(3,4)
      <0$, then $(3,4)$ is a saddle-point for $F$.}    
    \choice[correct]{If $F(x,y)$ has a critical point at $(-3,1)$, then
      $G(x,y)=\left(F(x,y)\right)^6$ also has a critical point at
      $(-3,1)$.}
    \choice[correct]{If $(2,-3)$ is a critical point of $F$, $F^{(1,1)}(2,-3) =
      0$, $F^{(2,0)}(2,-3)>0$, and $F^{(0,2)}(2,-3)<0$, then at
      $(2,-3)$ $F$ locally looks like a hyperbolic paraboloid.}
    \choice{None of these.}
  \end{selectAll}    
\end{problem}


\end{document}
