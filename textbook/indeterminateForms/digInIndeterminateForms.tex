\documentclass{ximera}
\usepackage[makeroom]{cancel}
% I suspect this package is supposed to go somewhere else and not just at the top of this document

\newcommand{\RR}{\mathbb R}
\renewcommand{\d}{\,d}
\newcommand{\dd}[2][]{\frac{d #1}{d #2}}
\renewcommand{\l}{\ell}
\newcommand{\ddx}{\frac{d}{dx}}
\newcommand{\dfn}{\textbf}
\newcommand{\eval}[1]{\bigg[ #1 \bigg]}


\title{Determinate and Indeterminate Forms}

\begin{document}
\begin{abstract}
  We want to be able to determine if a limit is of a determinate or indeterminate form and then compute limits with determinate forms.
\end{abstract}

\maketitle

%This chapter will consist of three dig in's 1. The Almost Equal Theorem 2. Limits of the form non-zero over zero 3. Determinate and indeterminate forms 


\textit{indeterminate form}.

\begin{definition}[List of Indeterminate Forms]\hfil
\begin{itemize}
\item[\textbf{0/0}] This refers to a limit of the form $\lim_{x\to a}
  \frac{f(x)}{g(x)}$ where $f(x)\to 0$ and $g(x)\to 0$ as $x\to a$.
\item[\textbf{$\pmb\infty$/$\pmb\infty$}] This refers to a limit of the form $\lim_{x\to a}
  \frac{f(x)}{g(x)}$ where $f(x)\to \infty$ and $g(x)\to \infty$ as $x\to a$.
\item[\textbf{0\,$\pmb{\cdot\infty}$}] This refers to a limit of the form $\lim_{x\to a}
  \left(f(x)\cdot g(x)\right)$ where $f(x)\to 0$ and $g(x)\to \infty$ as $x\to a$.
\item[\textbf{$\pmb\infty$--$\pmb\infty$}] This refers to a limit of the form $\lim_{x\to a}\left(
  f(x)-g(x)\right)$ where $f(x)\to \infty$ and $g(x)\to \infty$ as $x\to a$.

\item[\textbf{1$^{\pmb\infty}$}] This refers to a limit of the form $\lim_{x\to a}
  f(x)^{g(x)}$ where $f(x)\to 1$ and $g(x)\to \infty$ as $x\to a$.
\item[\textbf{0$^\text{0}$}] This refers to a limit of the form $\lim_{x\to a}
  f(x)^{g(x)}$ where $f(x)\to 0$ and $g(x)\to 0$ as $x\to a$.
\item[\textbf{$\pmb\infty^\text{0}$}] This refers to a limit of the form $\lim_{x\to a}
  f(x)^{g(x)}$ where $f(x)\to \infty$ and $g(x)\to 0$ as $x\to a$.
\end{itemize}
 
\end{definition}

In each of these cases, the value of the limit is \textbf{not}
immediately obvious.  For example, if we have a limit of the form
$\textbf{0$^\text{0}$}$, then the base of the exponent is ``driving the
limit to $0$,'' while the exponent is ``driving the limit to $1$.''
In this tug of war, it is not clear which of the two combatants will
triumph, or (if neither ``wins'') to what value they will reach a
compromise. Hence, a careful analysis is required.




\end{document}
