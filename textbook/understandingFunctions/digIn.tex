\documentclass{ximera}

\newcommand{\RR}{\mathbb R}
\renewcommand{\d}{\,d}
\newcommand{\dd}[2][]{\frac{d #1}{d #2}}
\renewcommand{\l}{\ell}
\newcommand{\ddx}{\frac{d}{dx}}
\newcommand{\dfn}{\textbf}
\newcommand{\eval}[1]{\bigg[ #1 \bigg]}


\begin{document}

\chapter{Understanding Functions}

\section{For Each Input, Exactly One Output}

Life is complex. Part of this complexity stems from the fact that
there are many relationships between seemingly unrelated events. Armed
with mathematics we seek to understand the world, and hence we need
tools for talking about these relationships.  In mathematics, we refer to these relationships as relations.  


A \textit{function} is a relation between sets of objects that can be
thought of as a ``mathematical machine.'' 
%% Insert picture 1
This means for each input,there is exactly one output. 
%% Insert picture 2

\marginnote{Something as simple as a dictionary could be thought of as
  a relation, as it connects \textit{words} to
  \textit{definitions}. However, a dictionary is not a function, as
  there are words with multiple definitions. On the other hand, if
  each word only had a single definition, then it would be a
  function.}

Let's say this explicitly.

\begin{definition}\index{function}
A \textbf{function} is a relation between sets, where for each input,
there is exactly one output.
\end{definition}

Moreover, whenever we talk about functions, we should try to
explicitly state what type of things the inputs are and what type of
things the outputs are.  In calculus, functions often define a
relation from (a subset of) the real numbers to (a subset of) the real
numbers.


\marginnote[.5in]{While the name of the function is technically ``$f$,'' we
  will abuse notation and call the function ``$f(x)$'' to remind the
  reader that it is a function.}
\begin{example}
Consider the function $f$ that maps from the real numbers to the real
numbers by taking a number and mapping it to its cube:
\begin{align*}
1 &\mapsto 1\\
-2 &\mapsto -8\\
1.5 &\mapsto 3.375
\end{align*}
and so on. This function can be described by the formula $f(x)=x^3$ or
by the plot shown in Figure~\ref{plot:fxn x^3}.
\end{example}

\begin{warning}
A function is a relation (such that for each input, there is exactly one
output) between sets and should not be confused with either its
formula or its plot.
\begin{itemize}
\item A formula merely describes the mapping using algebra.
\item A plot merely describes the mapping using pictures. 
\end{itemize}
\end{warning}


\begin{marginfigure}[0in]
\begin{tikzpicture}
	\begin{axis}[
            domain=-2:2,
            axis lines =middle, xlabel=$x$, ylabel=$y$,
            every axis y label/.style={at=(current axis.above origin),anchor=south},
            every axis x label/.style={at=(current axis.right of origin),anchor=west},
          ]
	  \addplot [very thick, penColor, smooth] {x^3};
        \end{axis}
\end{tikzpicture}
\caption{A plot of $f(x)=x^3$. Here we can see that for each input (a
  value on the $x$-axis), there is exactly one output (a value on the
  $y$-axis).}
\label{plot:fxn x^3}
\end{marginfigure}



\begin{example}
Consider the \textit{greatest integer function}, denoted by
\[
f(x) = \lfloor x \rfloor.
\]
This is the function that maps any real number $x$ to the greatest
integer less than or equal to $x$. See Figure~\ref{plot:greatest-integer fxn} for a plot of
this function. Some might be confused because here we have multiple
inputs that give the same output. However, this is not a problem. To
be a function, we merely need to check that for each input, there is exactly
one output, and this is satisfied.
\end{example}
\begin{marginfigure}[0in]
\begin{tikzpicture}
	\begin{axis}[
            domain=-2:4,
            axis lines =middle, xlabel=$x$, ylabel=$y$,
            every axis y label/.style={at=(current axis.above origin),anchor=south},
            every axis x label/.style={at=(current axis.right of origin),anchor=west},
            clip=false,
            %axis on top,
          ]
          \addplot [textColor, very thin, domain=(0:2.3)] {0}; % puts the axis back, axis on top clobbers our open holes
          \addplot [textColor, very thin] plot coordinates {(0,0) (0,2)}; % puts the axis back, axis on top clobbers our open holes
	  \addplot [very thick, penColor, domain=(-2:-1)] {-2};
          \addplot [very thick, penColor, domain=(-1:0)] {-1};
          \addplot [very thick, penColor, domain=(0:1)] {0};
          \addplot [very thick, penColor, domain=(1:2)] {1};
          \addplot [very thick, penColor, domain=(2:3)] {2};
          \addplot [very thick, penColor, domain=(3:4)] {3};
          \addplot[color=penColor,fill=penColor,only marks,mark=*] coordinates{(-2,-2)};  %% closed hole          
          \addplot[color=penColor,fill=penColor,only marks,mark=*] coordinates{(-1,-1)};  %% closed hole          
          \addplot[color=penColor,fill=penColor,only marks,mark=*] coordinates{(0,0)};  %% closed hole          
          \addplot[color=penColor,fill=penColor,only marks,mark=*] coordinates{(1,1)};  %% closed hole          
          \addplot[color=penColor,fill=penColor,only marks,mark=*] coordinates{(2,2)};  %% closed hole  
          \addplot[color=penColor,fill=penColor,only marks,mark=*] coordinates{(3,3)};  %% closed hole                  
          \addplot[color=penColor,fill=background,only marks,mark=*] coordinates{(-1,-2)};  %% open hole
          \addplot[color=penColor,fill=background,only marks,mark=*] coordinates{(0,-1)};  %% open hole
          \addplot[color=penColor,fill=background,only marks,mark=*] coordinates{(1,0)};  %% open hole
          \addplot[color=penColor,fill=background,only marks,mark=*] coordinates{(2,1)};  %% open hole
          \addplot[color=penColor,fill=background,only marks,mark=*] coordinates{(3,2)};  %% open hole
          \addplot[color=penColor,fill=background,only marks,mark=*] coordinates{(4,3)};  %% open hole
        \end{axis}
\end{tikzpicture}
\caption{A plot of $f(x)=\lfloor x\rfloor$. Here we can see that for each input (a
  value on the $x$-axis), there is exactly one output (a value on the
  $y$-axis).}
\label{plot:greatest-integer fxn}
\end{marginfigure}


Notice that both Figure 1 and Figure 2 pass the vertical line test.  A vertical line can be drawn through any part of the plot and the vertical line will only intersect the plot once.  This tells us the plot is a function.

Just to remind you, a function maps from one set to another. We call
the set a function is mapping from the \textbf{domain}\index{domain}
or \textit{source} and we call the set a function is mapping to the
\textbf{range}\index{range} or \textit{target}.  In our previous
examples the domain and range have both been the real numbers, denoted
by $\R$. In our next examples we show that this is not always the
case.


\begin{example}
Consider the function that maps non-negative real numbers to their positive square root. This function is denoted by 
\[
f(x) = \sqrt{x}.
\]
The domain is $\{x\left| x\ge 0\}$ written in set notation or$[0,\infty )$ in interval notation.  In calculus, we'll most often use intercal notation.  The range is $(-\infty,\infty)$.
See Figure~\ref{plot:sqrt fxn} for a plot of $f(x) = \sqrt{x}$.
\end{example}

\begin{marginfigure}[0in]
\begin{tikzpicture}
	\begin{axis}[
            xmin=-8,xmax=8,
            ymin=-5,ymax=5,
            domain=0:8,
            axis lines =middle, xlabel=$x$, ylabel=$y$,
            every axis y label/.style={at=(current axis.above origin),anchor=south},
            every axis x label/.style={at=(current axis.right of origin),anchor=west},
          ]
	  \addplot [very thick, penColor, smooth,samples=100] {sqrt(x)};
        \end{axis}
\end{tikzpicture}
\caption{A plot of $f(x)=\sqrt{x}$. Here we can see that for each
  input (a non-negative value on the $x$-axis), there is exactly one
  output (a positive value on the $y$-axis).}
\label{plot:sqrt fxn}
\end{marginfigure}

Note: $\sqrt{x^2} = |x|$  Why?  Although $\sqrt{x^2}$ may appear to simplify to $x$, let's see what happens when we plug in values.
%%Insert picture 3
We see that $\sqrt{x^2}\ne x$ as $f(-2)=2$ and $f(-1)=1$.  Instead, we see that $\sqrt{x^2} = |x|$.  The domain of $f(x)=\sqrt{x^2}$is $(-\infty,\infty)$ and the range is $[0,\infty)$.  See figure for plot.
%%Insert picture 4


Finally, we will consider a function whose domain is all real numbers
except for a single point.

\begin{example}
Consider the function defined by 
\[
f(x) = \frac{x^2 - 3x + 2}{x-2} = \frac{(x-2)(x-1)}{(x-2)}
\]
This function may seem innocent enough; however, it is undefined at
$x=2$. Why? Recall that we cannot divide bt zero and therefore the function is undefined when the denominator, $x-2$ equals $0$.  We have $x-2 = 0$ when $x=2$.  Therefore, the domain is $(\-infty ,2)\cup (2,\infty )$  
See Figure~\ref{plot:point undfed fxn} for a plot of this function.
\end{example}

Like Devyn and Riley, you may be wondering, are $f(x) = \frac{x^2 - 3x + 2}{x-2} = \frac{(x-2)(x-1)}{(x-2)}$ and $g(x) = x-1$ the same function?  What if we compare plots?  See figures 4 and 5.  What if we compare domain?  We see a hole in $f(x)$ at $x=2$.  This is where $f(X)$ is undefined.  On the other hand, there is no hole in $g(x)$ and the domain is $(-\infty ,\infty )$.  Thus, these are not the same function.  What we can say is that $f(x)=g(x)$ or $f(x)=x-2$ when $x \ne 2$.  It is important to specify the domain when we have algeraically simplified. 

%%Insert margin picture 5 which is fgure 5 referenced above


\begin{marginfigure}[0in]
\begin{tikzpicture}
	\begin{axis}[
            domain=-2:4,
            axis lines =middle, xlabel=$x$, ylabel=$y$,
            every axis y label/.style={at=(current axis.above origin),anchor=south},
            every axis x label/.style={at=(current axis.right of origin),anchor=west},
            xtick={-2,...,4},
            ytick={-3,...,3},
          ]
	  \addplot [very thick, penColor, smooth] {x-1};
          \addplot[color=penColor,fill=background,only marks,mark=*] coordinates{(2,1)};  %% open hole
        \end{axis}
\end{tikzpicture}
\caption{A plot of $f(x)=\protect\frac{x^2 - 3x + 2}{x-2}$. Here we
  can see that for each input (any value on the $x$-axis except for
  $x=2$), there is exactly one output (a value on the $y$-axis).}
\label{plot:point undfed fxn}
\end{marginfigure}

%%Insert picture 6


\section {Combining and Composition of Functions}

Functions can be combined using our standard operations: addition, subtraction, multiplication and division.

\being{example}
Suppose we have: 
 $\begin{align}
  & f(x)={{x}^{2}}+4 \\ 
 & g(x)=\frac{1}{x} \\ 
 & h(x)=3+\sqrt{x} \\ 
\end{align}$ 

Let's first note the domain of each function.  The domain of f is $(-\infty ,\infty )$ , domain of g is $(-\infty ,0)\cup (0,\infty )$ , domain of h is $[0,\infty )$ 

a)       Find$f+g$ 
Solution: $f+g={{x}^{2}}+4+\frac{1}{x}$  The domain is $(-\infty ,0)\cup (0,\infty )$
b)	Find$g-h$ 
Solution: $g-h=\frac{1}{x}-(3+\sqrt{x})=\frac{1}{x}-3-\sqrt{x}$ The domain is $(0,\infty )$
c)	Find$f\cdot h$ 
Solution: $f\cdot g=\left( {{x}^{2}}+4 \right)\cdot \left( 3+\sqrt{x} \right)=3{{x}^{2}}+4\sqrt{x}+{{x}^{3/2}}+12$ The domain is$[0,\infty )$
d)	Find$\frac{g}{h}$ 
Solution: $\frac{g}{h}=\frac{\left( \frac{1}{x} \right)}{3+\sqrt{x}}=\frac{1}{x}\div \left( 3+\sqrt{x} \right)=\frac{1}{x}\cdot \frac{1}{\left( 3+\sqrt{x} \right)}=\frac{1}{x\left( 3+\sqrt{x} \right)}=\frac{1}{3x+{{x}^{3/2}}}$  The domain is $(0,\infty )$
\end{example}

We can generalize to say:
\begin{definition}
If $f$ is a function with domain $A$ and $g$ is a function with domain $B$, then we can define operations on these functions as:
\begin{align*}
(f+g)(x)=f(x)+g(x) Domain A \cap B
(f-g)(x)=f(x)-g(x)  Domain A \cap B
(fg)(x)=f(x)g(x)  Domain A \cap B
\frac {f}{g} (x) = \frac {f(x)}{g(x)} Domain A \cap B except where g(x)=0
\end{align*}
\end{definition}

We also have what we call composition of functions.  Composition of functions can be thought of as putting one function inside another.  We denote this is $f(g(x))=(f\circ g) (x)$ 

\begin{example}
 Suppose we have:
$\begin{align}
  & f(x)={{x}^{2}}+5x+4 \\ 
 & g(x)=\frac{1}{x} \\ 
 & h(x)=3+\sqrt{x} \\ 
\end{align}$

Let’s first note the domain of each function.  The domain of f is$(-\infty ,\infty )$, domain of g is $(-\infty ,0)\cup (0,\infty )$, domain of h is $[0,\infty )$
 
a)	$f\circ g$ 
Solution: $f\circ g=f(g(x))$ We can think of this as everywhere we see an “x” in f, we’ll replace it with the function g. So we have: $f(g(x))={{\left( \frac{1}{x} \right)}^{2}}+5\left( \frac{1}{x} \right)+4$.  $(-\infty ,0)\cup (0,\infty )$
b)	$f\circ f$ 
Solution: $f\circ f={{\left( {{x}^{2}}+5x+4 \right)}^{2}}+5\left( {{x}^{2}}+5x+4 \right)+4$.  The domain is $(-\infty ,\infty )$
c)	$h\circ g$  
Solution: $h\circ g=h(g(x))=3+\sqrt{\frac{1}{x}}$ The domain is $(0,\infty )$
d)	$g\circ h$ 
Solution:$g\circ h=g(h(x))=\frac{1}{\sqrt{x}}$ The domain is $(0,\infty )$
\end{example}



\section{Inverses of Functions}


If a function maps every input to exactly one output, an inverse of that
function maps every ``output'' to exactly one ``input.'' While this
might sound somewhat esoteric, let's see if we can ground this in
some real-life contexts.

\begin{example}
Suppose that you are filling a swimming pool using a garden
hose---though because it rained last night, the pool starts with some
water in it. The volume of water in gallons after $t$ hours of
filling the pool is given by:
\[
v(t) = 700t + 200
\]
What does the inverse of this function tell you? What is the inverse
of this function?
\end{example}


\marginnote[.5in]{Here we abuse notation slightly, allowing $v$ and $t$
  to simultaneously be names of variables and functions.  This is
  standard practice in calculus classes.}
\begin{solution}
While $v(t)$ tells you how many gallons of water are in the pool after
a period of time, the inverse of $v(t)$ tells you how much time must
be spent to obtain a given volume. To compute the inverse function,
first set $v=v(t)$ and write
\[
v = 700t + 200.
\]
Now solve for $t$:
\[
t = v/700 - 2/7
\]
This is a function that maps volumes to times, and 
$t(v) = v/700-2/7$.

Let's look at how tables of values for functions and their inverses compare.  For this example, we have: 
%% insert picture 7
These tables reinforce the fact that our inputs become our outputs and viceversa.  We see that when the input for the function is $0$ the output is $200$ and when the input for the inverse is $200$, the output is $0$.  This isn't a coincidence.  This leads us to the face that the plot of the inverse is the plot of the function reflected over the line $y=x$ or $f(x)=x$.  See figure.
%% Insert picture 8
\end{solution}


Now let's consider a different example.

\begin{example}\label{E:example-ball-bridge}
Suppose you are standing on a bridge that is 60 meters above
sea-level. You toss a ball up into the air with an initial velocity of
30 meters per second.  If $t$ is the time (in seconds) after we toss
the ball, then the height at time $t$ is approximately $h(t) = -5 t^2
+30t+60$. What does the inverse of this function tell you? What is the inverse
of this function?
\end{example}


\begin{solution}
While $h(t)$ tells you how the height the ball is above sea-level at an
instant of time, the inverse of $h(t)$ tells you what time it is when
the ball is at a given height. There is only one problem: There is no
function that is the inverse of $h(t)$. Consider Figure~\ref{plot:fxn
  ball}, we can see that for some heights---namely 60 meters, there
are two times. This plot fails the horizontal line test to determine if the function is one-to-one.  Since the plot is not one-to-one, we cannot find an inverse function for $h(t)$.

While there is no inverse function for $h(t)$, we can find one if we
\textit{restrict the domain} of $h(t)$.  Looking at the plot of $h(t)$, we see tht it can pass the horizontal line test if we restrict the domain to half of the parabola.  The point at which we need to split the parabola is at the vertex. See figure.
%% Insert original figure 7 but with bolded point at vertex

To find the vertex, we need to write the equation in standard parabola form by completing the square.

\begin{align*}
$h &= -5t^2+30t+60
h &=-5[(t^2-6t+9)-12-9]
h &=-5[(t-3)^2-21]
h &=-5(t-3)^2+105$
\end{align*}

Therefore the vertex is $(3,105)$

Now we can find the inverse on the interval $[3,\infty)$.


\marginnote{Note: We have added and subtracted $9$ at the same time because that's equivalent to adding $0$.  The reason we have chosen 9 is by the completing the square process.  If we have a quadratic in the form $t^2-bt+c$, we need to find $(/frac {b}{2})^2$.  In this case, $(/frac {-6}{2})^2 &=9$.

From the standard form of the parabola: $h &= -5(t-3)^2+105$ we can find the inverse.
\begin{align*}
$h-105 &= -5(t-3)^2
\frac {h-105}{-5} &= (t-3)^2
\pm \sqrt{\frac {h-105}{-5}} &=t-3
3 \pm \sqrt{\frac {h-105}{-5}} &=t
3 \pm \sqrt{21-.2h} &=t $
\end{align*}

An alternate method to findin gthe inverse would be: 
Write
\begin{align*}
h &=  -5 t^2 +30t+60\\
0 &= -5 t^2 +30t+(60 - h)
\end{align*}
and solve for $t$ using the quadratic formula
\begin{align*}
t &= \frac{-30\pm \sqrt{30^2 -4(-5)(60-h)}}{2(-5)}\\
&= \frac{-30\pm \sqrt{30^2 +20(60-h)}}{-10}\\
&=3\mp \sqrt{3^2+ .2(60-h)}\\
&=3\mp \sqrt{9+ .2(60-h)}\\
&=3\mp \sqrt{21-.2h}
\end{align*}

But now we have two equations:
$t=3+ \sqrt{21-.2h}$
$t=3- \sqrt{21-.2h}$

We need to determine which equation should be used when $t>3$.
 Since $h(t)$ has its maximum value
  of $105$ when $t=3$, the largest $h$ could be is $105$. This means
  that $21-.2h \ge 0$ and so $\sqrt{21-.2h}$ is a real number. We know
  something else too, $t>3$. This means that the ``$\mp$'' that we see
  above must be a ``$+$.''  So the inverse of $h(t)$ on the interval
  $[3,\infty)$ is $t(h) = 3+ \sqrt{21-.2h}$. A similar argument will
    show that the inverse of $h(t)$ on the interval $(-\infty, 3]$ is
  $t(h) = 3- \sqrt{21-.2h}$.
\end{solution}
%%Insert picture 9

\begin{marginfigure}[-7in]
\begin{tikzpicture}
	\begin{axis}[
            clip=false, domain=0:7.58, axis lines =middle, xlabel=$t$,
            ylabel=$h$, every axis y label/.style={at=(current
              axis.above origin),anchor=south}, every axis x
            label/.style={at=(current axis.right of
              origin),anchor=west}, ] \addplot [very thick, penColor,
            smooth] {-5*x^2 +30*x+60};
        \end{axis}
\end{tikzpicture}
\caption{A plot of $h(t)=-5t^2+30t+60$. Here we can see that for each
  input (a value on the $t$-axis), there is exactly one output (a
  value on the $h$-axis). However, for each value on the $h$ axis,
  sometimes there are two values on the $t$-axis. Hence there is no
  function that is the inverse of $h(t)$.}
\label{plot:fxn ball}
\end{marginfigure}



\begin{marginfigure}[-2in]
\begin{tikzpicture}
	\begin{axis}[
            clip=false, domain=0:7.58, axis lines =middle, xlabel=$t$,
            ylabel=$h$, every axis y label/.style={at=(current
              axis.above origin),anchor=south}, every axis x
            label/.style={at=(current axis.right of
              origin),anchor=west}, ] 
          \addplot [very thick, penColor, smooth] {-5*x^2 +30*x+60};
          \addplot [very thick, penColor2] {80};
          \addplot [very thick, penColor4] plot coordinates {(5,0) (5,110)};
        \end{axis}
\end{tikzpicture}
\caption{A plot of $h(t)=-5t^2+30t+60$. While this plot passes the
  vertical line test, and hence represents $h$ as a function of $t$,
  it does not pass the horizontal line test, so the function is not one-to-one.}
\label{plot:fxn vert/horiz}
\end{marginfigure}



We see two different cases with our examples above. To clearly
describe the difference we need a definition.

\begin{definition}\index{one-to-one} 
A function is \textbf{one-to-one} if for every value in the range,
there is exactly one value in the domain.
\end{definition}

You may recall that a plot gives $y$ as a function of $x$ if every
vertical line crosses the plot at most once, this is commonly known as
the vertical line test. A function is one-to-one if every horizontal
line crosses the plot at most once, which is commonly known as the
horizontal line test, see Figure~\ref{plot:fxn vert/horiz}.  We can
only find an inverse to a function when it is one-to-one, otherwise we
must restrict the domain as we did in
Example~\ref{E:example-ball-bridge}.


Let's look at several examples.



\begin{example}
Consider the function
\[
f(x) = x^3.
\]
Does $f(x)$ have an inverse? If so what is it? If not, attempt to
restrict the domain of $f(x)$ and find an inverse on the restricted
domain.
\end{example}


\begin{solution}
In this case $f(x)$ is one-to-one and $f^{-1}(x) = \sqrt[3]{x}$. See Figure~\ref{plot:fxn and inverse x^3}.
\end{solution}

\begin{marginfigure}[-2in]
\begin{tikzpicture}
	\begin{axis}[
            domain=-2:2,
            xmin=-2, xmax=2,
            ymin=-2, ymax=2,
            axis lines =middle, xlabel=$x$, ylabel=$y$,
            every axis y label/.style={at=(current axis.above origin),anchor=south},
            every axis x label/.style={at=(current axis.right of origin),anchor=west},
          ]
	  \addplot [very thick, penColor, smooth] {x^3};
          \addplot [very thick, penColor2, smooth, samples=100,domain=.01:2] {x^(1/3)};
          \addplot [very thick, penColor2, smooth, samples=100,domain=-2:-.01] {-abs(x)^(1/3)};
          \addplot [very thick, penColor2] plot coordinates {(.01,.215) (-.01,-.215)};
          \addplot [dashed, textColor] {x};
          \node at (axis cs:-1.2,-.42) [penColor,anchor=west] {$f(x)$};
          \node at (axis cs:1.2,.9) [penColor2, anchor=west] {$f^{-1}(x)$};
        \end{axis}
\end{tikzpicture}
\caption{A plot of $f(x)=x^3$ and $f^{-1}(x) = \sqrt[3]{x}$. Note
  $f^{-1}(x)$ is the image of $f(x)$ after being flipped over the line
  $y=x$.}
\label{plot:fxn and inverse x^3}
\end{marginfigure}


\begin{example}
Consider the function
\[
f(x) = x^2.
\]
Does $f(x)$ have an inverse? If so what is it? If not, attempt to
restrict the domain of $f(x)$ and find an inverse on the restricted
domain.
\end{example}


\begin{solution}
In this case $f(x)$ is not one-to-one. However, it is one-to-one on
the interval $[0,\infty)$. Hence we can find an inverse of $f(x)=x^2$
  on this interval, and it is our familiar function $\sqrt{x}$.  See
  Figure~\ref{plot:fxn and inverse x^2}.
\end{solution}

\begin{marginfigure}[0in]
\begin{tikzpicture}
	\begin{axis}[
            domain=-2:2,
            xmin=-2, xmax=2,
            ymin=-2, ymax=2,
            axis lines =middle, xlabel=$x$, ylabel=$y$,
            every axis y label/.style={at=(current axis.above origin),anchor=south},
            every axis x label/.style={at=(current axis.right of origin),anchor=west},
          ]
	  \addplot [very thick, penColor, smooth] {x^2};
          \addplot [very thick, penColor2, smooth, samples=100,domain=0:2] {sqrt(x)};
          \addplot [dashed, textColor] {x};
          \node at (axis cs:-1.2,.55) [penColor,anchor=west] {$f(x)$};
          \node at (axis cs:1.4,1) [penColor2, anchor=west] {$f^{-1}(x)$};
        \end{axis}
\end{tikzpicture}
\caption{A plot of $f(x)=x^2$ and $f^{-1}(x) = \sqrt{x}$. While
  $f(x)=x^2$ is not one-to-one on $\R$, it is one-to-one on
  $[0,\infty)$.}
\label{plot:fxn and inverse x^2}
\end{marginfigure}


\subsection{A Word on Notation}

Given a function $f(x)$, we have a way of writing an inverse of $f(x)$, assuming it exists
\[
f^{-1}(x) = \text{the inverse of $f(x)$, if it exists.}
\]
On the other hand,
\[
f(x)^{-1} = \frac{1}{f(x)}.
\]
\end{document}
\begin{warning}
It is not usually the case that 
\[
f^{-1}(x) = f(x)^{-1}.
\]
\end{warning}

This confusing notation is often exacerbated by the fact that 
\[
\sin^2(x) = (\sin(x))^2=sin(x)*sin(x)\qquad \text{but} \qquad \sin^{-1}(x)
\ne(\sin(x))^{-1}.
\]

$sin^-1(x)=arcsin(x) but (sin(x))^-1 = \frac {1}{sin(x)}
In the case of trigonometric functions, this confusion can be avoided
by using the notation $\arcsin$ and so on for other trigonometric
functions.





