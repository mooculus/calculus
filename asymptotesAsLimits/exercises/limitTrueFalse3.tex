\documentclass{ximera}

\newcommand{\RR}{\mathbb R}
\renewcommand{\d}{\,d}
\newcommand{\dd}[2][]{\frac{d #1}{d #2}}
\renewcommand{\l}{\ell}
\newcommand{\ddx}{\frac{d}{dx}}
\newcommand{\dfn}{\textbf}
\newcommand{\eval}[1]{\bigg[ #1 \bigg]}


\outcome{Understand the relationship between limits and vertical asymptotes.}

\author{Nela Lakos \and Kyle Parsons \and Bobby Ramsey}

\begin{document}


\begin{exercise}
	Select True if the statement is \textbf{always} true; otherwise, select False.

	Let $f$ be a one-to-one function and $f^{-1}$ its inverse.  If the point $(2,5)$ lies on the graph of $f$, then the point $(5,2)$ lies on the graph of $f^{-1}$.
	\begin{multipleChoice}
		\choice[correct]{True}
		\choice{False}
	\end{multipleChoice}
	\begin{feedback}
		The graph of $f$ contains all points of the form $\left(x,f(x)\right)$ whereas the graph of $f^{-1}$ contains all points of the form $\left(f(x),x\right)$.
	\end{feedback}

	\begin{exercise}
		If $f$ and $g$ are two functions defined on $\left(-1,1\right)$, and if $g(0) = 0$, then it must be true that $\displaystyle \lim_{x\to0}\left[f(x)g(x)\right]=0$.

		\begin{multipleChoice}
			\choice{True}
			\choice[correct]{False}
		\end{multipleChoice}
		\begin{feedback}
			If $g(x) = x$ and $f(x)=\begin{cases} \frac{1}{x} \,; & \text{ for } x\neq0 \\ 0 \,; & \text{ for } x=0 \end{cases}$, then \[ \lim_{x\to0}\left[f(x)g(x)\right] = 1. \]
		\end{feedback}

		\begin{exercise}
    			If $f$ is continuous on $\left(-1,1\right)$ and if $f(0) = 10$ and $\displaystyle \lim_{x\to0}g(x) = 2$, then $\displaystyle \lim_{x\to0}\frac{f(x)}{g(x)} = 5$.    
	    		\begin{multipleChoice}
    				\choice[correct]{True}
    				\choice{False}
    			\end{multipleChoice}
    
    			\begin{feedback}
    				Since $f$ is continuous at 0 we know that $\displaystyle \lim_{x\to0}f(x) = f(0) = 10$ so we can use the quotient limit law to evaluate $\displaystyle \lim_{x\to0}\frac{f(x)}{g(x)}$.
    			\end{feedback}
    
   	 		\begin{exercise}
    				Let $f$ be a positive function with vertical asymptote $x=5$, then $\displaystyle \lim_{x\to5}f(x) = \infty$.
    				\begin{multipleChoice}
    					\choice{True}
    					\choice[correct]{False}
    				\end{multipleChoice}
    
    				\begin{feedback}
    					It could be that $\displaystyle \lim_{x\to5^-}f(x) = \infty$ so that $f$ has a vertical asymptote at 5, but $\lim_{x\to5^+}f(x)=10$.  In this case $\lim_{x\to5}f(x)$ does not exist.
    				\end{feedback}
    			\end{exercise}
		\end{exercise}
	\end{exercise}
\end{exercise}
\end{document}
