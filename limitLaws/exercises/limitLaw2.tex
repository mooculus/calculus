\documentclass{ximera}
\newcommand{\RR}{\mathbb R}
\renewcommand{\d}{\,d}
\newcommand{\dd}[2][]{\frac{d #1}{d #2}}
\renewcommand{\l}{\ell}
\newcommand{\ddx}{\frac{d}{dx}}
\newcommand{\dfn}{\textbf}
\newcommand{\eval}[1]{\bigg[ #1 \bigg]}

\outcome{Calculate limits using the limit laws.}
\outcome{Calculate limits by replacing a function with a continuous
  function.}
\tag{limit}
\author{Steven Gubkin}
\license{Creative Commons 3.0 By-NC}
\begin{document}

\begin{exercise}

	A student's attempt to evaluate the limit $\lim_{x \to 2} \left( 3x^2-1 \right)$ using limit laws is recorded below.  From one line to the next, only one limit law should be utilized.  Which of the following options best describes the student's work?
	
	\begin{align*}
		\lim_{x \to 2} \left( 3x^2-1 \right) &= \lim_{x \to 2} \left( 3x^2\right) + \lim_{x \to 2} \left( -1 \right)\\
		&= 3 \lim_{x \to 2} \left( x^2\right) + \lim_{x \to 2} \left( -1 \right)\\
		&= 3 \left( \lim_{x \to 2} \left( x\right) \right)^2 + \lim_{x \to 2} \left( -1 \right)\\
		&=3 \left( 2 \right)^2 + \left( -1 \right)\\
		&= 11
	\end{align*}
	
	\begin{multipleChoice}
		\choice[correct]{The work is fully correct.}
		\choice{The answer is correct, but the student skipped some steps, or made a mistake along the way}
		\choice{The answer is incorrect}
	\end{multipleChoice}
	
	\begin{feedback}
	  This work is completely correct.  You should be able to
          reproduce such a sequence of steps, even though you know
          that for polynomial functions it is sufficient to just
          evaluate the function at the limit point to get the limit
          (i.e.\ polynomial functions are ``continuous'').
	\end{feedback}
	
\end{exercise}
\end{document}
