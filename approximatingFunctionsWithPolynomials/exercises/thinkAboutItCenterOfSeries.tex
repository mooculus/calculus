\documentclass{ximera}

\newcommand{\RR}{\mathbb R}
\renewcommand{\d}{\,d}
\newcommand{\dd}[2][]{\frac{d #1}{d #2}}
\renewcommand{\l}{\ell}
\newcommand{\ddx}{\frac{d}{dx}}
\newcommand{\dfn}{\textbf}
\newcommand{\eval}[1]{\bigg[ #1 \bigg]}


\author{Jim Talamo}
\license{Creative Commons 3.0 By-bC}


\outcome{Understand why Taylor Polynomials are centered at $x-c$}


\begin{document}
\begin{exercise}
Recall that if a function $f(x)$ has at least $n$ derivatives at a point $x=c$, then the $n$-th order Taylor Polynomial is found by requiring that the function and its first $n$ derivatives are respectively equal to the value of the Taylor Polynomial and its first $n$ derivatives at $x=c$:

\begin{align*}
f(c) &= p_n(c) \\
f'(c) &= p_n'(c) \\
f''(c) &= p_n''(c) \\ 
& \vdots \\
f^{(n)}(c) &= p_n^{(n)}(c)
\end{align*}
or, more succinctly:

\[ f^{(k)}(c) = p_n^{(k)}(c) \textrm{ for all } k=0,1,2, \ldots , n \]

where $f^{(k)}(x)$ is notation for $\frac{d^k}{dx^k}\left[f(x)\right]$.

This lead to the requirement that:
\begin{multipleChoice}
\choice{$a_k = f^{(k)}(c) $}
\choice[correct]{$a_k = \frac{f^{(k)}(c)}{k!}$}
\choice{$f^{(k)}(c) = \frac{a_k}{k!}$}
\end{multipleChoice}

We thus take as the definition for the $n$-th order Taylor Polynomial, $p_n(x)$ centered at $x=c$ for $f(x)$:

\[
p_n(x) = a_0 +a_1(x-c)+a_2(x-c)^2+\ldots+a_n(x-c)^n
\]
where $a_k = \frac{f^{(k)}(c)}{k!}$ for $k=0,1,2, \ldots , n$.

One question that may arise is why we choose to write the polynomial in powers of $x-c$ rather than just $x$.  To get a feel for why the choice of using powers of $x-c$ is preferred, work through the exercise below.

Consider the function $f(x) = \sqrt{2x-1}$. We will find the second degree Taylor Polynomial centered at $x=1$.

\begin{exercise}
\begin{exercise}
Look for a polynomial of the form
\[
p_2(x) = a_0 +a_1(x-1)+a_2(x-1)^2
\]
The coefficients will be found by using the requirements that:

\begin{align*}
f(1) &= p_2(1) \\
f'(1) &= p_2'(1) \\
f''(1) &= p_2''(1) 
\end{align*} 

Complete the tables below:

\begin{tabular}{|c||c|c|}
\hline
$k$ \quad & \quad \quad $f^{(k)}(x)$  \quad \quad & \quad \quad $f^{(k)}(1)$ \quad \quad \\
\hline 
$0$ \quad & \quad \quad $(2x-1)^{1/2}$  \quad \quad & \quad \quad $\answer{1}$ \quad \quad  \\
\hline
$1$ \quad & \quad \quad $\answer{(2x-1)^{-1/2}}$ \quad \quad & \quad \quad $\answer{1}$ \quad \quad \\
\hline
$2$ \quad & \quad \quad $\answer{-(2x-1)^{-3/2}}$ \quad \quad & \quad \quad $\answer{-1}$ \quad \quad \\
\hline 
\end{tabular}

%%%%%TaylorPoly%%%%%%
\begin{tabular}{|c||c|c|}
\hline
$k$ \quad & \quad \quad $p_2^{(k)}(x)$  \quad \quad & \quad \quad $p_2^{(k)}(1)$ \quad \quad \\
\hline 
$0$ \quad & \quad \quad $a_0+a_1(x-1)+a_2(x-1)^2$  \quad \quad & \quad \quad $a_0$ \quad \quad  \\
\hline
$1$ \quad & \quad \quad $a_1+\answer{2}a_2(x-1)$ \quad \quad & \quad \quad $a_1+\answer{0}$ \quad \quad \\
\hline
$2$ \quad & \quad \quad $\answer{2}a_2$ \quad \quad & \quad \quad $2a_2$ \quad \quad \\
\hline
\end{tabular}

\begin{exercise}
Now, we can use the requirements to find the unknown coefficients:

\begin{itemize}
\item The requirement $f(1) = p_2(1)$ gives $a_0 = \answer{1}$. 
\item The requirement $f'(1) = p_2'(1)$ gives $a_1 = \answer{1}$. 
\item The requirement $f''(1) = p_2''(1)$ gives $a_2 = \answer{-\frac{1}{2}}$. 
\end{itemize}

Note that each condition easily allows us to isolate a singe unknown constant!  Hence, the second degree Taylor polynomial for $f(x) =\sqrt{2x-1}$ is:

\begin{align*}
p_2(x) &= \answer{1}+\answer{1}(x-1)+\answer{-\frac{1}{2}}(x-1)^2\\
\end{align*}
\end{exercise}
\end{exercise}

%%%%%%inefficeint way%%%%%%
\begin{exercise}
Instead, we can look for a polynomial of the form:
\[
P_2(x) = b_0 +b_1x+b_2x^2
\]
Note that we now have a polynomial in powers of $x$ instead of $x-1$.  The coefficients will still be found by using the requirements that:

\begin{align*}
f(1) &= P_2(1) \\
f'(1) &= P_2'(1) \\
f''(1) &= P_2''(1) 
\end{align*} 

Complete the tables below:

\begin{tabular}{|c||c|c|}
\hline
$k$ \quad & \quad \quad $f^{(k)}(x)$  \quad \quad & \quad \quad $f^{(k)}(1)$ \quad \quad \\
\hline 
$0$ \quad & \quad \quad $(2x-1)^{1/2}$  \quad \quad & \quad \quad $\answer{1}$ \quad \quad  \\
\hline
$1$ \quad & \quad \quad $\answer{(2x-1)^{-1/2}}$ \quad \quad & \quad \quad $\answer{1}$ \quad \quad \\
\hline
$2$ \quad & \quad \quad $\answer{-(2x-1)^{-3/2}}$ \quad \quad & \quad \quad $\answer{-1}$ \quad \quad \\
\hline 
\end{tabular}

%%%%%TaylorPoly%%%%%%
\begin{tabular}{|c||c|c|}
\hline
$k$ \quad & \quad \quad $P_2^{(k)}(x)$  \quad \quad & \quad \quad $P_2^{(k)}(1)$ \quad \quad \\
\hline 
$0$ \quad & \quad \quad $b_0+b_1x+b_2x^2$  \quad \quad & \quad \quad $\answer{1}b_0+\answer{1}b_1+\answer{1}b_2$ \quad \quad  \\
\hline
$1$ \quad & \quad \quad $b_1+\answer{2}b_2x$ \quad \quad & \quad \quad  $\answer{0}b_0+\answer{1}b_1+\answer{2}b_2$ \quad \quad \\
\hline
$2$ \quad & \quad \quad $\answer{2}b_2$ \quad \quad & \quad \quad  $\answer{0}b_0+\answer{0}b_1+\answer{2}b_2$ \quad \quad \\
\hline
\end{tabular}

\begin{exercise}
Now, we can use the requirements to find the unknown coefficients:

\begin{itemize}
\item The requirement $f(1) = P_2(1)$ gives $\answer{1}=b_0+b_1+b_2$. 
\item The requirement $f'(1) = P_2'(1)$ gives $\answer{1} = b_1 +\answer{2}b_2$. 
\item The requirement $f''(1) = P_2''(1)$ gives $\answer{-1} = \answer{2}b_2$. 
\end{itemize}

We now have a system of equations to solve!  It is not difficult to do this since the last condition gives: $b_2 = \answer{-\frac{1}{2}}$.

Now, substitute $b_2$ into the second to last condition to find $b_1 = \answer{2}$. 

Finally, substitute both $b_1$ and $b_2$ into the first condition to find $b_0 = \answer{-\frac{1}{2}}$.

Hence, this second degree Taylor polynomial for $f(x) =\sqrt{2x-1}$ is:

\begin{align*}
P_2(x) &= \answer{-\frac{1}{2}}+\answer{2}x+\answer{-\frac{1}{2}}x^2\\
\end{align*}


Note that the coefficients for both polynomials are different.  However, did we actually produce two different polynomials?  We can explore this by expand the first Taylor polynomial $p_3(x)$:

\begin{align*}
p_2(x) &=1+(x-1)-\frac{1}{2}(x-1)^2\\
&=1+(x-1)-\frac{1}{2} \left(\answer{x^2-2x+1}\right)\\
&= \answer{-\frac{1}{2}}+\answer{2}x+\answer{-\frac{1}{2}}x^2
\end{align*}

Are the polynomials $p_2(x)$ and $P_2(x)$ the same?

\begin{multipleChoice}
\choice[correct]{Yes}
\choice{No}
\end{multipleChoice}

\begin{feedback}[correct]
\begin{remark}
In general, these polynomials will always be the same.  We thus make the choice to expand in powers of $x-c$ rather than $x$ because expanding in powers of $x-c$ allows for a computationally efficient way to find the coefficients.  

Furthermore, we will shortly be looking for an ``infinite degree'' Taylor polynomial, and we will not be able to solve the infinite system of equations that arises if we require the polynomial to be written in powers of $x$ rather than $x-c$.
\end{remark}

\end{feedback}
\end{exercise}

\end{exercise}

\end{exercise}
\end{exercise}
\end{document}
