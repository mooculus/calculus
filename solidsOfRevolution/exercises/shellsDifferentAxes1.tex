\documentclass{ximera}
\newcommand{\RR}{\mathbb R}
\renewcommand{\d}{\,d}
\newcommand{\dd}[2][]{\frac{d #1}{d #2}}
\renewcommand{\l}{\ell}
\newcommand{\ddx}{\frac{d}{dx}}
\newcommand{\dfn}{\textbf}
\newcommand{\eval}[1]{\bigg[ #1 \bigg]}

\author{Jim Talamo }
\license{Creative Commons 3.0 By-NC}
\outcome{Set up a volume integral using the Shell Method}
\begin{document}
\begin{exercise}
  Let $R$ be the region in the $xy$-plane bounded between $y=x^2-1$
  and $y=1$.  This exercise gives practice using the Shell Method to
  set up an integral that gives the volume of the solid of revolution
  obtained when $R$ is rotated about several different axes.

  
  When using the Shell Method, the slices must be:
  \begin{multipleChoice}
    \choice[correct]{parallel to the axis of rotation.}
    \choice{perpendicular to the axis of rotation.}
  \end{multipleChoice}
  
  \begin{exercise}
    Suppose that $R$ is revolved about the $x$-axis.  To use the Shell Method:
    \begin{multipleChoice}
      \choice{the slices must be vertical.  We should integrate with respect to $x$.}
      \choice{the slices must be vertical.  We should integrate with respect to $y$.}
      \choice{the slices must be horizontal.  We should integrate with respect to $x$.}
      \choice[correct]{the slices must be horizontal.  We should integrate with respect to $y$.}
    \end{multipleChoice}
    \begin{exercise}
      
      We thus have a helpful version of the picture of the region $R$ below:

  \begin{image}
            \begin{tikzpicture}
            	\begin{axis}[
            		domain=-10:10, ymax=2.4,xmax=1.9, ymin=-1.4, xmin=-1.9,
            		axis lines =center, xlabel=$x$, ylabel=$y$,
            		every axis y label/.style={at=(current axis.above origin),anchor=south},
            		every axis x label/.style={at=(current axis.right of origin),anchor=west},
            		axis on top,
            		]
                      
            	\addplot [draw=penColor,very thick,smooth,samples=100,domain=0:2] {x^2-1};
		\addplot [draw=black!50!green,very thick,smooth,samples=100,domain=-2:0] {x^2-1};
	        \addplot [draw=penColor2,very thick,smooth,domain=-2:2] {1};
		\addplot [draw=penColor,very thick,smooth] coordinates {(2,-10)(2,10)};
		\addplot [draw=penColor5,very thick,dashed] coordinates {(-10,2)(10,2)};
		                       
            	\addplot [name path=A,domain=-1.4:1.4,draw=none,samples=200] {x^2-1};   
            	\addplot [name path=B,domain=-1.4:1.4,draw=none] {1};
            	\addplot [fillp] fill between[of=A and B];
	                
		\node at (axis cs:1.1,1.2) [penColor2] {$y=1$};
			\node at (axis cs:1.2,2.2) [penColor5] {$y=2$};
		\node at (axis cs:1.2,-.8) [penColor] {$x = +\sqrt{y+1}$};  
		\node at (axis cs:-.8,1.4) [black!50!green] {$x = -\sqrt{y+1}$};  
		
		\addplot [draw=penColor, fill = gray!50] plot coordinates {(-1.31,.72) (1.31,.72) (1.31,.82) (-1.31,.82) (-1.31,.72)};
          
          %Draw rho and h
          \addplot [draw=black!50!blue,very thick] coordinates {(.3,2)(.3,.82)};
          \node at (axis cs:.45,1.4) [black!50!blue] {$\rho$};
          
	  \draw[decoration={brace,raise=.1cm,mirror},decorate,thin] (axis cs:-1.31,.72)--(axis cs:1.31,.72);

	 \node at (axis cs:.1,.4)  [black]  {$h$};
                      
                  	\end{axis}
            \end{tikzpicture}
  \end{image}
            
 We see from the picture that $\rho$ is a:
 \begin{multipleChoice}
 \choice[correct]{vertical distance}
 \choice{horizontal distance}
 \end{multipleChoice}           
            
\begin{exercise}
Since $\rho$ is the distance from the axis of rotation to the slice, and this is a vertical distance, we find $\rho = y_{top}-y_{bot}$.
\begin{multipleChoice}
 \choice[correct]{$y_{top} = 2$}
 \choice{$y_{top} = y$}
 \choice{$y_{top} = 1$}
  \choice{$y_{top} = x^2-1$}
\end{multipleChoice}       

\begin{multipleChoice}
 \choice{$y_{bot} = 2$}
 \choice[correct]{$y_{bot} = y$}
 \choice{$y_{bot} = 1$}
  \choice{$y_{bot} = x^2-1$}
\end{multipleChoice}   

\begin{hint}
Remember that the process of find the volume requires us too express the volume of each shell in terms of the $y$-value where the shell is located.  An arbitrary shell is located at $y$!
\end{hint}

So, $\rho= \answer{2-y}$.
 \end{exercise}
 
  We see from the picture that $h$ is a:
 \begin{multipleChoice}
 \choice{vertical distance}
 \choice[correct]{horizontal distance}
 \end{multipleChoice}           
 
 \begin{exercise}
Since $h$ is the height of the slice, and this is a vertical distance, we find $h = x_{right}-x_{left}$.
\begin{multipleChoice}
 \choice[correct]{$x_{right} = \sqrt{y+1}$}
 \choice{$x_{right} = -\sqrt{y+1}$}
\end{multipleChoice}       

\begin{multipleChoice}
 \choice{$x_{left} = \sqrt{y+1}$}
 \choice[correct]{$x_{left} = -\sqrt{y+1}$}
\end{multipleChoice}   

So, $h= \answer{2\sqrt{y+1}}$.
 \end{exercise}


\begin{exercise}
Using the formula $V = \int_{y=c}^{y=d} 2\pi \rho h \d y$, an integral that gives the volume of this solid of revolution is:

\[
V = \int_{y=\answer{-1}}^{y=\answer{1}} \answer{4 \pi (2-y)\sqrt{y+1}} \d y
\]

Evaluating this, we find the volume is $\answer{ \frac{48 \sqrt{2}}{5} \pi }$ cubic units.

\begin{hint}
Make the substitution $u=y+1$ and do some algebra!
\end{hint}

\end{exercise}    

\end{exercise}



\end{exercise}
\end{exercise}




 
%%%%%
\begin{exercise}
Suppose that $R$ is revolved about the line $x=3$.

\begin{multipleChoice}
  \choice[correct]{the slices must be vertical.  We should integrate with respect to $x$.}
  \choice{the slices must be vertical.  We should integrate with respect to $y$.}
  \choice{the slices must be horizontal.  We should integrate with respect to $x$.}
  \choice{the slices must be horizontal.  We should integrate with respect to $y$.}
\end{multipleChoice}

\begin{exercise}

We thus have a helpful version of the picture of the region $R$ below:

   \begin{image}
            \begin{tikzpicture}
            	\begin{axis}[
            		domain=-10:10, ymax=2.4,xmax=3.9, ymin=-1.9, xmin=-1.9,
            		axis lines =center, xlabel=$x$, ylabel=$y$,
            		every axis y label/.style={at=(current axis.above origin),anchor=south},
            		every axis x label/.style={at=(current axis.right of origin),anchor=west},
            		axis on top,
            		]
                      
            	\addplot [draw=penColor,very thick,smooth,samples=100,domain=-2:2] {x^2-1};
	        \addplot [draw=penColor2,very thick,smooth,domain=-2:10] {1};
		\addplot [draw=penColor5,very thick,dashed] coordinates {(3,-10)(3,10)};
		                       
            	\addplot [name path=A,domain=-1.4:1.4,draw=none,samples=200] {x^2-1};   
            	\addplot [name path=B,domain=-1.4:1.4,draw=none] {1};
            	\addplot [fillp] fill between[of=A and B];
	                
		\node at (axis cs:.5,1.2) [penColor2] {$y=1$};
			\node at (axis cs:2.5,1.8) [penColor5] {$x=3$};
		\node at (axis cs:1.4,-.8) [penColor] {$y=x^2-1$};  
	
		
		\addplot [draw=penColor, fill = gray!50] plot coordinates {(.25,-.84) (.4,-.84) (.4,1) (.25,1) (.25,-.84)};
          
          %Draw rho and h
          \addplot [draw=black!50!blue,very thick] coordinates {(.4,.8)(3,.8)};
          \node at (axis cs:1.9,.6) [black!50!blue] {$\rho$};
          
	  \draw[decoration={brace,raise=.1cm},decorate,thin] (axis cs:.25,-.84)--(axis cs:.25,1);

	 \node at (axis cs:-.2,.3)  [black]  {$h$};
                      
                  	\end{axis}
            \end{tikzpicture}
  \end{image}
            
 We see from the picture that $h$ is a:
 \begin{multipleChoice}
 \choice[correct]{vertical distance}
 \choice{horizontal distance}
 \end{multipleChoice}           
            
\begin{exercise}
Since $h$ is the distance from the axis of rotation to the outer curve, and this is a vertical distance, we find $h = y_{top}-y_{bot}$.
\begin{multipleChoice}
 \choice[correct]{$y_{top} = 1$}
 \choice{$y_{top} = x^2-1$}
 \choice{$y_{top} = y$}
 \choice{$y_{top} = x$}
\end{multipleChoice}       

\begin{multipleChoice}
 \choice{$y_{bot} = 1$}
 \choice[correct]{$y_{bot} = x^2-1$}
 \choice{$y_{bot} = y$}
 \choice{$y_{bot} = x$}
\end{multipleChoice}    

So, $h= \answer{2-x^2}$.
 \end{exercise}
 
  We see from the picture that $\rho$ is a:
 \begin{multipleChoice}
 \choice{vertical distance}
 \choice[correct]{horizontal distance}
 \end{multipleChoice}           
            
\begin{exercise}
Since $\rho$ is the distance from the axis of rotation to the outer curve, and this is a horizontal distance, we find $\rho = x_{right}-x_{left}$.
\begin{multipleChoice}
 \choice{$x_{right} = x$}
 \choice[correct]{$x_{right} = 3$}
 \choice{$x_{right} = x^2-1$}
\end{multipleChoice}       

\begin{multipleChoice}
 \choice[correct]{$x_{left} = x$}
 \choice{$x_{left} = 3$}
 \choice{$x_{left} = x^2-1$}
\end{multipleChoice}    

So, $\rho= \answer{3-x}$.




\begin{exercise}
Using the formula $V = \int_{x=a}^{x=b} 2 \pi \rho h \d x$, an integral that gives the volume of this solid of revolution is:

\[
V = \int_{x=\answer{-\sqrt{2}}}^{x=\answer{\sqrt{2}}} 2\pi \answer{(3-x)(2-x^2)} \d x
\]

\end{exercise}    
\end{exercise}
\end{exercise}
\end{exercise}
\end{document}
