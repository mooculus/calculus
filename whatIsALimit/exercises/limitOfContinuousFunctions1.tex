\documentclass{ximera}

\newcommand{\RR}{\mathbb R}
\renewcommand{\d}{\,d}
\newcommand{\dd}[2][]{\frac{d #1}{d #2}}
\renewcommand{\l}{\ell}
\newcommand{\ddx}{\frac{d}{dx}}
\newcommand{\dfn}{\textbf}
\newcommand{\eval}[1]{\bigg[ #1 \bigg]}



\outcome{Understand the concept of a limit.}
\outcome{Calculate limits of continuous functions.}

\author{Nela Lakos}

\begin{document}
\begin{exercise}


For the given function $f$, evaluate the limit and justify your answer.\\



(A) $f(x)=x$
 \[
\lim_{x\to 7}f(x) = \answer{7}
\] 
Justification:\\ $f$ is continuous at $a=7$, which implies that
$\lim_{x\to 7}f(x)=f(\answer{7})=\answer{7}$.

\noindent\rule[0.5ex]{\linewidth}{.2pt}

(B) $f(x)=\sin{x}$
 \[
\lim_{x\to \frac{\pi}{2}}f(x) = \answer{1}
\] 
Justification:\\ $f$ is continuous at $a=\frac{\pi}{2}$, which implies that\\[1em]
$\lim_{x\to \frac{\pi}{2}}f(x)=f(\answer{\frac{\pi}{2}})=\sin{\Bigl(\answer{\frac{\pi}{2}}\Bigr)}=\answer{1}$.

\noindent\rule[0.5ex]{\linewidth}{.2pt}

(C) $f(x)=e^{x}$
 \[
\lim_{x\to 0}f(x) = \answer{1}
\] 
Justification:\\ $f$ is continuous at $a=0$, which implies that\\[1em]
$\lim_{x\to 0}f(x)=f(\answer{0})=e^{\answer{0}}=\answer{1}$.

\noindent\rule[0.5ex]{\linewidth}{.2pt}

(D) $f(x)=\ln{x}$
 \[
\lim_{x\to e^{4}}f(x) = \answer{4}
\] 
Justification:\\ $f$ is continuous at $a=e^{4}$, which implies that\\[1em]
$\lim_{x\to e^{4}}f(x)=f\Bigl(\answer{e^{4}}\Bigr)=\ln{\Bigl(\answer{e^{4}}\Bigr)}=\answer{4}$.

\noindent\rule[0.5ex]{\linewidth}{.2pt}

(E) $f(x)=\cos{x}$
 \[
\lim_{x\to \frac{2\pi}{3}}f(x) = \answer{-\frac{1}{2}}
\] 
Justification:\\ $f$ is continuous at $a= \frac{2\pi}{3}$, which implies that\\[1em]
$\lim_{x\to \frac{2\pi}{3}}f(x)=f\Bigl(\answer{ \frac{2\pi}{3}}\Bigr)=\cos{\Bigl(\answer{ \frac{2\pi}{3}}\Bigr)}=\answer{-\frac{1}{2}}$.

\noindent\rule[0.5ex]{\linewidth}{.2pt}

(F) $f(x)=x^3$
 \[
\lim_{x\to -2}f(x) = \answer{-8}
\] 
Justification:\\ $f$ is continuous at $a=-2$, which implies that\\[1em]
$\lim_{x\to -2}f(x)=f(\answer{ -2})=(\answer{-2 })^3=\answer{-8}$.

\noindent\rule[0.5ex]{\linewidth}{.2pt}
\end{exercise}
\end{document}
