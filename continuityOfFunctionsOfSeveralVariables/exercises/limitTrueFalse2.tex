\documentclass{ximera}

\author{Jim Talamo}

\newcommand{\RR}{\mathbb R}
\renewcommand{\d}{\,d}
\newcommand{\dd}[2][]{\frac{d #1}{d #2}}
\renewcommand{\l}{\ell}
\newcommand{\ddx}{\frac{d}{dx}}
\newcommand{\dfn}{\textbf}
\newcommand{\eval}[1]{\bigg[ #1 \bigg]}


\outcome{Evaluate limits of functions of several variables.}

\begin{document}
\begin{exercise}

Select all of the following statements below that are true.

\begin{selectAll}
\choice{If $\Lim{x}{1} F(x,1) = 5$ and $\Lim{y}{2} F(2,y)=5$, then $\Lim{\vector{x,y}}{\vector{1,2}} F(x,y) =5$.}
\choice[correct]{If $F(x,y)$ is continuous at $(3,5)$, then $\Lim{\vector{x,y}}{\vector{3,5}} F(x,y)$  exists.}
\choice{If $\Lim{\vector{x,y}}{\vector{3,5}} F(x,y)$ exists, then $F(x,y)$ is continuous at $(3,5)$.}
\choice{If $F(x,y)$ is not continuous at $(3,5)$, then $\Lim{\vector{x,y}}{\vector{3,5}} F(x,y)$ does not exist.}
\choice{If $F(x,y)$ approaches the same value along \emph{every} straight line path through $(a,b)$, then $\Lim{\vector{x,y}}{\vector{a,b}} F(x,y)$ must exist.}
\end{selectAll}

\begin{feedback}[correct] 

Here are some explanations.
\begin{itemize}

\item For the first choice, note that the given information tells you the values that the function approaches along the lines $x=1$ and $y=2$ is the same.  This is not enough to conclude that the limit exists.  As a counterexample, consider the function below.

\[
F(x,y) = \begin{cases} 5, & x=1 \\ 5, & y=2 \\ 0, & \textrm{otherwise } \end{cases}
\]

\item For the second option, the definition of continuity requires not only that  $\Lim{\vector{x,y}}{\vector{3,5}} F(x,y)$ exists, but that $F(3,5)$ is defined and that $\Lim{\vector{x,y}}{\vector{3,5}} F(x,y)=F(3,5).$ 

\item To be continuous at $(3,5)$, we need that $\Lim{\vector{x,y}}{\vector{3,5}} F(x,y)$ exists \emph{AND} $\Lim{\vector{x,y}}{\vector{3,5}} F(x,y) = F(3,5)$.

\item The issue for continuity may not be the nonexistence of the limit; it may be because both $\Lim{\vector{x,y}}{\vector{3,5}} F(x,y)$ and $F(3,5)$ exist, but do not line up.  For instance, consider the function below.

\[
F(x,y) = \begin{cases} 0, & (x,y) \neq (3,5) \\ 1, & (x,y) =(3,5) \end{cases}
\]

\item We need the function to approach the same value along \emph{every} path through $(a,b)$, not just all straight line paths.  A counterexample is the function in the text, or the function below, which approaches $0$ along all straight line paths through $(x,y) = (0,0)$, but approaches $\frac{1}{2}$ as $(x,y) \to (0,0)$ along $y=x^2$.

\[
F(x,y) = \begin{cases} \frac{x^2y}{x^4+y^2}, & (x,y) \neq (0,0) \\ 0, & (x,y) =(0,0) \end{cases}
\]

\end{itemize}
\end{feedback}
 \end{exercise}
\end{document}
