\documentclass{ximera}

\newcommand{\RR}{\mathbb R}
\renewcommand{\d}{\,d}
\newcommand{\dd}[2][]{\frac{d #1}{d #2}}
\renewcommand{\l}{\ell}
\newcommand{\ddx}{\frac{d}{dx}}
\newcommand{\dfn}{\textbf}
\newcommand{\eval}[1]{\bigg[ #1 \bigg]}



\outcome{Find derivatives of inverse functions in general.}
\outcome{Recall the meaning and properties of inverse trigonometric functions.}
\outcome{Derive the derivatives of inverse trigonometric functions.}
\outcome{Understand how the derivative of an inverse function relates to the original derivative.}
\outcome{Take derivatives which involve inverse trigonometric functions.}

\title[Dig-In:]{Derivatives of inverse trigonometric functions}

\begin{document}
\begin{abstract}
  We derive the derivatives of inverse trigonometric functions using implicit
  differentiation.
\end{abstract}
\maketitle

Now we will derive the derivative of arcsine, arctangent, and
arcsecant.

%% Since this is an inverse
%% function, we can find its derivative by using implicit
%% differentiation and the Inverse Function Theorem.


\begin{theorem}[The derivative of arcsine]\index{derivative!of arcsine}
\[
\ddx \arcsin(x) = \frac{1}{\sqrt{1-x^2}}.
\]
\begin{explanation} 
  %% To start, note that the Inverse Function Theorem assures us that this
  %% derivative actually exists.
  Recall
  \[
  \arcsin(x) = \theta\qquad \text{exactly when} \qquad\sin(\theta) = x\qquad\text{and}\qquad \answer[given]{\frac{-\pi}{2}}\le \theta\le \answer[given]{\frac{\pi}{2}}.
  \]
  Implicitly differentiating with respect $x$ we see
  \begin{align*}
    \sin(\theta) &= x\\
    \ddx \sin(\theta) &= \ddx x &\text{Differentiate both sides.}\\
    \cos(\theta) \cdot \theta' &= 1 &\text{Implicit differentiation.}\\
    \theta' &= \frac{1}{\cos(\theta)} &\text{Solve for $\theta'$.}
  \end{align*}
  While $\theta' = \frac{1}{\cos(\theta)}$, we need our answer written
  in terms of $x$. Since we are assuming that
  \[
  \sin(\theta) = x,
  \]
  consider the following triangle with the unit circle:
  \begin{image}
    \begin{tikzpicture}
      \begin{axis}[
          xmin=-1.1,xmax=1.1,ymin=-1.1,ymax=1.1,
          axis lines=center,
          width=4in,
          ticks=none,
          clip=false,
          unit vector ratio*=1 1 1,
          %xlabel=$x$, ylabel=$y$,
          every axis y label/.style={at=(current axis.above origin),anchor=south},
          every axis x label/.style={at=(current axis.right of origin),anchor=west},
        ]        
        \addplot [penColor2!50!white, very thick, smooth, domain=(-90:90)] ({cos(x)},{sin(x)}); %% unit circle
        \addplot [black!50!white, dashed, smooth, domain=(90:270)] ({cos(x)},{sin(x)}); %% unit circle
        \addplot[color=penColor2!50!white,fill=penColor2!50!white,only marks,mark=*] coordinates{(0,1)};  %% closed hole
        \addplot[color=penColor2!50!white,fill=penColor2!50!white,only marks,mark=*] coordinates{(0,-1)};  %% closed hole         
        \addplot [ultra thick] plot coordinates {(0,0) (.766,.643)}; %% 40 degrees
        
        \addplot [ultra thick] plot coordinates {(.766,0) (.766,.643)}; %% 40 degrees
        \addplot [ultra thick] plot coordinates {(0,0) (.766,0)}; %% 40 degrees
        
        %\addplot [ultra thick,penColor3] plot coordinates {(1,0) (1,.839)}; %% 40 degrees          
        
        \addplot [textColor,smooth, domain=(0:40)] ({.15*cos(x)},{.15*sin(x)});
        %\addplot [very thick,penColor] plot coordinates {(0,0) (.766,.643)}; %% sector
        %\addplot [very thick,penColor] plot coordinates {(0,0) (1,0)}; %% sector
        %\addplot [very thick, penColor, smooth, domain=(0:40)] ({cos(x)},{sin(x)}); %% sector
        \node at (axis cs:.12,.07) [anchor=west] {$\theta$};
        \node at (axis cs:.84,.322) {$x$};
        \node at (axis cs:.383,0) [anchor=north] {$\sqrt{1-x^2}$};
        \node at (axis cs:.38,.32) [anchor=south] {$1$};
      \end{axis}
    \end{tikzpicture}
  \end{image}
  From the unit circle above, we see that 
  \begin{align*}
    \theta' &= \frac{1}{\cos(\theta)}\\
    &=\frac{\mathrm{hyp}}{\mathrm{adj}}\\
    &= \answer[given]{\frac{1}{\sqrt{1-x^2}}}.
  \end{align*}
  To be completely explicit, 
  \[
  \ddx \theta = \ddx \arcsin(x) = \answer[given]{\frac{1}{\sqrt{1-x^2}}}.
  \]
\end{explanation}
\end{theorem}


\begin{question}
  Compute:
  \[
  \ddx \sin^{-1}(x)
  \begin{prompt}
    = \answer{1/\sqrt{1-x^2}}
  \end{prompt}
  \]
\end{question}




We can do something similar with arctangent. 


\begin{theorem}[The derivative of arctangent]\index{derivative!of arctangent}
  \[
  \ddx \arctan(x) = \frac{1}{1+x^2}.
  \]
  \begin{explanation} 
    %% To start, note that the Inverse Function Theorem assures us that this
    %% derivative actually exists.
    Recall
    \[
    \arctan(x) = \theta\qquad \text{exactly when} \qquad\tan(\theta) = x\qquad\text{and}\qquad \answer[given]{-\frac{\pi}{2}}< \theta< \answer[given]{\frac{\pi}{2}}.
    \]
    Implicitly differentiating with respect $x$ we see
    \begin{align*}
      \tan(\theta) &= x\\
      \ddx \tan(\theta) &= \ddx x         &\text{Differentiate both sides.}\\
      \sec^2(\theta) \cdot \theta' &= 1   &\text{Implicit differentiation.}\\
      \theta' &= \frac{1}{\sec^2(\theta)} &\text{Solve for $\theta'$.}\\
      \theta' &= \cos^2(\theta).
    \end{align*}
    While $\theta' = \cos^2(\theta)$, we need our answer written in terms
    of $x$. Since we are assuming that
    \[
    \tan(\theta) = \frac{\sin(\theta)}{\cos(\theta)}= x,
    \]
    consider the following triangle with the unit circle:
    \begin{image}
      \begin{tikzpicture}
	\begin{axis}[
            xmin=-1.1,xmax=1.1,ymin=-1.1,ymax=1.1,
            axis lines=center,
            width=4in,
            ticks=none,
            clip=false,
            unit vector ratio*=1 1 1,
            %xlabel=$x$, ylabel=$y$,
            every axis y label/.style={at=(current axis.above origin),anchor=south},
            every axis x label/.style={at=(current axis.right of origin),anchor=west},
          ]        
          \addplot [penColor3!50!white, very thick, smooth, domain=(-90:90)] ({cos(x)},{sin(x)}); %% unit circle
          \addplot [black!50!white, dashed, smooth, domain=(90:270)] ({cos(x)},{sin(x)}); %% unit circle
          \addplot[color=penColor3!50!white,fill=white,only marks,mark=*] coordinates{(0,1)};  %% open hole
          \addplot[color=penColor3!50!white,fill=white,only marks,mark=*] coordinates{(0,-1)};  %% open hole     
          \addplot [ultra thick] plot coordinates {(0,0) (.766,.643)}; %% 40 degrees

          \addplot [ultra thick] plot coordinates {(.766,0) (.766,.643)}; %% 40 degrees
          \addplot [ultra thick] plot coordinates {(0,0) (.766,0)}; %% 40 degrees
          
          %\addplot [ultra thick,penColor3] plot coordinates {(1,0) (1,.839)}; %% 40 degrees          

          \addplot [textColor,smooth, domain=(0:40)] ({.15*cos(x)},{.15*sin(x)});
          %\addplot [very thick,penColor] plot coordinates {(0,0) (.766,.643)}; %% sector
          %\addplot [very thick,penColor] plot coordinates {(0,0) (1,0)}; %% sector
          %\addplot [very thick, penColor, smooth, domain=(0:40)] ({cos(x)},{sin(x)}); %% sector
          \node at (axis cs:.12,.07) [anchor=west] {$\theta$};
          \node at (axis cs:.84,.322)[anchor=west] {$\frac{x}{\sqrt{1+x^2}}$};
          \node at (axis cs:.383,0) [anchor=north] {$\frac{1}{\sqrt{1+x^2}}$};
          \node at (axis cs:.38,.32) [anchor=south] {$1$};
        \end{axis}
\end{tikzpicture}
\end{image}
From the unit circle above, we see that 
\begin{align*}
  \theta' &= \cos^2(\theta)\\
  &= \left(\frac{\mathrm{adj}}{\mathrm{hyp}}\right)^2\\
  &= \answer[given]{\frac{1}{1+x^2}}.
\end{align*}
To be completely explicit, 
\[
\ddx \theta = \ddx \arctan(x) = \answer[given]{\frac{1}{1+x^2}}.
\]
\end{explanation}
\end{theorem}

\begin{question}
  Compute:
  \[
  \ddx \tan^{-1}(\sqrt{x})
  \begin{prompt}
    = \answer{1/(2\sqrt{x}(1+x))}
  \end{prompt}
  \]
\end{question}



Finally, we investigate the derivative of arcsecant.

\begin{theorem}[The derivative of arcsecant]\index{derivative!of arcsecant}
\[
\ddx \arcsec(x) = \frac{1}{|x|\sqrt{x^2-1}}\qquad\text{for $|x|>1$.}
\]
\begin{explanation} 
  %% To start, note that the Inverse Function Theorem assures us that this
  %% derivative actually exists.
  Recall
\[
\arcsec(x) = \theta\qquad\text{exactly when}\qquad\sec(\theta) = x\qquad\text{and}\qquad
      \answer[given]{0}\le \theta\le \answer[given]{\pi}\text{ with }x\ne \answer[given]{\pi/2}. 
\]
Implicitly differentiating with respect $x$ we see
\begin{align*}
\sec(\theta) &= x\\
\ddx \sec(\theta) &= \ddx x                     &\text{Differentiate both sides.}\\
\sec(\theta)\tan(\theta) \cdot \theta' &= 1     &\text{Implicit differentiation.}\\
\theta' &= \frac{1}{\sec(\theta)\tan(\theta)}   &\text{Solve for $\theta'$.}\\
\theta' &= \frac{\cos^2(\theta)}{\sin(\theta)}.
\end{align*}
While $\theta' = \frac{\cos^2(\theta)}{\sin(\theta)}$, we need our
answer written in terms of $x$. Since we are assuming that
\[
\sec(\theta) = \frac{1}{\cos(\theta)}= x,
\]
consider the following triangle with the unit circle:
\begin{image}
\begin{tikzpicture}
	\begin{axis}[
            xmin=-1.1,xmax=1.1,ymin=-1.1,ymax=1.1,
            axis lines=center,
            width=4in,
            ticks=none,
            clip=false,
            unit vector ratio*=1 1 1,
            %xlabel=$x$, ylabel=$y$,
            every axis y label/.style={at=(current axis.above origin),anchor=south},
            every axis x label/.style={at=(current axis.right of origin),anchor=west},
          ]        
          \addplot [penColor4!50!white, very thick, smooth, domain=(0:180)] ({cos(x)},{sin(x)}); %% unit circle
          \addplot [black!50!white, dashed, smooth, domain=(180:360)] ({cos(x)},{sin(x)}); %% unit circle
          \addplot[color=penColor4!50!white,fill=white,only marks,mark=*] coordinates{(0,1)};  %% open hole
          \addplot[color=penColor4!50!white,fill=penColor4!50!white,only marks,mark=*] coordinates{(1,0)};  %% closed hole
          \addplot[color=penColor4!50!white,fill=penColor4!50!white,only marks,mark=*] coordinates{(-1,0)};  %% closed hole     
          \addplot [ultra thick] plot coordinates {(0,0) (.766,.643)}; %% 40 degrees

          \addplot [ultra thick] plot coordinates {(.766,0) (.766,.643)}; %% 40 degrees
          \addplot [ultra thick] plot coordinates {(0,0) (.766,0)}; %% 40 degrees
          
          %\addplot [ultra thick,penColor3] plot coordinates {(1,0) (1,.839)}; %% 40 degrees          

          \addplot [textColor,smooth, domain=(0:40)] ({.15*cos(x)},{.15*sin(x)});
          %\addplot [very thick,penColor] plot coordinates {(0,0) (.766,.643)}; %% sector
          %\addplot [very thick,penColor] plot coordinates {(0,0) (1,0)}; %% sector
          %\addplot [very thick, penColor, smooth, domain=(0:40)] ({cos(x)},{sin(x)}); %% sector
          \node at (axis cs:.12,.07) [anchor=west] {$\theta$};
          \node at (axis cs:.84,.322)[anchor=west] {$\sqrt{1-\frac{1}{x^2}}$};
          \node at (axis cs:.383,0) [anchor=north] {$\frac{1}{x}$};
          \node at (axis cs:.38,.32) [anchor=south] {$1$};
        \end{axis}
\end{tikzpicture}
\end{image}
From the unit circle above, we see that 
\begin{align*}
  \theta' &= \frac{\cos^2(\theta)}{\sin(\theta)}\\
  &= \frac{
    \left(\frac{\mathrm{adj}}{\mathrm{hyp}}\right)^2}{\frac{\mathrm{opp}}{\mathrm{hyp}}}\\
  &= \frac{\left(\mathrm{adj}\right)^2}{\mathrm{opp}} &\text{Note, $\mathrm{hyp}=1$.}\\
  &= \frac{\answer[given]{1/x^2}}{\sqrt{1-1/x^2}}\\
  &= \frac{1}{|x|\sqrt{x^2-1}},
\end{align*}
To be completely explicit, 
\[
\ddx \theta = \ddx \arcsec(x) = \frac{1}{|x|\sqrt{x^2-1}}\qquad\text{for $|x|>1$}. 
\]
\end{explanation}
\end{theorem}

\begin{question}
  Compute:
  \[
  \ddx \sec^{-1}(3x)
  \begin{prompt}
    = \answer{\frac{1}{|x|\sqrt{(3x)^2-1}}}
  \end{prompt}
  \]
\end{question}

We leave it to you, the reader, to investigate the derivatives of
cosine, arccosecant, and arccotangent. However, as a gesture of
friendship, we now present you with a list of derivative formulas for
inverse trigonometric functions.

\begin{theorem}[The Derivatives of Inverse Trigonometric Functions] \hfil
\begin{itemize}
\item $\dd{x} \arcsin(x) = \frac{1}{\sqrt{1-x^2}}$.
\item $\dd{x} \arccos(x) = \frac{-1}{\sqrt{1-x^2}}$.
\item $\dd{x} \arctan(x) = \frac{1}{1+x^2}$.
\item $\dd{x} \arcsec(x) = \frac{1}{|x|\sqrt{x^2-1}}$ for $|x|>1$.
\item $\dd{x} \arccsc(x) = \frac{-1}{|x|\sqrt{x^2-1}}$ for $|x|>1$.
\item $\dd{x} \arccot(x) = \frac{-1}{1+x^2}$.
\end{itemize}
\end{theorem}








\end{document}
