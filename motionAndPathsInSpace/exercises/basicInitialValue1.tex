\documentclass{ximera}

\newcommand{\RR}{\mathbb R}
\renewcommand{\d}{\,d}
\newcommand{\dd}[2][]{\frac{d #1}{d #2}}
\renewcommand{\l}{\ell}
\newcommand{\ddx}{\frac{d}{dx}}
\newcommand{\dfn}{\textbf}
\newcommand{\eval}[1]{\bigg[ #1 \bigg]}

\author{Gregory Hartman \and Bart Snapp}
\license{Creative Commons 3.0 By-NC}
\acknowledgement{https://github.com/APEXCalculus}


\outcome{View a vector valued function as a position function.}
\outcome{Differentiate and integrate vector-valued functions.}
\outcome{For a position vector function of time, interpret the derivative as velocity.}
\outcome{For a position vector function of time, interpret the second derivative as acceleration.}

\begin{document}
\begin{exercise}
  The starting position of a particle is given by
  \[
  \vec{p}(0) =\vector{5,-2}
  \]
  Suppose the initial velocity is given by $\vec{v}(0) =\vector{1,2}$
  and the acceleration is given by $\vec{a}(t) =\vector{2,3}$.  Find:
\begin{itemize}
\item The velocity function: $\vec{v}(t) =\vector{\answer{2t+1},\answer{3t+2}}$
\item The speed function: $\vec{s}(t) = \answer{\sqrt{(1+2t)^2+(2+3t)^2}}$ 
\item The position function: $\vec{p}(t) = \vector{\answer{t^2+t+5},\answer{3t^2/2+2t-2}}$
\end{itemize}
\end{exercise}
\end{document}
