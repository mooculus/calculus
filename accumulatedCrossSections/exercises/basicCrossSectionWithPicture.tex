\documentclass{ximera}

\newcommand{\RR}{\mathbb R}
\renewcommand{\d}{\,d}
\newcommand{\dd}[2][]{\frac{d #1}{d #2}}
\renewcommand{\l}{\ell}
\newcommand{\ddx}{\frac{d}{dx}}
\newcommand{\dfn}{\textbf}
\newcommand{\eval}[1]{\bigg[ #1 \bigg]}


\author{Bart Snapp}
%Example from the original text
\license{Creative Commons 3.0 By-NC}


\outcome{Set up an integral or sum of integrals that gives the volume of a solid with known cross sections}



\begin{document}
\begin{exercise}

The base of a solid is the region bounded by $f(x)=x^2-1$ and
$g(x)=-x^2+1$:
\begin{image}
\begin{tikzpicture}
  \begin{axis}[
      xmin=-1, xmax=1,domain=-1:1,
      clip=false,
      axis lines =center, xlabel=$x$, ylabel=$y$,
      every axis y label/.style={at=(current axis.above origin),anchor=south},
      every axis x label/.style={at=(current axis.right of origin),anchor=west},
      axis on top,
    ] 
    \addplot [penColor2,very thick] {x^2-1};
    \addplot [penColor,very thick] {-x^2+1};
    \node at (axis cs:1,0.8) [penColor] {$y = -x^2+1$};
    \node at (axis cs:1,-0.8) [penColor2] {$y = x^2-1$};
  \end{axis}
\end{tikzpicture}
\end{image}
The cross-section of this solid consists of equilateral triangles
that are perpendicular to the $x$-axis:
\begin{image}
\begin{tikzpicture}
  \begin{axis}[
      xmin=-1.2, xmax=1.2,domain=-1:1,
      ymin=-.4,ymax=1,
      clip=false,
      ytick={0,.8},yticklabels={$0$,$1$},
      axis lines =center, xlabel=$x$, ylabel=$z$,
      every axis y label/.style={at=(current axis.above origin),anchor=south},
      every axis x label/.style={at=(current axis.right of origin),anchor=west},
      axis on top,
    ] 
    \addplot [penColor2,very thick] {(0.3)*x^2-(0.3)};
    \addplot [penColor,very thick] {(-0.3)*x^2+(0.3)};

    \addplot [draw=penColor, fill=fill1,very thick] plot coordinates {(-.11,-.3) (.1,.3) (0,.8) (-.11,-.3)};

    \addplot [draw=penColor, fill=fill1,very thick] plot coordinates {(-.67,-.17) (-.53,.21) (-.6,.6) (-.67,-.17)};

    \addplot [draw=penColor, fill=fill1,very thick] plot coordinates {(.53,-.22) (.67,.16) (.6,.6) (.53,-.22)};


    \addplot [black,very thick] {-.28 + .0086*sqrt(16031-14948*x^2)};

    \addplot [->] plot coordinates {(-.6,-.8) (.6,0.8)};
    \node[anchor=south west] at (axis cs:.6,0.8) {$y$};
  \end{axis}
\end{tikzpicture}
\end{image}
To find the volume of this solid:

\begin{exercise}
We want to find the volume of the triangular slab at a given $x$
value. We know the width of each slab is $\Delta x$:
\begin{image}
\begin{tikzpicture}
  \begin{axis}[
      xmin=-1.2, xmax=1.2,domain=-1:1,
      ymin=-.4,ymax=1,
      clip=false,
      ytick={0,.8},yticklabels={$0$,$1$},
      axis lines =center, xlabel=$x$, ylabel=$z$,
      every axis y label/.style={at=(current axis.above origin),anchor=south},
      every axis x label/.style={at=(current axis.right of origin),anchor=west},
      axis on top,
    ] 
    \addplot [penColor2,very thick] {(0.3)*x^2-(0.3)};
    \addplot [penColor,very thick] {(-0.3)*x^2+(0.3)};

    \addplot [draw=penColor, fill=fill1,very thick] plot coordinates {(-.11,-.3) (.1,.3) (0,.8) (-.11,-.3)};

    \addplot [draw=penColor, fill=fill1,very thick] plot coordinates {(-.11,-.3) (-.19,-.3) (-.08,.8) (0,.8) (-.11,-.3)};

    \draw[decoration={brace,raise=.1cm},decorate,thin] (axis cs:-.11,-.3)--(axis cs:-.19,-.3);  
    \node[anchor=north] at (axis cs:.-.15,-.35) {$\Delta x$};
    
    \addplot [black,very thick] {-.28 + .0086*sqrt(16031-14948*x^2)};

    \addplot [->] plot coordinates {(-.6,-.8) (.6,0.8)};
    \node[anchor=south west] at (axis cs:.6,0.8) {$y$};
  \end{axis}
\end{tikzpicture}
\end{image}
The area of each triangular cross-section is given by.
\[
A(x) = \answer[given]{\sqrt{3}(1-x^2)^2}
\]
\begin{hint}
  The base of the slab has length $\answer[given]{2(1-x^2)}$ since
  \[
  (1-x^2)- (x^2-1) = 2(1-x^2).
  \]
  By geometry, the height is $\answer[given]{\sqrt{3}(1-x^2)}$.  So
  the area of this triangle is
  \[
  \frac{1}{2} \text{base} \cdot \text{height} =
  \sqrt{3}(1-x^2)^2.
  \]
\end{hint}

We want to ``sum'' (integrate) all of the infinitesimal volumes
\[
\d V = A(x) \d x
\]
from $x=-1$ to $x=1$.  Thus the total volume is
\begin{align*}
  \text{Volume} &= \int_{-1}^1 A(x) \d x  \\
  &= \int_{-1}^1 \sqrt{3} (1-x^2)^2 \d x\\
  &=\sqrt{3} \int_{-1}^1 1-2x^2+x^4 \d x \\
  &=\sqrt{3} \eval{\answer[given]{x-\frac{2}{3}x^3+\frac{x^5}{5}}}_{-1}^1\\
  &=2\sqrt{3}\left(1-\frac{2}{3}+\frac{1}{5}\right)\\
  &=\frac{16\sqrt{3}}{15}.
\end{align*}


	\end{exercise}
	\end{exercise}
\end{document}