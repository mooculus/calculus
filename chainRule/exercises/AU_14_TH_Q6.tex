\documentclass{ximera}

\newcommand{\RR}{\mathbb R}
\renewcommand{\d}{\,d}
\newcommand{\dd}[2][]{\frac{d #1}{d #2}}
\renewcommand{\l}{\ell}
\newcommand{\ddx}{\frac{d}{dx}}
\newcommand{\dfn}{\textbf}
\newcommand{\eval}[1]{\bigg[ #1 \bigg]}


\begin{document}

\outcome{Understand the properties of trigonometric functions.}
\outcome{Evaluate expressions and solve equations involving trigonometric functions and inverse trigonometric functions.}
\outcome{Understand rate of change when quantities are dependent upon each other.}
\outcome{Apply chain rule to relate quantities expressed with different units.}
\begin{exercise}
A spherical meteorite enters the earth's atmosphere and burns up radially at a rate proportional to its surface area with constant of proportionality $\alpha<0$. We shall show that its radius decreases at a constant rate. 

First, given a sphere of radius $r$, recall the formula for its volume, $V$.
\[
V=\answer{\frac{4}{3}\pi r^3}
\]
\begin{exercise}
Similarly, recall that the surface area $A$ of a sphere of radius $r$ is given by
\[
A=\answer{4\pi r^2}
\]
\begin{exercise}
Since the meteorite burns up radially at a rate proportional to its surface area $A$ with constant of proportionality $\alpha$, 
\[
\dd[V]{t}=\answer{\alpha A}
\]
\begin{exercise}
Now compute $\dd[V]{t}$ and express it in terms of the surface area $A$ and $r'=\dd[r]{t}$. (If you \textit{must} enter a derivative of the function $r$ with respective time, simply write $r'$, not $\dd[r]{t}$
\[
\dd[V]{t}=\answer{A}\cdot r'
\]
\begin{exercise}
Equating these two expressions obtained for $\dd[V]{t}$, we find that $\alpha A=A r'$. If the meteorite is still burning up at a time $t$, then $A(t)\ne 0$. Therefore for such $t$,
\[
r'(t)=\answer{\alpha}
\]
Thus, $\dd[r]{t}$ is constant.
\end{exercise}
\end{exercise}
\end{exercise}
\end{exercise}
\end{exercise}
\end{document}
