\documentclass{ximera}

\newcommand{\RR}{\mathbb R}
\renewcommand{\d}{\,d}
\newcommand{\dd}[2][]{\frac{d #1}{d #2}}
\renewcommand{\l}{\ell}
\newcommand{\ddx}{\frac{d}{dx}}
\newcommand{\dfn}{\textbf}
\newcommand{\eval}[1]{\bigg[ #1 \bigg]}


\author{Jim Talamo}
\license{Creative Commons 3.0 By-NC}


\outcome{Understand the relationship between sequences, sequences of partial sums, and remainders.}

\begin{document}

\begin{exercise}

Suppose that $\{a_n\}_{n=1}$ is a sequence and it is known that $s_n =  \frac{2n+\ln(n)}{n}$.

Is it possible to define a sequence of remainders for $\sum_{k=1}^{\infty} a_k$?

\begin{multipleChoice}
\choice[correct]{Yes, because $\sum_{k=1}^{\infty} a_k$ converges.}
\choice{No, because $\sum_{k=1}^{\infty} a_k$ diverges.}
\end{multipleChoice}

\begin{exercise}
Find an explicit formula for $r_n$.

\[
r_n = \answer{-\frac{\ln(n)}{n}}
\]

\begin{hint}
Note that by using growth rates, we have that $\lim_{n \to \infty} s_n = \answer{2}$, meaning that $\sum_{k=1}^{\infty} a_k$ converges to $\answer{2}$.  Using the relationship

\[
\sum_{k=1}^{\infty} a_k = s_n+r_n
\]

and the formula $s_n = \frac{2n+\ln(n)}{n}$, we can substitute into the above formula to find that

\[
2 = \frac{2n+\ln(n)}{n} +r_n.
\]
\end{hint}

\begin{exercise}
What is the smallest integer $N$ needed so $\sum_{k=1}^N a_k$ approximates  $\sum_{k=1}^\infty a_k$ to within $.001$ of its true value?  What is $\sum_{k=1}^N a_k$ for this value of $N$ to four decimal places?
(You will have to use technology to find this value of $N$).

\[
N = \answer[tolerance=.0001]{9119} \qquad \textrm{ and } \qquad \sum_{k=1}^N a_k = \answer[tolerance=.001]{2.0009999}
\]

\begin{hint}
We need to find $N$ so $|r_N| \leq .001$.  This means we need to find the smallest \underline{integer} value of $N$ so $\frac{\ln(N)}{N} \leq .001$.  Technology can be used to find $N=9118.006$.  This means that $N=9118$ is just barely too small; we need to use $N=9119$.
\end{hint}
\end{exercise}

\end{exercise}
\end{exercise}
\end{document}