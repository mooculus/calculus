\documentclass{ximera}

\newcommand{\RR}{\mathbb R}
\renewcommand{\d}{\,d}
\newcommand{\dd}[2][]{\frac{d #1}{d #2}}
\renewcommand{\l}{\ell}
\newcommand{\ddx}{\frac{d}{dx}}
\newcommand{\dfn}{\textbf}
\newcommand{\eval}[1]{\bigg[ #1 \bigg]}


%\outcome{Given a velocity function, calculate displacement and distance traveled.}
%\outcome{Given a velocity function, find the position function.}
%\outcome{Given an acceleration function, find the velocity function.}
%\outcome{Understand the difference between displacement and distance traveled.}
%\outcome{Understand the relationship between position, velocity and acceleration.}

\author{Nela Lakos \and Kyle Parsons}

\begin{document}
\begin{exercise}

Consider a particle moving along a line.  Its acceleration, its initial velocity  and its initial position are given by
\begin{align*}
a(t) &= 2t-4\\
v(0) &= 3\\
s(0) &= -4
\end{align*}
for $0\leq t\leq4$.

(a) Find  the velocity of the particle at time $t$, for $0\leq t\leq4$.
\begin{hint}
Solve the IVP:
\begin{align*}
v'(t) &= 2t-4\\
v(0) &= 3
\end{align*}
\end{hint}
\[
v(t) = \answer{t^2-4t+3}.
\]

(b) Find the position of the particle at time $t$, for $0\leq t\leq4$.
\begin{hint}
Solve the IVP:
\begin{align*}
s'(t) &= v(t)\\
s(0) &= -4
\end{align*}
You have to use the expression for $v(t)$ obtained in part (a).
\end{hint}
\[
s(t) = \answer{\frac{1}{3}} t^3-\answer{2} t^2+\answer{3} t+\answer{-4}.
\]


\end{exercise}
\end{document}