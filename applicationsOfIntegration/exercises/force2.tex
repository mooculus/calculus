\documentclass{ximera}

\newcommand{\RR}{\mathbb R}
\renewcommand{\d}{\,d}
\newcommand{\dd}[2][]{\frac{d #1}{d #2}}
\renewcommand{\l}{\ell}
\newcommand{\ddx}{\frac{d}{dx}}
\newcommand{\dfn}{\textbf}
\newcommand{\eval}[1]{\bigg[ #1 \bigg]}


\author{Jim Talamo and Nicholas Hemleben}
\license{Creative Commons 3.0 By-NC}


\outcome{Use integrals to compute work done by a variable force}

\begin{document}
\begin{exercise}

A force described by $F(x) =\left|x-\frac{3}{2}\right|$ is applied to move a particle from $x=0$ to $x=4$.  Calculate the amount of work done. 

 The work done is $\answer{\frac{17}{4}}$ units.

\begin{hint}
Note that the absolute value can be written as a piecewise function:

\[
 |x| = \left\{ \begin{array}{cl} 
x &, x > 0 \\
-x &, x\leq 0
\end{array} \right.
\]

Using this, we can write:
\[
 F(x) = \left\{ \begin{array}{cl} 
x-\frac{3}{2} &, x > \answer{\frac{3}{2}} \\
-\left(\answer{x-\frac{3}{2}}\right)  &, x\leq \answer{\frac{3}{2}}
\end{array} \right.
\]
and use two integrals to compute the work.

\end{hint}
\end{exercise}
\end{document}
