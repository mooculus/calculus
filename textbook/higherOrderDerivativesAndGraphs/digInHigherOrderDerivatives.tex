\documentclass{ximera}

\newcommand{\RR}{\mathbb R}
\renewcommand{\d}{\,d}
\newcommand{\dd}[2][]{\frac{d #1}{d #2}}
\renewcommand{\l}{\ell}
\newcommand{\ddx}{\frac{d}{dx}}
\newcommand{\dfn}{\textbf}
\newcommand{\eval}[1]{\bigg[ #1 \bigg]}


\outcome{}

\title[Dig-In:]{Higher order derivatives}

\begin{document}
\begin{abstract}
  
\end{abstract}
\maketitle


Since the derivative gives us a formula for the slope of a tangent
line to a curve, we can gain information about a function purely from
the sign of the derivative.  In particular, we have the following theorem
\begin{theorem}
  If $f$ is differentiable on an interval, then
\begin{itemize}
\item $f'(x)>0$ on that interval whenever $f$ is increasing as $x$
  increases on that interval.
\item $f'(x)<0$ on that interval whenever $f$ is decreasing as $x$
  increases on that interval.
\end{itemize}
\end{theorem}
Moreover, we can apply this line of reasoning to the derivative
itself, by taking its derivative. We call the derivative of the
derivative the \textbf{second derivative}, the derivative of the
derivative of the derivative the \textbf{third derivative}, and so
on. We have special notation for higher derivatives, for example:
\begin{description}
\item[First derivative:] $\ddx f(x) = f'(x) = f^{(1)}(x)$.
\item[Second derivative:] $\dd[^2]{x^2} f(x) = f''(x) = f^{(2)}(x)$.
\item[Third derivative:] $\dd[^3]{x^3} f(x) = f'''(x) = f^{(3)}(x)$.
\end{description}

We use the facts above in our next example.

\begin{example}
  Consider these three graphs:
  \begin{image}
  \begin{tikzpicture}
	\begin{axis}[
            xmin=-2,xmax=2,ymin=-8,ymax=8,
            axis lines=center,
            ticks=none,
            width=6in,
            height=3in,
            every axis y label/.style={at=(current axis.above origin),anchor=south},
            every axis x label/.style={at=(current axis.right of origin),anchor=west},
          ]        
          \addplot [very thick,penColor,smooth, domain=(-2:2)] {x^3+.3*x^2-2*x)};
          \addplot [very thick, dotted,penColor,smooth, domain=(-2:2)] {3*x^2+2*.3*x-2)};
          \addplot [very thick, dashed,penColor,smooth, domain=(-2:2)] {6*.3*x+2)};
        \end{axis}
  \end{tikzpicture}
  \end{image}
  One is of $f$, another is of $f'$ and a third is of $f''$.  Explain
  what strategies you could use to identify which graph corresponds

\end{example}




\begin{example}
\begin{image}
 \begin{tikzpicture}
	\begin{axis}[
            domain=-4:4,
            ticks=none,
            ymax=2, ymin=-2,
            xmax=4, xmin=-4,
            axis lines =middle,
            every axis y label/.style={at=(current axis.above origin),anchor=south},
            every axis x label/.style={at=(current axis.right of origin),anchor=west},
            width=6in,
            height=3in,
          ]
	  \addplot [very thick, dashed,penColor,smooth,samples=100] {2*gauss(0,.75)};
          \addplot [very thick, penColor,smooth,samples=100] {2/(.75*sqrt(2*pi))*exp(-((x)^2)/(2*.75^2)) *(-x)/(.75^2)};
          \addplot [very thick, dotted,penColor,smooth,samples=100] {2/(.75*sqrt(2*pi))*exp(-((x)^2)/(2*.75^2)) *(-1)/(.75^2) +
2/(.75*sqrt(2*pi))*exp(-((x)^2)/(2*.75^2)) *(x^2)/(.75^4)};
        \end{axis}
          \end{tikzpicture}
\end{image}
\end{example}

\begin{example}
  Consider these three graphs:
  \begin{image}
  \begin{tikzpicture}
	\begin{axis}[
            xmin=-6.75,xmax=6.75,ymin=-1.5,ymax=1.5,
            axis lines=center,
            ticks=none,
            width=6in,
            height=3in,
            every axis y label/.style={at=(current axis.above origin),anchor=south},
            every axis x label/.style={at=(current axis.right of origin),anchor=west},
          ]        
          \addplot [very thick, dotted,penColor, samples=100,smooth, domain=(-6.75:6.75)] {sin(deg(x))};
          \addplot [very thick, dashed,penColor, samples=100,smooth, domain=(-6.75:6.75)] {cos(deg(x))};
          \addplot [very thick, penColor, samples=100,smooth, domain=(-6.75:6.75)] {-sin(deg(x))};
        \end{axis}
  \end{tikzpicture}
  \end{image}
  One is of $f$, another is of $f'$ and a third is of $f''$.  Explain
  what strategies you could use to identify which graph corresponds

\end{example}


\end{document}
