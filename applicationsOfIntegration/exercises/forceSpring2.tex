\documentclass{ximera}

\newcommand{\RR}{\mathbb R}
\renewcommand{\d}{\,d}
\newcommand{\dd}[2][]{\frac{d #1}{d #2}}
\renewcommand{\l}{\ell}
\newcommand{\ddx}{\frac{d}{dx}}
\newcommand{\dfn}{\textbf}
\newcommand{\eval}[1]{\bigg[ #1 \bigg]}


\author{Jim Talamo}
\license{Creative Commons 3.0 By-NC}


\outcome{Use integrals to compute work done by a variable force}

\begin{document}
\begin{exercise}

Hooke's Law states that the force required to displace a spring $x$ meters from equilibrium is given by $F(x) = kx$, where $k$ is a constant of proportionality.  If $60 J$ of work is required to stretch a spring $3 m$ from its equilibrium position, find the amount of force required to stretch the spring $2 m$ from its equilibrium position.

The force required is $\answer{\frac{240}{9}} N$.

\begin{hint}
Note that the work, not the force, is given!  Find the work in terms of $k$ first.  The integral is:

\[
W = \int_{\answer{0}}^{\answer{3}} \answer{ k x } \d x
\]

\begin{question}
Using $W=60$ and evaluating the integral gives:

\[
60 = \answer{\frac{9}{2}} k
\]
\end{question}

\end{hint}

\end{exercise}
\end{document}
