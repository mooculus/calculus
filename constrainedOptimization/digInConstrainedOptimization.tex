\documentclass{ximera}

\newcommand{\RR}{\mathbb R}
\renewcommand{\d}{\,d}
\newcommand{\dd}[2][]{\frac{d #1}{d #2}}
\renewcommand{\l}{\ell}
\newcommand{\ddx}{\frac{d}{dx}}
\newcommand{\dfn}{\textbf}
\newcommand{\eval}[1]{\bigg[ #1 \bigg]}



\title[Dig-In:]{Constrained optimization}


\outcome{Define a constrained optimization problem.}
\outcome{Optimize a function on a closed and bounded set.}
\outcome{Solve constrained optimization problems.}


\begin{document}
\begin{abstract}
  We learn to optimize surfaces along and within given paths.
\end{abstract}
\maketitle

When optimizing functions of one variable $f:\R\to\R$, we have the
Extreme Value Theorem:

\begin{theorem}[Extreme Value Theorem]\label{theorem:evt}\index{Extreme Value Theorem}
If $f$ is a continuous function for all $x$ in the closed interval
$[a,b]$, then there are points $c$ and $d$ in $[a,b]$, such that
$(c,f(c))$ is a global maximum and $(d,f(d))$ is a global
minimum on $[a, b]$.

Below, we see a geometric interpretation of this theorem.
\begin{image}
\begin{tikzpicture}
	\begin{axis}[
            domain=0:6, xmin=0, xmax=6, ymin=0, ymax=2.5,
            axis lines =left, xlabel=$x$, ylabel=$y$,
            every axis y label/.style={at=(current axis.above origin),anchor=south},
            every axis x label/.style={at=(current axis.right of origin),anchor=west},
            xtick={1,2,4,5}, ytick={.2,2.2},
            xticklabels={$a$,$c$,$d$,$b$}, yticklabels={$f(d)$,$f(c)$},
            axis on top,
          ]
          \addplot [draw=none, fill=fill1, domain=(1:5)] {2.5} \closedcycle;

          \addplot [textColor,dashed] plot coordinates {(0,2.2) (2,2.2)};
          \addplot [textColor,dashed] plot coordinates {(0,.2) (4,.2)};
          \addplot [textColor,dashed] plot coordinates {(2,0) (2,2.2)};
          \addplot [textColor,dashed] plot coordinates {(4,0) (4,.2)};

          \addplot [fill1,very thick] plot coordinates {(1,0) (1,2.5)};
          \addplot [fill1,very thick] plot coordinates {(5,0) (5,2.5)};

          \addplot [very thick,penColor, smooth,domain=(1.5:2.5)] {sin(deg(x*1.57-1.57)) + 1.2};%max
          \addplot [very thick,penColor, smooth,domain=(3.5:4.5)] {sin(deg(x*1.57-1.57)) + 1.2};%min
          \addplot [very thick,dashed,penColor!50!background, smooth,domain=(2.5:3.5)] {sin(deg(x*1.57 - 1.57)) + 1.2}; 
          \addplot [very thick,dashed,penColor!50!background, smooth,domain=(1:1.5)] {sin(deg(x*1.57 - 1.57)) + 1.2}; 
          \addplot [very thick,dashed,penColor!50!background, smooth,domain=(4.5:5)] {sin(deg(x*1.57 - 1.57)) + 1.2}; 
          
          \addplot [color=penColor,fill=penColor,only marks,mark=*] coordinates{(2,2.2)};  %% closed hole          
          \addplot [color=penColor,fill=penColor,only marks,mark=*] coordinates{(4,.2)};  %% closed hole          
        \end{axis}
\end{tikzpicture}
%% \caption{A geometric interpretation of the Extreme Value Theorem. A
%%   continuous function $f(x)$ attains both an global maximum and an
%%   global minimum on an interval $[a,b]$. Note, it may be the case
%%   that $a=c$, $b=d$, or that $d<c$.}
%% \label{figure:extreme-value}
%% \end{marginfigure}
\end{image}
\end{theorem}

A similar theorem applies to functions of several variables.

\begin{theorem}[Extreme Value Theorem]\index{Extreme Value Theorem}
  Let $F:\R^n\to\R$ be a continuous function on a closed, bounded set
  $S$. Then $F$ has a maximum and minimum value on $S$.
\end{theorem}

When finding extrema of functions in your first calculus course, you
had to check inside an interval and at the end-points. Now we need to
check the interior of a region and the boundary curve. If we check
those critical points, then the Extreme Value Theorem tells us that we
have found the extrema! No other tests are necessary.


To check the interior of a region, one finds where the gradient vector
is zero or undefined---we did that in the last section. Hence in this
section, we will focus on how one finds extrema on the boundary. In
this case, the Extreme Value Theorem \textbf{still} applies, and so we
will be sure that we found the extrema on the boundary.  Let's see
some examples, we start with a classic real-world problem.



\begin{example}
  The \link[U.S.\ Postal Service states that the girth plus the length of
  Standard Post Package must not exceed $108\unit{in}$]{https://pe.usps.com/text/qsg300/Q201e.htm\#1009536}. Given a
  rectangular box, the ``length'' is the longest side, and the
  ``girth'' is the perimeter of the base.

  Given a rectangular box where the width and height are equal, what
  are the dimensions of the box that give the maximum volume subject
  to the constraint of the size of a Standard Post Package?
  \begin{explanation}
    Let $w$, $h$ and $\l$ denote the width, height and length of a
    rectangular box. We assume here that $w=h$, so the girth is
    $4w$. Hence the volume of the box is
    \[
    V(w,\l) = w^2\cdot \l
    \]
    We wish to maximize this volume subject to the constraint
    $4w+\l\leq 108$. Note, common sense tells us that the volume will
    be maximized when
    \[
    4w + \l = \answer[given]{108}
    \]
    and that both $w$ and $\l$ are nonnegative. We may now rewrite our
    constraint as:
    \[
    \l = \answer[given]{108 - 4w}
    \]
    and this is applicable on the $w$-interval
    $\left[\answer[given]{0},\answer[given]{27}\right]$.

    We now consider the volume along the constraint $\l=108-4w$. Along
    this line, we have:
    \begin{align*}
      V(w,\l) &= V(w,108-4w)\\
      &= \answer[given]{108w^2-4w^3} 
    \end{align*}
    Set $v(w) = \answer[given]{108w^2-4w^3}$ and find the critical
    points. Write with me:
    \[
    v'(w) = \answer[given]{216w-12w^2}
    \]
    $v'(w)= 0$ when $w = 0$ and $w= \answer[given]{18}$.

    We found two critical values: when $w=0$ and when $w=18$. We
    ignore the $w=0$ solution. Thus the maximum volume, subject to the
    constraint, comes at $w=h=18$, $\l = 108-4(18) =36$.  This gives a
    volume of $V(18,36) = \answer[given]{11664}\unit{in}^3$.
    
    \begin{image}
      \begin{tikzpicture}
        \begin{axis}%
          [tick label style={font=\scriptsize},axis on top,
	    axis lines=center,
	    view={15}{25},
	    name=myplot,
	    %xtick=\empty,
	    %ytick={5},
	    ztick=\empty,
	    minor xtick=1,
	    minor ytick=1,
	    ymin=-5,ymax=135,
	    xmin=-5,xmax=35,
	    %zmin=-1, zmax=8,
	    every axis x label/.style={at={(axis cs:\pgfkeysvalueof{/pgfplots/xmax},0,0)},xshift=5pt,yshift=0pt},
	    xlabel={\scriptsize $w$},
	    every axis y label/.style={at={(axis cs:0,\pgfkeysvalueof{/pgfplots/ymax},0)},xshift=4pt,yshift=3pt},
	    ylabel={\scriptsize $l$},
	    every axis z label/.style={at={(axis cs:0,0,\pgfkeysvalueof{/pgfplots/zmax})},xshift=0pt,yshift=4pt},
	    zlabel={\scriptsize $V$},colormap/cool
	  ]
          
          \addplot3[domain=0:33,,y domain=0:135,mesh,samples=25,samples y=25,very thin,z buffer=sort] {x^2*y};
          
          \addplot3 [ultra thick,penColor, smooth,domain=0:32.5,samples=10,samples y=0] ({x},{108-4*x},{x^2*(108-4*x)});
          \addplot3 [ultra thick,penColor, smooth,dashed,opacity=.5,domain=0:32.5,samples=10,samples y=0] ({x},{108-4*x},{0});

          \filldraw [black,] (axis cs:18,36,11664) circle (3pt);
        \end{axis}
      \end{tikzpicture}
    \end{image}
    The volume function $V(w,\l)$ is shown above along with the
    constraint $\l = 108-4w$. As done previously, the constraint is
    drawn dashed in the $(w,\l)$-plane and also along the surface of
    the function. The Extreme Value Theorem guarantees that we found
    the extrema. The point where the volume is maximized is indicated.
  \end{explanation}
\end{example}



This portion of the text is entitled ``Constrained optimization''
because we want to find extrema of a function subject to a
\textit{constraint}, meaning there are limitations to what values the
function can attain. In our first example the constraint was set by
the U.S.\ Post office.  Constrained optimization problems are an
important topic in applied mathematics. The techniques developed here
are the basis for solving larger problems, where the constraints are
either more complex or more than two variables are involved.


\begin{example}
  Let $F(x,y) = x^2-y^2+5$ and let $T$ be the triangle with vertices
  $(-1,-2)$, $(0,1)$, and $(2,-2)$.
  \begin{image}  
    \begin{tikzpicture}
      \begin{axis}[,tick label style={font=\scriptsize},axis y line=middle,axis x line=middle,name=myplot,%
	  xtick={-2,2},% 
	  minor x tick num=1,
	  extra x ticks={-1,1},%
	  ymin=-2.5,ymax=1.5,%
	  xmin=-2.1,xmax=2.1%
        ]

        \draw [thick,penColor] (axis cs:-1,-2)-- node
              [black,pos=.2,above,sloped] {\scriptsize $y=3x+1$} (axis
              cs:0,1)--node [black,pos=.6,below,sloped] {\scriptsize
                $y=(-3/2)x+1$} (axis cs:2,-2)--node
              [black,pos=.5,below,sloped] {\scriptsize $y=-2$}(axis
              cs:-1,-2);
              
        \fill[black,draw=black] (axis cs:-1,-2) circle (1.5pt);
        \fill[black,draw=black] (axis cs:2,-2) circle (1.5pt);
        \fill[black,draw=black] (axis cs:0,1) circle (1.5pt);
        
      \end{axis}
      \node [right] at (myplot.right of origin) {\scriptsize $x$};
      \node [above] at (myplot.above origin) {\scriptsize $y$};
    \end{tikzpicture}
  \end{image}
  Find the maximum and minimum values of $F$ on $T$.
  \begin{explanation}
    Let's see a graph of $F$ along with the set $T$:
    \begin{image}
      \begin{tikzpicture}
        \begin{axis}%
          [width=175pt,tick label style={font=\scriptsize},axis on top,
	    axis lines=center,
	    view={55}{45},
	    name=myplot,
	    %xtick=\empty,
	    %ytick={5},
	    %ztick={.7,-.7},
	    minor xtick=1,
	    minor ytick=1,
	    ymin=-2.5,ymax=2,
	    xmin=-1.5,xmax=2.5,
	    zmin=-1, zmax=8,
	    every axis x label/.style={at={(axis cs:\pgfkeysvalueof{/pgfplots/xmax},0,0)},xshift=5pt,yshift=0pt},
	    xlabel={\scriptsize $x$},
	    every axis y label/.style={at={(axis cs:0,\pgfkeysvalueof{/pgfplots/ymax},0)},xshift=4pt,yshift=3pt},
	    ylabel={\scriptsize $y$},
	    every axis z label/.style={at={(axis cs:0,0,\pgfkeysvalueof{/pgfplots/zmax})},xshift=0pt,yshift=4pt},
	    zlabel={\scriptsize $z$},colormap/cool
	  ]
          
          \addplot3[domain=-1.4:2.4,,y domain=-2.5:2,mesh,samples=25,samples y=25,very thin,z buffer=sort] {x^2-y^2+5};
          
          \addplot3 [ultra thick,penColor, smooth,domain=-1:0,samples=10,samples y=0] ({x},{3*x+1},{x^2-(3*x+1)^2+5});
          \addplot3 [ultra thick,penColor, smooth,domain=0:2,samples=10,samples y=0] ({x},{-1.5*x+1},{x^2-(-1.5*x+1)^2+5});
          \addplot3 [ultra thick,penColor, smooth,domain=-1:2,samples=10,samples y=0] ({x},{-2},{x^2-4+5});
          
          \addplot3 [ultra thick,penColor, smooth,dashed,domain=-1:0,samples=10,samples y=0] ({x},{3*x+1},{0});
          \addplot3 [ultra thick,penColor, smooth,dashed,domain=0:2,samples=10,samples y=0] ({x},{-1.5*x+1},{0});
          \addplot3 [ultra thick,penColor, smooth,dashed,domain=-1:2,samples=10,samples y=0] ({x},{-2},{0});
          
        \end{axis}
      \end{tikzpicture}
    \end{image}
    The triangle defining $T$ in the $(x,y)$-plane is drawn with a
    dashed line. Above it is the surface described by $F$. 

    Since we are working within a closed and bounded set $T$, we now
    find the maximum and minimum values that $F$ attains along
    $T$. This means we find the extrema of $F$ along the edges of the
    triangle.

    Start with the bottom edge of the triangle, along the line
    $y=\answer[given]{-2}$ while $x$ runs from
    $\left[\answer[given]{-1},\answer[given]{2}\right]$:
    \[
    F\left(x,\answer[given]{-2}\right) = \answer[given]{x^2+1}
    \]
    Now set
    \[
    f_{\mathrm{bottom}}(x) = x^2+1,
    \]
    as this represents the bottom edge of the triangular boundary. We
    want to maximize/minimize $f_{\mathrm{bottom}}$ on the interval
    $[-1,2]$. To do so, we evaluate $f_1(x)$ at its critical points
    and at the endpoints.  First find the critical points of
    $f_{\mathrm{bottom}}$:
    \[
    f_{\mathrm{bottom}}'(x)= \answer[given]{2x}
    \]
    We see that $x= \answer[given]{0}$ is the only critical point of
    $f_{\mathrm{bottom}}$. Evaluating $f_{\mathrm{bottom}}$ at its
    critical point and at the endpoints of
    $\left[\answer[given]{-1},\answer[given]{2}\right]$ gives:
    \[
    \begin{array}{lcl}
      f_{\mathrm{bottom}}(-1) = \answer[given]{2} &\Rightarrow & F(-1,-2) = \answer[given]{2}\\
      f_{\mathrm{bottom}}(0)  = \answer[given]{1} &\Rightarrow & F(0,-2)  = \answer[given]{1}\\
      f_{\mathrm{bottom}}(2)  = \answer[given]{5} &\Rightarrow & F(2,-2)  = \answer[given]{5}.
    \end{array}
    \]
    We need to do this process twice more, for the other two edges of
    the triangle.  Along the left edge, along the line $y=3x+1$, we
    substitute $3x+1$ in for $y$ in $F(x,y)$:
    \begin{align*}
    f_{\mathrm{left}} &= F(x,3x+1)\\
    &= -8x^2-6x+4
    \end{align*}
    We want the maximum and minimum values of $f_{\mathrm{left}}$ on
    the interval $\left[\answer[given]{-1},\answer[given]{0}\right]$, so we
    evaluate $f_{\mathrm{left}}$ at its critical points and the
    endpoints of the interval. First find the critical points of
    $f_{\mathrm{left}}$:
    \[
    f'_{\mathrm{left}}(x) = \answer[given]{-16x-6}
    \]
    We see that $x = \answer[given]{-3/8}$ is the only critical point
    of $f_{\mathrm{left}}$.  Evaluating $f_{\mathrm{left}}$ at its
    critical point and the endpoints of $[-1,0]$ gives:
    \[
    \begin{array}{lcl}
      f_{\mathrm{left}}(-1) = \answer[given]{2} &\Rightarrow& F(-1,-2) = \answer[given]{2}\\
      f_{\mathrm{left}}(-3/8) = \answer[given]{41/8}  &\Rightarrow & F(-3/8,-1/8) = \answer[given]{41/8} \\
      f_{\mathrm{left}}(0) = \answer[given]{4} &\Rightarrow & F(0,1) = \answer[given]{4}.
    \end{array}
    \]
    Finally, we evaluate $F$ along the right edge of the triangle,
    where $y = (-3/2)x+1$:
    \begin{align*}
      f_{\mathrm{right}} &= F(x,(-3/2)x+1)\\
      &= \answer[given]{\frac{-5x^2}{4}+3x+4}
    \end{align*}
    We want the maximum and minimum values of $f_{\mathrm{right}}$ on
    the interval $\left[\answer[given]{0},\answer[given]{2}\right]$, so we
    evaluate $f_{\mathrm{right}}$ at its critical points and the
    endpoints of the interval. First find the critical points of
    $f_{\mathrm{right}}$:
    \[
    f'_{\mathrm{right}}(x) = \answer[given]{\frac{-5x}{2}+3}
    \]
    We see that $x = \answer[given]{6/5}$ is the only critical point.
    Evaluating $f_{\mathrm{right}}$ at this critical point and at the
    endpoints of the interval $[0,2]$:
    \[
    \begin{array}{lcl}
      f_{\mathrm{right}}(0) = \answer[given]{4} &\Rightarrow & F(0,1) = \answer[given]{4}\\
      f_{\mathrm{right}}(1.2) = \answer[given]{5.8} &\Rightarrow &  F(1.2,-0.8) = \answer[given]{5.8}\\
      f_{\mathrm{right}}(2) = \answer[given]{5} &\Rightarrow&  F(2,-2) = 5.
    \end{array}
    \]
    Check out the following graph:
    \begin{image}
      \begin{tikzpicture}[>=stealth]
        \begin{axis}%
          [tick label style={font=\scriptsize},axis on top,
	    axis lines=center,
	    view={15}{45},
	    name=myplot,
	    %xtick=\empty,
	    %ytick={5},
	    %ztick={.7,-.7},
	    minor xtick=1,
	    minor ytick=1,
	    ymin=-2.5,ymax=2,
	    xmin=-1.5,xmax=2.5,
	    zmin=-1, zmax=8,
	    every axis x label/.style={at={(axis cs:\pgfkeysvalueof{/pgfplots/xmax},0,0)},xshift=5pt,yshift=0pt},
	    xlabel={\scriptsize $x$},
	    every axis y label/.style={at={(axis cs:0,\pgfkeysvalueof{/pgfplots/ymax},0)},xshift=4pt,yshift=3pt},
	    ylabel={\scriptsize $y$},
	    every axis z label/.style={at={(axis cs:0,0,\pgfkeysvalueof{/pgfplots/zmax})},xshift=0pt,yshift=4pt},
	    zlabel={\scriptsize $z$},colormap/cool
	  ]
          
          \addplot3[domain=-1.4:2.4,,y domain=-2.5:2,mesh,samples=25,samples y=25,very thin,z buffer=sort] {x^2-y^2+5};
          
          \addplot3 [ultra thick,penColor, smooth,domain=-1:0,samples=10,samples y=0] ({x},{3*x+1},{x^2-(3*x+1)^2+5});
          \addplot3 [ultra thick,penColor, smooth,domain=0:2,samples=10,samples y=0] ({x},{-1.5*x+1},{x^2-(-1.5*x+1)^2+5});
          \addplot3 [ultra thick,penColor, smooth,domain=-1:2,samples=10,samples y=0] ({x},{-2},{x^2-4+5});
          
          \addplot3 [ultra thick,penColor, smooth,dashed,domain=-1:0,samples=10,samples y=0] ({x},{3*x+1},{0});
          \addplot3 [ultra thick,penColor, smooth,dashed,domain=0:2,samples=10,samples y=0] ({x},{-1.5*x+1},{0});
          \addplot3 [ultra thick,penColor, smooth,dashed,domain=-1:2,samples=10,samples y=0] ({x},{-2},{0});
          
          
          %\filldraw [black,] (axis cs:0,0,5) circle (3pt);
          \filldraw [black,] (axis cs:-1,-2,2) circle (3pt);
          \filldraw [black,] (axis cs:-.375,-0.125,5.125) circle (3pt);
          \filldraw [black,] (axis cs:0,-2,1) circle (3pt);
          \filldraw [black,] (axis cs:2,-2,5) circle (3pt);
          \filldraw [black,] (axis cs:0,1,4) circle (3pt);
          \filldraw [black,] (axis cs:1.2,-0.8,5.8) circle (3pt);
        \end{axis}
      \end{tikzpicture}
    \end{image}
    Above we see the surface described by $F$ along with important
    points along $T$.  We have evaluated $F$ at a total of $6$
    different places, all shown above. We checked each vertex of the
    triangle twice, as each showed up as the endpoint of an interval
    twice. Of all the $z$-values found, the maximum is
    $\answer[given]{5.8}$, found at
    $\left(\answer[given]{1.2},\answer[given]{-0.8}\right)$; the
    minimum is $\answer[given]{1}$, found at
    $\left(\answer[given]{0},\answer[given]{-2}\right)$.
  \end{explanation}
\end{example}

When working constrained optimization problems, there is often more
than one way to proceed. Let's do the previous problem again, but this
time we will use vector-valued functions and the chain rule.



\begin{example}
  Let $F(x,y) = x^2-y^2+5$ and let $T$ be the triangle with vertices
  $(-1,-2)$, $(0,1)$, and $(2,-2)$.
  \begin{image}  
    \begin{tikzpicture}
      \begin{axis}[,tick label style={font=\scriptsize},axis y line=middle,axis x line=middle,name=myplot,%
	  xtick={-2,2},% 
	  minor x tick num=1,
	  extra x ticks={-1,1},%
	  ymin=-2.5,ymax=1.5,%
	  xmin=-2.1,xmax=2.1%
        ]

        \draw [thick,penColor] (axis cs:-1,-2)-- node
              [black,pos=.5,above,sloped] {\scriptsize $\vecl_\mathrm{L}(t) = \vector{0,1}+t\vector{-1,-3}$} (axis
              cs:0,1)--node [black,pos=.5,below,sloped] {\scriptsize
                $\vecl_{\mathrm{R}}(t) = \vector{0,1}+t\vector{2,-3}$} (axis cs:2,-2)--node
              [black,pos=.5,below,sloped] {\scriptsize $\vecl_{\mathrm{B}}(t) = \vector{-1,-2} + t\vector{3,0}$}(axis
              cs:-1,-2);
              
        \fill[black,draw=black] (axis cs:-1,-2) circle (1.5pt);
        \fill[black,draw=black] (axis cs:2,-2) circle (1.5pt);
        \fill[black,draw=black] (axis cs:0,1) circle (1.5pt);
        
      \end{axis}
      \node [right] at (myplot.right of origin) {\scriptsize $x$};
      \node [above] at (myplot.above origin) {\scriptsize $y$};
    \end{tikzpicture}
  \end{image}
  Find the maximum and minimum values of $F$ on  $T$.
  \begin{explanation}
    Since we are working within a closed and bounded set $T$, the
    Extreme Value Theorem tells us that the extrema will occur. We now
    find the maximum and minimum values that $F$ attains along
    $T$. This means we find the extrema of $F$ along the edges of the
    triangle. This time we'll use the chain rule. Starting along the
    bottom edge, write with me:
    \begin{align*}
      \dd{t} F(\vecl_{\mathrm{B}}(t)) &= \vector{\answer[given]{2(-1+3t)},\answer[given]{4}}\dotp\vector{\answer[given]{3},\answer[given]{0}}\\
      &=\answer[given]{-6+18t}
    \end{align*}
    We see that $t= \answer[given]{1/3}$ is the only critical point of
    $F(\vecl_{\mathrm{B}}(t))$. Evaluating this at its critical point
    and at the endpoints of $\left[\answer[given]{0},\answer[given]{1}\right]$
    gives:
    \begin{align*}
      F(\vecl_{\mathrm{B}}(0)) &= \answer[given]{2}\\
      F(\vecl_{\mathrm{B}}(1/3)) &= \answer[given]{1}\\
      F(\vecl_{\mathrm{B}}(1))  &= \answer[given]{5}.
    \end{align*}
    We need to do this process twice more, for the other two edges of
    the triangle. For the left edge, write with me:
    \begin{align*}
      \dd{t} F(\vecl_{\mathrm{L}}(t)) &= \vector{\answer[given]{2(-t)},\answer[given]{-2(1-3t)}}\dotp\vector{\answer[given]{-1},\answer[given]{-3}}\\
      &=\answer[given]{6-16t}
    \end{align*}          
    We see that $t= \answer[given]{3/8}$ is the only critical point of
    $F(\vecl_{\mathrm{B}}(t))$. Evaluating this at its critical point
    and at the endpoints of $\left[\answer[given]{0},\answer[given]{1}\right]$
    gives:
    \begin{align*}
      F(\vecl_{\mathrm{L}}(0)) &= \answer[given]{4}\\
      F(\vecl_{\mathrm{L}}(3/8)) &= \answer[given]{5.125}\\
      F(\vecl_{\mathrm{L}}(1))  &= \answer[given]{2}.
    \end{align*}

    Now for the right edge, write with me:
    \begin{align*}
      \dd{t} F(\vecl_{\mathrm{R}}(t)) &= \vector{\answer[given]{2(2t)},\answer[given]{-2(1-3t)}}\dotp\vector{\answer[given]{2},\answer[given]{-3}}\\
      &=\answer[given]{6-10t}
    \end{align*}          
    We see that $t= \answer[given]{3/5}$ is the only critical point of
    $F(\vecl_{\mathrm{B}}(t))$. Evaluating this at its critical point
    and at the endpoints of $\left[\answer[given]{0},\answer[given]{1}\right]$
    gives:
    \begin{align*}
      F(\vecl_{\mathrm{R}}(0)) &= \answer[given]{4}\\
      F(\vecl_{\mathrm{R}}(3/5)) &= \answer[given]{5.8}\\
      F(\vecl_{\mathrm{R}}(1))  &= \answer[given]{5}.
    \end{align*}
    Again, check out the following graph:
    \begin{image}
      \begin{tikzpicture}[>=stealth]
        \begin{axis}%
          [tick label style={font=\scriptsize},axis on top,
	    axis lines=center,
	    view={15}{45},
	    name=myplot,
	    %xtick=\empty,
	    %ytick={5},
	    %ztick={.7,-.7},
	    minor xtick=1,
	    minor ytick=1,
	    ymin=-2.5,ymax=2,
	    xmin=-1.5,xmax=2.5,
	    zmin=-1, zmax=8,
	    every axis x label/.style={at={(axis cs:\pgfkeysvalueof{/pgfplots/xmax},0,0)},xshift=5pt,yshift=0pt},
	    xlabel={\scriptsize $x$},
	    every axis y label/.style={at={(axis cs:0,\pgfkeysvalueof{/pgfplots/ymax},0)},xshift=4pt,yshift=3pt},
	    ylabel={\scriptsize $y$},
	    every axis z label/.style={at={(axis cs:0,0,\pgfkeysvalueof{/pgfplots/zmax})},xshift=0pt,yshift=4pt},
	    zlabel={\scriptsize $z$},colormap/cool
	  ]
          
          \addplot3[domain=-1.4:2.4,,y domain=-2.5:2,mesh,samples=25,samples y=25,very thin,z buffer=sort] {x^2-y^2+5};
          
          \addplot3 [ultra thick,penColor, smooth,domain=-1:0,samples=10,samples y=0] ({x},{3*x+1},{x^2-(3*x+1)^2+5});
          \addplot3 [ultra thick,penColor, smooth,domain=0:2,samples=10,samples y=0] ({x},{-1.5*x+1},{x^2-(-1.5*x+1)^2+5});
          \addplot3 [ultra thick,penColor, smooth,domain=-1:2,samples=10,samples y=0] ({x},{-2},{x^2-4+5});
          
          \addplot3 [ultra thick,penColor, smooth,dashed,domain=-1:0,samples=10,samples y=0] ({x},{3*x+1},{0});
          \addplot3 [ultra thick,penColor, smooth,dashed,domain=0:2,samples=10,samples y=0] ({x},{-1.5*x+1},{0});
          \addplot3 [ultra thick,penColor, smooth,dashed,domain=-1:2,samples=10,samples y=0] ({x},{-2},{0});
          
          
%          \filldraw [black,] (axis cs:0,0,5) circle (3pt);
          \filldraw [black,] (axis cs:-1,-2,2) circle (3pt);
          \filldraw [black,] (axis cs:-.375,-0.125,5.125) circle (3pt);
          \filldraw [black,] (axis cs:0,-2,1) circle (3pt);
          \filldraw [black,] (axis cs:2,-2,5) circle (3pt);
          \filldraw [black,] (axis cs:0,1,4) circle (3pt);
          \filldraw [black,] (axis cs:1.2,-0.8,5.8) circle (3pt);
        \end{axis}
      \end{tikzpicture}
    \end{image}
    Above we see the surface described by $F$ along with important
    points along $T$.  We have evaluated $F$ at a total of $6$
    different places, all shown above. We checked each vertex of the
    triangle twice, as each showed up as the endpoint of an interval
    twice. Of all the $z$-values found, the maximum is
    $\answer[given]{5.8}$, found at
    $\left(\answer[given]{1.2},\answer[given]{-0.8}\right)$; the
    minimum is $\answer[given]{1}$, found at
    $\left(\answer[given]{0},\answer[given]{-2}\right)$. The Extreme Value Theorem guarantees that we have found the extrema on $T$.
  \end{explanation}
\end{example}

So far, our constraints have all been lines. This doesn't have to be the case. 


\begin{example}
Let $F(x,y) = 4x^3+4y^2-4x$ and let $C$ the set
\[
C = \{(x,y):x^2 + y^2 =1\}
\]
Find the maximum and minimum values of $F$ on $C$.
\begin{explanation}
  One way to do this is to find a vector-valued function that draws
  $C$. Since $C$ is a circle, we know $C$ is draw by:
  \[
  \vec{c}(t) = \vector{\cos(t),\sin(t)}
  \]
  for $0\le t<2\pi$. Now compute $\dd{t} F(\vec{c}(t))$. The chain rule says: 
  \[
  \dd{t} F(\vec{c}(t)) = \grad F(\vec{c}(t))\dotp \vec{c}'(t)
  \]
  So let's compute:
  \[
  \grad F(x,y) =\vector{\answer[given]{12x^2-4},\answer[given]{8y}}
  \]
  and
  \[
  \vec{c}'(t) = \vector{\answer[given]{-\sin(t)},\answer[given]{\cos(t)}}
  \]
  Putting this together, we see
  \begin{align*}
    \dd{t} F(\vec{c}(t)) &= \vector{\answer[given]{12\cos^2(t)-4},\answer[given]{8\sin(t)}}\dotp\vector{\answer[given]{-\sin(t)},\answer[given]{\cos(t)}}\\
    &=\answer[given]{4\sin(t)+8\cos(t)\sin(t) -12\cos^2(t)\sin(t)}
  \end{align*}
  Now we set $\dd[F]{t} = 0$. Here you may be tempted to solve for
  $t$, but this author suggestions that you instead solve for
  $\cos(t)$ and $\sin(t)$. Note that if
  \[
  4\sin(t)+8\cos(t)\sin(t) -12\cos^2(t)\sin(t) =0
  \]
  one solution is $\sin(t) = 0$. This corresponds to the points
  \[
  \left(1,0\right) \quad\text{and}\quad \left(-1,0\right)
  \]
  Now assume that $\sin(t) \ne 0$ and divide the equation above by
  $4\sin(t)$. We now have
  \[
  1+2\cos(t) -3\cos^2(t) =0
  \]
  Using the quadratic formula we find
  \[
  \cos(t) = \frac{-1\pm2}{-3}
  \]
  This corresponded to points
  \[
  \left(\frac{-1}{3},\frac{-\sqrt{8}}{3}\right) \quad\text{and}\quad\left(\frac{-1}{3},\frac{\sqrt{8}}{3}\right)
  \]
  Plugging these points back into $F$, we find that the minimum value
  of $F$ on $C$ is $\answer[given]{0}$ and the maximum value is
  $\answer[given]{128/27}$.
\end{explanation}
\end{example}          





      
It is hard to overemphasize the importance of optimization. As humans,
we routinely seek to make \textit{something} better. By expressing the
\textit{something} as a mathematical function, ``making
\textit{something} better'' means ``optimize \textit{some function.}''



\end{document}
