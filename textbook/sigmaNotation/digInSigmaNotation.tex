\documentclass{ximera}

\newcommand{\RR}{\mathbb R}
\renewcommand{\d}{\,d}
\newcommand{\dd}[2][]{\frac{d #1}{d #2}}
\renewcommand{\l}{\ell}
\newcommand{\ddx}{\frac{d}{dx}}
\newcommand{\dfn}{\textbf}
\newcommand{\eval}[1]{\bigg[ #1 \bigg]}


\title[Dig-In:]{Sigma notation}

\begin{document}
\begin{abstract}
  He we use sigma notation to express our sums that approximate area.
\end{abstract}
\maketitle

\section{Why sigma notation?}

The notation
\[
\sigma_{k = 1}^n f(k) 
\]
may look pretty scary, but if you just relax, you can do it.

In words this expression says:


If you are someone who is interested in computers, here is pseudocode
that performs the same operation as our expression above.

S = 0
For k from 1 to n,  S <= S + f(k)
Return S

%%BADBAD should we do a python exercise here?




\end{document}
