\documentclass{ximera}

\newcommand{\RR}{\mathbb R}
\renewcommand{\d}{\,d}
\newcommand{\dd}[2][]{\frac{d #1}{d #2}}
\renewcommand{\l}{\ell}
\newcommand{\ddx}{\frac{d}{dx}}
\newcommand{\dfn}{\textbf}
\newcommand{\eval}[1]{\bigg[ #1 \bigg]}


\author{Bart Snapp}

\outcome{Give the equation for a sphere or ball.}

\title[Dig-In:]{Drawing a sphere}

\begin{document}
\begin{abstract}
  Learn how to draw a sphere.
\end{abstract}
\maketitle

A key challenge in mathematics is converting formulas and equations
into ideas. We want to get to the point that when you see something
like
\[
S = \{(x,y,z):(x-1)^2+(y-2)^2+(z-3)^2=16\}
\]
you say to yourself, ``Hey, that's a sphere of radius $4$ centered at
the point $(1,2,3)$. Let's state this generally.
\begin{theorem}
  The set of points on the surface of a sphere of radius $r$ centered
  at $(a,b,c)$ are given by the set:
  \[
  S = \{(x,y,z):(x-a)^2+(y-b)^2+(z-c)^2=r^2\}
  \]
\end{theorem}
Above we give an implicit formula for the surface of the
sphere. Sometimes it is better to have parametric formulas.
\begin{theorem}
  The parametric formula for a sphere of radius $r$ centered at
  $(a,b,c)$ is give by:
  \begin{align*}
    x(\phi,\theta) &=r\cdot\cos(\theta)\sin(\phi)+a\\
    y(\phi,\theta) &=r\cdot\sin(\theta)\sin(\phi)+b\\
    z(\phi,\theta) &=r\cdot\cos(\phi)+c
  \end{align*}
  for $0\le \phi\le \pi$ and $0\le\theta< 2\pi$.
\end{theorem}


One thing to note above is that in some sense, the parametric formula
for the sphere ``draws'' the sphere (we'll get to that in a
second). For now, let me show you how to draw a sphere yourself. Get
out a sheet of paper, and play along---it will be fun! Start by
drawing a set of axes:
\begin{image}
  \begin{tikzpicture}  
    \begin{axis}[  
        xmin=-1.2,  
        xmax=1.2,  
        ymin=-1.2,  
        ymax=1.2,
        unit vector ratio=1 1 1,
        axis lines=center,
        ticks=none,
        xlabel=$y$,  
        ylabel=$z$,  
        every axis y label/.style={at=(current axis.above origin),anchor=south},  
        every axis x label/.style={at=(current axis.right of origin),anchor=west},  
      ]  
      \addplot [->] coordinates {(.7,.5) (-.7,-.5)};
      %\addplot [ultra thick, penColor,domain=0:360,smooth] ({cos(x)},{sin(x)});
      %\addplot [ultra thick, penColor,domain=180:360,smooth] ({cos(x)},{.4*sin(x)});
      %\addplot [ultra thick, dashed, penColor,domain=0:180,smooth] ({cos(x)},{.4*sin(x)});
      \node at (axis cs: -.75,-.52) {$x$};
    \end{axis}  
  \end{tikzpicture}  
\end{image}
Now draw a circle in the $(y,z)$-plane:
\begin{image}
  \begin{tikzpicture}  
    \begin{axis}[  
        xmin=-1.2,  
        xmax=1.2,  
        ymin=-1.2,  
        ymax=1.2,
        unit vector ratio=1 1 1,
        axis lines=center,
        ticks=none,
        xlabel=$y$,  
        ylabel=$z$,  
        every axis y label/.style={at=(current axis.above origin),anchor=south},  
        every axis x label/.style={at=(current axis.right of origin),anchor=west},  
      ]  
      \addplot [->] coordinates {(.7,.5) (-.7,-.5)};
      \addplot [ultra thick, penColor,domain=0:360,smooth] ({cos(x)},{sin(x)});
      %\addplot [ultra thick, penColor,domain=180:360,smooth] ({cos(x)},{.4*sin(x)});
      %\addplot [ultra thick, dashed, penColor,domain=0:180,smooth] ({cos(x)},{.4*sin(x)});
      \node at (axis cs: -.75,-.52) {$x$};
    \end{axis}  
  \end{tikzpicture}  
\end{image}
Pro-tip: If you have trouble drawing a circle, and most people do, try
drawing circles on graph paper. Practice makes perfect, and if you
practice enough, soon you'll be able to impress your friends and
enemies alike with your circle-drawing skills.  Now draw an ellipse,
dashing the part at the ``back'' of the sphere:
\begin{image}
  \begin{tikzpicture}  
    \begin{axis}[  
        xmin=-1.2,  
        xmax=1.2,  
        ymin=-1.2,  
        ymax=1.2,
        unit vector ratio=1 1 1,
        axis lines=center,
        ticks=none,
        xlabel=$y$,  
        ylabel=$z$,  
        every axis y label/.style={at=(current axis.above origin),anchor=south},  
        every axis x label/.style={at=(current axis.right of origin),anchor=west},  
      ]  
      \addplot [->] coordinates {(.7,.5) (-.7,-.5)};
      \addplot [ultra thick, penColor,domain=0:360,smooth] ({cos(x)},{sin(x)});
      \addplot [ultra thick, penColor,domain=180:360,smooth] ({cos(x)},{.4*sin(x)});
      \addplot [ultra thick, dashed, penColor,domain=0:180,smooth] ({cos(x)},{.4*sin(x)});
      \node at (axis cs: -.75,-.52) {$x$};
    \end{axis}  
  \end{tikzpicture}  
\end{image}
And voli\`a, we have a sphere! Now let me tell you something, people
who like mathematics really like asking (and answering) questions like
the following:

\begin{example}
  Verify that the two descriptions
  \[
  S = \{(x,y,z):(x-a)^2+(y-b)^2+(z-c)^2=r^2\}
  \]
  and
  \begin{align*}
    x(\phi,\theta) &=r\cdot\cos(\theta)\sin(\phi)+a\\
    y(\phi,\theta) &=r\cdot\sin(\theta)\sin(\phi)+b\\
    z(\phi,\theta) &=r\cdot\cos(\phi)+c
  \end{align*}
  where $0\le \phi\le \pi$ and $0\le \theta<2\pi$ describe the same geometric set.
  \begin{explanation}
    What we are being asked to do here is verify that both
    descriptions of the sphere in fact describe the same geometric
    object.  Here is how you do it. First show one description
    ``contains'' the other, and the other description ``contains'' the
    one. First we'll show that every point of the parametric formula
    \begin{align*}
    x(\phi,\theta) &=r\cdot\cos(\theta)\sin(\phi)+a\\
    y(\phi,\theta) &=r\cdot\sin(\theta)\sin(\phi)+b\\
    z(\phi,\theta) &=r\cdot\cos(\phi)+c
    \end{align*}
    is also a point of $S$. We do this by plugging the point
    \[
    \big(x(\phi,\theta),y(\phi,\theta),z(\phi,\theta)\big)
    \]
    into 
    \[
    (x-a)^2+(y-b)^2+(z-c)^2
    \]
    and through a manipulation of symbols, conclude that it does equal
    $r^2$. Write with me:
    \begin{align*}
      (x(\phi,\theta)&-a)^2+(y(\phi,\theta)-b)^2+(z(\phi,\theta)-c)^2 \\
      &= (x(\phi,\theta)r\cdot\cos(\theta)\sin(\phi)+a-a)^2+(y(\phi,\theta)-b)^2+(z(\phi,\theta)-c)^2\\
      &= (\answer[given]{r\cdot\cos(\theta)\sin(\phi)+a}-a)^2+(\answer[given]{r\cdot\sin(\theta)\sin(\phi)+b}-b)^2+(\answer[given]{r\cdot\cos(\phi)+c}-c)^2\\
      &= \answer[given]{r^2}\left(\cos^2(\theta)\sin^2(\phi) + \sin^2(\theta)\sin^2(\phi) + \cos^2(\phi)\right)\\
      &= \answer[given]{r^2}\left(\sin^2(\phi) + \cos^2(\phi)\right)\\
      &= r^2.
    \end{align*}
    So we see that every point drawn by our parametric description of
    the sphere is in the set $S$. 
  \end{explanation}
\end{example}


\end{document}
